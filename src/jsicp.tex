\documentclass[oneside]{book}
\usepackage{zxjatype}

\setjamainfont[BoldFont=IPAゴシック]{IPA明朝}
\setjasansfont{IPAゴシック}
\setjamonofont{IPAゴシック}
\renewcommand\indexname{索引} 
\renewcommand{\contentsname}{目次}

% New line height: 1.05 * 1.2 = 1.26
\renewcommand{\baselinestretch}{1.05}

\newfontfamily\greekfont[Mapping=TeX]{CMU Serif} % Computer Modern Unicode
\usepackage{polyglossia}
\setotherlanguage{greek}

% To be able to use '\-/' in place of '-' inside \code{}
% so that long function names containing hyphens 
% can be broken up after the hyphen:
\usepackage[shortcuts]{extdash} 

% So that file names with multiple dots don't confuse 
% graphicx package when using \includegraphics command:
\usepackage[multidot]{grffile}
\usepackage{graphicx}

\usepackage[usenames,dvipsnames,x11names]{xcolor}
\usepackage{amsmath}

% Workaround to fix mismatched left and right math delimiters. Taken from: 
% http://tex.stackexchange.com/questions/63410/parentheses-differ-xelatex-fontspec-newtxmath-libertine
\DeclareSymbolFont{parenthesis}{T1}{fxl}{m}{n}
\DeclareMathDelimiter{(}{\mathopen}{parenthesis}{"28}{largesymbols}{"00}
\DeclareMathDelimiter{)}{\mathclose}{parenthesis}{"29}{largesymbols}{"01}
\DeclareMathDelimiter{[}{\mathopen}{parenthesis}{"5B}{largesymbols}{"02} 
\DeclareMathDelimiter{]}{\mathclose}{parenthesis}{"5D}{largesymbols}{"03} 
\DeclareMathDelimiter{\lbrace}{\mathopen}{parenthesis}{"7B}{largesymbols}{"08} 
\DeclareMathDelimiter{\rbrace}{\mathclose}{parenthesis}{"7D}{largesymbols}{"09}

\usepackage{fancyvrb}
\usepackage{imakeidx}
\usepackage[totoc,font=footnotesize]{idxlayout}
\usepackage{fancyhdr}
\pagestyle{plain}
\usepackage[final]{pdfpages} % inserts pages from a pdf file

% Page geometry for 10-inch tablets:
\usepackage[papersize={148mm,197mm},
            top=21mm,
            textwidth=111mm,
            textheight=148mm,
            hcentering,
]{geometry}

\usepackage{titlesec}  % to change the appearance of section titles 
\usepackage{listings}  % for syntax highlighted code listings
\usepackage{verbatim}  % for simple verbatim and comment environments
\usepackage{enumerate} % allows customized labels in enumerations
\usepackage{hyperref}  % makes cross references and URLs clickable 
\definecolor{LinkRed}{HTML}{80171F}
\hypersetup{
  pdfauthor={Harold Abelson, Gerald Jay Sussman, Julie Sussman},
  pdftitle={Structure and Interpretation of Computer Programs, 2nd ed.},
  pdfsubject={computer science, programming, abstraction},
  colorlinks=true,
  linkcolor=LinkRed,
  urlcolor=LinkRed,
  bookmarks=true,
  bookmarksnumbered=true
}

% Document colors 
\definecolor{SchemeLight}  {HTML} {686868}
\definecolor{SchemeSteel}  {HTML} {787878}
\definecolor{SchemeDark}   {HTML} {262626}
\definecolor{SchemeBlue}   {HTML} {4172A3}
\definecolor{SchemeGreen}  {HTML} {487818}
\definecolor{SchemeBrown}  {HTML} {A07040}
\definecolor{SchemeRed}    {HTML} {AD4D3A}
\definecolor{SchemeViolet} {HTML} {7040A0}
\definecolor{DropCapGray}  {HTML} {A8A8A8}
\definecolor{ChapterGray}  {HTML} {C8C8C8}

\usepackage{lettrine}  % adds commands that make drop capitals
\renewcommand{\LettrineFontHook}{\rmfamily\bfseries\color{DropCapGray}}
\renewcommand{\DefaultLraise}{0.00}
\renewcommand{\DefaultLoversize}{0.02}
\renewcommand{\DefaultLhang}{0.12}
\setlength{\DefaultFindent}{1pt}
\setlength{\DefaultNindent}{0em}

\lstset{%
  % Scheme syntax highlighter
    columns=fixed,
    extendedchars=true,
    upquote=true,
    showstringspaces=false,
    sensitive=false,
    mathescape=true,
    escapechar=~,
    alsodigit={>,<,/,-,=,!,?,*},
    alsoletter=',
    morestring=[b]",
    morecomment=[l];,
    % Keyword list taken form functional.py in Pygments package:
    morekeywords={lambda, define, if, else, cond, and, or, case,%
      let, let*, letrec, begin, do, delay, set!, =>, quote,%
      quasiquote, unquote, unquote-splicing, define-syntax, let-syntax,%
      letrec-syntax, syntax-rules},
    % If keywords are quoted, they must not be highlighted:
    emph={'lambda, 'define, 'if, 'else, 'cond, 'and, 'or, 'case,%
      'let, 'let*, 'letrec, 'begin, 'do, 'delay, 'set!, '=>, 'quote,%
      'quasiquote, 'unquote, 'unquote-splicing, 'define-syntax, 'let-syntax,%
      'letrec-syntax, 'syntax-rules}, 
    emphstyle=\color{SchemeDark},
    % Paint error red:
    emph={[2]error},emphstyle=[2]\color{SchemeRed},%
    % Builtins taken from functional.py:
    emph={[3]*, +, -, /, <, <=, =, >, >=, abs, acos, angle,
        append, apply, asin, assoc, assq, assv, atan,
        boolean?, caaaar, caaadr, caaar, caadar, caaddr, caadr,
        caar, cadaar, cadadr, cadar, caddar, cadddr, caddr,
        cadr, call-with-current-continuation, call-with-input-file,
        call-with-output-file, call-with-values, call/cc, car,
        cdaaar, cdaadr, cdaar, cdadar, cdaddr, cdadr, cdar,
        cddaar, cddadr, cddar, cdddar, cddddr, cdddr, cddr,
        cdr, ceiling, char->integer, char-alphabetic?, char-ci<=?,
        char-ci<?, char-ci=?, char-ci>=?, char-ci>?, char-downcase,
        char-lower-case?, char-numeric?, char-ready?, char-upcase,
        char-upper-case?, char-whitespace?, char<=?, char<?, char=?,
        char>=?, char>?, char?, close-input-port, close-output-port,
        complex?, cons, cos, current-input-port, current-output-port,
        denominator, display, dynamic-wind, eof-object?, eq?,
        equal?, eqv?, eval, even?, exact->inexact, exact?, exp,
        expt, floor, for-each, force, gcd, imag-part,
        inexact->exact, inexact?, input-port?, integer->char,
        integer?, interaction-environment, lcm, length, list,
        list->string, list->vector, list-ref, list-tail, list?,
        load, log, magnitude, make-polar, make-rectangular,
        make-string, make-vector, map, max, member, memq, memv,
        min, modulo, negative?, newline, not, null-environment,
        null?, number->string, number?, numerator, odd?,
        open-input-file, open-output-file, output-port?, pair?,
        peek-char, port?, positive?, procedure?, quotient,
        rational?, rationalize, read, read-char, real-part, real?,
        remainder, reverse, round, scheme-report-environment,
        set-car!, set-cdr!, sin, sqrt, string, string->list,
        string->number, string->symbol, string-append, string-ci<=?,
        string-ci<?, string-ci=?, string-ci>=?, string-ci>?,
        string-copy, string-fill!, string-length, string-ref,
        string-set!, string<=?, string<?, string=?, string>=?,
        string>?, string?, substring, symbol->string, symbol?,
        tan, transcript-off, transcript-on, truncate, values,
        vector, vector->list, vector-fill!, vector-length,
        vector-ref, vector-set!, vector?, with-input-from-file,
        with-output-to-file, write, write-char, zero?},
    emphstyle=[3]\color{SchemeViolet},%
    %
    basicstyle=\color{SchemeLight}\ttfamily,
    keywordstyle=\color{SchemeBlue}\bfseries,
    identifierstyle=\color{SchemeDark},
    stringstyle=\color{SchemeGreen},
    commentstyle=\color{SchemeLight}\itshape,
}
  
\newcommand{\acronym}[1]{\textsc{\MakeLowercase{#1}}}
\newcommand{\newterm}[1]{\index{#1}\emph{#1}}
\newcommand{\jnewterm}[1]{\index{#1}{\bf #1}}
\newcommand{\strong}[1]{\textbf{#1}}
\newcommand{\var}[1]{\textsl{#1}}
\newcommand{\code}[1]{\texttt{#1}}
\newcommand{\link}[1]{\hyperref[#1]{#1}}
\newcommand{\heading}[1]{{\sffamily\bfseries #1}}
\newcommand{\dark}{\color{SchemeDark}}

\newenvironment{example}%
  {\verbatim\small}%
  {\endverbatim}

\newenvironment{smallexample}%
  {\verbatim\footnotesize}%
  {\endverbatim}

\lstnewenvironment{scheme}[1][]
{\lstset{basicstyle=\ttfamily\small\color{SchemeLight},#1}}
{}

\lstnewenvironment{smallscheme}[1][]
{\lstset{basicstyle=\ttfamily\footnotesize\color{SchemeLight},#1}}
{}

\titleformat{\chapter}[display]
  {\color{SchemeDark}\normalfont\sffamily\bfseries\LARGE}
  {\filright \color{ChapterGray}\fontsize{3em}{0em}\selectfont
    \oldstylenums{\thechapter}}
  {1em}
  {\filright}
  
\titleformat{\section}
{\color{SchemeDark}\normalfont\Large\sffamily\bfseries}
{\color{SchemeSteel}\thesection}{0.8em}{}

\titleformat{\subsection}
{\color{SchemeDark}\normalfont\large\sffamily\bfseries}
{\color{SchemeSteel}\thesubsection}{0.8em}{}

\titleformat{\subsubsection}
{\color{black}\normalfont\normalsize\sffamily\bfseries}
{\color{SchemeSteel}\thesubsubsection}{0.8em}{}

\setcounter{secnumdepth}{3}
\setcounter{tocdepth}{3}

\frenchspacing
\makeindex

%====================%
%  End of preamble.  %
%====================%

\begin{document}
\pagenumbering{roman}
\VerbatimFootnotes

%             **********************************************************
%         sicp
%            Structure and Interpretation of Computer Programs, 2e
%             Unofficial Texinfo Format
%
% utfversion      2.andresraba5.2 
% utfversiondate  February 10, 2014
%
%             This file is licensed under a Creative Commons 
%             Attribution-ShareAlike 3.0 Unported License 
%             http://creativecommons.org/licenses/by-sa/3.0/
%             
%             This is a Texinfo file.  To convert it to Info hypertext
%             format, you will need the `makeinfo' program from the GNU
%             Texinfo package.  To produce a PDF, use `texi2pdf'. 
%             For more information about this file,
%             see the text under `\label{UTF' below.}
%             
%             Various versions of sicp.texi and preformatted sicp.info
%             can be found at the following Web pages:
%             
%                 http://www.neilvandyke.org/sicp-texi/
%                 http://sicpebook.wordpress.com/
%                 [add your own here]
%             
%             **********************************************************

% HISTORY:
%
% * Version 1 (April, 2001) by Lytha Ayth.
%
% * Version 2 (April 20, 2001) by Lytha Ayth.
%
% * Version 2.nwv1 (March 11, 2002) by Neil W. Van Dyke.
%   Cosmetic change to heading in Info format, and comment changes.
% 
% * Version 2.neilvandyke1 (February 10, 2003) by Neil W. Van Dyke
%   Correction to Exercise 1.39 formula, spotted by Steve VanDevender.
%   Added URL of Abelson and Sussman video lectures.
%
% * Version 2.neilvandyke2 (unreleased)
%
% * Version 2.neilvandyke3 (April 20, 2006) by Neil W. Van Dyke
%   Pedro Kr\"oger patch to add missing Lisp example.
%
% * Version 2.neilvandyke4 (January 10, 2007) by Neil W. Van Dyke
%   Brad Walker patch to add \code{@dircategory} and \code{@direntry}.
%
% * Version 2.andresraba1 (May 23, 2011) by Andres Raba.
%   Mathematics typeset in TeX, figures redrawn in vector graphics,
%   typeface changed, cross-references improved, hyperlinks added,
%   known errors and typos corrected.
%
% * Version 2.andresraba2 (November 21, 2011) by Andres Raba.
%   Minor change to the appearance of diagrams. Adjusted page layout.
%   Fixed some typos. License changed from CC BY-NC to CC BY-SA.
%
% * Version 2.andresraba3 (November 22, 2012) by Andres Raba.
%   Improved layout and pagination. Included list of figures.
%   Added punctuation to displayed math. Updated citation links.

% * (Version 2.andresraba4 is a pocket version, described in
%   sicp-pocket.texi.)

% * Version 2.andresraba5 (September 20, 2013) by Andres Raba.
%   Texinfo source is converted to LaTeX. Pages are redesigned.

% The Algorithmic Language Scheme

\frontmatter

\includepdf[scale=0.92]{coverpage.pdf}

\pagebreak

\vspace*{\fill}
\thispagestyle{empty}

\begin{small}

\noindent
{\copyright}1996 by The Massachusetts Institute of Technology

\vspace{1.26em}
\noindent
Structure and Interpretation of Computer Programs,\\
second edition

\vspace{1.26em}
\noindent
Harold Abelson and Gerald Jay Sussman\\
with Julie Sussman, foreword by Alan J. Perlis

\vspace{1.6em}
\noindent
\includegraphics[width=25mm]{fig/icons/by-nc-sa.pdf}

\vspace{0.4em}
\noindent
This work is licensed under a Creative Commons\\ 
Attribution-NonCommercial-ShareAlike 3.0 Unported License\\
(\href{http://creativecommons.org/licenses/by-nc-sa/3.0/}{\acronym{CC BY-NC-SA} 3.0}).
Based on a work at \href{http://mitpress.mit.edu/sicp/}{mitpress.mit.edu}.

\vspace{1.26em}
\noindent
The \acronym{MIT} Press\\
Cambridge, Massachusetts\\ 
London, England

\vspace{1.26em}
\noindent
McGraw-Hill Book Company\\
New York, St. Louis, San Francisco,\\ 
Montreal, Toronto

\vspace{1.26em}
\noindent
Unofficial Texinfo Format \href{http://sicpebook.wordpress.com}{2.andresraba5.2} (February 10, 2014),\\ 
based on \href{http://www.neilvandyke.org/sicp-texi/}{2.neilvandyke4} (January 10, 2007).

\vspace{1.26em}
\noindent
日本語: by \href{http://github.com/minghai/sicp-pdf/}{minghai} based on 2.andresraba5.2 (March 31, 2014).

\end{small}

\pagebreak

\tableofcontents

\small  % Added by minghai. (Japanese fonts looks too big.)

%=======================================================================================================

\chapter*{非公式Texinfoフォーマット}
\addcontentsline{toc}{chapter}{非公式Texinfoフォーマット}
\label{UTF}

これは\acronym{SICP}の第二版非公式Texinfo版です。

あなたは恐らくこれをEmacsのInfoモードの様なハイパーテキストブラウザで読んで
いることでしょう。他にも{\TeX}で組版した物を画面や印刷して読んでいるかもしれませんが
それはバカバカしい上に高くつきます。


公式に無料で公開された\acronym{HTML}-and-\acronym{GIF}版を
Lytha Aythが最初に私的に、2001年4月の長いEmacs Lovefest Weekendの間に
非公式Texinfo版(\acronym{UTF})バージョン1へと変換しました。



\acronym{UTF}は\acronym{HTML}版よりも検索がより簡単です。また寄付された古い386の様な
質素な計算機上で行う人々にとってよりアクセスが容易です。386は理論的にはLinux、Emacs、
Schemeインタプリタを同時に実行できます。しかし多くの386は恐らくNetscapeと必要なX Window
Systemを事前に芽の出かけた資金不足の若いハッカーに\newterm{thrashing}(\jnewterm{スラッシング})の
概念を教えることなしに動かすことはできないでしょう。UTFはまた圧縮無しでも1.44\acronym{MB}
のフロッピーディスケットに収まります。これはインターネットやLANへの接続環境の無いPC
にインストールする場合に役立つでしょう。



Texinfoへの変換は可能な範囲での直接的な翻字でした。{\TeX}-to-\acronym{HTML}変換の様に
ある程度の破れが含まれること無しにはできませんでした。非公式TexInfo形式においては
図が「失なわれた技術」であるアスキーアートによる下手糞な"復活"を被りました。また
多量の上付き文字と下付き文字のいくつかの変換の間に不明瞭さによる変換の
失敗が含まれてしまった可能性が大いにあります。読者への課題として残されたと予測します。
しかし、最低でも\emph{``以上''}の記号を\texttt{<u>\&gt;</u>}と符号化することで
我等の勇敢な宇宙飛行士を危険に晒すようなことはしませんでした。



もしあなたが\texttt{sicp.texi}を変更しエラーを訂正したり、アスキーアートを向上させたなら
\code{@set utfversion {utfversion}}の行を更新し、あなたの修正を反映して下さい。
例えば、もしあなたがLythaのバージョン\code{1}で開始し、あなたの名前がBobなら、改訂版は
\code{1.bob1}, \code{1.bob2}, \dots , \code{1.bob\textit{n}}です。また\code{utfversiondate}も更新
して下さい。もしあなたが自分の改訂版をWeb上で配布したいのなら文字列``sicp.texi''を
ファイルやWebページのどこかに埋め込んでおけば人々にとってWeb検索エンジンから
探すことが簡単になるでしょう。



非公式Texinfo形式は寛大にも自由の下に配布された\acronym{HTML}版の魂を引き継いで
いると信じられています。しかし、いつ誰かの法律家の大艦隊が良心に基づく小さな事に対して非常に腹を立て
何かを行わなければならなくなるかもしれません。ですのであなたのフルネームを
使ったり、あなたのアカウントやマシン名を含むInfo, \acronym{DVI}, PostScript, \acronym{PDF}形式
を配布する前に良く良く考えて下さい。

\noindent
\textit{Peath, Lytha Ayth}

\vspace{1.0em}
\noindent
\textbf{付録:}AbelsonとSussmanによる\acronym{SICP}のビデオレクチャーもご覧下さい。\\
\href{http://groups.csail.mit.edu/mac/classes/6.001/abelson-sussman-lectures/}{\acronym{MIT CSAIL}}, 
\href{http://ocw.mit.edu/courses/electrical-engineering-and-computer-science/6-001-structure-and-interpretation-of-computer-programs-spring-2005/video-lectures/}{\acronym{MIT OCW}}.

\vspace{0.5em}
\noindent 
\textbf{付録2:} 上記は2001年の元の\acronym{UTF}の紹介です。
10年後、\acronym{UTF}は一変しました。数学上の記号と式は適切に組版され、図は
ベクターグラフィックにより描かれています。元のテキスト形式とアスキーアートの図
は今でもTexinfoのソースに残っていますが、Info形式でコンパイルした場合のみ
表示されます。電子書籍リーダーとタブレットの夜明けに画面上で\acronym{PDF}を
読むことは正式に、最早バカバカしいことでは無くなりました。楽しんで下さい!

\vspace{0.5em}
\noindent
\textit{A.R, May, 2011}

%============================================================================================================
\chapter*{非公式日本語版}
\addcontentsline{toc}{chapter}{非公式日本語版}
\label{Unofficail Japanese Edition}

SICPはかつて第一版、第二版共に日本にて公式に翻訳が商業出版されていました。
第二版を出版していたピアソン桐原が2013年8月に
\href{http://slashdot.jp/story/13/08/09/0517250/}{ピアソングループから撤退し技術書の取扱を終了したため}、
日本語でSICPを読む機会は失われました。
このことがこの翻訳を行うことの契機となりました。

実際にはその後、2014年1月付近に、寛大にも第二版の訳者、和田英一先生がオンライン上にてSICPの訳書、
\href{http://sicp.iijlab.net/}{「計算機プログラムの構造と解釈」}全文を公開して下さいました。この時点でこの非公式日本語版の
価値は随分と小さくなりました。

しかし、その時、既に3章まで翻訳していたこと、そして非公式TexInfo版が2013年11月に大改訂を行い、
当初の日本語には正式に対応していないtexi2pdfから変更を行い、XeLaTeXを採用したために、
日本語でも美しい組版ができる可能性が出てきたことが、この原稿を廃棄することを押し止めました。

SICPのライセンスについてはインターネットアーカイブにて調べてみました。
2001年1月にMITがSICPを寛大にもオンラインで無料で読むことができるように公開された時にはライセンスが指定されていませんでした。

2008年4月にMITはSICPのライセンスをCC BY-NCと指定しました。その後ライセンスは2011年10月に一旦CC BY-SAに変更されます。
そして2年後の2013年9月に再びCC BY-NCへと戻されました。この事実がSICP原文のライセンスの解釈を難しくしています。
ライセンスの変更はオーナーの自由ですが、ライセンシーはコンテンツ取得時のライセンスを尊重すれば良いからです。

最初に非公式TexInfo版を作成したLytha Aythはライセンス指定の無いSICP公開をWeb文化に基づくものだと理解しました。
次にLaTeXの組版を開発したAndres RabaはCC BY-SAに基き正式な許諾の下、PDF版を作成しました。
私の翻訳はPDF版のライセンスであるCC BY-SAに従うことが求められます。しかし、現在のMITが非商業を求めて
いることを鑑みて、Raba氏に許可を頂いた上で非商業制約を追加した
\href{http://creativecommons.org/licenses/by-nc-sa/3.0/}{\acronym{CC BY-NC-SA} 3.0}
にてリリースすることにしました。

CC BY-NC、及びBY-SAは共に翻訳の許可を明記しています。従ってこの翻訳にはLythaが心配したような法的問題は
起こらないと信じています。しかし同時に、法的問題は常に一方的に起こされることがあることもまた現実です。
従って読者の皆様には常にネットワーク上のデータは(そしてプログラムも!)消えてなくなってしまうシャボン玉で
あることを忘れずに御用心願います。

TeX、LaTeX環境の日本語対応を進めて下さった全ての関係者の皆様に感謝します。
特に最新の情報を常に更新し続けて下さっている\href{http://oku.edu.mie-u.ac.jp/~okumura/texwiki/}{TeX Wiki}の奥村~晴彦氏、
W32TeXを自動でインストールし更新可能な
\href{http://www.math.sci.hokudai.ac.jp/~abenori/soft/abtexinst.html}{TeXインストーラ}作者の阿部~紀行氏、
XeLaTeX向け日本語パッケージ
\href{http://zrbabbler.sp.land.to/zxjatype.html}{``ZXjatype''}
を開発して下さった八登~崇之氏に感謝致します。

海外ではSICPの新しい形の開発が非常に盛んです。PDFはもちろん、epubやインタラクティブ版、Kindle版(mobi形式)、ClojureやJavaScriptに
よるSICP等が公開されています。この翻訳はCC BY-NC-SAですので非商業であればそのような
派生や翻案に利用することが可能です。日本でもSICPの世界が広がっていくことを期待しています。

\vspace{1em}
\noindent
※ 校正御協力者様 (順不同、敬称略)

\begin{itemize}

\item \href{https://github.com/kei-s}{Kei Shiratsuchi}

\item \href{https://github.com/kimurakoichi}{Kimura, Koichi}

\item \href{https://github.com/nna774}{のな}

\end{itemize}

%============================================================================================================


\chapter*{献辞}
\addcontentsline{toc}{chapter}{献辞}
\label{Dedication}



この本を、尊敬と賛美を込めて、コンピュータの中に住む妖精に捧げます。

\begin{quote}
``コンピュータサイエンスに関わる私達にとってコンピュータを使用することを
楽しむことはとても大事だと私は考えます。コンピュータサイエンスが始まった時、それは
とても多くの楽しみに溢れていました。
ご存知のとおり、お金を払うお客様達は時折酷く騙されました。そして暫くして私達は
彼らの不満を真面目に受け取り始めてしまいました。
私達は考え始めてしまったのです。成功裏に、障害の無い完全なコンピュータの使用法に
ついて私達に責任があるのではないかと。
私はそうは思いません。
私は、私達がコンピュータサイエンスを伸展し、新しい方向に向かわせ、
そして仲間達と共に楽しむことに責任があると考えます。
私はコンピュータサイエンスの現場が楽しむことの感覚を失わないことを望みます。
さらに、我々が伝道師になることは望みません。
自分が聖書のセールスマンだとは思わないで下さい。
世界には既にそのような人が溢れています。
あなたが他の人々が学ぶコンピュータ利用法について何を知っているでしょう。
コンピュータ利用に成功する鍵があなたの手の中にのみあるとは決っして思わないで
下さい。
 私が思うに、そして期待することは、あなたの手の中にあるものは知性です。
それはあなたが初めて計算機に出会った時よりもより多くのことを知ることができる
能力であり、それはより多くのことを生むことができるのです。''

\noindent
---Alan J. Perlis (April 1, 1922 -- February 7, 1990)
\end{quote}

%===========================================================================================================


\chapter*{前書き}
\addcontentsline{toc}{chapter}{前書き}
\label{Foreword}

\vspace{-0.6em}


教育者、将軍、栄養士、精神分析医、そして両親はプログラムします。軍隊、学生、そして
いくつかの社会はプログラムされます。大きな問題に対する解決は一連のプログラムを
利用します。それらのほとんどは途中でひょっこり表れます。これらのプログラムは手近な
問題に特化されて現れる成果に溢れています。プログラミングを独立した知的な活動として
理解するためにはあなたはコンピュータプログラミングに向かわねばなりません。
コンピュータプログラムを読み、書かねばなりません。それも数多くです。そのプログラムが何に
ついてであるか、またはどのような適用を担うのかは多くは関係ありません。
重要なことはそれらがどのように実行され、どれだけ滑らかに他のプログラムに対してより
大きなプログラムの作成のために適合するのかです。プログラマは部分の完全性と集合の
妥当性の両方を追求せねばなりません。この本では``プログラム''の使用はデジタル計算機上にて
実行されるためのLispの方言で書かれたプログラムの創造、実行、それに学習に焦点を当てて
います。Lispの使用はプログラム記述の表記法のみを制約、制限し、私達が何をプログラムするか
については影響を与えません。


この本の主題は3つの事象に焦点を当てます。人の心、コンピュータプログラムの集合、そして
コンピュータです。全てのコンピュータプログラムは人の心の中で生まれる現実の、または
精神的な過程のモデルです。これらの過程は人の経験と思考から浮かび上がり、数はとても
多く、詳細は入り組んで、いつでも部分的にしか理解されません。それらはコンピュータ
プログラムにより稀にしか永遠の充足としてモデル化されることはありません。従って、
例え私達のプログラムが注意深く手作りされた別個の記号の集合だとしても、連動する機能の
寄せ集めだとしても、それらは絶えず発展します。私達のモデルの知覚がより深まるにつれ、
増えるにつれ、一般化されるにつれ、モデルが究極的に準安定な位置に逹っするまで変更を
行い、その中には依然として私達が格闘するモデルが存在します。コンピュータプログラミングに
関連する歓喜の源はプログラムとして表現された仕組みの心の中とコンピュータ上で絶え間無く
続く発展であり、それにより生まれる知力の爆発です。もし技巧が私達の夢を解釈するならば、
コンピュータはプログラムとして現わされるそれらを実行するのです!




その力全てに対して、コンピュータは厳しい親方です。そのプログラムは正しくなければ
なりません。私達が伝えたいと望む事柄は委細全て正確に伝えられねばなりません。
全ての他の象徴的な活動と同じく、私達は議論を通してプログラムの真理を確信するように
なります。Lispそれ自身に意味論を割り当てることも可能です。(ところでこれはまた別の
モデルです)。そしてもしプログラムの機能を指定できるのなら、例えば述語論理においてなら、
論理の証明方法が容認可能な正確性の議論に使用できます。残念なことにプログラムが巨大で
複雑になるにつれ、そしてほとんど常にそうなるのですが、仕様の妥当性、一貫性、正確さそれら
自身が疑わしくなります。そのため完全に形式化された正確さの議論は巨大なプログラムには
伴いません。巨大プログラムは小さな物から成長するため正確さに確信を持てる標準的な
プログラム構造の武器庫を開発することは重要です。私達はこれをidiom(イディオム)と呼びます。
そしてそれらを組み合わせて価値が検証された構成技術を用いてより大きな構造にすることを
学びます。これらの技術はこの本の中で長々と扱われます。そしてそれらを理解することは
プログラミングと呼ばれるプロメテウスの進取性(Promethean enterprise)に参加するのに
絶対に必要なことです。他の何事でもなく、強力な構成技術を暴き熟達することは巨大で
重要なプログラムを作成する能力を加速します。反対に、巨大なプログラムを書くことはとても
苦労が多いため、私達は多大な機能や詳細を巨大プログラムに合うように減らす新しい手法
を開発することを促されています。



プログラムとは異なり、コンピュータは物理法則に従わなければなりません。もし
それらを迅速に動かしたいのならば---状態変更当たり2、3ナノ秒で---コンピュータは
電子を極小の距離で転送せねばなりません(高々\(1{1\over2}\) フィート)。
巨大な数の端子により生じる熱は空間に集中しますがこれは取り除かねばなりません。
精緻な工学の技芸が機能の多重度と端子の密度の間のバランスを取るために開発されました。
任意のイベントにおいて、ハードウェアは常に私達がプログラムを行うのに気にするよりも
よりプリミティブなレベルで動作します。私達のLispプログラムを``機械の''プログラムに
変換する処理はそれ自体が私達がプログラムする抽象モデルです。それらの学習と作成は
とても多くの見識をプログラミングの自由裁量なモデルに関連する組織的なプログラムに
対して与えます。もちろんコンピュータそれ自身もそのようにモデル化可能です。そのことを
考えてみましょう。最小の物理スイッチング要素の振舞は量子力学でモデル化され、微分
方程式により記述され、その詳細な振舞は近似値の数値演算により獲得され、それは
コンピュータプログラムにより表現され、それはコンピュータ上で実行され、それは
組み立てられ\dots !



3つの焦点を別々に判別することは戦術上の利便性の問題でしかありません。
例え良く言われるように全てが頭の中にあるとしても、この論理的分割はこれらの焦点の
間の記号的通信量の加速を引き起します。焦点の豊かさ、活力、潜在力は人間の
経験の中で人生自体の発展により増加します。最良時には焦点の間の関係は準安定に
なります。コンピュータは絶対に十分に大きく、速くはなりません。ハードウェア技術の
全ての飛躍的進歩がより大規模なプログラミング計画、新しい組織化原理、抽象モデルの
向上へと導きます。読者の全員が自身に対し繰り返し``どの終点に向かって? どの終端に向かって?''と
問わねばなりません。しかしあまり問い過ぎてもいけません。ほろ苦い哲学の便秘のために
プログラミングの楽しさを逸っしてしまいます。



私達が書くプログラムの間で、いくつか(しかし絶対に十分ではない)は厳格な数学上の
関数、例えばソートや数列の最大値を見つける、素数性判定、平方根を求める等が実行されます。
私達はそのようなプログラムをアルゴリズムと呼びます。多数の物がそれらの最適な振舞を、
特に2つの重要なパラメタである実行時間とデータストレージの必要量に関して知られています。
プログラマは良いアルゴリズムとイディオムを獲得しなければなりません。
例えいくつかのプログラムが厳格な仕様に反しても、それらのパフォーマンスに関して
見積り、常に改善に努めることはプログラマの責務です。


Lispは``生存者''であり約四半世紀の間利用されてきました。活発なプログラミング言語の中で
FortranのみがLispより長い人生を経ています。LispとFortranはどちらもアプリケーションの重要な領域の
プログラミング上の必要性に対処してきました。すなわちFortranは科学計算や工学計算に対して、
Lispは人工知能に対してです。これらの2つの領域は重要で有り続けており、そこに携わっている
プログラマ達はこれら2つの言語に専念しているため、LispとFortranは少なくとももう四半期は
活発に使われ続けることでしょう。


Lispは変化します。このテキストで使用されるScheme方言はオリジナルのLispから発展し
いくつかの重要な手法に関して異なっています。違いには変数束縛に対する静的スコーピングや
関数の値として関数の生成を許可している点等が含まれます。その意味構造においてSchemeは
初期のLispと同等にAlgol 60に近い物です。Algol 60は再び現役となることはないでしょうが、
SchemeとPascalの遺伝子に受け継がれています。
これらの2つの言語の周りに集った言語よりも、もう2つの異なる文化の流通貨幣としての
2つの言語を見つけることのほうが難しいでしょう。
Pascalはピラミッドを建築するための物です---印象的で、息を飲むような、軍隊が重い
ブロックを所定の位置に押すことで建築された静的な構造物です。Lispは有機体を構築
するための物です---印象的で、息を飲むような、小分隊が不安定で無数のより単純な有機体を
所定の位置に嵌め込むことで構築された動的な構築物です。使用された体系化の原則は
両者の場合で同じです。ただし並外れて重要な違いが1つあります。個々のLispプログラマに
委ねられた任意のエクスポート可能な機能の数はPascalの進取性の中に見つかるそれらよりも桁違いに多いのです。
Lispプログラムは機能のライブラリを膨らませます。その機能の実用性はそれらを生成した
アプリケーションを越えます。Lisp生来のデータ構造であるリストがそのような実用性の成長の
大きな原因です。簡単な構造と自然なリストの適用可能性が驚くべき程に非特異的に機能に
反映されています。Pascalでは宣言可能なデータ構造の過剰さがカジュアルな連携を抑止し、ペナルティを科す
機能の中に特殊化することを促しています。1つのデータ構造の上で操作する100の機能を持つほうが
10のデータ構造の上で操作する10の機能を持つよりも優れています。結果としてピラミッドは1000年の間
変わらぬままでいるに違いありませんが、有機体は発展できなければ滅んでしまうのです。


この違いを説明するためにはこの本の中にある教材と課題の扱いを任意の初級課程の
Pascalを用いるテキストのそれと比べてみて下さい。\acronym{MIT}だけが消費できる、
そこで見つかる血統書付きの良馬のためのものという幻想の下で苦悩しないで下さい。学生が
誰であるかとかどこで利用されるかが問題ではありません。まさに、
Lispプログラミングに対して真剣な本はどんな物であるべきかが問題です。



これはプログラミングに関するテキストであることに注意して下さい。人工知能の仕事のための
予習に使われる他の多くのLispの本とは違います。結局、ソフトウェア工学と人工知能の重大な
プログラミングの課題は研究がより大きくなるにつれシステムとして融合する傾向にあります。
このことがなぜそのようなLispへの興味が人工知能の外側で大きくなっているのかを説明します。



誰かがそのゴールから予測したように、人工知能研究は多くの明確なプログラミング上の問題を
生成しました。他のプログラミング文化ではこの相次ぐ問題は新しい言語を生みます。実際に
どんなとても大きなプログラミングタスクにおいても効果的な体系化原理はタスクモジュール内の
情報量を言語の発明を通してコントロールし、分離することです。これらの言語は
私達、人間が最も良く操作を行うシステムの境界へと辿り着くに従いプリミティブではなくなっていく
傾向にあります。結果として、そのようなシステムは何度も複製された複雑な言語処理機能を含みます。
Lispはとてもシンプルな文法と意味論を持ち、パースが初歩的なタスクとして扱えます。
従ってパースの技術はLispプログラムにおいてはほとんどルール無用の役割を演じます。そして
言語処理機の構築は巨大なLispシステムの変化と成長の程度に対しほとんど障害になりません。
最後に、全てのLispプログラマにより負われている義務と自由に対して責任を持つものこそが
このとても単純な文法と意味論です。数行のサイズを越えるLispプログラムなら自由裁量による
関数で満たすことなく書くことはできません。開発し、合わせる。合わせて、また開発する!
括弧の入れ子の中に自身の考えを記述するLispプログラマに乾杯。

\vspace{0.5em}
\noindent
Alan J. Perlis\\
New Haven, Connecticut

%=========================================================================================================


\chapter*{第二版~序文}
\addcontentsline{toc}{chapter}{第二版~序文}
\label{Preface}

\begin{quote}
ソフトウェアが他の何物にも似ていないと言うことはできるでしょうか。それが捨てられる
べき物だと。つまり、常にシャボン玉だと見なすことだと。

---Alan J. Perlis
\end{quote}

\vspace{0.7em}

\noindent
この本の中の教材は1980年から\acronym{MIT}の入門者レベルの計算機科学の科目の中心となる物です。
私達はこの教材を4年間、最初の版が出版された時点で教えてきました。そしてこの第二版が出現する
までにさらに12年が経過しました。私達の成果が広く受け入れられ、他のテキストに取り込まれている
ことを喜ばしく思っています。私達の生徒がこの本の考えとプログラムを学び新しい計算機システムと
言語の核としてそれらを組み込んでいるのを見てきました。古代のタルムードの多義語の文字認識では、
私達の生徒が開発者になってくれました。そのような能力有る学生と熟練した開発者を得たことは
とても幸運なことでした。



この版を準備するにあたって、私達自身の教育上の経験と\acronym{MIT}や他の同僚達からの
コメントにより提案された幾百もの説明を統合しました。この本の中の主なプログラミングシステムの
多くを包括的数値演算システム、インタプリタ、レジスタマシンシミュレータ、コンパイラを含めて再設計しました。
そして全てのプログラム例を、任意の\acronym{IEEE} Scheme標準(\link{IEEE 1990})に従うScheme実装が
そられのコードを実行できることを確実にするために、書き直しました。


この版はいくつかの新しいテーマを重視しています。これらの内、最も重要なものは
計算モデル内での時間を取り扱うための異なる取り組みにより演じられる中心的な役割です。
状態を伴うオブジェクト、並行プログラミング、関数型プログラミング、遅延評価、そして
非決定性プログラミングです。私達は並行性と非決定性に関わる新しい節を含め、そして
このテーマをこの本を通してまとめることを試みました。



この本の第一版は\acronym{MIT}の一学期の科目の講義概要を密接に追っていました。
第二版の全ての新しい教材により、一学期で全てをカバーすることは不可能となりました。
そのためインストラクタは選択をしなければなりません。私達自身の教育現場では、
時々論理プログラミング(\link{Section 4.4})を飛ばします。学生にはレジスタマシンの
シミュレータを使用させるのでその実装(\link{Section 5.2})はカバーしません。
そしてコンパイラ(\link{Section 5.5})は概観のみを大雑把に教えています。それでもこれは
依然として強烈な授業です。何人かのインストラクタは最初の3章から4章のみをカバーし、
他の教材を続きの授業に残したいと願うでしょう。



World-Wide-Webサイト \href{http://mitpress.mit.edu/sicp}{http://mitpress.mit.edu/sicp} はこの本のユーザへのサポートを提供します。
これにはこの本のプログラム、プログラミング課題のサンプル、補助教材、ダウンロード可能なLispの
Scheme方言の実装が含まれます。

%============================================================================================================

\chapter*{第一版~序文}
\addcontentsline{toc}{chapter}{第一版~序文}
\label{Preface 1e}

% \vspace{-0.6em}
\begin{quote}
コンピュータはヴァイオリンのような物です。初心者が最初に蓄音機、そして次に
ヴァイオリンを試すことを想像して下さい。彼は後者の音は酷いと言います。
これが人間主義者と多くの計算機科学者から聞こえてくる議論です。
計算機のプログラムは特定の目的には良い物だ、しかし柔軟性が無いと彼らは言います。
ヴァイオリンやタイプライタだって同じです。あなたがその使い方を学ぶまでは。

---Marvin Minsky, ``Why Programming Is a Good Medium for Expressing
Poorly-Understood and Sloppily-Formulated Ideas''
\end{quote}

\vspace{0.8em}

\noindent
``The Structure and Interpretation of Computer Programs''(SICP, 計算機プログラムの構造と解釈)は
マサチューセッツ工科大学(MIT)での入門者レベルの
計算機科学の科目です。\acronym{MIT}にて電気工学、または計算機工学を専攻する全ての
学生が``共通コアカリキュラム''の4つの内の1つとして履修しなければなりません。
共通コアカリキュラムは回路と線形システムについて2つの科目とデジタルシステムの
設計についての科目を含みます。私達はこの科目の開発を1978年から行なってきました。
そしてこの教材を現行様式として1980年の秋から、600名から700名の学生に毎年、教えて
きました。これらの学生の多くは少ししか、または全く事前に公式な計算機利用についての
トレーニングを受けてはいませんでした。ただし、多くは事前に計算機で少々遊んだ経験が
有り、ほんの少数は広範囲のプログラミングの経験やハードウェア設計の経験がありました。

私達のこの計算機科学の入門科目の設計は2つの主な関心事を反映しています。1つは、
コンピュータ言語はコンピュータに命令を実行させるための単なる方法等ではなく、
新しい種類の方法論に関する考えを表現するための公式なメディアであるという考えを
証明することです。従ってプログラムは人々が読むために書かれねばならず、そして
ただ偶然に機械にとって実行する物でなければなりません。2つ目は、このレベルの
科目により扱われる本質的な教材とは、特定のプログラミング言語が構築する構文ではなく、
また特定の関数を効率的に演算するための賢いアルゴリズムでもなく、増してアルゴリズムと
演算基盤の数理解析でないという信念です。そうではなく、大きなソフトウェアシステムの
知的な複雑性をコントロールするために用いる技術でなければなりません。

私達の目標は、この教科を完了した学生がプログラミングの美学とスタイルの原理に対して
必ず良い感触を得ることです。学生達が大きなシステムの複雑性をコントロールするための
主な技術の能力を得られなければなりません。学生達が50ページの長さのプログラムを、
それが模範的なスタイルで書かれているのならば、読めるようにならなければなりません。
学生達がプログラムの変更を行う時に、元の作者の魂とスタイルを維持しながら安心できな
ければなりません。

これらのスキルは決してコンピュータプログラミングに対して独自なことではありません。
私達が教え、利用する技術は全ての工学設計に対して共通な物です。私達は
適切な場合に、詳細を隠す抽象概念を構築することにより複雑性をコントロールします。
標準的な、良く理解された部品を``mix and match''(様々な物をうまく組み合わせる方法)の方法により
組み合わせることにより、システムを構築することを可能にする慣習的なインターフェイスを
確立することで、複雑性をコントロールします。私達は設計を記述するための新しい言語を
確立することで複雑性をコントロールします。そして各言語は設計の特定の側面を重要視し、
他の側面の重要性を緩和します。

私達のこの教科に対する取り組み方の根底を成す物は、``計算機科学''は科学ではなく、
その意義は計算機とは関係が無いという信念です。計算機革命とは私達の考え方と
私達の考えの表現方法における革命です。この変化の本質を恐らく最もうまく言い表わす
のは\newterm{procedural epistemology}(\jnewterm{手続的認識論})---古典的な数学上の主題により
取られるより宣言的な視点に対立する、命令型の視点からの知識構造の研究---の出現でしょう。
数学は``何であるか''の概念を正確に扱うためのフレームワークを提供します。計算機の使用は
``行い方''の概念を正確に扱うためのフレームワークを提供します。

私達の教材を教えるにあたって、プログラミング言語Lispの一方言を使用します。
私達は正式にこの言語を教えることはしません。する必要がないからです。
ただそれを使用し、そして学生は2、3日で習熟してしまいます。これはLispの様な
言語の1つの利点です。これらの言語は複合式を形成する方法があまり多くありません。
そしてほとんど構文構造が存在しません。形式的な特性の全ては一時間もあれば
カバーできます。まるでチェスのルールの様なものです。少しの時間の後にはこの
言語の構文上の詳細を忘れてしまいます。(ほとんど存在しないからです)。そして
本当の問題---私達が演算したい物を把握すること、どのように問題を扱いやすい
部分へと分解するか、そしてどのようにその部品上で働くかについて取り掛かります。
Lispのもう1つの利点は私達が知っている他のどの言語よりもプログラムを分解した
モジュラに対するより多くの大規模な戦略をサポートする(しかし強制はしない)ことです。
手続化とデータ抽象化を行い、公開関数を用いて処理の共通なパターンを獲得し、代入と
データの変更を用いて局所状態のモデル化を行い、プログラムの部品をストリームと遅延評価に
結び付け、簡単に組込言語を実装することができます。これら全てがインタラクティブ(相互作用)な
環境にインクリメンタル(漸増的な)プログラム設計、構築、テスト、デバッグのための
優れたサポートと共に組込まれています。私達は前例の無い力と洗練さを供えた素晴しいツールを
創り出したJohn McCarthyを始めとする全ての世代のLisp wizard(ウィザード、魔法使い、
最上級のプログラマの賞賛を込めた呼び名)に感謝します。

私達が用いるLispの方言、SchemeはLispとAlgolの力と洗練を一緒にもたらそうとしました。
Lispからは単純な構文から導き出されるメタ言語の力、データオブジェクトとしてのプログラムの
単一の表現、ガベージコレクションを持つヒープ上に取得されるデータを得ました。
AlgolからはAlgol委員会に在籍したプログラム設計の開拓者からの贈り物である
レキシカルスコープとブロック構造を得ました。私達はJohn ReynoldsとPeter Landinの
Church(チャーチ)の\(lambda\)-calculus(ラムダ計算)のプログラミング言語の構造に
対する関係についての彼等の洞察に対して言及したいと願います。
またコンピュータがこの世界に現れる何十年も前にこの領域を偵察された数学者達に対する
恩義も忘れておりません。これらの開拓者にはAlonzo Church, Barkley Rosser, Stephen Kleene,
Haskell Curry等が含まれております。

%============================================================================================================

\chapter*{謝辞}
\addcontentsline{toc}{chapter}{謝辞}
\label{Acknowledgements}


この本とこのカリキュラムの開発を手助けして下さった多くの人々に感謝致します。



私達の教科は明らかに1960年代の終わりに\acronym{MIT}にてJack Wozencraftと
Arthur Evans, Jr.により教えられたプログラミング言語学と\( \lambda \)計算上の
素晴しい科目、``6.231''の知的末裔です。



私達はRobert Fanoに大きな借りがあります。彼は\acronym{MIT}の電気工学と計算機科学の
導入部のカリキュラムを再編成し、工学設計の原理を重視しました。彼はこの進取性への
着手に導き、またこの本への発展の元となる最初の教科ノートのまとめを記述しました。



私達が教えようとするプログラミングのスタイルと美学の多くは
Guy Lewis Steele Jr.の協力の下に開発されました。彼は初期のSchemeの開発において
Gerald Jay Sussmanと協力を行いました。加えてDavid Turner, Peter Henderson, 
Dan Friedman, David Wise, Will Clingerが私達にこの本の中に現れる関数型プログラミングの
テクニックの多くを教えてくれました。



Joel Mosesは私達に巨大システムの構造化について教えてくれました。彼の記号演算のための
Macsymaシステムにおける経験が、人は制御の複雑性を回避し、データの体系化に集中して
モデル化されていく世界の真の構造を反映するべきだという見識を与えてくれました。


Marvin MinskyとSeymour Papertは私達のプログラミングに関する態度の多くと、
私達の知的な生活内にそれの場所を形作りました。彼等に対して、考えを探求するための
式の意味を演算が与えることについての理解に借りがあります。そうでなければ、
正確に取り扱うためには複雑過ぎることになってしまいます。彼らは学生のプログラムを
書き、変更する能力が、その中で探求が自然な活動になる強力なメディアを提供すると
強調します。



私達はまたプログラミングは大いに楽しく、このプログラミングの楽しみをサポートする
ために十分に注意しなければならない点についてAlan Perlisに強く同意します。
この楽しみの一部は作業中の偉大な職人達を観察することから得られます。
私達は幸運なことに、Bill GosperとRichard Greenblattの下で見習いプログラマで
いることができました。



私達のカリキュラムの開発に貢献して下さった全ての人々を特定することは難しいことです。
私達は過去15年私達と共に働き、多くの時間を私達の教科に費してくれた全ての講師、
口答の指導者、チューターに、
特に、Bill Siebert, Albert Meyer,
Joe Stoy, Randy Davis, Louis Braida, Eric Grimson, Rod Brooks, Lynn Stein and
Peter Szolovitsに感謝します。
私達は特に卓越した教育上の貢献として現在はウェルズリーのFranklyn Turbakに感謝します。
彼の学部生向け指導要項は私達皆が目指す基準を打ち立てました。
Jerry SaltzerとJim Millerには私達が並行性のミステリーに取り組むのを手助けして下さった
ことに感謝します。そしてPeter SzolovitsとDavid McAllesterには\link{Chapter
4}における非決定性評価の説明に対する貢献に感謝します。



多くの人々は他大学でこの資料を紹介するのに大きな努力を費してくださいました。
私達が親密に働いたそれらの人々の幾人かはイスラエル工科大学のJacob Katzenelson、
カリフォルニア大学アーバイン校のHardy Mayer、オックスフォード大学のJoe Stoy、
パデュー大学のElisha Sacks、ノルウェー技術科学大学のJan Komorowskiです。
私達は他大学においてこの科目を受け入れることで主要な教育の賞を受けた同僚達を
非常に誇りに思います。この中にはイェール大学のKenneth Yip、カリフォルニア大学
バークリー校のBrian Harvey、コーネル大学のDan Huttenlocherを含みます。


Al Moy\'eは私たちのためにこの教材をHPの技術者達に教える手筈とこのレクチャーのビデオ
テープの製品化を準備してくれました。私たちはまた才能あるインストラクター達にも
感謝致します。具体的にはJim Miller, Bill Siebert, Mike Eisenbergです。彼等は
これらのテープを組み込んで生涯教育のコースを設計し、世界中の大学と業界にて
教育を行いました。



他国の多くの教育者が多大な時間を第一版の翻訳に費して下さいました。
Michel Briand, Pierre Chamard, and Andr\'e Picはフランス語版をプロデュースして下さいました。
Susanne Daniels-Heroldはドイツ語版をプロデュースして下さいました。
元吉文男は日本語版をプロデュースして下さいました。私たちはどなたが中国語版を
プロデュースして下さったのか知りません。しかし``未許可''の翻訳の題材として
選ばれたことを光栄に思います。



私たちが教育の目的のために使用するSchemeシステムの開発に技術的な貢献をされた全ての
人々を列挙することは難しいことです。Guy Steeleに加えて、主要なウィザードの中にはChris Hanson, Joe
Bowbeer, Jim Miller, Guillermo Rozas, Stephen Adamsが含まれます。
多大な時間を費して下さった他の人々はRichard Stallman, Alan Bawden, Kent Pitman, Jon Taft,
Neil Mayle, John Lamping, Gwyn Osnos, Tracy Larrabee, George Carrette, Soma
Chaudhuri, Bill Chiarchiaro, Steven Kirsch, Leigh Klotz, Wayne Noss, Todd Cass,
Patrick O'Donnell, Kevin Theobald, Daniel Weise, Kenneth Sinclair, Anthony
Courtemanche, Henry M. Wu, Andrew Berlin, それにRuth Shyuです。


\acronym{MIT}の実装を越えて、私たちは\acronym{IEEE}のScheme標準仕様について
働いた多くの人々に感謝したいと思います。\( \rm R^4RS \)を編集したWilliam Clingerと
Jonathan Rees、\acronym{IEEE}標準を準備したChris Haynes, David Bartley,
Chris Hanson, Jim Millerを含みます。


Dan Friedmanは長い間Schemeコミュニティのリーダーでした。コミュニティの広範な仕事は
言語設計の問題を越えて、Schemer's Inc.によるEdSchemeを基にした高校生向けカリキュラムや
Mike EisenbergやBrian HarveyとMatthew Wrightによる素晴しい本のような、特筆すべき
教育上のイノベーションを含むまでに至りました。



私たちはこの本を現実にすることに貢献して下さった人々の働きに感謝致します。
特に\acronym{MIT}出版のTerry Ehling, Larry Cohen, Paul Bethgeです。
Ella Mazelは素晴しいカバーの絵を見つけてくれました。第二版に対しては特にこの本の
デザインを助けてくれたBernardとEllaのMazel夫妻、非凡な{\TeX}ウィザードである
David Jonesに感謝致します。私たちはまた新しいドラフトに対し洞察力のあるコメントを
して下さった読者の方々、Jacob Katzenelson, Hardy Mayer, Jim Miller, そして特に
Brian Harveyに対して、Julieが彼の本\textit{Simply Scheme}に行ったように、この本に
行ってくれたことを感謝致します。



最後に、何年にも渡ったこの仕事を励まして下さった組織のサポートに感謝したいと思います。
Hewlett-Packardからのサポートを可能にして下さったIra GoldsteinとJoel Birnbaum、
それに\acronym{DARPA}からのサポートを可能にして下さったBob Kahnを含みます。

%============================================================================================================
%\pagenumbering{arabic}

\mainmatter

\chapter{手続を用いた抽象化の構築}
\label{Chapter 1}

\begin{quote}
心がその中で、その力を単純な考えの上に発揮する``心の働き''は、主としてこれら3つです。1. いくつかの
簡単な考えを1つの複合物に組み合わせます。全ての複雑な考えはこのようにして作られます。
2. 2つの考えをそれが簡単であるか複雑であるかに係らず一緒にもたらし、お互いに
合わせることでそれらを統合することは無しに、全ての関係性の考えを得ることで、一度に
それらを見渡します。3. 考えをそれらの実在に付随する全ての他の考えから分離します。
これは抽象化と呼ばれ、このようにして全てのその一般的な考えは作られます。

---John Locke, \emph{An Essay Concerning Human Understanding} (1690)
\end{quote}

\vspace{1.0em}

\noindent
私達は\newterm{computational process}(\jnewterm{演算プロセス})について学びます。演算プロセスとは抽象的な存在で
コンピュータ
の中に複数が住んでいます。それらが進化するとプロセスは\newterm{data}(\jnewterm{データ})と呼ばれる
また別の抽象的な物を扱います。プロセスの進化は\newterm{program}(\jnewterm{プログラム})と呼ばれるルールの
パターン(型、類型)により命じられます。人はプログラムを作成してプロセスに命ずるのです。
つまり私達はコンピュータの精霊に私達の呪文で魔法をかけるのです。

演算プロセスは本当に魔法使いの精霊の考えに似ています。それは見たり触れたりは
できません。物理的な物では構成されていません。しかしとてもリアルな存在です。
知的な仕事を行います。質問にも答えます。銀行でお金を払ったり、工場でロボットの
腕をコントロールすることで世界に影響を与えることも可能です。私達が利用する
プロセスに魔法をかけるプログラムとは魔法使いの呪文のような物です。難解な秘伝の
\newterm{programming languages}(\jnewterm{プログラミング言語})の中で記号的表現にて慎重に組み立てられプロセスに
実行してほしいタスク(仕事、任務)を指示します。



演算プロセスは、正しく動くコンピュータでは、精密に正しくプログラムを実行します。
従って魔法使いの見習いのように、初心者のプログラマは魔法の結果について理解し、
予測することを学ばねばなりません。例えプログラムの小さなエラー
(通常は\newterm{bugs}(\jnewterm{バグ})、または\newterm{glitches}(\jnewterm{グリッチ}、誤作動)と呼ばれます)でも
複雑で予測不可能な結果をもたらす場合もあるのです。



幸運なことに、プログラムを学ぶことは魔法を学ぶことより少しも危険ではありません。
私達が相手にする精霊は都合良く安全な方法で封じ込まれています。
しかし、実際の世界でのプログラミングには注意力、専門知識、堅実さを必要とします。
例えばCAD(計算機による設計支援)プログラムの小さなバグが飛行機や
ダムの最悪な崩壊に繋ったり、工業ロボットの自己破壊を起こしたりします。



ソフトウェアエンジニアのマスター達は最終的にプロセスが望まれたタスクを実行する
ことに自信を持てるだけの技能を、プログラムの構築に対して持っています。
彼らは前もってシステムの行いを図で説明することができます。予測不可能な問題が
最悪な結果をもたらさぬようプログラムをどのように構造化を行うのか知っています。
そして問題が発生した時にはプログラムの \newterm{debug}(\jnewterm{デバッグ}、バグ取り)を行えます。
良い設計のコンピュータシステムは、良い設計の自動車や原子炉のように
モジュール方式で設計されており、パーツは個別に組み立て、置き換え、デバッグが
可能です。

\subsubsection*{Lispプログラミング}



私達はプロセスを記述するのに適切な言語を必要とします。この目的に対しプログラミング
言語Lispを利用します。私達の日々の考えが通常、自然言語(例えば英語やフランス語、
日本語)で表されるように、定量的な現象が数学の記号で表されるように、手続的な
思考はLispで表現されます。Lispは1950年代後半に\newterm{recursion equations}(\jnewterm{再帰方程式})
と呼ばれるある種の論理表現に関する推論のための形式化として開発されました。
この言語はJohn McCarthyにより着想され、彼の論文``記号式の再帰方程式とそれらの機械による演算''
(\link{McCarthy 1960})を基にしています。



数学上の形式主義としての始まりにも関わらず、Lispは実用的なプログラミング言語
です。Lisp\newterm{interpreter}(\jnewterm{インタプリタ}、逐次翻訳処理器)はLisp言語にて記述されたプロセスを実行する
機械です。最初のLispインタプリタはMcCarthyと\acronym{MIT}研究所の人工知能部門の
同僚、学生による手助けにて実装されました。\footnote{
\textit{Lisp 1 Programmer's Manual}は1960年に初出し、
\textit{Lisp 1.5 Programmer's Manual} (\link{McCarthy et al. 1965})は1962年に出版
されました。Lispの初期の歴史は\link{McCarthy 1978}にて説明されています。}
Lispはその名前をLIst Processing(リスト処理)の頭文字から取っており、
記号微分や代数式の積分の様なプログラミング上の問題に着手するための
記号操作能力を提供するために設計されました。この目的のためにアトムとリストとして
知られる新しいデータオブジェクトを含みます。これはその時代の他の全ての言語から
著しく際立たせる物でした。



Lispは計画的な設計の取り組みから生まれた製品ではありませんでした。そうではなく、
非公式に試験的なやり方で、ユーザの要求と実利的な実装上の考慮への対応として発展しました。
Lispの非公式な進化は何年も続き、Lispユーザのコミュニティは伝統的に言語の``公式な''
どんな定義の公表に対しても抵抗しました。この進化は初期構想の柔軟性と洗練さと共に、今日
世界中で広く使用される言語で2番目に古い(Fortranのみがより古い)言語として、Lispに継続的に
最新のプログラム設計についての考えを受け入れることを可能にしてきました。
従ってLispは今では複数の方言の系統が存在し、それらはオリジナルの機能の多くを共有しながらも、
お互いに大きな違いを持ちます。この本で使用されるLispの方言はSchemeと呼ばれます。\footnote{
1970年代に最もメジャーなLispプログラムの記述に用いられた2つの方言は\acronym{MIT}のプロジェクト
\acronym{MAC}で開発されたMacLisp (\link{Moon 1978}; \link{Pitman 1983})と
Bolt Beranek and Newman Inc.とXerox Palo Alto研究センターににて開発された
Interlisp (\link{Teitelman 1974})でした。Portable Standard Lisp (\link{Hearn 1969}; \link{Griss 1981})
は簡単に、異なるマシンの間で移植可能にするよう設計されたLisp方言です。MacLispは
カリフォルニア大学バークレー校により開発されたFranz Lispや\acronym{MIT}人工知能研究所が
Lispをとても効率良く実行するために設計した特定目的プロセッサ(処理機)をベースにした
Zetalisp (\link{Moon and Weinreb 1981})といったいくつかの下位方言を生みました。
この本で使用するLisp方言はSheme (\link{Steele and Sussman 1975})と呼ばれ、1975年に
\acronym{MIT}人工知能研究所のGuy Lewis Steele Jr.とGerald Jay Sussmanにより開発され、
後に\acronym{MIT}にて教育目的のために再実装されました。
Common Lisp (\link{Steele 1982}, \link{Steele 1990})はLispコミュニティにより初期のLisp方言の
機能を集約し、Lispの業界標準を作成するために開発されました。Common Lispは1994年に
\acronym{ANSI}標準(\link{ANSI 1994})になりました。
}



実験的であるという特徴と記号操作の重要性のため、Lispは初期においては数値演算に対し
少くともFortranとの比較にてとても非効率でした。しかし年を追って、プログラムを
機械語に変換し、数値演算を適度に効率良く実行可能なLispコンパイラが開発されました。
特別なアプリケーションに対してはLispは最高の効果を発揮しています。\footnote{
そのような特別なアプリケーションの1つは自然科学上の重大な計算、太陽系の動きの統合
におけるブレイクスルーでした。これは以前の結果より二桁も良く、太陽系の活動が混沌
であることを実演しました。この計算は全てLispで書かれたソフトウェアツールの手助け
により実装された新しい統合アルゴリズム、特定目的のコンパイラ、特定目的の計算機に
より可能となりました。(\link{Abelson et al. 1992}; \link{Sussman and Wisdom 1992})}
Lispは今でもどうしようもなく非効率であるという古い評判を乗り越えられてはいませんが、
Lispは今では多くのアプリケーションにて、効率が問題の中心ではない場合において
利用されています。例えばLispはOSのシェル言語やエディタの拡張言語、CADシステム等に
おいて選択言語となっています。



もしLispがメインストリームの言語でなければなぜ私達はプログラミングの議論のための
フレームワークとしてそれを用いるのでしょうか? なぜならこの言語は
重要なプログラミング構成概念とデータ構造を学ぶため、またそれらをサポートする
言語上の機能にそれらを関連付けするために、言語自身を洗練された媒体と成す個有の機能
を持っているためです。
これらの機能で最も著しい物は、Lispによる\newterm{procedures}(\jnewterm{手続})と
呼ばれるプロセスの記述が、それ自身がLispのデータとして表現され、また操作されることが可能
であるという事実です。これの重要性は、伝統的な``受動的な''データと``能動的な''
プロセスとの間の区別をぼかす能力に依存する、強力なプログラム設計のテクニックが存在する
ということです。私達がそれを発見するにつれ、手続をデータとして扱うLispの柔軟性は
Lispをこれらのテクニックを探求するのに、既存で最も便利な言語の1つとします。
手続をデータとして表現する能力はまた、Lispを他のプログラムをデータとして操作しなければ
ならないプログラムを書く目的に対し洗練された言語にします。例えばコンピュータ言語に
対応するインタプリタやコンパイラのようなプログラムです。これらの考慮点に加えて、
Lispによるプログラミングはとても楽しいのです。



\section{プログラミングの要素}
\label{Section 1.1}



強力なプログラミング言語はコンピュータにタスクの実行を指示するだけではありません。
そのような言語は私達がプロセスについての自らの考えを体系化するフレームワークとして
の役目を担います。従って言語を記述する時、簡単なアイデアを組み合わせてより複雑な
アイデアを形成するという手段をその言語が提供することには特に注意を払わねばなりません。
強力な言語全てがこれを達成するために3つのメカニズムを持っています。

\begin{itemize}

\item \textbf{プリミティブな式},
言語に関わる最も単純な要素を表現する

\item \textbf{合成化の手段},
これにより、より単純なものより複合要素が構築される

\item \textbf{抽象化の手段},
これにより複合要素は名前を付けて個体として扱える

\end{itemize}

\noindent
プログラミングにおいては2つの種類の要素を扱います。手続(procedure)とデータです。
(後でそれらはあまりはっきりとは区別できないことを明かします。)
簡単に説明するとデータは操作対象の``物''で手続はデータの操作のためのルールの
記述です。従って強力なプログラミング言語はどれもプリミティブ(原始的な、最低レベルの、組込の)なデータと
プリミティブな手続を記述可能でなければならず、また手続とデータを
合成化、抽象化する手法を持たなければなりません。

この章では単純な数値データのみを扱うことにより、手続構築のためのルールに
集中します。\footnote{数値を``単純なデータ''と特徴付けるのは公然なウソです。
実際に数値の扱いは任意のプログラミング言語において最も油断ならない、混乱を招く
要素です。いくつかの典型的な問題は次のものです。いくつかのコンピュータシステムは
2のような\newterm{integers}(\jnewterm{整数})と2.71のような\newterm{real numbers}(\jnewterm{実数})の区別をします。実数2.00は
整数2とは異なるでしょうか? 整数に用いられる算術演算は実数に対する物と同じでしょ
うか? 6を2で割ったら3? それとも3.0? どれだけ大きな数値を表示できますか? 精度は
小数何桁まで正しく表わされますか? 整数の範囲は実数の範囲と同じですか? もちろん
これらの質問の他にも丸めと切り捨てに関する誤差の問題の蓄積といった数値解析の科学全体が
存在します。この本のフォーカスは大規模なプログラム設計であり数値演算向けの
テクニックではないのでこれらの問題は無視することにします。この章の数値演算の
例では非整数演算において精度上正確な桁数に制限を持つ算術演算を用いる場合に
一般的な丸めの方法を示します。}後の章では同じこれらのルールにより複合データも
また構築できることを学びます。




\subsection{式}
\label{Section 1.1.1}

プログラミングを始める1つの簡単な方法はいくつかの典型的な対話を
Lispの方言であるSchemeのインタプリタを用いて試してみることです。コンピュータ
の端末の前に座っていると想像してみてください。あなたが\newterm{expression}(\jnewterm{式})を
入力するとインタプリタはその式の\newterm{evaluation}(\jnewterm{評価})の結果を表示することで
応答します。

あなたが入力するプリミティブな式の一種として数値があります。(より正確には
あなたが入力する式は10進数の数値を表す数字から成り立ちます。)
もし数値をLispに与えた場合、

\begin{scheme}
486
\end{scheme}

\noindent
インタプリタは以下を表示することで応答します。\footnote{この本を通して、ユーザの入力
とインタプリタが表示した応答を区別したい場合、傾いた文字で表します。}

\begin{scheme}
~\textit{486}~
\end{scheme}

\noindent
数値を表す式はプリミティブな手続を表す式(例えば\code{+} や \code{*})
と接続することで複合式を形成し、それら数値に対し手続を適用することを表現します。
例えば:

\begin{scheme}
(+ 137 349)
~\textit{486}~
\end{scheme}

\begin{scheme}
(- 1000 334)
~\textit{666}~
\end{scheme}

\begin{scheme}
(* 5 99)
~\textit{495}~
\end{scheme}

\begin{scheme}
(/ 10 5)
~\textit{2}~
\end{scheme}

\begin{scheme}
(+ 2.7 10)
~\textit{12.7}~
\end{scheme}

\noindent
これらのような式は括弧の中の式のリストを区切ることにより形成され手続の適用を示し、
\newterm{combinations}(\jnewterm{組み合わせ})と呼ばれます。リストの最も左の要素は\newterm{operator}(\jnewterm{オペレータ}、\jnewterm{演算子})と呼ばれ、
他の要素は\newterm{operand}(\jnewterm{オペランド}、被演算数)と呼ばれます。組み合わせの値はオペレータにより
与えられた手続をオペランドの値である\newterm{arguments}(\jnewterm{引数})に適用することで得られます。


オペレータをオペランドの左に置く決まりは\newterm{prefix notation}(\jnewterm{前置表記法})として
知られています。最初の内は数学の決まりから明らかに逸脱するので混乱するかもしれません。
しかし、前置表記法にはいくつかの利点が存在します。その1つは以下の例のように、
任意の数の引数を取る手続に適応できることです。

\begin{scheme}
(+ 21 35 12 7)
~\textit{75}~
\end{scheme}

\begin{scheme}
(* 25 4 12)
~\textit{1200}~
\end{scheme}

\noindent
曖昧さが全くありません。オペレータが常に最も左の要素であり、合成全体
は括弧で区切られているためです。



前置表記法の2つ目の利点は直接的な方法にて組み合わせを\textit{nested}(ネスト、入れ子)にすることが可能です。
つまり、組み合わせの要素それ自体が組み合わせである場合です。

\begin{scheme}
(+ (* 3 5) (- 10 6))
~\textit{19}~
\end{scheme}

\noindent
原理的にはそのようなネストの深さとLispインタプリタが評価可能な式全体の複雑さには
制限がありません。しかし私達人間は以下のような比較的単純な式でも混乱してしまいます。

\begin{scheme}
(+ (* 3 (+ (* 2 4) (+ 3 5))) (+ (- 10 7) 6))
\end{scheme}

\noindent
インタプリタは直ちに57だと評価するでしょう。このような式を次のような形式で
記述することで私達自身を助けることが可能です。

\begin{scheme}
(+ (* 3
      (+ (* 2 4)
         (+ 3 5)))
   (+ (- 10 7)
      6))
\end{scheme}

\noindent
\newterm{pretty-printing}(\jnewterm{プリティプリント}、整形)として知られるフォーマットの決まりに個々の
長いオペランドを従わせることで、オペランドが垂直方向で位置合わせされます。
結果的に式の構造が明確にインデント(字下げ)されることになります。\footnote{典型的な
Lispシステムは式を整形しユーザを手助けするための機能を提供します。特に便利な
2つの機能において、1つは新しい行がどこで始まろうとも自動的に正しい整形位置に
インデントします。もう1つは右括弧が入力された時に対応する左括弧がハイライト
されます。}



例え複雑な式でもインタプリタは常に同じ基本的なサイクルにて処理を行います。
式を端末から読み、その式を評価し、結果を表示します。この操作モードはしばしば
インタプリタが\newterm{read-eval-print loop}(REPL:\jnewterm{レプル})で実行されていると
呼ばれます。特に明示的にインタプリタに式の値を表示しろと命令する必要がないこと
に注意して下さい。\footnote{Lispは各式が値を持つという決まりに従います。Lispが
非効率な言語であるという古い噂と共に、この決まりがAlan PerilsによるOscar Wilde
をもじった皮肉のネタ元になっています。曰く``Lispプログラマは全ての値を知っている
がそのコストはどれについても知らない''}


\subsection{名前付けと環境}
\label{Section 1.1.2}
プログラミング言語の重要な特徴は演算対象を参照するための名前を利用するために
それが提供する手段です。名前は\newterm{value}(\jnewterm{値})としてオブジェクトを持つ\newterm{variable}(\jnewterm{変数})を
識別します。


Lispの方言Schemeでは対象に\code{define}(定義)を用いて名前を付けます。
以下のように入力すると

\begin{scheme}
(define size 2)
\end{scheme}

\noindent
インタプリタは名前\code{size}と値2を関連付けます。\footnote{この本ではインタプリタの
定義に対する評価の応答を表示しません。とても実装依存であるためです。}
一度名前\code{size}が数値2に関連付けられれば値2を名前で参照することが可能です。

\begin{scheme}
size
~\textit{2}~
\end{scheme}

\begin{scheme}
(* 5 size)
~\textit{10}~
\end{scheme}

\noindent
より多くの\code{define}の使用例を見ましょう。

\begin{scheme}
(define pi 3.14159)
(define radius 10)
(* pi (* radius radius))
~\textit{314.159}~
(define circumference (* 2 pi radius))
circumference
~\textit{62.8318}~
\end{scheme}

\noindent
\code{define}は言語の最も単純な抽象化の手段です。簡単な名前を使用して合成命令の
結果を参照することを可能にします。例えば上の例で計算した\code{circumference}(円周)
です。一般的に演算対象はとても複雑な構造を持ち、それを覚えて使用時に詳細を
繰り返し記述することはとても面倒です。実際に、複雑なプログラムは少しづつ
複雑さを増していく演算対象を1つづつ構築して組み立てられます。インタプリタは
このプログラム組立の各ステップに特に便利です。なぜなら名前とオブジェクトの
関連性が連続した対話を通して少しづつ作成可能なためです。この機能は
漸進的開発とプログラムのテストを促進し、Lispプログラムが通常数多くの比較的
単純な手続により構成される理由です。



値に記号を関連付け、後にそれらを取り出すことがあるのは、インタプリタが名前と
オブジェクトのペアを追跡するためのある種のメモリを持たなければいけないことを
意味することは明白でしょう。このメモリは\newterm{environment}(\jnewterm{環境})と呼ばれます。(より正確
には\newterm{global environment}(\jnewterm{グローバル環境}、大域環境)、演算には複数の異なる環境が利用され
ることを後に学ぶため)\footnote{\link{Chapter 3}にてこの環境という
概念がインタプリタがどのように働くか、またどのようにインタプリタを実装するかに
おいて重要であることを示します。}


\subsection{組み合わせの評価}
\label{Section 1.1.3}
この章の目標の1つは手続的な思考上の問題を分離することです。
代表例として、組み合わせの評価においてインタプリタはそれ自身が手続に従うことを
考えてみましょう。

\begin{quote}
組み合わせを評価するため、以下を行います

\begin{enumerate}

\item
組み合わせの部分式を評価する

\item
最も左の部分式(オペレータ)の値である手続を他の部分式の値である引数(オペランド)
に対し適用する

\end{enumerate}
\end{quote}

\noindent
この単純なルールでさえ、一般的な過程におけるいくつかの重要な点を示します。
最初に第一のステップが組み合わせの評価過程を達成すためには先に組み合わせの各要素の
評価過程の実行を行う必要があります。従って評価ルールは事実上
\newterm{recursive}(\jnewterm{再帰})的です。つまり評価ルールの1ステップとしてそれ自身を
実行する必要性があります。\footnote{評価ルールが第一のステップの部分として
組み合わせの最も左の要素を評価しなければいけないというのは奇妙に写るかもしれませ
ん。この時点ではそれは\code{+}や\code{*}が表す足し算やかけ算のような組込の
プリミティブな手続でしかないためです。後に組み合わせのオペレータそのものが組み合わせ
である場合を扱えることが便利であることを学びます。}



再帰の考えがいかに簡潔に、深くネストした複合式を表現できるかに注目してください。
再帰でなければとても複雑な経過に見えてしまうでしょう。例えば以下の式を評価して
みます。

\begin{scheme}
(* (+ 2 (* 4 6))
   (+ 3 5 7))
\end{scheme}

\noindent
この式は評価ルールが4つの異なる組み合わせに適用される必要があります。この過程を
式の組立を木形式にて表現することで図解することが可能です。\link{Figure 1.1}を
ご覧下さい。各組み合わせは枝の付いたノードで表され、枝にはオペレータと別の組み合わせ
への茎となるオペランドが付いています。終端ノード(他のノードへのと続く枝の無い物)
はオペレータか数値を表しています。評価を木の用語で表すと、オペランドの値は
上へと流れていくことが想像できます。終端ノードから始まり上のレベル、さらに上のレベルにて
合成されます。一般的に、再帰は階層的な木のような対象を扱うのにとても強力な技術
です。実際に評価ルールの``値を情報に流す''形式は\newterm{tree accumulation}
(\jnewterm{集積木})として知られます。


\begin{figure}[tb]
\phantomsection\label{Figure 1.1}
\centering
\begin{comment}
\heading{Figure 1.1:} Tree representation, showing the value of each subcombination.

\begin{example}
   390
   /|\____________
  / |             \
 *  26            15
    /|\           /|\
   / | \         // \\
  +  2  24      / | | \
        /|\    +  3 5  7
       / | \
      *  4  6
\end{example}
\end{comment}
\includegraphics[width=31mm]{fig/chap1/Fig1.1g.pdf}
\begin{quote}
\heading{Figure 1.1:} 部分的組合せの値を示す木表現
\end{quote}
\end{figure}

\noindent
次に第一ステップの適用の繰り返しが、組み合わせでなく、プリミティブ
な式、例えば数値や組込オペレータ、その他の名前を評価することが必要となる点へと
導くことに注目して下さい。
以下を規定することにより、プリミティブな場合を取り扱います。

\begin{itemize}

\item
数字の値はそれが意味する値です

\item
組込オペレータの値は機械語の列であり対応する操作を実行します。

\item
その他の名前の値は現在の環境にてその名前に対応するオブジェクトです。

\end{itemize}

\noindent
2つ目のルールは
\code{+}と\code{*}のような記号もまたグローバル(大域)環境に含まれており、それらの``値''として
一連の機械語命令に関係付けられていると規定することにより、3番目のルールの特別な場合
であると見做すことができます。
注意すべき
鍵となる点は式の中の記号の意味の決定に環境が果たす役割です。Lispのような
インタラクティブな言語では\code{(+ x 1)}のような式の値について記号\code{x}
(またはの\code{+}のような記号についてさえ)意味を与える環境の説明無しに話すことは
無意味です。\link{Chapter 3}にて学びますが、評価が行われる文脈を提供する環境の
一般的概念は我々がプログラムの実行を理解する上で重要な役割を果たします。

上で与えられた評価ルールが定義を扱わないことにも注意して下さい。
例えば\code{(define x 3)}の評価は\code{define}を2つの引数、シンボル\code{x}の値と3に
適用しません。\code{define}の目的はまさに\code{x}に対する値の関連付けだからです。
(つまり\code{(define x 3)}は合成式ではありません。)

そのような一般的な評価ルールに対する例外は\jnewterm{特殊形式}と呼ばれます。
\code{define}は特殊形式の一例に過ぎません。すぐに他の例に出会うことになります。
特殊形式は全てそれ自身の評価ルールを持ちます。色々な種類の式(それぞれが
関連する評価ルールを持つ)はプログラミング言語の構文を構成します。
他の多くのプログラミング言語と比較してLispはとても簡単な構文を持ちます。
式の評価ルールは簡単な一般ルールと少しの特殊形式にて説明可能です。
\footnote{事物をより統一的な方法で表記可能な、簡単で便利な代替的表面構造である
特別な構文形式を、Peter Landinの作成した語句ですが、\newterm{syntactic sugar}
(シンタックスシュガー、\jnewterm{糖衣構文})と呼ぶ場合があります。他言語のユーザと比較して
Lispプログラマは一般に構文上の問題に気をつかいません。(Pascalのマニュアルを調査
するとどれだけ多くのページが構文の記述に割り当てられているのかに気付くのとは逆
です。)この構文の軽視はLispの柔軟性の理由の一部になります。Lispの柔軟性
は表面上の構文の変更を簡単にします。また多くの``便利な''構文の構築を見かける
理由にも繋ります。それらの構文は言語をあまり統一的でないものにし、プログラムが
巨大で複雑になるにつれ元の価値よりも多くの問題を起こすことになります。Alan Peril
曰く、``糖衣構文はセミコロンの癌を引き起こす''}



\subsection{複合手続}
\label{Section 1.1.4}
他の強力なプログラミング言語に必ず存在する要素をいくつかLispでも確認しました。

\begin{itemize}

\item
数値と算術命令はプリミティブなデータと手続です。

\item
組み合わせのネストは演算の結合手法を提供します。

\item
名前と値を関連付けする定義は抽象化の限定された手法を与えます。

\end{itemize}

\noindent
ここでは\newterm{procedure definitions}(\jnewterm{手続の定義})を学びます。より強力な抽象化のテクニックであり
組み立てられた操作に名前を与え、1つの単位としてアクセス可能にします。


``二乗の値''をどのように表現するかから始めましょう。``二乗の値を求めるためには
その値をその値自身にかける''と言えるでしょう。

\begin{scheme}
(define (square x) (* x x))
\end{scheme}

\noindent
これを以下のように理解することが可能です。

\begin{example}
(define (square    x)         (*      x         x))
  |        |       |           |      |         |
 定義    二乗する  xを       かける  xを       xで.
\end{example}

\noindent
ここで\code{square}と名付けられた\newterm{compound procedure}(\jnewterm{複合手続})が出てきました。この手続はある数値を
それ自身にてかけ算することを表しています。かけられる数には\code{x}という名前が
付けられており代名詞が自然言語にて果たすのと同じ役割を果たします。
この定義の評価はこの複合手続を作成し、私達はそれに\code{square}という名前を
与えています。\footnote{ここでは次の2つを区別できるようになることが本当に重要だと言え
ます。1つは手続を名前を付けずに作成すること、もう1つは既に作成された手続に
名前を付けることです。どのように行うかについては\link{Section 1.3.2}にて学びます。}


一般的な手続の定義形式は以下の通りです。

\begin{scheme}
(define (~\( \dark \langle\kern0.08em \)~~\var{\dark name}~~\( \dark \kern0.03em\rangle \)~ ~\( \dark \langle \)~~\var{\dark formal parameters}~~\( \dark \kern0.02em\rangle \)~) 
  ~\( \dark \langle\kern0.08em \)~~\var{\dark body}~~\( \dark \rangle \)~)
\end{scheme}

\noindent
\( \langle\hbox{\sl name}\kern0.08em\rangle \)はその環境における手続定義に関連付けられる記号です。
\footnote{この本では一般的な式の構文をかぎ括弧にて閉じたイタリックの記号
---例えば、 \( \langle \)\var{name}\( \kern0.08em\rangle \)---を用いて実際に
式が利用される時に埋められるべき式中の``枠''を示します。}
\( \langle\hbox{\sl formal parameters}\kern0.08em\rangle \)は手続の中で
利用される名前で手続の関連する引数を参照します。
\( \langle\hbox{\sl body}\kern0.08em\rangle \)は形式上のパラメータが、
適用される手続の実際の引数に置換される時、手続適用の値を返す式です。
\footnote{より実際には、手続のボディは連続する式です。この場合インタプリタは
連続する各式を順に評価し最後の式の値を手続適用全体の値として返します。}
\( \langle \)\var{name}\( \kern0.08em\rangle \)と
\( \langle \)\var{formal parameters}\( \kern0.08em\rangle \)は括弧を用いて
グループ化され、実際の手続呼出しのように定義されます。


\code{square}を定義したので使ってみましょう。

\begin{scheme}
(square 21)
~\textit{441}~
(square (+ 2 5))
~\textit{49}~
(square (square 3))
~\textit{81}~
\end{scheme}

\noindent
\code{square}を構築要素として他の手続の構築に用いることも可能です。
例えば、 \( x^2 + y^2 \)は次のように表現できます。

\begin{scheme}
(+ (square x) (square y))
\end{scheme}

\noindent
2つの数値を引数として取りそれらの二乗の和を求める\code{sum\-/of\-/squares}を定義する
ことも簡単です。

\begin{scheme}
(define (sum-of-squares x y)
  (+ (square x) (square y)))
(sum-of-squares 3 4)
~\textit{25}~
\end{scheme}

\noindent
\code{sum\-/of\-/squares}をさらに別の手続構築に利用することもできます。

\begin{scheme}
(define (f a)
  (sum-of-squares (+ a 1) (* a 2)))
(f 5)
~\textit{136}~
\end{scheme}

\noindent
複合手続はプリミティブな手続と全く同じように利用可能です。
実際に上の\code{sum\-/of\-/squares}の定義を見ても\code{square}が\code{+} や \code{*}のように
インタプリタに組込まれているのか、複合手続として定義されているのか見分けがつか
ないでしょう。


\subsection{手続適用の置換モデル}
\label{Section 1.1.5}
オペレータの名前が合成式を示す組合せを評価する時、 インタプリタは
\link{Section 1.1.3}で説明した組合せのオペレータがプリミティブである場合と
ほぼ同じ手順を追います。インタプリタは合成の各要素を評価し、
(組合せのオペレータの値である)手続を(組合せのオペランドである)
引数に対して適用します。




プリミティブな手続を引数に対して適用するメカニズムはインタプリタに組込まれている
ことが想像できます。複合手続に対しては適用プロセスは以下のようになります。

\begin{quote}
複合手続を引数に適用するために、手続のボディを、各形式パラメタを対応する引数にて
置換してから評価します。
\end{quote}


\noindent
過程を説明するために以下のコンビネーションを評価してみましょう。

\begin{scheme}
(f 5)
\end{scheme}

\noindent
\code{f}は\link{Section 1.1.4}にて定義された手続です。 \code{f}のボディを取得することから始めます。

\begin{scheme}
(sum-of-squares (+ a 1) (* a 2))
\end{scheme}

\noindent
次に形式パラメタの\code{a}を引数5で置き換えます。

\begin{scheme}
(sum-of-squares (+ 5 1) (* 5 2))
\end{scheme}

\noindent
従って問題は2つのオペランドとオペレータ\code{sum\-/of\-/squares}に換算されます。
この組み合わせの評価は3つの部分問題に分かれます。まずオペレータを
評価して適用する手続を得て、オペランドを評価して引数を得る必要があります。
さて\code{(+ 5 1)}は6になり、\code{(* 5 2)}は10になりますので\code{sum\-/of\-/squares}手続
を6と10に適用しなければなりません。これらの値は\code{sum\-/of\-/squares}のボディの
パラメタ\code{x}と\code{y}を置き換え、式は以下のように置換されます。

\begin{scheme}
(+ (square 6) (square 10))
\end{scheme}

\noindent
\code{square}の定義を用いるとこれはさらに以下のように置換されます。

\begin{scheme}
(+ (* 6 6) (* 10 10))
\end{scheme}

\noindent
乗算を置換することで以下になります。

\begin{scheme}
(+ 36 100)
\end{scheme}

\noindent
最終的には次のとおりです。

\begin{scheme}
136
\end{scheme}

\noindent
ここまで説明したプロセスは手続適用の\newterm{substitution model}(\jnewterm{置換モデル}、代入モデル)
と呼ばれます。この章にて扱われた手続の過程においては、
手続適用の``意味''を決定するモデルとして捉えることができます。
しかし、強調すべき2つの事があります。

\begin{itemize}

\item
置換の目的は私達が手続適用について考えることを手助けすることであり、
インタプリタが実際にどのように働くかの説明を与えることではありません。
典型的なインタプリタは形式パラメータのための値を置き換えるために手続のテキストを
操作することで、手続適用を評価することはしません。実際には``置換''は形式パラメタに
ローカルの環境を用いることで行われます。このことについてはより完全に\link{Chapter 3}
と\link{Chapter 4}にてインタプリタの実装の詳細について調査する時に議論します。

\item
この本のコース全体ではインタプリタがどのように働くかについて、一連の徐々に
精巧なモデルを紹介して行きます。最終的にはインタプリタとコンパイラの完全な実装
を\link{Chapter 5}で見せます。置換モデルはこれらのモデル
の最初---評価手続について正式な考え得るための始まりに過ぎません。
一般的に科学とエンジニアリングについての事象をモデリングする場合、単純化した
不完全なモデルから始めます。より詳細な調査を行うにつれ、これらの単純なモデルは
不適切になり、より正確なモデルにて置き換えられます。置換モデルもまた例外ではあり
ません。実際に \link{Chapter 3}で示しますが手続を``mutable(変わりやすい)データ''
と共に扱う場合に置換モデルは破綻し、より複雑な手続適用のモデルにより置き換えなけ
ればならなくなります。
\footnote{置換のアイデアの簡明性にもかかわらず、置換処理の厳密な数学上の定義を
与えることは驚くほど複雑になることが知られています。問題は手続の形式パラメタの名前と
手続が適用される式で利用されている(同じである可能性のある)名前の間の混乱の可能性から生じます。
実際に論理とプログラミング意味論の文献における\jnewterm{置換}の間違った定義には
長い歴史があります。\link{Stoy 1977}の置換に関する注意深い議論を参照下さい。}

\end{itemize}

\subsubsection*{適用順~対~正規順}
\link{Section 1.1.3}で与えられた評価の記述に従えば、インタプリタは最初にオペレータとオペランド
を評価し、次に結果の手続を結果の引数に適用します。評価の仕方はこれだけでは
ありません。代替としての評価モデルはオペランドをそれらの値が必要になるまで
評価しません。その代わりに最初はオペランドの式にそれがプリミティブなオペレータ
のみ持つまでパラメタで置換します。それから評価を実行します。この手法を用いた
場合、\code{(f 5)}の評価は展開の流れに従って進行します。

\begin{scheme}
(sum-of-squares (+ 5 1) (* 5 2))
(+   (square (+ 5 1))      (square (* 5 2))  )
(+   (* (+ 5 1) (+ 5 1))   (* (* 5 2) (* 5 2)))
\end{scheme}

\noindent
簡約が続きます。


\begin{scheme}
(+      (* 6 6)      (* 10 10))
(+         36           100)
                136
\end{scheme}

\noindent
今回も前回の評価モデルと同じ答になりました。しかし経過が異なります。
具体的には\code{(+ 5 1)}と\code{(* 5 2)}がここではそれぞれ二回づつ実行されます。
式\code{(* x x)}における\code{x}がそれぞれ \code{(+ 5 1)}と\code{(* 5 2)}に置き換えることで
換算されているのに相当していています。




この代替である``完全に展開してから簡約する''評価方法は\jnewterm{正規順序評価}
として知られています。一方、``引数を評価してから適用''する方法はインタプリタ
が実際に利用するもので\jnewterm{適用順序評価}と呼ばれます。、
(この本の最初の2つの章の手続全てを含めて)置換を使用してモデリング可能、かつ正当な値を生む手続適用において
正規順序と適用順序の評価は同じ値を生むことが見てとれるでしょう。
(正規順序と適用順序の評価が同じ値を返さない``不当な''値の例は\link{Exercise 1.5}をご覧下さい)




Lispは適用順序評価を用いています。理由の一部は上の\code{(+ 5 1)} と \code{(* 5 2)}で
示されたような式の複数回評価を避けることで付加的な効率を得るためです。
そしてより重要な理由は正規順序評価は置換によりモデル化可能な手続の範囲を離れる時の
取扱がとても複雑なためです。一方で、正規順序評価はとても価値のあるツールです。
その意味のいくらかを\link{Chapter 3} と \link{Chapter 4}にて調査します。
\footnote{\link{Chapter 3}では\code{stream processing}(ストリーム処理)を紹介します。これは
一見して``無限''のデータ構造を正規順評価の制約形式に立脚して取り扱う手法です。
\link{Section 4.2}ではSchemeインタプリタを変更しSchemeの正規順異種を作成します。}


\subsection{条件式と述語}
\label{Section 1.1.6}

この時点で私達が定義可能な種類の手続の表現力はとても限られています。テストを作成し、
テストの結果により異なる命令を実行する方法が無いためです。例えば数値の絶対値を
演算する手続を定義できません。数値が正、負、零であるかテストを行いルールに従い
異なる場合に対し異なる行動をしなければなりません。

\begin{comment}

\begin{example}
      /
      |   x  if x > 0
|x| = <   0  if x = 0
      |  -x  if x < 0
      \
\end{example}

\end{comment}
\begin{displaymath}
 |x| = \left\{ \begin{array}{r@{\quad \mathrm{if} \quad}l}	 	 
        x  &  x > 0, \\
	0  &  x = 0, \\
  \!\! -x  &  x < 0. \end{array} \right. 
\end{displaymath}

この考えは\newterm{case analysis}(ケース分析、\jnewterm{事例分析})と呼びLispにはそのようなケース分析の
ための特殊形式が存在します。\code{cond}(``conditional''(条件文)を表わす)と呼ばれ、
以下のように利用されます。

\begin{scheme}
(define (abs x)
  (cond ((> x 0) x)
        ((= x 0) 0)
        ((< x 0) (- x))))
\end{scheme}

\noindent
条件式の一般的な形式は以下のとおりです。

\begin{scheme}
(cond (~\( \dark \langle{p_1}\kern0.08em\rangle \)~ ~\( \dark \langle{e_1}\kern0.08em\rangle \)~)
      (~\( \dark \langle{p_2}\kern0.08em\rangle \)~ ~\( \dark \langle{e_2}\kern0.08em\rangle \)~)
      ~\( \dots \)~
      (~\( \dark \langle{p_n}\kern0.08em\rangle \)~ ~\( \dark \langle{e_n}\kern0.08em\rangle \)~))
\end{scheme}

\noindent
記号\code{cond}から構成され、続く括弧で括った複数の式のペア

\begin{scheme}
(~\( \dark \langle{p}\kern0.08em\rangle \)~ ~\( \dark \langle{e}\kern0.08em\rangle \)~)
\end{scheme}

\noindent
\newterm{clauses}(\jnewterm{クローズ}、節)と呼ばれます。各ペアの最初の式は\newterm{predicate}(\jnewterm{述語})
---値が真か偽になる式です。
\footnote{``真か偽のどちかに解釈される''とは次を意味します。Schemeでは2つの区別される
値が存在し、それらは \code{\#t} と \code{\#f}の定数で示されます。インタプリタが述語の値をチェック
する時、\code{\#f}を偽と訳します。それ以外の任意の値は全て真だと扱われます。(従って
\code{\#t}を与えることは論理的には必要ありません。しかしそのほうが便利です。)
この本では\code{true} と \code{false}という名前を用います。それらは\code{\#t} と \code{\#f}という値に
それぞれ関連付けられます。}


条件式は次のように評価されます。まず述語\( \langle{p_1}\kern0.08em\rangle \)が最初に評価されます。
もしその値が偽であれば次に\( \langle{p_2}\kern0.08em\rangle \)が評価されます。もし\( \langle{p_2}\kern0.08em\rangle \)
の値もまた偽であるならば、その次は\( \langle{p_3}\kern0.08em\rangle \)が評価されます。
この過程は値が真となる述語が見つかるまで続きます。その場合インタプリタは対応するクローズの
\newterm{consequent expression}(\jnewterm{結果式}) \( \langle{e}\kern0.08em\rangle \)の値が条件式の値として返されます。
もし真となる\( \langle{p}\kern0.08em\rangle \)が見つからない場合には\code{cond}の値は未定義です。



述語という単語は真か偽を返す手続に利用されます。真か偽と評価される式にも用いられます。
絶対値の手続\code{abs}はプリミティブな述語\code{>}, \code{<}, \code{=}を利用します。
\footnote{\code{abs}はまた``マイナス''演算子\code{-}を使用します。
\code{(- x)}のように単一のオペランドに利用された時、符号の反転を示します。}

これらは2つの数値を引数として取り最初の数が2つ目の数に対し式の順に、より大きい、より小さい、等しい
かどうかテストを行い適宜に真か偽を返します。


絶対値の手続を書くもう1つの方法が次です。

\begin{scheme}
(define (abs x)
  (cond ((< x 0) (- x))
        (else x)))
\end{scheme}

\noindent
これは日本語で``もし\( x \)が零より小さい場合\( -x \)を返す。そうでなければ\( x \)を返す''と
表現できます。\code{else}は特別なシンボルで\code{cond}の最後の節(クローズ)の\( \langle{p}\kern0.08em\rangle \)
の場所にて利用可能です。こうすることで\code{cond}がその値として対応する\( \langle{e}\kern0.08em\rangle \)
の値をそれ以前のクローズ全てが回避された場合に返すことが可能です。
本当の所はここで\( \langle{p}\kern0.08em\rangle \)に常に値が真となる任意の式を使用することも可能です。


次はさらにもう1つ別の絶対値手続の書き方です。

\begin{scheme}
(define (abs x)
  (if (< x 0)
      (- x)
      x))
\end{scheme}

\noindent
これは特殊形式の\code{if}という制約のある条件の型を使用しており、ケース分析にて正確に2つのケースが存在する場合に用います。
\code{if}式の一般的な形式は以下のとおりです。



\begin{scheme}
(if ~\( \dark \langle\kern0.07em \)~~\var{\dark predicate}~~\( \dark \kern0.06em\rangle \)~ ~\( \dark \langle\kern0.07em \)~~\var{\dark consequent}~~\( \dark \kern0.05em\rangle \)~ ~\( \dark \langle\kern0.06em \)~~\var{\dark alternative}~~\( \dark \kern0.06em\rangle \)~)
\end{scheme}

\noindent
\code{if}式を評価するためにインタプリタは式の\( \langle \)\var{predicate}\( \kern0.08em\rangle \)の部分を評価すること
から始めます。もし\( \langle \)\var{predicate}\( \kern0.08em\rangle \)の評価が真値になる場合、インタプリタは次に
\( \langle \)\var{consequent}\( \kern0.08em\rangle \)を評価しその値を返します。そうでなければ
\( \langle \)\var{alternative}\( \kern0.08em\rangle \)を評価しその値を返します。
\footnote{\code{if} と \code{cond}の小さな違いは\code{cond}の各クローズの\( \langle{e}\kern0.08em\rangle \)は連続する式になっても
良いことです。もし対応する\( \langle{p}\kern0.08em\rangle \)が真になる場合、\( \langle{e}\kern0.08em\rangle \)内の式は
順に評価され連なりの最後の式の値が\code{cond}の値として返されます。しかし\code{if}式の中では
\( \langle \)\var{consequent}\( \kern0.08em\rangle \)と\( \langle \)\var{alternative}\( \kern0.08em\rangle \)
は単一の式でなければなりません}



\code{<}や\code{=}, \code{>}のようなプリミティブな述語に追加して論理複合命令が存在し、複合述語を構築することを可能にします。
最も良く利用される3つは以下の物です。

\begin{itemize}

\item
\( \hbox{\tt(and }\langle{e_1}\rangle\;\;\dots\;\;\langle{e_n}\rangle\hbox{\tt)} \)

インタプリタは式\( \langle{e}\kern0.08em\rangle \)を左から右へ1つずつ評価します。もし\( \langle{e}\kern0.08em\rangle \)の
どれかが偽と評価された場合\code{and}式の値は偽となり、残りの\( \langle{e}\kern0.08em\rangle \)は評価されません。
もし全ての\( \langle{e}\kern0.08em\rangle \)の評価が真となれば\code{and}式の値は最後の値になります。

\item
\( \hbox{\tt(or }\langle{e_1}\rangle\;\;\dots\;\;\langle{e_n}\rangle\hbox{\tt)} \)

インタプリタは式 \( \langle{e}\kern0.08em\rangle \)を1つずつ左から右へ評価します。
もし\( \langle{e}\kern0.08em\rangle \)のどれかが真と評価されればその値が\code{or}式の値として返され、
\( \langle{e}\kern0.08em\rangle \)の残りは評価されません。
もし全ての\( \langle{e}\kern0.08em\rangle \)が偽と評価された場合、\code{or}の値は偽となります。

\item
\( \hbox{\tt(not }\langle{e}\rangle\hbox{\tt)} \)

\code{not}式の値は式\( \langle{e}\kern0.08em\rangle \)が偽と評価される時は真であり、そうでなければ偽となります。

\end{itemize}

\noindent
\code{and}と\code{or}が特殊形式であり手続ではないことに注意して下さい。
部分式が全て評価される必要が無いためです。
\code{not}は通常の手続です。

これらがどのように利用されるかの例として、数値\( x \)が\( 5 < x < 10 \)の値域に存在するかという条件は
次のように表現されます。

\begin{scheme}
(and (> x 5) (< x 10))
\end{scheme}

\noindent
別の例として、ある数値が別の数値に対し等しいかより大きいかを示す述語は以下の通りです。

\begin{scheme}
(define (>= x y) (or (> x y) (= x y)))
\end{scheme}

\noindent
または代替法として

\begin{scheme}
(define (>= x y) (not (< x y)))
\end{scheme}

\begin{quote}
\heading{\phantomsection\label{Exercise 1.1}Exercise 1.1:} 以下の一連の式について、
各式に対するインタプリタの応答としての表示結果は何か?
式の列は下記に表示された順で評価されるものと考えよ。

\begin{scheme}
10
(+ 5 3 4)
(- 9 1)
(/ 6 2)
(+ (* 2 4) (- 4 6))
(define a 3)
(define b (+ a 1))
(+ a b (* a b))
(= a b)
(if (and (> b a) (< b (* a b)))
    b
    a)
\end{scheme}

\begin{scheme}
(cond ((= a 4) 6)
      ((= b 4) (+ 6 7 a))
      (else 25))
\end{scheme}

\begin{scheme}
(+ 2 (if (> b a) b a))
\end{scheme}

\begin{scheme}
(* (cond ((> a b) a)
         ((< a b) b)
         (else -1))
   (+ a 1))
\end{scheme}
\end{quote}

\begin{quote}
\heading{\phantomsection\label{Exercise 1.2}Exercise 1.2:} 以下の式を接頭辞形式にて翻訳せよ
\begin{comment}

\begin{example}
5 + 4 + (2 - (3 - (6 + 4/5)))
-----------------------------
       3(6 - 2)(2 - 7)
\end{example}

\end{comment}
\begin{displaymath}
{5 + 4 + (2 - (3 - (6 + {4\over5})))\over3(6 - 2)(2 - 7)}\,.
\end{displaymath}
\end{quote}

\begin{quote}
\heading{\phantomsection\label{Exercise 1.3}Exercise 1.3:} 3つの数値を引数として取り、
内2つの大きな数値の二乗の和を返す手続を定義せよ。
\end{quote}

\begin{quote}
\heading{\phantomsection\label{Exercise 1.4}Exercise 1.4:} 我々の評価モデルが
オペレータが複合式である組み合わせを可能にすることを観察せよ。
この観察結果を用いて次の手続の挙動を説明せよ:


\begin{scheme}
(define (a-plus-abs-b a b)
  ((if (> b 0) + -) a b))
\end{scheme}
\end{quote}

\begin{quote}
\heading{\phantomsection\label{Exercise 1.5}Exercise 1.5:} Ben Bitdiddleは自分が
直面するインタプリタが適用順評価と正規順評価のどちらを用いるか
決定するテストを開発した。まず以下の2つの手続を定義する。

\begin{scheme}
(define (p) (p))
(define (test x y)
  (if (= x 0) 0 y))
\end{scheme}



次に以下の式を評価する。

\begin{scheme}
(test 0 (p))
\end{scheme}



Benは適用順評価を用いるインタプリタではどのような挙動を観察するだろうか?
Benは正規順評価を用いるインタプリタではどのような挙動を観察するだろうか?
あなたの回答を説明せよ。(特殊形式の\code{if}はインタプリタが適用順評価でも正規順評価でも
同じ挙動を行うと仮定せよ: 述語式が最初に評価され、結果がconsequentとalternativeのどちらを評価するか決定する)


\end{quote}

\subsection{例: ニュートン法による平方根}
\label{Section 1.1.7}


ここまでで説明された通り、手続は普通の数学の関数にとても似ています。
手続は1つ以上のパラメタにより決定される値を特定します。しかし数学の関数と計算機の
手続の間には重要な違いが存在します。手続は効果的である必要があります。



その一例として、平方根の演算問題について考えましょう。
square-root関数を以下のように定義できます。

\begin{comment}

\begin{example}
sqrt(x) = the y such that y >= 0 and y^2 = x
\end{example}

\end{comment}
\begin{displaymath}
\sqrt{x}\;\; = {\rm\;\; the\;\;} y 
{\rm\;\; such\;\; that\;\;} y \ge 0 {\rm\;\; and\;\;} y^2 = x.
\end{displaymath}


これは完全に正しい数学の関数です。これを用いてある数値が他の数値の平方根であるか
分かりますし、平方根の一般的な事実を導出可能です。しかし一方でこの定義は手続の
記述ではありません。与えられた数値から実際どのようにして平方根を求めるのか、
これはほとんど何も教えてくれません。この定義を疑似Lispにて言い換えようとも
問題の何の手助けにもなりません。

\begin{scheme}
(define (sqrt x)
  (the y (and (>= y 0) 
              (= (square y) x))))
\end{scheme}

\noindent
これはただ問題を提起するだけです。



関数と手続の間の対称性は事物の属性の説明と行いの説明との間の一般的区別に関する
反映です。または時には宣言的知識と手続的知識の間の区別だと参照できるでしょう。
数学では通常宣言的(what is)記述を用い、コンピュータサイエンスでは通常手続的
(how to)記述を用います。\footnote{宣言的、手続的記述は数学とコンピュータサイエンスの
ように実際に深く関わっています。例えばプログラムの生成した答が``正しい''という
ことはプログラムについて宣言的な文を作成することです。プログラムが正しいことを
証明するための立証技術を目的とした非常に多くの研究が存在します。この問題の技術的
難度の多くは(プログラムが構築される)手続的文と(事象を推論するのに用いられる)
宣言的文との間の移行に関連します。関連領域において、プログラミング言語の設計に
おける現在の重要な領域は超高水準言語と呼ばれる物の調査です。それは実際に
プログラムを宣言的文の用語にて作成します。その意図はインタプリタを十分に洗練する
ことでプログラマより与えられた``what is''の知識より``how to''の知識を自動的に
生成可能とします。これは一般的には可能ではありませんが、成果が達成された
重要な領域が存在します。この考え方については\link{Chapter 4}にて再度触れることに
致します。}



人はどのようにして平方根を求めることができるのでしょうか? 最も一般的な方法は
ニュートンの漸次接近法を用いる方法です。ニュートン法はある数値\( x \)の平方根
の推定値として\( y \)を持つ場合に、より良い推定値(実際の平方根により近い値)を
求めるために\( y \)と\( x / y \)の平均を取るという簡単な操作を実行します。
\footnote{この平方根アルゴリズムは実際にはニュートン法の特別なケースです。
ニュートン法は方程式の根を求める一般的な技法です。平方根アルゴリズム自体は
アレキサンドリアのHeronにより\acronym{A.D.}1世紀に開発されました。一般的な
ニュートン法をLispの手続によりどのように表わすかについては\link{Section 1.3.4}にて
学びます。}
例として、2の平方根は以下のようにして求められます。
推定値の初期値を1とします:

\begin{smallexample}
推定値    商                     平均
1         (2/1) = 2              ((2 + 1)/2) = 1.5
1.5       (2/1.5) = 1.3333       ((1.3333 + 1.5)/2) = 1.4167
1.4167    (2/1.4167) = 1.4118    ((1.4167 + 1.4118)/2) = 1.4142
1.4142    ...                    ...
\end{smallexample}

\noindent
この過程を繰り返すことにより平方根のより良い近似値を得られます。



では手続の表現にてこの過程を形式化してみましょう。radicand(被開法数: 根号の中身。平方根を求める値)と
guess(推定値)を用います。もし推定値の品質が十分であれば終了します。そうでなければ
処理をより良い推定値にて繰り返さなければなりません。この基本的戦略を手続として
以下のように記述しました。

\begin{scheme}
(define (sqrt-iter guess x)
  (if (good-enough? guess x)
      guess
      (sqrt-iter (improve guess x) x)))
\end{scheme}

\noindent
推定値は旧推定値と商の平均を取ることで改善されます。

\begin{scheme}
(define (improve guess x)
  (average guess (/ x guess)))
\end{scheme}

\noindent
averageの定義は以下です。

\begin{scheme}
(define (average x y)
  (/ (+ x y) 2))
\end{scheme}

\noindent
``十分に良い''の定義を決めねばなりません。以下に説明しますが、これは本当はあまり良い
テストではありません。(\link{Exercise 1.7}をご覧下さい)
考え方は回答を十分に近い値にするために、その二乗と被開法数の差が事前に決定した
許容誤差(ここでは0.001)より小さくなるまで改善します。
\footnote{通常は述語にはクエスチョンマークで終わる名前を与えます。そうすることで
それが述語だと理解できるようにです。これは単にスタイル上の慣例です。インタプリタの
受け取り方に関する限り、クエスチョンマークは通常の文字でしかありません。}


\begin{scheme}
(define (good-enough? guess x)
  (< (abs (- (square guess) x)) 0.001))
\end{scheme}

\noindent
最後にどのように始めるかが必要です。例えば任意の数値の平方根の推定値を常に1とすることも
可能です。\footnote{推定値の初期値を1ではなく1.0と表現していることに注意して下さい。
これは多くのLispの実装では何の違いもありません。しかし\acronym{MIT} Schemeは
整数と小数の値を厳格に区別します。2つの整数を割ると小数ではなく分数を返します。
例として10を6で割ると5/3を返します。しかし10.0を6.0で割れば1.6666666666666667を返すのです。
(分数の演算の実装法については\link{Section 2.1.1}で学びます。)
もし推定値の初期値をsquare-rootプログラムにおいて1にして開始した場合、\( x \)も
実際に整数である場合には全ての続くsquare-rootの演算により生成される値は小数ではなく分数
になります。分数と小数を混ぜた演算は小数を返します。従って推定値の初期値を1.0にすることで
全ての続く値を小数にすることが可能です。}

\begin{scheme}
(define (sqrt x)
  (sqrt-iter 1.0 x))
\end{scheme}

\noindent
この定義をインタプリタに入力すれば\code{sqrt}を他の手続のように利用可能です。

\begin{scheme}
(sqrt 9)
~\textit{3.00009155413138}~

(sqrt (+ 100 37))
~\textit{11.704699917758145}~

(sqrt (+ (sqrt 2) (sqrt 3)))
~\textit{1.7739279023207892}~

(square (sqrt 1000))
~\textit{1000.000369924366}~
\end{scheme}

\noindent
\code{sqrt}プログラムはまた私達がここまでで紹介した単純な手続き型言語がCやPascalで
記述可能などんな純粋数値演算プログラムを書くのにも十分であることを示しています。
これには驚かれるかもしれません。私達はまだコンピュータに何かを繰り返し繰り返し
行わせるどのような繰り返し(ループ)要素もこの言語には入れていないためです。
一方で\code{Sqrt\-/iter}はどのように繰り返しが特別な記法を全く使わずに通常の手続
呼出能力のみで成し遂げられるかを実演して見せています。
\footnote{反復実装における手続呼出上の効率の問題を気にされている読者の方は
\link{Section 1.2.1}の``末尾再帰''上の備考に注目して下さい}

\begin{quote}
\heading{\phantomsection\label{Exercise 1.6}Exercise 1.6:} Alyssa P. Hackerはなぜ
\code{if}が特殊形式として提供される必要があるのか理解できなかった。
``なぜ\code{cond}を用いた通常手続として定義できないのだろう?''と彼女は訝った。
Alyssaの友達であるEva Lu Atorはこれは実際にできると主張し、\code{if}の新バージョンを
定義した。

\begin{scheme}
(define (new-if predicate then-clause else-clause)
  (cond (predicate then-clause)
        (else else-clause)))
\end{scheme}

EvaはAlyssaに対してプログラムのデモを行った。

\begin{scheme}
(new-if (= 2 3) 0 5)
~\textit{5}~
(new-if (= 1 1) 0 5)
~\textit{0}~
\end{scheme}

喜びながらAlyssaは\code{new\-/if}を用いてsquare-rootプログラムを書き直した。

\begin{scheme}
(define (sqrt-iter guess x)
  (new-if (good-enough? guess x)
          guess
          (sqrt-iter (improve guess x) x)))
\end{scheme}



Alyssaが平方根の計算にこれの使用を試した時に何が起こるだろうか?
\end{quote}

\begin{quote}
\heading{\phantomsection\label{Exercise 1.7}Exercise 1.7:} 平方根の演算で使用された
\code{good\-/enough?}テストはとても小さい数値の平方根を
見つける場合にはあまり効果的ではないだろう。また実際のコンピュータでは数値演算
命令はほとんど常に精度に制限のある状態で実行される。これが我々のテストをとても
大きな数値に対して不適切にする。ここまでの記述についてテストがどのように小さな値と
大きな値にて失敗するか例を用いて説明せよ。\code{good\-/enough?}実装の代替戦略は
\code{guess}がある試行から次に向けどのように変化するか監視し、変化が推定値の割合に
おいてとても小さい時に止めることである。このような終了テストを用いるsquare-rootを
設計せよ。これは小さな、及び、大きな数値に対してより良く働くだろうか?

\end{quote}

\begin{quote}
\heading{\phantomsection\label{Exercise 1.8}Exercise 1.8:} 立方根に対する
ニュートン法は\( y \)が\( x \)の立方根である場合において以下の値により
良く近似される。
\begin{comment}

\begin{example}
x/y^2 + 2y
----------
    3
\end{example}

\end{comment}
\begin{displaymath}
{{x / y^2} + 2y\over 3}.
\end{displaymath}
\noindent
この式を用いてsquare-rootに類似したcube-rootを実装せよ。(\link{Section 1.3.4}にてこれらの
square-rootとcube-rootの抽象化としての汎用なニュートン法の実装方法を学習します。)
\end{quote}

\subsection{ブラックボックス抽象化としての手続}
\label{Section 1.1.8}



\code{sqrt}は私達にとり相互に定義された手続の集合により定義されたプロセスの例でした。
\code{sqrt\-/iter}の定義が\newterm{recursive}(\jnewterm{再帰的})であることに注意して下さい。再帰とは
手続がそれ自身の語により定義されていることです。手続をそれ自体の名前を用いて定義する
考え方は不安になるかもしれません。そのような``循環的''な定義がどのようにしてつじつま
を合わせるのか全く不明に見えるかもしれません。コンピュータにより実行するために良く
定義された手続には指定が足りなく見えるかもしれません。これについては\link{Section 1.2}にて
より注意深く触れることにします。最初はしかし\code{sqrt}の例にて説明されたいくつか別の
重要な点について考えましょう。



平方根を演算する問題が自然にいくつかの部分的な問題へ分割されることに注意して下さい。
推定値が十分に良いかどのように判断するか、推定値をどのように改善するか、等です。
これらのタスクの1つ1つは分離された手続により達成されます。\code{sqrt}プログラム全体は
(\link{Figure 1.2}にて表される)手続の群れに見てとることが可能です。この図が問題を
部分問題へと分解することを映し出しています。

\begin{figure}[tb]
\phantomsection\label{Figure 1.2}
\centering
\begin{comment}
\heading{Figure 1.2:} Procedural decomposition of the \code{sqrt} program.
\end{comment}
\begin{example}
                       sqrt
                        |
                    sqrt-iter
                    /       \
            good-enough    improve
              /     \          \
          square    abs      average
\end{example}

\begin{quote}
\heading{Figure 1.2:} \code{sqrt}プログラムの手続分解
\end{quote}
\end{figure}

\noindent
この分解戦略の重要性はプログラムを部分---最初の10行、次の10行、その他へと分割
するような単純なものではありません。そうではなく、各手続が他の手続の定義にて
モジュールとして利用可能な特定のタスクを担うことが不可欠です。例えば\code{good\-/enough?}
手続を\code{square}の語を用いて定義する時、\code{square}手続を``ブラックボックス''として
考えることが可能です。その時、その手続が\emph{どのように}結果を計算するのか気にして
いません。それが二乗を計算するという事実のみです。二乗がどのように計算されるかという
詳細は隠し、後の時点で考慮することが可能です。実際に\code{good\-/enough?}手続について考える
限り、\code{square}は手続では無く手続の抽象に過ぎないのです。\jnewterm{手続の抽象化}と
呼ばれるものです。この抽象化のレベルでは二乗を計算するどんな手続も等しく相応しいのです。



従って返り値のみを考えるため、以下の2つの二乗する手続は区別不可能となります。
それぞれが数値の引数を取りその数値の二乗を値として生成します。
\footnote{これら手続のどちらがより効率的な実装であるかは全く明確ではありません。
これは実行環境依存です。``明白な''実装が効率的ではない機械が存在します。
広範な対数と逆対数のテーブルをとても効率の良い方法で持つ機械について考えてみて下さい}

\begin{scheme}
(define (square x) (* x x))
(define (square x) (exp (double (log x))))
(define (double x) (+ x x))
\end{scheme}

\noindent
従って手続定義は詳細を隠すことができなければなりません。手続のユーザはその手続を
彼等自身で書いたとは限りません。しかし他のプログラマからブラックボックスとして
取得したかもしれません。ユーザはその手続がどのように実装されているのかそれを使用
するためには知る必要が無いのです。

\subsubsection*{ローカル名}



手続のユーザにとっては問題とならない手続実装の詳細の1つには手続の形式パラメタに
対する実装者が選択した名前があります。従って以下の異なる手続は区別不可能でなけれ
ばなりません。


\begin{scheme}
(define (square x) (* x x))
(define (square y) (* y y))
\end{scheme}

\noindent
この指針---手続の意味はその作者が使用したパラメタの名前から独立すべきである---は
表面上では自明なことに思えますがそこから導きださえる結論は重要です。最も単純な結論は
手続のパラメタ名はその手続のボディに対してローカルであるべきというものです。
例えば、まずsquare-root手続の中の\code{good\-/enough?}の定義においては\code{square}を使用しました。


\begin{scheme}
(define (good-enough? guess x)
  (< (abs (- (square guess) x))
     0.001))
\end{scheme}

\noindent
\code{good\-/enough?}の作者の意図する所は第一引数の二乗が第二引数にて与えられた
許容差の範囲であるかを決定することです。\code{good\-/enough?}の作者が\code{guess}を
第一引数の参照に用い\code{x}を第二引数に用いたことが見てとれます。\code{square}の
引数は\code{guess}です。もし\code{square}の作者が\code{x}を(上で見たように)使用した場合
\code{good\-/enough?}の\code{x}は\code{square}の\code{x}とは異ならなければならないことが
わかります。手続\code{square}の実行は\code{good\-/enough?}の\code{x}の値に影響を与えては
いけません。なぜならその\code{x}の値は\code{square}が演算を終えた後にも\code{good\-/enough?}
にて必要だからです。



もしパラメタがそれらが関連する手続のボディに対してローカルでない場合、\code{square}の
パラメタ\code{x}は\code{good\-/enough?}のパラメタ\code{x}と混同される可能性があります。
そして\code{good\-/enough?}の挙動はどのバージョンの\code{square}を利用するかに依存するでしょう。
従って\code{square}は私達が望んだブラックボックスではなくなるでしょう。



手続の形式パラメタは手続定義においてとても特別な役割を持ちます。形式パラメタには
どんな名前を用いてもかまわないのです。そのような名前は\newterm{bound variable}
(\jnewterm{束縛変数})と呼ばれます。そして手続定義はその形式パラメタを\newterm{binds}(\jnewterm{束縛}する)と
呼びます。もし束縛変数が静的に定義中においてリネームされても手続定義の意味は変わり
ません。\footnote{静的なリネームのコンセプトは実際には微妙で正式に定義するのは難しい
ことです。有名な論理学者達も恥しい間違いをここで犯してきました}
もし変数が束縛されていなければそれは\newterm{free}(\jnewterm{自由})だと呼びます。
束縛が名前を定義する式の集合はその名前の\newterm{scope}(\jnewterm{スコープ})と呼ばれます。
手続定義においてはその手続の形式パラメタとして宣言された束縛変数はその手続のボディを
そのスコープとします。



上記の\code{good\-/enough?}の定義において、\code{guess}と\code{x}は束縛変数ですが、
\code{<}, \code{-}, \code{abs}, \code{square}は自由変数です。私達が
選んだ\code{guess}と\code{x}の名前が、\code{<}, \code{-}, \code{abs}, \code{square}と異なり区別可能
である限り、\code{good\-/enough?}の意味はそれらの名前から独立せねばなりません。
(もし\code{guess}を\code{abs}にリネームした場合、変数\code{abs}を\newterm{capturing}(\jnewterm{占領})
することで自由変数を束縛変数に変化させるのでバグを持ち込むことになるでしょう。)
しかしながら\code{good\-/enough?}の意味はその自由変数の名前からは独立していません。
記号\code{abs}は(この定義の外部の)数値の絶対値を求める手続に名付けられているという事実に
当然、依存します。もし\code{cos}を\code{abs}にその定義において置き換えれば\code{good\-/enough?}は
異なる関数を計算することでしょう。


\subsubsection*{内部定義とブロック構造}



私達は今の所、一種類の名前の分離について学びました。手続の形式パラメタは手続の
ボディに対してローカルです。square-rootプログラムは我々が望むだろう名前使用を
コントロールする別な方法を示します。

\begin{scheme}
(define (sqrt x)
  (sqrt-iter 1.0 x))
(define (sqrt-iter guess x)
  (if (good-enough? guess x)
      guess
      (sqrt-iter (improve guess x) x)))
(define (good-enough? guess x)
  (< (abs (- (square guess) x)) 0.001))
(define (improve guess x)
  (average guess (/ x guess)))
\end{scheme}

\noindent
このプログラムの問題は\code{sqrt}のユーザにとって重要な手続は\code{sqrt}のみであることです。
他の手続(\code{sqrt\-/iter}, \code{good\-/enough?}, \code{improve})は彼らにとって余計なものです。
ユーザは他に\code{good\-/enough?}という名の手続を、square-rootプログラムと一緒に使用する
他のプログラムの一部として定義することができません。なぜなら\code{sqrt}がそれを必要と
するからです。この問題は多くの異なるプログラマにより巨大システムを構築する場合に特に
深刻な問題となります。例えば数値演算の巨大ライブラリの構築において、多くの数値演算
関数は一連の近似値演算として計算されるため補助的な手続として\code{good\-/enough?}と\code{improve}
と名付けられた手続を持つかもしれません。私達は部分手続を局所化し\code{sqrt}の中に隠し
たいと思うでしょう。そうすれば\code{sqrt}が他の一連の近似値演算と共存し、それぞれが
自身のプライベートな\code{good\-/enough?}手続を持つことができます。これを可能にするために
手続はその手続に対して局所的な内部定義を持つことが可能です。例えばsqaure-rootプログラムは
以下のように書き換えることが可能です。

\begin{scheme}
(define (sqrt x)
  (define (good-enough? guess x)
    (< (abs (- (square guess) x)) 0.001))
  (define (improve guess x) (average guess (/ x guess)))
  (define (sqrt-iter guess x)
    (if (good-enough? guess x)
        guess
        (sqrt-iter (improve guess x) x)))
  (sqrt-iter 1.0 x))
\end{scheme}

\noindent
このような定義の入れ子は\newterm{block structure}(\jnewterm{ブロック構造})と呼ばれ、最も単純な名前パッケージ
問題解決に基本的に正しい解決方法です。しかしより良いアイデアがここに隠れています。
補助的な手続を内在化させることに加えてそれらを簡潔化することができます。\code{x}は\code{sqrt}に
束縛されているため、\code{sqrt}の内部に定義された手続\code{good\-/enough?}, \code{improve}, \code{sqrt\-/iter}は
\code{x}のスコープ内にあります。従って\code{x}を明示的にこれらの手続それぞれに渡す必要はありません。
その代わりに\code{x}を以下で示すように内部の定義にて自由変数にすることができます。
そして\code{x}は包括する手続\code{sqrt}が呼ばれた時にその値を得ます。このような規律を
\newterm{lexical scoping}(\jnewterm{レキシカルスコープ})と呼びます。\footnote{レキシカルスコープは
手続内の自由変数が包括する手続定義により作られた束縛を参照するため用いられるよう指示します。
それはつまり、手続が定義された環境の中でそれらが探されることを意味します。これがどのように
働くのか、その詳細については第3章にて環境とインタプリタの詳細な挙動について学ぶ時に理解
します。}


\begin{scheme}
(define (sqrt x)
  (define (good-enough? guess)
    (< (abs (- (square guess) x)) 0.001))
  (define (improve guess)
    (average guess (/ x guess)))
  (define (sqrt-iter guess)
    (if (good-enough? guess)
        guess
        (sqrt-iter (improve guess))))
  (sqrt-iter 1.0))
\end{scheme}

\noindent
ブロック構造は巨大なプログラムを取扱の簡単な部品に分割するために広範囲にて
利用されます。\footnote{\label{Footnote 28}組込定義は手続本体の最初に来なければなりません。
相互依存の定義と使用を行うプログラムの実行結果についてはこのような管理も
責任を持つことができません}
ブロック構造の考え方はプログラミング言語Algol 60に起源を持ちます。多くの
先進的なプログラミング言語に存在し、巨大プログラム構築の体系化を手助けす
る重要なツールです。


\section{手続とそれが生成するプロセス}
\label{Section 1.2}



私達はプログラミングの要素について考えてきました。プリミティブな算術演算子を用い
これらの演算子を組み合わせ、その合成演算子を複合手続として定義することで抽象化
を行ってきました。しかしそれらは私達がプログラムをどのように書くか知っていると
言えるためには十分ではありません。私達の状況はチェスにおいて各駒がどのように動く
のかルールを覚えたが典型的な序盤や戦術、戦略について何も知らない人に似ています。
チェスの初心者棋士のように、私達はまだこの領域での慣習としての一般的パターンを
知りません。私達はどの手が打つ価値があるのか(どの手続が定義する価値があるのか)の
知識を欠いています。打った手の(手続実行の)結果を予想する経験を欠いています。


熟慮下の行動の結果を思い描く能力はエキスパートプログラマになるために重大です。
それはどんな統合的、かつ創造的な活動についても同じです。熟練の写真家になるには
例えば、景色の見方を学び、各可能な露出と現像条件の組み合わせにおいて各領域が
どれだけ暗く写真に表れるかを知らなければなりません。そうして初めてフレームの計画、
光量、露出、現像を逆向きに推測して望んだ効果を得ることが可能となります。
プログラミングにおいてもまた同じです。プロセスが取り得る行動がどのような進行を
経るのか計画し、プログラムを用いてプロセスをコントロールします。


エキスパートになるためには、数多くの種類の手続により生成されるプロセスを心に
描けられるようにならなければなりません。そのようなスキルを開発した後に初めて
望んだ挙動を示すプログラムを信頼できる形で構築する方法を学ぶことが可能になります。



手続は計算過程の\newterm{local evolution}(\jnewterm{局所展開})のためのパターンです。プロセス
の各ステージが以前のステージの上にどのように構築されるかを指定します。
ここで手続により局所展開が指示されたプロセスの全体的な、または\newterm{global}(\jnewterm{大域的})な
挙動について説明を行えればと思います。しかしこれは一般的にはとても難しいので、最低でも
いくつかのプロセス展開の典型的パターンについて説明することを試してみましょう。



この節では簡単な手続により生成されたプロセスのためのいくつかの共通な``形''について
検討してみます。またこれらのプロセスが時間と記憶域の重要な計算資源をどの程度消費する
かについても調査してみます。ここで考慮する手続はとても簡単なものです。それらの役割は
写真撮影におけるテストパターンにより演じられる様なものです。非常に単純化した原型的な
パターンであり、それら自身の目的に沿った現実的な例ではありません。


\subsection{線形再帰と反復}
\label{Section 1.2.1}


階乗を求める関数を考えることから始めましょう。定義を以下に示します。

\[
n! = n \cdot [(n - 1)  \cdot (n - 2)  \dots 3 \cdot 2 \cdot 1] = n \cdot (n - 1)!
\]


階乗を計算する方法は数多くあります。1つの方法は任意の正の整数 \( n \)において、\( n! \)は
\( n \)と\( (n - 1)! \)の積に等しいという観察結果を利用します。
\begin{comment}

\begin{example}
n! = n * [(n - 1) * (n - 2) ... 3 * 2 * 1] = n * (n - 1)!
\end{example}

\end{comment}
\begin{displaymath}
n! = n \cdot [(n - 1) \dot (n - 2) \dots 3 \cdot 2 \cdot 1] = n \cdot (n - 1)!.
\end{displaymath}

従って\( (n - 1)! \)を演算し、\( n \)を掛けることで\( n! \)を求めることが可能です。
もし1!が1に等しいという規約を追加すればこの観察結果は直接手続に翻訳できます。

\begin{scheme}
(define (factorial n)
  (if (= n 1)
      1
      (* n (factorial (- n 1)))))
\end{scheme}

\noindent
\link{Section 1.1.5}の置換モデルを用いてこの手続が6!の計算を実行する様子を
\link{Figure 1.3}に示すように観察できます。

\begin{figure}[tb]
\phantomsection\label{Figure 1.3}
\centering
\begin{comment}
\heading{Figure 1.3:} A linear recursive process for computing 6!.

\begin{example}
(factorial 6)        ----------------
(* 6 (factorial 5))                   \
(* 6 (* 5 (factorial 4)))               \
(* 6 (* 5 (* 4 (factorial 3))))           \
(* 6 (* 5 (* 4 (* 3 (factorial 2)))))       \
(* 6 (* 5 (* 4 (* 3 (* 2 (factorial 1))))))  |
(* 6 (* 5 (* 4 (* 3 (* 2 1)))))             /
(* 6 (* 5 (* 4 (* 3 2))))                 /
(* 6 (* 5 (* 4 6)))                     /
(* 6 (* 5 24))                        /
(* 6 120)                           /
720          <---------------------
\end{example}
\end{comment}
\includegraphics[width=82mm]{fig/chap1/Fig1.3c.pdf}
\par\bigskip
\noindent
\heading{Figure 1.3:} 6!を求めるための線形再帰プロセス
\end{figure}

\noindent
では階乗の演算について異なる視点を得てみましょう。\( n! \)を計算するルールを最初に
1を2で掛け、その結果を3に掛け、次に4に掛け\( n \)に辿り着くまで繰り返すと説明する
ことも可能でした。より形式的には、積の実行と、1から\( n \)までカウントするカウンタ
とを一緒に保持します。カウンタと積は同時にあるステップから次へとルールに従い
変更されると言うことでこの演算を説明できます。

\begin{scheme}
product ~\( \dark \gets \)~ counter * product
counter ~\( \dark \gets \)~ counter + 1
\end{scheme}

\noindent
そして\( n! \)とはカウンタが\( n \)を越えた時点での積の値であると規定します。

再び、今までの階乗を求める手続の説明を次のように書き換えられます。
\footnote{実際のプログラムでは恐らく前の節で紹介したブロック構造を\code{fact\-/iter}の
定義を隠すために用いるでしょう。

\begin{smallscheme}
(define (factorial n)
  (define (iter product counter)
    (if (> counter n)
        product
        (iter (* counter product)
              (+ counter 1))))
  (iter 1 1))
\end{smallscheme}

ここでそれを避けたのは一度に考えなければならなことを最小にするためです。}


\begin{scheme}
(define (factorial n)
  (fact-iter 1 1 n))
(define (fact-iter product counter max-count)
  (if (> counter max-count)
      product
      (fact-iter (* counter product)
                 (+ counter 1)
                 max-count)))
\end{scheme}

\noindent
前回と同じく、置換モデルを用いて6!の演算を\link{Figure 1.4}として示します。



2つのプロセスを比べてみて下さい。1つの見方としては、これらはほとんど同じに見えます。
両者は同じ数学の関数を同じ領域で計算し、それぞれが\( n! \)を求めるのに\( n \)に比例
したステップ数を必要とします。実際に両者のプロセスが全く同じ一連の乗算を実行し、全く
同じ一連の部分的な積を得ます。一方で2つのプロセスの``形''を考えた時、全く異なった展開を
していることに気がつきます。


\begin{figure}[tb]
\phantomsection\label{Figure 1.4}
\centering
\begin{comment}
\heading{Figure 1.4:} A linear iterative process for computing 6!.

\begin{example}
(factorial 6)   -----.
(fact-iter   1 1 6)  |
(fact-iter   1 2 6)  |
(fact-iter   2 3 6)  |
(fact-iter   6 4 6)  |
(fact-iter  24 5 6)  |
(fact-iter 120 6 6)  |
(fact-iter 720 7 6)  V
720
\end{example}
\end{comment}
\includegraphics[width=36mm]{fig/chap1/Fig1.4c.pdf}
\par\bigskip
\noindent
\heading{Figure 1.4:} 6!を求めるための線形反復プロセス
\end{figure}

\noindent
最初のプロセスについて考えます。\link{Figure 1.3}の矢印で示されるように、
置換モデルが展開の後、収縮する状態を明らかにしています。
展開は\newterm{deferred operations}(遅延演算)の連鎖(このケースでは乗算の連なり)を構築するプロセスとして
起こります。収縮は演算が実際に実行されることにより起こります。遅延演算の連鎖として
示されるこのタイプのプロセスは\newterm{recursive process}(再帰プロセス)と呼ばれます。このプロセスの実行には
インタプリタが後の実行ために操作の過程を記録する必要があります。\( n! \)の演算では遅延乗算の連鎖
の長さ、そしてそれに従う追跡の必要な情報の量が、\( n \)に従い線形に(\( n \)に比例して)ステップ数と同様に増加します。
このようなプロセスは\newterm{linear recursive process}(線形再帰プロセス)と呼ばれます。



対照的に、2つ目のプロセスは展開も収縮もしません。各ステップにおいて追跡が必要な物は
どの\( n \)に対しても変数\code{product}, \code{counter}, \code{max\-/count}の現在値です。
これを\newterm{iterative process}(\jnewterm{反復プロセス})と呼びます。
一般的に、反復プロセスは限られた数の\newterm{state variables}(\jnewterm{状態変数})により状態が、
集約されることが可能な物です。
状態変数がプロセスが状態毎にどのように更新されるかという固定ルールと
プロセスが停止する条件を指定する(任意の)終了試験と一緒に用います。
\( n! \)の演算では\( n \)に従い必要なステップ数が線形に増加します。
このようなプロセスは\newterm{linear iterative process}(\jnewterm{線形反復プロセス})と呼ばれます。



2つのプロセスの対称性は他の見方もできます。反復の場合、プログラムの変数は任意の
ポイントにおいてプロセスの状態について完全な描写を提供します。もしステップの間で
計算を停止した場合に、計算の再開を行うのに必要な全てはインタプリタに対し3つの
プログラム変数の値を提供することです。再帰プロセスではそうはいきません。
この場合、いくつかの追加の``隠された''情報が存在し、インタプリタにより保持されており、
プログラムの変数には保存されていません。その情報には遅延命令の連鎖を辿る中での
``プロセスの現在地''が示されています。鎖が長い程、より多くの情報が保持される必要が
あります。\footnote{\link{Chapter 5}にてレジスタマシン上での手続の実装について
議論する時に、任意の反復プロセスが``ハードウェア内にて''固定長のレジスタ集合を持ち、
補助的なメモリは持たない機械であると認識できることを学びます。対照的に、再帰プロセスを
理解するには\newterm{stack}(\jnewterm{スタック})として知られる補助的なデータ構造が必要です。}



反復と再帰の対称性において、再帰\jnewterm{プロセス}の概念と再帰\jnewterm{手続}の
概念を混同しないように注意せねばなりません。私達が手続を再帰だと説明する時、
手続の定義が(直接、または間接的に)その手続自身を参照するという構文上の事実を
参照します。しかし、プロセスがあるパターン、例えば、線形再帰に従うと
説明する時、私達はプロセスがどのように展開するかについて話しており、手続が
どのように書かれているかという構文については話していません。\code{fact\-/iter}のような
再帰手続を反復プロセスの生成として言及することは当惑させるかもしれません。
しかし、そのプロセスは実際に反復的です。その状態は3つの状態変数により完全に補足され、
インタプリタはプロセスを実行するために、ただ3つの変数を追跡することのみが必要です。



プロセスとプロシジャ(手続)の区別が混乱を招き易いのは、(AdaやPascal、C言語を含む)多くの一般的言語
の実装が、例えプロセスが本質的には反復で記述されていても、任意の再帰手続の逐次実行が手続呼出の回数に伴い
多くのメモリ容量を消費するように設計されているためです。結果としてこれらの言語は反復プロセスのみを特別な目的の
``ループ構成概念''である\code{do}, \code{repeat}, \code{until}, \code{for}, \code{while}のような物を用いて
記述します。私達が\link{Chapter 5}にて考えるSchemeの実装はこの短所を共有しません。
例え反復プロセスが再帰手続により記述されていても定量的な記憶域にて実行します。
この属性を持つ実装は\newterm{tail-recursive}(末尾再帰)と呼ばれます。末尾再帰の実装を用いれば
反復は一般的な手続呼出メカニズムを用いて表現可能であり、特別な反復構成概念は糖衣構文としてのみ
実益のあるものとなります。\footnote{末尾再帰は長い間コンパイラの最適化のための裏技として知られ
てきました。末尾再帰の論理的な意味論上の基礎はCarl \link{Hewitt (1977)}により与えられました。
彼はそれを演算の``メッセージパッシング''モデルにて説明しました。\link{Chapter 3}にて議論します。
これに影響を受けて、Gerald Jay SussmanとGuy Lewis Steele Jr. (\link{Steele and Sussman 1975}参照)
はSchemeのための末尾再帰インタプリタを構築しました。Steeleは後に末尾再帰が手続呼出をコンパイルするのに
どれだけ自然な方法の結果であるかを示しました(\link{Steele 1977})。Schemeの\acronym{IEEE}標準仕様は
Schemeの実装が末尾再帰であることを必須要件としています。}

\begin{quote}
\heading{\phantomsection\label{Exercise 1.9}Exercise 1.9:} 次の2つの各手続は
2つの正の整数を加算する手段を定義している。手続\code{inc}は引数を
1増やし、\code{dec}は引数を1減らす。

\begin{scheme}
(define (+ a b)
  (if (= a 0) b (inc (+ (dec a) b))))
(define (+ a b)
  (if (= a 0) b (+ (dec a) (inc b))))
\end{scheme}


置換モデルを用いて各手続が\code{(+ 4 5)}の評価において生成するプロセスを図示せよ。
これらのプロセスは反復であるか、再帰であるか?

\end{quote}

\begin{quote}
\heading{\phantomsection\label{Exercise 1.10}Exercise 1.10:} 以下の手続は
アッカーマン関数と呼ばれる数学の関数を計算する。

\begin{scheme}
(define (A x y)
  (cond ((= y 0) 0)
        ((= x 0) (* 2 y))
        ((= y 1) 2)
        (else (A (- x 1) (A x (- y 1))))))
\end{scheme}


以下の式の値はいくつであるか?

\begin{scheme}
(A 1 10)
(A 2 4)
(A 3 3)
\end{scheme}


\code{A}が上で定義された手続である時、以下の手続について考察せよ。

\begin{scheme}
(define (f n) (A 0 n))
(define (g n) (A 1 n))
(define (h n) (A 2 n))
(define (k n) (* 5 n n))
\end{scheme}


\( n \)が正の整数である場合に手続\code{f}, \code{g}, \code{h}により計算される関数の
数学上の定義について簡明に答えよ。例として\code{(k n)}は\( 5n^2 \)を計算する。

\end{quote}


\subsection{木再帰}
\label{Sec. 1.2.2}
\label{Section 1.2.2}
もう1つの演算の一般的パターンは\newterm{tree recursion}(\jnewterm{木再帰})と呼ばれます。
例として、フィボナッチ数の計算について考えてみましょう。各数値は先行する2つの数の
和となります。

\begin{comment}
0, 1, 1, 2, 3, 5, 8, 13, 21, \( \dots \)
\end{comment}
\begin{displaymath}
 0,\; 1,\; 1,\; 2,\; 3,\; 5,\; 8,\; 13,\; 21,\; \dots. 
\end{displaymath}


一般に、フィボナッチ数は次のルールにて定義可能です。

\begin{comment}

\begin{example}
         /
         |  0                        if n = 0
Fib(n) = <  1                        if n = 1
         |  Fib(n - 1) + Fib(n - 2)  otherwise
         \
\end{example}

\end{comment}
\begin{displaymath}
 {\rm Fib}(n) = 
\begin{cases}	
        \; 0 & {\rm if} \;\; n=0, \\
	\; 1 & {\rm if} \;\; n=1, \\
	\; {\rm Fib}(n-1) + {\rm Fib}(n-2) \quad & {\rm otherwise}. 
\end{cases} 
\end{displaymath}


私達は直ぐにこの定義をフィボナッチ数を計算する再帰手続の定義に翻訳が可能です。

\begin{scheme}
(define (fib n)
  (cond ((= n 0) 0)
        ((= n 1) 1)
        (else (+ (fib (- n 1))
                 (fib (- n 2))))))
\end{scheme}

\noindent
この計算のパターンについて考えてみましょう。\code{(fib 5)}を計算するには
\code{(fib 4)}と\code{(fib 3)}を計算します。\code{(fib 4)}を計算するには\code{(fib 3)}と
\code{(fib 2)}を計算します。一般的に展開されたプロセスは\link{Figure 1.5}で示すように
木のように見えます。枝が各レベル(最下層を除く)にて2つに分かれることに注意して
下さい。これが\code{fib}手続が実行される度に毎回、自身を二回呼び出す事実を反映
しています。



この手続は典型的な木再帰としては有益です。しかしフィボナッチ数を計算するには酷い
方法です。あまりにも冗長な計算を行うためです。\link{Figure 1.5}において\code{(fib 3)}
の計算全体が---ほぼ仕事の半分が---重複していることに注意して下さい。実際には手続が
\code{(fib 1)}や\code{(fib 0)}の演算回数(上記の木全体においての葉の数)が正確にFib(\( n+1 \))
であることを示すのは難しくありません。この方法の酷さを知るために、
Fib(\( n \))の値が\( n \)に対し指数関数的に増加することを示すことができます。
より正確にはFib(\( n \))は以下の条件の場合に\( \varphi^n / \sqrt{5} \)に最も近い
整数になります。(\link{Exercise 1.13}参照)

\begin{comment}

\begin{example}
[phi] = (1 + [sqrt]5)/2 ~= 1.6180
\end{example}

\end{comment}
\begin{displaymath}
\varphi = {1 + \sqrt{5}\over2} \approx 1.6180 
\end{displaymath}
\noindent
\( \varphi \)は\newterm{golden ratio}(\jnewterm{黄金比})であり次の等式を満たします。

\begin{comment}

\begin{example}
[phi]^2 = [phi] + 1
\end{example}

\end{comment}
\begin{displaymath}
\varphi^2 = \varphi + 1. 
\end{displaymath}

\begin{figure}[tb]
\phantomsection\label{Figure 1.5}
\centering
\begin{comment}
\heading{Figure 1.5:} The tree-recursive process generated in computing \code{(fib 5)}.

\begin{example}

                   ..<............ fib5   <.......... 
                ...     ___________/  \___________   .  
             ...       /       . .....            \    . 
           ..       fib4     .        . . . .     fib3  .  
         ..     ____/. \____  ..             .  __/  \__  .  
       ..      /  . .  ..   \    .        ..   /  . .   \   . 
     ..     fib3 .       .  fib2 .        . fib2 .   .  fib1 .
   ..      / . \  .     .   /  \  .      .  /  \ ...  .  |  .
 ..       / . . \   .  .   /  . \   .  .   / .  \   .  . 1 .
.      fib2 . . fib1.  .fib1 .  fib0 . .fib1. . fib0 .  .  .
.      /  \  . . |  .  . |  .  . |   . . |   . . |   .   .>
V     /  . \   . 1  .  . 1  .  . 0  .  . 1  .  . 0  ..
.  fib1 .. fib0..  .   .   .   .   .   V   .   ..  . 
.   |  .  . |  . .>     .>.     . .    ..>.      .>
.   1 .   . 0  .      
 .   .     .  .       
  .>.       ..        

\end{example}
\end{comment}
\includegraphics[width=90mm]{fig/chap1/Fig1.5c.pdf}
\begin{quote}
\heading{Figure 1.5:} \code{(fib 5)}を求める際に生成された木再帰プロセス
\end{quote}
\end{figure}

\noindent
従ってプロセスは入力に伴ない指数関数的に増加するステップ数を要します。一方で
要求される記憶域は入力に対し線形にしか増加しません。なぜなら計算過程の任意のポイントに
おいて、木の中のどのノードが上にあるのかのみ追跡する必要があるためです。一般的に、
木再帰プロセスにおいて必要とされるステップ数は木の中のノードの数に比例します。
必要とされる記憶域は木の最大の深さに対して比例します。


フィボナッチ数の計算を反復プロセスに定式化することも可能です。この考えは\( a \)と\( b \)
の整数のペアを用い、Fib(1) = 1とFib(0) = 0の初期化を行い、以下の変換を同時に行うというものです。


\begin{comment}

\begin{example}
a <- a + b
b <- a
\end{example}

\end{comment}
\begin{displaymath}
% \eqalign{a &\gets a + b, \cr 
% b &\gets a. \cr}
\begin{array}{l@{\quad\gets\quad}l}
  a & a + b, \\ 
  b & a.
\end{array}
\end{displaymath}
\noindent
この変換を\( n \)回行った後に\( a \)と\( b \)がそれぞれFib(\( n+1 \))と
Fib(\( n \))に等しいことを示すのは難しくありません。従ってフィボナッチ数を反復的に
以下の手続を用いて計算可能です。

\begin{scheme}
(define (fib n)
  (fib-iter 1 0 n))
(define (fib-iter a b count)
  (if (= count 0)
      b
      (fib-iter (+ a b) a (- count 1))))
\end{scheme}

\noindent
このFib(\( n \))を計算する2つ目の方法は線形反復です。2つの方法により要求される
ステップ数の差は、一方は\( n \)に対し線形であり、もう一方はFib(\( n \))自身の値の速さで
増加するため、例え入力値が小さくてもその差は非常に大きくなります。



これより木再帰プロセスが役に立たないと結論づけるべきではありません。数値ではなく
階層構造のデータを操作するプロセスを考えた場合、木再帰は自然で強力なツールです。
\footnote{これの例は\link{Section 1.1.3}:インタプリタ自身が木再帰プロセスを用いて式を評価する
ことから暗示されます}しかし、例え数値演算においても木再帰はプログラムの設計と理解を
手助けするのに役立ちます。例えば最初の\code{fib}手続は2つ目に比べてとても非効率ですが、
より直感的でフィボナッチ数列の定義とLisp翻訳の違いは大差がありません。反復アルゴリズム
の定式化を行うためには、計算が3つの状態変数に再定義できることに気付く必要があります。

\subsubsection*{例: 両替方法を数える}



反復的フィボナッチアルゴリズムに至るには多少の知恵が必要です。一方で、次の問題
について考えてみて下さい:\$1.00を両替するにはいくつの方法があるでしょうか?
50セント、25セント、10セント、5セント、1セント硬貨があります。
より一般的に、任意の量の金額に対して両替方法がいくつ存在するか計算する手続を書くこと
ができますか?


この問題には再帰手続としての簡単な答が存在します。利用可能なコインのタイプをある
順序で並べると考えてみましょう。すると以下の関係が成り立ちます。


\( n \)種類の硬貨を用いた場合、金額\( a \)の両替方法の数は

\begin{itemize}

\item
最初の種類の硬貨を除いた残り全てを用いた金額\( a \)の両替方法の数、\\

プラス

\item
\( d \)が最初の種類の硬貨の額面である場合に、\( n \)種類の硬貨全てを用いた
金額\( a - d \)の両替方法の数

\end{itemize}

\noindent
なぜこれが正しいのか考えるためには両替方法が2つのグループに分けられることに注目し
ます。最初の種類の硬貨を用いないものと、用いるものです。従ってある金額に対する
両替方法の数の総数は最初の種類の硬貨を全く使わないその金額に対する両替方法の数と最初の
コインを用いる両替方法の数の和です。しかし後者の数は最初の種類の硬貨を用いた後の
残りの金額に対する両替方法の数に等しくなります。


従って与えられた金額の両替問題から少ない種類の硬貨を用いたより少ない金額の両替問題へと
再帰的に縮小することが可能です。この集約ルールについて注意深く考えてください。そして
自分自身でそのルールを用いて以下の縮退ケースを指定した場合アルゴリズムを記述できるよう
に準備して下さい。\footnote{例えば5セント硬貨と1セント硬貨を用いて10セントの両替を
行う問題に縮退ルールをどのように適用するかについて、詳細に通してやって見て下さい}

\begin{itemize}

\item
もし\( a \)が0である場合、両替方法は1と数える

\item
もし\( a \)が0未満の場合、両替方法は0と数える

\item
もし\( n \)が0の場合、両替方法は0と数える

\end{itemize}

\noindent
この記述は簡単に再帰手続に翻訳できます。

\begin{scheme}
(define (count-change amount) (cc amount 5))
(define (cc amount kinds-of-coins)
  (cond ((= amount 0) 1)
        ((or (< amount 0) (= kinds-of-coins 0)) 0)
        (else (+ (cc amount
                     (- kinds-of-coins 1))
                 (cc (- amount
                        (first-denomination 
                         kinds-of-coins))
                     kinds-of-coins)))))
(define (first-denomination kinds-of-coins)
  (cond ((= kinds-of-coins 1) 1)
        ((= kinds-of-coins 2) 5)
        ((= kinds-of-coins 3) 10)
        ((= kinds-of-coins 4) 25)
        ((= kinds-of-coins 5) 50)))
\end{scheme}

\noindent
(\code{first\-/denomination}手続は利用可能な硬貨の種類の数を入力に取り、最初の種類の
硬貨の額面を返します。ここでは硬貨は最大額面から最小への順に並んでいると仮定し
ますが、どのような順でもうまく行きます)これで元々の質問である\$1の両替について
回答ができます。

\begin{scheme}
(count-change 100)
~\textit{292}~
\end{scheme}

\noindent
\code{count\-/change}は\code{fib}の最初の実装と同様に冗長な木再帰プロセスを生成します。
(292が演算されるのに暫く時間がかかるでしょう)
一方で結果を求めるのにより良いアルゴリズムをどのように設計するかは自明ではあり
ません。この問題は読者への宿題とします。木再帰プロセスはとても非効率ですが
多くの場合、指示と理解が簡単であることが人々に対し、ユーザが両者の世界の良い面を
得られる、木再帰手続をより効率的で等価な手続へと変換を行う``賢いコンパイラ''
の設計を提案する方向へと向かわせています。
\footnote{冗長な演算に対処する1つの取り組み方法は演算結果
に従い、自動的に値のテーブルを構築することです。手続をある引数に適用するよう要求される度に、最初にその値が
既にテーブルに存在するかを確認します。その場合、冗長な演算を防ぐことが可能です。
この戦略は\newterm{tabulation}(\jnewterm{表形式化})や\newterm{memoization}(\jnewterm{メモ化})として知られ
直感的な方法で実装が可能です。表形式化は時折、(\code{count\-/change}のような)指数関数的な
ステップ数を要するプロセスを、入力に対し時間と記憶域の要求が線形に増加するプロセスへと
変換するのに利用されます。(\link{Exercise 3.27}参照)}

\begin{quote}
\heading{\phantomsection\label{Exercise 1.11}Exercise 1.11:} 関数\( f \)は
\( n < 3 \)の場合\( f(n) = n \)と\( n \ge 3 \)の場合、\( f(n) = f(n-1) + 2f(n-2) + 3f(n-3) \)
のルールの下に定義される。 \( f \)を演算する手続を再帰プロセスを用いて書け。また\( f \)を演算する手続を
反復プロセスを用いて書け。
\end{quote}

\newpage

\begin{quote}
\heading{\phantomsection\label{Exercise 1.12}Exercise 1.12:} 以下の
数値のパターンは\newterm{Pascal's triangle}(\jnewterm{パスカルの三角形})と呼ばれる。

\begin{example}
        1
      1   1
    1   2   1
  1   3   3   1
1   4   6   4   1
      . . .
\end{example}



三角の端の数値は全て1であり、三角内部の各数値はそれの上2つの数値の和である。
\footnote{パスカルの三角形の各要素は\newterm{binomial coefficients}(二項係数)と
呼ばれます。\( n \)番目の行が\( (x + y)^n \)の展開式における各項の係数であるため
です。係数を計算するこのパターンはBlaise Pascalの1653年の確率理論の独創的な成果
である \textit{Trait\'e du triangle arithm\'etique}に現れました。\link{Knuth (1973)}
によると、1303年に同様のパターンが中国の数学者、朱世傑により出版された\textit{Szu-yuen Y\"u-chien} 
(``The Precious Mirror of the Four Elements'')(四元玉鑑)の中にも記載されています。
また12世紀のペルシャの詩人であり数学者であったOmar Khayyam、同じく12世紀
のインド人数学者Bh\'ascara \'Ach\'aryaについても同様です。}パスカルの三角形の要素を
再帰プロセスを用いて求める手続を書け。
\end{quote}

\begin{quote}
\heading{\phantomsection\label{Exercise 1.13}Exercise 1.13:} Fib(\( n \))が
\( \varphi^n / \sqrt{5} \)に最も近い整数であることを証明せよ。
\( \varphi = (1 + \sqrt{5}) / 2 \)とする。
ヒント:\( \psi = (1 - \sqrt{5}) / 2 \)と置く。帰納法とフィボナッチ数の定義
(\link{Section 1.2.2}参照)を用いてFib(\( n \)) = \( (\varphi^n - \psi^n) / \sqrt{5} \)
であることを証明せよ。
\end{quote}

\subsection{増加のオーダー}
\label{Section 1.2.3}



前節の例ではプロセスが消費する計算資源の割合が大幅に異なることを示しました。
この違いを説明する1つの便利な方法には\newterm{order of growth}(\jnewterm{増加のオーダー})の記法を
用いて入力が大きくなるにつれプロセスが要求するリソース(資源)の総体的量の
説明を得ることがあります。



\( n \)が問題サイズを測るパラメータ、\( R(n) \)をサイズ\( n \)の問題に対し
プロセスが要求するリソースの量だとします。前節の例では\( n \)を与えられた
関数が何回計算されるかの数としました。しかし他の可能性もあります。例えば、
もし私達のゴールが数値の平方根の近似値を求めることであれば、\( n \)を必要な
精度の桁数と取ることもありえるでしょう。行列の乗算では\( n \)を行列の行数と
取るかもしれません。一般的に与えられた問題を分析するのに望ましい問題の属性は
いくつもあります。同様に、\( R(n) \)が使用される内部保管レジスタの数を量った
り、実行された基本的機械語命令の数であったり等します。一度に固定数の命令を
実行する計算機においては必要とされる時間は実行される基本的機械語命令の数に
比例します。



もし任意の十分に大きな\( n \)の値に対して正の定数\( k_1 \)と\( k_2 \)が
\( n \)に独立して存在し\( k_1f(n) \le R(n) \le k_2f(n) \)を満たす時、
\( R(n) \)は増加の次数\( \Theta(f(n)) \)を持ち\( R(n) \) = \( \Theta(f(n)) \)
(``シータ\( f(n) \)''と発音する)と記述されます。




例として、\link{Section 1.2.1}で説明した階乗を求める線形再帰プロセスではステップ数は入力
\( n \)に比例します。従ってこのプロセスに必要なステップ数は\( \Theta(n) \)に
従い増加します。必要とされる記憶域もまた\( \Theta(n) \)に従い増加します。反復式階乗では
ステップ数はまだ\( \Theta(n) \)ですが、記憶域は\( \Theta(1) \)---定数です。
\footnote{これらの記述は多くの過度な単純化を隠しています。例えばもし``機械語命令''を
プロセスのステップ数として数えた場合に、一例として、乗算に対し必要な機械語命令の数は乗算される
2つの数に対し独立していると想定するとします。それは数値が十分に大きな場合は
間違いです。同様の見解が記憶域に対する見積に対しても取られます。プロセスの設計と記述
のように、プロセスの分析は抽象化の色々なレベルに対して行えます。}
木再帰フィボナッチ演算は\( \Theta(\varphi^n) \)ステップと記憶域\( \Theta(n) \)を
必要とします。
この時\( \varphi \)は\link{Section 1.2.2}で示したとおりの黄金比です。



増加の次数はプロセスの行いについて概観的な説明のみを与えます。
例えば\( n^2 \)ステップ、\( 1000n^2 \)ステップ、
\( 3n^2 + 10n + 17 \)ステップを必要とするプロセスは全て増加の次数は
\( \Theta(n^2) \)になります。一方で増加の次数は問題のサイズを変更した場合に
どの程度プロセスの挙動が変化するかを推測するのに実用的な指標です。
\( \Theta(n) \)の線形プロセスに対しサイズを2倍にした場合、概ね2倍のリソースを
使用します。指数関数的プロセスに対しては問題サイズを1増やす度、定数因子を
リソース使用率にかけることになります。\link{Section 1.2}の最後にて増加の次数が対数である
2つのアルゴリズムをでは、問題サイズを倍にした時に必要とするリソースが定数量
増えることを調査します。

\begin{quote}
\heading{\phantomsection\label{Exercise 1.14}Exercise 1.14:} \link{Section 1.2.2}の
\code{count\-/change}手続により11セントの両替を求めた場合に生成される
プロセスの木を図示せよ。量が増えるに従いこのプロセスにより使用される記憶域とステップ数
の増加の次数はいくつか?
\end{quote}

\begin{quote}
\heading{\phantomsection\label{Exercise 1.15}Exercise 1.15:} (ラジアンで指定される)
角度の正弦値は\( x \)が十分に小さい時\( \sin x  \approx x \)
の近似式を用いることで計算できる。そして三角法の恒等式、
\begin{comment}

\begin{example}
               x             x
sin x = 3 sin --- - 4 sin^3 ---
               3             3
\end{example}

\end{comment}
\begin{displaymath}
\sin x = 3\sin {x\over3} - 4\sin^3 {x\over3} 
\end{displaymath}
\noindent
を用いてsinの引数の大きさを縮小することができる。(この課題の目的では``十分に小さい''
とはその大きさが0.1ラジアンよりも大きくないこととする) これらの考えが以下の手続に
組込まれている。

\begin{scheme}
(define (cube x) (* x x x))
(define (p x) (- (* 3 x) (* 4 (cube x))))
(define (sine angle)
   (if (not (> (abs angle) 0.1))
       angle
       (p (sine (/ angle 3.0)))))
\end{scheme}

\begin{enumerate}[a]

\item
手続\code{p}は\code{(sine 12.15)}を評価した時、何回適用されるか?

\item
\code{(sine a)}が評価された時、\code{sine}手続により生成されたプロセスにより使用
された(\( a \)の関数としての)記憶域とステップ数の増加の次数を求めよ。

\end{enumerate}
\end{quote}


\subsection{指数計算}
\label{Section 1.2.4}



与えられた数値の指数関数を求める問題について考えましょう。基数\( b \)と正の整数
である指数\( n \)を引数に取り\( b^n \)を求める手続にします。再帰定義により
これを行う1つの方法は次の通りです。

\begin{comment}

\begin{example}
b^n = b * b^(n - 1)
b^0 = 1
\end{example}

\end{comment}
\begin{displaymath}
% \eqalign{b^n &= b\cdot b^{n-1}, \cr 
% b^0 &= 1, \cr} 
\begin{array}{l@{{}={}}l}
  b^n & b\cdot b^{n-1}, \\ 
  b^0 & 1,
\end{array}
\end{displaymath}

早速、手続に翻訳します。

\begin{scheme}
(define (expt b n)
  (if (= n 0)
      1
      (* b (expt b (- n 1)))))
\end{scheme}

\noindent
これは線形再帰プロセスであり、\( \Theta(n) \)ステップと記憶域\( \Theta(n) \)を
必要とします。階乗と同様にすぐに等価な線形反復へと定式化可能です。

\begin{scheme}
(define (expt b n)
  (expt-iter b n 1))
(define (expt-iter b counter product)
  (if (= counter 0)
      product
      (expt-iter b
                 (- counter 1)
                 (* b product))))
\end{scheme}

\noindent
このバージョンは\( \Theta(n) \)ステップと記憶域\( \Theta(1) \)を必要とします。


指数関数は二乗を連続して用いることでより少ないステップで計算できます。
例えば、\( b^8 \)を以下のように計算するのではなく、
\begin{comment}

\begin{example}
b * (b * (b * (b * (b * (b * (b * b))))))
\end{example}

\end{comment}
\begin{displaymath}
 b\cdot (b\cdot (b\cdot (b\cdot (b\cdot (b\cdot (b\cdot b))))))\,, 
\end{displaymath}

3回の乗算で求めることが可能です。
\begin{comment}

\begin{example}
b^2 = b * b
b^4 = b^2 * b^2
b^8 = b^4 * b^4
\end{example}

\end{comment}
\begin{displaymath}
% \eqalign{	b^2 &= b\cdot b, \cr
%		b^4 &= b^2\cdot b^2, \cr
%		b^8 &= b^4\cdot b^4. \cr} 
\begin{array}{l@{{}={}}l}
  b^2 & b\cdot b, \\
  b^4 & b^2\cdot b^2, \\
  b^8 & b^4\cdot b^4.
\end{array}
\end{displaymath}

この方法は2の冪乗である指数関数についてはうまく働きます。また連続する二乗の利点を
一般的な指数関数の演算に対し以下のルールに従うことで利用可能です。
\begin{comment}

\begin{example}
b^n = (b^(n/2))^2    if n is even
b^n = b * b^(n - 1)  if n is odd
\end{example}

\end{comment}
\begin{displaymath}
% \eqalign{b^n &= (b^{n / 2})^2 \qquad \;\, {\rm if\;} n {\rm\; is\; even}, \cr
% b^n &= b\cdot b^{n-1} \qquad {\rm if\;} n {\rm\; is\; odd}. \cr} 
\begin{array}{l@{{}={}}lr@{\ n\ }l}
  b^n & (b^{n / 2})^2  \;\; & \mbox{if\,} & \mbox{\,is\, even}, \\
  b^n & b\cdot b^{n-1} \;\; & \mbox{if\,} & \mbox{\,is\, odd}.
\end{array}
\end{displaymath}

この方法を手続として表現します。

\begin{scheme}

(define (fast-expt b n)
  (cond ((= n 0) 1)
        ((even? n) (square (fast-expt b (/ n 2))))
        (else (* b (fast-expt b (- n 1))))))
\end{scheme}

\noindent
整数が偶数であるかテストする述語はプリミティブな手続、\code{remainder}を用い
以下のように定義される。

\begin{scheme}
(define (even? n)
  (= (remainder n 2) 0))
\end{scheme}

\noindent
\code{fast\-/expt}により展開されるプロセスは\( n \)の対数に従い記憶域とステップ数の両者が
増加します。これを理解するために\( b^{2n} \)を\code{fast\-/expt}を用いて演算するのに
\( b^n \)の演算よりただ1度のみ多くの乗算が必要であることに注目して下さい。
従って計算可能な指数のサイズは、可能な新規の乗算の度に(大体)倍になります。
このため\( n \)の指数により必要とされる乗算の数は2を底とする\( n \)の対数と同等の
早さにて増加します。このプロセスは\( \Theta(\log n) \)で増加します。
\footnote{より正確に言えば、必要とされる乗算の数は1から\( n \)の基数2の対数未満と
\( n \)の二進数表現における1の数の和になります。この合計が常に\( n \)の基数2の対数
の2倍よりも小さくなります。オーダ(次数)記法の定義に従う任意の定数 \( k_1 \)と
\( k_2 \)により、対数プロセスに対し、その対数の基数は問題ではないため、そのような
プロセス全ては\( \Theta(\log n) \)と説明されることが示されます}



\( \Theta(\log n) \)の増加と\( \Theta(n) \)の増加の違いは\( n \)が大きくなる程
顕著になります。例えば\( n \) = 1000の時\code{fast\-/expt}は14回しか乗算を必要としません。
\footnote{誰が数値を1000乗まで上げることを気にするのだろうかと思うかもしれません。
\link{Section 1.2.6}を参照して下さい}
連続する二乗の考えを用いて対数ステップ数の指数関数を求める反復アルゴリズムを考案
することも可能です。(\link{Exercise 1.16}参照) しかし反復アルゴリズムでは良くあること
ですが、これは再帰アルゴリズムのように直接的に書下すことができません。
\footnote{この反復アルゴリズムは古代から存在します。紀元前200年以前に\'Ach\'arya Pingala
により書かれた\textit{Chandah-sutra}には現れています}
\link{Knuth 1981}の4.6.3節にてこれと指数関数の他の方法について完全な議論と分析を行って
います。

\begin{quote}
\heading{\phantomsection\label{Exercise 1.16}Exercise 1.16:} 連続二乗と対数ステップ数を
用いる\code{fast\-/expt}のような反復指数関数プロセスを展開する
手続を設計せよ。(ヒント:\( (b^{n / 2})^2 = (b^2)^{n / 2} \)を用い、指数\( n \)、
基数\( b \)と共に追加の状態変数\( a \)を保持し状態変換を積\( ab^n \)が状態間に
おいて一定であるという方法にて定義せよ。
プロセスの最初において\( a \)は1を取り、回答はプロセスの終了時に\( a \)の値として
得られる。一般的に、状態間において一定である\newterm{invariant quantity}(\jnewterm{不変量})を
定義する技法は反復アルゴリズムの設計を考える上で強力な方法である。)
\end{quote}

\begin{quote}
\heading{\phantomsection\label{Exercise 1.17}Exercise 1.17:} この節における
指数演算アルゴリズムは指数関数を乗算の繰り返しを用いて実行することを
基本としている。同様な手段で、整数の乗算を加算の繰り返しを用いて実行することも可能だ。
以下の乗算手続(私達の言語が足し算だけ可能で乗算はできないと仮定する)は\code{expt}手続の
類似である。

\begin{scheme}
(define (* a b)
  (if (= b 0)
      0
      (+ a (* a (- b 1)))))
\end{scheme}



このアルゴリズムは\code{b}に対し線形のステップ数を取る。今、和に加えて整数を
倍にする\code{double}と偶数を2で割る\code{halve}があるとする。これらを用いて
\code{fast\-/expt}に類似して対数ステップ数を持つ乗算手続を設計せよ。
\end{quote}

\begin{quote}
\heading{\phantomsection\label{Exercise 1.18}Exercise 1.18:} \link{Exercise 1.16}と
\link{Exercise 1.17}の結果を用いて反復プロセスを生成する2つの
整数を乗算する手続を考案せよ。足し算、double、halveを用い対数ステップ数のアルゴリズム
を使用すること。
\footnote{このアルゴリズムは時折``ロシア農民のかけ算''として知られており、古くから
存在します。その使用例は最も古い数学の書籍の1つ、リンドパピルスにも見られます。これは
紀元前1700年頃にエジプトの筆記者、A'h-moseにより書かれた(そしてより古い書物から写本された)本です。}

\end{quote}

\begin{quote}
\heading{\phantomsection\label{Exercise 1.19}Exercise 1.19:} フィボナッチ数を
対数ステップ数にて求める巧みなアルゴリズムが存在する。\link{Section 1.2.2}の
\code{fib\-/iter}にて\( a \)と\( b \)状態変数の変換\( a \gets a + b \)と\( b \gets a \)を思い出そう。
この変換を\( T \)と呼び、1と0から始めて\( n \)回繰り返して\( T \)を適用した時に
Fib(\( n+1 \))とFib(\( n \))のペアを算出することに注意せよ。言い換えれば、フィボナッチ数は
変換\( T \)の\( n \)乗である\( T^n \)をペア(1, 0)から始めて適用するということである。
ここで\( T \)は\( p=0 \)、\( q=1 \)である時の変換\( T_{pq} \)の特別な形であると考えてみよう。
この時\( T_{pq} \)は\( (a, b) \)を\( a \gets bq + aq + ap \) and \( b \gets bp + aq \)とする。
もしそのような変換\( T_{pq} \)を二回適用した場合にその効果は同形変換 \( T_{p'q'} \)を一回
適用した場合と同じであることを示せ。また \( p \)と\( q \)に対する\( p' \) and \( q' \)を求めよ。
これは\code{fast\-/expt}手続におけるように、\( T^n \)を連続する平方にて求める。
これらを全て一緒に考慮して次の手続を完成させよ。これは対数ステップ数にて実行される。
\footnote{この課題は\link{Kaldewaij 1990}の例をベースにJoe Stoyにより提案されました}

\begin{scheme}
(define (fib n)
  (fib-iter 1 0 0 1 n))
(define (fib-iter a b p q count)
  (cond ((= count 0) b)
        ((even? count)
         (fib-iter a
                   b
                   ~\( \dark \langle \)~~\var{\dark ??}~~\( \dark \rangle \)~   ~\textrm{; compute \textit{p}'}~
                   ~\( \dark \langle \)~~\var{\dark ??}~~\( \dark \rangle \)~   ~\textrm{; compute \textit{q}'}~
                   (/ count 2)))
        (else (fib-iter (+ (* b q) (* a q) (* a p))
                        (+ (* b p) (* a q))
                        p
                        q
                        (- count 1)))))
\end{scheme}
\end{quote}

\subsection{最大公約数}
\label{Section 1.2.5}



2つの整数\( a \)と\( b \)のGreatest Common divisor(\acronym{GCD}:最大公約数)とは
\( a \)と\( b \)の両者を余り無しで割り切れる最大の整数だと定義されます。例えば16と28
の\acronym{GCD}は4です。\link{Chapter 2}で分数の計算の実装方法について調査する時に
分数を約分するために\acronym{GCD}を求められるようになる必要が出てきます。
(分数を約分するためには分母と分子をそれらの\acronym{GCD}で割らねばなりません。例えば
16/28は4/7になります) 2つの整数の\acronym{GCD}を求める1つの方法はそれらを因数分解し、
共通因数を求める方法です。しかしより効率的な有名なアルゴリズムが存在します。



そのアルゴリズムの考えはもし\( r \)が\( a \)を \( b \)で割った時の余りである場合に
\( a \)と\( b \)の共通因数は正確に\( b \)と\( r \)の共通な因数であるという結果を基にして
います。従って次の等式を利用可能です。

\begin{example}
GCD(a,b) = GCD(b,r)
\end{example}

\noindent
引き続いて\acronym{GCD}を求める問題からより小さな値の整数のペアの\acronym{GCD}を
求める問題へと縮小していくことができます。例えば、

\begin{example}
GCD(206,40) = GCD(40,6)
            = GCD(6,4)
            = GCD(4,2)
            = GCD(2,0)
            = 2
\end{example}


\noindent
上の例では\acronym{GCD}(206, 40)を\acronym{GCD}(2, 0)へと縮小しています。その答は2です。
任意の2つの正の整数から始めて収縮を繰り返し実行することで常に最終的には2つ目の数値が0である
ペアにすることができます。その時、\acronym{GCD}の値はもう1つの値です。この\acronym{GCD}を
求める方法は\newterm{Euclid's Algorithm}\jnewterm{ユークリッドの互除法})として知られています。
\footnote{ユークリッドの互除法はユークリッドの\textit{Elements}(およそ紀元前300年
の原論第7巻)に載っていたためにそう呼ばれます。\link{Knuth (1973)}によると最も古く良く知られた
重要なアルゴリズムであると考えられるそうです。(\link{Exercise 1.18})の古代のエジプト人の
乗算方法は確かにこれよりも古いのですが、Knuthの説明ではユークリッドのアルゴリズムは最も古く
知られた一般的なアルゴリズムとして紹介されたものであり、説明的な例の集合では無いとのことです}


ユークリッドの互除法を手続として表すのは簡単です。

\begin{scheme}
(define (gcd a b)
  (if (= b 0)
      a
      (gcd b (remainder a b))))
\end{scheme}

\noindent
これが反復プロセスを生成し、そのステップ数は与えられた数値の対数で増加します。


ユークリッドの互除法により必要とされるステップ数が対数増加する事実がフィボナッチ数に
対する興味深い関係を持ちます。

\begin{quote}
\heading{Lam\'eの定理:}もしユークリッドの互除法があるペアの\acronym{GCD}を求めるのに
\( k \)ステップを必要とする場合、必ずペアの小さな値が\( k \)番目のフィボナッチ数より
大きいか等しい。
\footnote{この定理は1845年にフランスの数学者でありかつエンジニアでもあるGabriel Lam\'eにより
証明されました。彼は数理物理学への貢献の第一人者としても有名です。この定理を証明するには
\( a_k \ge b_k \)であるペア(\( a_k, b_k \))がユークリッドの互除法にて\( k \)ステップで停止
するか考えます。証明は\( (a_{k+1}, b_{k+1}) \to (a_k, b_k) \to (a_{k-1}, b_{k-1}) \)が
縮小プロセスにおける連続する3つのペアである場合、必ず\( b_{k+1} \ge b_k + b_{k-1} \)であること
を基にします。この仮定を確認するために縮小ステップの変換定義が\( a_{k-1} = b_k, b_{k-1} =  \)``\( a_k \)を
\( b_k \)で割った余り''であることについて考えます。2つ目の等式はある整数
\( q \)に対し\( a_k = qb_k + b_{k-1} \)が成り立つことを意味します。\( q \)は少なくとも1ですから
\( a_k = qb_k + b_{k-1} \ge b_k + b_{k-1} \)が成り立ちます。しかし以前の収縮ステップより
\( b_{k+1} = a_k \)です。従って\( b_{k+1} = a_k \ge b_k + b_{k-1} \)が成り立ちます。これで先程の
仮定は立証できました。アルゴリズムが停止するのに必要なステップ数を\( k \)とした場合に、これで定理は
\( k \)を用いた数学的帰納法にて証明可能となりました。\( k = 1 \)の時、これは単に\( b \)が
少なくともFib(1) = 1と同じ大きさであることを必要としますので真です。次に\( k \)に等しいかより小さい
整数全てにおいて定理が真であると仮定します。そして\( k + 1 \)でも成立することを証明します。
\( (a_{k+1}, b_{k+1}) \to (a_k, b_k) \to (a_{k-1}, b_{k-1}) \)が縮小プロセスにおける連続するペア
である場合に、数学的帰納法の仮定より、\( b_{k-1} \ge {\rm Fib}(k - 1) \)と
\( b_k \ge {\rm Fib}(k) \)が成り立ちます。ここで先程フィボナッチ数の定義と共に証明した式を
適用すると\( b_{k+1} \ge b_k + b_{k-1} \ge {\rm Fib}(k) + {\rm Fib}(k-1) = {\rm Fib}(k+1) \)
が導出されます。これで定理の証明は終了です。
}

\end{quote}

\noindent
この定理を用いてユークリッドの互除法の増加のオーダーを推測することが可能です。
\( n \)が手続の入力値の小さな値だとします。もしプロセスが\( k \)ステップ必要と
する場合、\( n \ge {\rm Fib}(k) \approx \varphi^k / \sqrt{5} \)が必ず成り立ちます。
従ってステップ数\( k \)は\( n \)の(\( \varphi \)を底とする)対数で増加します。
つまり増加のオーダーは\( \Theta(\log n) \)となります。


\begin{quote}
\heading{\phantomsection\label{Exercise 1.20}Exercise 1.20:} ある手続が
生成するプロセスはもちろんインタプリタにより使用されるルールに依存する。
例として上で説明した反復\code{gcd}手続について考える。この手続を\link{Section 1.1.5}で議論した
正規順評価で解釈、実行したと想定する。(\code{if}に対する正規順評価ルールは\link{Exercise 1.5}
を参照)。置換法を(正規順に)用いて\code{(gcd 206 40)}の評価により生成されるプロセスを
説明せよ。次に実際に実行された\code{remainder}命令を示せ。\code{(gcd 206 40)}の正規順評価に
おいて実際に実行された\code{remainder}命令は何回だろうか? 適用順評価では?
\end{quote}


\subsection{例: 素数判定}
\label{Section 1.2.6}



この節では整数\( n \)が素数であるかをテストする2つの方法について述べます。
1つは増加のオーダーが\( \Theta(\sqrt{n}) \)であり、他は``確率的''な
アルゴリズムで増加のオーダーが\( \Theta(\log n) \)です。
この節の最後の課題ではこれらのアルゴリズムに基づいたプログラミングのプロジェクトを
提案します。

\subsubsection*{約数を探す}



古代の時代から数学者は素数についての問題に魅惑されてきました。多くの人々が数値が素数であるか
の決定法の問題に取り組んできました。数値が素数であるかのテストの1つの方法は数値の約数を
求めることです。次のプログラムは1より大きな最も小さい整因子(約数)を与えられた\( n \)
に対して求めます。このプログラムはそれを直接的な方法、つまり2で始まる一連の整数により割り切れる
かどうかをテストすることにより行います。

\begin{scheme}
(define (smallest-divisor n) (find-divisor n 2))
(define (find-divisor n test-divisor)
  (cond ((> (square test-divisor) n) n)
        ((divides? test-divisor n) test-divisor)
        (else (find-divisor n (+ test-divisor 1)))))
(define (divides? a b) (= (remainder b a) 0))
\end{scheme}

\noindent
数値が素数であるか以下のようにテストします: \( n \)は\( n \)自身が最小の約数である
場合、かつその場合に限り素数である。

\begin{scheme}
(define (prime? n)
  (= n (smallest-divisor n)))
\end{scheme}

\noindent
\code{find\-/divisor}の終了条件はもし\( n \)が素数でないならば\( \sqrt{n} \)より小さいか
または等しい約数を持つという事実に基づいています。
\footnote{もし\( d \)が\( n \)の約数である時、\( n / d \)もまた約数です。
しかし\( d \)と\( n / d \)の両者が共に\( \sqrt{n} \)より大きいことは有りません}
これはこのアルゴリズムが1から\( \sqrt{n} \)までの約数についてのみテストすれば良いことを
示します。結果として、\( n \)が素数であるかを判定するのに必要なステップ数の増加のオーダーは
\( \Theta(\sqrt{n}) \)となります。

\subsubsection*{フェルマーテスト}



\( \Theta(\log n) \)の素数判定はフェルマーの小定理として知られる数論の結果に
基づきます。\footnote{Pierre de Fermat (1601-1665)は現在の整数論の創始者と
考えられています。彼は多くの重要な数論上の事実について発見しました。しかし彼は
通常その結果のみを公表し、証明を与えませんでした。フェルマーの小定理は彼が1640年に
書いた手紙に記録されています。最初に出版された証明は1736年にオイラーにより与え
られました。(それより早く、同様の証明がライプニッツの出版されなかった原稿に
見つかっています)。最も有名なフェルマーの数式は---フェルマーの最終定理として知られ---
1637年に彼の所有した書籍(3世紀のギリシャ人数学者Diophantusによる)\textit{Arithmetic}
に``私は真に驚くべき証明を発見したが、書き残すにはこの余白は狭すぎる''という所感
と共にメモされた物です。フェルマーの最終定理の証明を見つけることは数論において
最も有名な挑戦の1つとなりました。完全な解はついに1995年にプリンストン大学の
Andrew Wilesにより与えられました。}

\begin{quote}
\heading{フェルマーの小定理:} \( n \)が素数かつ\( a \)が\( n \)より小さい任意の正の整数である時、
\( a \)の\( n \)乗は法\( n \)に関して\( a \)と合同である。
\end{quote}

\noindent
(2つの数値はその両方が\( n \)で割った時に同じ余りを持つ場合、\newterm{congruent modulo} \( n \)
(\jnewterm{法\( n \)に関して合同})と
呼ばれます。また\( a \)を\( n \)で割った時の余りは\( a \) \newterm{modulo} \( n \)の\newterm{remainder}(剰余)、
または単純に\( a \) \newterm{modulo} \( n \)と呼ばれます。)


もし\( n \)が素数でなければ一般に\( a < n \)の多くの値は上記の関係を満しません。
これが次の素数判定のアルゴリズムへと導きます: ある値\( n \)が与えられた時、\( a < n \)と
なる乱数を取り\( a^n \) modulo \( n \)の剰余を求めます。もし結果が\( a \)に等しくない時、
\( n \)は確実に素数ではありません。もし\( a \)に等しいならば\( n \)が素数である確率は良いと言えます。
ここで別の乱数\( a \)を取り同じ方法でテストを行います。それもまた等式を満すのであれば\( n \)が
素数である確率はより確からしくなります。より多くの\( a \)について試験を行えば、結果の確からしら
を増すことが可能です。このアルゴリズムはフェルマーテストとして知られています。


フェルマーテストを実装するには(ある数値の指数関数 modulo 別の数値)を求める手続が必要です。

\begin{scheme}
(define (expmod base exp m)
  (cond ((= exp 0) 
         1)
        ((even? exp)
         (remainder 
          (square 
           (expmod base (/ exp 2) m))
          m))
        (else
         (remainder
          (* base 
             (expmod base (- exp 1) m))
          m))))

\end{scheme}

\noindent
これは\link{Section 1.2.4}の\code{fast\-/expt}手続にとても似ています。連続する二乗を用いるため、ステップ数の増加は
"指数"引数の対数になります。
\footnote{指数\( e \)が1より大きい場合の縮小ステップは、任意の整数\( x \),
\( y \), \( m \)に対し\( x \) modulo \( m \)と\( y \) modulo \( m \)を別々に求め、これらを
掛け、その結果の法\( m \)に関する剰余を求めることで(\( x \)と\( y \)の積 modulo \( m \))を求めることが
できるという事実に基づきます。例えば\( e \)が偶数の場合に\( b^{e / 2} \) modulo \( m \)を求め、
その二乗を取り、法\( m \)に関する剰余を得ます。このテクニックはとても役に立ちます。\( m \)より
もはるかに大きな数値を一切扱う必要無しに演算を行うことが可能だからです。
(\link{Exercise 1.25}と比較せよ)}



フェルマーテストは1から\( n-1 \)までの乱数\( a \)を選択し、\( a \)の\( n \)乗のmodulo \( n \)
が\( a \)に等しいかをチェックすることで行います。乱数\( a \)は手続\code{random}を用いて選択しますが、それは
Schemeのプリミティブな手続に存在する前提です。\code{random}は入力の整数よりも少ない非負数な整数
を返します。そのため1から\( n-1 \)の乱数を得るには\code{random}に\( n-1 \)を入力とし、結果に1を足します。

\begin{scheme}
(define (fermat-test n)
  (define (try-it a)
    (= (expmod a n n) a))
  (try-it (+ 1 (random (- n 1)))))
\end{scheme}

\noindent
以下の手続はパラメータにより与えられた数値の回数分、テストを実行します。
テストが毎回成功すればtrueを、そうでなければfalseを返します。

\begin{scheme}
(define (fast-prime? n times)
  (cond ((= times 0) true)
        ((fermat-test n) (fast-prime? n (- times 1)))
        (else false)))
\end{scheme}


\subsubsection*{確率的手法}


フェルマーテストは正確さが保証された多くの親しみのあるアルゴリズムとは性格が異なっています。
ここでは得られた結果は確率的にのみ正しいと言えます。より正確には、\( n \)が常にフェルマーテストに
失敗するのであれば\( n \)が素数でないことは確実です。しかし\( n \)がテストをパスした
という結果は、とても強い目安ではありますが、\( n \)が素数であることを保証しません。
ここで言いたかったのは任意の数値\( n \)に対し、十分な回数のテストを行い\( n \)が常にテストを
パスする場合、この素数判定が間違いである可能性は思い通りに小さくすることが
可能だということでした。



残念ながらこの主張は完全には正しくありません。実はフェルマーテストを騙してしまう
数値が存在します。\( a < n \)となる全ての整数において、\( n \)が素数ではなく、しかし、
\( a^n \)が\( n \)を法とする\( a \)に合同であるような数値\( n \)。
そのような数値はとても稀です。そのためフェルマーテストは実際にとても信用が高いと言えます。
\footnote{\label{Footnote 1.47} フェルマーテストを騙してしまう数は\newterm{Carmichael numbers}(\jnewterm{カーマイケル数})
と呼ばれとても稀であるということ以外はあまり良くわかっていません。100,000,000未満には255の
カーマイケル数が存在します。最小の物からいくつか上げると561, 1105, 1729,
2465, 2821,  6601です。任意に選ばれたとても巨大な数値の素数性をテストする場合にフェルマーテスト
を騙す数値に当る確率はコンピュータが``正確な''アルゴリズムを実行する際に宇宙放射線がエラーを
引き起す確率よりも低いです。2つ目の理由でなく、最初の理由からアルゴリズムを不適切だと考えることは
数学とエンジニアリングの間の違いを示しています。}



フェルマーテストのバリエーションには騙されない物も複数あります。これらのテストでは
フェルマーテストと同様に、整数\( n \)の素数判定を乱数\( a < n \)を選択し、
\( n \)と\( a \)に依る何らかの条件をチェックします。(そのようなテストの例は
\link{Exercise 1.28}を参照して下さい)。一方でフェルマーテストとは対照的に、任意の
\( n \)に対し\( n \)が素数でなければ\( a < n \)の多くに対し条件が成立しないことを
証明できます。従って\( n \)がいくつかの乱数\( a \)に対してテストが通るのであれば、
\( n \)が素数である可能性は五分五分より高くなります。もし\( n \)が2つの乱数である\( a \)
に対してテストを通れば、\( n \)が素数である確率は4分の3よりも高くなります。テストを
何度も乱数\( a \)を選択しながら実行することでエラーの確率を思い通りに小さくすることが
可能です。



エラーの確率が自由裁量で小さくできることが証明可能なテストの存在はこのタイプの
アルゴリズムへの興味を起こしました。それらは\newterm{probabilistic algorithms}
(\jnewterm{確率的アルゴリズム})と呼ばれます。この領域にはとても多くの研究活動が存在し、
確率的アルゴリズムは多くの現場に効果的に適用されてきました。\footnote{確率的
素数判定法の最も特筆すべき適用例は暗号の領域です。現時点では任意の200桁の数値を
因数分解することは計算能力上不可能ですが、そのような数値の素数判定はフェルマー
テストにより数秒で行うことが可能です。この事実が\link{Rivest et al. (1977)}により
提案された``解読不能な符号''を構築するためのテクニックの基を形成しました。
その結果として\jnewterm{RSAアルゴリズム}は電子通信上のセキュリティを拡張する
テクニックとして広く利用されるようになりました。このことと関連する開発により、
素数の研究は一時は``純粋''数学のトピックの典型例でありその世界自身のために
のみ研究される物だと考えられてきましたが、現在では暗号、電子資産の転送と情報検索に
対する重要で現実的な適用例を持つことが判明しました。}

\begin{quote}
\heading{\phantomsection\label{Exercise 1.21}Exercise 1.21:}\code{smallest\-/divisor}手続を
用いて次の数値の最小の約数を求めよ:199、1999、19999
\end{quote}

\begin{quote}
\heading{\phantomsection\label{Exercise 1.22}Exercise 1.22:} 多くのLisp実装は
\code{runtime}と呼ばれるプリミティブな手続を持っておりそれはシステムが実行
している間の(例えばマイクロ秒で測定された)時間を整数にて返す。次の\code{timed\-/prime\-/test}
手続は整数\( n \)と共に呼んだ時、\( n \)を表示し、\( n \)が素数であるかチェックする。
\( n \)が素数であれば手続は3つのアスタリスクとテスト実行に掛った時間を表示する。

\begin{scheme}
(define (timed-prime-test n)
  (newline) (display n) (start-prime-test n (runtime)))
(define (start-prime-test n start-time)
  (if (prime? n) 
      (report-prime (- (runtime) start-time))))
(define (report-prime elapsed-time)
  (display " *** ") (display elapsed-time))
\end{scheme}

この手続を用いて指定した範囲の連続した奇数について素数判定を行う手続、\code{search\-/for\-/primes}
を書け。その手続を用いて1000、10,000、100,000より大きな素数を3つ見つけよ。
各素数のテストに必要な時間を記録せよ。テストアルゴリズムは10,000辺りの素数を判定する時、
約\( \sqrt{10} \)倍、1000辺りの素数をテストするより時間がかかるはずである。
あなたの結果はこれに従っているか?
100,000や1,000,000のデータに対して\( \Theta(\sqrt{n}) \)の予想は当たっているか?
あなたの結果は演算に必要なステップ数に比例して実行時間が増えるという考えに矛盾していないか?

\end{quote}

\begin{quote}
\heading{\phantomsection\label{Exercise 1.23}Exercise 1.23:} この節の最初で示された
\code{smallest\-/divisor}手続は必要の無いテストを数多く行っている:
数値が2で割ることができるかチェックした後に、それがより大きな偶数にて割り切れるかチェックを
行う必要は無い。これにより\code{test\-/divisor}の値は2, 3, 4, 5, 6, \( \dots \)ではなく、
2, 3, 5, 7, 9, \( \dots \)であるべきだと提案できる。この変更を実装するために、入力が2であれば3
を返し、それ以外では入力に2を足した値を返す手続\code{next}を定義せよ。\code{smallest\-/divisor}手続を
変更し、\code{(+ test\-/divisor 1)}の代わりに\code{(next test\-/divisor)}を使用せよ。
\code{timed\-/prime\-/test}をこの変更したバージョンの\code{smallest\-/divisor}を用いて
\link{Exercise 1.22}で見つけた12の素数に対しテストを行え。
この変更はテストステップを半分にするため2倍速く実行されることをが予測される。
この予測が確認できるだろうか? もしそうでなければ2つのアルゴリズムのスピードの比率は
どのような値が確認できるか? 2と異なる結果をどのように説明するか?
\end{quote}

\begin{quote}
\heading{\phantomsection\label{Exercise 1.24}Exercise 1.24:} \link{Exercise 1.22}の
\code{timed\-/prime\-/test}手続を変更し\code{fast\-/prime?}(フェルマー法)を用い、
課題で見つけた12の素数をそれぞれテストせよ。フェルマーテストは\( \Theta(\log n) \)で
増加するが、1000に近い素数をテストするのに必要な時間と比べ1,000,000付近の素数をテストする
のに必要な時間をどれ程と見繕うだろうか? 実際との相違をどのように説明できるか?
\end{quote}

\begin{quote}
\heading{\phantomsection\label{Exercise 1.25}Exercise 1.25:} Alyssa P. Hackerは
\code{expmod}を書くにあたって多くの余分な仕事を行ったと文句を
言った。結局のところ我々は既に指数演算のやり方を知っているのだから単純に
以下のように書くことができたはずだと彼女は言った。

\begin{scheme}
(define (expmod base exp m)
  (remainder (fast-expt base exp) m))
\end{scheme}

彼女は正しいだろうか? この手続は最初の素数判定と同様にうまく行えるだろうか?
\end{quote}

\begin{quote}
\heading{\phantomsection\label{Exercise 1.26}Exercise 1.26:} Louis Reasonerは
\link{Exercise 1.24}を行うのに随分と苦労した。彼の\code{fast\-/prime?}
テストは彼の\code{prime?}テストよりも随分遅いようだ。Louisは友達のEva Lu Atorを呼んで
助けを求めた。彼らがLouisのコードを試してみると、Louisが\code{expmod}手続を\code{square}手続を呼ぶのでなく、
明示的に乗算を用いていることを見つけた。

\begin{scheme}
(define (expmod base exp m)
  (cond ((= exp 0) 1)
        ((even? exp)
         (remainder (* (expmod base (/ exp 2) m)
                       (expmod base (/ exp 2) m))
                    m))
        (else
         (remainder (* base 
                       (expmod base (- exp 1) m))
                    m))))
\end{scheme}



``これがどんな違いを生んでいるのかわからないよ''とLouisは言った。``私にはわかる''と
Evaが言う。``手続をそのように記述することで、\( \Theta(\log n) \)のプロセスを
\( \Theta(n) \)のプロセスに変えてしまったの。''
説明せよ。
\end{quote}

\begin{quote}
\heading{\phantomsection\label{Exercise 1.27}Exercise 1.27:} \link{Footnote 1.47}にて
並べられたカーマイケル数が実際にフェルマーテストを騙すか実演せよ。
整数\( n \)を取り\( a^n \)が\( a \)と法\( n \)に関して合同であるか全ての\( a < n \)
に対しテストを行う手続を書き、与えられたカーマイケル数に対してその手続を試せ。
\end{quote}

\begin{quote}
\heading{\phantomsection\label{Exercise 1.28}Exercise 1.28:} 騙されないフェルマーテストの
1つの変形として\newterm{Miller-Rabin test} (\link{Miller 1976};
\link{Rabin 1980})がある。これはフェルマーの小定理の代替形から始めるが、それは\( n \)
が素数でかつ\( a \)が\( n \)以下の任意の正の整数である時、\( a \)の\( (n-1) \)乗は
法\( n \)に関して1と合同であると定める。Miller-Rabinテストで数値\( n \)の素数判定を行うには
乱数\( a < n \)を選択し、\( a \)の\( (n-1) \)乗の\( n \)を法とする剰余を
\code{expmod}手続を用いて求める。しかし、\code{expmod}手続中で二乗するステップにおいて毎回、
``自明でない法\( n \)に関する1の平方根''を見つけたかチェックを行う。これは1または\( n-1 \)に
等しくない数値でかつ、法\( n \)に関して二乗した値の剰余が1に等しい数値である。そのような
自明でない1の平方根が存在すれば\( n \)が素数ではないことが証明可能である。
またもし\( n \)が素数でない奇数である時、少なくとも\( a < n \)の半分において
このような方法で\( a^{n-1} \)を演算すると自明でない法\( n \)に関する1の平方根が現れることが
証明可能である。(これがなぜMiller-Rabinテストが騙されないかである)。
\code{expmod}手続を変更し自明でない1の平方根を見つけた時合図を送るようにし、それを用いて
\code{fermat\-/test}に似たMiller-Rabinテストを実装せよ。既知の素数、非素数を用いてあなたの手続を
チェックせよ。ヒント:\code{expmod}に合図を送らせる簡単な方法は0を返させることである。

\end{quote}

\vspace{0.5em}
\section{高階手続による抽象の形式化}
\label{Section 1.3}



私達はここまでで手続が事実上、特定の値から独立した数値への複合命令を記述する
抽象化であることを見てきました。例えば、

\begin{scheme}
(define (cube x) (* x x x))
\end{scheme}

\noindent
これは特定の値の立方について述べているのではなく、任意の数値の立方を得るための手法
について述べている訳です。もちろんこの手続を定義することなく常に以下のような式を書くことで
やっていくことも可能です。
\begin{scheme}
(* 3 3 3)
(* x x x)
(* y y y)
\end{scheme}

\noindent
そして明示的に\code{cube}について触れないことも可能でしょう。しかしこれはとても
大きな不便を与えます。高レベルな命令の用語ではなく常に言語内にプリミティブとして
偶然存在するレベルの特定の命令レベルにて働かざるを得なくなります(このケースでは乗算です)。
私達のプログラムは立方を計算可能ですが、私達の言語は立方のコンセプトを表現する能力が
欠けているかもしれません。私達が強力なプログラミング言語から望むべき物の1つは共通の
パターンに対し名前を付けることで抽象を構築し、その後抽象化の用語にて直接働く能力です。
手続はこの能力を与えます。これがなぜ原始的な物を除いた全てのプログラミング言語にて
手続を定義するメカニズムが含まれているかの理由です。



それにも関わらず数値演算ですら、もし手続のパラメータが数値のみであると制約されていれば
抽象化を行うには我々の能力は非常に大きく制限されていると言えるでしょう。
しばしば同じプログラムのパターンがいくつもの異なる手続にて使用されます。そのような
パターンを表現するには引数として手続を受け入れることができるか、手続を値として返すような
手続を構築する必要が出てきます。手続を操作する手続は\newterm{higher-order procedures}(\jnewterm{高階手続})
と呼ばれます。この節では高階手続がどのように強力な抽象化メカニズムを果たし、言語の表現力を
幅広く増大するかを示します。



\subsection{引数としての手続}
\label{Sec. 1.3.1}
\label{Section 1.3.1}


次の3つの手続について考えてみて下さい。1つ目は\code{a}から\code{b}の整数の合計を
計算します。

\begin{scheme}
(define (sum-integers a b)
  (if (> a b) 
      0
      (+ a (sum-integers (+ a 1) b))))
\end{scheme}

\noindent
2つ目は与えられた範囲の整数の立方の合計を計算します。

\begin{scheme}
(define (sum-cubes a b)
  (if (> a b)
      0
      (+ (cube a)
         (sum-cubes (+ a 1) b))))
\end{scheme}

\noindent
3つ目は以下の級数の一連の項の合計を計算します。
\begin{comment}

\begin{example}
  1       1       1
----- + ----- + ------ + ...
1 * 3   5 * 7   9 * 11
\end{example}

\end{comment}
\begin{displaymath}
 {1\over1\cdot 3} +  {1\over5\cdot 7} + {1\over9\cdot 11} + \dots, 
\end{displaymath}
\noindent
これは\( \pi / 8 \)に(とてもゆっくりと)収束します。\footnote{この級数は
一般に等価である形式\( {\pi\over4} = 1 - {1\over3} + {1\over5} - {1\over7} + \dots \)
にて記述されるLeibnizによる物です。私達はこれがある高級な数学上のトリックに使われるのを
\link{Section 3.5.3}で見ることになります。}

\begin{scheme}
(define (pi-sum a b)
  (if (> a b)
      0
      (+ (/ 1.0 (* a (+ a 2)))
         (pi-sum (+ a 4) b))))
\end{scheme}

\noindent
これらの3つの手続は明確に共通な基礎をなすパターンを共有しています。それらは
ほとんどの部分が同一で、手続の名前、和を求める項を\code{a}を用いて演算する関数、\code{a}の
次の値を与える関数のみが異なります。各手続を同じテンプレートを用いて枠を埋めることで
生成することができそうです。

\begin{scheme}
(define (~\( \dark \langle \)~~\var{\dark name}~~\( \dark \rangle \)~ a b)
  (if (> a b)
      0
      (+ (~\( \dark \langle \)~~\var{\dark term}~~\( \dark \rangle \)~ a)
         (~\( \dark \langle \)~~\var{\dark name}~~\( \dark \rangle \)~ (~\( \dark \langle \)~~\var{\dark next}~~\( \dark \rangle \)~ a) b))))
\end{scheme}

\noindent
このような共通パターンの存在は便利な抽象化が表に浮かび上がるのを待っていることを
示す強力な証拠です。実際に数学者は大昔に\newterm{summation of a series}(\jnewterm{級数の和})の
抽象化を特定し``シグマ記法''を開発しました。つまり、
\begin{comment}

\begin{example}
  b
 ---
 >    f(n) = f(a) + ... + f(b)
 ---
 n=a
\end{example}

\end{comment}
\begin{displaymath}
 \sum_{n=a}^b f(n) = f(a) + \ldots + f(b), 
\end{displaymath}
\noindent
このように表現します。シグマ記法の力は数学者に特定の合計のみについてではなく、
総和のコンセプト自身について取り扱うことを可能にしました。例えば特定の級数の和を求めることから
独立して一般的な総和についての結果を形式化することを可能としたのです。





同様に、プログラムの設計者である私達は言語に、特定の総和を求める手続のみでなく、
総和自身のコンセプトを表現する手続を書くことができるのに十分に強力になって欲しいと
願うでしょう。そうすることが直ぐに私達の手続言語にて上記にて示された共通テンプレートを
用いて、``枠''を形式パラメータに変換することで可能です。

\begin{scheme}
(define (sum term a next b)
  (if (> a b)
      0
      (+ (term a)
         (sum term (next a) next b))))
\end{scheme}

\noindent
\code{sum}が引数として下限と上限の\code{a}と\code{b}を手続\code{term}と\code{next}と一緒に
取ることに注意して下さい。\code{sum}はこれから行うように好きな手続を使用することが
できます。例えばそれを(引数に1を足す手続\code{inc}と共に)\code{sum\-/cubes}の定義に利用
可能です。

\begin{scheme}
(define (inc n) (+ n 1))
(define (sum-cubes a b)
  (sum cube a inc b))
\end{scheme}

\noindent
これを用いて整数1から10の立方の和を求めることができます。

\begin{scheme}
(sum-cubes 1 10)
~\textit{3025}~
\end{scheme}

\noindent
termを求めるidentityプロシジャの助けを借りて、\code{sum}を用いて\code{sum\-/integers}
の定義ができます。

\begin{scheme}
(define (identity x) x)
(define (sum-integers a b)
  (sum identity a inc b))
\end{scheme}

\noindent
これで1から10までの整数の和を求められます。

\begin{scheme}
(sum-integers 1 10)
~\textit{55}~
\end{scheme}

\noindent
\code{pi\-/sum}も同様に定義可能です。\footnote{ブロック構造(\link{Section 1.1.8})を\code{pi\-/sum}の中に
\code{pi\-/next}と\code{pi\-/term}の定義を埋め込むために使用していることに注意して下さい。
これらの手続は任意の他の手続に対し有用ではなかろうためです。
それらを一緒にどのように取り除くかについては\link{Section 1.3.2}で説明します。}

\begin{scheme}
(define (pi-sum a b)
  (define (pi-term x)
    (/ 1.0 (* x (+ x 2))))
  (define (pi-next x)
    (+ x 4))
  (sum pi-term a pi-next b))
\end{scheme}

\noindent
これらの手続を用いて\( \pi \)の近似値を求められます。

\begin{scheme}
(* 8 (pi-sum 1 1000))
~\textit{3.139592655589783}~
\end{scheme}

\noindent
\code{sum}を手に入れたことで、それを構築用ブロックとしてより多くのコンセプトの形式化にて
利用可能です。例えば関数\( f \)の\( a \)と\( b \)の限度値の間の定積分は以下の式を
用いて小さな値\( dx \)に対し数値的に近似可能です。
\begin{comment}

\begin{example}
/b     /  /     dx \    /          dx \    /           dx \      \
|  f = | f| a + -- | + f| a + dx + -- | + f| a + 2dx + -- | + ...| dx
/a     \  \     2  /    \          2  /    \           2  /      /
\end{example}

\end{comment}
\begin{displaymath}
{\int_a^b \!\!\! f} \, = {\left[\;f\!\left(a + {dx\over 2}\right) 
		+ f\!\left(a + dx + {dx\over 2}\right) 
		+ f\!\left(a + 2dx + {dx\over 2}\right) + \,\dots \;\right] \! dx} 
\end{displaymath}
\noindent
これを直接、手続として表現します。

\begin{scheme}
(define (integral f a b dx)
  (define (add-dx x) 
    (+ x dx))
  (* (sum f (+ a (/ dx 2.0)) add-dx b) 
     dx))

(integral cube 0 1 0.01)
~\textit{.24998750000000042}~

(integral cube 0 1 0.001)
~\textit{.249999875000001}~
\end{scheme}

\noindent
(\code{cube}の0から1の実際の定積分の値は1/4です。)

\begin{quote}
\heading{\phantomsection\label{Exercise 1.29}Exercise 1.29:} シンプソンの公式は
上記にて示された方法よりもより正確な数値積分の方法である。
シンプソンの公式を用いて\( a \)と\( b \)の間の\( f \)の定積分は次のように
近似される。
\begin{comment}

\begin{example}
h
- (y_0 + 4y_1 + 2y_2 + 4y_3 + 2y_4 + ... + 2y_(n-2) + 4y_(n-1) + y_n)
3
\end{example}

\end{comment}
\begin{displaymath}
 {h\over 3}(y_0 + 4y_1 + 2y_2 + 4y_3 + 2y_4 + \ldots + 2y_{n-2} + 4y_{n-1} + y_n), 
\end{displaymath}
\noindent
ここで\( h = (b - a) / n \)、\( n \)は偶数、\( y_k = f(a + kh) \)である。
(\( n \)を増やすことで近似の精度を高めることができる)。
\( f \), \( a \), \( b \), \( n \)を引数に取りシンプソンの公式を用いて求めた定積分の
値を返す手続を定義せよ。
\end{quote}

\begin{quote}
\heading{\phantomsection\label{Exercise 1.30}Exercise 1.30:} 上の\code{sum}手続は
線形再帰を生成する。手続は和の計算が線形で行われるよう書き直すことが可能だ。
次の定義にて消された表記を埋め、どのように行うのか示せ。

\begin{scheme}
(define (sum term a next b)
  (define (iter a result)
    (if ~\( \dark \langle \)~~\var{\dark ??}~~\( \dark \rangle \)~
        ~\( \dark \langle \)~~\var{\dark ??}~~\( \dark \rangle \)~
        (iter ~\( \dark \langle \)~~\var{\dark ??}~~\( \dark \rangle \)~ ~\( \dark \langle \)~~\var{\dark ??}~~\( \dark \rangle \)~)))
  (iter ~\( \dark \langle \)~~\var{\dark ??}~~\( \dark \rangle \)~ ~\( \dark \langle \)~~\var{\dark ??}~~\( \dark \rangle \)~))
\end{scheme}
\end{quote}

\begin{quote}
\heading{\phantomsection\label{Exercise 1.31}Exercise 1.31:} \begin{enumerate}[a]

\item
\code{sum}手続は高階手続として捉えられる非常に多くの数の同様な抽象化の最も簡単な物に
すぎない。\footnote{\link{Exercise 1.31}から\link{Exercise 1.33}の目的は、
多くの一見では異なる操作を統一するために、適切な抽象化を利用することで達成される表現力を実演することです。
しかし、集積やフィルタリングは洗練された考えではありますが、この時点ではそれらを使用するのに我々の両手が縛られて
いるようなものです。私達はまだこれらの抽象化のための適切な組み合わせの手段を与えるためのデータ構造
を持っていないためです。私達は\link{Section 2.2.3}にてこれらの考えに立ち戻り、集積とフィルタを
組み合わせるためのインターフェイスとして\newterm{sequences} (\jnewterm{列})をどのように使うのかを示し
さらに一層強力な抽象化を構築します。そこではこれらの手法が実際にどのようにしてプログラムを設計するのに
強力で洗練されたアプローチとして役に立つのかを学びます。}
\code{product}と呼ぶ与えられた範囲の点の関数値の積を返す類似の手続を書け。
どのようにして\code{product}を用いて\code{factorial}を定義するのか示せ。また\code{product}を用いて
次の式を使用して\( \pi \)の近似値を計算せよ。\footnote{この式は17世紀に英国人数学者John Wallisにより
発見されました。}

\begin{comment}

\begin{example}
pi   2 * 4 * 4 * 6 * 6 * 8 ...
-- = -------------------------
 4   3 * 3 * 5 * 5 * 7 * 7 ...
\end{example}

\end{comment}
\begin{displaymath}
 {\pi\over 4} = {2\cdot 4\cdot 4\cdot 6\cdot 6\cdot 8\dots\over 
		   3\cdot 3\cdot 5\cdot 5\cdot 7\cdot 7\dots}\,. 
\end{displaymath}
\item
もしあなたの\code{product}手続が再帰プロセスを生成するのであれば線形プロセスを生成するものを書け。
もし線形プロセスを生成するのであれば再帰プロセスを生成するものを書け。

\end{enumerate}
\end{quote}

\begin{quote}
\heading{\phantomsection\label{Exercise 1.32}Exercise 1.32:} \begin{enumerate}[a.]

\item
\code{sum}と\code{product} (\link{Exercise 1.31})は両方とも、汎用の集積関数を用いて
項の集合を結合する\code{accumulate}(集積)と呼ばれるより一般的な目的の特別なケースに過ぎない。

\begin{scheme}
(accumulate combiner null-value term a next b)
\end{scheme}



\code{accumulate}は引数として\code{sum}と\code{product}と同じく項と範囲の指定を
(2つの引数の)\code{combiner}手続と \code{null-value}を共に得る。\code{combiner}は
どのように現在の項が以前の項の集積と結合されるかを指定し、 \code{null-value}は
項が尽きた時に使用する基となる値を指定する。\code{accumulate}を書き\code{sum}と\code{product}
の両者がどのように簡単な\code{accumulate}の呼び出しで定義できるかを示せ。

\item
あなたの\code{accumulate}が再帰プロセスを生成するのなら線形プロセスを生成する物を書け。
もし線形プロセスを生成するのならば再帰プロセスを生成する物を書け。

\end{enumerate}
\end{quote}

\begin{quote}
\heading{\phantomsection\label{Exercise 1.33}Exercise 1.33:} より汎用的なバージョンの
\code{accumulate} (\link{Exercise 1.32})を結合される項の
\newterm{filter}(\jnewterm{フィルタ})の概念を紹介することで得ることが可能だ。指定された条件を
満たす範囲の値から導かれる項のみを連結する。結果としての\code{filtered\-/accumulate}抽象は
\code{accumulate}と同じ引数を追加の1引数の述語と共に取り、述語はフィルタを指定する。
手続としての\code{filtered\-/accumulate}を書け。以下を\code{filtered\-/accumulate}を用いて
どのように表現するかを示せ。

\begin{enumerate}[a]

\item
\( a \)と\( b \)の区間の素数の二乗の和(あなたは既に\code{prime?}述語を書いていると前提する)

\item
全ての\( n \)未満の正の整数でかつ\( n \)に対して互いに素(つまり
\( {\rm GCD}(i, n) = 1 \)となる全ての整数\( i < n \))の積

\end{enumerate}
\end{quote}


\subsection{\code{lambda}を用いた手続の構築}
\label{Section 1.3.2}



\link{Section 1.3.1}に示すよう\code{sum}を使用する時、\code{pi\-/term}や\code{pi\-/next}のような自明な
手続を高階手続にて引数として使うためだけに定義せねばならないのはひどく不恰好に見えます。
\code{pi\-/next}や\code{pi\-/term}を定義する代わりに、``入力値に4を足す手続''や``入力値と入力値に
2を足した数の積の逆数を返す手続''を直接指定する方法を持つほうがより便利になるでしょう。
これは手続を作成する特殊形式\code{lambda}を紹介することで可能です。
\code{lambda}を用いることで先程行いたかったことを以下のように記述できます。

\begin{scheme}
(lambda (x) (+ x 4))
\end{scheme}

\noindent
そして

\begin{scheme}
(lambda (x) (/ 1.0 (* x (+ x 2))))
\end{scheme}

\noindent
次に\code{pi\-/sum}手続は補助的な手続を定義すること無しに表現が可能となります。

\begin{scheme}
(define (pi-sum a b)
  (sum (lambda (x) (/ 1.0 (* x (+ x 2))))
       a
       (lambda (x) (+ x 4))
       b))
\end{scheme}

\noindent
同様に\code{lambda}を使用して、\code{integral}手続を補助的な手続\code{add\-/dx}を定義することなく
書くことが可能です。

\begin{scheme}
(define (integral f a b dx)
  (* (sum f
          (+ a (/ dx 2.0))
          (lambda (x) (+ x dx))
          b)
     dx))
\end{scheme}

\noindent
一般的に、\code{lambda}は\code{define}と同様に手続を作成しますが、手続に対して
名前が指定されないことが異なります。

\begin{scheme}
(lambda (~\( \dark \langle \)~~\var{\dark formal-parameters}~~\( \dark \rangle \)~) ~\( \dark \langle \)~~\var{\dark body}~~\( \dark \rangle \)~)
\end{scheme}

\noindent
結果としての手続は\code{define}を用いて作成した手続と同じです。ただ1つの違いは
それが環境においてどのような名前にも結び付けられていないことです。

\begin{scheme}
(define (plus4 x) (+ x 4))
\end{scheme}

\noindent
上記は以下と等価です。

\begin{scheme}
(define plus4 (lambda (x) (+ x 4)))
\end{scheme}

\noindent
\code{lambda}式は以下のように読むことができます。

\begin{scheme}
(lambda                     (x)     (+   x     4))
    |                        |       |   |     |
手続は                    引数xを持ち 足す x と  4
\end{scheme}

\noindent
任意の値として手続を持つ式と同様に、\code{lambda}式は複合式において
オペレータとして使用することが可能です。例えば、

\begin{scheme}
((lambda (x y z) (+ x y (square z)))
 1 2 3)
~\textit{12}~
\end{scheme}

\noindent
またはより一般的に、私達が通常手続の名前を使用する任意の文脈において使用可能です。
\footnote{Lispを学ぶ人にとっては\code{lambda}という名前よりは\code{make\-/procedure}の様な
名前を使用したほうがより判りやすいか、または恐しく思わせたりはしないでしょう。
しかしこの慣習はしっかりと根付いた物です。この表記は\( \lambda \)-calculus(ラムダ計算)
という数理論理学者Alonzo \link{Church (1941)}により発表された数学上の形式主義の
名から受け入れられています。Churchは\( \lambda \)計算を関数と関数適用の概念を学ぶための
厳格な基礎として与えるために開発しました。 \( \lambda \)計算はプログラミング言語の
意味の数学上の研究のための基礎的なツールとなりました。}

\subsubsection*{ローカル変数使用のため\code{let}を用いる}



別の\code{lambda}使用法にはローカル変数の作成があります。形式的パラメータに
束縛されていないローカル変数を手続で必要とする場合は良くあります。
例えば以下の関数を演算したいとします。
\begin{comment}

\begin{example}
f(x,y) = x(1 + xy)^2 + y(1 - y) + (1 + xy)(1 - y)
\end{example}

\end{comment}
\begin{displaymath}
 f(x,y) = x(1 + xy)^2 + y(1 - y) + (1 + xy)(1 - y), 
\end{displaymath}
\noindent
これは以下のようにも表現できます。
\begin{comment}

\begin{example}
     a = 1 + xy
     b = 1 - y
f(x,y) = xa^2 + yb + ab
\end{example}

\end{comment}
\begin{displaymath}
% \eqalign{	a 	&= 1 + xy, \cr
% 		b 	&= 1 - y,  \cr
% 		f(x,y) 	&= xa^2 + yb + ab. \cr}
\begin{array}{r@{{}={}}l}
  a 	  &  1 + xy, \\
  b 	  &  1 - y,  \\
  f(x,y)  &  xa^2 + yb + ab.
\end{array}
\end{displaymath}


\( f \)を求める手続を書く場合、\( x \)と\( y \)のみでなく中間値の名前として
\( a \)や\( b \)をローカル変数として含みたくなるでしょう。これを実現する1つの
方法として補助的な手続をローカル変数を束縛するため使用することがあります。

\begin{scheme}
(define (f x y)
  (define (f-helper a b)
    (+ (* x (square a))
       (* y b)
       (* a b)))
  (f-helper (+ 1 (* x y))
            (- 1 y)))
\end{scheme}

\noindent
もちろん、\code{lambda}式を用いて無名手続をローカル変数の束縛のため指定することも
可能です。\code{f}のボディはすると手続への単一の呼び出しになります。

\begin{scheme}
(define (f x y)
  ((lambda (a b)
     (+ (* x (square a))
        (* y b)
        (* a b)))
   (+ 1 (* x y))
   (- 1 y)))
\end{scheme}

\noindent
この構成はとても便利で\code{let}と呼ばれる特殊形式がその使用をより便利にするために
用意されています。\code{let}を用いることで手続\code{f}は以下のようになります。

\begin{scheme}
(define (f x y)
  (let ((a (+ 1 (* x y)))
        (b (- 1 y)))
    (+ (* x (square a))
       (* y b)
       (* a b))))
\end{scheme}

\noindent
\code{let}式の一般的な形式は次のとおりです。

\begin{scheme}
(let ((~\( \dark \langle \)~~\( \dark var_1 \)~~\( \dark \rangle \)~ ~\( \dark \langle \)~~\( \dark exp_1 \)~~\( \dark \rangle \)~)
      (~\( \dark \langle \)~~\( \dark var_2 \)~~\( \dark \rangle \)~ ~\( \dark \langle \)~~\( \dark exp_2 \)~~\( \dark \rangle \)~)
      ~\( \dots \)~
      (~\( \dark \langle \)~~\( \dark var_n \)~~\( \dark \rangle \)~ ~\( \dark \langle \)~~\( \dark exp_n \)~~\( \dark \rangle \)~))
   ~\( \dark \langle \)~~\var{\dark body}~~\( \dark \rangle \)~)
\end{scheme}

\noindent
これは以下のように考えることが可能です。

\begin{scheme}
let ~\( \dark \langle \)~~\( \dark var_1 \)~~\( \dark \rangle \)~ have the value ~\( \dark \langle \)~~\( \dark exp_1 \)~~\( \dark \rangle \)~ and
    ~\( \dark \langle \)~~\( \dark var_2 \)~~\( \dark \rangle \)~ have the value ~\( \dark \langle \)~~\( \dark exp_2 \)~~\( \dark \rangle \)~ and
    ~\( \dots \)~
    ~\( \dark \langle \)~~\( \dark var_n \)~~\( \dark \rangle \)~ have the value ~\( \dark \langle \)~~\( \dark exp_n \)~~\( \dark \rangle \)~
in  ~\( \dark \langle \)~~\var{\dark body}~~\( \dark \rangle \)~
\end{scheme}

(letは使役の意ですので、\( \langle \)\var{body}\( \rangle \)の中では\( \langle \)\( var_1 \)\( \rangle \)は
\( \langle \)\( exp_1 \)\( \rangle \)の値を持たせる、以下繰り返しと読めます。)

\noindent
\code{let}式の最初の部分は名前と式のペアのリストです。\code{let}が評価される時、
各名前は関連する式の値と関連付けされます。\code{let}のボディはこれらのローカルな値に束縛された
名前と共に評価されます。\code{let}式は以下の代替文法として評価されるためこれが起こります。

\begin{scheme}
((lambda (~\( \dark \langle \)~~\( \dark var_1 \)~~\( \dark \rangle \)~ ~\( \dots \)~ ~\( \dark \langle \)~~\( \dark var_n \)~~\( \dark \rangle \)~)
    ~\( \dark \langle \)~~\var{\dark body}~~\( \dark \rangle \)~)
 ~\( \dark \langle \)~~\( \dark exp_1 \)~~\( \dark \rangle \)~
 ~\( \dots \)~
 ~\( \dark \langle \)~~\( \dark exp_n \)~~\( \dark \rangle \)~)
\end{scheme}

\noindent
インタプリタ内にはローカル変数を提供するために新しいメカニズムが必要とされません。
\code{let}式は中で行われる\code{lambda}適用に対する糖衣構文でしかありません。



この等価式から\code{let}式にて指定された変数のスコープが\code{let}のボディであることが
わかります。これが以下のことを暗示します。

\begin{itemize}

\item
\code{let}は変数を可能な限り使用される場所に局地的に束縛します。例えばもし\code{x}の値が
5である時、次の式の値は

\begin{scheme}
(+ (let ((x 3))
     (+ x (* x 10)))
   x)
\end{scheme}

\noindent
38です。ここで\code{let}のボディの中の\code{x}は3ですので\code{let}式の値は33です。
一方で最も外側の\code{+}の第二引数である\code{x}は依然5です。

\item
変数の値は\code{let}の外側にて計算されます。これはローカル変数の値を提供する式が
ローカル変数自身と同じ名前を持っている変数に依存する場合に問題となります。
例えば、もし\code{x}の値が2の時、次の式では

\begin{scheme}
(let ((x 3)
      (y (+ x 2)))
  (* x y))
\end{scheme}

\noindent
式の値は12になります。\code{let}のボディ内部では\code{x}は3、\code{y}は4(外側の\code{x}足す2)に
なるためです。

\end{itemize}

\noindent
時には\code{let}と同様の効果を得るために内部定義を利用することもあります。
例として、上記の手続\code{f}を次のように定義することも可能でした。

\begin{scheme}
(define (f x y)
  (define a (+ 1 (* x y)))
  (define b (- 1 y))
  (+ (* x (square a))
     (* y b)
     (* a b)))
\end{scheme}

\noindent
しかしこのような状況では\code{let}を使用し、内部定義は内部手続のみのために利用することを
好みます。\footnote{内部定義を十分に良く理解し、プログラムが私達がそれに意図した
意味を意味することを確実にするには私達がこの章で紹介したよりもより複雑な評価過程のモデル
を必要とします。しかし手続の内部定義と共にはその機微は浮かび上がりません。
この問題については\link{Section 4.1.6}にて評価についてより学んだ後に立ち戻ります。}

\begin{quote}
\heading{\phantomsection\label{Exercise 1.34}Exercise 1.34:} 以下の手続を定義したとする。

\begin{scheme}
(define (f g) (g 2))
\end{scheme}

すると以下の結果を得る。

\begin{scheme}
(f square)
~\textit{4}~
(f (lambda (z) (* z (+ z 1))))
~\textit{6}~
\end{scheme}


もし(天邪鬼にも)インタプリタに\code{(f f)}の組み合わせを評価させたらどのような結果が起こるか?
説明せよ。
\end{quote}


\subsection{汎用手法としての手続}
\label{Section 1.3.3}


\link{Section 1.1.4}にて複合手続を数値演算の抽象化パターンのメカニズムとして紹介し、
関係する特定の数値から独立させました。\link{Section 1.3.1}の\code{integral}手続のような
高階手続ではより強力な種類の抽象化について学び始めました。
関係する特定の関数から独立した汎用的演算手法を表現するのに利用される手続でした。
この節では2つのより複雑な例---零と関数の不動点を見付けるための汎用手法---について議論します。
そしてこれらの手法がどのように手続として直接的に表現されるのかを示します。

\subsubsection*{半区間手法により方程式の根を求める}



\newterm{half-interval method}(\jnewterm{半区間手法})は方程式\( f(x) = 0 \)の根を求めるのに
単純ながら強力なテクニックです。ここで\( f \)は連続関数とします。
この考えは\( f(a) < 0 < f(b) \)となる点\( a \)と\( b \)を与えた時、\( f \)は最低でも1つの0を
\( a \)と\( b \)の間に持つことになります。ゼロを特定するために\( a \)と\( b \)の平均\( x \)を
求め\( f(x) \)を計算します。もし\( f(x) > 0 \)なら\( f \)は0を\( a \)と\( x \)の間に持ちます。
もし\( f(x) < 0 \)なら\( f \)は0を\( x \)と\( b \)の間に持ちます。このように繰り返すことで
\( f \)が0を持つより小さな区間を特定できます。区間が十分に小さな時点に辿りついたら処理は停止します。
不確かな区間が処理の各ステップにて半分になるため、必要とされるステップ数は\( \Theta(\log(L / T)) \)
に従い増加します。このとき\( L \)は元の区間の長さで\( T \)は許容誤差(私達が``十分に小さい''と
考える区間のサイズ)になります。この戦略を実装した手続が以下になります。

\begin{scheme}
(define (search f neg-point pos-point)
  (let ((midpoint (average neg-point pos-point)))
    (if (close-enough? neg-point pos-point)
        midpoint
        (let ((test-value (f midpoint)))
          (cond ((positive? test-value)
                 (search f neg-point midpoint))
                ((negative? test-value)
                 (search f midpoint pos-point))
                (else midpoint))))))
\end{scheme}

\noindent
最初に関数\( f \)を値が負と正になる2つの点と共に与えられると想定します。
最初に2つの与えられた点の中間点を求めます。次に与えられた区間が十分に小さいかチェックし、
もしそうであれば単純に中間点を答とします。そうでなければ中間点における\( f \)の
値をtest-value(試験値)として計算します。もし試験値が正ならば、元の負の地点から中間点までの
新しい区間にて処理を続けます。もし試験値が負ならば中間値から正の地点までの区間にて
処理を続けます。最終的に試験値が0になる可能性がありますが、その場合、中間地点そのものが
我々が探している根となります。


終了地点が``十分に近い''か試験するためには\link{Section 1.1.7}にて平方根を求めるために
利用した物と同様の手続が利用可能です。\footnote{私達は``小さな''値の表現として
0.001を用い、計算にて受け入れられる誤差の許容範囲を示しました。実際の演算に
おける適切な許容範囲は解決すべき問題、計算機とアルゴリズムの制約に依存します。
これはしばしばとても微妙な考慮事項であり、数値解析者や他の魔法使いのような人達の
助けを必要とします。}

\begin{scheme}
(define (close-enough? x y) (< (abs (- x y)) 0.001))
\end{scheme}

\noindent
\code{search}は直接利用するのは扱いにくいです。\( f \)の値が必要な符号を持たない点を
意図せず与えてしまうことが可能なためです。そのような場合では間違った答を得てしまいます。
代わりに\code{search}を次の手続を経由して使用することにしましょう。これは終端のどちらが負の
関数値を持ち、どちらが正の関数値を持つか検査します。そして\code{search}手続を適切に
呼び出します。もし関数が2つの与えられた点にて同じ符号を持つ場合、半区間手法は使用
できません。この場合この手続はエラーを伝えます。\footnote{これは\code{error}を用いて達成でき
ます。\code{error}は引数としていくつかの項目を受け取りそれらをエラーメッセージとして出力します。}

\begin{scheme}
(define (half-interval-method f a b)
  (let ((a-value (f a))
        (b-value (f b)))
    (cond ((and (negative? a-value) (positive? b-value))
           (search f a b))
          ((and (negative? b-value) (positive? a-value))
           (search f b a))
          (else (error "Values are not of 
                        opposite sign" a b)))))
\end{scheme}

\noindent
次の例は半区間手法を使用して\( \pi \)の近似を\( \sin x = 0 \)の2と4の間の根として
求めています。

\begin{scheme}
(half-interval-method sin 2.0 4.0)
~\textit{3.14111328125}~
\end{scheme}

\noindent
また次の別の例では半区間手法を用いて方程式\( x^3 - 2x - 3 = 0 \)において1と2の間で
根を探しています。

\begin{scheme}
(half-interval-method (lambda (x) (- (* x x x) (* 2 x) 3))
                      1.0
                      2.0)
~\textit{1.89306640625}~
\end{scheme}

\subsubsection*{関数の不動点を求める}


数値\( x \)は\( x \)が等式\( f(x) = x \)を満たす時、関数\( f \)の\newterm{fixed point}(\jnewterm{不動点})と呼ばれます。
いくつかの関数\( f \)に対し不動点を初期推測値から始めて\( f \)を値があまり変わらなくなるまで
繰り返し適用することで求めることができます。
\begin{comment}

\begin{example}
f(x), f(f(x)), f(f(f(x))), ...
\end{example}

\end{comment}
\begin{displaymath}
 f(x),\quad f(f(x)),\quad f(f(f(x))), \quad\dots, 
\end{displaymath}
\noindent
この考えを用いて関数と初期推定値を入力とし、関数の不動点への近似を生成する手続、
\code{fixed\-/point}を開発できます。指示した許容範囲未満の差に二点が収まるまで関数を
繰り返し適用します。

\begin{scheme}
(define tolerance 0.00001)
(define (fixed-point f first-guess)
  (define (close-enough? v1 v2)
    (< (abs (- v1 v2)) 
       tolerance))
  (define (try guess)
    (let ((next (f guess)))
      (if (close-enough? guess next)
          next
          (try next))))
  (try first-guess))
\end{scheme}

\noindent
例えばこの手法をコサイン関数の不動点を近似するのに利用できます。初期推測値は1
とします。\footnote{以下を暇な授業の間に実行してみて下さい:電卓をラジアンモードに設定し
不動点に到達するまで\code{cos}を連打してみましょう。}

\begin{scheme}
(fixed-point cos 1.0)
~\textit{.7390822985224023}~
\end{scheme}

\noindent
同様にして、次の方程式の答を見つけられます。
\( y = \sin y + \cos y \):

\begin{scheme}
(fixed-point (lambda (y) (+ (sin y) (cos y))) 1.0)
~\textit{1.2587315962971173}~
\end{scheme}

\noindent
不動点処理は\link{Section 1.1.7}にて平方根を求めるのに使用した処理を思い出させます。
両者は共に結果がある判定基準を満たすまで推測値を繰り返し改善する考えを基にしています。
実際に直ぐに平方根の計算を不動点検索として形式化が可能です。
ある数値\( x \)の平方根を求めるには\( y^2 = x \)を満たす\( y \)を探す必要があります。
この等式を等価な形の\( y = x / y \)
\footnote{\( \mapsto \) (``maps to''(写す)と読みます。)は数学者による\code{lambda}の記述法です。
\( y \mapsto x / y \)は\code{(lambda (y) (/ x y))}を意味し、\( y \)における関数の値は
\( x / y \)ということです。}
にするとその関数\( y \mapsto x / y \)の不動点を探してることに気付きます。従って平方根を以下のように
求めることを試すことができます。

\begin{scheme}
(define (sqrt x)
  (fixed-point (lambda (y) (/ x y)) 1.0))
\end{scheme}

\noindent
残念ながらこの不動点検索は収束しません。初期推測値を\( y_1 \)とします。次の推測値は
\( y_2 = x / y_1 \)でさらに次は\( y_3 = x / y_2 = x / (x / y_1) = y_1 \)
です。この結果は2つの推測値\( y_1 \)と\( y_2 \)がずっと繰り返し、答が振動する無限ループです。



そのような振動をコントロールする1つの方法は推測値が大きく変化することを防ぐことです。
回答は常に推測値\( y \)と\( x / y \)の間にあるはずですから\( y \)と
\( x / y \)の両方から同じ位遠くはない地点にできるはずです。
従って\( y \)と\( x / y \)の平均を取って\( y \)の次の推測値は
\( {1\over2}(y + x / y) \)となります。

\begin{scheme}
(define (sqrt x)
  (fixed-point (lambda (y) (average y (/ x y))) 1.0))
\end{scheme}

\noindent
(\( y = {1\over2}(y + x / y) \)は等式\( y = x / y \)を単純に変形したものである
ことに注意して下さい。得るためには等式の両辺に\( y \)を足し、2で割ります。)


この変更により平方根手続がうまく行きます。実際に、もし定義をひも解いた場合、ここで
生成された平方根の近似の連続は元々の\link{Section 1.1.7}の平方根手続が生成するものと正確に
同じです。この一連の近似値の平均から回答へのアプローチは、\newterm{average damping}(\jnewterm{平均減衰})
と呼ぶテクニックであり、良く不動点検索の収束に対し手助けとなります。

\begin{quote}
\heading{\phantomsection\label{Exercise 1.35}Exercise 1.35:} 黄金比率
\( \varphi \)(\link{Section 1.2.2})は変形\( x \mapsto 1 + 1 / x \)の不動点である
ことを示せ。この比率を用いて\( \varphi \)を\code{fixed\-/point}手続を用いて求めよ。
\end{quote}

\begin{quote}
\heading{\phantomsection\label{Exercise 1.36}Exercise 1.36:} \code{fixed\-/point}を変更し、
\link{Exercise 1.22}にて示された\code{newline}
と\code{display}プリミティブを用いて生成する一連の近似値を表示するようにせよ。
次に\( x^x = 1000 \)の答を\( x \mapsto \log(1000) / \log(x) \)の不動点を
求める方法で求めよ。
(Schemeのプリミティブである\code{log}手続を利用せよ。これは自然対数を計算する)。
平均減衰を用いる場合と用いない場合にてステップ数を比較せよ。(\code{fixed\-/point}を
推定値1では開始できないことに注意せよ。これは\( \log(1) = 0 \)での割り算を
引き起すためである。)
\end{quote}

\begin{quote}
\heading{\phantomsection\label{Exercise 1.37}Exercise 1.37:} \begin{enumerate}[a]

\item
無限\newterm{continued fraction}(\jnewterm{連分数})とは以下の形式の式である。
\begin{comment}

\begin{example}
           N_1
f = ---------------------
               N_2
    D_1 + ---------------
                   N_3
          D_2 + ---------
                D_3 + ...
\end{example}

\end{comment}
\begin{displaymath}
 {f} = {N_1 \over {D_1 + {N_2 \over {D_2 + {N_3 \over D_3 + \dots}}}}}\,. 
\end{displaymath}


例として、無限連分数の展開として\( N_i \)と\( D_i \)の全てが1の場合、\( 1 / \varphi \)
を生成し、この時\( \varphi \)は(\link{Section 1.2.2}で説明した)黄金比率である。
無限連分数の近似を求める1つの方法として与えられた項の数を越えた後、展開を切り捨てる方法がある。
そのような切り捨て---所謂\newterm{\textit{k}-term finite continued fraction}(\jnewterm{k項有限連分数})---は以下
の形式になる。
\begin{comment}

\begin{example}
       N_1
-----------------
          N_2
D_1 + -----------
      ...    N_K
          + -----
             D_K
\end{example}

\end{comment}
\begin{displaymath}
 {N_1 \over {D_1 + {N_2 \over {\ddots + {N_k \over D_k}}}}}\,. 
\end{displaymath}


\code{n}と\code{d}は連分数の\( N_i \)\( D_i \)の項を返す1引数(項の索引\( i \))の手続であると考える。
\code{(cont\-/frac n d k)}を評価すると\( k \)項有限連分数の値を求める手続、\code{cont\-/frac}を定義せよ。
あなたの手続を以下を用いて\( 1 / \varphi \)の近似を求めることで、一連の\code{k}の値についてチェックせよ。

\begin{scheme}
(cont-frac (lambda (i) 1.0)
           (lambda (i) 1.0)
           k)
\end{scheme}

\noindent
小数点以下4桁の精度の近似を得るには\code{k}はどれだけの大きさでなければならないか?

\item
もしあなたの\code{cont\-/frac}手続が再帰プロセスを生成するのならば線形プロセスを生成するものを書け。
もし線形プロセスを生成するのであれば、再帰プロセスを生成するものを書け。

\end{enumerate}
\end{quote}

\begin{quote}
\heading{\phantomsection\label{Exercise 1.38}Exercise 1.38:} 1737年にスイスの数学者、
Leonhard Euler(レオンハルト・オイラー)は学術論文\textit{De Fractionibus Continuis}を出版した。
それには\( e \)が自然対数の底である時の、\( e - 2 \)に対する連分数展開が含まれている。
この分数では\( N_i \)は全て1であり、\( D_i \)は数列1, 2, 1, 1, 4, 1, 1, 6, 1, 1, 8, \( \dots \)である。
\link{Exercise 1.37}のあなたの\code{cont\-/frac}手続を用いてオイラー展開を基にし\( e \)の近似を求める
プログラムを書け。
\end{quote}

\begin{quote}
\heading{\phantomsection\label{Exercise 1.39}Exercise 1.39:} タンジェント(正接)
関数の連分数表現は1770年にドイツの数学者J.H. Lambert
(ヨハン・ハインリッヒ・ランベルト)により発表された。
\begin{comment}

\begin{example}
              x
tan x = ---------------
                x^2
        1 - -----------
                  x^2
            3 - -------
                5 - ...
\end{example}

\end{comment}
\begin{displaymath}
 {\tan x} = {x \over {1 - {x^2 \over {3 - {x^2 \over 5 - \dots}}}}}\,, 
\end{displaymath}
\noindent
ここで\( x \)はラジアンである。ランベルトの式を基にして正接関数の近似値を求める
手続\code{(tan\-/cf x k)}を定義せよ。\code{k}は\link{Exercise 1.37}と同様に
求める項の数を指定する。
\end{quote}

\label{Section 1.3.4}
\subsection{返り値としての手続}



ここまでの一連の例は手続を引数として渡す能力が著しく私達のプログラミング言語の表現力を
拡張することを実演しました。返り値自体が手続である手続を作成することでさらに表現力を
獲得することができます。


この考えを\link{Section 1.3.3}の終わりにて説明された不動点の例を振り返ることで説明できます。
\code{square\-/root}手続の新しいバージョンを、\( \sqrt{x} \)は\( y \mapsto x / y \)関数の
不動点であるという観察結果から始めて、不動点検索の形で形式化しました。
次に平均減衰を用いて近似値を収束させました。平均減衰はそれ自体が便利な汎用技法です。
即ち関数\( f \)を与えられた時、\( x \)における関数の値が\( x \)と\( f(x) \)の平均だと
考えます。


平均減衰の考えを次の手続を用いて説明できます。

\begin{scheme}
(define (average-damp f) (lambda (x) (average x (f x))))
\end{scheme}

\noindent
\code{average\-/damp}は引数として手続\code{f}を取りその値として(\code{lambda}で生成された)手続を返します。
その手続は数値\code{x}に適用された時、\code{x}と\code{(f x)}の平均を返します。例えば\code{average\-/damp}を
\code{square}手続に適用した時、生成された手続の値は、ある値\( x \)において\( x \)と\( x^2 \)の
平均となります。この結果の手続に10を適用すれば10と100の平均として55を返します。\footnote{これは
組み合わせであり、かつそのオペレータもまた合成式であることに注意して下さい。\link{Exercise 1.4}にて
既にそのような合成式を形式化する能力については実演しました。しかしあれは単に簡単な例にすぎません。
ここではそのような合成式に対する真の要求---高階手続により値として返されることで得られた手続を
いつ適用するのかについて学び始めます。}

\begin{scheme}
((average-damp square) 10)
~\textit{55}~
\end{scheme}

\noindent
\code{average\-/damp}を用いて\code{square\-/root}手続を次のように再公式化できます。

\begin{scheme}
(define (sqrt x)
  (fixed-point (average-damp (lambda (y) (/ x y))) 1.0))
\end{scheme}

\noindent
この形式化が手法内においてどれだけ3つの考え、不動点検索、平均減衰、
関数\( y \mapsto x / y \)について明快にしているかに注目して下さい。
このsquare-rootの手法の形式化と\link{Section 1.1.7}で与えた元のバージョンの比較は
示唆的です。これらの手続が同じ処理について表現していることを心に留めて下さい。
そして同じ処理をこれらの抽象化を用いて表現した時にどれだけ明白になるのか
に注目して下さい。一般的に処理を手続に形式化する手法はとても多くの数有ります。
経験の豊富なプログラマはどのように手続形式化を選ぶのか、特に明快な方法を知っています。
そしてどこで処理の便利な要素が他のアプリケーションにて再使用可能な独立した要素として
浮かび上がるかについて知っているのです。再使用の簡単な例として\( x \)の立方根は
関数\( y \mapsto x / y^2 \)の不動点であることに注意して下さい。従って
直ぐにsquare-root手続を立方根を求める手続に汎化することが可能です。\footnote{さらなる
一般化については\link{Exercise 1.45}を参照して下さい。}


\begin{scheme}
(define (cube-root x)
  (fixed-point (average-damp (lambda (y) (/ x (square y))))
               1.0))
\end{scheme}

\subsubsection*{ニュートン法}



\link{Section 1.1.7}にて初めてsquare-root手続を紹介した時に、これは\newterm{Newton's method}(\jnewterm{ニュートン法})
の特別な場合であると伝えました。もし\( x \mapsto g(x) \)が微分可能な関数である時、方程式
\( g(x) = 0 \)の答は以下の条件で関数\( x \mapsto f(x) \)の不動点となります。
\begin{comment}

\begin{example}
           g(x)
f(x) = x - -----
           Dg(x)
\end{example}

\end{comment}
\begin{displaymath}
 {f(x) = x} - {g(x)\over Dg(x)} 
\end{displaymath}
\noindent
そして\( Dg(x) \)は\( x \)により微分した導関数です。ニュートン法は上で学んだ
不動点を用いる手法で、関数\( f \)の不動点を探すことで方程式の解の近似を求めます。
\footnote{初歩的な微積分学の教科書は通常ニュートン法を近似の数列\( x_{n+1} = x_n -
g(x_n) / Dg(x_n) \)を用いて説明しています。処理に関する言語を持ち不動点の
考えを用いることで手法の説明を平易にできます。}



多くの関数\( g \)において、また十分に良い初期推測値\( x \)においてニュートン法は
\( g(x) = 0 \)の解に急速に収束します。\footnote{ニュートン法は常に解へと収束は
しません。しかし好ましい場合においては各繰り返しにおいて解の近似値の精度の桁数は
二倍になることが示されます。そのような場合ではニュートン法は半区間手法よりも大変
速く収束します。}



ニュートン法を手続として実装するために、最初に微分の考えを表現せねばなりません。
``微分''は平均減衰と同様にある関数を別の関数へと変形することに注目して下さい。
例えば関数\( x \mapsto x^3 \)の微分は\( x \mapsto 3x^2 \)です。
一般的に\( g \)が関数であり\( dx \)が小さな値である時、\( g \)の導関数
\( Dg \)はその値が任意の数\( x \)が与えられた時に(小さな値\( dx \)の極限において)
\begin{comment}

\begin{example}
        g(x + dx) - g(x)
Dg(x) = ----------------
               dx
\end{example}

\end{comment}
\begin{displaymath}
 {Dg(x)} = {g(x + {\it dx}) - g(x) \over {\it dx}}\,. 
\end{displaymath}
\noindent
従って微分の考えを(\( dx \)を例えば0.00001として)手続として次のように
表現できます。

\begin{scheme}
(define (deriv g)
  (lambda (x) (/ (- (g (+ x dx)) (g x)) dx)))
\end{scheme}

\noindent
次の定義と一緒に用います。

\begin{scheme}
(define dx 0.00001)
\end{scheme}

\noindent
\code{average\-/damp}と同様に、\code{deriv}は引数として手続を取り、値として手続を返す手続です。
例えば導関数\( x \mapsto x^3 \)の5における値(正確な値は75です)の近似を求めるために
以下のように評価できます。

\begin{scheme}
(define (cube x) (* x x x))
((deriv cube) 5)
~\textit{75.00014999664018}~
\end{scheme}

\noindent
\code{deriv}の助けを借りて、ニュートン法を不動点処理として次のように表現できます。

\begin{scheme}
(define (newton-transform g)
  (lambda (x) (- x (/ (g x) ((deriv g) x)))))
(define (newtons-method g guess)
  (fixed-point (newton-transform g) guess))
\end{scheme}

\noindent
\code{newton\-/transform}手続はこの節の最初の式を表現しています。そして\code{newtons\-/method}
が直ぐにそれを用いて定義されています。これは初期推測値と一緒に手続を引数として取り
その手続はゼロを見つけたい関数を計算します。例えば、\( x \)の平方根を見つけたい時、
ニュートン法を用いて関数\( y \mapsto y^2 - x \)のゼロを初期推測値1から始めて探す
ことが可能です。\footnote{平方根を探す場合、ニュートン法は任意の開始値から急速に
正しい解に収束します。}


これが平方根手続の別の形を与えます。

\begin{scheme}
(define (sqrt x)
  (newtons-method
   (lambda (y) (- (square y) x)) 1.0))
\end{scheme}

\subsubsection*{抽象化と一級手続}



より一般的な手法の事例として、平方根の演算の表現方法を2つ見てきました。1つは
不動点検索で、もう1つはニュートン法です。ニュートン法はそれ自体が不動点処理として
表現されているため、実際には平方根を不動点として計算する2つの方法を見た訳になります。
各手法は関数と共に開始し、その関数のある変形の不動点を探します。この一般的な考えそれ
自身を手続として表現できます。

\begin{scheme}
(define (fixed-point-of-transform g transform guess)
  (fixed-point (transform g) guess))
\end{scheme}

\noindent
このとても汎用的な手続は引数としてある関数を計算する手続\code{g}、\code{g}を変形する
手続、初期推測値を取ります。結果としての返り値は変形された関数の不動点です。



この抽象化を用いて、この節最初の(平均減衰バージョンの\( y \mapsto x / y \)の
不動点を探した)平方根演算をこの汎用手法の例として変更できます。

\begin{scheme}
(define (sqrt x)
  (fixed-point-of-transform
   (lambda (y) (/ x y)) average-damp 1.0))
\end{scheme}

\noindent
同様にこの節の2つ目の平方根演算(\( y \mapsto y^2 - x \)のニュートン変形の不動点を
探すニュートン法の例)を以下の様に表現できます。

\begin{scheme}
(define (sqrt x)
  (fixed-point-of-transform
   (lambda (y) (- (square y) x)) newton-transform 1.0))
\end{scheme}

\noindent
\link{Section 1.3}では複合手続は重大な抽象化メカニズムであるという考えから始めました。
私達のプログラミング言語において演算の一般的な手法を明示的な要素として表現することを
可能にするためです。ここでは高階手続がどのようにこれらの一般的手法を操作してさらなる
抽象化を作成することを可能にするのかについて学びます。


プログラマとして、私達のプログラムに内在する抽象化を判別する機会を迅速に学ばねば
なりません。そしてその上に構築し、それらを汎化してより強力な抽象化を作成する術を
学ばねばなりません。これは常にプログラムを可能な限り抽象化して書かねばならないと
言う訳ではありません。エキスパートプログラマは彼等のタスクにとって適切な抽象化
レベルの選択方法を知っています。しかし、これらの抽象化を用いて考えられるように
なることが重要です。そうすればそれらを新しいコンテキストでも適用することに準備する
ことができます。高階手続の有用性は、それらがここまでの抽象化を私達の
プログラミング言語の要素として明示的に表現することを可能にしてくれることです。
そうすることで抽象化は他の計算上の要素と同様に扱われることが可能となります。



通常、プログラミング言語は計算要素が取扱可能になるような方法に制約を課します。
制約が最も少ない要素は\newterm{first-class}(\jnewterm{第一級})の地位にあると言われます。
第一級要素の``権利と特権''のいくつかを次に示します。
\footnote{プログラミング言語の要素の第一級の地位の概念はイギリスの計算機科学者
Christopher Strachey (1916-1975)によるものです。}

\begin{itemize}

\item
変数により名付けることが可能

\item
手続に対し引数として渡すことが可能

\item
手続の結果として返すことが可能

\item
データ構造に含まれることが可能\footnote{
これの例は\link{Chapter 2}にてデータ構造を紹介した後に学びます。}

\end{itemize}

\noindent
Lispは他の言語と異なり、手続に完全な第一級の地位を与えます。このことが効率の良い
実装に対して課題を課しますが、結果的に表現力に得る物は莫大な物となります。\footnote{
第一級手続の主な実装コストは、手続が値として返ることを可能とするために、手続の
自由変数に対して予備の領域を、例え手続が実行中でなくとも必要とします。私達が
\link{Section 4.1}にて学ぶSchemeの実装では、これらの変数は手続の環境に保存されます。}

\begin{quote}
\heading{\phantomsection\label{Exercise 1.40}Exercise 1.40:} \code{newtons\-/method}手続と
共に以下の形式の式にて使用が可能な手続\code{cubic}を
定義せよ。

\begin{scheme}
(newtons-method (cubic a b c) 1)
\end{scheme}

\noindent
次に三次方程式\( x^3 + ax^2 + bx + c = 0 \)の近似解を求めよ。

\end{quote}

\begin{quote}
\heading{\phantomsection\label{Exercise 1.41}Exercise 1.41:} 引数が1つの手続を
引数として取り、その手続を二回適用する手続を返す手続\code{double}を定義せよ。
例えば\code{inc}が引数に1を足す手続であれば、\code{(double inc)}は2を足す手続になる。
次の式はどんな値を返すか?

\begin{scheme}
(((double (double double)) inc) 5)
\end{scheme}
\end{quote}

\begin{quote}
\heading{\phantomsection\label{Exercise 1.42}Exercise 1.42:} \( f \)と\( g \)が
2つの1引数関数だとする。\( g \)に\( f \)を\newterm{composition}(\jnewterm{合成})するとは
関数\( x \mapsto f(g(x)) \)と定義される。合成を実装する手続\code{compose}を定義せよ。
例えば\code{inc}が引数に1を足す手続である場合、

\begin{scheme}
((compose square inc) 6)
~\textit{49}~
\end{scheme}
\end{quote}

\begin{quote}
\heading{\phantomsection\label{Exercise 1.43}Exercise 1.43:} \( f \)が
数値演算関数であり\( n \)が正の整数である時、\( f \)を\( n \)回適用する、
\( x \)における値が\( f(f(\dots (f(x))\dots )) \)である関数を定義できる。
例えば\( f \)が関数\( x \mapsto x + 1 \)である時、\( f \)を\( n \)回適用した関数は
\( x \mapsto x + n \)となる。もし\( f \)が数値を二乗する操作ならば、\( f \)を
\( n \)回適用した関数は引数を\( 2^n \)乗する。入力として\( f \)を計算する手続と正の整数
\( n \)を取り、\( f \)の\( n \)回適用を計算する手続を返す手続を書け。
その手続は以下のように使用可能でなければならない。

\begin{scheme}
((repeated square 2) 5)
~\textit{625}~
\end{scheme}

ヒント:\link{Exercise 1.42}の\code{compose}を使うと便利でしょう。
\end{quote}

\begin{quote}
\heading{\phantomsection\label{Exercise 1.44}Exercise 1.44:} 関数の
\newterm{smoothing}(\jnewterm{補間})という考えは信号処理において重要な概念である。
\( f \)が関数であり\( dx \)がある小さな値である時、\( f \)の補間とは
\( x \)における値が\( f(x - dx) \), \( f(x) \), and \( f(x + dx) \)
の平均である関数である。入力として\( f \)を計算する手続を取り、補間された
\( f \)を計算する手続を返す手続\code{smooth}を書け。
時には関数の補正を繰り返し(つまり補間された関数をさらに補間
することを繰り返す)、\newterm{\textit{n}-fold smoothed function}(\jnewterm{\textit{n}次畳み込み補間関数})
を得ることには価値がある。任意の与えられた関数の\textit{n}次畳み込み補間関数を\link{Exercise 1.43}の
\code{smooth}と\code{repeated}を用いてどのように生成するかを示せ。
\end{quote}

\begin{quote}
\heading{\phantomsection\label{Exercise 1.45}Exercise 1.45:} \link{Section 1.3.3}にて
平方根を求める試みにおいて単純に\( y \mapsto x / y \)の不動点を探すのでは
収束しないのを見た。この問題は平均減衰にて解決できた。同じ手法が
平均減衰を行った\( y \mapsto x / y^2 \)の不動点として立方根を求める場合においても
うまく行く。残念ながらこの処理は4乗根ではうまくいかない---単一の平均減衰は\( y \mapsto x / y^3 \)
の不動点検索を収束させるのに十分ではない。一方でもし平均減衰を二回行えば(すなわち
\( y \mapsto x / y^3 \))の平均減衰の平均減衰)不動点検索は収束する。\( n \)乗根を
\( y \mapsto x / y^{n-1} \)の平均減衰の繰り返しを基として不動点探索して求める場合に
何回の平均減衰が必要であるかを試行せよ。
この結果を用いて\link{Exercise 1.43}の\code{fixed\-/point}, \code{average\-/damp}, \code{repeated}手続を用いて\( n \)乗根を
求める単一の手続を実装せよ。
必要な数値演算はプリミティブとして存在すると仮定する。

\end{quote}

\begin{quote}
\heading{\phantomsection\label{Exercise 1.46}Exercise 1.46:} この章にて
説明されたいくつかの数値解析手法は非常に汎用的な計算戦略であり
\newterm{iterative improvement}(\jnewterm{反復改善法})として知られている。
反復改善法は何かを求めるために解の初期推測値から始め、推測値が十分に良いかをテストし、
そうでなければ推測値を改善し、改善された推測値を新しい推測値として用いて
処理を継続する。2つの手続を引数として取る手続\code{iterative\-/improve}を書け。1つは
推測値が十分に良いか判断する手続であり、もう1つは推測値を改善する手続である。
\code{iterative\-/improve}は推測値を引数として取り、推測値を十分に良くなる
まで繰り返す手続をその値として返さなければならない。
\link{Section 1.1.7}の\code{sqrt}手続と\link{Section 1.3.3}の\code{fixed\-/point}手続を\code{iterative\-/improve}を
用いて書き直せ。
\end{quote}


%=======================================================================================================
%=======================================================================================================
%=======================================================================================================

\chapter{データを用いた抽象化の構築}
\label{Chapter 2}

\begin{quote}
私達は今、数学上の抽象化の重要なステップに到達しました。
記号がどんな意味を持つのか忘れるのです。
\( \dots \)[数学者]に遊んでいる暇はありません。
これらの記号を用いて実行する演算はいくらでもあります、
これらが何を意味するのか全く考える必要無しに。

---Hermann Weyl, \textit{The Mathematical Way of Thinking}
\end{quote}

\vspace{1.0em}

\noindent
\link{Chapter 1}では演算処理とプログラム設計における手続の役割について集中しました。
私達はプリミティブなデータ(数値)とプリミティブな命令(算術演算)の使い方、
組み合わせ、条件式、パラメタの使用を通して複合手続を形成するための手続の結合方法、
\code{define}を用いた抽象化の方法について学びました。
また手続が処理の局地展開のためのパターンとして見なされ得る
ことを学びました。そして手続内で具体化されたプロセスに対するいくつかの共通パターンの、
簡単なアルゴリズム上の解析を分類し、推論し、実行しました。
また高階手続が、一般的な演算の手段を操作し、その結果を用いて推測することを可能にすることにより、
私達の言語を強化することも学びました。
これはプログラミングの本質の大部分です。

この章では私達はより複雑なデータについて目を向けることにします。
第一章での全ての手続は単純な数値データを操作しましたが、単純なデータは私達が
演算を用いて解決したいと願う多くの問題には不十分です。
プログラムは一般的に複雑な事象をモデル化するために設計され、大抵の場合、
複数の側面を持つ実世界の事象をモデル化するため、いくつかのパーツを持つ演算対象の
オブジェクトを構築せねばなりません。
従って第一章での焦点は手続を組み合わせることで複合手続を形成し抽象化を構築すること
でしたが、この章では任意のプログラミング言語においてもう1つの鍵となる側面に向かい
ます。データオブジェクトを組み合わせ\newterm{compound data}(複合データ)
を形成することによる、プログラミング言語が抽象化の構築に対して与える意味です。


私達はなぜプログラミング言語にてデータを組合せたいのでしょうか?
手続を組み合わせたいのと同じ理由のためです。プログラムを設計可能な概念上のレベル
に持ち上げ、設計の部品化を進め、言語の表現力を拡張したいがためです。
手続を定義する能力が、言語のプリミティブな命令のレベルよりもより高い概念のレベル
において処理を扱うことを可能にしてくれるのと同様に、複合データオブジェクトを構築
できる能力は、言語のプリミティブなデータオブジェクトが与えるよりも
より高い概念レベルのデータを扱うことを可能にします。




分数を用いて数値演算を実行するシステム設計の課題を考えます。
2つの分数を取りそれらの和を実行する命令\code{add\-/rat}を想像します。
単純なデータを用いる場合、分数は2つの整数として考えられます。分子と分母です。
すると各分数が2つの整数(分子と分母)で表現されるプログラムの設計が可能です。
そして\code{add\-/rat}は2つの手続(1つは和の分子を求め、もう1つは分母を求める)にて
実装されるでしょう。しかしこれは不恰好です。それではどの分子がどの分母に
関係するのか明示的に追跡をせねばなりません。多くの分数に対して多くの命令を
実行する目的のシステムにおいては、そのような詳細な記録はプログラムを大幅に
散乱させるのみでなく、それらが私達の心にどんな影響を与えるかについは言うまでも
ありません。もし分子と分母をプログラムが分数を単一の概念上の単位として見做し静的な方法で扱えることができる
ペア---\newterm{compound data object}(\jnewterm{複合データオブジェクト})---
に``貼り合せ''られればずっと良くなることでしょう。



複合データの使用はまたプログラムのモジュラリティ(部品化)を推進します。
もし分数を独自に、直接それ自身をオブジェクトとして扱うことができれば、分数それ自体を
扱うプログラムの一部を、分数が整数のペアとして表現されるだろうという詳細から分離する
ことができます。
データオブジェクトがどのように表現されるかを扱うプログラムの部分を、
データオブジェクトがどのように利用されるかを扱うプログラムの部分から分離する一般的なテクニック
は強力な設計手法であり\newterm{data abstraction}(\jnewterm{データ抽象化})と呼ばれます。
どのようにデータ抽象化がプログラムの設計、保守、変更をより簡単にするかを
これから学びます。



複合データの使用はプログラミング言語の表現力を実際に増加させます。
``一次結合''\( ax + by \)の形式化について考えてみて下さい。
\( a \), \( b \), \( x \), \( y \)を引数として取り、\( ax + by \)の
値を返す手続を書こうと思うかもしれません。これは引数が数値であるならば少しも
難しいとは思えません。私達は既に手続を定義できます。

\begin{scheme}
(define (linear-combination a b x y)
  (+ (* a x) (* b y)))
\end{scheme}

\noindent
しかし数値のみが対象ではないと考えてみましょう。手続の項目として加算と乗算が定義され
ているならば分数、複素数、多項式、その他何でも一次結合を形式化できるというアイデアを
表現したいとします。これを以下の形式の手続として表現できるでしょう。

\begin{scheme}
(define (linear-combination a b x y)
  (add (mul a x) (mul b y)))
\end{scheme}

\noindent
\code{add}と\code{mul}はプリミティブな手続\code{+}と\code{*}ではなく、より複雑なものです。
適切な操作を引数\code{a}, \code{b}, \code{x}, \code{y}として与えたどのような種類のデータに対しても行います。
キーポイントは\code{linear\-/combination}が\code{a}, \code{b}, \code{x}, \code{y}について知らねばならないことは
手続\code{add}とは\code{mul}が適切な操作を行うだろうことのみです。手続\code{linear\-/combination}の視点からは
\code{a}, \code{b}, \code{x}, \code{y}が何であるかは無関係であり、それらがどのようによりプリミティブなデータを
用いて表現されるのかについては尚更無関係です。
この同じ例がなぜプログラミング言語が複合オブジェクトを直接操作する能力を提供することが重要であるのか
を示しています。もしこれが無ければ、\code{linear\-/combination}のような手続に対してその引数を \code{add}と
\code{mul}に向けてそれらの詳細な構造を知らずに渡す方法がありません。\footnote{手続を直接操作する能力は
プログラミング言語の表現力に対して類似の増強を与えます。例えば\link{Section 1.3.1}において\code{sum}手続を紹介
しましたが、これは手続\code{term}を引数として取り、ある指定した区間の\code{term}の値の和を求めました。
\code{sum}を定義するためには\code{term}のような手続を、\code{term}がよりプリミティブな命令にてどのように
表現されているのかに関わらず、それ自身の要素として表現できることが重要でした。実際に、もし``手続''
という概念が無ければ\code{sum}のような命令の定義の可能性について考えつくことすら疑わしいことだったでしょう。
その上、加算の実行を考慮する範囲では\code{term}がどのようによりプリミティブな命令から構築され得るのかの
詳細は無関係なのです。}



私達はこの章を先に触れられた分数の数値演算システムを実装することで始めます。
これが複合データとデータ抽象化の議論の背景を形作ります。
複合手続と同様に、解決すべき主な問題は複雑さを対処するための技術としての抽象化であり
どのようにデータ抽象化が適切な\newterm{abstraction barriers}(\jnewterm{抽象化バリア})を
異なるプログラムの部分の間に構築することを可能にするかについて学びます。



複合データを形成するための鍵はプログラミング言語はある種の``糊''を提供しなければ
ならないことであり、そうすることでデータオブジェクトはより複雑なデータオブジェクト
を形成するために組み合わせることが可能になります。多くの有力な種類の糊が存在します。
実際に、全く特別ではない手続のみの``データ''操作を用いて複合データをどのように形成するかに
ついて発見するでしょう。これは第一章の終りに向かい既に希薄となっていた
``手続''と``データ''の区別をよりボカすことになるでしょう。
また列と木を表現するいくつかの保守的な技術についても探検します。複合データを扱う場合の
鍵となる考えの1つは\newterm{closure}(\jnewterm{クロージャ})の概念です---そのデータオブジェクトを組み合わせ
るのに用いる糊はプリミティブなデータオブジェクトのみではなく、複合データオブジェクトもまた
組み合わせられなければなりません。
もう1つの鍵となる考えは複合データオブジェクトは種々様々な方法でプログラムモジュールを組み立てるための
\newterm{conventional interfaces}(\jnewterm{慣習的インターフェイス})の役を演じることができることです。
これらのアイデアのいくつかについてはクロージャを用いる簡単なグラフィック言語を与えることで
説明します。


次に\newterm{symbolic expressions}(\jnewterm{記号表現})---その基本的な部分は任意の記号であり
数字のみではないデータ---を紹介することで言語の具象的な力を増補します。
オブジェクトの集合を表現するための様々な代替方法について探検します。
与えられた数値関数が多くの異なる演算処理により計算され得るのと同様に、
与えられたデータ構造が多くの方法にてより単純なオブジェクトを用いて表現され得ること、
表現の選択がデータを扱う処理の時間と記憶域の要件に対し重大な影響を与えることについて発見するでしょう。
記号微分、集合の表現、情報符号化のコンテキストにてこれらの考えについて調査します。


次にプログラムの異なる部分において異なって表現され得るデータを用いて処理する問題にとりかかります。 
これが\newterm{generic operations}(\jnewterm{総称命令})の実装の必要性へと導きます。総称命令は
多くの異なるデータの型を扱わなければなりません。総称命令の存在時における部品化の保守は
単純なデータ抽象化のみにより構築可能な場合に比べて、より強力な抽象化バリアを、必要とします。
具体的には\newterm{data-directed programming}(\jnewterm{データ適従プログラミング})を
個別のデータ表現に対し分離した設計と\newterm{additively}(\jnewterm{付加的})に(つまり変更無しに)
組み合わせることを可能にする技術として紹介します。
システム設計に対するこのアプローチを説明するために、
多項式上の記号演算の実行向けパッケージを実装するために私達が学んだことを適用することでこの章を終わります。
その実装の中では多項式の係数は整数、分数、複素数、さらには多項式にもなり得ます。


\section{データ抽象化のイントロダクション}
\label{Section 2.1}



\link{Section 1.1.8}においてより複雑な手続を作成する要素として使われる手続は特定の命令の
集合としてのみでなく、手続の抽象化としても見做されることを伝えました。
その手続がどのように実装されたのかの詳細は抑制可能であり、特定の手続それ自身は
相対的に同じ振舞を持つ任意の他の手続で置き換えられます。
言い換えれば、手続がどのように使われるかをその手続がどのようによりプリミティブな
手続を用いて実装されたかの詳細から分離する抽象化を作成できます。
複合データのための類似の概念は\newterm{data abstraction}(\jnewterm{データ抽象化})と呼ばれます。
データ抽象化はどのように複合データオブジェクトが使用されるかをどれがどのようにより
プリミティブなデータオブジェクトから構築されたのかの詳細から分離することを可能にする
方法論(methodology, メソドロジ)です。



データ抽象化の基本的なアイデアは複合データオブジェクトを使用するためのプログラムを構造化
することで``抽象データ''上で操作を行うことです。それはつまり、私達のプログラムが
手元でタスクを実行するためには厳密には必要と言えないデータに関する想定を一切持たないような
方法でデータを利用しなければいけないということです。同時に、``具体的''なデータ表現はデータ
を利用するプログラムとは独立に定義されます。システムにおけるこれらの2つのパーツの間の
インターフェイスは手続の集合であり、\newterm{selectors}(セレクタ)と\newterm{constructors}(コンストラクタ)
と呼ばれ、抽象データを具体的な表現を用いて実装します。このテクニックを説明するために、
分数を扱う手続の集合をどのように設計するかについてこれから考えます。



\subsection{例: 分数のための数値演算命令}
\label{Section 2.1.1}



分数を用いて数値演算を行いたいとします。足し算、引き算、かけ算、割り算をそれらに
対して行い2つの分数が等しいかテストします。



分子と分母から分数を構築する方法を既に持っていると仮定することから始めましょう。
分数を与えられた時にそれの分子と分母を抽出する(または選択(セレクト)する)方法を持っているとも
仮定します。さらにコンストラクタとセレクタが手続として存在すると仮定します。

\begin{itemize}

\item
\( \hbox{\tt(make-rat}\;\langle{n}\kern0.08em\rangle\;\langle{d}\kern0.06em\rangle\hbox{\tt)} \)は分子が整数\( \langle{n}\kern0.08em\rangle \)であり、かつ分母が整数\( \langle{d}\kern0.06em\rangle \)である分数を返す。

\item
\( \hbox{\tt(numer}\;\;\langle{x}\kern0.08em\rangle\hbox{\tt)} \)は分数\( \langle{x}\kern0.08em\rangle \)の分子を返す。

\item
\( \hbox{\tt(denom}\;\;\langle{x}\kern0.08em\rangle\hbox{\tt)} \)は分数\( \langle{x}\kern0.08em\rangle \)の分母を返す。

\end{itemize}

\noindent
ここで統合のための強力な戦略、\newterm{wishful thinking}(\jnewterm{希望的観測})を用います。
私達はまだ分数がどのように表現されるのか、または手続\code{numer}, \code{denom}, \code{make\-/rat}が
どのように実装されるべきであるのかについて語っていません。そうであっても、
もし私達がこれら3つの手続を持っているのならば、足し算、引き算、かけ算、割り算、等値テスト
を以下の関係性を用いて行うことができるでしょう。

\begin{comment}

\begin{example}
n_1   n_2   n_1 d_2 + n_2 d_1
--- + --- = -----------------
d_1   d_2        d_1 d_2

n_1   n_2   n_1 d_2 - n_2 d_1
--- - --- = -----------------
d_1   d_2        d_1 d_2

n_1   n_2   n_1 n_2
--- * --- = -------
d_1   d_2   d_1 d_2

n_1 / d_1   n_1 d_2
--------- = -------
n_2 / d_2   d_1 n_2

n_1   n_2
--- = ---  if and only if n_1 d_2 = n_2 d_1
d_1   d_2
\end{example}
\end{comment}

\begin{align*}
  {n_1 \over d_1} + {n_2 \over d_2} 	&= {n_1 d_2 + n_2 d_1 \over d_1 d_2}, \\
  {n_1 \over d_1} - {n_2 \over d_2} 	&= {n_1 d_2 - n_2 d_1 \over d_1 d_2}, \\
  {n_1 \over d_1} \cdot {n_2 \over d_2}	&= {n_1 n_2 \over d_1 d_2}, \\
  {n_1 / d_1} \over {n_2 / d_2} 	&= {n_1 d_2 \over d_1 n_2}, \\
  {n_1 \over d_1} 			&= {n_2 \over d_2} \quad 
						{\rm\ if\ and\ only\ if\quad} 
						n_1 d_2 = n_2 d_1.
\end{align*}

\noindent
これらのルールを手続として表現できます。

\begin{scheme}
(define (add-rat x y)
  (make-rat (+ (* (numer x) (denom y))
               (* (numer y) (denom x)))
            (* (denom x) (denom y))))
(define (sub-rat x y)
  (make-rat (- (* (numer x) (denom y))
               (* (numer y) (denom x)))
            (* (denom x) (denom y))))
(define (mul-rat x y)
  (make-rat (* (numer x) (numer y))
            (* (denom x) (denom y))))
(define (div-rat x y)
  (make-rat (* (numer x) (denom y))
            (* (denom x) (numer y))))
(define (equal-rat? x y)
  (= (* (numer x) (denom y))
     (* (numer y) (denom x))))

\end{scheme}


これでセレクタとコンストラクタの手続である\code{numer}, \code{denom}, \code{make\-/rat}を用いて
分数の操作を定義できました。必要な物は分子と分母を貼り合せて分数を形成する何らかの
方法です。

\subsubsection*{ペア}



データ抽象化の具体的レベルを実装できるようになるために、私達の言語は\newterm{pair}(\jnewterm{ペア})
と呼ばれる複合構造を提供します。それはプリミティブな手続\code{cons}を用いて構築できます。
この手続は2つの引数を取り、2つの引数を部分として持つ複合データオブジェクトを返します。
ペアを与えられた時、プリミティブな手続\code{car}と\code{cdr}を用いてそれの部分を抽出することが
できます。\footnote{\code{cons}という名前は``construct''によります。\code{car}と\code{cdr}という
名前はIBM 704上でのオリジナルのLisp実装に由来します。このマシンはアドレシングの仕組み
としてメモリロケーションの``アドレス''と``デクリメント''の部分を参照可能でした。
\code{car}は``Contents of Address part of Register''(レジスタのアドレス部分の中身)を表し、
\code{cdr}(``クダー''と読みます)は``Contents of Decrement part of Register.''(レジスタの
デクリメント部分の中身)を表します。}
従って、\code{cons}, \code{car}, \code{cdr}を以下のように使用できます。

\begin{scheme}
(define x (cons 1 2))
(car x)
~\textit{1}~
(cdr x)
~\textit{2}~
\end{scheme}

\noindent
ペアは名前を与えることができ、プリミティブなデータオブジェクトと同様に扱うこと
ができるデータオブジェクトです。さらに\code{cons}はその要素がペアであるペアや、その繰り返しを
作ることも可能です。

\begin{scheme}
(define x (cons 1 2))
(define y (cons 3 4))
(define z (cons x y))
(car (car z))
~\textit{1}~
(car (cdr z))
~\textit{3}~
\end{scheme}

\noindent
\link{Section 2.2}においてペアを組み立てるこの能力が、
全ての種類の複雑なデータ構造を作成するために汎用目的構築ブロックとしてペアが利用可能
であることに対し、どのような意味を持つのかについて学びます。
\code{cons}, \code{car}, \code{cdr}により実装された単一の複合データプリミティブペアが
私達が必要とするただ1つの糊です。ペアから構築されたデータオブジェクトは
\newterm{list-structured}(\jnewterm{リスト構造化})データと呼ばれます。

\subsubsection*{分数を表現する}



ペアは分数システムを仕上げるための自然な方法を提供します。単純に分数を2つの
整数、分子と分母のペアとして表現します。そして\code{make\-/rat}, \code{numer}, \code{denom}は
簡単に次のように実装することが可能です。\footnote{
セレクタとコンストラクタを
定義するもう1つの実装として次が挙げられます。

\begin{smallscheme}
(define make-rat cons)
(define numer car)
(define denom cdr)
\end{smallscheme}



最初の定義は名前\code{make\-/rat}を式\code{cons}の値に関連付けます。それはペアを
構築するプリミティブな手続です。従って\code{make\-/rat}と\code{cons}は同じプリミティブな
コンストラクタになります。



セレクタとコンストラクタをこのように定義するのは効率が良いです。\code{make\-/rat}が
\code{cons}を\emph{calling}(呼び出す)代わりに、\code{make\-/rat}が\code{cons}で\emph{is}(ある)
ためです。そのため\code{make\-/rat}が呼ばれた時に2つでなく、1つの手続が呼ばれるだけに
なります。しかし一方で、これを行うことは手続呼出のトレースや手続呼出に対するブレイクポイント
の設定に対するデバッグ上の手助けを無効にしてしまいます。あなたは\code{make\-/rat}の呼出を
見たくなるのであって、\code{cons}への全ての呼出を見たい訳ではないからです。



この本ではこの定義スタイルを使用しないことにしました。}

\begin{scheme}
(define (make-rat n d) (cons n d))
(define (numer x) (car x))
(define (denom x) (cdr x))
\end{scheme}

\noindent
また演算結果を表示するために、分数を分子、スラッシュ、分母で表示することにします。\footnote{
\code{display}はデータを表示するSchemeのプリミティブです。Schemeのプリミティブである\code{newline}
は表示を新しい行から始めます。これらの手続のどちらも意味のある値は返しません。そのため
下記の\code{print\-/rat}内での使用においては\code{print\-/rat}が表示する物のみを示し、インタプリタが
\code{print\-/rat}の返り値として表示する物は示していません。}

\begin{scheme}
(define (print-rat x)
  (newline)
  (display (numer x))
  (display "/")
  (display (denom x)))
\end{scheme}

\noindent
これで分数手続を試すことができます。

\begin{scheme}
(define one-half (make-rat 1 2))
(print-rat one-half)
~\textit{1/2}~
(define one-third (make-rat 1 3))
(print-rat (add-rat one-half one-third))
~\textit{5/6}~
(print-rat (mul-rat one-half one-third))
~\textit{1/6}~
(print-rat (add-rat one-third one-third))
~\textit{6/9}~
\end{scheme}

\noindent
最後の例が示すとおり、私達の分数実装は分数を最も小さな項に約分しません。
これを\code{make\-/rat}を変更することで改良できます。もし\link{Section 1.2.5}で扱った
2つの整数の最大公約数を生成する\code{gcd}手続を持っていれば、\code{gcd}を用いて
分子と分母を最小の項に、ペアを構築する前に縮小することができます。

\begin{scheme}
(define (make-rat n d)
  (let ((g (gcd n d)))
    (cons (/ n g) (/ d g))))
\end{scheme}

\noindent
これで次の希望した結果を得ます。

\begin{scheme}
(print-rat (add-rat one-third one-third))
~\textit{2/3}~
\end{scheme}

\noindent
この変更は(\code{add\-/rat}や\code{mul\-/rat}のような)実際の命令を実装する他の手続の変更無しに、
コンストラクタ\code{make\-/rat}を変更することで達成されました。

\begin{quote}
引数を扱えるより良い版の\code{make\-/rat}を定義せよ。
\heading{\phantomsection\label{Exercise 2.1}Exercise 2.1:} 正と負の両方の
\code{make\-/rat}は符号の正常化を行わなければならない。従ってもし分数が正であれば
分子と分母の両方が正であるし、もし分数が負であれば分子のみが負でなければならない。
\end{quote}


\subsection{抽象化バリア}
\label{Section 2.1.2}


さらなる複合データとデータ抽象化の例を続ける前に、分数の例にて持ち上がったいくつかの
問題について考えてみましょう。私達は分数操作をコンストラクタ\code{make\-/rat}とセレクタ\code{numer}と
\code{denom}を用いて定義しました。一般的にデータ抽象化の基となる考えは
データオブジェクトの各型に対し、その型のデータオブジェクトの全ての操作が表される命令を用いて、
命令の基本的な集合を判断し、そのデータを操作する時にそれらの命令のみを用いることです。





私達は\link{Figure 2.1}にて示された分数システムの構造を想像することができます。
水平線は\newterm{abstraction barriers}(\jnewterm{抽象化バリア})を表現し、システムの異なる``レベル''
を分離します。各レベルではバリアはデータ抽象化を利用する(上側の)プログラムをデータ抽象化
を実装する(下側の)プログラムから分離します。
分数を利用するプログラムはもっぱら分数パッケージにより``公用向け''に提供された
手続を用いて分数を操作します。それら手続とは\code{add\-/rat},
\code{sub\-/rat}, \code{mul\-/rat}, \code{div\-/rat}, それに \code{equal\-/rat?}です。
これらは順に、もっぱらコンストラクタとセレクタである\code{make\-/rat}, \code{numer}, \code{denom}を
用いて実装されます。この3つはペアを用いて実装されます。ペアがどのように実装されているかの
詳細はペアが\code{cons}, \code{car}, \code{cdr}の使用により操作できる限りにおいては分数パッケージの
他の物に取っては重要ではありません。実質的に、各レベルにおける手続は抽象化バリアを定義する
インターフェイスであり、異なるレベルを接続します。



この単純な考えは多くの利点を持ちます。1つの利点はプログラムの保守と変更をより簡単に
することです。任意の複雑なデータ構造が、プログラミング言語により提供されるプリミティブな
データ構造を用いて多彩な方法で表現されます。もちろん、表現の選択がその上で操作を行う
プログラムに影響を与えます。従ってもし表現がある程度後に変更された場合、全ての当該
プログラムはそれに応じて変更されなければなりません。この作業は大きなプログラムの
場合においては
表現上の依存が設計によりとても少ないプログラムモジュールに対してのみに制限されていなければ
時間のかかる高コストな物に成り得ます。

\begin{figure}[tb]
\phantomsection\label{Figure 2.1}
\centering
\begin{comment}
\begin{quote}
\heading{Figure 2.1:} Data-abstraction barriers in the rational-number package.

\begin{example}
        +------------------------------------+
--------| Programs that use rational numbers |--------
        +------------------------------------+
          Rational numbers in problem domain
            +---------------------------+
------------|   add-rat  sub-rat  ...   |-------------
            +---------------------------+
   Rational numbers as numerators and denominators
              +------------------------+
--------------| make-rat  numer  denom |--------------
              +------------------------+
              Rational numbers as pairs
                  +----------------+
------------------| cons  car  cdr |------------------
                  +----------------+
            However pairs are implemented
\end{example}
\end{quote}
\end{comment}
\includegraphics[width=91mm]{fig/chap2/Fig2.1c.pdf}
\begin{quote}
\heading{Figure 2.1:} 分数パッケージ内の抽象化バリア
\end{quote}
\end{figure}

\noindent
例として、分数を最小の項へと約分する問題の解法の代替法には、分数を組み立てた時でなく、
分数のパーツにアクセスする度に約分を実行する方法があります。
これは異なるコンストラクタとセレクタ手続に導きます。

\begin{scheme}
(define (make-rat n d) (cons n d))
(define (numer x)
  (let ((g (gcd (car x) (cdr x))))
    (/ (car x) g)))
(define (denom x)
  (let ((g (gcd (car x) (cdr x))))
    (/ (cdr x) g)))
\end{scheme}

\noindent
この実装と依然の実装との間の違いはいつ\code{gcd}を求めるかにあります。
私達の典型的な分数の使用においては、同じ分数の分子と分母に何度もアクセスする場合、
分数が組み立てられる時に\code{gcd}を求めるほうが好ましいです。そうでなければ
\code{gcd}を求めるのはアクセスする時まで待ったほうが良いかもしれません。
どちらの場合でも、一方の表現からもう一方の表現へと変更する場合、手続、\code{add\-/rat}, 
\code{sub\-/rat}, その他は全く変更する必要がありません。



表現上の依存対象を少ないインターフェイス手続に制約することはプログラムの設計と共に
それらの変更をも手助けします。なぜなら代替的な実装を考えるための柔軟性を保つ
ことを可能にするためです。私達の簡単な例で続けるために、私達は分数パッケージ
を設計中で、早期に\code{gcd}を構築時と選択時のどちらで実行するか決められないと
想像して下さい。データ抽象化メソドロジはその決定をシステムの他の部分上の進行の
可能性を失わせずに決定を遅らせる方法を与えます。

\begin{quote}
\heading{\phantomsection\label{Exercise 2.2}Exercise 2.2:} 平面上の線分を
表現する問題について考える。各線分は点のペアにて表現する。始点と
終点である。コンストラクタ\code{make\-/segment}とセレクタ\code{start\-/segment}と\code{end\-/segment}
を定義せよ。それらは点を用いて線分の表現を定義する。さらに点は数値のペアにて表現できる。
\( x \)座標と\( y \)座標である。それに沿ってこの表現を定義するコンストラクタ\code{make\-/point}と
セレクタ\code{x\-/point}を\code{y\-/point}を定めよ。
最後に、セレクタとコンストラクタを用いて引数として線分を取りその中点(その座標が両端点の
座標の平均である点)を返す手続\code{midpoint\-/segment}を定義せよ。
あなたの手続をテストするためには以下の点を表示する方法が必要だろう。

\begin{scheme}
(define (print-point p)
  (newline)
  (display "(")
  (display (x-point p))
  (display ",")
  (display (y-point p))
  (display ")"))
\end{scheme}
\end{quote}

\begin{quote}
\heading{\phantomsection\label{Exercise 2.3}Exercise 2.3:} 平面上の
三角のための表現を実装せよ。(ヒント:\link{Exercise 2.2}を利用したいだろう。)
コンストラクタとセレクタを利用して、与えられた三角の周辺の長さと面積を求める手続を
作れ。適切な抽象化バリアを用いてどんな表現を用いても同じ周辺長と面積の手続が働くよう、
あなたのシステムを設計できるだろうか?

\end{quote}

\subsection{データにより何が意味されるのか}
\label{Section 2.1.3}




\link{Section 2.1.1}にて分数実装を分数演算\code{add\-/rat}, \code{sub\-/rat}, その他を
3つの定められていない手続、\code{make\-/rat}, \code{numer}, \code{denom}を用いて
実装することから始めました。その時点では命令はデータオブジェクト---分子、分母と
分数を用いて定義されると考えることができました。データオブジェクトの振舞は
後者の3つの手続により指定されました。



しかし\newterm{data}(\jnewterm{データ})とは正確には何を意味するのでしょうか。
``与えられたセレクタとコンストラクタにより実装された物全て''と言うのみでは十分ではありません。
明かに3つの手続の任意の集合全てが分数実装に対する適切な基準としての役割を果せる訳ではありません。
もし分数\code{x}を整数のペア\code{n}と\code{d}から組み立てた場合、\code{x}の\code{numer}と\code{denom}の抽出し
それらを割ることは、\code{n}を\code{d}で割るのと同じ結果になることを保証せねばなりません。
言い替えれば、\code{make\-/rat}, \code{numer}, \code{denom}は任意の整数\code{n}と零でない整数\code{d}に対し
もし\code{x}が\code{(make\-/rat n d)}である時、その場合以下の条件を満たさなければなりません。
\begin{comment}

\begin{example}
(numer x)    n
--------- = ---
(denom x)    d
\end{example}

\end{comment}
\begin{displaymath}
 {\hbox{\tt(numer x)} \over \hbox{\tt(denom x)}} = {{\tt n} \over {\tt d}}\,.  
\end{displaymath}


実際にこれが\code{make\-/rat}, \code{numer}, \code{denom}が分数表現のための適切な基準を形成する
ために満たさなければならないただ1つの条件です。
一般的に、私達はデータをセレクタとコンストラクタのある集合と共に、これらの手続が
有効な表現となるために満たさなければならない制約により定義されると考えることができます。\footnote{
意外にもこの考えは厳格に形式化することがとても難しいのです。そのような形式化を与える試みは
2つあります。1つはC. A. R. \link{Hoare (1972)}により開拓され、\newterm{abstract models}(\jnewterm{抽象モデル})
として知られています。``手続プラス制約''の仕様を上の分数の例内で概説されたように形式化します。
分数表現上の条件は整数に関する事実(等値関係と除算)を用いて規定されています。
一般的に抽象モデルは新しい種類のデータオブジェクトを以前に定義されたデータオブジェクトの型を用いて
定義します。従ってデータオブジェクトに関する成立条件はそれらを以前に定義されたデータオブジェクト
に関する成立条件へと還元していくことでチェックできます。
もう1つの試みは\acronym{MIT}のZillesとIBMのGoguen, Thatcher, Wagner, Wrightにより紹介され
(\link{Thatcher et al. 1978}を参照)、またトロント大学のGuttagにより紹介されました。(\link{Guttag 1977}
を参照)。その試みは``手続''を抽象代数システムの要素と見做し、その振舞は``条件''に相当する公理により
指定されました。そして抽象代数のテクニックを用いてデータオブジェクトに関する成立条件をチェックしました。
両者の手法が\link{Liskov and Zilles (1975)}により論文として調査されています。}



この視点は分数のような``高階データオブジェクト''のみを定義するのではなく、
より低いレベルのオブジェクトの定義も提供することができます。
私達が分数を定義するために使用したペアの概念について考えてみます。
私達はまだペアとは実際には何であるのか述べていません。言語が手続\code{cons}, \code{car}, \code{cdr}を
ペア上の命令として提供するとのみ説明しています。しかしこれら3つの命令について
知らなければいけないことはもし私達が2つのオブジェクトを\code{cons}を用いて貼り合わせた時、
\code{car}と\code{cdr}を用いてそれらのオブジェクトを取得することができることのみです。
つまり、それらの命令は任意のオブジェクト\code{x}と\code{y}に対し、もし\code{z}が\code{(cons x y)}であるなら
\code{(car z)}は\code{x}であり、\code{(cdr z)}は\code{y}であるという制約を満たしています。
実際に、これらの3つの手続は言語にプリミティブとして含まれていることについて既に述べました。
しかし、上記の制約を満たす任意の3つの手続ならペアを実装するための基盤として使用することが
可能です。この点は私達が\code{cons}, \code{car}, \code{cdr}をどんなデータ構造も全く利用せずに、
しかし手続のみを用いて実装できることにより、著しく説明されます。
これがその定義です。

\begin{scheme}
(define (cons x y)
  (define (dispatch m)
    (cond ((= m 0) x)
          ((= m 1) y)
          (else (error "Argument not 0 or 1: CONS" m))))
  dispatch)
(define (car z) (z 0))
(define (cdr z) (z 1))
\end{scheme}

\noindent
この手続の使用はデータが何であるべきかという私達の直感的概念のような物には全く
関係しません。それでもなお、これがペアを表現するのに有効な方法であると示すのに
必要なこと全てはこれらの手続が上で与えられた制約を満たすことです。



注意すべき微細な点は\code{(cons x y)}により返される値は手続---すなわち内部で定義された
手続\code{dispatch}であることです。それが1つの引数を取り\code{x}か\code{y}のどちらかを
引数が0であるか1であるかに従って返します。相応して、\code{(car z)}は\code{z}を0に適用します。
故にもし\code{z}が\code{(cons x y)}により作られた手続であるのなら、\code{z}を0に適用すれば
\code{x}を返します。従って、\code{(car (cons x y))}が希望通りに\code{x}を返すことを示しました。
同様に\code{(cdr (cons x y))}は\code{(cons x y)}の返り値としての手続を1に適用し、
\code{y}を返します。従ってこのペアの手続としての実装は有効な実装であり、もし私達が
\code{cons}, \code{car}, \code{cdr}のみを用いてペアにアクセスする場合、この実装を``本物の''データ構造を
用いる実装と区別することはできません。



ペアの手続による表現を提示することのポイントは私達の言語がこのように働いているという
ことではなく(Schemeや一般的なLispシステムは効率上の理由からペアを直接的に実装します)、
しかしそれがこのように働くことができるということです。手続による表現は曖昧ですが、ペアを
表現するのに完璧に適切な方法です。ペアが満たすべき必要な条件を満たすからです。
この例はまた手続をオブジェクトとして操作する能力が自動的に複合データを表現する能力を
提供することを実演しました。これは今は珍しく見えるかもしれませんが、しかし手続による
データの表現は私達のプログラミングレパートリの中心的役割を演じます。このプログラミング
スタイルは時折\newterm{message passing}(\jnewterm{メッセージパッシング})と呼ばれ、私達はこれを
\link{Chapter 3}にてモデリングとシミュレーションの問題を解決する時に基本的なツールとして
用います。

\begin{quote}
\heading{\phantomsection\label{Exercise 2.4}Exercise 2.4:} ここにペアの
代替的な手続上の表現がある。この表現に対して\code{(car (cons
x y))}が任意のオブジェクト\code{x}と\code{y}に対して\code{x}を返すか確認せよ。

\begin{scheme}
(define (cons x y)
  (lambda (m) (m x y)))
(define (car z)
  (z (lambda (p q) p)))
\end{scheme}



対応する\code{cdr}の定義はどうなるか? (ヒント:これが正しく働くか確認するには
\link{Section 1.1.5}の置換モデルを使用せよ)
\end{quote}

\begin{quote}
\heading{\phantomsection\label{Exercise 2.5}Exercise 2.5:} 負ではない
整数のペアを数値と数値演算命令のみを用いて表現できることを、もし\( a \)と\( b \)の
ペアを積\( 2^a 3^b \)の整数で表現すれば可能であることにより示せ。
対応する手続\code{cons}, \code{car}, \code{cdr}.の定義を与えよ。
\end{quote}

\begin{quote}
\heading{\phantomsection\label{Exercise 2.6}Exercise 2.6:} ペアを
手続として表現することが十分に驚かせるに値するものでない場合、
手続を操作可能なある言語においては0と1を足すことを以下のように実装することで
数値が無くてもやっていける(少なくとも負ではない整数のみを考える場合においては)ことを
考えてみよ。

\begin{scheme}
(define zero (lambda (f) (lambda (x) x)))
(define (add-1 n)
  (lambda (f) (lambda (x) (f ((n f) x)))))
\end{scheme}



この表現はその開発者に因んで\newterm{Church numerals}(\jnewterm{チャーチ数})として知られる。
Alonzo Churchは\( \lambda \)-演算を発明した論理学者である。



\code{one}と\code{two}を直接(\code{zero}と\code{add\-/1}を用いずに)定義せよ。(ヒント:
加算手続の直接的な定義\code{+}を与えよ。(\code{add\-/1}の繰り返し適用は用いない)
\end{quote}


\subsection{延長課題: 区間演算}
\label{Section 2.1.4}



Alyssa P. Hackerは人々が工学上の問題を解くのを手助けするシステムを設計しています。
システムにおいて彼女が提供したい1つの機能は(物理機器の測定されたパラメータのような)不正確な量を
既知の精度にて扱う能力です。演算がそのような近似量にて行われた時、結果が既知の
精度の値になるようにするためです。



電気技術者達がAlyssaのシステムを電気の量を計算するために使用します。彼らは時折2つの
抵抗\( R_1 \), \( R_2 \)の並列に等価な抵抗値\( R_p \)を次の式を用いて計算する
必要があります。
\begin{comment}

\begin{example}
            1
R_p = -------------
      1/R_1 + 1/R_2
\end{example}

\end{comment}
\begin{displaymath}
 R_p = {1 \over 1 / R_1 + 1 / R_2}\,.  
\end{displaymath}


抵抗値は通常抵抗の生産者により保証されるいくらかの許容誤差未満であることが知られています。
例えばもしあなたが``10\%の許容誤差で6.8Ω''とラベリングされた抵抗を買ったとしたら、
確かなのはその抵抗は\( 6.8 - 0.68 = 6.12 \)と\( 6.8 + 0.68 = 7.48 \)Ωの間の抵抗を
持つことのみです。従って、もし6.8Ω 10\%の抵抗と並列に4.7Ω 5\%の抵抗を接続した場合に、
組み合わせの抵抗は約2.58Ω(2つの抵抗が低限である場合)から約2.97Ω(2つの抵抗が上限である場合)
の区間になります。



Alyssaのアイデアは``interval arithmetic''(区間演算)を``区間''(不正確な量の取り得る
値の区間を表現するオブジェクト)を連結する演算命令の集合として実装することです。
2つの区間の加算、減算、乗算、除算の結果はそれ自身が区間であり、結果の範囲を表します。



Alyssaは2つの終端、下限と上限を持つ``区間''と呼ばれる抽象オブジェクトの存在を仮定しました。
彼女はまた区間の終端を与えられた時、データコンストラクタ\code{make\-/interval}を用いて区間の
構築ができると仮定しました。Alyssaは最初に2つの区間を足す手続を書きました。彼女は和の最小値は
2つの下限の和であり、最大値は2つの上限の和になるだろうと推測しました。

\begin{scheme}
(define (add-interval x y)
  (make-interval (+ (lower-bound x) (lower-bound y))
                 (+ (upper-bound x) (upper-bound y))))
\end{scheme}

\noindent
Alyssaはまた2つの区間の積を限界値の積の最小値と最大値を見つけることで算出し、
そしてそれらを結果区間の限界値として用いました。(\code{min}と\code{max}は任意の数の引数の
最小値と最大値を見つけるプリミティブです)。

\begin{scheme}
(define (mul-interval x y)
  (let ((p1 (* (lower-bound x) (lower-bound y)))
        (p2 (* (lower-bound x) (upper-bound y)))
        (p3 (* (upper-bound x) (lower-bound y)))
        (p4 (* (upper-bound x) (upper-bound y))))
    (make-interval (min p1 p2 p3 p4)
                   (max p1 p2 p3 p4))))
\end{scheme}

\noindent
2つの区間を割るために、Alyssaは一つ目に2つ目の逆数を掛けました。区間の逆数の限界値は
上限の逆数と下限の逆数をその順で用いることに注意して下さい。

\begin{scheme}
(define (div-interval x y)
  (mul-interval 
   x
   (make-interval (/ 1.0 (upper-bound y))
                  (/ 1.0 (lower-bound y)))))
\end{scheme}

\begin{quote}
\heading{\phantomsection\label{Exercise 2.7}Exercise 2.7:} Alyssaの
プログラムは未完成である。なぜなら彼女は区間の抽象の実装を特定していない。
以下に区間のコンストラクタの定義を置く。

\begin{scheme}
(define (make-interval a b) (cons a b))
\end{scheme}


セレクタ\code{upper\-/bound}と\code{lower\-/bound}を定義し実装を完成させよ。
\end{quote}

\begin{quote}
\heading{\phantomsection\label{Exercise 2.8}Exercise 2.8:} Alyssaの
考えと同様の推論を用いて、2つの区間の差がどのように計算されるかを説明せよ。
対応する減算手続\code{sub\-/interval}を定義せよ。
\end{quote}

\begin{quote}
\heading{\phantomsection\label{Exercise 2.9}Exercise 2.9:} 区間の
\newterm{width}(\jnewterm{幅})は上限と下限の差の半値である。幅は区間で指定された数値
の不確かさの基準である。いくつかの数値演算に対しては、2つの区間を結合した結果の幅は
引数区間の幅のみによる関数である。一方で他の演算においては結合の幅は引数の幅の
関数ではない。2つの区間の和、または差の幅は足される、または引かれる区間の幅の
関数であることを示せ。これが乗算と除算においては正しくないことを例をもって示せ。
\end{quote}

\begin{quote}
\heading{\phantomsection\label{Exercise 2.10}Exercise 2.10:} エキスパートシステム
プログラマのBen BitdiddleはAlyssaの肩越しに覗いて、
区間の長さが0の時に割ったらどうなるのか不明だよとコメントした。
Alyssaのコードを変更し、この条件をチェックしてもしそれが起こればエラーを返すようにせよ。
\end{quote}

\begin{quote}
\heading{\phantomsection\label{Exercise 2.11}Exercise 2.11:} 通り過ぎながら
Benはまた曖昧なコメントを残した。``区間の終端の符号をテストすることで
\code{mul\-/interval}を7つに場合分けできる。その1つのみが2つ以上の乗算を必要とする。''
この手続をBenの提案に従い書き直せ。
\end{quote}
\noindent
プログラムをデバッグした後で、Alyssaはユーザ候補の一人に見せた。彼は彼女の
プログラムは間違った問題を解いていると文句を言った。彼が欲しい のは中央値として
表現された数値と追加の許容誤差を扱えるプログラムだ。例えば彼は\( 3.5 \pm 0.15 \)の
ような区間を扱いたく、[3.35, 3.65]ではない。Alyssaは彼女の机に戻りこの問題を代替となる
コンストラクタとセレクタを提供することで直した。

\begin{scheme}
(define (make-center-width c w)
  (make-interval (- c w) (+ c w)))
(define (center i)
  (/ (+ (lower-bound i) (upper-bound i)) 2))
(define (width i)
  (/ (- (upper-bound i) (lower-bound i)) 2))
\end{scheme}
\noindent
不運なことに、Alyssaのユーザの多くはエンジニアです。実際の工学の場では通常、小さな不確かさ
を伴なう計測を伴い、区間の中央値に対する区間の幅の割合として測定されます。
エンジニアは通常パーセンテージにて許容誤差を端末のパラメータ上に、以前に与えた抵抗の
仕様のように指定します。

\begin{quote}
\heading{\phantomsection\label{Exercise 2.12}Exercise 2.12:} コンストラクタ
\code{make\-/center\-/percent}を中央値とパーセンテージ許容誤差を取り
望まれた区間を返すように定義せよ。セレクタ\code{percent}を与えられた区間に対する
パーセンテージ許容誤差を返すように定義することも行うこと。\code{center}セレクタは
上で見たものと同じである。
\end{quote}

\begin{quote}
\heading{\phantomsection\label{Exercise 2.13}Exercise 2.13:} 小さな
パーセンテージ許容誤差の前提の下では、2つの区間の積のパーセンテージ許容誤差を
因数の許容誤差を用いて近似するための簡単な式が存在することを示せ。
全ての数値は正であると前提して問題を簡単にしても良い。
\end{quote}

\noindent
大変な仕事を終え、Alyssa P. Hackerは完了したシステムを受け渡しました。何年か後、彼女が全てを忘れた頃に、
彼女は興奮した電話を、怒ったユーザ、Lem E. Tweakitから受けました。どうやらLemは並列接続の抵抗の式が2つの
代数的に等価な方法で書くことができることに気付いたようです。
\begin{comment}

\begin{example}
 R_1 R_2
---------
R_1 + R_2
\end{example}

\end{comment}
\begin{displaymath}
 R_1 R_2 \over R_1 + R_2 
\end{displaymath}
\noindent
と、
\begin{comment}

\begin{example}
      1
-------------
1/R_1 + 1/R_2
\end{example}

\end{comment}
\begin{displaymath}
 {1 \over 1 / R_1 + 1 / R_2}\,. 
\end{displaymath}

彼は以下の2つのプログラムを書きました。それぞれが並列接続の抵抗値を異なる式で計算します。

\begin{scheme}
(define (par1 r1 r2)
  (div-interval (mul-interval r1 r2)
                (add-interval r1 r2)))
\end{scheme}

\begin{scheme}
(define (par2 r1 r2)
  (let ((one (make-interval 1 1)))
    (div-interval 
     one (add-interval (div-interval one r1)
                       (div-interval one r2)))))
\end{scheme}

\vspace{0.3em}

LemはAlyssaのプログラムは2つの方法の演算にて異なる値を返すと抗議しました。
これは深刻な苦情です。

\begin{quote}
\heading{\phantomsection\label{Exercise 2.14}Exercise 2.14:} Lemが正しいことを
確認せよ。様々な数値演算にてシステムの挙動を調べよ。
ある区間\( A \)と\( B \)を作成し、式\( A /\! A \)と\( A / B \)の計算において
それらを用いよ。幅が中央値の小さなパーセンテージである区間を用いることで多くの実態を掴む
ことができるだろう。center-percent形式(\link{Exercise 2.12}参照)の演算の結果を
調査せよ。
\end{quote}

\begin{quote}
\heading{\phantomsection\label{Exercise 2.15}Exercise 2.15:} Eva Lu Atorは
もう一人のユーザで、彼女もまた異なるが代数的には等価な式により異なる区間が
算出されることに気付いた。彼女はAlyssaのシステムを用いて区間の計算をする式が、
もし式が不確かな値を表現する変数がどれも繰り返されない形であれば、より厳しい
エラーの限界を算出すると言う。
従って彼女は抵抗の並列に対し、\code{par2}の方が\code{par1}より``より良い''プログラムであると述べた。
彼女は正しいだろうか? それは何故か?
\end{quote}

\begin{quote}
\heading{\phantomsection\label{Exercise 2.16}Exercise 2.16:} 一般的に、
なぜ等価な代数式が異なる答に導くのか説明せよ。
この欠点を持たない区間演算パッケージを開発することは可能だろうか。
または不可能だろうか。(警告:この問題はとても難しい)
\end{quote}

%=======================================================================================================

\section{階層データと閉包性}
\label{Section 2.2}



ここまで学んだように、ペアは私達が複合データオブジェクトを構築するのに利用可能な
プリミティブな``糊''を提供します。\link{Figure 2.2}はペア---この場合は\code{(cons 1 2)}
にて形成されたペアを図示する標準的な方法を示しています。
この\newterm{box-and-pointer notation}(\jnewterm{箱と点表記法})と呼ばれる表現において、
各オブジェクトは箱への\newterm{pointer}(\jnewterm{ポインタ})として表わされています。
プリミティブオブジェクトの箱はオブジェクトの表現を持っています。例えば数値の箱は
数字を持っています。ペアの箱は実際には二重の箱で、左部分はペアの\code{car}(へのポインタ)
を持っており、右部分は\code{cdr}を持っています。




私達は既に\code{cons}が数値のみでなくペアもまた組み合わせられることについて学びました。
(\link{Exercise 2.2}と\link{Exercise 2.3}であなたはこの事実を用いたか、または用いざるを得な
かったでしょう)。結果としてペアは全ての種類のデータ構造を構築可能な普遍的な構築ブロックを
提供します。\link{Figure 2.3}は数値1, 2, 3, 4を組み合わせるためにペアを用いる
2つの方法を示しています。

\begin{figure}[tb]
\phantomsection\label{Figure 2.2}
\centering
\begin{comment}
\heading{Figure 2.2:} Box-and-pointer representation of \code{(cons 1 2)}.

\begin{example}
     +---+---+     +---+
---->| * | *-+---->| 2 |
     +-|-+---+     +---+
       |
       V
     +---+
     | 1 |
     +---+
\end{example}
\end{comment}
\includegraphics[width=34mm]{fig/chap2/Fig2.2c.pdf}
\begin{quote}
\heading{Figure 2.2:} \code{(cons 1 2)}の箱とポインタ表現
\end{quote}
\end{figure}

\begin{figure}[tb]
\phantomsection\label{Figure 2.3}
\centering
\begin{comment}
\heading{Figure 2.3:} Two ways to combine 1, 2, 3, and 4 using pairs.

\begin{example}
     +---+---+     +---+---+         +---+---+     +---+
---->| * | *-+---->| * | * |    ---->| * | *-+---->| 4 |
     +-|-+---+     +-|-+-|-+         +-|-+---+     +---+
       |             |   |             |
       V             V   V             V
   +---+---+      +---+ +---+      +---+---+     +---+---+
   | * | * |      | 3 | | 4 |      | * | *-+---->| * | * |
   +-|-+-|-+      +---+ +---+      +-|-+---+     +-|-+-|-+
     |   |                           |             |   |
     V   V                           V             V   V
  +---+ +---+                      +---+        +---+ +---+
  | 1 | | 2 |                      | 1 |        | 2 | | 3 |
  +---+ +---+                      +---+        +---+ +---+

  (cons (cons 1 2)                 (cons (cons 1
        (cons 3 4))                            (cons 2 3))
                                         4)
\end{example}
\end{comment}
\includegraphics[width=96mm]{fig/chap2/Fig2.3c.pdf}
\begin{quote}
\heading{Figure 2.3:} ペアを用いて1,2,3,4を組み合わせる2つの方法
\end{quote}
\end{figure}

\noindent
要素がペアであるペアを作成する能力は表現ツールとしてのリスト構造の重要性の本質です。
私達はこの能力を\code{cons}の\jnewterm{閉包性}(closure property)と呼びます。
一般的に、データオブジェクトを組み合わせる操作は
もしその命令による組み合わせの結果それ自身が同じ命令を用いて組み合わせることが可能ならば
閉包性を満たします。\footnote{
ここでの``closure''(閉包)という用語の使用は抽象代数から来ており、
もし操作の集合の要素への適用により生成される要素が再び同じ集合の要素である場合、
要素の集合が操作の下において閉じられていると呼ばれます。Lispコミュニティでは(残念なことに)
用語``closure''を全く関係のない概念にも使用しています。closureとは自由変数を持つ
手続を表現するための実装テクニックです。私達は``closure''をこの本の中では2つ目の意味では
用いません。}
閉包はどのような目的の組み合わせをも強力にする鍵となります。なぜなら\newterm{hierarchical}(\jnewterm{階層})構造---
複数のパーツから成る構造であり、パーツ自身も複数のパーツから成るような構造を作成することが
可能になるためです。



\link{Chapter 1}の始めから、手続の取扱において閉包を本質的に利用してきました。
とても簡単なプログラムを除けば全てのプログラムは組み合わせの要素はそれ自身も
また組み合わせであるという事実に依存しているためです。
この節では複合データにとっての閉包の重要性を取り上げます。
ペアを使用して列と木を表現するための、いくつかの便利なテクニックを説明します。
そして鮮烈な方法でクロージャを図示するグラフィック言語を提示します。\footnote{
組み合わせは閉包であるべきだという手段の概念は単純なアイデアです。
残念なことに多くの人気の有るプログラミング言語が提供するデータの組み合わせ手法は
閉包性を満たしませんし、閉包性の活用が面倒です。FortranやBASICではデータ要素を
組合せる典型的な1つの手段は配列にそれらをまとめることです。しかし配列の要素自身が
配列である配列を形成できません。PascalとCは構造体の要素が構造体であることを認めます。
しかしこれはプログラマが明示的にポインタを取り扱うことを要求し、構造体の各フィールドが
事前に指定された形式の要素のみを保管できるという制約に帰着します。Lispのペアとは
異なりこれらの言語は複合データを統一的な方法で扱うことを簡単にする組み込みの汎用目的な糊
を持っていません。この制約がこの本の前書きにおけるAlan Perlisのコメントの背景にあります。
``Pascalにおける過剰な宣言可能なデータ構造は関数内にて特殊化を引き起こし、カジュアルな連携を
不利にし、抑制してしまう。1つのデータ構造を操作する100の関数を持つほうが、10のデータ構造を
操作する10の関数を持つよりも良い。''}

\subsection{列の表現}
\label{Section 2.2.1}



ペアを用いて構築可能な便利な構造の1つが\newterm{sequence}(\jnewterm{列})---順に並べたデータ
オブジェクトの集合です。もちろんペアを用いて列を表現する方法は数多く存在します。
特に簡単な表現方法の1つを\link{Figure 2.4}に示します。列1, 2, 3, 4がペアの連鎖として
表わされています。各ペアの\code{car}は鎖内で相対するアイテムであり、各ペアの\code{cdr}は
鎖内での次のペアです。最後のペアの\code{cdr}は列の終端をペアではないことを識別する値を
指し示すことで合図します。箱とポインタの図では斜めの線にて表現され、プログラムでは
変数\code{nil}の値にて示されます。列全体は入れ子の\code{cons}命令にて構築されます。

\begin{scheme}
(cons 1
      (cons 2
            (cons 3
                  (cons 4 nil))))
\end{scheme}

\begin{figure}[tb]
\phantomsection\label{Figure 2.4}
\centering
\begin{comment}
\heading{Figure 2.4:} The sequence 1, 2, 3, 4 represented as a chain of pairs.

\begin{example}
     +---+---+     +---+---+     +---+---+     +---+---+
---->| * | *-+---->| * | *-+---->| * | *-+---->| * | / |
     +-|-+---+     +-|-+---+     +-|-+---+     +-|-+---+
       |             |             |             |
       V             V             V             V
     +---+         +---+         +---+         +---+
     | 1 |         | 2 |         | 3 |         | 4 |
     +---+         +---+         +---+         +---+
\end{example}
\end{comment}
\includegraphics[width=76mm]{fig/chap2/Fig2.4c.pdf}
\begin{quote}
\heading{Figure 2.4:} ペアの鎖として表現された列1, 2, 3, 4
\end{quote}
\end{figure}

\noindent
そのようなペアの列は入れ子の\code{cons}にて形成され、\newterm{list}(\jnewterm{リスト})と呼ばれます。
そしてSchemeは\code{list}と呼ばれるプリミティブを提供しリストの構築を手助けします。\footnote{
この本では\newterm{list}をリスト終端マーカーにて終端化されたペアの鎖を意味するように
使用します。一方で用語\newterm{list structure}(\jnewterm{リスト構造})はペアから作り上げられた
任意のデータ構造を参照し、ただのリストは意味しません。}
上の列は\code{(list 1 2 3 4)}により生成可能です。

\begin{scheme}
(list ~\( \dark \langle \)~~\( \dark a_1 \)~~\( \dark \rangle \)~ ~\( \dark \langle \)~~\( \dark a_2 \)~~\( \dark \rangle \)~ ~\( \dots \)~ ~\( \dark \langle \)~~\( \dark a_n \)~~\( \dark \rangle \)~)
\end{scheme}

\noindent
is equivalent to

は以下と等価です。

\begin{scheme}
(cons ~\( \dark \langle \)~~\( \dark a_1 \)~~\( \dark \rangle \)~
      (cons ~\( \dark \langle \)~~\( \dark a_2 \)~~\( \dark \rangle \)~
            (cons ~\( \dots \)~
                  (cons ~\( \dark \langle \)~~\( \dark a_n \)~~\( \dark \rangle \)~
                        nil)~\( \dots \)~)))
\end{scheme}

\noindent
Lispシステムは慣習としてリストを括弧で括られた要素の列を表示することで表します。
従って\link{Figure 2.4}のデータオブジェクトは\code{(1 2 3 4)}の様に表示されます。

\begin{scheme}
(define one-through-four (list 1 2 3 4))
one-through-four
~\textit{(1 2 3 4)}~
\end{scheme}

\noindent
式\code{(list 1 2 3 4)}とリスト\mbox{\code{(1 2 3 4)}}を取り違えないよう気をつけて下さい。
リストは式が評価された時に得られた結果です。式\code{(1 2 3 4)}を評価しようとする試みは
インタプリタが手続\code{1}を引数\code{2}, \code{3}, \code{4}に適用しようとした時にエラーを
発します。



\code{car}をリスト内の最初のアイテムを選択すると考えることもでき、\code{cdr}を
最初のアイテム以外の全てにより成り立つサブ(副)リストを選択すると考えることも
可能です。\code{car}と\code{cdr}の入れ子の適用はリスト内の2つ目、3つ目、そして
その後に続く複数のアイテムを抽出するために利用可能です。\footnote{
\code{car}と\code{cdr}の入れ子の適用は書くのが面倒なためLispの各種方言はそれらに対する
略記法を提供しています。例えば、

\begin{smallscheme}
(cadr ~\( \dark \langle \)~~\( \dark arg \)~~\( \dark \rangle \)~) = (car (cdr ~\( \dark \langle \)~~\( \dark arg \)~~\( \dark \rangle \)~))
\end{smallscheme}

そのような手続全ての名前は\code{c}で始まり\code{r}で終わります。それらの間の各\code{a}は\code{car}命令を
意図し、\code{d}は\code{cdr}命令を意図し、その名前に現れた順と同じ順にて適用されます。\code{car}と\code{cdr}の
名前は存続します。なぜなら\code{cadr}のような単純な組み合わせが発音可能だからです。}


コンストラクタ\code{cons}は元のリストと同様のリストを作りますが、最初に追加のアイテムを入れます。

\begin{scheme}
(car one-through-four)
~\textit{1}~
(cdr one-through-four)
~\textit{(2 3 4)}~
(car (cdr one-through-four))
~\textit{2}~
(cons 10 one-through-four)
~\textit{(10 1 2 3 4)}~
(cons 5 one-through-four)
~\textit{(5 1 2 3 4)}~
\end{scheme}

\noindent
code{nil}の値はペアの鎖を終了するために使用されますが、全く要素が無い列\newterm{empty list}
(\jnewterm{空リスト})として考えることもできます。\newterm{nil}という単語はラテン語の単語\emph{nihil}
の省略形で、``無''を意味します。\footnote{
どれだけのエネルギーがLisp方言の標準化において文字通り意味の無い議論に浪費されたかについて
は特筆に値します。\code{nil}は普通の名前であるべきか? \code{nil}の値は記号であるべきか?
それはリストであるべきか? それはペアであるべきか? Schemeでは\code{nil}は普通の名前であり
この節では変数として扱いその値はリスト終端マーカーです。(\code{true}が普通の変数であり、真の値を
持つのと同様です)。Common Lispを含む他のLisp方言は\code{nil}を特別な記号として扱います。
この本の著者達は、言語の標準化における数多くの乱闘に耐えてきたので、この問題全体を避けたいと
思います。\link{Section 2.3}にてquoteを紹介した後には空リストに\code{'()}という名前を付け、全体的に
変数\code{nil}を免除します。}

\subsubsection*{リスト命令}


複数のペアを使用して要素の列をリストのように表現することは、慣習的なプログラミングテクニック
である連続してリストを``\code{cdr}で縮小する''ことによりリストを操作するのと同時に生じます。
例えば手続\code{list\-/ref}は引数としてリストと数値\( n \)を取り、リストの\( n \)番目の項目を
返します。リストの要素を数えるのに0から始めるのが慣習です。
\code{list\-/ref}を計算する方法は以下の通りです。

\begin{itemize}

\item
\( n = 0 \)の場合、\code{list\-/ref}はリストの\code{car}を返す。

\item
そうでなければ、\code{list\-/ref}はリストの\code{cdr}の\( (n - 1) \)番目の項目を返す。

\end{itemize}

\begin{scheme}
(define (list-ref items n)
  (if (= n 0)
      (car items)
      (list-ref (cdr items) (- n 1))))
(define squares (list 1 4 9 16 25))
(list-ref squares 3)
~\textit{16}~
\end{scheme}

\noindent
時折、私達はリスト全体を\code{cdr}で下ります。これを手助けするために、Schemeは
プリミティブな手続\code{null?}を持っており、その引数が空リストであるかどうかを試験します。
手続\code{length}はリスト内の要素数を返しますが、\code{null?}の使用の典型的なパターンを説明します。

\begin{scheme}
(define (length items)
  (if (null? items)
      0
      (+ 1 (length (cdr items)))))
(define odds (list 1 3 5 7))
(length odds)
~\textit{4}~
\end{scheme}

\noindent
手続\code{length}は単純な最近計画を実装します。集約ステップは以下の通りです。

\begin{itemize}

\item
任意のリストの\code{length}はリストの\code{cdr}の\code{length}に1を足した値

\end{itemize}

\noindent
これが底となるケースに到達するまで繰り返し適用される

\begin{itemize}

\item
空リストの\code{length}は0

\end{itemize}

\noindent
また\code{length}は反復スタイルでも計算可能です。

\begin{scheme}
(define (length items)
  (define (length-iter a count)
    (if (null? a)
        count
        (length-iter (cdr a) (+ 1 count))))
  (length-iter items 0))
\end{scheme}

\noindent
もう1つの慣習的なプログラミングテクニックは\code{cdr}を繰り返し利用しリストを下る間に、
答のリストを``\code{cons}で積み上げ''ることです。これは手続\code{append}にて利用され、
\code{append}は2つのリストを引数として取り、それらの要素を結合し、新しいリストを作ります。

\begin{scheme}
(append squares odds)
~\textit{(1 4 9 16 25 1 3 5 7)}~
(append odds squares)
~\textit{(1 3 5 7 1 4 9 16 25)}~
\end{scheme}

\noindent
\code{append}もまた再帰計画を用いて実装されます。リスト\code{list1}と\code{list2}を
\code{append}するためには以下の通りに行います。

\begin{itemize}

\item
もし\code{list1}が空リストであれば、結果は単に\code{list2}

\item
そうでない場合、\code{list1}の\code{cdr}と\code{list2}を\code{append}し、その結果の上に
\code{list1}の\code{car}を\code{cons}する

\end{itemize}

\begin{scheme}
(define (append list1 list2)
  (if (null? list1)
      list2
      (cons (car list1) (append (cdr list1) list2))))
\end{scheme}

\begin{quote}
\heading{\phantomsection\label{Exercise 2.17}Exercise 2.17:} 与えられた(空でない)
リストの最初の要素のみを持つリストを返す手続\code{last\-/pair}を定義せよ。

\begin{scheme}
(last-pair (list 23 72 149 34))
~\textit{(34)}~
\end{scheme}
\end{quote}

\begin{quote}
\heading{\phantomsection\label{Exercise 2.18}Exercise 2.18:}リストを引数として取り、同じ要素を逆順に持つリストを返す手続\code{reverse}を定義せよ。

\begin{scheme}
(reverse (list 1 4 9 16 25))
~\textit{(25 16 9 4 1)}~
\end{scheme}
\end{quote}

\begin{quote}
\heading{\phantomsection\label{Exercise 2.19}Exercise 2.19:} \link{Section 1.2.2}の
両替数え上げプログラムについて考える。プログラムにて用いられる
通貨を容易に変更できるようになればとても良いだろう。そうすることで例えば
イギリスのポンドの両替方法の数を計算できるようになるだろう。プログラムが書かれた時には、
通貨の知識はある部分は手続\code{first\-/denomination}の中に、またある部分は手続\code{count\-/change}の
中に存在した。(\code{count\-/change}は米国の貨幣には5種類あることを知っていた)。
両替を行うため利用される貨幣のリストが提供できるようになればより良くなるだろう。


\code{cc}を変更することで、その2つ目の引数がどの貨幣を使用するかを指定する整数ではなく、
使用する貨幣の値のリストとなるようにしたいと考える。そして通貨の各種類を定義する
リストを持つことにする。

\begin{scheme}
(define us-coins (list 50 25 10 5 1))
(define uk-coins (list 100 50 20 10 5 2 1 0.5))
\end{scheme}


次に\code{cc}を以下のように呼び出す。

\begin{scheme}
(cc 100 us-coins)
~\textit{292}~
\end{scheme}


これを行うためにはプログラム\code{cc}に何らかの変更が必要だ。同じ形態を保つが、
2つ目の引数に異なる方法でアクセスする。以下のようになる。

\begin{scheme}
(define (cc amount coin-values)
  (cond ((= amount 0) 1)
        ((or (< amount 0) (no-more? coin-values)) 0)
        (else
         (+ (cc amount
                (except-first-denomination 
                 coin-values))
            (cc (- amount
                   (first-denomination 
                    coin-values))
                coin-values)))))
\end{scheme}

リスト構造に対するプリミティブな命令を用いて
手続\code{first\-/denomination}, \code{except\-/first\-/de- nomination}, and \code{no\-/more?}を定義せよ。
リスト\code{coin\-/values}の順は\code{cc}により生成される解答に影響を与えるか?
それは何故か? または何故そうでないのか?
\end{quote}

\begin{quote}
\heading{\phantomsection\label{Exercise 2.20}Exercise 2.20:}手続
\code{+}, \code{*}, \code{list}は任意の数の引数を取る。そのような手続を定義する1つの
方法として\newterm{dotted-tail notation}(\jnewterm{ドット付き末尾記法})と共に\code{define}を使用することが上げられる。
手続定義において、最後のパラメータ名の前にドットがあるパラメータリストは手続が呼び出された時に、
最初以下のパラメータが(もし存在したら)初期引数の値を通常通りに持つが、最後のパラメータの値は
残りの引数全てのリストとなる。例えば、以下の定義を与えられた時に、

\begin{scheme}
(define (f x y . z) ~\( \dark \langle \)~~\var{\dark body}~~\( \dark \rangle \)~)
\end{scheme}

\noindent
手続\code{f}は2つ以上の引数で呼び出すことが可能だ。もし次を評価すれば、

\begin{scheme}
(f 1 2 3 4 5 6)
\end{scheme}

\noindent
\code{f}のボディでは\code{x}が1、 \code{y}が2、そして\code{z}はリスト\mbox{\code{(3 4 5 6)}}となる。
以下の定義を与えられた時、

\begin{scheme}
(define (g . w) ~\( \dark \langle \)~~\var{\dark body}~~\( \dark \rangle \)~)
\end{scheme}

\noindent
手続\code{g}はゼロ個以上の引数にて呼び出し可能となる。
次を評価すれば、

\begin{scheme}
(g 1 2 3 4 5 6)
\end{scheme}

\noindent
\code{g}のボディでは\code{w}はリスト\code{(1 2 3 4 5 6)}となる。\footnote{
To define \code{f} and \code{g} using \code{lambda} we would
write


\code{f}と\code{g}を\code{lambda}を用いて定義するには、以下のように記述する。

\begin{smallscheme}
(define f (lambda (x y . z) ~\( \dark \langle \)~~\var{\dark body}~~\( \dark \rangle \)~))
(define g (lambda w ~\( \dark \langle \)~~\var{\dark body}~~\( \dark \rangle \)~))
\end{smallscheme}
}



この記法を用いて手続\code{same\-/parity}を書け。\code{same\-/parity}は1つまたは
それ以上の整数を引数として取り、最初の引数と同じ偶奇性を持つ全ての引数の
リストを返す。例えば、

\begin{scheme}
(same-parity 1 2 3 4 5 6 7)
~\textit{(1 3 5 7)}~
(same-parity 2 3 4 5 6 7)
~\textit{(2 4 6)}~
\end{scheme}
\end{quote}

\subsubsection*{リストに渡るmap}

あるとても便利な命令は、ある変換をリストの各要素に適用し、結果のリストを返します。
例えば以下の手続はリストの各数値を与えられた因数で拡大します。

\begin{scheme}
(define (scale-list items factor)
  (if (null? items)
      nil
      (cons (* (car items) factor)
            (scale-list (cdr items) 
                        factor))))
(scale-list (list 1 2 3 4 5) 10)
~\textit{(10 20 30 40 50)}~
\end{scheme}

\noindent
私達は\link{Section 1.3}のように、この一般的な考えを抽象化し、高階手続にて表現された共通なパターン
として捉えることができます。ここでの高階手続は\code{map}と呼ばれます。\code{map}は引数として
1引数の手続とリストを取り、返り値としてその手続をリストの各要素に適用することで得られた
結果のリストを返します。\footnote{\label{Footnote 12}
Schemeは標準としてここで説明される物よりもより汎用的な\code{map}手続を提供します。
このより汎用的な\code{map}は\( n \)引数の手続を、\( n \)個のリストと共に取り、
全てのリストの最初の要素を手続に適用し、次に全ての2つ目の要素を適用し、以下それを繰り返し、
結果のリストを返します。例えば、

\begin{smallscheme}
(map + (list 1 2 3) (list 40 50 60) (list 700 800 900))
~\textit{(741 852 963)}~
(map (lambda (x y) (+ x (* 2 y)))
     (list 1 2 3)
     (list 4 5 6))
~\textit{(9 12 15)}~
\end{smallscheme}
}

\begin{scheme}
(define (map proc items)
  (if (null? items)
      nil
      (cons (proc (car items))
            (map proc (cdr items)))))
(map abs (list -10 2.5 -11.6 17))
~\textit{(10 2.5 11.6 17)}~
(map (lambda (x) (* x x)) (list 1 2 3 4))
~\textit{(1 4 9 16)}~
\end{scheme}

\noindent
これで\code{map}を用いた新しい\code{scale\-/list}の定義を与えられる。

\begin{scheme}
(define (scale-list items factor)
  (map (lambda (x) (* x factor))
       items))
\end{scheme}

\noindent
\code{map}は重要な構造です。それが共通なパターンを掴むからだけでなく、リストを扱う
より高いレベルの抽象化を確立させるためです。\code{scale\-/list}の元の定義では
プログラムの再帰構造はリストのエレメント毎の処理に注意を引きました。
\code{map}を用いた\code{scale\-/list}の定義はそのレベルの詳細を抑制し、要素のリストから
結果のリストへの拡大変換を強調します。2つの定義の間の違いはコンピュータが異なる
処理を行うことではなく(異なりますが)、私達が過程について異って考えていることです。
実際に、\code{map}はリストの要素がどのように抽出され、また結合されるかの詳細から
リストを変換する手続の実装を分離する抽象化バリアを強化することを手助けします。
\link{Figure 2.1}にて示されるバリアのように、この抽象化は私達に列がどのように実装さ
れるかの低レベルの詳細を変更する柔軟性を提供し、その上で列から列へと変換する操作の
概念上のフレームワークを保っている。\link{Section 2.2.3}はこのプログラムを構成するためのフレームワーク
としての列の利用を拡張している。

\begin{quote}
\heading{\phantomsection\label{Exercise 2.21}Exercise 2.21:} T手続
\code{square\-/list}は数値のリストを引数として取りそれらの数値の二乗のリストを返す。

\begin{scheme}
(square-list (list 1 2 3 4))
~\textit{(1 4 9 16)}~
\end{scheme}


ここに2つの異なる\code{square\-/list}がある。失なわれた式を埋めることで両者を
完成させよ。

\begin{scheme}
(define (square-list items)
  (if (null? items)
      nil
      (cons ~\( \dark \langle \)~~\var{\dark ??}~~\( \dark \rangle \)~ ~\( \dark \langle \)~~\var{\dark ??}~~\( \dark \rangle \)~)))
(define (square-list items)
  (map ~\( \dark \langle \)~~\var{\dark ??}~~\( \dark \rangle \)~ ~\( \dark \langle \)~~\var{\dark ??}~~\( \dark \rangle \)~))
\end{scheme}
\end{quote}

\begin{quote}
\heading{\phantomsection\label{Exercise 2.22}Exercise 2.22:} Louis Reasonerは
\link{Exercise 2.21}の最初の\code{square\-/list}手続を書き直し、
反復プロセスを展開させようと試みている。

\begin{scheme}
(define (square-list items)
  (define (iter things answer)
    (if (null? things)
        answer
        (iter (cdr things)
              (cons (square (car things))
                    answer))))
  (iter items nil))
\end{scheme}



残念なことに、\code{square\-/list}をこのように定義しては解答のリストは希望の逆順に
なってしまう。何故か?


Louisはそこで彼のバグを\code{cons}への引数を逆順にすることで直そうと試みた。

\begin{scheme}
(define (square-list items)
  (define (iter things answer)
    (if (null? things)
        answer
        (iter (cdr things)
              (cons answer
                    (square (car things))))))
  (iter items nil))
\end{scheme}

これもまたうまく行かない。説明せよ。
\end{quote}

\begin{quote}
\heading{\phantomsection\label{Exercise 2.23}Exercise 2.23:} 手続
\code{for\-/each}は\code{map}に似ている。手続と要素のリストを引数として取る。しかし
結果のリストを形成するのではなく、\code{for\-/each}はただ手続を左から右へと毎回各要素に適用する。
手続を要素に適用し返された値は全く利用しない---\code{for\-/each}は表示のような行動を起こす手続と共に
利用される。例えば、

\begin{scheme}
(for-each (lambda (x)
            (newline)
            (display x))
          (list 57 321 88))
~\textit{57}~
~\textit{321}~
~\textit{88}~
\end{scheme}



(上では示されていない)\code{for\-/each}呼出による返り値は真のような不定な何かである。
\code{for\-/each}の実装を与えよ。
\end{quote}

\subsection{階層構造}
\label{Section 2.2.2}



リストを用いた列の表現は要素が列自身である列を表現することを自然に一般化する。
例えば以下の様に構築されたオブジェクト\code{((1 2) 3 4)}を

\begin{scheme}
(cons (list 1 2) (list 3 4))
\end{scheme}

\noindent
最初の項目はそれ自身がリスト\code{(1 2)}である、3つの項目のリストであると見做すことができる。
実際に、インタプリタにより表示される結果の形式によりこれは推奨されている。
\link{Figure 2.5}がペアを用いたこの構造の表現を示している。

\begin{figure}[tb]
\phantomsection\label{Figure 2.5}
\centering
\begin{comment}
\heading{Figure 2.5:} Structure formed by \code{(cons (list 1 2) (list 3 4))}.

\begin{example}
                                          (3 4)
                                            |
                                            V
((1 2) 3 4)  +---+---+                  +---+---+     +---+---+
        ---->| * | *-+----------------->| * | *-+---->| * | / |
             +-|-+---+                  +-|-+---+     +-|-+---+
               |                          |             |
               V                          V             V
      (1 2)  +---+---+     +---+---+    +---+         +---+
        ---->| * | *-+---->| * | / |    | 3 |         | 4 |
             +-|-+---+     +-|-+---+    +---+         +---+
               |             |
               V             V
             +---+         +---+
             | 1 |         | 2 |
             +---+         +---+
\end{example}
\end{comment}
\includegraphics[width=91mm]{fig/chap2/Fig2.5c.pdf}
\begin{quote}
\heading{Figure 2.5:} \code{(cons (list 1 2) (list 3 4))}により形作られた構造
\end{quote}
\end{figure}

\noindent
要素それ自身が列である列のもう1つの考え方は木としての考え方である。
列の要素は木の枝であり、それ自身が列である\mbox{elements}(複数の要素)は部分木である。
\link{Figure 2.6}は木として見た場合の\link{Figure 2.5}を示している。

\begin{figure}[tb]
\phantomsection\label{Figure 2.6}
\centering
\begin{comment}
\heading{Figure 2.6:} The list structure in \link{Figure 2.5} viewed as a tree.

\begin{example}
 ((1 2) 3 4)
     /\\
    /  | \
(1 2)  3 4
 / \
 1 2
\end{example}
\end{comment}
\includegraphics[width=22mm]{fig/chap2/Fig2.6a.pdf}
\begin{quote}
\heading{Figure 2.6:} 木として見た\link{Figure 2.5}のリスト構造
\end{quote}
\end{figure}

\noindent
再帰は木構造を扱うのに自然なツールです。良く木に対する操作を枝に対する操作へと
還元でき、それは順に枝の枝への操作へと還元され、木の葉に辿り着くまで繰り返され
ます。例として、\link{Section 2.2.1}の\code{length}手続を木の葉の総数を求める\code{count\-/leaves}
手続と比べてみましょう。

\begin{scheme}
(define x (cons (list 1 2) (list 3 4)))
(length x)
~\textit{3}~
(count-leaves x)
~\textit{4}~
(list x x)
~\textit{(((1 2) 3 4) ((1 2) 3 4))}~
(length (list x x))
~\textit{2}~
(count-leaves (list x x))
~\textit{8}~
\end{scheme}

\noindent
\code{count\-/leaves}を実装するには\code{length}を求める再帰計画を思い出します。

\begin{itemize}

\item
リスト\code{x}の\code{length}は\code{x}の\code{cdr}の\code{length}に1を足した物

\item
空リストの\code{length}は0

\end{itemize}

\noindent
\code{count\-/leaves}も同様えす。空リストの値は同じで

\begin{itemize}

\item
空リストの\code{count\-/leaves}は0

\end{itemize}

\noindent
しかし集約ステップにおいて、リストの\code{car}を取り除く時、\code{car}はそれ自身が
後で数えねばならない木である可能性があることを計算に入れねばなりません。
従って適切な集約ステップは

\begin{itemize}

\item
木\code{x}の\code{count\-/leaves}は\code{x}の\code{car}の\code{count\-/leaves}と、
\code{x}の\code{cdr}の\code{count\-/leaves}の和

\end{itemize}

\noindent
最終的に\code{car}を取ることにより実際の葉に届くので別の規範を必要とする。

\begin{itemize}

\item
葉の\code{count\-/leaves}は1

\end{itemize}

\noindent
木に対する再帰手続を書くのを助けるために、Schemeはプリミティブな手続\code{pair?}を提供します。
\code{pair?}は引数がペアであるかをテストします。以下に完成した手続を置きます。\footnote{
\code{cond}の最初の2つの項の順が大事です。空リストは\code{null?}を満たし、その上でペアでも
ありません}

\begin{scheme}
(define (count-leaves x)
  (cond ((null? x) 0)
        ((not (pair? x)) 1)
        (else (+ (count-leaves (car x))
                 (count-leaves (cdr x))))))
\end{scheme}

\begin{quote}
\heading{\phantomsection\label{Exercise 2.24}Exercise 2.24:} 式
\code{(list 1 (list 2 (list 3 4)))}を評価したとする。インタプリタの表示する
結果、対応する箱と点構造、木としての解釈(\link{Figure 2.6}相当)を示せ。
\end{quote}

\begin{quote}
\heading{\phantomsection\label{Exercise 2.25}Exercise 2.25:} 以下の
各リストから7を抽出する\code{car}と\code{cdr}の組み合わせを与えよ。

\begin{scheme}
(1 3 (5 7) 9)
((7))
(1 (2 (3 (4 (5 (6 7))))))
\end{scheme}
\end{quote}

\begin{quote}
\heading{\phantomsection\label{Exercise 2.26}Exercise 2.26:} 2つのリスト\code{x}と\code{y}を定義したとする。

\begin{scheme}
(define x (list 1 2 3))
(define y (list 4 5 6))
\end{scheme}


以下の各式を評価した場合にレスポンスとしてインタプリタがどのような結果を表示するか?

\begin{scheme}
(append x y)
(cons x y)
(list x y)
\end{scheme}
\end{quote}

\begin{quote}
\heading{\phantomsection\label{Exercise 2.27}Exercise 2.27:} \link{Exercise 2.18}の
\code{reverse}手続を変更してリストを引数として取り、
全ての要素が逆順に、さらに全てのサブリストも同様に逆順にされたリストをその値として
返す手続\code{deep\-/reverse}を作れ。例として、

\begin{scheme}
(define x (list (list 1 2) (list 3 4)))
x
~\textit{((1 2) (3 4))}~
(reverse x)
~\textit{((3 4) (1 2))}~
(deep-reverse x)
~\textit{((4 3) (2 1))}~
\end{scheme}
\end{quote}

\begin{quote}
\heading{\phantomsection\label{Exercise 2.28}Exercise 2.28:} リストとして
表現された木を引数に取り、その木の全ての葉を左から右への順で要素とした
リストを返す手続\code{fringe}を書け。

\begin{scheme}
(define x (list (list 1 2) (list 3 4)))
(fringe x)
~\textit{(1 2 3 4)}~
(fringe (list x x))
~\textit{(1 2 3 4 1 2 3 4)}~
\end{scheme}
\end{quote}

\begin{quote}
\heading{\phantomsection\label{Exercise 2.29}Exercise 2.29:} バイナリモバイル(binary mobile)\footnote{
訳注:天井から糸で釣ってあり、絶妙なバランスで揺れ、回る数々の棒のインテリア。
枝が必ず2つに分かれるのでバイナリ(二進)と名付けられている。Google Imagesで
mobileを検索すると実物が見られる。}
は左の枝と右の枝の2つの枝で構成される。各枝はある長さを持つ
棒であり、そこから重りか別のバイナリモバイルをぶら下げる。
バイナリモバイルを複合データを用いて2つの枝から組み立てることで表現できる。
(例えば\code{list}を用いる。)

\begin{scheme}
(define (make-mobile left right)
  (list left right))
\end{scheme}


枝は\code{length}(数値であること)と\code{structure}から組み立てられ、\code{structure}は
数値(簡単に重りを表わす)かまたは他のモバイルである。

\begin{scheme}
(define (make-branch length structure)
  (list length structure))
\end{scheme}

\begin{enumerate}[a]

\item
対応するセレクタ\code{left\-/branch}を\code{right\-/branch}を書け。このセレクタは
モバイルの複数の枝を返す。また\code{branch\-/length}と\code{branch\-/structure}
は枝のそれぞれのコンポーネント(構成要素)を返す。

\item
セレクタを用いて手続\code{total\-/weight}を定義せよ。それはモバイルの総重量を
返す。

\item
モバイルは一番上の左枝にかかるトルク(回転力)が一番上の右の枝にかかるトルクと等しい時
(これはつまり、もし左の棒の長さと
その棒にかかる重さを掛けた値が、相対する右側の積の値と同じ場合である)、かつ
各部分モバイルも全て同様である場合に限り、
\newterm{balanced}(\jnewterm{バランスが取れた状態})であると言う。
あるバイナリモバイルがバランスが取れているかテストする述語を設計せよ。

\item
モバイルの表現をコンストラクタが以下になるように変更すると考える。

\begin{scheme}
(define (make-mobile left right) (cons left right))
(define (make-branch length structure)
  (cons length structure))
\end{scheme}



新しい表現へとあなたのプログラムを変更するのにどれほどが必要か?

\end{enumerate}
\end{quote}

\subsubsection*{木に渡るmap}



\code{map}が列を扱うのに強力な抽象化であるのと同様に、再帰を伴なう\code{map}は木を
扱うのに強力な抽象化です。例えば\link{Section 2.2.1}の\code{scale\-/list}に同類な\code{scale\-/tree}
手続は引数として因数と葉が数値である木を取ります。これは同じ形の木を返しますが、
各数値は因数により乗算されます。\code{scale\-/tree}の再帰計画は\code{count\-/leaves}に対する物に
似ています。

\begin{scheme}
(define (scale-tree tree factor)
  (cond ((null? tree) nil)
        ((not (pair? tree)) (* tree factor))
        (else (cons (scale-tree (car tree) factor)
                    (scale-tree (cdr tree) factor)))))
(scale-tree (list 1 (list 2 (list 3 4) 5) (list 6 7)) 10)
~\textit{(10 (20 (30 40) 50) (60 70))}~
\end{scheme}

\noindent
\code{scale\-/tree}を実装するもう1つの方法は木を部分木の列と見做し\code{map}を使用します。
列全体に\code{map}をかけ、各部分木を順に拡大し、結果のリストを返します。その木が葉である
場合には単純に因数を掛けます。

\begin{scheme}
(define (scale-tree tree factor)
  (map (lambda (sub-tree)
         (if (pair? sub-tree)
             (scale-tree sub-tree factor)
             (* sub-tree factor)))
       tree))
\end{scheme}

\noindent
多くの木の操作が同様な列操作と再帰の組み合わせにて実装可能です。

\begin{quote}
\heading{\phantomsection\label{Exercise 2.30}Exercise 2.30:} \link{Exercise 2.21}の
\code{square\-/list}と同様の手続\code{square\-/tree}を定義せよ。
\code{square\-/tree}は以下の振舞を行う。

\begin{scheme}
(square-tree
 (list 1
       (list 2 (list 3 4) 5)
       (list 6 7)))
~\textit{(1 (4 (9 16) 25) (36 49))}~
\end{scheme}

\code{square\-/tree}を直接な(つまり高階関数を全く用いない)方法と\code{map}と再帰を
用いる方法の両者を定義せよ。
\end{quote}

\begin{quote}
\heading{\phantomsection\label{Exercise 2.31}Exercise 2.31:} 
\link{Exercise 2.30}への解答を抽象化し、手続\code{tree\-/map}を作れ。
\code{tree\-/map}を用いて\code{square\-/tree}は以下のように定義できる。

\begin{scheme}
(define (square-tree tree) (tree-map square tree))
\end{scheme}
\end{quote}

\begin{quote}
\heading{\phantomsection\label{Exercise 2.32}Exercise 2.32:} 集合は
識別可能な要素のリストとして表現できる。そして集合の全ての部分集合集合を
リストのリストとして表わせられる。例えば、集合が\code{(1 2 3)}である時、全ての
部分集合の集合は\code{(() (3) (2) (2 3) (1) (1 3) (1 2) (1 2 3))}だ。
以下の集合の全ての部分集合の集合を生成する手続の定義を完成し、なぜうまくいくのか
を明確に説明せよ。

\begin{scheme}
(define (subsets s)
  (if (null? s)
      (list nil)
      (let ((rest (subsets (cdr s))))
        (append rest (map ~\( \dark \langle \)~~\var{\dark ??}~~\( \dark \rangle \)~ rest)))))
\end{scheme}
\end{quote}

\subsection{慣習的インターフェイスとしての列}
\label{Section 2.2.3}



複合データを用いて働く場合、これまでデータ抽象化がどれだけプログラムの設計を
データ表現の詳細に陥らずに行えるか、また抽象化がどれだけ代替的な表現方法を
試みる柔軟性を保つかについて強調してきました。この節ではもう1つの強力なデータ構造を用いる
設計原則を紹介します。\newterm{conventional interfaces}(\jnewterm{慣習的インターフェイス})の使用です。



\link{Section 1.3}においてプログラム抽象化、高階手続としての実装がどのようにして数値データを取り扱う
プログラムの共通パターンを掴むことができるのかを学んできました。複合データを扱う類似の操作を
形式化する能力は決定的にデータ構造を扱うスタイルに依存します。例えば次の手続について考えて
みて下さい。\link{Section 2.2.2}の\code{count\-/leaves}手続に類似しており、木を引数として取り、奇数の葉の
二乗の合計を求めます。

\begin{scheme}
(define (sum-odd-squares tree)
  (cond ((null? tree) 0)
        ((not (pair? tree))
         (if (odd? tree) (square tree) 0))
        (else (+ (sum-odd-squares (car tree))
                 (sum-odd-squares (cdr tree))))))
\end{scheme}

\noindent
表面上では、この手続は以下の物ととても異なっています。以下では全ての偶数のフィボナッチ数
\( {\rm Fib}(k) \)のリストを、\( k \)が与えられた\( n \)以下の範囲にて作成しています。

\begin{scheme}
(define (even-fibs n)
  (define (next k)
    (if (> k n)
        nil
        (let ((f (fib k)))
          (if (even? f)
              (cons f (next (+ k 1)))
              (next (+ k 1))))))
  (next 0))
\end{scheme}

\noindent
これらの2つの手続は構造的にとても異なっているという事実にも係らず、2つの計算の
より抽象的な記述は大きな類似性を明らかにします。最初のプログラムは

\begin{itemize}

\item
木の葉を列挙する

\item
フィルタを通して奇数のみを選ぶ

\item
選択された数の二乗を求める

\item
初期値0にて\code{+}を用いて集積する。

\end{itemize}

\noindent
2つ目のプログラムは

\begin{itemize}

\item
0から\( n \)を列挙する

\item
各整数のフィボナッチ数を求める

\item
フィルタを通して偶数を選択する

\item
初期値は空リストにて\code{cons}を用いて結果を集積する

\end{itemize}

\noindent
信号処理のエンジニアはこれらの処理をステージのカスケードを通して流れる信号を
用いて処理するという概念的説明を自然だと思われるでしょう。各ステージはプログラム設計の
部分を\link{Figure 2.7}に示すように実装しています。
\code{sum\-/odd\-/squares}では\newterm{enumerator}(\jnewterm{エニュメレータ})にて始めました。それは
与えられた木の葉から成る``信号''を生成します。この信号は\newterm{filter}(\jnewterm{フィルタ})を
通して奇数要素以外を全て取り除きます。残った信号は順に``変換器''である\newterm{map}を通し、
それが\code{square}手続を各要素に適用します。mapの出力は次に\newterm{accumulator}(\jnewterm{集積機})に
与えられ、それが要素を初期値0と\code{+}を用いて連結します。\code{even\-/fibs}の設計も同様です。



残念ながら上記の2つの手続の定義はこの信号の流れの構造を提示するのは失敗しています。
例えば\code{sum\-/odd\-/squares}を調べてみるとenumeration(列挙)は部分的に\code{null?}の\code{pair?}の
テストにて実装され、別の部分では手続の木再帰構造により実装されています。
同様に集積は部分的にテストの中に見つかり、また部分的に再帰中で使用される足し算に見つかり
ます。全体的にどちらの手続も信号の流れの記述内の要素に関連する明確な部分は存在しません。
2つの手続は演算を異なる方法で分解し、列挙をプログラム全体に広げてmap, filter, accumulation
に混ぜました。もしプログラムを手続中に信号処理構造の宣言を作るように構成できるのであれば
結果としてのコードの概念の明快さを増すことができるでしょう。

\begin{figure}[tb]
\phantomsection\label{Figure 2.7}
\centering
\begin{comment}
\begin{quote}
\heading{Figure 2.7:} The signal-flow plans for the
procedures \code{sum\-/odd\-/squares} (top) and \code{even\-/fibs} (bottom) reveal
the commonality between the two programs.

\begin{example}
+-------------+   +-------------+   +-------------+   +-------------+
| enumerate:  |-->| filter:     |-->| map:        |-->| accumulate: |
| tree leaves |   | odd?        |   | square      |   | +, 0        |
+-------------+   +-------------+   +-------------+   +-------------+

+-------------+   +-------------+   +-------------+   +-------------+
| enumerate:  |-->| map:        |-->| filter:     |-->| accumulate: |
| integers    |   | fib         |   | even?       |   | cons, ()    |
+-------------+   +-------------+   +-------------+   +-------------+
\end{example}
\end{quote}
\end{comment}
\includegraphics[width=111mm]{fig/chap2/Fig2.7d.pdf} 
\par\bigskip
\noindent
\heading{Figure 2.7:} 手続\code{sum\-/odd\-/squares}(上)と\code{even\-/fibs}(下)の信号の流れの計画が
二つのプログラムの間の共通点を明かす
\end{figure}

\subsubsection*{列命令}



プログラムを体系化し信号伝達構造をより明確に反映する鍵はある段階の処理から次へと流れる
``信号''に集中することです。もしこれらの信号をリストとして表現するなら、各段階の処理を
リスト操作を用いて実装できます。例えば信号伝達図のmapの段階を\link{Section 2.2.1}の\code{map}手続を
用いて実装できます。

\begin{scheme}
(map square (list 1 2 3 4 5))
~\textit{(1 4 9 16 25)}~
\end{scheme}

\noindent
列をフィルタリングして与えられた述語を満足する要素のみを選択することは以下の様に達成できます。

\begin{scheme}
(define (filter predicate sequence)
  (cond ((null? sequence) nil)
        ((predicate (car sequence))
         (cons (car sequence)
               (filter predicate (cdr sequence))))
        (else (filter predicate (cdr sequence)))))
\end{scheme}

\noindent
例として、

\begin{scheme}
(filter odd? (list 1 2 3 4 5))
~\textit{(1 3 5)}~
\end{scheme}

\noindent
集積は次のように実装します。

\begin{scheme}
(define (accumulate op initial sequence)
  (if (null? sequence)
      initial
      (op (car sequence)
          (accumulate op initial (cdr sequence)))))
(accumulate + 0 (list 1 2 3 4 5))
~\textit{15}~
(accumulate * 1 (list 1 2 3 4 5))
~\textit{120}~
(accumulate cons nil (list 1 2 3 4 5))
~\textit{(1 2 3 4 5)}~
\end{scheme}

\noindent
信号伝達図を実装するのに残っているもの全ては処理すべき要素の列を列挙することです。
\code{even\-/fibs}のためには与えられた区間の整数の列を生成しなければならず、以下のように
行うことができます。

\begin{scheme}
(define (enumerate-interval low high)
  (if (> low high)
      nil
      (cons low (enumerate-interval (+ low 1) high))))
(enumerate-interval 2 7)
~\textit{(2 3 4 5 6 7)}~
\end{scheme}

\noindent
木の葉を列挙するには、以下の様にして可能です。\footnote{これは実際には
\link{Exercise 2.28}の\code{fringe}手続そのものです。ここではその名を変えて
列操作手続一般に属するパーツであることを強調しています。}

\begin{scheme}
(define (enumerate-tree tree)
  (cond ((null? tree) nil)
        ((not (pair? tree)) (list tree))
        (else (append (enumerate-tree (car tree))
                      (enumerate-tree (cdr tree))))))
(enumerate-tree (list 1 (list 2 (list 3 4)) 5))
~\textit{(1 2 3 4 5)}~
\end{scheme}

\noindent
これで\code{sum\-/odd\-/squares}と\code{even\-/fibs}を信号伝達図の様に再形式化することができます。
\code{sum\-/odd\-/squares}のためには木の葉の列を列挙し、これをフィルタにかけ列の奇数のみを保持し、
各要素を二乗し、結果の合計を求めます。

\begin{scheme}
(define (sum-odd-squares tree)
  (accumulate
   + 0 (map square (filter odd? (enumerate-tree tree)))))
\end{scheme}

\noindent
\code{even\-/fibs}に対しては0から\( n \)の整数を列挙し、これらの整数のそれぞれに対する
フィボナッチ数を生成し、結果列をフィルタにかけ偶数の要素のみを保持し、結果をリストの
中に集積します。

\begin{scheme}
(define (even-fibs n)
  (accumulate
   cons
   nil
   (filter even? (map fib (enumerate-interval 0 n)))))
\end{scheme}

\noindent
列操作としての伝達プログラムの価値はこれがモジュラ形式のプログラムデザインを
行うことを手助けしてくれることにあります。モジュラであるとは相対的に独立した
部品を組み立てることで構築される設計です。柔軟な形でコンポーネントを
接続するための慣習的なインターフェイスと共に、標準コンポーネントのライブラリを
提供することで、モジュラ設計を推進することができます。



モジュラ構築は複雑性を工学的設計において複雑性をコントロールすることに対して強力な戦略です。
例えば現実の信号処理アプリケーションでは、設計者は恒常的にフィルタと変換器の標準化された
グループから選択された要素を繋げることでシステムを構築します。同様に列操作は標準的プログラム要素
を様々に組合せたライブラリを提供します。実例として私達は\code{sum\-/odd\-/squares}と\code{even\-/fibs}の手続の
部品を用いて、フィボナッチ数の最初から\( n + 1 \)個の二乗のリストを作成できます。


\begin{scheme}
(define (list-fib-squares n)
  (accumulate
   cons
   nil
   (map square (map fib (enumerate-interval 0 n)))))
(list-fib-squares 10)
~\textit{(0 1 1 4 9 25 64 169 441 1156 3025)}~
\end{scheme}

\noindent
部品を再配置し、列の奇数の二乗の積を計算するのに使うことも可能です。

\begin{scheme}
(define (product-of-squares-of-odd-elements sequence)
  (accumulate * 1 (map square (filter odd? sequence))))
(product-of-squares-of-odd-elements (list 1 2 3 4 5))
~\textit{225}~
\end{scheme}

\noindent
慣習的なデータ処理アプリケーションを列操作を用いて説明することもできます。
個人の記録の列があるとし、最も高給なプログラマの給料を見つけたいとします。
記録の給料を返すセレクタ\code{salary}と記録がプログラマの物であるかを判定する
述語\code{programmer?}があるとします。すると以下のように書けます。

\begin{scheme}
(define (salary-of-highest-paid-programmer records)
  (accumulate 
   max 0 (map salary (filter programmer? records))))
\end{scheme}

\noindent
この例は列操作として表わすことができる広範囲な処理のヒントを与えたにすぎません。\footnote{
Richard \link{Waters (1979)}は伝統的なFrotranプログラムを自動的に解析し、map, フィルタ,
集積を用いてそらを俯瞰するプログラムを開発しました。彼はFortranの科学サブルーチンパッケージの
コードの実に90\%がこのパラダイムにうまくはまることを発見しました。
Lispがプログラミング言語として成功した理由の1つにリストが順序有り集合を表すのに
標準的な手段を提供したことがあり、そのため高階手続を用いて操作することが可能になりました。
プログラミング言語APLはその力と魅力の多くを同様の選択のおかげで得ました。APLでは
全てのデータは配列として表現され、統一的、かつ便利な全ての種類の配列操作のための
包括的な命令集合が存在します。}



ここではリストとして実装された列は処理モジュールを接続することを可能にする
慣習的インターフェイスとしての役割を行います。その上、私達が構造を列として
統一的に表現した時、私達はプログラム中のデータ構造依存性を少ない数の列操作へと
局所化しました。これらを変更することで、プログラム設計を全体的に保存したまま
列の代替的表現方法を試みることができます。私達はこの能力を\link{Section 3.5}にて
列処理パラダイムを無限列を許可するよう一般化する時に利用します。

\begin{quote}
\heading{\phantomsection\label{Exercise 2.33}Exercise 2.33:} 
欠けた式を埋めて次の集積としてのいくつかの基本的なリスト操作命令の定義を完成させよ。

\begin{scheme}
(define (map p sequence)
  (accumulate (lambda (x y) ~\( \dark \langle \)~~\var{\dark ??}~~\( \dark \rangle \)~) nil sequence))
(define (append seq1 seq2)
  (accumulate cons ~\( \dark \langle \)~~\var{\dark ??}~~\( \dark \rangle \)~ ~\( \dark \langle \)~~\var{\dark ??}~~\( \dark \rangle \)~))
(define (length sequence)
  (accumulate ~\( \dark \langle \)~~\var{\dark ??}~~\( \dark \rangle \)~ 0 sequence))
\end{scheme}
\end{quote}

\begin{quote}
\heading{\phantomsection\label{Exercise 2.34}Exercise 2.34:} 
\( x \)の多項式を\( x \)の与えられた値にて評価することは集積として表すことができる。
以下の多項式を良く知られた\newterm{Horner's rule}(\jnewterm{ホーナー法})と呼ばれるアルゴリズムを用いて評価する。
\begin{comment}

\begin{example}
a_n x^n + a_(n-1) x^(n-1) + ... + a_1 x + a_0
\end{example}

\end{comment}
\begin{displaymath}
 a_n x^n + a_{n-1} x^{n-1} + \ldots + a_1 x + a_0 
\end{displaymath}
\noindent
ホーナー法は上記の計算を以下のような構造にする。

\begin{comment}

\begin{example}
(... (a_n x + a_(n-1)) x + ... + a_1) x + a_0
\end{example}

\end{comment}
\begin{displaymath}
 (\ldots (a_n x + a_{n-1}) x + \ldots + a_1) x + a_0. 
\end{displaymath}
\noindent
すなわち\( a_n \)で始め、\( x \)を掛け、\( a_{n-1} \)を足し、\( x \)を掛け、を\( a_0 \)
に到達するまで繰り返す。\footnote{\link{Knuth 1981}によるとこの方法はW. G. Hornerにより
19世紀始めに考案された。しかしその手法は実際にはニュートンにより100年を越えた前に使用されていた。
ホーナー法は多項式を最初に\( a_n x^n \)を計算し、\( a_{n-1}x^{n-1} \)を足すを繰り返す直接的な
方法より少ない回数の和と積を用いて評価する。実際に任意の多項式を評価するためのどんなアルゴリズムも
ホーナー法が必要な数と同じ数の和と積を使用する必要があることを証明することが可能である。
従ってホーナー法は多項式評価において最適なアルゴリズムである。
これは(和の数において)A. M. Ostrowskiによる1954年の論文にて証明され、これが現代の最適アルゴリズム
研究の基礎を築いた。同様の説明が積の数についてV. Y. Panにより1966年に証明された。
\link{Borodin and Munro (1975)}による本がこれらと他の最適アルゴリズムについての結果について概観
している。}


以下のテンプレートを埋めホーナー法を用いて多項式を評価する手続を作り出せ。
多項式の係数\( a_0 \)から\( a_n \)は列で用意されると想定せよ。

\begin{scheme}
(define (horner-eval x coefficient-sequence)
  (accumulate (lambda (this-coeff higher-terms) ~\( \dark \langle \)~~\var{\dark ??}~~\( \dark \rangle \)~)
              0
              coefficient-sequence))
\end{scheme}



例えば\( 1 + 3x + 5x^3 + x^5 \)を\( x = 2 \)の時の値を求める場合、
次のように評価を行う。

\begin{scheme}
(horner-eval 2 (list 1 3 0 5 0 1))
\end{scheme}
\end{quote}

\begin{quote}
\heading{\phantomsection\label{Exercise 2.35}Exercise 2.35:} 
\link{Section 2.2.2}の\code{count\-/leaves}を集積として再定義せよ。

\begin{scheme}
(define (count-leaves t)
  (accumulate ~\( \dark \langle \)~~\var{\dark ??}~~\( \dark \rangle \)~ ~\( \dark \langle \)~~\var{\dark ??}~~\( \dark \rangle \)~ (map ~\( \dark \langle \)~~\var{\dark ??}~~\( \dark \rangle \)~ ~\( \dark \langle \)~~\var{\dark ??}~~\( \dark \rangle \)~)))
\end{scheme}
\end{quote}

\begin{quote}
\heading{\phantomsection\label{Exercise 2.36}Exercise 2.36:} 
手続\code{accumulate\-/n}は\code{accumulate}に似ているが3番目の引数として列の列を取り、
その要素の列の長さは全て一定である。指定された集積手続を複数の列の最初の要素、二番目の要素、
以下繰り返し、を全て連結するため適用し、結果の列を返す。例えばもし\code{s}が4つの列を含む列、
\code{((1 2 3) (4 5 6) (7 8 9) (10 11 12))}である時、\code{(accumulate\-/n + 0 s)}の値は列
\code{(22 26 30)}にならなければならない。以下の\code{accumulate\-/n}の定義の欠けた式を補え。

\begin{scheme}
(define (accumulate-n op init seqs)
  (if (null? (car seqs))
      nil
      (cons (accumulate op init ~\( \dark \langle \)~~\var{\dark ??}~~\( \dark \rangle \)~)
            (accumulate-n op init ~\( \dark \langle \)~~\var{\dark ??}~~\( \dark \rangle \)~))))
\end{scheme}
\end{quote}

\begin{quote}
\heading{\phantomsection\label{Exercise 2.37}Exercise 2.37:}
ベクトル\( {\bf v} = (v_i) \)を数値の列として表現し、行列\( {\bf m} = (m_{ij}) \)を
ベクトル(行列の行)の列として表現するとする。例えば以下の行列は、
\begin{comment}

\begin{example}
+-         -+
|  1 2 3 4  |
|  4 5 6 6  |
|  6 7 8 9  |
+-         -+
\end{example}

\end{comment}
\begin{displaymath}
%  
% \left(\matrix{	1 & 2 & 3 & 4 \cr
% 		4 & 5 & 6 & 6 \cr
% 		6 & 7 & 8 & 9 \cr }\right) 
\left(
\begin{array}{cccc}
  1 & 2 & 3 & 4 \\
  4 & 5 & 6 & 6 \\
  6 & 7 & 8 & 9 
\end{array}
\right) 
\end{displaymath}
\noindent
列\code{((1 2 3 4) (4 5 6 6) (6 7 8 9))}として表現される。この表現と共に列操作を
用いることで簡潔に基本的な行列とベクトルの操作を表現することができる。これらの
操作は(行列演算のどんな本にも記述されている)次のものである。
\begin{comment}

\begin{example}
                                       __
(dot-product v w)      returns the sum >_i v_i w_i

(matrix-*-vector m v)  returns the vector t,
                                   __
                       where t_i = >_j m_(ij) v_j

(matrix-*-matrix m n)  returns the matrix p,
                                      __
                       where p_(ij) = >_k m_(ik) n_(kj)

(transpose m)          returns the matrix n,
                       where n_(ij) = m_(ji)
\end{example}

\end{comment}
\begin{displaymath}
%  
% \eqalign{ 
% 	\hbox{\tt (dot-product v w)} 		&\; {\rm returns\;the\;sum\;} 
% 		\Sigma_i v_i w_i, \cr
% 	\hbox{\tt (matrix-*-vector m v)} 	&\; {\rm returns\;the\;vector\;} {\bf t}, 
% 		\; {\rm where\;} t_i = \Sigma_j m_{ij} v_j, \cr
% 	\hbox{\tt (matrix-*-matrix m n)} 	&\; {\rm returns\;the\;matrix\;} {\bf p},
% 		\; {\rm where\;} p_{ij} = \Sigma_k m_{ik} n_{kj}, \cr
% 	\hbox{\tt (transpose m)} 			&\; {\rm returns\;the\;matrix\;} {\bf n}, 
% 		\; {\rm where\;} n_{ij} = m_{ji}. \cr
% } 
\begin{array}{rl}
	\hbox{\tt (dot-product v w)} 		&\; {\rm returns\;the\;sum\;} 
		\Sigma_i v_i w_i, \\
	\hbox{\tt (matrix-*-vector m v)} 	&\; {\rm returns\;the\;vector\;} \hbox{\bf t}, \\ 
		&\; {\rm where\;} t_i = \Sigma_j m_{ij} v_j, \\
	\hbox{\tt (matrix-*-matrix m n)} 	&\; {\rm returns\;the\;matrix\;} \hbox{\bf p}, \\
		&\; {\rm where\;} p_{ij} = \Sigma_k m_{ik} n_{kj}, \\
	\hbox{\tt (transpose m)} 			&\; {\rm returns\;the\;matrix\;} \hbox{\bf n}, \\
		&\; {\rm where\;} n_{ij} = m_{ji}.
\end{array}
\end{displaymath}
ドット積を次のように定義できる。\footnote{この定義は\link{Footnote 12}にて説明した
\code{map}の拡張バージョンを使用する}

\begin{scheme}
(define (dot-product v w)
  (accumulate + 0 (map * v w)))
\end{scheme}



以下の他の行列操作を演算するための手続の欠けた式を補え。(手続\code{accumulate\-/n}は
\link{Exercise 2.36}で定義されている。)

\begin{scheme}
(define (matrix-*-vector m v)
  (map ~\( \dark \langle \)~~\var{\dark ??}~~\( \dark \rangle \)~ m))
(define (transpose mat)
  (accumulate-n ~\( \dark \langle \)~~\var{\dark ??}~~\( \dark \rangle \)~ ~\( \dark \langle \)~~\var{\dark ??}~~\( \dark \rangle \)~ mat))
(define (matrix-*-matrix m n)
  (let ((cols (transpose n)))
    (map ~\( \dark \langle \)~~\var{\dark ??}~~\( \dark \rangle \)~ m)))
\end{scheme}
\end{quote}

\begin{quote}
\heading{\phantomsection\label{Exercise 2.38}Exercise 2.38:} 
\code{accumulate}手続はまた\code{fold\-/right}としても知られている。それが列の最初の要素と
右側の要素全てを結合した結果とを結合するためである。\code{fold\-/left}も存在し、\code{fold\-/right}と
似ているが、要素の結合を逆の向きに行う。

\begin{scheme}
(define (fold-left op initial sequence)
  (define (iter result rest)
    (if (null? rest)
        result
        (iter (op result (car rest))
              (cdr rest))))
  (iter initial sequence))
\end{scheme}


以下の式の値はいくらか。

\begin{scheme}
(fold-right / 1 (list 1 2 3))
(fold-left / 1 (list 1 2 3))
(fold-right list nil (list 1 2 3))
(fold-left list nil (list 1 2 3))
\end{scheme}



\code{op}が\code{fold\-/right}と\code{fold\-/left}にて同じ値を任意の列に対し生成することを保証するのに
必要な特性を答えよ。
\end{quote}

\begin{quote}
\heading{\phantomsection\label{Exercise 2.39}Exercise 2.39:} 
以下の\code{reverse}(\link{Exercise 2.18})の\code{fold\-/right}と\code{fold\-/left}(\link{Exercise 2.38})を用いた
定義を完成させよ。

\begin{scheme}
(define (reverse sequence)
  (fold-right (lambda (x y) ~\( \dark \langle \)~~\var{\dark ??}~~\( \dark \rangle \)~) nil sequence))
(define (reverse sequence)
  (fold-left (lambda (x y) ~\( \dark \langle \)~~\var{\dark ??}~~\( \dark \rangle \)~) nil sequence))
\end{scheme}
\end{quote}

\subsubsection*{入れ子のmap}



列のパラダイムを拡張し、一般的に入れ子ループを用いて表現される多くの演算を含めてみます。\footnote{
この入れ子マッピングへの取り組み方はDavid Turnerにより示されました。彼の言語であるKRCとMirandaは
これらの構成概念を取り扱うための洗練された形式主義を与えました。この節の例(また\link{Exercise 2.42}
も参照)は\link{Turner 1981}より翻案されました。\link{Section 3.5.3}ではこのやり方が無限長列に対しどのように
一般化されるかを学びます。}
次の問題について考えてみて下さい:正の整数\( n \)を与えられた時、異なる正の整数\( i \)と\( j \)の
全ての順序付けペアを見つけよ。条件として\( 1 \le j < i \le n \)、かつ\( i + j \)は素数である。
例として、もし\( n \)が6ならばペアは以下の通りである。

\begin{comment}

\begin{example}
  i   | 2 3 4 4 5 6 6
  j   | 1 2 1 3 2 1 5
------+---------------
i + j | 3 5 5 7 7 7 11
\end{example}

\end{comment}
\begin{displaymath}
\vbox{
\offinterlineskip
\halign{
\strut \hfil \quad #\quad \hfil & \vrule 
	\hfil \quad #\quad \hfil &
	\hfil \quad #\quad \hfil &
	\hfil \quad #\quad \hfil &
	\hfil \quad #\quad \hfil &
	\hfil \quad #\quad \hfil &
	\hfil \quad #\quad \hfil &
	\hfil \quad #\quad \hfil \cr

$i$ 	& 2 & 3 & 4 & 4 & 5 & 6 & 6 \cr
$j$ 	& 1 & 2 & 1 & 3 & 2 & 1 & 5 \cr
\noalign{\hrule}
$i + j$	& 3 & 5 & 5 & 7 & 7 & 7 & 11 \cr}
}
\end{displaymath}


この演算を体系化する自然な方法は全ての順序付けられた\( n \)以下の正の整数のペア
を生成し、フィルタを通してその合計が素数であるもののみを選択し、フィルタを通った
各ペア\( (i, j) \)に対し三つ組\( (i, j, i + j) \)を生成することです。



ここでペアの列を生成する方法を上げます:全ての整数\( i \le n \)に対し、
整数\( j < i \)を列挙し、全てのそのような\( i \)と\( j \)に対し、
ペア\( (i, j) \)を生成します。列操作を用いて、列\code{(enumerate\-/interval 1 n)}に
沿ってmapを行います。この列の各\( i \)に対し、列\code{(enumerate\-/interval 1 (- i 1))}に
沿ってmapを行います。この後者の列の\( j \)に対し、ペア\code{(list i j)}を生成します。
これが全ての\( i \)に対するペアの列を与えます。全ての\( i \)に対する全ての列を
(\code{append}を用いて集積することにより)接続することで要求されたペアの列を生成します。\footnote{
ここではペアを2つの要素のリストとして表現しており、Lispのペアとしてではありません。
従って``ペア''\( (i, j) \)は\code{(list i j)}であり、\code{(cons i j)}ではありません。}

\begin{scheme}
(accumulate
 append nil (map (lambda (i)
                   (map (lambda (j) (list i j))
                        (enumerate-interval 1 (- i 1))))
                 (enumerate-interval 1 n)))
\end{scheme}

\noindent
mapと集積の組み合わせを\code{append}と共に用いるのはこの種のプログラムにおいて
とても一般的ですので、これを分離したプログラムとして分けます。

\begin{scheme}
(define (flatmap proc seq)
  (accumulate append nil (map proc seq)))
\end{scheme}

\noindent
これでこのペアの列をフィルタにかけ和が素数であるものを探します。フィルタの述語が
各要素に対して呼ばれます。その引数はペアであり、ペアから整数を抽出せねばなりません。
従って列の各要素に適用される述語は以下のようになります。

\begin{scheme}
(define (prime-sum? pair)
  (prime? (+ (car pair) (cadr pair))))
\end{scheme}

\noindent
最後に、フィルタを通ったペア全体に以下の手続を用いてmapをかけた結果の列を生成します。
以下の手続は2つの要素のペアとそれらの和を用いて3つ組を構築します。

\begin{scheme}
(define (make-pair-sum pair)
  (list (car pair) (cadr pair) (+ (car pair) (cadr pair))))
\end{scheme}

\noindent
これらのステップ全てを接続すれば手続は完了です。

\begin{smallscheme}
(define (prime-sum-pairs n)
  (map make-pair-sum
       (filter prime-sum?
               (flatmap
                (lambda (i)
                  (map (lambda (j) (list i j))
                       (enumerate-interval 1 (- i 1))))
                (enumerate-interval 1 n)))))
\end{smallscheme}

\noindent
入れ子のmapは区間を列挙するもの以外の列に対しても便利です。ある集合\( S \)の全ての順列を
生成したいとします。つまり集合内の項目の全ての並べ方です。例えば\( \{1, 2, 3\} \)の順列は
\( \{1, 2, 3\} \), \( \{1, 3, 2\} \), \( \{2, 1, 3\} \), \( \{2, 3, 1\} \),
\( \{3, 1, 2\} \), and \( \{3, 2, 1\} \)です。
ここに集合\( S \)の順列を生成するための計画を上げます:\( S \)中の全て項目\( x \)に対し
再帰的に\( S - x \)の順列の列を生成し、\footnote{集合\( S - x \)は\( S \)の全ての要素から
\( x \)を除いた集合}次に\( x \)をそれぞれの先頭に置く。これは\( S \)の全ての\( x \)に対し
\( S \)の\( x \)で始まる順列の列を生成する。これらの全ての\( x \)に対する列を接続すると
\( S \)の全ての順列が与えられる。\footnote{Schemeのコードではセミコロンは\newterm{comments}(\jnewterm{コメント})
を書く場合に利用されます。セミコロンから始まり行末までの全てはインタプリタに無視されます。この本では
あまり多くのコメントを使用していません。私達はプログラムに対し説明的な名前を付けることで
それ自身がドキュメントであるかのように作るよう努力しています。}

\begin{scheme}
(define (permutations s)
  (if (null? s)              ~\textrm{; 集合は空か?}~
      (list nil)             ~\textrm{; 空集合を持つ列}~
      (flatmap (lambda (x)
                 (map (lambda (p) (cons x p))
                      (permutations (remove x s))))
               s)))
\end{scheme}

\noindent
この戦略がどのように\( S \)の順列を生成する問題から\( S \)よりもより少ない要素の
集合の順列生成の問題へと縮小しているかに注意して下さい。境界条件に関しては要素無しの
集合を表す空リストまで順に処理を繰り返します。空リストに対して\code{(list nil)}を
生成しました。これは1要素の列であり、要素無しの集合を表します。\code{permutations}内で
利用される\code{remove}手続は与えられた式から与えられた項目以外の全ての要素を返します。
これは簡単なフィルタにて表すことができます。


\begin{scheme}
(define (remove item sequence)
  (filter (lambda (x) (not (= x item)))
          sequence))
\end{scheme}

\begin{quote}
\heading{\phantomsection\label{Exercise 2.40}Exercise 2.40:} 
整数\( n \)を与えられ、ペア\( (i, j) \)を\( 1 \le j < i \le n \)の条件で
生成する手続\code{unique\-/pairs}を定義せよ。\code{unique\-/pairs}を用いて
上で与えられた\code{prime\-/sum\-/pairs}の定義をより簡単にせよ。
\end{quote}

\begin{quote}
\heading{\phantomsection\label{Exercise 2.41}Exercise 2.41:} 
与えられた整数\( n \)以下でかつ合計が与えられた整数\( s \)である、全ての異なる正の整数
\( i \), \( j \), \( k \)の順序有りの3つ組を求める手続を書け。
\end{quote}

\begin{quote}
\heading{\phantomsection\label{Exercise 2.42}Exercise 2.42:} 

\begin{figure}[tb]
\phantomsection\label{Figure 2.8}
\centering
\begin{comment}
\heading{Figure 2.8:} A solution to the eight-queens puzzle.

\begin{example}
+---+---+---+---+---+---+---+---+
|   |   |   |   |   | Q |   |   |
+---+---+---+---+---+---+---+---+
|   |   | Q |   |   |   |   |   |
+---+---+---+---+---+---+---+---+
| Q |   |   |   |   |   |   |   |
+---+---+---+---+---+---+---+---+
|   |   |   |   |   |   | Q |   |
+---+---+---+---+---+---+---+---+
|   |   |   |   | Q |   |   |   |
+---+---+---+---+---+---+---+---+
|   |   |   |   |   |   |   | Q |
+---+---+---+---+---+---+---+---+
|   | Q |   |   |   |   |   |   |
+---+---+---+---+---+---+---+---+
|   |   |   | Q |   |   |   |   |
+---+---+---+---+---+---+---+---+
\end{example}
\end{comment}
\includegraphics[width=48mm]{fig/chap2/Fig2.8c.pdf}      % 73mm
\par\bigskip
\noindent
\heading{Figure 2.8:} 8クイーンパズルの解の一例
\end{figure}

``8クイーンパズル''は8つのクイーンをチェス盤の上に、どのクイーンも他のクイーンを
取ることができないようにするにはどのように置くかを尋ねる。(これはつまりどの2つのクイーンも
同じ列、行、または斜めの線上に有ってはならないということである)。
考えられる解の1つを\link{Figure 2.8}に示す。このパズルを解く1つの方法は盤上に渡って
各列にクイーンを置く。\( k - 1 \)個のクイーンを置いたら\( k \)番目のクイーンは
既に盤上に置いてあるどのクイーンも取れない場所に置かなければならない。この取り組み方を
再帰的に形式化できる:盤上の最初の\( k - 1 \)列内の\( k - 1 \)個のクイーンの可能な置き方
全ての列を既に生成したと想定する。これら全ての方法に対し拡張した位置の拡張集合を
\( k \)番目の列の各行にクイーンを置くことで生成する。次にこれらをフィルタにかけて
\( k \)番目の列のクイーンが他のクイーンを考慮しても安全な位置のみを保持する。
これは\( k \)個のクイーンを最初の\( k \)列内に置く全ての方法を生成する。
この過程を繰り返すことで1つの解答のみでなく、このパズルの全ての解答を生成できる。


この解法を手続\code{queens}として実装した。\( n \)個のクイーンを\( n \times n \)の
チェス盤上に置く問題に対する全ての解の列を返す。\code{queens}は内部手続\code{queen\-/cols}を
持ち、それは盤の最初の\( k \)列中にクイーンを置く全ての方法の列を返す。

\begin{scheme}
(define (queens board-size)
  (define (queen-cols k)
    (if (= k 0)
        (list empty-board)
        (filter
         (lambda (positions) (safe? k positions))
         (flatmap
          (lambda (rest-of-queens)
            (map (lambda (new-row)
                   (adjoin-position new-row 
                                    k 
                                    rest-of-queens))
                 (enumerate-interval 1 board-size)))
          (queen-cols (- k 1))))))
  (queen-cols board-size))
\end{scheme}



この手続の中で、\code{rest\-/of\-/queens}は最初の\( k - 1 \)列内に\( k - 1 \)個のクイーンを
置く方法であり、\code{new\-/row}は\code{k}番目の列に対してクイーンを置くように提案された行である。
盤上の位置の集合に対する表現を、新しい列の位置を位置の集合に付け足す手続\code{adjoin\-/position}と
位置の空集合を表す\code{empty\-/board}を含めて実装することでプログラムを完成させよ。
あなたは\code{k}番目の列にあるクイーンが他に対して安全であるかどうかを位置の集合に対して決定する
\code{safe?}もまた書かなければならない。(新しいクイーンが安全であるかどうかのみをチェックする
必要であることに注意すること---他のクイーンは既にお互いに安全であることが保証されている)。
\end{quote}

\begin{quote}
\heading{\phantomsection\label{Exercise 2.43}Exercise 2.43:} 
Louis Reasonerは\link{Exercise 2.42}を行うことで酷い時間を過している。彼の\code{queens}手続は
動いているように見える。しかし実行がとても遅い。(Louisは\( 6\times6 \)の場合でさえそれを
解くのにかかる長い時間を待つことができなかった)。LouisがEva Lu Atorに助けを求めた時、
彼女はLouisが\code{flatmap}内の入れ子のマッピングの順を交換してしまったことを指摘した。
以下のように書いていた。

\begin{scheme}
(flatmap
 (lambda (new-row)
   (map (lambda (rest-of-queens)
          (adjoin-position new-row k rest-of-queens))
        (queen-cols (- k 1))))
 (enumerate-interval 1 board-size))
\end{scheme}

この交換がなぜプログラムの実行を遅くするのか説明せよ。Louisのプログラムが8クイーンパズルを
解くのにどれだけの時間がかかるか推察せよ。\link{Exercise 2.42}のプログラムが同じパズルを解く
のに必要な時間が\( T \)であるとの前提で行え。
\end{quote}

\subsection{例: ピクチャー言語}
\label{Section 2.2.4}


この節では絵を描く簡単な言語をお見せします。これがデータ抽象と閉包の力を図示し、
また高階手続を本質的な方法で利用します。この言語は\link{Figure 2.9}のようなパターンを
試験することを簡単にするように設計されております。この図は要素が移動し、縮小しを
繰り返しながら組み立てられています。\footnote{ピクチャー言語はPeter Hendersonが
作成した言語を基にしており、この言語はM.C. Escherの木版画``Square Limit''(\link{Henderson 1982}参照)
のようなイメージを構築するために作成されました。その木版画は繰り返しサイズが変更された
パターンが組込まれており、この節の\code{square\-/limit}手続を用いて描かれた配置と
似ております。}
この言語内では、組み立てられるデータオブジェクトはリスト構造ではなく手続として
表現されます。閉包の特性を見たす\code{cons}が簡単に自由に複雑なリスト構造を構築できるように、
この言語内の命令もまた閉包の特性を満たし、簡単に自由に複雑なパターンを構築できます。

\subsubsection*{この本のピクチャー言語}



\link{Section 1.1}でプログラミングの学習を始めたとき、言語のプリミティブ、その組み合わせの手段、
抽象化の手段に集中することが言語の説明の重要性だと強調しました。
ここではその枠組みに従います。

\begin{figure}[tb]
\phantomsection\label{Figure 2.9}
\centering
\begin{comment}
\heading{Figure 2.9:} Designs generated with the picture language.

[two graphic images not included]
\end{comment}
\includegraphics[width=111mm]{fig/chap2/Fig2.9-bigger.png}
\par\bigskip
\noindent
\heading{Figure 2.9:} ピクチャー言語を用いて生成したデザイン
\end{figure}

\begin{figure}[tb]
\phantomsection\label{Figure 2.10}
\centering
\begin{comment}
\heading{Figure 2.10:} Images produced by the \code{wave}
painter, with respect to four different frames.  The frames, shown with dotted
lines, are not part of the images.

[four graphic images not included]
\end{comment}
\includegraphics[width=50mm]{fig/chap2/Fig2.10.pdf}
\par\bigskip
\noindent
\heading{Figure 2.10:} \code{wave}により生成されたイメージ
\end{figure}

\noindent
このピクチャー言語の優雅さの部分は要素の種類が\newterm{painter}(\jnewterm{ペインタ})と呼ばれる
ものたった1つしかないことです。ペインタは指定された平行四辺形の枠の中にイメージを
移動し、拡大縮小して描きます。例えば\code{wave}と呼ぶペインタがありそれは\link{Figure 2.10}
に見られるような粗野な線の絵を描きます。実際の絵の下腿はフレームに依存します---\link{Figure 2.10}の
4つの絵全ては同じ\code{wave}ペインタにより生成されていますが、4つの異なるフレームを
考慮しています。ペインタはこれよりもより複雑にすることが可能です。\code{rogers}と
呼ばれるプリミティブなペインタは\acronym{MIT}の創始者であるWilliam Barton Rogersの
絵を\link{Figure 2.11}に示されるように描きます。\footnote{
William Barton Rogers (1804-1882)は
\acronym{MIT}の創始者であり、かつ初代学長です。地質学者であり、才能溢れる教師である彼は
William and Mary CollegeとUniversity of Virginiaにて教鞭を取りました。1859年に
彼はボストンへ移りそこでより研究に打ち込み、``技術専門の研究所''を設立する計画を
進めました。またマサチューセッツ州の最初のガスメータの州検査官も務めました。




\acronym{MIT}が1861年に創設された時、Rogersは最初の学長に選ばれました。
Rogersは``useful learning''(実用的学習)の活用を信奉しました。これは当時の
大学教育からは異なるものでした。彼に依れば古典の過度の強調が``より幅広く、
高く、より現実的な教育と自然科学、及び社会科学の前に立ち塞がっている''と書いています。
同様に彼の教育は職業専門学校の狭い教育からも異なるものになろうとしていました。

\begin{quote}
実利的であることと科学実務者の間の区別を強制する世界は全く無益だ。現代の経験全てが
その完全な無益さを示している。
\end{quote}



Rogersは\acronym{MIT}の学長を健康上の理由で辞任する1870年まで務めました。
1878年に二代目の\acronym{MIT}学長John Runkleは1873年からの大不況によりもたらさ
れた財政危機のプレッシャーとHarvardによる\acronym{MIT}の買収の試みに対する抵抗
の緊張により辞任しました。
Rogerは学長のオフィスを支えるため1881年まで戻りました。



Rogersは1882年、\acronym{MIT}の大学院の卒業試験に取り組む最中に倒れ、亡くなられました。
RunkleがRogerの最後の言葉を同年に送られた弔事から引用しています。

\begin{quote}
``本日ここに立ち本校とは何であるかを考えると\( \dots \) 科学の始まりを思い受かべます。
150年前にStephen Halesは灯用のガスを主題にした小論文を発表しました。その中で彼は
彼の研究が128グレインの瀝青炭-- '' ``瀝青炭'' これが地上での彼の最後の言葉でした。
ここで彼は体を前に曲げ、彼の前のテーブルの上にある端書を確認するようでいて、そして
ゆっくりと直立した体制を取り戻し、両腕を上げ、そして
彼の地上の労働と業績の場面から``死の明日''へと形を変えたのです。そこでは人生の謎は
解決され、肉体から解放された魂は新しく、未だ測りしれない無限の未来の謎を熟考する
ことに終りの無い充足を見つけるのです。
\end{quote}

Francis A. Walker(\acronym{MIT}の三代目の学長)の言葉では

\begin{quote}
彼自身が負うた彼の人生全ては最も誠実で雄々しく、そして彼は騎士が心から望んだか
のごとく、仕事中に、その役職のまま、公務の行いの最中に亡くなった。
\end{quote}
} \link{Figure 2.11}の4つのイメージは\link{Figure 2.10}の\code{wave}のイメージと
同じ4つのフレームを考慮して描かれています。

\begin{figure}[tb]
\phantomsection\label{Figure 2.11}
\centering
\begin{comment}
\heading{Figure 2.11:} Images of William Barton Rogers,
founder and first president of \acronym{MIT}, painted with respect to the same
four frames as in \link{Figure 2.10} (original image reprinted with the
permission of the \acronym{MIT} Museum).

[four graphic images not included]
\end{comment}
\includegraphics[width=48mm]{fig/chap2/Fig2.11.pdf}
\par\bigskip
\noindent
\heading{Figure 2.11:} William Barton Rogersのイメージ
\end{figure}

\noindent
イメージを結合するには与えられたペインタから新しいペインタを構築する種々の命令を
使用します。例えば\code{beside}命令は2つのペインタを取り新しい最初のペインタのイメージを
フレームの左半分に、2つ目のペインタのイメージをフレームの右半分に描く複合ペインタを
生成します。同様に\code{below}は2つのペインタを取り、1つ目のペインタのイメージを2つ目の
ペインタのイメージの下に描きます。いくつかの命令は単一のペインタを変換し新しいペインタを
生成します。例えば\code{flip\-/vert}はペインタを取りそのイメージを上下逆さに描くペインタを
生成し、\code{flip\-/horiz}は元のペインタのイメージを左右逆に描くペインタを生成します。


\link{Figure 2.12}は\code{wave4}を呼んだペインタの描画を見せており、これは\code{wave}で始め
2段階を経て構築されています。

\begin{scheme}
(define wave2 (beside wave (flip-vert wave)))
(define wave4 (below wave2 wave2))
\end{scheme}

\begin{figure}[tb]
\phantomsection\label{Figure 2.12}
\centering
\begin{comment}
\heading{Figure 2.12:} Creating a complex figure, starting
from the \code{wave} painter of \link{Figure 2.10}.

[two graphic images not included]

\begin{scheme}
(define wave2                      (define wave4
  (beside wave (flip-vert wave)))    (below wave2 wave2))
\end{scheme}
\end{comment}
\includegraphics[width=50mm]{fig/chap2/Fig2.12.pdf}
\par\bigskip
\noindent
\heading{Figure 2.12:} \link{Figure 2.10}の\code{wave}ペインタから始めて複雑な図を作成する
\end{figure}

\noindent
複雑なイメージをこの様式で構築する場合はペインタが言語の接続手段の下で閉じている
という事実を利用しています。2つのペインタの\code{beside}や\code{below}はそれ自身がペインタです。
従ってそれをより複雑なペインタを作るための要素として使用できます。
\code{cons}を用いてリスト構造を構築するのと同様に、結合手段の下のデータの閉包は
ほんのわずかな命令を用いて複雑な構造を作成する能力にとって重大です。



ペインタを結合できれば直ぐに、ペインタを接続する典型的なパターンを抽象化できるように
なりたいと願うでしょう。ペインタ操作をSchemeの手続として実装することにします。それは
私達がピクチャー言語内のメカニズムとして専用の抽象化を必要としないことを意味します。
接続の手段が普通のSchemeの手続ですから、手続の範囲で行えるペインタの操作を用いて、何でもできる
能力が自動的に得られます。例えば\code{wave4}内のパターンを抽象化できます。

\begin{scheme}
(define (flipped-pairs painter)
  (let ((painter2 (beside painter (flip-vert painter))))
    (below painter2 painter2)))
\end{scheme}

\noindent
そして\code{wave4}をこのパターンのインスタンスとして定義します。

\begin{scheme}
(define wave4 (flipped-pairs wave))
\end{scheme}

\begin{figure}[tb]
\phantomsection\label{Figure 2.13}
\centering
\begin{comment}
\begin{quote}
\heading{Figure 2.13:} Recursive plans for \code{right\-/split} and \code{corner\-/split}.

\begin{example}
+-------------+-------------+    +------+------+-------------+
|             |             |    | up-  | up-  |             |
|             | right-split |    | split| split| corner-split|
|             |             |    |      |      |             |
|             |     n-1     |    |  n-1 |  n-1 |     n-1     |
|             |             |    |      |      |             |
|  identity   +-------------+    +------+------+-------------+
|             |             |    |             | right-split |
|             | right-split |    |             |     n-1     |
|             |             |    |  identity   +-------------+
|             |     n-1     |    |             | right-split |
|             |             |    |             |     n-1     |
+-------------+-------------+    +-------------+-------------+

       right-split n                    corner-split n
\end{example}
\end{quote}
\end{comment}
\includegraphics[width=111mm]{fig/chap2/Fig2.13a.pdf}
\begin{quote}
\heading{Figure 2.13:} \code{right\-/split}と\code{corner\-/split}の再帰計画
\end{quote}
\end{figure}

\noindent
また再帰命令を定義することも可能です。以下はペインタを分割し、\link{Figure 2.13}に示すように
右へ向けて枝分かれします。

\begin{scheme}
(define (right-split painter n)
  (if (= n 0)
      painter
      (let ((smaller (right-split painter (- n 1))))
        (beside painter (below smaller smaller)))))
\end{scheme}

\noindent
右に向けてと同じように上方向にも枝分かれすることでバランスの取れたパターンを生成することも可能です。
(課題\link{Exercise 2.44}と図\link{Figure 2.13}と\link{Figure 2.14}を参照して下さい)。

\begin{scheme}
(define (corner-split painter n)
  (if (= n 0)
      painter
      (let ((up (up-split painter (- n 1)))
            (right (right-split painter (- n 1))))
        (let ((top-left (beside up up))
              (bottom-right (below right right))
              (corner (corner-split painter (- n 1))))
          (beside (below painter top-left)
                  (below bottom-right corner))))))
\end{scheme}

\noindent
\code{corner\-/split}の4つのコピーを置くことで\code{square\-/limit}と呼ばれるパターンを
獲得することができ、\code{wave}と\code{rogers}に対する適用が\link{Figure 2.9}に
示されます。

\begin{scheme}
(define (square-limit painter n)
  (let ((quarter (corner-split painter n)))
    (let ((half (beside (flip-horiz quarter) quarter)))
      (below (flip-vert half) half))))
\end{scheme}

\begin{quote}
\heading{\phantomsection\label{Exercise 2.44}Exercise 2.44:} 
\code{corner\-/split}にて使用された手続\code{up\-/split}を定義せよ。\code{right\-/split}に
似ているが、\code{below}と\code{beside}の役割を入れ替える。
\end{quote}

\begin{figure}[tbp]
\phantomsection\label{Figure 2.14}
\centering
\begin{comment}
\heading{Figure 2.14:} The recursive operations \code{right\-/split} and \code{corner\-/split} applied to the painters \code{wave} and \code{rogers}.  Combining four \code{corner\-/split} figures produces symmetric \code{square\-/limit} designs as shown in \link{Figure 2.9}.

[two graphic images not included]

\begin{scheme}
(right-split wave 4)         (right-split rogers 4)
\end{scheme}

[two graphic images not included]

\begin{scheme}
(corner-split wave 4)        (corner-split rogers 4)
\end{scheme}
\end{comment}
\includegraphics[width=91mm]{fig/chap2/Fig2.14b.pdf}
\begin{quote}
\heading{Figure 2.14:} ペインタ\code{wave}と\code{rogers}に適用された再帰命令\code{right\-/split}と\code{corner\-/split}。
4つの図\code{corner\-/split}を組み合わせることで\link{Figure 2.9}で示された対照的な\code{square\-/limit}のデザインを生成する。
\end{quote}
\end{figure}

\subsubsection*{高階命令}



ペインタ命令のパターンを抽象化するのに加えて、より高いレベルのペインタ命令接続の
抽象化パターンについて取り組むことができます。
それはペインタ命令を操作を行うための要素---ペインタ命令を引数として
取り新しいペインタ命令を作成する手続と見做し、そしてこれらの要素のための
組み合わせの手段の記述が可能だということです。



例として、\code{flipped\-/pairs}と\code{square\-/limit}はペインタのイメージを四角のパターン内にて
4つのコピーをそれぞれが準備します。それらはどのような位置と向きに置くかということのみ
において異なります。このペインタ接続のパターンを抽象化する1つの方法は以下のプロシジャを
用いて、4つの1引数ペインタ命令を取り与えられたペインタをそれら4つの命令で変換する命令を
生成し、結果を四角の中に配置します。\code{tl}, \code{tr}, \code{bl}, and \code{br}は左上、右上、
左下、右下のコピーに対応する変換です。

\begin{scheme}
(define (square-of-four tl tr bl br)
  (lambda (painter)
    (let ((top (beside (tl painter) (tr painter)))
          (bottom (beside (bl painter) (br painter))))
      (below bottom top))))
\end{scheme}

\noindent
すると\code{flipped\-/pairs}は\code{square\-/of\-/four}を以下のように用いて定義可能です。\footnote{
同等に、こうも書けます。

\begin{smallscheme}
(define flipped-pairs
  (square-of-four identity flip-vert identity flip-vert))
\end{smallscheme}
}

\begin{scheme}
(define (flipped-pairs painter)
  (let ((combine4 (square-of-four identity flip-vert
                                  identity flip-vert)))
    (combine4 painter)))
\end{scheme}

\noindent
そして\code{square\-/limit}は以下の様に表現可能です。\footnote{\code{Rotate180}は
ペインタを180度回転します(\link{Exercise 2.50}参照)。\code{rotate180}の代わりに
\code{(compose flip\-/vert flip\-/horiz)}と言うこともできます。\code{compose}手続は
\link{Exercise 1.42}から使用しました。}

\begin{scheme}
(define (square-limit painter n)
  (let ((combine4 (square-of-four flip-horiz identity
                                  rotate180 flip-vert)))
    (combine4 (corner-split painter n))))
\end{scheme}

\begin{quote}
\heading{\phantomsection\label{Exercise 2.45}Exercise 2.45:} 
\code{right\-/split}と\code{up\-/split}は一般的な分割命令のインスタンスだと言うことができる。
手続\code{split}を以下の式を評価する場合に、

\begin{scheme}
(define right-split (split beside below))
(define up-split (split below beside))
\end{scheme}

\noindent
既に定義住みのものと全く同じ振舞を行う手続\code{right\-/split}と\code{up\-/split}を
生成するよう定義せよ。
\end{quote}


\subsubsection*{フレーム}


ペインタとどのように実装しその接続手段を示す前に、始めにフレームについて考えなけ
ればなりません。フレームは3つのベクトル---始点ベクトルと2つの辺ベクトルで説明でき
ます。始点ベクトルは平面上においてある絶対的な始点からフレームの始点までのオフセットを
指定します。そして辺ベクトルは始点から角までのオフセットを指定します。もし2つの辺が
垂直であればフレームは長方形になります。それ以外ではフレームはより一般的な平行四辺形に
なります。


\link{Figure 2.15}はフレームとその対応するベクトルを示します。
データ抽象化に従い、まだフレームがどのように表現されるかについては、
3つのベクトルを取りフレームを生成するコンストラクタ\code{make\-/frame}と関連する3つの
セレクタ\code{origin\-/frame}, \code{edge1\-/frame}, \code{edge2\-/frame}が存在すること以外を
特定する必要がありません。(\link{Exercise 2.47}を参照して下さい)。

\begin{figure}[tb]
\phantomsection\label{Figure 2.15}
\centering
\begin{comment}
\heading{Figure 2.15:} A frame is described by three vectors
-- an origin and two edges.

\begin{example}
                         __
                     __--  \
                 __--       \
      __     __--            \   __
     |\  __--                 \__-|
       \-                  __--
frame   \              __--
edge2    \         __--    frame
vector    \    __--        edge1
           \_--            vector
            -   <--+
          frame    |
          origin   +-- (0, 0) point
          vector       on display screen
\end{example}
\end{comment}
\includegraphics[width=51mm]{fig/chap2/Fig2.15a.pdf}
\begin{quote}
\heading{Figure 2.15:} 3つのベクトル --- 始点と2つの辺により記述されたフレーム
\end{quote}
\end{figure}

\noindent
私達は単位正方形\( (0 \le x, y \le 1) \)内の座標をイメージを指定するのに用いることにします。
各フレームは、フレームに適合するようにイメージの移動と拡大縮小をするのに使われる
\newterm{frame coordinate map}(\jnewterm{フレーム座標マップ})に関連付けられます。
マップは単位正方形をベクトル\( {\bf v} = (x, y) \)を次のベクトルの和にマッピングすることで
変換します。

\begin{comment}

\begin{example}
Origin(Frame) + x * Edge_1(Frame) + y * Edge_2(Frame)
\end{example}

\end{comment}
\begin{displaymath}
 {\rm Origin(Frame)} + x \cdot {\rm Edge_1(Frame)} + y \cdot {\rm Edge_2(Frame)}. 
\end{displaymath}
\noindent
例えば、(0, 0)はフレームの始点に、(1, 1)は対角線上に始点の反対の頂点へ、そして
(0.5, 0.5)はフレームの中心点にマッピングされます。フレーム座標マップは以下の手続により
作成できます。\footnote{\code{frame\-/coord\-/map}はこの先の\link{Exercise 2.46}にて説明される
ベクトル操作を用います。ここでは何らかのベクトルの表現を用いて実装済みと仮定します。
データ抽象化のおかげでこのベクトルの表現がどんなものかは、ベクトル操作が正しく振る舞われる
限りにおいて問題にはなりません。}

\begin{scheme}
(define (frame-coord-map frame)
  (lambda (v)
    (add-vect
     (origin-frame frame)
     (add-vect (scale-vect (xcor-vect v)
                           (edge1-frame frame))
               (scale-vect (ycor-vect v)
                           (edge2-frame frame))))))
\end{scheme}

\noindent
\code{frame\-/coord\-/map}をフレームに適用すると、ベクトルを取りベクトルを返す手続を返すことに
注意して下さい。もし引数ベクトルが単位正方形の中なら、結果のベクトルはフレームの範囲内に
なります。例として、

\begin{scheme}
((frame-coord-map a-frame) (make-vect 0 0))
\end{scheme}

\noindent
は以下のベクトルと同じものを返します。

\begin{scheme}
(origin-frame a-frame)
\end{scheme}

\begin{quote}
\heading{\phantomsection\label{Exercise 2.46}Exercise 2.46:} 
始点からある点へと走る2次元ベクトル\( \bf v \)は\( x \)-座標と\( y \)-座標から成る
ペアにより表現できる。ベクトルに対するデータ抽象をコンストラクタ\code{make\-/vect}と関連する
セレクタ\code{xcor\-/vect}と\code{ycor\-/vect}を与えることにより実装せよ。
セレクタとコンストラクタを用いてベクトルの足し算、引き算、スカラによる乗算を求める操作を実行する
手続 \code{add\-/vect}, \code{sub\-/vect}, \code{scale\-/vect}を実装せよ。

\begin{comment}

\begin{example}
(x_1, y_1) + (x_2, y_2) = (x_1 + x_2, y_1 + y_2)
(x_1, y_1) - (x_2, y_2) = (x_1 - x_2, y_1 - y_2)
             s * (x, y) = (sx, sy)
\end{example}
\end{comment}
\begin{displaymath}
%  
% \eqalign{
% 	(x_1, y_1) + (x_2, y_2) 	&= (x_1 + x_2, y_1 + y_2), \cr
% 	(x_1, y_1) - (x_2, y_2) 	&= (x_1 - x_2, y_1 - y_2), \cr
% 	s \cdot (x, y) 			&= (sx, sy). \cr }
% 
\begin{array}{r@{{}={}}l}
	(x_1, y_1) + (x_2, y_2) 	& (x_1 + x_2, y_1 + y_2), \\
	(x_1, y_1) - (x_2, y_2) 	& (x_1 - x_2, y_1 - y_2), \\
	s \cdot (x, y) 			& (sx, sy). 
\end{array}
\end{displaymath}
\end{quote}


\begin{quote}
\heading{\phantomsection\label{Exercise 2.47}Exercise 2.47:} 
ここに2つの有り得そうなフレームのコンストラクタがある

\begin{scheme}
(define (make-frame origin edge1 edge2)
  (list origin edge1 edge2))
(define (make-frame origin edge1 edge2)
  (cons origin (cons edge1 edge2)))
\end{scheme}


各コンストラクタに適切な、フレームに対応する実装を生成するセレクタを提供せよ。
\end{quote}

\subsubsection*{ペインタ}


ペインタはフレームを引数として与え、特定のイメージをフレームにはまるように移動、拡大縮小
して描く手続として表現される。すなわちもし\code{p}がペインタで\code{f}がフレームである場合、
\code{f}を引数として\code{p}を呼び出すことで\code{f}の中に\code{p}のイメージを生成する。



プリミティブなペインタがどのように実装されているかの詳細は特定のグラフィックシステムの
特質と描画されるイメージのタイプに依存します。例えばスクリーン上の2つの指定された点の間に
線を引く手続\code{draw\-/line}があると想定します。すると線分のリストから線を引くためのペインタ、
例えば\link{Figure 2.10}の\code{wave}ペインタのようなものを以下のように作ることができます。\footnote{
\code{Segments\-/>painter}は線分の表現に下記の\link{Exercise 2.48}で説明されたものを使っています。
また\link{Exercise 2.23}で説明された\code{for\-/each}を使っています。}

\begin{scheme}
(define (segments->painter segment-list)
  (lambda (frame)
    (for-each
     (lambda (segment)
       (draw-line
        ((frame-coord-map frame) 
         (start-segment segment))
        ((frame-coord-map frame)
         (end-segment segment))))
     segment-list)))
\end{scheme}

\noindent
線分は単位正方形に対しての座標を用いて与えられます。リスト中の各線分に対して
ペインタは線分の終端をフレーム座標マップを用いて変換し、変換後の点の間に線を引きます。



手続としてペインタを表現することは強力な抽象化バリアをピクチャーランゲージの中に
確立します。私達は全ての種類のプリミティブなペインタを種々のグラフィック機能の基盤の
上に作り、混ぜることができます。それらの実装の詳細は問題ではありません。
フレームを引数として取りフレームに適切なサイズにスケールして何かを描く
任意の手続がペインタの役を演じることができます。\footnote{
例えば\link{Figure 2.11}の\code{rogers}ペインタはグレーレベルのイメージから構築されています。
与えられたフレームの中の各点に対し\code{rogers}ペインタはイメージ中のマッピングされる位置を
フレーム座標マップの下に決定し、適切に影を付けます。異なるタイプのペインタを許可することで、
\link{Section 2.1.3}で議論された分数表現は適切な条件を満たせば全く任意でかまわないという抽象データの
考えからより大きな利点を得ています。ここではペインタは指定されたフレーム内に何かを描くので
あれば全くどのように実装されても構わないという事実を用いています。\link{Section 2.1.3}はまた
ペアがどのように手続として実装され得るかということも示しました。
ペインタはデータに対する手続表現の二つ目の例です。}

\begin{quote}
\heading{\phantomsection\label{Exercise 2.48}Exercise 2.48:} 
平面上で方向を持つ線分はベクトルのペア---原点から
線分の始点へと向かうベクトルと原点から線分の終点へと向かうベクトルとして
表現可能だ。\link{Exercise 2.46}のベクトル表現を用いて、コンストラクタ\code{make\-/segment}と
セレクタ\code{start\-/segment}と\code{end\-/segment}を持つ線分表現を定義せよ。
\end{quote}

\begin{quote}
\heading{\phantomsection\label{Exercise 2.49}Exercise 2.49:} 
\code{segments\-/>painter}を用いて以下のプリミティブなペインタを定義せよ。

\begin{enumerate}[a]

\item
指定したフレームの外枠を描くペインタ

\item
フレームの反対の角を繋いで``X''を描くペインタ

\item
フレームの辺の中点を結んでダイヤモンドの形を描くペインタ

\item
\code{wave}ペインタ

\end{enumerate}
\end{quote}

\subsubsection*{変形とペインタの組み合わせ}



ペインタに対する操作(例えば\code{flip\-/vert}や\code{beside})はフレーム引数に由来する
フレームに対して元のペインタを実行するペインタを作成することで働いています。
従って例えば\code{flip\-/vert}は引っくり返す場合にもそれがどのように描かれるのかは
知る必要がありません---ただフレームをどのように引っくり返すのか知る必要がある
のみです。逆転したペインタはただ元のペインタを用いますが、フレームは逆転されて
いるのです。



ペインタ操作は\code{transform\-/painter}手続を基にしており、それはペインタとどのように
フレームを変換するかの情報を引数に取り、新しいペインタを生成します。変換されたペインタは
フレーム上にて呼ばれた時に、フレームを変換して基のペインタを変換済みのフレーム上で
呼び出します。\code{transform\-/painter}に対する引数は新しいフレームの角を指定する
(ベクトルとして表現される)複数の点です。フレームにマッピングされる時、最初の点は
新しいフレームの始点を指定し、他の2つのは辺ベクトルの終点を指定します。従って、
単位正方形内の引数は元のフレームの中に含まれるフレームを指定します。

\begin{scheme}
(define (transform-painter painter origin corner1 corner2)
  (lambda (frame)
    (let ((m (frame-coord-map frame)))
      (let ((new-origin (m origin)))
        (painter (make-frame 
                  new-origin
                  (sub-vect (m corner1) new-origin)
                  (sub-vect (m corner2) new-origin)))))))
\end{scheme}

\noindent
次はどのようにペインタのイメージを縦方向に逆向きにするかです。

\begin{scheme}
(define (flip-vert painter)
  (transform-painter painter
                     (make-vect 0.0 1.0)   ~\textrm{; new \code{origin}}~
                     (make-vect 1.0 1.0)   ~\textrm{; new end of \code{edge1}}~
                     (make-vect 0.0 0.0))) ~\textrm{; new end of \code{edge2}}~
\end{scheme}

\noindent
\code{transform\-/painter}を用いることで簡単に新しい変換を定義することができます。
右上4分の1のフレームは次のようにして与えられます。

\begin{scheme}
(define (shrink-to-upper-right painter)
  (transform-painter
   painter (make-vect 0.5 0.5)
   (make-vect 1.0 0.5) (make-vect 0.5 1.0)))
\end{scheme}

\noindent
他の変換はイメージを時計回りの逆に90度回転したり、\footnote{\code{rotate90}は
四角形のフレームに対してのみの純粋な回転です。イメージもまた拡大縮小して
回転したフレームに合わせられるためです。}

\begin{scheme}
(define (rotate90 painter)
  (transform-painter painter 
                     (make-vect 1.0 0.0)
                     (make-vect 1.0 1.0)
                     (make-vect 0.0 0.0)))
\end{scheme}

\noindent
イメージをフレームの中心に向けて潰したりします。\footnote{\link{Figure 2.10}と
\link{Figure 2.11}内のひし形のイメージは\code{squash\-/inwards}を
\code{wave}と\code{rogers}に適用することで作成されました。}

\begin{scheme}
(define (squash-inwards painter)
  (transform-painter painter
                     (make-vect 0.0 0.0)
                     (make-vect 0.65 0.35)
                     (make-vect 0.35 0.65)))
\end{scheme}

\noindent
フレーム変換は2つ以上のペインタを接続する手段を定義するための鍵でもあります。
例えば\code{beside}手続は2つのペインタを取りそれらを引数のフレームの左半分と右半分に
それぞれ描画するように変換する新しい複合ペインタを生成します。複合ペインタがフレームを
与えられ時、1つ目の変換済みペインタを呼びフレームの左半分に描き、次に二つ目の変換済み
ペインタを呼びフレームの右半分を描きます。

\begin{scheme}
(define (beside painter1 painter2)
  (let ((split-point (make-vect 0.5 0.0)))
    (let ((paint-left
           (transform-painter 
            painter1
            (make-vect 0.0 0.0)
            split-point
            (make-vect 0.0 1.0)))
          (paint-right
           (transform-painter
            painter2
            split-point
            (make-vect 1.0 0.0)
            (make-vect 0.5 1.0))))
      (lambda (frame)
        (paint-left frame)
        (paint-right frame)))))
\end{scheme}

\noindent
ペインタのデータ抽象化と、特にペインタの手続としての表現がどのように\code{beside}の
実装を簡単にしているのか注目して下さい。\code{beside}手続はコンポーネントのペイントの
詳細について各ペインタが指定されたフレームに何かを描くこと以外は一切知る必要がありません。

\begin{quote}
\heading{\phantomsection\label{Exercise 2.50}Exercise 2.50:} 
ペインタを水平方向に引っくり返す変換\code{flip\-/horiz}を定義せよ。
またペインタを時計と逆回りに180度と270度回す変換を定義せよ。
\end{quote}

\begin{quote}
\heading{\phantomsection\label{Exercise 2.51}Exercise 2.51:} 
ペインタに対する\code{below}命令を定義せよ。\code{below}は2つのペインタを引数に取る。
結果のペインタはフレームを与えられ、1つ目のペインタにてフレームの底部を描き、
2つ目のペインタにて上部を描く。\code{below}を2つの異なる方法で定義せよ。
1つは上で与えた\code{beside}と同様な方法で、2つ目は\code{beside}と適切な(\link{Exercise 2.50}の)
回転命令を利用せよ。
\end{quote}

\subsubsection*{堅牢な設計のための言語のレベル}



ピクチャー言語は私達が紹介した手続とデータにようる抽象化についての重要なアイデアの
いくつかを訓練しました。基本的なデータ抽象化であるペインタは手続表現を用いて実装され
言語に異なる基礎的な描画能力を統一した方法で扱うことを可能にしました。
接続手段は閉包の性質を満たし簡単に複雑な設計を組み上げることを可能にしました。
最後に、抽象化手続に対する全てのツールはペインタに対する接続手段の抽象化にとって
有効でした。


私達はまた言語とプログラム設計に関する素晴しい考えを垣間見ることができました。
これは\newterm{stratified design}(\jnewterm{階層化設計})の方法で、複雑なシステムは一連の
言語を用いて記述される一連のレベルとして構造化されるべきであるという概念です。
各レベルはパーツをパーツを接続して構築され、それらは次のレベルではプリミティブとして
参照されます。そして各レベルで構築されたパーツは次のレベルにてプリミティブとして
使用されます。階層化された設計の各レベルで使用される言語はプリミティブ、
接続手段、そしてそのレベルの詳細さに適切な抽象化手段を持っています。



階層化された設計は複雑なシステムの設計において普及しています。例えば
計算機設計では抵抗とトランジスタは接続され(そしてアナログ回路
言語を用いて記述され)ANDゲートやORゲートのようなパーツを生じ、それらがデジタル回路
の言語のプリミティブを形成します。\footnote{\link{Section 3.3.4}がそのような言語について
記述します。} このようなパーツはプロセッサ、バス構造、メモリシステムを構築するために
接続され、それらはコンピュータを形成するために接続され、コンピュータアーキテクチャに
相応しい言語を用います。コンピュータは分散システムを形成するために接続され、
ネットワーク相互接続その他を記述するに適切な言語を用います。


階層化の簡単な例として、ピクチャー言語はプリミティブな要素(プリミティブペインタ)を
用い、それらは点と線を指定し\code{segments\-/>painter}のための線分のリストを提供したり、
\code{rogers}のようなペインタに対するシェーディングの詳細を提供したりする言語を
用いて作成されました。私達のピクチャー言語の説明の大部分がこれらのプリミティブを接続し
\code{beside}や\code{below}のような幾何学的なコンバイナ(結合器)に充てられました。
私達はまたより高階なレベルにおいて\code{beside}と\code{below}をプリミティブとして見做し
\code{square\-/of\-/four}のような命令を持つ言語において幾何学的結合器を接続する共通の
パターンを獲得することに努めました。



階層化設計はプログラムを\newterm{robust}(\jnewterm{堅牢})にすることを手助けします。
それはつまりプログラムにおける仕様上の小さな変更が相応した小さな変更を
要求することを意味します。例えば\link{Figure 2.9}で示された\code{wave}のイメージ
を変更したいとします。\code{wave}要素の詳細な表現を変更する最も低レベルで行うことも
可能ですが、中間のレベルにおいて\code{corner\-/split}が\code{wave}をどのように複製するか
について行うことも可能ですし、最高レベルにおいて\code{square\-/limit}がどのように角の
4つのコピーを配置するかについて変更することも可能です。一般的に階層設計の各レベル
は異なる語彙をシステムの特徴を表すのに提供します。そして異なる種類の変更方法をも
提供します。

\begin{quote}
\heading{\phantomsection\label{Exercise 2.52}Exercise 2.52:} 
\link{Figure 2.9}の\code{wave}のsquare-limitに、上で説明された各レベルで働くことで
変更を加えよ。より詳細には、

\begin{enumerate}[a]

\item
\link{Exercise 2.49}のプリミティブな\code{wave}ペインタにいくつか線分を加えよ。
(例えば笑顔を追加せよ)

\item
\code{corner\-/split}により構築されるパターンを変更せよ
(例えば\code{up\-/split}や\code{right\-/split}のイメージを2つでなく1つにせよ)

\item
\code{square\-/of\-/four}を用いる\code{square\-/limit}のバージョンを変更し角を異なるパターンで
組み立てるようにせよ。
(例えば正方形の各角にてMr. Rogersをそれぞれ外に向けよ)

\end{enumerate}
\end{quote}

\section{記号データ}
\label{Section 2.3}



私達がここまで使用した全ての複合データオブジェクトは最終的には数値から構築されていました。
この節では任意のシンボル(記号)をデータとして用いる能力を紹介することで、言語の表現力を拡張します。



\subsection{クォート}
\label{Section 2.3.1}



もし記号を用いて複合データを形成できれば以下のようなリストを持つことができます。

\begin{scheme}
(a b c d)
(23 45 17)
((Norah 12) (Molly 9) (Anna 7) (Lauren 6) (Charlotte 4))
\end{scheme}

\noindent
記号を含むリストは言語の式と全く同じように見えます。

\begin{scheme}
(* (+ 23 45) 
   (+ x 9))
(define (fact n) 
  (if (= n 1) 1 (* n (fact (- n 1)))))
\end{scheme}

\noindent
記号を扱う目的のためには言語に新しい要素を必要とします。データオブジェクトを
\newterm{quote}(\jnewterm{引用})する能力です。例えばリスト\code{(a b)}を構築したいとします。
私達はこれを\code{(list a b)}を用いては達成できません。なぜならこの式は\code{a}と
\code{b}の値からリストを構築する式であり、記号それ自体ではないからです。
この問題は自然言語の文脈では良く知られていて、単語と文が意味上の要素として見做さ
れているか、または文字列(文法上の要素)として見做されているかの場合が有り得ます。
自然言語での共通な慣例は単語や文が文字通りに扱われることを示すためにクォーテーション
マークを用いることです。例えば``John''の最初の文字は明らかに``J''です。
もし私達が誰かに``貴方の名前を大きな声で言って''と伝えれば、その人の名前を
聞くことを期待します。しかしもし誰かに`` `貴方の名前'と大きな声で言って''と伝えれば
``貴方の名前''という語を聞くことを期待するでしょう。私達がクォーテーションマークを
入れ子にすることを第三者が何を言うだろうかを説明するために強制されていることに
注意して下さい。\footnote{言語の中でクォーテーションを許可することが簡単な語で
言語について推論する能力を与えると共に大きな破壊をもたらしています。それが
等値な物は等値な物と置換できるという概念を破壊するためです。例えば1足す2は3ですが
``3''という語は``1足す2''という語句ではありません。クォーテーションは他の表現を
操作する表現を構築する手段を提供するため強力です。(\link{Chapter 4}でインタプリタを
書く時に学びます)。しかし言語の中で、同じ言語の他の文について話す文を許すことは
``等値な物は等値な物と交換できる''が何を意味すべきかという任意の一貫性を保守すること
をとても難しくします。例えばもし私達が宵の明星が明けの明星と同じであることを知っている
場合、``宵の明星は金星''という文から``明けの明星は金星''であることを推論できます。
しかし``Johnは宵の明星が金星であることを知っている''を与えられても``Johnは明けの明星
が金星であることを知っている''とは推論することはできません。}



データオブジェクトとして扱われるべきであり、式として評価
されるべきではないリストとシンボルを区別するために、これと同じ習慣に従うことができます。
しかしクォートする形式は自然言語のそれとは異なり、クォーテーションマーク(伝統的に
シングルクォートの記号 \code{'})はクォートされるべきオブジェクトの先頭にのみ置かれます。
Schemeの文法でこの様に逃れられるのはオブジェクトを区切るのに空白と括弧を信頼することが
できるためです。従ってシングルクォート文字の意味は次のオブジェクトをクォートすることに
なります。\footnote{シングルクォートは表示される文字列を囲むのに使用してきたダブルクォート
とは異なります。シングルクォートがリストやシンボルを示すのに対し、ダブルクォートは
文字列と共にのみ利用されます。この本では文字列の使用方法は表示されるための項目としてのみ
です。}


これでシンボルとその値を区別することが可能です。

\begin{scheme}
(define a 1)
(define b 2)
(list a b)
~\textit{(1 2)}~
(list 'a 'b)
~\textit{(a b)}~
(list 'a b)
~\textit{(a 2)}~
\end{scheme}

\noindent
クォーテーションはまた慣習的なリストに対する印字された表現を用いて
複合オブジェクトの入力も可能にします。\footnote{厳密には私達のクォーテーションマークの
使用方法は言語における全ての複合式は括弧で区切られリストのように見えるという
全体のルールを破ります。この整合性に対しては特殊形式\code{quote}を紹介することで
回復することが可能です。これはクォーテーションマークと同じ目的を演じます。
従って\code{'a}の代わりに\code{(quote a)}と入力できますし、\code{'(a b c)}の代わりに
\code{(quote (a b c))}と入力できます。これはインタプリタは正確にはどのように働くか
ということです。クォーテーションマークは単一文字による省略形に過ぎず次の完全な
式を\code{quote}でラッピングすることで
\( \hbox{\ttfamily(quote}\;\langle\kern0.06em\hbox{\ttfamily\slshape expression}\kern0.08em\rangle\hbox{\ttfamily)} \)
を形成します。これは重要なことです。なぜならインタプリタに読まれる任意の式がデータオブジェクト
として扱うことができるという原則を保持するからです。例えば\code{(car '(a b c))}という式は
\code{(car (quote (a b c)))}と同じで、\code{(list 'car (list 'quote '(a b c)))}
を評価することで構築できます。}

\begin{scheme}
(car '(a b c))
~\textit{a}~
(cdr '(a b c))
~\textit{(b c)}~
\end{scheme}

\noindent
これを守ることで、空リストを\code{'()}を評価して得ることができます。
従って、変数\code{nil}の使用を止められます。



記号を操作するのに使われるもう1つ追加のプリミティブとして\code{eq?}があります。
これは2つのシンボルを引数として取りそれらが同じであるかテストします。\footnote{
2つのシンボルが同じ文字で同じ順に構成されている場合にそれらが``同じ''であると
考えることができます。そのような定義はまだ私達が解決するには準備の足りない深い
問題を回避しています。プログラミング言語における``同一性''の意味です。
私達はこの問題に\link{Chapter 3} (\link{Section 3.1.3})にて戻ります。}
\code{eq?}を用いることで\code{memq}と呼ばれる便利な手続を実装できます。これは2つの引数、
シンボルとリストを取ります。もしシンボルがリストに含まれていない場合(つまり
リスト中のどの項目にも\code{eq?}でない場合)\code{memq}はfalseを返します。そうでなければ
リスト中のそのシンボルが最初に出現する場所からのサブリストを返します。

\begin{scheme}
(define (memq item x)
  (cond ((null? x) false)
        ((eq? item (car x)) x)
        (else (memq item (cdr x)))))
\end{scheme}

\noindent
例えば、次の式の値は

\begin{scheme}
(memq 'apple '(pear banana prune))
\end{scheme}

\noindent
falseになります。そして次の式の値は

\begin{scheme}
(memq 'apple '(x (apple sauce) y apple pear))
\end{scheme}

\noindent
\code{(apple pear)}です。

\begin{quote}
\heading{\phantomsection\label{Exercise 2.53}Exercise 2.53:} 
以下の式のそれぞれを評価した応答としてインタプリタは何を表示するか?

\begin{scheme}
(list 'a 'b 'c)
(list (list 'george))
(cdr '((x1 x2) (y1 y2)))
(cadr '((x1 x2) (y1 y2)))
(pair? (car '(a short list)))
(memq 'red '((red shoes) (blue socks)))
(memq 'red '(red shoes blue socks))
\end{scheme}
\end{quote}

\begin{quote}
\heading{\phantomsection\label{Exercise 2.54}Exercise 2.54:} 
2つのリストはそれぞれが同じ要素を同じ順で持っている場合に\code{equal?}と言える。
例えば

\begin{scheme}
(equal? '(this is a list) '(this is a list))
\end{scheme}

\noindent
は真であるが

\begin{scheme}
(equal? '(this is a list) '(this (is a) list))
\end{scheme}

\noindent
は偽である。具体的には基本となる\code{eq?}の記号の等価性を再帰的に用いて\code{equal?}
を定義できる。\code{a}と\code{b}が\code{equal?}であるとはそれらが両方とも記号である場合、かつ
記号が\code{eq?}である場合、または両方ともリストであり\code{(car a)}が\code{(car b)}に
\code{equal?}であり、かつ\code{(cdr a)}が\code{(cdr b)}に\code{equal?}であるような場合である。
この考えを用いて\code{equal?}を手続として実装せよ。\footnote{
実際には、プログラマは\code{equal?}を数値と同じくシンボルを含むリストの比較に用いる。
数値は記号とは認識されない。数の上で等しい二つの数値(\code{=}でテストした場合の様に)が
\code{eq?}でもそうであるかという問題は高度に実装依存である。\code{equal?}の
(Schemeにプリミティブとして提供されているような)より良い定義でも、もし\code{a}と\code{b}が
数値であるなら、それらが数値として等しい場合に\code{equal?}であると明記するだろう。}
 
\end{quote}

\begin{quote}
\heading{\phantomsection\label{Exercise 2.55}Exercise 2.55:} 
Eva Lu Atorはインタプリタに次の式を入力した。

\begin{scheme}
(car ''abracadabra)
\end{scheme}

驚いたことにインタプリタは\code{quote}を応答として表示した。説明せよ。
\end{quote}


\subsection{例: 記号微分}
\label{Section 2.3.2}



記号操作の説明として、及びより一層のデータ抽象の説明として、代数式の記号微分を
行う手続の設計について考えてみましょう。手続は引数として代数式と変数を取り
変数に対する式の導関数を返すことにします。例えば手続に対する引数が
\( ax^2 + bx + c \)と\( x \)の時、手続は\( 2ax + b \)を返さなければいけません。
記号微分はLispにとって歴史的に特別な意味があります。記号操作のためのコンピュータ言語
の開発の裏にこれがその動機の一例として存在しました。さらにこれが現在増大する応用数学者
及び科学者に用いられている記号数理の成果の
ための強力なシステムの開発へと導く一連の研究の始まりに跡を残しました。



記号微分プログラムの開発においても\link{Section 2.1.1}の分数システムの開発と
同じデータ抽象化の戦略に従います。最初に``sums,'' ``products,'' ``variables''
のような抽象オブジェクトを操作する微分アルゴリズムを定義します。
これらがどのように表現されているのかについて心配することありません。
後程、表現上の問題については解決しましょう。

\subsubsection*{抽象データを用いた微分プログラム}



問題を簡単にするために、二引数の足し算とかけ算の命令のみから構築される式のみを扱う
とても簡単な記号微分プログラムについて考えましょう。任意のそのような式の微分は以下の
簡約ルールを適用することで実行されます。

\begin{comment}

\begin{example}
dc
-- = 0  for c a constant, or a variable different from x
dx

dx
-- = 1
dx

d(u + v)   du   dv
-------- = -- + --
   dx      dx   dx

d(uv)     / dv \     / du \
----- = u | -- | + v | -- |
 dx       \ dx /     \ dx /
\end{example}

\end{comment}
\begin{displaymath}
{{\it dc} \over {\it dx}} 		= 0, 
	\quad c\ \mbox{は定数か、} x\ \mbox{と異なる変数}, 
\end{displaymath}
\begin{displaymath}
{{\it dx} \over {\it dx}} 		= 1, 
\end{displaymath}
\begin{displaymath}
{{\it d\,(u + v\,)} \over {\it dx}} 	= {{\it du} \over {\it dx}} + {{\it dv} \over {\it dx}}, 
\end{displaymath}
\begin{displaymath}
{{\it d\,(uv\,)} \over {\it dx}} 	= u {{\it dv} \over {\it dx}} + v {{\it du} \over {\it dx}}. 
\end{displaymath}


\noindent
後者の2つのルールは自然に再帰的であることに注意して下さい。つまり和の導関数を得る
ためには最初に項の導関数を求めそれらを足す必要があります。各項は順に分解が必要な
式に成り得ます。順により小さな部分へと分解していくことはやがて定数か、その導関数が
0か1のどちらかになる変数になります。



これらのルールを手続で具体化するために、分数実装の設計で行ったように少し希望的観測
に耽ります。もし代数式を表現するための手段があるのなら式が和か積か定数であるかを
判別することができるはずです。式のパーツを抽出することができるはずです。例えば
足し算に対して加数(第一項)と被加数(第二項)を抽出できるはずです。またパーツから式を
構築することもできるはずです。既に以下のセレクタ、コンストラクタ、述語を実装するための
手続を持っていると仮定しましょう。

\begin{scheme}
(variable? e)            ~\textrm{\code{e}は変数であるか?}~
(same-variable? v1 v2)   ~\textrm{\code{v1}と\code{v2}は同じ変数であるか?}~
(sum? e)                 ~\textrm{\code{e}は和か?}~
(addend e)               ~\textrm{和\code{e}の加数.}~
(augend e)               ~\textrm{和\code{e}の被加数.}~
(make-sum a1 a2)         ~\textrm{\code{a1}と\code{a2}の和を構築する}~
(product? e)             ~\textrm{\code{e}は積か?}~
(multiplier e)           ~\textrm{積\code{e}の乗数}~
(multiplicand e)         ~\textrm{積\code{e}の被乗数}~
(make-product m1 m2)     ~\textrm{\code{m1}と\code{m2}の積を構築する}~
\end{scheme}

\noindent
これらと数値であるかを判断するプリミティブな述語\code{number?}を用いて、以下の手続の
様に微分のルールを表現できます。

\begin{scheme}
(define (deriv exp var)
  (cond ((number? exp) 0)
        ((variable? exp) (if (same-variable? exp var) 1 0))
        ((sum? exp) (make-sum (deriv (addend exp) var)
                              (deriv (augend exp) var)))
        ((product? exp)
         (make-sum
           (make-product (multiplier exp)
                         (deriv (multiplicand exp) var))
           (make-product (deriv (multiplier exp) var)
                         (multiplicand exp))))
        (else
         (error "unknown expression type: DERIV" exp))))
\end{scheme}

\noindent
この\code{deriv}手続は完全な微分アルゴリズムに立脚しています。代数データの項により
表現されているため、適切なセレクタとコンストラクタを設計する限りにおいて、どのように
代数式を表現しても動きます。この条件の部分が次に解決すべき問題です。

\subsubsection*{代数式を表現する}



代数式を表現するリスト構造を使用する手法は数多く想像できます。例えば通常の代数
記法を真似する記号のリストを用い、\( ax + b \)をリスト\code{(a * x + b)}の様に
表現することもできるでしょう。しかし特に直接的な1つの選択はLispが複合式に用いる
のと同じく括弧で括った接頭辞記述法です。つまり\( ax + b \)は\code{(+ (* a x) b)}と
表現されます。従って微分問題に対するデータ表現は以下のとおりです。

\begin{itemize}

\item
変数はシンボルである。プリミティブな述語\code{symbol?}で判別される。

\begin{scheme}
(define (variable? x) (symbol? x))
\end{scheme}

\item
2つの変数はそれらを表現するシンボルが\code{eq?}である時同じだ。

\begin{scheme}
(define (same-variable? v1 v2)
  (and (variable? v1) (variable? v2) (eq? v1 v2)))
\end{scheme}

\item
和と積はリストとして構築される。

\begin{scheme}
(define (make-sum a1 a2) (list '+ a1 a2))
(define (make-product m1 m2) (list '* m1 m2))
\end{scheme}

\item
和は最初の要素がシンボル\code{+}のリストだ。

\begin{scheme}
(define (sum? x) (and (pair? x) (eq? (car x) '+)))
\end{scheme}

\item
加数は和のリストの二つ目の項だ。

\begin{scheme}
(define (addend s) (cadr s))
\end{scheme}

\item
被加数は和のリストの三つめの項だ。

\begin{scheme}
(define (augend s) (caddr s))
\end{scheme}

\item
積は最初の要素がシンボル\code{*}のリストだ。

\begin{scheme}
(define (product? x) (and (pair? x) (eq? (car x) '*)))
\end{scheme}

\item
乗数は積のリストの二つ目の項だ。

\begin{scheme}
(define (multiplier p) (cadr p))
\end{scheme}

\item
被乗数は積のリストの三つめの項だ。

\begin{scheme}
(define (multiplicand p) (caddr p))
\end{scheme}

\end{itemize}

\noindent
従って記号微分プログラムを得るためには、これらを\code{deriv}により組込まれた
アルゴリズムを用いて組み立てることのみが必要でです。いくつかの例とその振舞を
見てみましょう。

\begin{scheme}
(deriv '(+ x 3) 'x)
~\textit{(+ 1 0)}~
(deriv '(* x y) 'x)
~\textit{(+ (* x 0) (* 1 y))}~
(deriv '(* (* x y) (+ x 3)) 'x)
~\textit{(+ (* (* x y) (+ 1 0))}~
   ~\textit{(* (+ (* x 0) (* 1 y))}~
      ~\textit{(+  x 3)))}~
\end{scheme}

\noindent
プログラムは正しい解答を生成します。しかし、それらは簡略化されていません。
\begin{comment}

\begin{example}
d(xy)
----- = x * 0 + 1 * y
 dx
\end{example}

\end{comment}
\begin{displaymath}
 {{\it d\,(xy)} \over {\it dx}} = x \cdot 0 + 1 \cdot y, 
\end{displaymath}
\noindent
しかし私達はこのプログラムに\( x \cdot 0 = 0 \), \( 1 \cdot y = y \),
\( 0 + y = y \)を理解して欲しいと望みます。二つ目の例の解答は単純に\code{y}となるべき
です。三つめの例が示すように、これは式が複雑な場合には深刻な問題となります。



この困難は分数実装において出くわしたものととても似ています。最も単純な形式に解答を
約分していませんでした。分数を約分するためには実装のコンストラクタとセレクタのみを
変更する必要がありました。ここでも同様の戦略を受け入れることができます。\code{deriv}には
全く変更は加えません。その代わりに\code{make\-/sum}を変更し両方の加数が数値である場合、
\code{make\-/sum}はそれらを足してその和を返します。また加数の1つが0ならば\code{make\-/sum}は
もう一方の加数を返します。

\begin{scheme}
(define (make-sum a1 a2)
  (cond ((=number? a1 0) a2)
        ((=number? a2 0) a1)
        ((and (number? a1) (number? a2))
         (+ a1 a2))
        (else (list '+ a1 a2))))
\end{scheme}

\noindent
これには手続\code{=number?}を用いました。式が与えられた数値と等しいかチェックします。

\begin{scheme}
(define (=number? exp num) (and (number? exp) (= exp num)))
\end{scheme}

\noindent
同様に\code{make\-/product}を変更し任意の項に0を掛ければ0、任意の項に1を掛ければそれ自身に
するルールを構築します。


\begin{scheme}
(define (make-product m1 m2)
  (cond ((or (=number? m1 0) (=number? m2 0)) 0)
        ((=number? m1 1) m2)
        ((=number? m2 1) m1)
        ((and (number? m1) (number? m2)) (* m1 m2))
        (else (list '* m1 m2))))
\end{scheme}

\noindent
これがこのバージョンが先程の3つの例でどのように動くかです

\begin{scheme}
(deriv '(+ x 3) 'x)
~\textit{1}~
(deriv '(* x y) 'x)
~\textit{y}~
(deriv '(* (* x y) (+ x 3)) 'x)
~\textit{(+ (* x y) (* y (+ x 3)))}~
\end{scheme}

\noindent
これはとても改善が見られますが、三つめの例は
式を``最も単純''だと同意を得られる形式に変形するプログラムを得るまでには
今だ長い道程があることを示します。代数の簡約の問題は複雑です。他の理由の中でも、
ある目的にとって最も単純な形式が他の目的にとってはそうではないことが有り得る
ためです。

\begin{quote}
\heading{\phantomsection\label{Exercise 2.56}Exercise 2.56:} 
より多くの種類の式を扱うために基本的な微分をどのように拡張すべきか示せ。
例として、以下の微分ルールを実装せよ。
\begin{comment}

\begin{example}
d(u^n)            du
------ = nu^(n-1) --  
  dx              dx
\end{example}

\end{comment}
\begin{displaymath}
 {{\it d\,(u^n\,)} \over {\it dx}} = nu^{n-1} {{\it du} \over {\it dx}} 
\end{displaymath}
\noindent
\code{deriv}プログラムに新しい節を追加し、適切な手続\code{exponentiation?}, \code{base}, \code{exponent},
\code{make\-/exponentiation}を定義せよ。(シンボル\code{**}を指数演算の表記に用いても良い)。
任意の数の0乗は1であり、任意の数の1乗はそれ自身であるというルールを構築せよ。
\end{quote}

\begin{quote}
\heading{\phantomsection\label{Exercise 2.57}Exercise 2.57:} 
微分プログラムを拡張し、(2以上の)任意の数の項の和と積を扱えるようにせよ。
すると上の最後の例は以下のように表現できる。

\begin{scheme}
(deriv '(* x y (+ x 3)) 'x)
\end{scheme}



この問題を和と積の表現のみを変更することで行え。\code{deriv}手続には全く変更を加えない。
例えば和の\code{addend}(加数)は最初の項になり、\code{augend}(被加数)は残りの項の和と
なるであろう。
\end{quote}

\begin{quote}
\heading{\phantomsection\label{Exercise 2.58}Exercise 2.58:} 
微分プログラムを変更し通常の数学の記法を扱えるようにしたいとする。\code{+}と\code{*}は
接中辞となり接頭辞演算子ではなくなる。微分プログラムは抽象データを用いて定義されて
いるので、もっぱら微分プログラムが操作する代数式を表現する述語、セレクタ、コンストラクタを
変更することで式の異なる表現を対応するように変更することができる。

\begin{enumerate}[a]

\item
\code{(x + (3 * (x + (y + 2))))}の様な接中辞で表される代数式を微分することを
どのように行うのか示せ。作業を簡単にするために\code{+}と\code{*}は常に2つの引数を取り
式は完全に括弧で括られていると仮定せよ。

\item
\code{(x + 3 * (x + y + 2))}のような標準的な代数記法を認めることで問題は大幅に難しくなる。
これは必要の無い括弧を省略し、乗算は加算の前に行われると仮定している。
私達の微分プログラムがそれでも働くこの記法に対する適切な述語、セレクタ、コンストラクタを設計できるだろうか?

\end{enumerate}
\end{quote}

\subsection{例: 集合を表現する}
\label{Section 2.3.3}



以前の例において2つ種類の複合データオブジェクトの表現を構築しました。分数と代数式
です。これらの例の1つでは組立時と選択時のどちらかで式を単純化(簡約化)を行うか選択肢が
ありました。しかしそれ以外では
リストを用いたこれらの構造に対する表現の選択肢は直接的なものでした。
私達が集合の表現に向かう時、表現の選択肢はあまり明白ではありません。
本当に数多くの可能な表現が存在し、それらはお互いからいくつかの点において著しく
異なります。



非公式には集合は異なる要素の単純な集まりです。より正確な定義を与えるために
私達はデータ抽象の手法を用いることができます。それは``集合''を集合上で用いられる
操作を特定することで定義することです。これらは\code{union\-/set}, \code{intersection\-/set}, 
\code{element\-/of\-/set?}, \code{adjoin\-/set}です。\code{element\-/of\-/set?}は与えられた要素が
集合のメンバーであるかを判定する述語です。\code{adjoin\-/set}はオブジェクトと集合を
引数として取り、元の集合の要素と挿入された要素をも含む集合を返します。
\code{union\-/set}はどちらかの引数に現れる全ての
要素含む集合である、2つの集合の和集合を計算します。
\code{intersection\-/set}は両方の引数の中に現れる要素のみを含む、2つの集合の共通集合を計算します。
データ抽象の視点から見れば、私達は上で与えられた解釈と一致する方法であれば、これらの
命令を実装するどんな表現を設計することも自由です。\footnote{
もしより正式でありたければ、``上で与えられた解釈と一致する''の部分を、命令群が
以下のようなルールの集合を満たすと指定することができます。

\noindent
任意の集合\code{S}と任意のオブジェクト\code{x}に対し、
\code{(element\-/of\-/set? x (adjoin\-/set x S))}は真(非公式には``オブジェクトを集合にadjoin
すればそのオブジェクトを含む集合を生成する'')

\noindent
任意の集合\code{S}と\code{T}と任意のオブジェクト\code{x}に対し、
\code{(element\-/of\-/set? x (union\-/set S T))}は
\code{(or (element\-/of\-/set? x S) (element\-/of\-/set? x T))}に等しい
(非公式には``\code{(union S T)}の要素は\code{S}または\code{T}に存在する要素'')

\noindent
任意のオブジェクト\code{x}に対し\code{(element\-/of\-/set? x '())}は偽
(非公式には``どのオブジェクトも空集合の要素ではない'')
}

\subsubsection*{順序無しリストとしての集合}


集合を表現するための1つの方法は、どの要素も一度より多くは現れない要素のリストとします。
空集合は空リストとして表現されます。この表現では\code{element\-/of\-/set?}は\link{Section 2.3.1}の
手続\code{memq}と似ています。\code{eq?}の代わりに\code{equal?}を用いているので集合要素は
シンボルである必要はありません。

\begin{scheme}
(define (element-of-set? x set)
  (cond ((null? set) false)
        ((equal? x (car set)) true)
        (else (element-of-set? x (cdr set)))))
\end{scheme}

\noindent
これを用いて\code{adjoin\-/set}を書けます。adjoinされるオブジェクトが既に集合に
存在する場合、単に元の集合を返します。そうでなければ\code{cons}を用いて
オブジェクトを集合を表すリストに追加します。

\begin{scheme}
(define (adjoin-set x set)
  (if (element-of-set? x set)
      set
      (cons x set)))
\end{scheme}

\noindent
\code{intersection\-/set}に対して再帰の戦略を使用できます。もし\code{set2}と
\code{set1}の\code{cdr}の共通集合の求める方がわかれば、これに\code{set1}の\code{car}を
含めるかどうか決定することのみが必要です。しかしこれは\code{(car set1)}が\code{set2}
にも存在するかに依存します。以下に結果の手続を示します。

\begin{scheme}
(define (intersection-set set1 set2)
  (cond ((or (null? set1) (null? set2)) '())
        ((element-of-set? (car set1) set2)
         (cons (car set1)
               (intersection-set (cdr set1) set2)))
        (else (intersection-set (cdr set1) set2))))
\end{scheme}

\noindent
表現の設計において、私達が考慮しなければならない問題の1つは効率です。
集合操作により必要とされるステップ数を考えて下さい。
それら全てが\code{element\-/of\-/set?}を使用するので、この命令のスピードは総じて
集合実装の効率上に主要な影響を与えます。ここで、あるオブジェクトが集合の要素であるか
をチェックするために、\code{element\-/of\-/set?}は集合全体をスキャンしなければならない
かもしれません。(最悪の場合、そのオブジェクトがその集合の中に存在しないことが分かる
かもしれません)。それ故にもしその集合が\( n \)要素を持つ場合、\code{element\-/of\-/set?}は
最大\( n \)ステップかかるかもしれません。従って必要とされるステップ数は\( \Theta(n) \)
で増加します。\code{adjoin\-/set}により必要とされるステップ数は、それがこの命令を用いるので、
これもまた\( \Theta(n) \)で増加します。
\code{intersection\-/set}は、\code{set1}の各要素に対し\code{element\-/of\-/set?} のチェックを行うため、
必要とされるステップ数は関係する集合のサイズの積か、またはサイズ\( n \)の2つの集合に対し
\( \Theta(n^2) \)で増加します。\code{union\-/set}に対しても同じことが言えます。

\begin{quote}
\heading{\phantomsection\label{Exercise 2.59}Exercise 2.59:} 
集合の順序無しリスト表現に対する\code{union\-/set}命令を実装せよ。
\end{quote}

\begin{quote}
\heading{\phantomsection\label{Exercise 2.60}Exercise 2.60:}
集合は重複無しのリストとして表現されると指示した。ここで重複を許可すると
仮定してみる。例として集合\( \{1, 2, 3\} \)はリスト\code{(2 3 2 1
3 2 2)}として表現できるだろう。この表現上で操作を行う手続\code{element\-/of\-/set?}, \code{adjoin\-/set},
\code{union\-/set}, \code{intersection\-/set}を設計せよ。
それぞれの効率は対応する重複無し表現に対する手続に比べてどれ程だろうか?
重複無しの集合に優先してこの表現を用いるだろうアプリケーションはあるだろうか?
\end{quote}

\subsubsection*{順序有りリストとしての集合}


私達の集合操作を高速化するための1つの方法として表現を変更することで集合要素を
昇順に並べる方法があります。これを行うには2つのオブジェクトを比較する何らかの
方法が必要です。それによりどちらが大きいを言うことができます。例えばシンボルを
辞書順で比較したり、オブジェクトに一意の番号を付けその後要素を対応する番号で比較
するための何らかの方法について同意できるでしょう。議論を単純にするため私達は集合
要素が数値である場合のみについて考えます。それにより要素を\code{>}と\code{<}を用いて
比較することができます。数値の集合をその要素を昇順に並べることで表現しましょう。
上の最初の表現は集合\( \{1, 3, 6, 10\} \)を要素を任意の順で並べることで表現でき
る一方で、新しい表現はリスト\code{(1 3 6 10)}のみを許します。



順序付けの1つの利点は\code{element\-/of\-/set?}にて現れます。項目の存在をチェックする
場合において、集合全体をスキャンする必要がありません。もし探している項目よりも
大きな要素に出会ったならばその集合にこの項目が無いことがわかります。

\begin{scheme}
(define (element-of-set? x set)
  (cond ((null? set) false)
        ((= x (car set)) true)
        ((< x (car set)) false)
        (else (element-of-set? x (cdr set)))))
\end{scheme}

\noindent
これがどれだけのステップを割引くでしょうか? 最悪の場合、探している項目は
集合の中で一番大きい物かもしれません。その場合ステップ数は順序無し表現と
同じです。しかし一方でもし多くの異なるサイズの項目を探している場合、時々は
リストの先頭近くの点で検索を停止することができることを期待できます。そして
他の場合にはやはりリストのほとんどを試験しなければいけません。平均では集合の
項目数の半分近くを試験しなければいけないことが期待できるはずです。
従って必要とされる平均のステップ数は約\( n / 2 \)になります。これはそれでも
\( \Theta(n) \)で増加しますが、以前の実装に対して平均的にはステップ数において
半分に節約します。



\code{intersection\-/set}ではより目覚ましい高速化を得ます。順序無し表現ではこの命令
は\( \Theta(n^2) \)ステップを必要としました。\code{set1}の各要素に対して\code{set2}の
完全なスキャンを実行していたためです。しかし順序有り表現ではより賢い方法を用いる
ことができます。二つの集合の最初の要素\code{x1}と\code{x2}を比較することで始め、もし
\code{x1}と\code{x2}が等しい場合にはそれらは共通集合の要素です。そして共通集合の残りは
2つの集合の\code{cdr}の共通集合です。そうでなく\code{x1}が\code{x2}より小さい場合を考えます。
\code{x2}は\code{set2}の最小の要素ですから直ぐに\code{x1}は\code{set2}のどこにも現れず、従って
共通集合には有り得ません。従って共通集合は\code{set2}と\code{set1}の\code{cdr}の共通集合に
等しいとなります。同様にもし\code{x2}が\code{x1}より小さい場合、共通集合は\code{set1}と
\code{set2}の\code{cdr}の共通集合にて与えられます。以下に手続を与えます。

\begin{scheme}
(define (intersection-set set1 set2)
  (if (or (null? set1) (null? set2))
      '()
      (let ((x1 (car set1)) (x2 (car set2)))
        (cond ((= x1 x2)
               (cons x1 (intersection-set (cdr set1)
                                          (cdr set2))))
              ((< x1 x2)
               (intersection-set (cdr set1) set2))
              ((< x2 x1)
               (intersection-set set1 (cdr set2)))))))
\end{scheme}

\noindent
この処理により必要とされるステップ数を推定するために、各ステップにて共通集合問題は
縮小され、\code{set1}か\code{set2}、又はその両方の最初の要素を削除することで、
より小さな集合の共通部分を求める問題になっていることに注意して下さい。
従って必要とされるステップ数は最大でも\code{set1}と\code{set2}のサイズの合計であり、
順序無し表現におけるサイズの積とはなりません。これは\( \Theta(n^2) \)でなく、
\( \Theta(n) \)で増加するため、例え中程度のサイズの集合に対してでも考慮に値する高速化です。

\begin{quote}
\heading{\phantomsection\label{Exercise 2.61}Exercise 2.61:} 
順序有り表現を用いた\code{adjoin\-/set}の実装を与えよ。\code{element\-/of\-/set?}との類似点にて、
どのように順序の利点を用いて順序無し表現に対し平均で約半分のステップを必要とする手続を
生成するかを示せ。
\end{quote}

\begin{quote}
\heading{\phantomsection\label{Exercise 2.62}Exercise 2.62:} 
順序付きリストとして表現された集合に対する\code{union\-/set}の実装を\( \Theta(n) \)の
範囲で行え。
\end{quote}

\subsubsection*{二分木としての集合}



集合要素を木の形式にて準備することで順序有りリスト表現よりも良く行なうことができます。
木の各ノードはそのノードにおける``エントリ''と呼ばれる集合の1つの要素と他の2つの(空にも有り得る)
ノードへのリンクを持ちます。``左''のリンクはそのノードよりも小さな値を差し、``右''のリンクはその
ノードの値より大きな値のノードを差します。\link{Figure 2.16}は集合\( \{1, 3, 5, 7, 9, 11\} \)
を表現するいくつかの木を示しています。

\begin{figure}[tb]
\phantomsection\label{Figure 2.16}
\centering
\begin{comment}
\heading{Figure 2.16:} Various binary trees that represent the set \( \{1, 3, 5, 7, 9, 11\} \).

\begin{example}
   7          3             5
   /\         /\            /\
  3  9       1  7          3  9
 /\   \         /\        /   /\
1  5  11       5  9      1   7  11
                   \
                   11
\end{example}
\end{comment}
\includegraphics[width=70mm]{fig/chap2/Fig2.16b.pdf}
\begin{quote}
\heading{Figure 2.16:} 集合\( \{1, 3, 5, 7, 9, 11\} \)を表現するさまざまな二分木
\end{quote}
\end{figure}

\noindent
木表現の優位点は次のとおりです。ある数値\( x \)がある集合に含まれているかどうかをチェック
したいと想定します。\( x \)をトップノードのエントリと比較することから始めます。
もし\( x \)がこれよりも小さければ、左の部分木のみを探索すれば良いことがわかります。
もし\( x \)が大きければ、右の部分木のみを探索する必要があります。ここで、木が
``バランスが取れた''状態であるとは各部分木のサイズが元の約半分であるということです。
従って一度のステップにおいてサイズ\( n \)の木の探索問題を、サイズ\( n / 2 \)の
木の探索問題に縮小したことになります。各ステップにより木の探索に必要なステップは半分に
なるのでサイズ\( n \)の木の探索に必要なステップ数は\( \Theta(\log n) \)で増加することが
期待されます。\footnote{
\link{Section 1.2.4}の高速指数アルゴリズムや\link{Section 1.2.4}の半区間検索手法で学んだように各ステップにて
問題のサイズを半分にすることは対数増加の特徴的な性質です。}
大きな集合に対しては以前の表現に比べこれは著しい高速化になるでしょう。


木はリストを用いて表現できます。各ノードは3つの項目のリストになります。ノードのエントリ、
左部分木、右部分木です。左、または右部分木が空リストの場合はそこに接続された部分木が存在
しないことを示します。この表現を以下の手続にて説明できます。\footnote{
私達は集合を木を用いて表現しており、そして木はリストを用いています---事実上、データ抽象化が
データ抽象化の上に構築されています。手続\code{entry}, \code{left\-/branch},\code{right\-/branch},
\code{make\-/tree}を
私達がそのような木をリスト構造を用いて表現することを望んだ特定の方法から``二分木''の抽象化を
分離する方法として見做すことができます。}

\begin{scheme}
(define (entry tree) (car tree))
(define (left-branch tree) (cadr tree))
(define (right-branch tree) (caddr tree))
(define (make-tree entry left right)
  (list entry left right))
\end{scheme}

\noindent
これで\code{element\-/of\-/set?}手続を上で説明された戦略を用いて書くことができます。

\begin{scheme}
(define (element-of-set? x set)
  (cond ((null? set) false)
        ((= x (entry set)) true)
        ((< x (entry set))
         (element-of-set? x (left-branch set)))
        ((> x (entry set))
         (element-of-set? x (right-branch set)))))
\end{scheme}

\noindent
集合に項目を付加することは同様に実装され、そしてまた\( \Theta(\log n) \)ステップを
必要とします。項目\code{x}を付加するためには、\code{x}をノードのエントリと比較し\code{x}が右か
左のどちらの枝に追加されるべきを判断し、\code{x}を適切な枝に追加し、この新しく構築された枝を元の
エントリともう一方の枝と共に接続します。もし\code{x}を空の木に付加するよう求められたら
エントリに\code{x}を持ち、右と左の枝は空である木を生成します。以下がこの手続です。

\begin{scheme}
(define (adjoin-set x set)
  (cond ((null? set) (make-tree x '() '()))
        ((= x (entry set)) set)
        ((< x (entry set))
         (make-tree (entry set)
                    (adjoin-set x (left-branch set))
                    (right-branch set)))
        ((> x (entry set))
         (make-tree (entry set) (left-branch set)
                    (adjoin-set x (right-branch set))))))
\end{scheme}

\noindent
上の木の検索は対数ステップで実行可能であるとの主張は木は``バランスが取れている''
という前提に依存しています。すなわち、全ての木の左と右の部分木は大体同じ要素の
数を持っているため、各部分木はその親の約半分の要素持っていることになります。
しかしどのようにすれば私達が構築した木がバランスが取れていると確信することが
できるのでしょうか。例えもしバランスの取れた木で開始したとしても、\code{adjoin\-/set}
にて要素を足していけばバランスが取れていない結果を生み出します。新しく付加される
要素の位置は集合に既に存在する項目とどのように比較されるかに依存するために、
もし要素を``ランダム''に追加すればその木が平均ではバランスが取れることが予想でき
ます。しかしこれは保証されません。例えばもし空集合から始めて数値を1から7まで順番に
追加していけば\link{Figure 2.17}で示されるとてもアンバランスな木になってしまいます。
この木では全ての左の部分木は空であり、単純な順序有りリストに対する優位点が存在
しません。この問題を解く1つの方法として任意の木をバランスの取れた木に同じ要素を
用いて変換する操作を定義することが上げられます。そうすれば数回毎の\code{adjoin\-/set}
の後にこの変換を実行することで集合のバランスを保つことができます。この問題を解く
他の方法もまた存在しますが、その多くは検索と挿入の両方が\( \Theta(\log n) \)
ステップで行える新しいデータ構造を設計することを含みます。\footnote{そのような
構造の例には\newterm{B-trees}(\jnewterm{B木})や\newterm{red-black trees}(\jnewterm{赤黒木})があります。この問題に
ささげられた巨大な文献が存在します。 \link{Cormen et al. 1990}を参照して下さい。}

\begin{figure}[tb]
\phantomsection\label{Figure 2.17}
\centering
\begin{comment}
\heading{Figure 2.17:} Unbalanced tree produced by adjoining 1 through 7 in sequence.

\begin{example}
1
 \
  2
   \
    4
     \
      5
       \
        6
         \
          7
\end{example}
\end{comment}
\includegraphics[width=40mm]{fig/chap2/Fig2.17a.pdf}
\begin{quote}
\heading{Figure 2.17:} 1から7まで順にadjoinすることで生成した不均衡な木
\end{quote}
\end{figure}

\begin{quote}
\heading{\phantomsection\label{Exercise 2.63}Exercise 2.63:} 
以下の2つの手続はそれぞれ二分木をリストに変換する。

\begin{scheme}
(define (tree->list-1 tree)
  (if (null? tree)
      '()
      (append (tree->list-1 (left-branch tree))
              (cons (entry tree)
                    (tree->list-1 
                      (right-branch tree))))))
(define (tree->list-2 tree)
  (define (copy-to-list tree result-list)
    (if (null? tree)
        result-list
        (copy-to-list (left-branch tree)
                      (cons (entry tree)
                            (copy-to-list 
                              (right-branch tree)
                              result-list)))))
  (copy-to-list tree '()))
\end{scheme}

\begin{enumerate}[a]

\item
2つの手続は全ての木に対して同じ結果を生成するか? もしそうでなければどのように
結果は異なるか? \link{Figure 2.16}の木に対して2つの手続はどんなリストを生成するか?

\item
2つの手続は\( n \)要素のバランスの取れた木をリストに変換するのに同じステップ数増加の
オーダであるか? もしそうでなければどちらがより遅く増加するか?

\end{enumerate}
\end{quote}

\begin{quote}
\heading{\phantomsection\label{Exercise 2.64}Exercise 2.64:} 
以下の手続\code{list\-/>tree}は順序有りリストをバランスの取れた木に変換する。
ヘルパ手続\code{partial\-/tree}は引数として整数\( n \)と少なくとも\( n \)要素のリスト
を取り、リストの最初の\( n \)要素を含むバランスの取れた木を生成する。
\code{partial\-/tree}の結果として返されるのは(\code{cons}で構築された)ペアであり、
\code{car}が構築された木で\code{cdr}が木に含まれなかった要素のリストである、

\begin{scheme}
(define (list->tree elements)
  (car (partial-tree elements (length elements))))
(define (partial-tree elts n)
  (if (= n 0)
      (cons '() elts)
      (let ((left-size (quotient (- n 1) 2)))
        (let ((left-result 
               (partial-tree elts left-size)))
          (let ((left-tree (car left-result))
                (non-left-elts (cdr left-result))
                (right-size (- n (+ left-size 1))))
            (let ((this-entry (car non-left-elts))
                  (right-result 
                   (partial-tree 
                    (cdr non-left-elts)
                    right-size)))
              (let ((right-tree (car right-result))
                    (remaining-elts 
                     (cdr right-result)))
                (cons (make-tree this-entry
                                 left-tree
                                 right-tree)
                      remaining-elts))))))))
\end{scheme}

\begin{enumerate}[a]

\item
できるだけ明確に\code{partial\-/tree}がどのように働くのか文章で答えよ。
リスト\code{(1 3 5 7 9 11)}に対し\code{list\-/>tree}により生成される木を描け。

\item
\code{list\-/>tree}が\( n \)要素のリストを変換するのに必要とされるステップ数の
増加んオーダはいくらか?

\end{enumerate}
\end{quote}

\begin{quote}
\heading{\phantomsection\label{Exercise 2.65}Exercise 2.65:} 
\link{Exercise 2.63}と\link{Exercise 2.64}の結果を用いて(バランスの取れた)二分木として
実装された集合の\( \Theta(n) \)における実装を与えよ。\footnote{
\link{Exercise 2.63}から\link{Exercise 2.65}はPaul Hilfingerによるものである。}
\end{quote}

\subsubsection*{集合と情報検索}


リストを集合を表現するのに使用する選択肢について調べ、データオブジェクトに対する
表現の選択がどのようにそのデータを使用するプログラムのパフォーマンスに大きな
影響を与えるかについて学びました。集合に専念するもう1つの理由としてここで議論された
テクニックが情報検索を含むアプリケーションにおいて何度も何度も現われることが
上げられます。



企業が持つ個人情報や会計システムの取引等、大量の個人レコードを持つデータベースに
ついて考えてみて下さい。典型的なデータ管理システムはレコードの中のデータへの
アクセスや変更に多大な時間を過ごします。従ってレコードにアクセスする効率的な
手法を必要とします。これは各レコードの一部に識別子である\newterm{key}(\jnewterm{キー})としての
役割を果たさせることで行われます。キーはレコードを一意に識別する任意の物で
かまいません。個人情報に対しては従業員番号であったりします。会計システムにおいては
取引番号であったりします。キーが何であれレコードをデータ構造として定義する時、
与えられたレコードに関連するキーを取得する\code{key}セレクタ手続を含まなければなりません。



さて、データベースをレコードの集合として表現します。与えられたキーでレコードを指し示す
ためには手続\code{lookup}を用い、引数としてキーとデータベースを取り、そのキーを持つレコード
を返すか、そのようなレコードが無ければfalseを返します。例えばもしレコードの集合が
順序無しリストで実装されていれば、以下を用いることができます。

\begin{scheme}
(define (lookup given-key set-of-records)
  (cond ((null? set-of-records) false)
        ((equal? given-key (key (car set-of-records)))
         (car set-of-records))
        (else (lookup given-key (cdr set-of-records)))))
\end{scheme}

\noindent
もちろん、巨大な集合を表現するのには順序無しリストよりもより良い方法が存在します。
レコードが``ランダムアクセス''される情報検索システムは一般的に木をベースにした手法で
実装されます。以前に議論された二分木のような物です。そのようなシステムを設計する場合、
データ抽象化の方法論はとても大きな手助けになります。設計者は順序無しリストの様な、
簡単で直接的な表現を用いて初期実装を作成することができます。これは最終的なシステムには
相応しくありません。しかし、``quick and dirty''(迅速だが汚い)データベースを残りのシステム
をテストするために提供する目的には便利でしょう。後でデータ表現はより洗練された物に
変更することが可能です。もしデータベースが抽象セレクタとコンストラクタによりアクセス
されるのならば、この表現上の変更は残りのシステムに対し何の変更も要求しません。

\begin{quote}
\heading{\phantomsection\label{Exercise 2.66}Exercise 2.66:} 
レコードの集合が二分木として構造化され、キーの数値で順序付けられている場合の
\code{lookup}手続を実装せよ。
\end{quote}

\subsection{例: ハフマン符号化木}



この節では集合と木を操作するためのリスト構造とデータ抽象化の使用のための練習を提供
します。アプリケーションの狙いは1と0の(ビットの)列としてのデータを表現するための
手法です。例えばASCII標準コードはコンピュータ内にて各文字を7ビットの列に
符号化してテキストを表現するのに利用されます。7ビットを用いることは\( 2^7 \)、
または128の異なる文字を区別することができます。一般的に、もし\( n \)個の異なる
記号を区別したい場合、記号当たりに\( \log_2\!n \)ビットの使用が必要となります。
もし全てのメッセージが8つのシンボル、A, B, C, D, E, F, G, Hで作られている場合、
一文字当たり3ビットのコードを選択することができます。以下に例を上げます。

\begin{example}
A 000    C 010    E 100    G 110
B 001    D 011    F 101    H 111
\end{example}

\noindent
この符号を用いて、以下のメッセージは

\begin{example}
BACADAEAFABBAAAGAH
\end{example}

\noindent
54ビットの列として符号化されます。

\begin{example}
001000010000011000100000101000001001000000000110000111
\end{example}

\noindent
ASCIIや上記のAからHの符号は\newterm{fixed-length}(\jnewterm{固定長})符号として知られています。
それらがメッセージの各記号を同じ数のビットを用いて表現するためです。
時には\newterm{variable-length}(\jnewterm{可変長})符号を使用することが有利な場合もあります。
異なるシンボルが異なる数のビットで表現され得るものです。例えばモールス符号は
アルファベットの各文字に対して同じ数の点と長音を用いはしません。
具体的にはEは最も頻繁に現われる文字ですので単一のドットで表現されます。
一般的にはもしメッセージにおいてある記号がとても良く現れ、ある記号はとても稀に
現れる場合、短い符号を頻出のシンボルに割り振ることでデータをより効率的に(つまり
メッセージ当たりでより少ないビット数で)符号化することができます。

\begin{example}
A 0      C 1010    E 1100    G 1110
B 100    D 1011    F 1101    H 1111
\end{example}

\noindent
この符号では上の同じメッセージが以下の列として符号化されます。

\begin{example}
100010100101101100011010100100000111001111
\end{example}

\noindent
この列は42ビットですから上で示した固定長符号に比べ記憶域において20\%以上節約できています。



可変長符号を用いる上での難点の1つとして0と1の列を読んでいる時にいつシンボルの終わりに
辿り着いたか知ることが上げられます。モールス符号はこの問題を各文字に対するトンとツーの
列の後に特別な\newterm{separator code}(\jnewterm{分離符号})(この場合には一息置くこと)を用いることで
解決しました。他の解法としてはどの任意のシンボルに対する完全な符号も他のシンボルの符号の
始め(または\newterm{prefix}(\jnewterm{接頭辞}))ではない様に符号を設計するという物があります。
このような符号は\newterm{prefix code}(\jnewterm{接頭符号})と呼ばれます。上の例ではAは0で符号化され
Bは100で符号化されるので、他のどのシンボルも0、または100で始まる符号を持つことができません。



一般的に、もし符号化対象のメッセージ中のシンボルの相対頻度の利点を得られる
可変長接頭符号を用いれば著しい倹約を達成することができます。これを行うための
1つの特定な理論体系としてその発見者David Huffmanに因んでハフマン符号と呼ばれる
手法があります。ハフマン符号は葉が符号化された記号である二分木として表現するこ
とができます。木の葉でないノードのそれぞれには
そのノードの下に位置する葉の中のシンボル全てを含む集合があります。
加えて各葉のシンボルには重み(相対的な頻度)が割り振られており、葉でないノードの
それぞれはその下に位置する葉の重み全ての合計である重みを持っています。
重みはエンコード、またはデコード処理では利用されません。以下では重みがどのように
木の構築を手助けするかについて学びます。



\link{Figure 2.18}は上で与えられたAからH符号に対するハフマン木を示しています。
葉の重みは
この木がAは相対頻度8、Bは相対頻度3、他の文字は相対頻度1で現われるメッセージに対し
設計されたことを示しています。

\begin{figure}[tb]
\phantomsection\label{Figure 2.18}
\centering
\begin{comment}
\heading{Figure 2.18:} A Huffman encoding tree.

\begin{example}
           {A B C D E F G H} 17
                    *
                   / \
                  /   \
                A 8    * {B C D E F G H} 9
            __________/ \_____________
           /                          \
{B C D} 5 *                            * {E F G H} 4
         / \                       ___/ \___
        /   \                     /         \
      B 3    * {C D} 2   {E F} 2 *           * {G H} 2
            / \                 / \         / \
           /   \               /   \       /   \
         C 1   D 1           E 1   F 1   G 1   H 1
\end{example}
\end{comment}
\includegraphics[width=81mm]{fig/chap2/Fig2.18a.pdf}
\par\bigskip
\noindent
\heading{Figure 2.18:} ハフマン符号化木
\end{figure}

\noindent
ハフマン木を与えられることで、任意のシンボルの符号を木の根から始めてそのシンボル
を持つ葉に辿り着くまで降りていくことで見つけることができます。
左の枝に降りる度に符号に0を追加し、右の枝に降りる度に1を追加します。(どの枝を降り
るかはどの枝がそのシンボルに対する葉を含むか、つまりその集合にシンボルを含むか
をテストして判断することで決定します)。例えば\link{Figure 2.18}の木の根から始めて
Dに対する葉に辿り付くには右の枝を選択し、次に左の枝、次に右の枝、次に右の枝を
辿ります。従ってDに対する符号は1011になります。



ビット列をハフマン木を用いて複合するには、根から始めてビット列の一連の0と1を用いて
左か右の枝を下りるのか決定します。葉に着く度に、メッセージの新しいシンボルを生成し、
その時点で木の根から再開し次のシンボルを見つけます。例えば上記の木と列10001010を
与えられたとします。根から始めて右の枝へと下ります。(列の最初のビットが1だからです)。
次に左の枝を下ります。(2つ目のビットが0だからです)。次に左の枝を下ります。(3つ目の
ビットもまた0だからです)。この様にしてBに対する葉に辿り着くので複合されたメッセージ
の最初のシンボルはBです。ここでまた根から再開し、次のビットが0なので左に移動します。
これによりAの葉に辿り着きます。そしてまた根から残りの列1010と共に再開し、右、左、右
左と動きCに辿り着きます。従ってメッセージ全体はBACです。

\subsubsection*{ハフマン木の生成}


シンボルの``アルファベット''とそれらに対応する頻度を与えられた時、どのように
``最高の''符号を構築できるでしょうか? (言い替えれば、どの木がメッセージを最も少ない
ビット数で符号化するでしょうか?)。ハフマンはこれを行うアルゴリズムを与え、結果の符号が
相対的なシンボルの頻度と符号が構築された時の頻度が合致した場合に、実際にメッセージに
対する最良の可変長符号であることを示しました。
このハフマン符号の最適性についてはここでは証明しません。しかしハフマン木がどのように構築
されるかについては示します。\footnote{ハフマン符号の数学上の特性についての議論に対しては
\link{Hamming 1980}を参照して下さい。}


ハフマン木を生成するためのアルゴリズムはとても簡単です。その考えは木を再配置することで
最も低い頻度のシンボルが根から最も遠く現れるようにします。シンボルと頻度を含む葉のノード
の集合と共に
符号が構築される初期データにより決定されるに従う
まず重みが最低の2つの葉を見つけ、それらをマージしてこの2つのノードを左と右の枝に持つ
ノードを生成します。新しいノードの重みは2つの重みの和です。元の集合から2つの葉を削除し
それらをこの新しいノードで置き換えます。この処理を続けます。各ステップにて最も小さな
重みを持つ2つのノードをマージし、集合から削除し、これらの2つを左と右の枝に持つノードで
置き換えします。処理は1つのノードのみが残った時に停止し、それが木全体の根になります。
以下に\link{Figure 2.18}のハフマン木がどのように生成されるかを示します。

\begin{example}
Initial  
leaves   {(A 8) (B 3) (C 1) (D 1) (E 1) (F 1) (G 1) (H 1)}
Merge    {(A 8) (B 3) ({C D} 2) (E 1) (F 1) (G 1) (H 1)}
Merge    {(A 8) (B 3) ({C D} 2) ({E F} 2) (G 1) (H 1)}
Merge    {(A 8) (B 3) ({C D} 2) ({E F} 2) ({G H} 2)}
Merge    {(A 8) (B 3) ({C D} 2) ({E F G H} 4)}
Merge    {(A 8) ({B C D} 5) ({E F G H} 4)}
Merge    {(A 8) ({B C D E F G H} 9)}
Final    {({A B C D E F G H} 17)}
merge
\end{example}

\noindent
このアルゴリズムは常に同じ木を特定しません。各ステップにおいて重みが最小の
ノードのペアが一意となるとは限らないためです。またどの2つのノードがマージされるかの順も
決定しません。(つまりどれが右になり、どれが左になるのかはわかりません)。

\subsubsection*{ハフマン木の表現}


以下ではハフマン木を用いてメッセージの符号化・複合化を行い、かつ上で概説した
アルゴリズムに基づきハフマン木を作成します。木はどのように表現されるかの議論から
始めます。



木の葉はシンボル\code{leaf}、葉に対するシンボル、そして重みから構成される
リストにて表現されます。

\begin{scheme}
(define (make-leaf symbol weight)
  (list 'leaf symbol weight))
(define (leaf? object)
  (eq? (car object) 'leaf))
(define (symbol-leaf x) (cadr x))
(define (weight-leaf x) (caddr x))
\end{scheme}

\noindent
一般的な木は左側の枝、右側の枝、シンボルの集合、そして重みのリストになります。
シンボルの集合は単純にシンボルのリストとなり、より洗練された集合の表現を用いは
しません。2つのノードをマージして木を作る時、木の重みを各ノードの重みの和として取得し、
シンボルの集合は各ノードのシンボルの集合の和集合とします。シンボルの集合はリストとして
表現されていますので和集合は\link{Section 2.2.1}で定義した\code{append}手続を用いて形成することが
できます。

\begin{scheme}
(define (make-code-tree left right)
  (list left
        right
        (append (symbols left)
                (symbols right))
        (+ (weight left)
           (weight right))))
\end{scheme}

\noindent
もしこのように木を作るのなら以下のようなセレクタを持つことになります。

\begin{scheme}
(define (left-branch  tree) (car  tree))
(define (right-branch tree) (cadr tree))

(define (symbols tree)
  (if (leaf? tree)
      (list (symbol-leaf tree))
      (caddr tree)))
(define (weight tree)
  (if (leaf? tree)
      (weight-leaf tree)
      (cadddr tree)))
\end{scheme}

\noindent
手続\code{symbols}と\code{weight}はそれらが葉と一般的な木のどちらと共に呼ばれたかに
より少々異なったことをしなければなりません。これらは\newterm{generic procedures}
(\jnewterm{ジェネリック手続}:二種類以上のデータを扱える手続)の簡単なサンプルであり、\link{Section 2.4}と
\link{Section 2.5}にてより多くのことについて述べます。

\subsubsection*{複合化手続}


以下の手続は複合化アルゴリズムを実装します。0と1のリストをハフマン木と共に引数として
取ります。

\begin{scheme}
(define (decode bits tree)
  (define (decode-1 bits current-branch)
    (if (null? bits)
        '()
        (let ((next-branch
               (choose-branch 
                (car bits) current-branch)))
          (if (leaf? next-branch)
              (cons (symbol-leaf next-branch)
                    (decode-1 (cdr bits) tree))
              (decode-1 (cdr bits) next-branch)))))
  (decode-1 bits tree))
(define (choose-branch bit branch)
  (cond ((= bit 0) (left-branch branch))
        ((= bit 1) (right-branch branch))
        (else (error "bad bit: CHOOSE-BRANCH" bit))))
\end{scheme}

\noindent
手続\code{decode\-/1}は2つの引数を取ります。残りのビット列のリストと木における現在の
位置です。木を``下り''続けるためリストの中の次のビットが0であるか1であるかに
従って左、または右の枝を選択します。(これは手続\code{choose\-/branch}と共に行われます)。
葉に辿りついた時、その葉のシンボルをメッセージの次のシンボルとして、残りのメッセージ
を木の根から再開して複合した結果の頭に\code{cons}することで返します。\code{choose\-/branch}
の最終条項のエラーチェックに注目して下さい。もし手続が0または1以外の物を入力データに
見つけた場合にエラーを発します。

\subsubsection*{重み付き要素の集合}

私達の木の表現において、各葉ではないノードは簡単にリストとして表現したシンボルの
集合を持ちます。しかし上で議論した木の生成アルゴリズムはまた葉と木の集合に対しても
働き、2つの最小の項目のマージを続けなければなりません。集合の中の最も小さな項目を
繰り返し見つけなければなりませんから、このような集合に対しては順序有りの表現を用いる
と便利です。

葉と木の集合を重みの昇順で配置した要素のリストとして表現することにします。
以下の集合を構築するための\code{adjoin\-/set}手続は\link{Exercise 2.61}にて説明した
ものと似ています。しかし項目はその重みにて比較され、集合に追加される要素はその中に
既に存在はしないとします。

\begin{scheme}
(define (adjoin-set x set)
  (cond ((null? set) (list x))
        ((< (weight x) (weight (car set)))
         (cons x set))
        (else (cons (car set)
                    (adjoin-set x (cdr set))))))
\end{scheme}

\noindent
以下の手続は\code{((A 4) (B 2) (C 1) (D 1))}の様なシンボルと頻度のペアのリストを
取り、葉の初期順序有り集合を構築し、ハフマンアルゴリズムに従いマージできるよう
準備します。

\begin{scheme}
(define (make-leaf-set pairs)
  (if (null? pairs)
      '()
      (let ((pair (car pairs)))
        (adjoin-set (make-leaf (car pair)    ~\textrm{; symbol}~
                               (cadr pair))  ~\textrm{; frequency}~
                    (make-leaf-set (cdr pairs))))))
\end{scheme}

\begin{quote}
\heading{\phantomsection\label{Exercise 2.67}Exercise 2.67:} 
符号化木とサンプルのメッセージを定義する。

\begin{scheme}
(define sample-tree
  (make-code-tree (make-leaf 'A 4)
                  (make-code-tree
                   (make-leaf 'B 2)
                   (make-code-tree (make-leaf 'D 1)
                                   (make-leaf 'C 1)))))
(define sample-message '(0 1 1 0 0 1 0 1 0 1 1 1 0))
\end{scheme}

\code{decode}手続を用いてメッセージを複合し結果を与えよ。
\end{quote}

\begin{quote}
\heading{\phantomsection\label{Exercise 2.68}Exercise 2.68:} 
\code{encode}手続は引数としてメッセージと木を取り、符号化されメッセージを与える
ビットのリストを生成する。

\begin{scheme}
(define (encode message tree)
  (if (null? message)
      '()
      (append (encode-symbol (car message) tree)
              (encode (cdr message) tree))))
\end{scheme}


\code{encode\-/symbol}はあなたが書かねばならぬ手続である。与えられた木に従って与えられた
シンボルの符号化を行いビットのリストを返す。\code{encode\-/symbol}をもしシンボルが
その木に存在しない場合にエラーを発するように設計せよ。あなたの手続を\link{Exercise 2.67}
で得られた結果とサンプルの木で符号化することでテストし、元のサンプルメッセージと
同じであるか確認せよ。
\end{quote}

\begin{quote}
\heading{\phantomsection\label{Exercise 2.69}Exercise 2.69:} 
以下の手続はその引数としてシンボルと頻度のペアのリストを取り(どのシンボルも1つ
より多くのペアには存在しない)、ハフマンアルゴリズムに従いハフマン符号化木を生成する。

\begin{scheme}
(define (generate-huffman-tree pairs)
  (successive-merge (make-leaf-set pairs)))
\end{scheme}



\code{make\-/leaf\-/set}は上で与えられた手続でペアのリストを葉の順序有り集合へと変換する。
\code{successive\-/merge}はあなたが書かなければならぬ手続である。\code{make\-/code\-/tree}を
用いて集合の重みが最小の要素を残り要素が1つになるまで繰り返しマージせよ。残った一要素こそ
が望まれたハフマン木である。(この手続は少しトリッキーであるがそんなに複雑ではない。
もしあなたが自分の設計が複雑だと思うなら、ほとんど確実に何かを間違えている。順序有りの
集合表現を用いているという事実から著しい利点を得ることができる。)
\end{quote}

\begin{quote}
\heading{\phantomsection\label{Exercise 2.70}Exercise 2.70:} 
以下の8つのシンボルによるアルファベット(文字体系)と関連する頻度は効率的に1950年代のロック
の歌を符号化するために設計された。(``アルファベット''の``シンボル''は単体の
文字である必要は無いことに注意せよ。)

\begin{example}
A    2   GET 2   SHA 3   WAH 1
BOOM 1   JOB 2   NA 16   YIP 9
\end{example}


\code{generate\-/huffman\-/tree}(\link{Exercise 2.69}参照)を用いて対応するハフマン木を
生成し、\code{encode} (\link{Exercise 2.68}参照)を用いて以下のメッセージを符号化
せよ。

\begin{example}
Get a job
Sha na na na na na na na na
Get a job
Sha na na na na na na na na
Wah yip yip yip yip yip yip yip yip yip
Sha boom
\end{example}

符号化には何ビットが必要であるか? この歌をもし固定長符号を8つシンボルによる
アルファベットに用いた時、最小で何ビットが必要であるか?
\end{quote}

\begin{quote}
\heading{\phantomsection\label{Exercise 2.71}Exercise 2.71:} 
\( n \)個のシンボルのアルファベットに対するハフマン木を持っているとする。
各シンボルに関連する頻度は\( 1, 2, 4, \dots, 2^{n-1} \)だとしよう。
\( n=5 \)、\( n=10 \)の場合の木をスケッチせよ。そのような(任意の\( n \)に対する)
木において、最も頻度の高いシンボルを符号化するには何ビットが必要であるか?
最も頻度の低いシンボルに対してはいくらか?
\end{quote}

\begin{quote}
\heading{\phantomsection\label{Exercise 2.72}Exercise 2.72:} 
あなたが\link{Exercise 2.68}にて設計した符号化手続について考える。シンボルを符号化する
のに必要なステップ数の増加のオーダーはいくらか? 各ノードに辿りついた時にシンボルリスト
を検索するのに必要なステップ数を含めることを確認するように。この質問の一般的な解答は
難しい。\( n \)個のシンボルの相対頻度が\link{Exercise 2.71}で説明された特別な場合に
ついて考えよ。そしてアルファベットにおける最大頻度と最小頻度のシンボルを符号化するのに
必要なステップ数の(\( n \)の関数としての)増加のオーダーを与えよ。
\end{quote}

\section{抽象データの多重表現}
\label{Section 2.4}

ここまでデータ抽象化について紹介してきました。これはプログラムが操作するデータ
オブジェクトの実装における選択に独立して多くのプログラムが指定され得る様な方法で
システムを構造化するメソドロジ(方法論)でした。例えば\link{Section 2.1.1}にてどのように
分数を用いるプログラムの設計タスクを、複雑なデータを構築するためのコンピュータ言語の
プリミティブなメカニズムを用いて分数を実装するタスクから分離するのかについて学び
ました。鍵となる考えは抽象化バリアを建てること---分数の使用法をそれらの根底にある
リスト構造を用いた表現から分離することでした。同様の抽象化バリアは分数演算(
\code{add\-/rat}, \code{sub\-/rat}, \code{mul\-/rat}, \code{div\-/rat})を実行する手続の詳細を
分数を用いる``より高いレベル''の手続から分離します。結果のプログラムは\link{Figure 2.1}
で示される構造を持ちます。

これらのデータ抽象化バリアは複雑さをコントロールする強力なツールです。
基礎をなすデータオブジェクトの表現を分離することで、大きなプログラムを設計する
タスクを小さなタスクに分割し別々に実行することを可能にします。しかしこの種の
データ抽象化はまだ十分に強力ではありません。データオブジェクトに対して
``基礎をなす表現''について話すことが常に意味があるとは限らないためです。



一例として、あるデータオブジェクトに対しては便利な表現が1つよりも多くあるかも
しれません。そして私達はシステムを複数の表現を扱えるように設計したいと願う
でしょう。簡単な例を得るために、複素数が2つのほとんど同様な方法で表現される
としましょう。(実数と虚数から成る)直行形式と(大きさと角度から成る)極形式です。
ある時は直行形式がより適切で、ある時には極形式がより適切になります。実際に
複素数が両者の方法にて表現され、複素数を操作する手続がどちらの表現でも働くこと
ができるシステムを考えることは完全に適切であろうと思われます。



より重要なこととして、プログラミングシステムは時折、長期間に渡り多くの人々が
働くことにより設計されるため、長い間仕様変更にさらされます。
そのような環境では、データ表現の選択を前もって同意することが誰にとっても可能と
いうことは単純に有り得ません。そのため使用から表現を分離するデータ抽象化バリア
に加えて、お互いから異なる設計の選択を分離し、単一のプログラム内にて異なる選択の
共存を許す抽象化バリアを必要とします。さらに、巨大なプログラムは時折、以前から
存在する独立して設計されたモジュールを組み合わせることで作成されるため、
プログラマにモジュールを\newterm{additively}(\jnewterm{付加的})に組み合わせて巨大なシステムに
することを許可するための約束事が必要です。それはつまりこれらのモジュールを再設計、
または再実装する必要が無いようにするためです。


この節ではプログラムの異なるパーツにより異なる方法で表現され得るデータをどのように
処理するかについて学びます。これは\newterm{generic procedures}(\jnewterm{ジェネリック手続})---
二種類以上の方法で表現され得るデータを処理可能な手続の構築が必要となります。
ジェネリック手続を構築する主なテクニックは\newterm{type tags}(\jnewterm{タイプタグ})を持つ
データオブジェクトを利用して処理することになります。それはつまりデータオブジェクト自身が
どのように処理されるべきであるかについての情報を明示的に含むことです。
また\newterm{data-directed}(\jnewterm{データ適従})プログラミングについても議論します。これは強力、かつ便利な
ジェネリック命令を用いて付加的にシステムを組み立てる実装戦略です。


簡単な複素数の例から始めます。どのようにタイプタグとデータに従うスタイルが
抽象的な``複素数''データオブジェクトの概念を維持しながら
複素数の直交形式と極形式の表現を分けて設計を行うことを可能にしているかについて学びます。
私達はこれを、複素数に対する数値演算手続(\code{add\-/complex}, \code{sub\-/complex},
\code{mul\-/complex}, \code{div\-/complex})を、複素数がどのように表現されるのかから独立して
複素数の部分にアクセスするジェネリックなセレクタを用いて定義することで達成します。
結果としての複素数システムは\link{Figure 2.19}に示されるとおり、2つの異なる種類の
抽象化バリアを含みます。``水平方向''の抽象化バリアは\link{Figure 2.1}
と同じ役割を演じます。それらは``高レベル''の命令を``低レベル''の表現から分離します。
それに加えて``垂直方向''のバリアが存在し、代替的な表現を分離して設計しインストールする
能力を与えます。

\begin{figure}[tb]
\phantomsection\label{Figure 2.19}
\centering
\begin{comment}
\begin{quote}
\heading{Figure 2.19:} Data-abstraction barriers in the complex-number system.

\begin{example}
           Programs that use complex numbers
  +-------------------------------------------------+
--| add-complex sub-complex mul-complex div-complex |--
  +-------------------------------------------------+
              Complex arithmetic package
---------------------------+---------------------------
          Rectangular      |         Polar
        representation     |     representation
---------------------------+---------------------------
    List structure and primitive machine arithmetic
\end{example}
\end{quote}
\end{comment}
\includegraphics[width=108mm]{fig/chap2/Fig2.19a.pdf}
\begin{quote}
\heading{Figure 2.19:} 複素数システムのデータ抽象化バリア
\end{quote}
\end{figure}

\noindent
\link{Section 2.5}にてどのようにタイプタグとデータ適従スタイルを用いてジェネリックな数値演算パッケージを
開発するかについて示します。これは全ての種類の``数値''を操作するのに用いることができる手続(\code{add},
\code{mul}, その他)を提供します。\link{Section 2.5.3}ではどのようにして記号代数を実行するジェネリックな
数値演算をシステム内にて用いるかについて示します。


\subsection{複素数のための表現}
\label{Section 2.4.1}
\label{Sec. 2.4.1}



ジェネリックな命令を用いる単純な代わりに非現実的なプログラムの例として複素数上で
数値演算命令を実行するシステムを開発します。順序有りペアとしての複素数に対する
2つのもっともらしい表現について議論することから始めます。直行形式(実数部と虚数部)と
極形式(大きさと角度)です。\footnote{実際の計算システムにおいては直行形式のほうが
極形式よりも多くの場合には好まれます。直行形式と極形式の間の変換における丸め誤差
のためです。これがなぜ複素数システムのサンプルが非現実的であるかの理由です。
それにもかかわらず、この例はジェネリック命令を用いたシステムの設計の明確な説明を
提供し、またこの章の中で後に開発されるより実質的なシステムに対する良い導入部で
あります。}
\link{Section 2.4.2}がどのようにして両方の表現がタイプタグとジェネリック命令の使用を通して
単一のシステム内にて共存できるように作成され得るのかについてを示します。


分数と同様に、複素数は自然に順序有りペアとして表現されます。複素数の集合は
2つの直行する軸を持つ二次元空間として考えることができます。この視点から複素数
\( z = x + iy \) (ここで \( i^2 = -1 \))はその平面中の実数座標が\( x \)かつ、
虚数座標が\( y \)の点として考えることが可能です。
複素数の和はこの表現において座標の和と還元できます。
\begin{comment}

\begin{example}
Real-part(z_1 + z_2) = Real-part(z_1) + Real-part(z_2)

Imaginary-part(z_1 + z_2) = Imaginary-part(z_1) + Imaginary-part(z_2)
\end{example}

\end{comment}
\begin{displaymath}
%  \eqalign{
% \hbox{Real-part} (z_1 + z_2)\; 		&= 
% 	\hbox{ Real-part} (z_1)\; + \hbox{ Real-part} (z_2), \cr
% \hbox{Imaginary-part} (z_1 + z_2)\; 	&= 
% 	\hbox{ Imaginary-part} (z_1)\; + \hbox{ Imaginary-part} (z_2). \cr 
% } 
\begin{array}{r@{{}={}}l}
  \hbox{Real-part} (z_1 + z_2)\; 	& 
	\hbox{ Real-part} (z_1)\; + \hbox{ Real-part} (z_2), \\
  \hbox{Imaginary-part} (z_1 + z_2)\; 	& 
	\hbox{ Imaginary-part} (z_1)\; + \hbox{ Imaginary-part} (z_2). 
\end{array}
\end{displaymath}

\begin{figure}[tb]
\phantomsection\label{Figure 2.20}
\centering
\begin{comment}
\heading{Figure 2.20:} Complex numbers as points in the plane.

\begin{example}
 Imaginary
    ^
    |
  y |.........................* z = x + ?y = r e^(?A)
    |                    __-- .
    |                __--     .
    |          r __--         .
    |        __--             .
    |    __-- \               .
    |__--    A |              .
----+----------+-------------------> Real
                              x
\end{example}
\end{comment}
\includegraphics[width=79mm]{fig/chap2/Fig2.20.pdf}
\par\bigskip
\noindent
\heading{Figure 2.20:} 平面上の点としての複素数
\end{figure}

\noindent
複素数をかけ算する場合、複素数を大きさと角度(\link{Figure 2.20}内の\( r \)と\( A \))
としての極形式の表現を用いて考える方がより自然です。2つの複素数の積は一方の複素数を
もう一方の長さで延し次にもう一方の角度の分、回転することで得られるベクトルになります。
\begin{comment}

\begin{example}
Magnitude(z_1 * z_2) = Magnitude(z_1) * Magnitude(z_2)

Angle(z_1 * z_2) = Angle(z_1) + Angle(z_2)
\end{example}

\end{comment}
\begin{displaymath}
%  \eqalign{
% 	\hbox{Magnitude} (z_1 \cdot z_2)\; 	&= 
% 		\hbox{ Magnitude} (z_1)\; \cdot \hbox{ Magnitude} (z_2), \cr
% 	\hbox{Angle} (z_1 \cdot z_2)\; 		&= 
% 		\hbox{ Angle} (z_1)\; + \hbox{ Angle} (z_2). \cr
% } 
\begin{array}{r@{{}={}}l}
	\hbox{Magnitude} (z_1 \cdot z_2)\; 	& 
		\hbox{ Magnitude} (z_1)\; \cdot \hbox{ Magnitude} (z_2), \\
	\hbox{Angle} (z_1 \cdot z_2)\; 		& 
		\hbox{ Angle} (z_1)\; + \hbox{ Angle} (z_2). 
\end{array}
\end{displaymath}
\begin{displaymath}
\begin{array}{r@{{}={}}l}
	\hbox{大きさ} (z_1 \cdot z_2)\; 	&
		\hbox{ 大きさ} (z_1)\; \cdot \hbox{ 大きさ} (z_2), \\
	\hbox{角度} (z_1 \cdot z_2)\; 		&
		\hbox{ 角度} (z_1)\; + \hbox{ 角度} (z_2).
\end{array}
\end{displaymath}

\noindent
従って複素数には2つの異なる表現が存在し、それぞれは異なる操作に適しています。
けれども、複素数を用いるプログラムを書いている誰かさんの視点からは、データ抽象化の
主義が複素数を操作するための全ての命令はどの表現がコンピュータにより用いられるかに
係らず存在するべきだと提案します。例えば直行形式にて指定される複素数の大きさを
求められることはしばしば便利であります。同様に極形式にて指定される複素数の実数部を
決定できることも時折便利であります。


そのようなシステムを設計するために、\link{Section 2.1.1}にて分数パッケージの設計において従った
のと同じデータ抽象化戦略に従えます。複素数上の命令が4つのセレクタを使用して実装される
と想定します。\code{real\-/part}, \code{imag\-/part}, \code{magnitude}, \code{angle}です。
また複素数を構築する2つの手続を持っているとも想定します。\code{make\-/from\-/real\-/imag}は指定
された実数部と虚数部を持つ複素数を返し、\code{make\-/from\-/mag\-/ang}は指定された大きさと角度
を持つ複素数を返します。これらの手続は任意の複素数に対し同じ特性を持ちます。

\begin{scheme}
(make-from-real-imag (real-part z) (imag-part z))
\end{scheme}

\noindent
と

\begin{scheme}
(make-from-mag-ang (magnitude z) (angle z))
\end{scheme}

\noindent
の両方が\code{z}に等しい複素数を生成します。



これらのコンストラクタとセレクタを用いて、\link{Section 2.1.1}で分数に対して行ったのと全く
同様に、コンストラクタとセレクタにて指定された``抽象データ''を用いて複素数上での
数値演算を実装可能です。前述の式にて示されたように、複素数の和と差は実数部と虚数部
を用いることで、また複素数の積と商は大きさと角度を用いることで実装できます。

\begin{scheme}
(define (add-complex z1 z2)
  (make-from-real-imag (+ (real-part z1) (real-part z2))
                       (+ (imag-part z1) (imag-part z2))))
\end{scheme}

\begin{scheme}
(define (sub-complex z1 z2)
  (make-from-real-imag (- (real-part z1) (real-part z2))
                       (- (imag-part z1) (imag-part z2))))
\end{scheme}

\begin{scheme}
(define (mul-complex z1 z2)
  (make-from-mag-ang (* (magnitude z1) (magnitude z2))
                     (+ (angle z1) (angle z2))))
\end{scheme}

\begin{scheme}
(define (div-complex z1 z2)
  (make-from-mag-ang (/ (magnitude z1) (magnitude z2))
                     (- (angle z1) (angle z2))))
\end{scheme}

\noindent
複素数パッケージを完了させるためには表現を選択し、コンストラクタとセレクタを
プリミティブな数値とリスト構造を用いて実装しなければなりません。これを行うために
2つ明らかな方法があります。``直行形式''の複素数はペア(実数部, 虚数部)として表現し、
また極形式はペア(大きさ, 角度)にて表現します。どちらを選択するべきでしょうか?


異なる選択を具体的にするために、二人のプログラマ、Ben BitdiddleとAlyssa P. Hacker
がいると想像して下さい。二人は複素数システムのための表現を独立して設計します。
Benは複素数を直行形式にて表現することを選択しました。この選択により現状として
複素数を与えられた実数部と虚数部から構築するため、実数部と虚数部を複素数から選択する
のは直接的です。大きさと角度を求めるためには、または複素数を与えられた大きさと角度から
構築するために彼は三角法の関係を用いました。

\begin{comment}

\begin{example}
                      __________
x = r cos A     r = ./ x^2 + y^2

y = r sin A     A = arctan(y,x)
\end{example}

\end{comment}
\begin{displaymath}
%  \eqalign{
% 	x = r \cos A, \qquad 	& r = \sqrt{x^2 + y^2}, \cr
% 	y = r \sin A, \qquad 	& A = \arctan(y, x), \cr
% } 
\begin{array}{r@{{}={}}lr@{{}={}}l}
	x & r \cos A, \qquad 	& r & \sqrt{x^2 + y^2}, \\
	y & r \sin A, \qquad 	& A & \arctan(y, x), 
\end{array}
\end{displaymath}
\noindent
これは実数部と虚数部\( (x, y) \)を大きさと角度\( (r, A) \)へと関係づけます。\footnote{
ここで参照されたアークタンジェント関数は、Schemeの\code{atan}手続にて計算されますが、2つの
引数\( y \)と\( x \)を取りタンジェントが\( y / x \)となる角度を返すように定義
されました。引数の符号が角度の象限を決定します。}
Benの表現は従って以下のセレクタとコンストラクタにより与えられます。

\begin{scheme}
(define (real-part z) (car z))
(define (imag-part z) (cdr z))
(define (magnitude z)
  (sqrt (+ (square (real-part z))
           (square (imag-part z)))))
(define (angle z)
  (atan (imag-part z) (real-part z)))
(define (make-from-real-imag x y) (cons x y))
(define (make-from-mag-ang r a)
  (cons (* r (cos a)) (* r (sin a))))
\end{scheme}

\noindent
一方、Alyssaは複素数を極形式にて表現することを選択しました。彼女にとっては大きさと
角度を選択するのは直接的です。しかし実数部と虚数部を得るためには三角法の関係を用い
ねばなりません。Alyssaの表現は次のとおりです。

\begin{scheme}
(define (real-part z) (* (magnitude z) (cos (angle z))))
(define (imag-part z) (* (magnitude z) (sin (angle z))))
(define (magnitude z) (car z))
(define (angle z) (cdr z))
(define (make-from-real-imag x y)
  (cons (sqrt (+ (square x) (square y)))
        (atan y x)))
(define (make-from-mag-ang r a) (cons r a))
\end{scheme}

\noindent
データ抽象化の規律は\code{add\-/complex}, \code{sub\-/complex}, \code{mul\-/complex},
\code{div\-/complex}の同じ実装がBenの表現とAlyssaの表現のどちらに対してもうまく
いくことを保証します。

\subsection{タグ付きデータ}
\label{Section 2.4.2}



データ抽象化を考え方の1つは``最小責務の原則''の適用としてです。\link{Section 2.4.1}の
複素数システムの実装において、私達はBenの直行形式表現とAlyssaの極形式表現の
どちらも使用することができました。セレクタとコンストラクタにより形成された
抽象化バリアが最後の可能な瞬間にデータオブジェクトに対する具体的な表現の
選択に従うことを可能にしています。従ってシステム設計において最高の柔軟性を
維持することができるのです。



最小責務の原則はさらにもっと高みへと到達することができます。もし私達が望めば、
セレクタとコンストラクタを設計した``後''にさえ表現の多義性を維持することが
可能です。そしてBenの表現``と''Alyssaの表現の両方の使用を選択できます。
もし両方の表現が単一のシステムに含まれる場合、極形式のデータを直行形式のデータから
識別するための何らかの方法が必要になります。そうでなけば、例えばペア(3, 4)の大きさ
を求めるよう尋ねられた場合に(数値を直行形式だと考えて)5と答えるべきか(数値が極形式
であると考えて)3と答えるべきであるのか分かりません。この識別を直接的な方法で達成する
ために\newterm{type tag}(\jnewterm{タイプタグ})---\code{rectangular}または\code{polar}のシンボル---を
各複素数の部分として導入します。すると複素数を操作せねばならない時にタグを用いてどち
らのセレクタを適用するべきか決定することができます。



タグ付きデータを操作するためにデータオブジェクトからタグと(複素数の場合には極形式、
または直行形式の)実際のコンテンツを抽出する手続\code{type\-/tag}と\code{contents}を持つと
想定します。またタグとコンテンツを取りタグ付きデータオブジェクトを生成する
手続\code{attach\-/tag}を仮定します。これを実装する直接的な方法は普通のリスト構造を用いる
ことです。

\begin{scheme}
(define (attach-tag type-tag contents)
  (cons type-tag contents))
(define (type-tag datum)
  (if (pair? datum)
      (car datum)
      (error "Bad tagged datum: TYPE-TAG" datum)))
(define (contents datum)
  (if (pair? datum)
      (cdr datum)
      (error "Bad tagged datum: CONTENTS" datum)))
\end{scheme}

\noindent
これら手続を用いて述語\code{rectangular?}と\code{polar?}を定義し、直行形式と
極形式のそれぞれを認識することができます。

\begin{scheme}

(define (rectangular? z) (eq? (type-tag z) 'rectangular))
(define (polar? z) (eq? (type-tag z) 'polar))
\end{scheme}

\noindent
タイプタグを用いてBenとAlyssaはこれで彼等のコードを変更し2つの異なる表現が同じ
システム内にて共存させることができるようになりました。Benが複素数を構築する度に
彼は直行形式であるとタグを付けます。Alyssaが複素数を構築する度に、彼女はそれを
極形式であるとタグを付けます。加えて、BenとAlyssaは手続の名前が衝突しないように
確認しなければなりません。これを行う1つの方法としてBenは彼の各表現手続に接尾辞
\code{rectangular}を追加し、Alyssaは彼女の手続に対し\code{polar}を付け加えます。
以下はBenの\link{Section 2.4.1}から改正した直行形式表現です。

\begin{scheme}
(define (real-part-rectangular z) (car z))
(define (imag-part-rectangular z) (cdr z))
(define (magnitude-rectangular z)
  (sqrt (+ (square (real-part-rectangular z))
           (square (imag-part-rectangular z)))))
(define (angle-rectangular z)
  (atan (imag-part-rectangular z)
        (real-part-rectangular z)))
(define (make-from-real-imag-rectangular x y)
  (attach-tag 'rectangular (cons x y)))
(define (make-from-mag-ang-rectangular r a)
  (attach-tag 'rectangular
              (cons (* r (cos a)) (* r (sin a)))))
\end{scheme}

\noindent
そして以下はAlyssaの改訂版極形式表現です。

\begin{scheme}
(define (real-part-polar z)
  (* (magnitude-polar z) (cos (angle-polar z))))
(define (imag-part-polar z)
  (* (magnitude-polar z) (sin (angle-polar z))))
(define (magnitude-polar z) (car z))
(define (angle-polar z) (cdr z))

(define (make-from-real-imag-polar x y)
  (attach-tag 'polar
              (cons (sqrt (+ (square x) (square y)))
                    (atan y x))))
(define (make-from-mag-ang-polar r a)
  (attach-tag 'polar (cons r a)))
\end{scheme}

\noindent
各ジェネリックなセレクタは引数のタグをチェックし、そのタイプのデータを扱うのに
適切な手続を呼び出す様に実装されます。例として、複素数の実数部を得る場合、
\code{real\-/part}はタグを確かめBenの\code{real\-/part\-/rectangular}かAlyssaの\code{real\-/part\-/polar}の
どちらを使うのかを決定します。どちらの場合でも\code{contents}を用いて生のタグの無いデータを
抽出し直交形式、または極形式の手続を必要に応じて呼び出します。

\begin{scheme}
(define (real-part z)
  (cond ((rectangular? z)
         (real-part-rectangular (contents z)))
        ((polar? z)
         (real-part-polar (contents z)))
        (else (error "Unknown type: REAL-PART" z))))

(define (imag-part z)
  (cond ((rectangular? z)
         (imag-part-rectangular (contents z)))
        ((polar? z)
         (imag-part-polar (contents z)))
        (else (error "Unknown type: IMAG-PART" z))))

(define (magnitude z)
  (cond ((rectangular? z)
         (magnitude-rectangular (contents z)))
        ((polar? z)
         (magnitude-polar (contents z)))
        (else (error "Unknown type: MAGNITUDE" z))))
(define (angle z)
  (cond ((rectangular? z)
         (angle-rectangular (contents z)))
        ((polar? z)
         (angle-polar (contents z)))
        (else (error "Unknown type: ANGLE" z))))
\end{scheme}

\noindent
複素数演算命令を実装するためには\link{Section 2.4.1}から同じ手続\code{add\-/complex}, 
\code{sub\-/complex}, \code{mul\-/complex}, \code{div\-/complex}を使うことができます。
なぜならそれらが呼び出すセレクタはジェネリックであるためどちらの表現に対して
も働くからです。例として手続\code{add\-/complex}は今でも以下のとおりです。

\begin{scheme}
(define (add-complex z1 z2)
  (make-from-real-imag (+ (real-part z1) (real-part z2))
                       (+ (imag-part z1) (imag-part z2))))
\end{scheme}

\noindent
最後に、複素数をBenの表現とAlyssaの表現のどちらを利用して構築するのか決定
しなければなりません。妥当な選択として実数部と虚数部がある場合には直行形式を
用い、大きさと角度がある場合には極形式を用いて構築します。

\begin{scheme}
(define (make-from-real-imag x y)
  (make-from-real-imag-rectangular x y))
(define (make-from-mag-ang r a)
  (make-from-mag-ang-polar r a))
\end{scheme}

\noindent
結果としての複素数システムは\link{Figure 2.21}にて表される構造を持ちます。
システムは3つの関連する独立した部分に分離されます。複素数演算命令、Alyssaの
極形式実装、そしてBenの直行形式実装です。極形式と直行形式の実装はBenとAlyssaが
別々に働きながら書かれることが可能でした。そして両者が
抽象コンストラクタ、セレクタのインターフェイスを用いながら複素数演算手続を実装する
第三者のプログラマにより基礎を成す表現として利用されることが可能です。

\begin{figure}[tb]
\phantomsection\label{Figure 2.21}
\centering
\begin{comment}
\begin{quote}
\heading{Figure 2.21:} Structure of the generic complex-arithmetic system.

\begin{example}
    +-------------------------------------------------+
----| add-complex sub-complex mul-complex div-complex |----
    +-------------------------------------------------+
                Complex arithmetic package
                 +-----------------------+
                 | real-part   imag-part |
-----------------|                       |------------------
                 | magnitude   angle     |
                 +-----------+-----------+
           Rectangular       |          Polar
          representation     |     representation
-----------------------------+------------------------------
       List structure and primitive machine arithmetic
\end{example}
\end{quote}
\end{comment}
\includegraphics[width=108mm]{fig/chap2/Fig2.21a.pdf}
\begin{quote}
\heading{Figure 2.21:} ジェネリックな複素数演算システムの構造
\end{quote}
\end{figure}

\noindent
各データオブジェクトはその型にてタグ付けられているので、セレクタはデータに対し
ジェネリックな方法で操作します。これは各セレクタがそれが適用される個々のデータの
型に従う振舞を持つように定義されているということです。分けられた表現を結び付ける
ための一般的なメカニズムについて注意して下さい。与えられた表現実装(例えばAlyssaの
極形式パッケージ)の中では複素数は型の無いペア(大きさ, 角度)です。ジェネリックな
セレクタが極形式の型(タイプ)の複素数を操作する時、タグを取り中身をAlyssaのコードに
渡します。反対にAlyssaが通常の使用のために数値を構築する時、彼女が型でタグを付ける
ことでより高いレベルの手続により適切に認識されることができます。データオブジェクトが
あるレベルから別のレベルへと渡されるに従い、このタグの取り付けと除去の規律が重要な
組織的戦略となります。\link{Section 2.5}にてこれについて学びます。


\subsection{データ適従プログラミングと付加性}
\label{Section 2.4.3}



データの型をチェックし適切な手続を呼ぶ一般的な戦略は
\newterm{dispatching on type}(\jnewterm{タイプ別処理})と呼ばれる。これはシステム設計において
モジュール方式を得るための強力な戦略です。一方で\link{Section 2.4.2}のような呼出の実装は
2つの明らかな弱点が存在します。1つはジェネリックインターフェイス手続(\code{real\-/part},
\code{imag\-/part}, \code{magnitude}, \code{angle})は全ての異なる表現について知っていなければ
なりません。例えば複素数に対する新しい表現を複素数システムに組み入れたいとしましょう。
この新しい表現を型にて識別し、次に全てのジェネリックインターフェイス手続に新しい型を
チェックする条項を追加し、その表現に対する適切なセレクタを適用する必要が出てくるでしょう。


もう1つのこのテクニックの弱点は例え個々の表現が別々に設計できたとしても、システム全体の
中でどの2つの手続も同じ名前を持たないことを保証せねばなりません。これがなぜBenとAlyssa
が\link{Section 2.4.1}の彼等の元の手続の名前を変更しなければいけないかの理由でした。



両者の弱点の根底にある問題はジェネリックインターフェイスを実装するためのテクニック
が\newterm{additive}(\jnewterm{付加的})でないことです。ジェネリックセレクタ手続を実装する人は
これらの手続を新しい表現がインストールされる度に変更せねばならず、また個々の表現を
接続する人々は名前衝突が起こらぬ様に彼等のコードを変更せねばなりません。
これらのケースのそれぞれでコードに対して加えられなければならない変更は簡単ですが、
それでも必ず行わねばならず、不自由さと障害の原因となります。これは複素数システムに
対しては現時点ではあまり大きな問題ではありません。しかしただ2つではなく数百もの異なる
表現が複素数に対して存在すると仮定してみて下さい。その上どのプログラマも全ての
インターフェイス手続や全ての表現について知らないと想定してみて下さい。問題は現実的で
あり大規模なデータベース管理システムのようなプログラムでは必ず解決される必要があります。



私達に必要なものはより一層のシステム設計のモジュール化のための手段です。
これは\newterm{data-directed programming}(\jnewterm{データ適従プログラミング})として知られる
プログラミングテクニックにより提供されます。データ適従プログラミングがどのように
働くかを理解するためには、
異なる型の集合に対して共通なジェネリックな命令の集合を扱う度に、
実際に予想される命令を1つの軸に、予想される型をもう一方の軸に持つ二次元の表に取り組み、
その観察結果から始めます。
表の項目には与えられた各引数の型に対する各命令を実装する手続です。
前の章にて開発された複素数システムでは命令の名前、データタイプ、実際の手続の間の対応は
ジェネリックなインターフェイス手続の種々の条件節の間に広がっています。
しかし同じ情報が\link{Figure 2.22}の中に示されるように1つのテーブルの中に組込まれることが
できたはずです。



データ適従プログラミングはそのようなテーブルと直接連携するためのプログラム設計の
テクニックです。以前は私達は複素数演算コードをそれぞれが明示的に型に従う呼び出し
を行う手続の集合としての2つの表現パッケージと接続するメカニズムを実装しました。
ここではインターフェイスを命令の名前と引数タイプの組み合わせをテーブルの中から調べ
適用すべき正しい手続を見つける単一の手続として実装します。そして次にその手続を
引数の中身に対して適用します。これを行なえば、システムに対して新しい
表現パッケージの追加するために既存の手続に何の変更を行う必要もありません。
必要なのは表に新しい項目を追加することです。

\begin{figure}[tb]
\phantomsection\label{Figure 2.22}
\centering
\begin{comment}
\begin{quote}
\heading{Figure 2.22:} Table of operations for the complex-number system.

\begin{example}
           |               Types
Operations | Polar           | Rectangular
===========+=================+======================
real-part  | real-part-polar | real-part-rectangular
imag-part  | imag-part-polar | imag-part-rectangular
magnitude  | magnitude-polar | magnitude-rectangular
angle      | angle-polar     | angle-rectangular
\end{example}
\end{quote}
\end{comment}
\includegraphics[width=102mm]{fig/chap2/Fig2.22.pdf}
\begin{quote}
\heading{Figure 2.22:} 複素数システムの命令表
\end{quote}
\end{figure}

\noindent
この計画を実装するために、2つの手続\code{put}と\code{get}を命令と型のテーブルを操作するために
持っていると仮定します。

\begin{itemize}

\item
\( \hbox{\tt(put}\;\langle \)\var{op}\( \kern0.1em\rangle\;\langle \)\var{type}\( \kern0.08em\rangle\;\langle \)\var{item}\( \kern0.08em\rangle\hbox{\tt)} \)は\( \langle \)\var{item}\( \kern0.08em\rangle \)をテーブルに挿入し、
\( \langle \)\var{op}\( \kern0.1em\rangle \)と\( \langle \)\var{type}\( \kern0.08em\rangle \)で索引付けられる

\item
\( \hbox{\tt(get}\;\langle \)\var{op}\( \kern0.1em\rangle\;\langle \)\var{type}\( \kern0.08em\rangle\hbox{\tt)} \)は
\( \langle \)\var{op}\( \kern0.08em\rangle \), \( \langle \)\var{type}\( \kern0.08em\rangle \)の項目をテーブルから探し
見つかった項目を返す。 もし見つからない場合には\code{get}はfalseを返す

\end{itemize}

\noindent
今のところは\code{put}と\code{get}が私達の言語に含まれていると仮定しましょう。
\link{Chapter 3} (\link{Section 3.3.3})においてこれらと他のテーブル操作の命令をどのように実装するか
について学びます。



ここからはデータ適従プログラミングが複素数システムにおいてどのように使用できるかについて
示します。直行形式表現を開発したBenは彼が元々行ったとおりにコードを実装しました。
彼は手続の集合、つまりは\newterm{package}(\jnewterm{パッケージ})を定義し、システムに
どのように直行形式の数値を取り扱うかを教えるテーブルに項目を追加する
ことで、パッケージをシステムの残りに対して接続します。
これは以下の手続を呼び出すことにより達成されます。

\begin{scheme}
(define (install-rectangular-package)
  ~\textrm{;; internal procedures}~
  (define (real-part z) (car z))
  (define (imag-part z) (cdr z))
  (define (make-from-real-imag x y) (cons x y))
  (define (magnitude z)
    (sqrt (+ (square (real-part z))
             (square (imag-part z)))))
  (define (angle z)
    (atan (imag-part z) (real-part z)))
  (define (make-from-mag-ang r a)
    (cons (* r (cos a)) (* r (sin a))))

  ~\textrm{;; interface to the rest of the system}~
  (define (tag x) (attach-tag 'rectangular x))
  (put 'real-part '(rectangular) real-part)
  (put 'imag-part '(rectangular) imag-part)
  (put 'magnitude '(rectangular) magnitude)
  (put 'angle '(rectangular) angle)
  (put 'make-from-real-imag 'rectangular
       (lambda (x y) (tag (make-from-real-imag x y))))
  (put 'make-from-mag-ang 'rectangular
       (lambda (r a) (tag (make-from-mag-ang r a))))
  'done)
\end{scheme}

\noindent
この中の内部手続はBenが\link{Section 2.4.1}にて分離を行った時に彼が書いたものと同じ
手続であることに注意して下さい。これらをシステムの残りに接続するためには
全く変更が必要がありません。さらに、これらの手続の定義はインストールを行う
手続の内部であるため、Benは直行形式パッケージの外部の他の手続に対して
名前の衝突が起こることを全く心配する必要がありません。
これらをシステムの残りに対し接続するために、Benは彼の\code{real\-/part}手続を
命令名\code{real\-/part}と型\code{(rectangular)}の元にインストールしました。
そして他のセレクタに対しても同様に行いました。\footnote{
私達はシンボル\code{rectangular}ではなくリスト\code{(rectangular)}を用いました。
全てが同じ型ではない複数の引数を伴なう命令の可能性を考慮するためです。}
このインターフェイスはまた外部システムにより利用されるコンストラクタも
定義します。\footnote{コンストラクタがその下にインストールされる型は
リストである必要がありません。なぜならコンストラクタは常にある特定の型の
オブジェクトを作成するために使用されるためです。}これらはBenの内部定義
コンストラクタと全く同じです。ただしタグを付加することが異なります。


Alyassaの極形式パッケージも同様です。

\begin{scheme}
(define (install-polar-package)
  ~\textrm{;; internal procedures}~
  (define (magnitude z) (car z))
  (define (angle z) (cdr z))
  (define (make-from-mag-ang r a) (cons r a))
  (define (real-part z)
    (* (magnitude z) (cos (angle z))))
  (define (imag-part z)
    (* (magnitude z) (sin (angle z))))
  (define (make-from-real-imag x y)
    (cons (sqrt (+ (square x) (square y)))
          (atan y x)))
  ~\textrm{;; interface to the rest of the system}~
  (define (tag x) (attach-tag 'polar x))
  (put 'real-part '(polar) real-part)
  (put 'imag-part '(polar) imag-part)
  (put 'magnitude '(polar) magnitude)
  (put 'angle '(polar) angle)
  (put 'make-from-real-imag 'polar
       (lambda (x y) (tag (make-from-real-imag x y))))
  (put 'make-from-mag-ang 'polar
       (lambda (r a) (tag (make-from-mag-ang r a))))
  'done)
\end{scheme}

\noindent
BenとAlyssaの両者が今でも御互いに同じ名前(例えば\code{real\-/part})にて定義された
彼等の元々の手続を使用してようとも、これらの定義は今では異なる手続の内部定義(\link{Section 1.1.8}参照)
です。従って名前の衝突は起こりません。



複素数演算のセレクタは\code{apply\-/generic}と呼ばれる普遍的な``operation''手続を
用いてテーブルにアクセスします。これはジェネリックな命令を引数に対して適用します。
\code{apply\-/generic}は命令の名前と引数の型の下に表を調べ結果としての手続が存在すれば
適用します。\footnote{


\code{apply\-/generic}は\link{Exercise 2.20}で説明したドット付き
末尾記法を用います。異なるジェネリック命令は異なる数の引数を取る場合が考えられる
ためです。\code{apply\-/generic}では\code{op}がその値として\code{apply\-/generic}の第一引数を
持ち、\code{args}はその値として残りの引数のリストを持ちます。



\code{apply\-/generic}はまたプリミティブな手続\code{apply}を用います。これは2つの引数、
手続とリストを取ります。\code{apply}はリストの要素を引数として手続を適用します。
例えば、

\begin{smallscheme}
(apply + (list 1 2 3 4))
\end{smallscheme}

\noindent
は10を返します。}

\begin{scheme}
(define (apply-generic op . args)
  (let ((type-tags (map type-tag args)))
    (let ((proc (get op type-tags)))
      (if proc
          (apply proc (map contents args))
          (error
            "No method for these types: APPLY-GENERIC"
            (list op type-tags))))))
\end{scheme}

\noindent
\code{apply\-/generic}を用いることで、私達のジェネリックなセレクタを以下のように定義する
ことができます。

\begin{scheme}
(define (real-part z) (apply-generic 'real-part z))
(define (imag-part z) (apply-generic 'imag-part z))
(define (magnitude z) (apply-generic 'magnitude z))
(define (angle z) (apply-generic 'angle z))
\end{scheme}

\noindent
もし新しい表現がシステムに追加されたとしてもこれらが全く変更されないことに注意して
下さい。


またテーブルからコンストラクタを抽出することもできます。コンストラクタはパッケージの
外部プログラムにより使用でき、実数部と虚数部か大きさと角度から複素数を作ります。
\link{Section 2.4.2}にあるとおり、実数部と虚数部がある場合には直交形式で構築し、大きさと
角度がある場合には極形式にて構築します。

\begin{scheme}
(define (make-from-real-imag x y)
  ((get 'make-from-real-imag 'rectangular) x y))
(define (make-from-mag-ang r a)
  ((get 'make-from-mag-ang 'polar) r a))
\end{scheme}

\begin{quote}
\heading{\phantomsection\label{Exercise 2.73}Exercise 2.73:} 
\link{Section 2.3.2}は記号微分を行うプログラムについて説明した。

\begin{scheme}
(define (deriv exp var)
  (cond ((number? exp) 0)
        ((variable? exp) 
         (if (same-variable? exp var) 1 0))
        ((sum? exp)
         (make-sum (deriv (addend exp) var)
                   (deriv (augend exp) var)))
        ((product? exp)
         (make-sum (make-product
                    (multiplier exp)
                    (deriv (multiplicand exp) var))
                   (make-product 
                    (deriv (multiplier exp) var)
                    (multiplicand exp))))
        ~\( \dark \langle \)~~\var{\dark more rules can be added here}~~\( \dark \rangle \)~
        (else (error "unknown expression type: 
                      DERIV" exp))))
\end{scheme}



このプログラムを微分する式のタイプにより呼出を行っていると解釈することもできる。
このシチュエーションではデータの``タイプタグ''が代数演算子の記号(例えば\code{+})であり
実行される命令は\code{deriv}である。このプログラムを基本の微分手続を書き直すことで
データ適従プログラミングスタイルに変換することができる。

\begin{scheme}
(define (deriv exp var)
  (cond ((number? exp) 0)
        ((variable? exp) 
         (if (same-variable? exp var) 1 0))
        (else ((get 'deriv (operator exp)) 
               (operands exp) var))))
(define (operator exp) (car exp))
(define (operands exp) (cdr exp))
\end{scheme}

\begin{enumerate}[a]

\item
上で何が行われたのか説明せよ。なぜ手続\code{number?}と\code{variable?}をデータ適従呼出に
吸収することができないのか?

\item
和と積の微分のための手続とそれらを上記のプログラムで使用されたテーブルにインストールする
補助コードを書け。

\item
貴方の好きな追加の微分ルール、例えば指数に対する物(\link{Exercise 2.56})を選択し、
このデータ適従システムにインストールせよ。

\item
この単純な代数操作において、式の型はそれを一緒に束縛する代数演算子である。
しかし手続を逆の向きに索引付けし\code{deriv}の呼出行を以下のようにした場合、

\begin{scheme}
((get (operator exp) 'deriv) (operands exp) var)
\end{scheme}

\noindent
微分システムへの対応する変更は何が必要か?

\end{enumerate}
\end{quote}

\begin{quote}
\heading{\phantomsection\label{Exercise 2.74}Exercise 2.74:} 
Insatiable Enterprises, Inc.(強欲エンタープライズ社)は高いレベルで非集中化された
数多くの独立事業所を世界中に抱える複合企業である。社のコンピュータ施設は接続された
ばかりであるが、賢いネットワーク接続計画を用いてネットワーク全体がどのユーザに対し
ても1台のコンピュータとして現れる。強欲社の社長は初めてネットワークの機能を用い
て事業所ファイルから管理者情報を取得しようと試みたが、全ての事業所ファイルは
Schemeのデータ構造として実装されているのにも係らず、使用されている個々のデータ構造は
事業所の間で異なっていることに狼狽した。事業所長の会議が大急ぎで開催され既存の事業所の
自立性を保ちつつ本社の要求を満足できるファイル統合の戦略を探すことになった。



そのような戦略がデータ適従戦略を用いてどのように実装できるか示せ。例として各事業所
の職員記録は単一のファイルから成る従業員の名前をキーにしたレコードの集合であると
想定せよ。集合の構造は事業所毎に変わる。さらに各従業員のレコードはそれ自身が集合
(事業所毎で異なる構造)であり\code{address}と\code{salary}のような識別子の下で鍵付けられた
情報を含んでいる。具体的には

\begin{enumerate}[a]

\item
本社のために指定された従業員のレコードを指定された職員記録ファイルから取得する
\code{get\-/record}手続を実装せよ。手続は任意の事業所のファイルに適用できなければならない。
個々の事業所のファイルがどのように構造化されねばならないか説明せよ。
具体的にはどんな型の情報が提供されねばならないか

\item
本社のために任意の事業所の職員記録ファイルから与えられた職員記録から給与情報を返す
\code{get\-/salary}手続を実装せよ。
記録はこの操作が動くようどのように構造化されねばならないか?

\item
本社のために\code{find\-/employee\-/record} 手続を実装せよ。これは全ての事業所のファイルに対し
与えられた従業員のレコードを探し、レコードを返さねばならない。この手続が引数として
従業員の名前と全ての事業所のファイルのリストを与えられると仮定せよ。

\item
強欲社が新しい会社を吸収した時、どんな変更が新しい職員情報を中央システムに受け入れるため
に必要であるか?

\end{enumerate}
\end{quote}

\subsubsection*{メッセージパッシング}



データ適従プログラミングの鍵となる考えはプログラム中のジェネリックな命令を
\link{Figure 2.22}の様な命令と型のテーブルを明示的に処理することで扱うことです。
\link{Section 2.4.2}で用いたプログラミングスタイル要求された型に基く呼出を各命令がそれ自身の
呼出の世話を行うことで組織化しました。実際にこれは命令と型のテーブルを、
テーブルの行を表す各ジェネリックな操作手続を用いて行に分解します。



代替的な実装戦略はテーブルを列に分解し、データ型に基き呼び出しを行う``知的な命令''を
用いる代わりに、命令名に基づき呼び出しを行う``知的なデータオブジェクト''を用いて動かす
ものです。直行形式の複素数の様なデータオブジェクトが入力として必要な命令名を取り指定され
た命令を実行するように準備を行うことで行うことができます。そのような規律の下では
\code{make\-/from\-/real\-/imag}は以下のように書くことができます。

\begin{scheme}
(define (make-from-real-imag x y)
  (define (dispatch op)
    (cond ((eq? op 'real-part) x)
          ((eq? op 'imag-part) y)
          ((eq? op 'magnitude)
           (sqrt (+ (square x) (square y))))
          ((eq? op 'angle) (atan y x))
          (else (error "Unknown op: 
                        MAKE-FROM-REAL-IMAG" op))))
  dispatch)
\end{scheme}

\noindent
対応する\code{apply\-/generic}手続はジェネリックな命令を引数に適用しますが、ここでは
単純に命令の名前をデータオブジェクトに与えオブジェクトに仕事を行わせます。\footnote{
この構造の1つの制約は一引数のジェネリック手続のみを許容することです。}

\begin{scheme}
(define (apply-generic op arg) (arg op))
\end{scheme}

\noindent
\code{make\-/from\-/real\-/imag}により返される値は手続---内部手続\code{dispatch}のである
ことに注意して下さい。これが\code{apply\-/generic}が命令に実行を要求した時に起動
される手続です。



このプログラミングスタイルは\newterm{message passing}(\jnewterm{メッセージパッシング})と
呼ばれます。その名前はデータオブジェクトが要求された命令の名前を``メッセージ''として
受け取った要素であるというイメージから来ています。私達は既にメッセージパッシングの例を
\link{Section 2.1.3}にて見ています。その時は\code{cons}, \code{car}, \code{cdr}がデータオブジェクト無し、
手続のみでどのように定義され得るかを学びました。ここではメッセージパッシングは数学上の
トリックではなくジェネリック命令を用いてシステムを構造化するのに便利なテクニックである
ことを学びます。この章の残りではメッセージパッシングではなくデータ適従プログラミングの
使用を続け、全般的な数値演算操作について議論します。そしてそれがシミュレーションプログラム
の構造化に対して強力なツールに成り得ることを学びます。

\begin{quote}
\heading{\phantomsection\label{Exercise 2.75}Exercise 2.75:} 
コンストラクタ\code{make\-/from\-/mag\-/ang}をメッセージパッシングスタイルにて実装せよ。
この手続は上で与えられた\code{make\-/from\-/real\-/imag}と同様でなければならない。
\end{quote}

\begin{quote}
\heading{\phantomsection\label{Exercise 2.76}Exercise 2.76:} 
ジェネリックな命令を用いた巨大システムが発展するにつれ、新しい型のデータオブジェクトや
命令が必要となるかもしれない。3つの戦略---ジェネリック命令の明示的呼出、データ適従スタイル、
メッセージパッシング---のそれぞれに対して新しい型や命令を追加するために必要な
システムに対する変更について説明せよ。
どの構造化が新しい型が良く追加されるシステムに対して最も適切であるか?
どれが新しい命令が良く追加されねばならぬシステムに対して最も適切であるか?
\end{quote}

\section{ジェネリック命令を持つシステム}
\label{Section 2.5}



前の節ではデータオブジェクトが2つ以上の方法で表現されるシステムをどのように設計するか
について学んだ。鍵となる考えはデータ操作を指定するコードをいくつかの表現に対し
ジェネリックなインターフェイス手続を用いてリンクすることでした。ここではこれと同じ
考えを異なる表現上のジェネリックな命令の定義のみでなく、異なる種類の引数上の
ジェネリックな命令を定義するためにどのように用いるかについて学びます。私達は既に
いくつかの数値演算命令の異なるパッケージを見てきました。言語内に構築された
プリミティブ数値演算(\code{+}, \code{-}, \code{*}, \code{/})、\link{Section 2.1.1}の分数演算
(\code{add\-/rat}, \code{sub\-/rat}, \code{mul\-/rat}, \code{div\-/rat})、\link{Section 2.4.3}で実装した
複素数演算です。ここではデータ適従のテクニックを用いて私達がこれまでに構築した
全ての数値演算パッケージを内蔵する数値演算のパッケージを構築します。


\begin{figure}[tb]
\phantomsection\label{Figure 2.23}
\centering
\begin{comment}
\label{Figure 2.23}
\heading{Figure 2.23:} Generic arithmetic system.

\begin{example}
                        Programs that use numbers
                           +-----------------+
---------------------------| add sub mul div |-------------------
                           +-----------------+
                        Generic arithmetic package
 +-----------------+   +-------------------------+
 | add-rat sub-rat |   | add-complex sub-complex |   +---------+
-|                 |-+-|                         |-+-| + - * / |-
 | mul-rat div-rat | | | mul-complex div-complex | | +---------+
 +-----------------+ | +-------------------------+ |
      Rational       |     Complex artithmetic     |   Ordinary
     arithmetic      +--------------+--------------+  arithmetic
                     | Rectangular  |     Polar    |
---------------------+--------------+--------------+-------------
             List structure and primitive machine arithmetic
\end{example}
\end{comment}
\includegraphics[width=111mm]{fig/chap2/Fig2.23a.pdf}
\par\bigskip
\noindent
\heading{Figure 2.23:} ジェネリックな数値演算システム
\end{figure}

\noindent
\link{Figure 2.23}は私達が構築するシステムの構造を示しています。
抽象化バリアに注目して下さい。``数値''を扱う第三者の視点からはそこにあるのは
どの種類の数値が提供されても単一の手続\code{add}です。\code{add}はジェネリックインターフェイスの
部分で別々の実数演算、分数演算、複素数演算のパッケージに、数値を使用するプログラムから統一的な
アクセスを可能にします。(複素数の様な)任意の個別数値演算パッケージはそれ自身が
(直行形式と極形式の様な)異なる表現のために設計されたパッケージを結合する
(\code{add\-/complex}の様な)ジェネリックな手続を通してアクセスできます。
さらに、システムの構造は付加的なため個々の数値演算パッケージは別々に設計することが可能で、
それらを結合してジェネリックな数値演算システムを生成できます。



\subsection{ジェネリックな数値演算命令}
\label{Section 2.5.1}


ジェネリックな数値演算命令の設計タスクはジェネリックな複素数命令を設計するのと同様です。
例えば、実数上での通常の加算のプリミティブ\code{+}、分数上の\code{add\-/rat}や複素数上の
\code{add\-/complex}のように振る舞うジェネリックな加算手続\code{add}を持ちたいとします。
\code{add}と他のジェネリックな数値演算命令を\link{Section 2.4.3}にて複素数に対するジェネリックな
セレクタを実装するのに用いたのと同じ戦略に従うことで実装することが可能です。
全ての種類の数値にタイプタグをアタッチすることでジェネリック手続にその引数のデータタイプ
に従って適切なパッケージを呼び出す理由とします。

ジェネリックな数値演算は以下のように定義されます。

\begin{scheme}
(define (add x y) (apply-generic 'add x y))
(define (sub x y) (apply-generic 'sub x y))
(define (mul x y) (apply-generic 'mul x y))
(define (div x y) (apply-generic 'div x y))
\end{scheme}

\noindent
\newterm{ordinary}(通常の)数値を扱うためのパッケージをインストールすることで始めます。
これは私達の言語のプリミティブな数値のことです。これらにシンボル\code{scheme\-/number}で
タグを付けます。このパッケージ内の数値演算命令はプリミティブな数値演算手続です。
(そのためタグの無い数値を扱うために拡張手続を定義する必要はありません)。
これらの命令はそれぞれが2つの引数を取るためリスト\code{(scheme\-/number scheme\-/number)}を
鍵にしてテーブルにインストールされます。

\begin{scheme}
(define (install-scheme-number-package)
  (define (tag x) (attach-tag 'scheme-number x))
  (put 'add '(scheme-number scheme-number)
       (lambda (x y) (tag (+ x y))))
  (put 'sub '(scheme-number scheme-number)
       (lambda (x y) (tag (- x y))))
  (put 'mul '(scheme-number scheme-number)
       (lambda (x y) (tag (* x y))))
  (put 'div '(scheme-number scheme-number)
       (lambda (x y) (tag (/ x y))))
  (put 'make 'scheme-number (lambda (x) (tag x)))
  'done)
\end{scheme}

\noindent
scheme-numberパッケージのユーザは(タグ付きの)普通の数値を手続を用いて作成します。

\begin{scheme}
(define (make-scheme-number n)
  ((get 'make 'scheme-number) n))
\end{scheme}

\noindent
さてこのジェネリック数値演算システムのフレームワークが準備できたので新しい種類の
数値も容易に含めることができます。ここに分数演算を実行するパッケージがあります。
付加的あることの利点として\link{Section 2.1.1}の分数コードをパッケージ内の内部手続として変更無しに
利用できることに注目して下さい。

\begin{scheme}
(define (install-rational-package)
  ~\textrm{;; internal procedures}~
  (define (numer x) (car x))
  (define (denom x) (cdr x))
  (define (make-rat n d)
    (let ((g (gcd n d)))
      (cons (/ n g) (/ d g))))
  (define (add-rat x y)
    (make-rat (+ (* (numer x) (denom y))
                 (* (numer y) (denom x)))
              (* (denom x) (denom y))))
  (define (sub-rat x y)
    (make-rat (- (* (numer x) (denom y))
                 (* (numer y) (denom x)))
              (* (denom x) (denom y))))
  (define (mul-rat x y)
    (make-rat (* (numer x) (numer y))
              (* (denom x) (denom y))))
  (define (div-rat x y)
    (make-rat (* (numer x) (denom y))
              (* (denom x) (numer y))))
  ~\textrm{;; interface to rest of the system}~
  (define (tag x) (attach-tag 'rational x))
  (put 'add '(rational rational)
       (lambda (x y) (tag (add-rat x y))))
  (put 'sub '(rational rational)
       (lambda (x y) (tag (sub-rat x y))))
  (put 'mul '(rational rational)
       (lambda (x y) (tag (mul-rat x y))))
  (put 'div '(rational rational)
       (lambda (x y) (tag (div-rat x y))))
  (put 'make 'rational
       (lambda (n d) (tag (make-rat n d))))
  'done)
(define (make-rational n d)
  ((get 'make 'rational) n d))
\end{scheme}

\noindent
複素数を扱うために同様のパッケージをタグ\code{complex}を用いてインストールできます。
パッケージを作る際に、直行形式と極形式のパッケージにて定義された\code{make\-/from\-/real\-/imag}と
\code{make\-/from\-/mag\-/ang}の命令をテーブルから抽出します。付加性が内部命令として同じ
\link{Section 2.4.1}の手続\code{add\-/complex}, \code{sub\-/complex}, \code{mul\-/complex}, \code{div\-/complex}を
使用することを可能にします。

\begin{scheme}
(define (install-complex-package)
  ~\textrm{;; imported procedures from rectangular and polar packages}~
  (define (make-from-real-imag x y)
    ((get 'make-from-real-imag 'rectangular) x y))
  (define (make-from-mag-ang r a)
    ((get 'make-from-mag-ang 'polar) r a))
  ~\textrm{;; internal procedures}~
  (define (add-complex z1 z2)
    (make-from-real-imag (+ (real-part z1) (real-part z2))
                         (+ (imag-part z1) (imag-part z2))))
  (define (sub-complex z1 z2)
    (make-from-real-imag (- (real-part z1) (real-part z2))
                         (- (imag-part z1) (imag-part z2))))
  (define (mul-complex z1 z2)
    (make-from-mag-ang (* (magnitude z1) (magnitude z2))
                       (+ (angle z1) (angle z2))))
  (define (div-complex z1 z2)
    (make-from-mag-ang (/ (magnitude z1) (magnitude z2))
                       (- (angle z1) (angle z2))))
  ~\textrm{;; interface to rest of the system}~
  (define (tag z) (attach-tag 'complex z))
  (put 'add '(complex complex)
       (lambda (z1 z2) (tag (add-complex z1 z2))))
  (put 'sub '(complex complex)
       (lambda (z1 z2) (tag (sub-complex z1 z2))))
  (put 'mul '(complex complex)
       (lambda (z1 z2) (tag (mul-complex z1 z2))))
  (put 'div '(complex complex)
       (lambda (z1 z2) (tag (div-complex z1 z2))))
  (put 'make-from-real-imag 'complex
       (lambda (x y) (tag (make-from-real-imag x y))))
  (put 'make-from-mag-ang 'complex
       (lambda (r a) (tag (make-from-mag-ang r a))))
  'done)
\end{scheme}

\noindent
複素数パッケージの外側のプログラムは複素数を実数部と虚数部からでも大きさと角度からでも
構築することができます。元は直行形式と極形式のパッケージ内にて定義された内在する手続が
どのように複素数パッケージにエクスポートされているか、そしてそこからどのようにして
外部の世界へとエクスポートされているかについて注意して下さい。

\begin{scheme}
(define (make-complex-from-real-imag x y)
  ((get 'make-from-real-imag 'complex) x y))
(define (make-complex-from-mag-ang r a)
  ((get 'make-from-mag-ang 'complex) r a))
\end{scheme}

\noindent
ここで私達が行ったのは2つのレベルのタグシステムです。典型的な複素数、直交形式で
\( 3 + 4i \)のような物は\link{Figure 2.24}で示されるように表現されます。
外型のタグ(\code{complex})は数値を複素数パッケージへと導きます。複素数パッケージに入れば、
次のタグ(\code{rectangular})が数値を直行形式パッケージへと導きます。
巨大で複雑なシステムでは多くのレベルが存在するかもしれず、それぞれはジェネリックな命令を
用いて次へと接続されます。データオブジェクトが``下方''へ渡されるにつれ、適切なパッケージへ
導く外側のタグは(\code{contents}を適用することで)取り去られ、次のレベルのタグ(もし存在すれば)が
さらなる呼出のために使用されるため見えるようになります。

\begin{figure}[tb]
\phantomsection\label{Figure 2.24}
\centering
\begin{comment}
\heading{Figure 2.24:} Representation of \( 3 + 4i \) in rectangular form.

\begin{example}
     +---+---+     +---+---+     +---+---+
---->| * | *-+---->| * | *-+---->| * | * |
     +-|-+---+     +-|-+---+     +-|-+-|-+
       |             |             |   |
       V             V             V   V
 +---------+   +-------------+  +---+ +---+
 | complex |   | rectangular |  | 3 | | 4 |
 +---------+   +-------------+  +---+ +---+
\end{example}
\end{comment}
\includegraphics[width=64mm]{fig/chap2/Fig2.24c.pdf}
\begin{quote}
\heading{Figure 2.24:} 直行形式による\( 3 + 4i \)の表現
\end{quote}
\end{figure}

\noindent
上記のパッケージでは、\code{add\-/rat}, \code{add\-/complex}, それに他の数値演算手続を
全く元々書かれた状態で利用しました。しかし、これらの定義が異なるインストール手続の内部と
なれば直ぐに、お互いから識別可能である名前にする必要は無くなります。単純に両者のパッケージにて
\code{add}, \code{sub}, \code{mul}, \code{div}と名付けることが可能になります。

\begin{quote}
\heading{\phantomsection\label{Exercise 2.77}Exercise 2.77:} 
Louis Reasonerは\code{z}が\link{Figure 2.24}で示されるオブジェクトである場合に式
\code{(magnitude z)}を評価しようとした。驚いたことに、答の5の代わりに彼が受け取ったのは
\code{apply\-/generic}からのエラーメッセージで、型\code{(complex)}上に\code{magnitude}命令の
手段が存在しないと言う。彼はこの応答をAlyssa P. Hackerに見せた所、彼女は``問題は複素数セレクタが
\code{complex}の数値に対して定義されていない、\code{polar}と\code{rectangular}の数値に対してのみ
行われている。これを動かすためにしなければならないことは以下を\code{complex}パッケージに追加する
ことだ。''と述べた。

\begin{scheme}
(put 'real-part '(complex) real-part)
(put 'imag-part '(complex) imag-part)
(put 'magnitude '(complex) magnitude)
(put 'angle '(complex) angle)
\end{scheme}


これでなぜ動くのか詳細を説明せよ。例として式\code{(magnitude z)}を\code{z}が\link{Figure 2.24}にて
示されるデータオブジェクトの場合に評価する時、呼び出される全ての手続をトレースせよ
具体的には、\code{apply\-/generic}は何回起動されるか? どの手続が各ケースに対して呼び出されるか?
\end{quote}

\begin{quote}
\heading{\phantomsection\label{Exercise 2.78}Exercise 2.78:} 
\code{scheme\-/number}パッケージの内部手続は本質的にプリミティブな手続\code{+}, \code{-}, その他の
呼出し以上の物ではない。言語のプリミティブを直接使用することはできない。
私達のタイプタグシステムが各データオブジェクトに対し型付けられていることを要件とするため
である。しかし実際には全てのLisp実装は型システムを持っており、内部にて使用している。
\code{symbol?}や\code{number?}のようなプリミティブな述語はデータオブジェクトが特定の型を持つか
決定する。\link{Section 2.4.2}の\code{type\-/tag}, \code{contents}, and \code{attach\-/tag}の定義を変更し
私達のジェネリックシステムがSchemeの内部型システムの利点を得るようにせよ。
これは言い替えれば、
システムは以前と同じように動作する必要があるが、ただし普通の数値はその\code{car}がシンボル
\code{scheme\-/number}であるペアでなく、単純にSchemeの数値として表現されるようにせよ。
\end{quote}

\begin{quote}
\heading{\phantomsection\label{Exercise 2.79}Exercise 2.79:} 
2つの数値の等値関係をテストするジェネリックな等値の述語\code{equ?}を定義し、
ジェネリック数値演算パッケージにインストールせよ。この命令は通常の数値、分数、
複素数に対しても働くこと。
\end{quote}

\begin{quote}
\heading{\phantomsection\label{Exercise 2.80}Exercise 2.80:} 
引数が0であるかテストするジェネリックな述語\code{=zero?}を定義しジェネリック数値
演算パッケージにインストールせよ。この命令は通常の数値、分数、
複素数に対しても働くこと。
\end{quote}

\subsection{異なる型のデータを組み合わす}
\label{Section 2.5.2}



通常の数値、複素数、分数、そして開発するだろう任意の他の型の数値を包括する
統一数値演算システムをどのように定義するかについて学びました。しかし私達は
重要な問題を無視してきました。今まで私達が定義した命令は異なるデータの型を
完全に独立しているとして扱ってきました。従って追加すべき分かれたパッケージが、
例えば2つの普通の数値や2つの複素数が存在します。私達がまだ考慮していないことは
型の境界を渡る命令を定義することには意義があるという事実です。例えば複素数と実数の
加算です。私達はこれまでプログラムの間にバリアを築くために大きな努力をしてきました。
それが分離して開発、理解されることを可能にするためでした。私達は型を渡る命令を
ある程度注意深くコントロールされた手段にて導入したいと思います。そうすることで
私達のモジュール境界を重大な侵害が起こらないようにそれらをサポートすることがで
きるようにです。


クロスタイプ(型を渡る)命令を扱う1つの方法は命令が有効な型の可能な組み合わせそれぞれに
対して異なる手続を設計することです。例えば複素数パッケージを拡張し、それが複素数と実数
の加算を提供し、タグ\code{(complex scheme\-/number)}を用いてテーブルにインストールするように
します。\footnote{私達はまたほとんど同一の手続を型\code{(scheme\-/number complex)}を扱うために
提供しなければなりません。}

\begin{scheme}
~\textrm{;; to be included in the complex package}~
(define (add-complex-to-schemenum z x)
  (make-from-real-imag (+ (real-part z) x) (imag-part z)))
(put 'add '(complex scheme-number)
     (lambda (z x) (tag (add-complex-to-schemenum z x))))
\end{scheme}

\noindent
このテクニックはうまく行きますが、面倒です。このようなシステムでは新しい型を導入する
コストはその型のための手続のパッケージを構築するだけでなく、クロスタイプの命令を
実装する手続の構築とインストールに及びます。これは簡単にその型自身の命令を定義するため
に必要なものよりより多くのコードとなるでしょう。この手法はまた分かれたパッケージを付加的に接続する
能力を弱めたり、最低でも個々のパッケージの実装者が他のパッケージの考慮をしなければなら
ない範囲を制約する能力をダメにしてしまいます。
例えば、上の例では複素数と実数上の混合命令の扱いが複素数パッケージの責任となるのは
妥当に見えます。しかし分数と複素数の接続においては複素数パッケージで行われるかもしれないし、
分数パッケージかもしれないし、これらの2つのパッケージから抽出した命令を用いる
何らかの第三者パッケージかもしれません。パッケージ間の区分上における整合性のポリシー
の形式化が、多くのパッケージと多くのクロスタイプ命令を伴なうシステム設計において
計り知れなくなってしまいます。

\subsubsection*{型の強制}



完全に依存しない型達上にて振る舞う完全に依存しない命令群が一般的な状況においては
明示的にクロスタイプ命令を実装することは、面倒かもしれませんが、人が望む最高の
物かもしれません。幸運なことに私達は通常、私達の型システム内の潜在的に存在する
だろう付加的な構造の利点を用いることによりより良く行うことが可能です。
時折、異なるデータの型は完全には独立しておらず、ある型のオブジェクトが
他の型であるように見られる場合が複数存在するでしょう。
この過程は\newterm{coercion}(\jnewterm{強制})と呼ばれます。例えばもし私達が算術上、
実数と複素数を合成するよう求められた場合に、私達は実数を虚数部が0の複素数だと
見做すことができます。これはこの問題を2つの複素数の合成へと変換し、複素数パッケージ
により通常の方法にて取り扱うことが可能になります。



一般的に、ある型のオブジェクトを等価な他の型のオブジェクトに変換する強制手続を
設計することでこの考えを実装することができます。以下は典型的な強制手続です。
これは与えられた普通の数値(実数)を実数部とゼロである虚数部を持つ複素数に変換します。

\begin{scheme}
(define (scheme-number->complex n)
  (make-complex-from-real-imag (contents n) 0))
\end{scheme}

\noindent
これらの強制手続を2つの型の名前により索引付けした特別な強制テーブルにインストールします。

\begin{scheme}
(put-coercion 'scheme-number
              'complex 
              scheme-number->complex)
\end{scheme}

\noindent
(このテーブルを操作するために手続\code{put\-/coercion}と\code{get\-/coercion}が存在すると仮定します)。
一般にこのテーブルの枠のいくつかは空になります。全ての型の任意のデータオブジェクトを全ての
他の型へと強制することは一般には不可能です。例えば任意の複素数を実数に強制することはできません。
そのため普遍的な\code{complex\-/>scheme\-/number}手続はテーブルに含まれることはありません。



強制テーブルが準備されれば、\link{Section 2.4.3}の\code{apply\-/generic}手続を変更することで
統一的な作法で強制を取り扱うことができます。命令を適用するよう求められた時、
最初にその命令が引数の型に対して定義されているかどうかを以前と同様にチェックします。
もしそうであれば命令と型のテーブルで見つかった手続を呼び出します。そうでなければ
強制を試みます。単純化のために、2つの引数を伴う場合のみについて考えることにします。\footnote{
一般化については\link{Exercise 2.82}を参照して下さい。}
強制テーブルをチェックし、最初の型のオブジェクトが2つ目の型に強制できるか確認します。
もしそうであれば、最初の引数を強制し、命令の試行を再び行います。もし最初の型のオブジェクトが
一般に2つ目の型に強制できない場合、逆に2つ目の引数を1つ目の引数の型に強制できるか試します。
最後にどちらの型も他方の型に強制できない場合、諦めます。以下がこの手続です。

\begin{smallscheme}
(define (apply-generic op . args)
  (let ((type-tags (map type-tag args)))
    (let ((proc (get op type-tags)))
      (if proc
          (apply proc (map contents args))
          (if (= (length args) 2)
              (let ((type1 (car type-tags))
                    (type2 (cadr type-tags))
                    (a1 (car args))
                    (a2 (cadr args)))
                (let ((t1->t2 (get-coercion type1 type2))
                      (t2->t1 (get-coercion type2 type1)))
                  (cond (t1->t2 
                         (apply-generic op (t1->t2 a1) a2))
                        (t2->t1 
                         (apply-generic op a1 (t2->t1 a2)))
                        (else (error "No method for these types"
                                     (list op type-tags))))))
              (error "No method for these types"
                     (list op type-tags)))))))
\end{smallscheme}

\noindent
この強制スキームは上で概説された様に明示的なクロスタイプ命令の定義手法上に多くの利点を持ちます。
私達は依然、型に関係する強制手続を書かねばなりませんが(\( n \)個の型のシステムに対し
場合により\( n^2 \)の手続)、全ての型の集合と各ジェネリック命令に対し異なる手続を書くのでなく、
型のペア1組につき1つの手続を書くだけで済みます。\footnote{
もし私達が賢いならば普通は\( n^2 \)よりも少ない強制手続で済みます。例えばもし型1から型2
への変換方法と型2から型3への変換方法を知っている場合、この知識を用いて型1から型3へ変換することが
できます。これはシステムに新しい型を追加する時に明示的に提供せねばならない強制手続の数を
劇的に減らします。もしシステムに必要なだけの洗練を組み入れたいのなら、システムにタイプ間の
``グラフ''を検索させて自動的に明示的に提供された物から推論可能な強制手続を生成させることが可能です。}
ここで私達が信頼しているものはタイプ間の適切な変換は型それ自身のみに依存し、適用される命令には
依存しないという事実です。



一方で、私達の強制スキームが十分に汎用ではないアプリケーションが存在するかもしれません。
たとえ合成されるオブジェクトの両方ともが他方に変換できないとしても両者を第三の型に
変換することで命令を実行することが可能になるかもしれません。そのような複雑さに対処する
ため、そしてそれでもプログラムのモジュール方式を維持するために、通常はより一層タイプ間の
関係の構造の利点を得るシステムを構築することが、次で議論するように必要です。

\subsubsection*{型の階層}



上で展開された強制スキームは型のペアの間の自然な関係の存在に当てにしていました。
より``一般的な''構造が、異なる型のお互いへの関係の仕方には良く存在します。
例えば、私達が整数、分数、実数、複素数を扱う一般的な数値演算システムを構築していると
仮定します。そのようなシステムにおいては整数を特別な種類の分数として見做すことは
とても自然です。分数は同様に特別な種類の実数であり、実数は同様に特別な種類の
複素数であります。私達が実際に手にしている物は\newterm{hierarchy of types}(\jnewterm{型の階層})
呼ばれるもので、その中では例えば整数は分数の\newterm{subtype}(\jnewterm{サブタイプ})です(すなわち
分数に適用できる任意の命令は自動的に整数に適用できます)。逆に分数は整数の
\newterm{supertype}(\jnewterm{スーパータイプ}、親の型)と呼びます。今手にした階層はとても単純な
種類で、各型はたかだか1つのスーパータイプを持ち、たかだか1つのサブタイプを持ちます。
そのような構造は\newterm{tower}(\jnewterm{タワー}、塔)と呼ばれ\link{Figure 2.25}で示されます。

\begin{figure}[tb]
\phantomsection\label{Figure 2.25}
\centering
\begin{comment}
\heading{Figure 2.25:} A tower of types.

\begin{example}
 complex
   ^
   |
  real
   ^
   |
rational
   ^
   |
integer
\end{example}
\end{comment}
\includegraphics[width=11mm]{fig/chap2/Fig2.25.pdf}
\par\bigskip
\noindent
\heading{Figure 2.25:} 型の塔
\end{figure}

\noindent
もしタワー構造を持つ場合、階層に新しい型を追加する問題を著しく単純化できます。
新しい型がどのようにその上のスーパータイプの隣に組込まれるか、そしてどのように
その型がその下の型に対してスーパータイプであるかを指定するだけです。
例えばもし複素数に対して整数を追加したい場合、明示的に特別な強制手続
\code{integer\-/>complex}を定義する必要はありません。その代わりに整数がどのように分数に
変換できるか、分数がどのように実数に変換できるか、実数がどのように複素数に変換でき
るかを定義します。そうしたらシステムに整数を複素数に変換することをこれらのステップ
を通して変換することを許可し、次に2つの複素数を加算します。


\code{apply\-/generic}手続を以下のように再設計することもできます。各型に対して
\code{raise}手続を与える必要があります。これはある型のオブジェクトをタワーにおいて
1レベル上げます。そうすればシステムが異なる型のオブジェクト上にて操作する必要がある時、
全てのオブジェクトがタワー内にて同じレベルになるまで連続して上げることができます
(\link{Exercise 2.83}と\link{Exercise 2.84}がそのような戦略の実装の詳細について
考察しています)。



タワーの別の利点には全ての型がスーパータイプ上に定義された全ての命令を``継承''する
概念を簡単に実装できることが上げられます。例えばもし整数の実数部を求めるための
特別な手続を提供しない場合、それにもかかわらず
整数は複素数のサブタイプであるという事実のおかげで、整数のための\code{real\-/part}が
定義されることが期待できます。タワーでは\code{apply\-/generic}を変更するという統一的な
方法でこの様なことが起こるよう準備することが可能です。
もし必要な命令が与えられたオブジェクトの型のために直接定義されていない場合、
オブジェクトをそのスーパータイプに上げることで再試行できます。従ってタワーを
這い上がりながら望まれた命令が実行可能になるまで引数を変換するか、頂上まで辿り
ついてそこで諦めることができます。



別のより一般的な階層に比べた場合、もう1つタワーの利点はデータオブジェクトをより簡単な
表現へ``下げる''簡単な方法を提供することです。例えば\( 2 + 3i \)を\( 4 - 3i \)に
足した場合、その答は複素数\( 6 + 0i \)よりも整数6で得るほうがより良いと言えるでしょう。
\link{Exercise 2.85}はそのようなレベルを下げる命令の実装について議論します。
(この仕掛けには\( 6 + 0i \)のような階層のレベルを下げられるオブジェクトを
\( 6 + 2i \)のような下げられないオブジェクトから見分ける一般的な方法が必要です)。

\subsubsection*{階層の不十分さ}



もしシステムのデータの型が自然にタワーに配置できる場合、ここまで見てきた通りに、
異なる型上のジェネリック命令の取扱の問題を著しく単純化できます。
残念なことに、これは普通の場合ではありません。\link{Figure 2.26} は雑多な型のより複雑な
配置を図示しています。この図は幾何学的図形の異なる型の間の関係を見せています。
一般的に1つの型が複数のサブタイプを持つことがわかります。例えば三角形と四角形は
共に多角形のサブタイプです。加えてある型は複数のスーパータイプを持つことがあり得ます。
例えば二等辺直角三角形は二等辺三角形、または直角三角形と見做すことができます。
この複数スーパータイプ問題は特に困難で、階層内において型を``上げる''単一の方法が
存在しません。オブジェクトに命令を適用するため``正しい''スーパータイプを求めることは
\code{apply\-/generic}の様な手続に不可欠な型ネットワーク全体を通しての多大な検索を巻き起す
可能性があります。一般的にある型に対して複数のサブタイプが存在するので値に対し型階層を
``下げる''強制にも同様の問題が存在します。巨大システムの設計におけるモジュール化方式を
それでも維持しながら多くの数の相互に関係する型の取り扱うことはとても難しく、現在の多くの
研究領域です。\footnote{
第一版でも存在したこの文は12年前と同じく今も変わりません。実用的で汎用的な異なる型の
要素間の関係の表現するフレームワーク(哲学者が``オントロジー''(存在論)と呼ぶもの)を
開発することは不可能に見えるほど難しいことです。10年前に存在した混乱と現在に存在する混乱との間の
違いは、種々の不適切な存在論上の理論が、相応して不適切なプログラミング言語に過剰に組込まれてい
ることです。例えばオブジェクト指向言語の複雑性の多くは---そして現在のオブジェクト指向言語間の
微妙で混乱させる違いは---相互に関係する型上のジェネリック命令の扱いを中心とします。
\link{Chapter 3}での私達自身による計算オブジェクトの議論はこれらの問題を完全に避けます。
オブジェクト指向言語に親しみのある読者は\link{Chapter 3}においてローカルの状態について多くの
触れるべきことが存在すると気付くでしょう。しかし私達は``クラス''や``継承''についてさえ
述べることはしません。実際に私達はこれらの問題が知識表現上の成果の利用と自動的な推論無しに
コンピュータ言語設計のみで適切に解決されることは無いと疑っています。}

\begin{figure}[tb]
\phantomsection\label{Figure 2.26}
\centering
\begin{comment}
\heading{Figure 2.26:} Relations among types of geometric figures.

\begin{example}
                     polygon
                    /       \
                   /         \
            triangle         quadrilateral
            /     \              /     \
           /       \            /       \
     isosceles   right      trapezoid   kite
     triangle    triangle       |         |
      |     \      |            |         |
      |      \     |            |         |
equilateral   isosceles   parallelogram   |
triangle      right          |       \    |
              triangle       |        \   |
                          rectangle  rhombus
                                \    /
                                 \  /
                                square
\end{example}
\end{comment}
\includegraphics[width=96mm]{fig/chap2/Fig2.26e.pdf}
\par\bigskip
\noindent
\heading{Figure 2.26:} 幾何学的図形の型の間の関係
\end{figure}

\begin{quote}
\heading{\phantomsection\label{Exercise 2.81}Exercise 2.81:} 
Louis Reasonerは\code{apply\-/generic}が引数に対しそれらが既に同じ型であってもお互いの
型に強制を試行することに気付いた。そのため彼は強制テーブルに各型の引数をそれら自身の
型に\newterm{coerce}(\jnewterm{強制})するための手続を追加する必要があるのではないかと``reason''(推論)
した。

\begin{scheme}
(define (scheme-number->scheme-number n) n)
(define (complex->complex z) z)
(put-coercion 'scheme-number 
              'scheme-number
              scheme-number->scheme-number)
(put-coercion 'complex 'complex complex->complex)
\end{scheme}

\begin{enumerate}[a]

\item
Louisの強制手続がインストールされると\code{apply\-/generic}が命令に対する
2つの引数の型が\code{scheme\-/number}、または2つの引数の型が\code{complex}であり
それらの型に対する命令がテーブルに存在しない場合何が起こるだろうか?
例えばジェネリックな指数関数命令を定義したとしよう。

\begin{scheme}
(define (exp x y) (apply-generic 'exp x y))
\end{scheme}

\noindent
そしてScheme-numberパッケージの指数関数に対する手続を追加したとする。
ただし他の型に関しては全て行わない。

\begin{scheme}
~\textrm{;; following added to Scheme-number package}~
(put 'exp '(scheme-number scheme-number)
     (lambda (x y) (tag (expt x y)))) 
     ~\textrm{; using primitive \code{expt}}~
\end{scheme}

\noindent
2つの複素数引数により\code{exp}を呼び出した場合、何が起こるだろうか?

\item
同じ型の引数に伴なう強制に関して何かが行われるべきかについてLouisは正しいだろうか?
それとも\code{apply\-/generic}はそのままで正しく動作するだろうか?

\item
\code{apply\-/generic}を変更し2つの引数が同じ型である場合に強制を試行しないようにせよ。

\end{enumerate}
\end{quote}

\begin{quote}
\heading{\phantomsection\label{Exercise 2.82}Exercise 2.82:} 
\code{apply\-/generic}を複数引数の全体的な場合に強制を扱わせるような一般化を行わせるには
どのように行うかを示せ。1つの戦略として全ての引数を最初の引数の型に強制するよう
試行し、次に2つ目、以降繰り返しが上げられる。この戦略(と上で与えられた2引数版が
同様に)全体には不十分である例を示せ。(ヒント:テーブルにいくつか適切な型が混ざった
命令が存在し、それが試行されない場合について考えよ。)
\end{quote}

\begin{quote}
\heading{\phantomsection\label{Exercise 2.83}Exercise 2.83:} 
\link{Figure 2.25}で示される型のタワーを取り扱うジェネリックな数値演算システムの
設計を行っているとする。整数、分数、実数、複素数に対応する。各型(複素数を除く)に
対してその型のオブジェクトをタワー内にて1レベル上げる手続を設計せよ。
(複素数を除く)各型に対し動作するジェネリックな\code{raise}命令をどのようにインストールするか
示せ。
\end{quote}

\begin{quote}
\heading{\phantomsection\label{Exercise 2.84}Exercise 2.84:} 
\link{Exercise 2.83}の\code{raise}命令を用いて\code{apply\-/generic}手続を変更し、
複数の引数が一連の``上げる''動作を行うことによりこの節で語られた様に同じ型を持つようにせよ。
2つの型のどちらがタワー内にてより高いレベルであるかテストする手段を開発する必要がある。
これを残りのシステムと``互換性''を保ち、タワーに新しいレベルを追加する場合にも
問題が無いような手段で行え。
\end{quote}

\begin{quote}
\heading{\phantomsection\label{Exercise 2.85}Exercise 2.85:} 
この節では可能な限りタワー内の型レベルを下げることによりデータオブジェクトの``単純化''を
行う手段について説明した。\link{Exercise 2.83}に記述されたタワーに対しこれを達成する手続\code{drop}
を設計せよ。いくつかの一般的な方法の中から決定する鍵は、オブジェクトを下げることができるか
どうかである。例えば複素数\( 1.5 + 0i \)は\code{real}(実数)である限り下げられ、複素数\( 1 + 0i \)
は\code{integer}(整数)である限り下げることができ、複素数\( 2 + 3i \)は下げることが絶対にできない。
以下に、あるオブジェクトが下げることができるか決定する計画を示す。
オブジェクトをタワー内にて``押し下げる''ジェネリックな命令\code{project}(射影)を定義することから始める。
例えば複素数の射影は虚数部を捨てることになる。すると数値は\code{project}した結果を
元の型に\code{raise}(上げ)た時に開始した時点と同じ値になれば\code{drop}(落とす)ことができる
ことになる。可能な場合にオブジェクトを落とす手続\code{drop}を書くことで、この考えをどのように実装するか
詳細に示せ。色々な射影命令を設計し、ジェネリックな命令として\code{project}をシステム内にインストール
する必要がある。\footnote{
実数は引数に最も近い整数を返すプリミティブ\code{round}を用いて整数に射影することができる。}
また\link{Exercise 2.79}で説明した等値関係のジェネリックな述語を利用する必要もある。
最後に\code{drop}を用いて\link{Exercise 2.84}の\code{apply\-/generic}を書き直し解答を``単純化''する。
\end{quote}

\begin{quote}
\heading{\phantomsection\label{Exercise 2.86}Exercise 2.86:} 
実数部、虚数部、大きさ、角度が通常の数値、分数、またはシステムに追加したくなるかも
しれない数のどれかを用いることができる複素数を扱えるようにしたいとする。
これを達成するために必要なシステムに対する変更を説明し、実装せよ。
普通の数と分数に対してジェネリックな\code{sine}や\code{cosine}のような命令を定義する必要が
出てくるであろう。
\end{quote}

\subsection{例: 記号代数}
\label{Section 2.5.3}


記号代数表現の操作は巨大なスケールのシステムの設計において起こり得る最も困難な問題の
多くを説明する複雑な処理です。代数表現は一般的に階層構造であると見ることができ、
演算子の木がオペランドに適用されます。代数表現を定数と変数のようなプリミティブな
オブジェクトの集合から始めて、これらを加算や乗算のような代数演算子を用いて接続する
ことで構築することができます。他の言語と同様に、複合オブジェクトに簡単な用語で参照する
ことを可能にするための抽象化を形式化します。典型的な記号代数における抽象化は線形結合、
多項式、有理関数、三角関数のような考えです。これらを式の処理を方向付けするのによく便利である
複合``型''と見做すことができます。例えば私達は以下の式を
\begin{comment}

\begin{example}
x^2 sin (y^2 + 1) + x cos 2y + cos(y^3 - 2y^2)
\end{example}

\end{comment}
\begin{displaymath}
 x^2 \sin (y^2 + 1) + x \cos 2y + \cos(y^3 - 2y^2) 
\end{displaymath}
\noindent
係数を伴なう\( x \)の多項式と係数が整数である\( y \)の三角関数として記述することが
できます。


私達は完全な代数操作システムをここで開発しようとはしません。
そのようなシステムは非常に複雑なプログラムであり、深い代数学の知識と洗練されたアルゴリズムを
具体化する必要があります。私達が行うのは代数操作の単純だが重要な部分について考えること、
つまり多項式の演算です。そのようなシステムの設計者が直面する決定すべきことや、
この試みのまとめを手助けするために抽象データやジェネリックな命令をどのようにして適用するか
のような事柄について説明します。

\subsubsection*{多項式の計算}

多項式上の数値演算を実行するシステムの設計における最初のタスクは多項式とは何かを
決定することです。多項式は通常いくつかの変数(多項式の\newterm{indeterminates}(\jnewterm{不定元}))
に関連して定義されます。簡単にするために多項式はただ1つの不定元
(\newterm{univariate polynomials}(\jnewterm{一変数多項式}))に制約します。\footnote{
一方で係数は別の変数にてそれ自身が多項式であることを許可します。これにより本質的に
完全に多変量システムと同じ表現力を得ますが、強制においてこの先で記述される問題が発生します。}
多項式とは項の和であり、各項は係数、不定元の累乗数、または係数と不定元の累乗数の積であると
定義します。係数は多項式の不定元に依存しない代数表現であると定義します。例えば、
\begin{comment}

\begin{example}
5x^2 + 3x + 7
\end{example}

\end{comment}
\begin{displaymath}
 5x^2 + 3x + 7 
\end{displaymath}
\noindent
は簡単な\( x \)の多項式であり、
\begin{comment}

\begin{example}
(y^2 + 1)x^3 + (2y)x + 1
\end{example}

\end{comment}
\begin{displaymath}
 (y^2 + 1)x^3 + (2y)x + 1 
\end{displaymath}
\noindent
は係数が\( y \)の多項式である\( x \)の多項式です。


既にいくつかの困難な問題を回避しています。これらの多項式の最初の物は多項式
\( 5y^2 + 3y + 7 \)と同じかそれとも異なるでしょうか? 妥当な答は``多項式を純粋に
数学の関数であると考えれば答はYESです。しかしもし多項式を文法上の形式であると
考えれば答はNOです''となるでしょう。2つ目の多項式は代数学的に係数が
\( x \)の多項式である\( y \)の多項式に等価です。私達のシステムはこれを認識する
べきでしょうか? さらに他にも多項式を表現する方法は存在します---例えば因数の積として
や(1変数多項式に対しては)累乗根の集合として、また指定した点の集合における多項式の
値の列挙として。\footnote{
1変数多項式に対しては与えられた点の集合における多項式の値を与えることは特に良い表現
に成り得ます。これは多項式数値演算をとても簡単にすることができます。例として
この方法で表現された2つの多項式の和を求めるには相対する点の多項式の当たい を足すだけ
で済みます。より親しみ易い表現に戻すには\( n + 1 \)個の点における多項式の値を与えら
れた場合に\( n \)次の多項式の係数を取り戻すラグランジュ補完公式を用いることができます。}
これらの問題を私達の数値演算操作システムにおいて、根底にある数学上の意味でなく、
``多項式''が特定の文法形式であることを決定することでうまく行うことができます。



さて、多項式上で数値演算を行なうことについてどのように進めるか考えねばなりません。
この簡単なシステムでは加算と乗算についてのみしか考えません。さらに接続される2つの多項式は
同じ不定元を持たなければならないとします。



私達のシステムの設計はデータ抽象化にて馴染のある規律に従うことで取り組みます。
多項式を\newterm{poly}と呼ぶ新しいデータ構造を用いて表現します。\newterm{poly}は
変数と項の係数により構成されます。polyからそれらの部分を抽出するセレクタ\code{variable}と
\code{term\-/list}と与えられた変数と項のリストからpolyを組み上げるコンストラクタ\code{make\-/poly}
が既にあると仮定します。変数はただのシンボルであり\link{Section 2.3.2}の\code{same\-/variable?}手続を
用いて変数の比較が可能です。以下の手続はpolyの加算と乗算を定義します。

\begin{scheme}
(define (add-poly p1 p2)
  (if (same-variable? (variable p1) (variable p2))
      (make-poly (variable p1)
                 (add-terms (term-list p1)
                            (term-list p2)))
      (error "Polys not in same var: ADD-POLY"
             (list p1 p2))))

(define (mul-poly p1 p2)
  (if (same-variable? (variable p1) (variable p2))
      (make-poly (variable p1)
                 (mul-terms (term-list p1)
                            (term-list p2)))
      (error "Polys not in same var: MUL-POLY"
             (list p1 p2))))
\end{scheme}

\noindent
多項式を私達の数値演算システムに組込むためにはそれらをタイプタグと共に提供する
必要があります。タグ\code{polynomial}を用いることにし、タグ付き多項式上の適切な
命令を命令テーブルにインストールします。\link{Section 2.5.1}と同様に、多項式パッケージに
対するインストール手続に私達の全てのコードを組み込んでしまうことにします。

\begin{scheme}
(define (install-polynomial-package)
  ~\textrm{;; 内部手続}~
  ~\textrm{;; polyの表現}~
  (define (make-poly variable term-list)
    (cons variable term-list))
  (define (variable p) (car p))
  (define (term-list p) (cdr p))
  ~\( \dark \langle \)~~\emph{procedures \code{same\-/variable?} and \code{variable?} from section 2.3.2}~~\( \dark \rangle \)~

  ~\textrm{;; 項と項のリストの表現}~
  ~\( \dark \langle \)~~\emph{procedures \code{adjoin\-/term} \( \dots \) \code{coeff} from text below}~~\( \dark \rangle \)~

  (define (add-poly p1 p2) ~\( \dots \)~)
  ~\( \dark \langle \)~~\emph{\code{add\-/poly}で使用される手続}~~\( \dark \rangle \)~
  (define (mul-poly p1 p2) ~\( \dots \)~)
  ~\( \dark \langle \)~~\emph{\code{mul\-/poly}で使用される手続}~~\( \dark \rangle \)~

  ~\textrm{;; システムの残りへのインターフェイス}~
  (define (tag p) (attach-tag 'polynomial p))
  (put 'add '(polynomial polynomial)
       (lambda (p1 p2) (tag (add-poly p1 p2))))
  (put 'mul '(polynomial polynomial)
       (lambda (p1 p2) (tag (mul-poly p1 p2))))
  (put 'make 'polynomial
       (lambda (var terms)
         (tag (make-poly var terms))))
  'done)
\end{scheme}

\noindent
多項式の加算は項別に実行されます。同じ次数の項(つまり同じ指数の不定元)が合成され
ねばなりません。これは係数は加数の係数の合計である同じ次数の新しい項を形成すること
により行われます。ある加数の項の同じ次数の項がもう一方に無い場合には単純に構築される
和の多項式に積み上げられます。



項のリストを操作するために、空の項リストを返すコンストラクタ\code{the\-/empty\-/termlist}と
新しい項を項リストに挿入するコンストラクタ\code{adjoin\-/term}を既に持っていると仮定します。
また与えられた項リストが空であるか判断する述語\code{empty\-/termlist?}と項リストから最大次数の項を
抽出するセレクタ\code{first\-/term}、最大次数の項を除く全てを返すセレクタ\code{rest\-/terms}もまた
持っていると仮定します。項を操作するために、与えられた次数と係数から項を構築する
コンストラクタ\code{make\-/term}と項の次数と係数をそれぞれ返すセレクタ\code{order}と\code{coeff}
を既に持っていると仮定します。これらの命令は項と項のリストの両方を実際の表現については分離して考えられる
データ抽象として捉えることを許します。



以下は2つの多項式の和のために項リストを構築する手続です。\footnote{
この命令は\link{Exercise 2.62}にて開発した\code{union\-/set}命令にとても似ています。
実際にもし多項式の項を不定元の指数に従い並べた集合だと考えるなら、和のために
項リストを生成するプログラムは\code{union\-/set}とほとんど同じです。}

\begin{scheme}
(define (add-terms L1 L2)
  (cond ((empty-termlist? L1) L2)
        ((empty-termlist? L2) L1)
        (else
         (let ((t1 (first-term L1)) 
               (t2 (first-term L2)))
           (cond ((> (order t1) (order t2))
                  (adjoin-term
                   t1 (add-terms (rest-terms L1) L2)))
                 ((< (order t1) (order t2))
                  (adjoin-term
                   t2 (add-terms L1 (rest-terms L2))))
                 (else
                  (adjoin-term
                   (make-term (order t1)
                              (add (coeff t1) (coeff t2)))
                   (add-terms (rest-terms L1)
                              (rest-terms L2)))))))))
\end{scheme}

\noindent
ここで注意すべき最も重要な点はジェネリックな加算手続\code{add}を用いて合成される
2つの項の係数を一緒に足したことです。これは以下で見るように強力な帰結です。



2つの項リストを乗算するために最初のリストの各項をもう一方のリストの全ての項で乗算するのに
繰り返し\code{mul\-/term\-/by\-/all\-/terms}を使用ます。\code{mul\-/term\-/by\-/all\-/terms}は与えられた項を
全ての与えられた項リストの項で乗算します。結果の項リスト(最初のリストの各項に対して1つ)は
合計に積み上げられます。2つの項の乗算は次数が乗数の次数の和で係数が乗数の係数の積となる項
を形成します。

\begin{scheme}
(define (mul-terms L1 L2)
  (if (empty-termlist? L1)
      (the-empty-termlist)
      (add-terms (mul-term-by-all-terms (first-term L1) L2)
                 (mul-terms (rest-terms L1) L2))))
(define (mul-term-by-all-terms t1 L)
  (if (empty-termlist? L)
      (the-empty-termlist)
      (let ((t2 (first-term L)))
        (adjoin-term
         (make-term (+ (order t1) (order t2))
                    (mul (coeff t1) (coeff t2)))
         (mul-term-by-all-terms t1 (rest-terms L))))))
\end{scheme}

\noindent
これが本当に多項式の和と積のためにあるもの全てです。ジェネリック手続\code{add}と\code{mul}を
用いて項を操作するため、多項式パッケージはジェネリック数値演算パッケージにより知られている
任意の型の係数を自動的に取り扱うことが可能であることに注意して下さい。
もし\link{Section 2.5.2}で議論されたような強制メカニズムを含めていた場合、型の異なる係数の多項式上でも
命令を取り扱うことが自動的にできます。
\begin{comment}

\begin{example}
                         /        2                 \
[3x^2 + (2 + 3i)x + 7] * | x^4 + --- x^2 + (5 + 3i) |
                         \        3                 /
\end{example}

\end{comment}
\begin{displaymath}
 [3x^2 + (2 + 3i)x + 7] \cdot \! \left[ x^4 + {2\over3} x^2 + (5 + 3i) \right]\!\!. 
\end{displaymath}

多項式の加算と乗算の手続、\code{add\-/poly}と\code{mul\-/poly} をジェネリックな数値演算システムに
型\code{polynomial}のための命令\code{add}と\code{mul}としてインストールしたため、私達のシステムは
また自動的に以下のような多項式操作を取り扱うことが可能です。
\begin{comment}

\begin{example}
[(y + 1)x^2 + (y^2 + 1)x + (y - 1)] * [(y - 2)x + (y^3 + 7)]
\end{example}

\end{comment}
\begin{displaymath}
 \Big[(y + 1)x^2 + (y^2 + 1)x + (y - 1)\Big] \cdot \Big[(y - 2)x + (y^3 + 7)\Big]. 
\end{displaymath}

その理由はシステムが係数を合成しようと試す時、\code{add}と\code{mul}を通して呼出を行うためです。
係数はそれ自身(\( y \)の)多項式ですから、これらは\code{add\-/poly}と\code{mul\-/poly}を用いて合成され
ます。結果は``データ適従再帰''のような物で、例えば\code{mul\-/poly}の呼出は係数の乗算のために
\code{mul\-/poly}の再帰呼出に帰着します。もし係数の係数がそれ自身多項式(多項式を3変数で表現した場合)
の場合、データ適従はシステムがまた別のレベルの再帰呼出に従うことを保証します。そして
データの構造が指示するだけのより多くのレベルについてもまた同様です。\footnote{

これを完全に順調に行うには私達のジェネリック数値演算システムに``数値''を次数が0で係数がその数である
多項式であると見做すことで多項式に強制する能力も追加する必要があります。これは以下のような
式に対して実行を行いたい場合に必要です。
\begin{comment}

\begin{example}
[x^2 + (y + 1)x + 5] + [x^2 + 2x + 1]
\end{example}

\end{comment}
\begin{displaymath}
 [x^2 + (y + 1)x + 5] + [x^2 + 2x + 1], 
\end{displaymath}
\noindent
これは係数\( y + 1 \)を係数2に対し足す必要があります。}

\subsubsection*{項リストの表現}


ようやく項リストに対する良い表現を実装する仕事に直面せねばなりません。
項リストは実際には項の次数をキーにした係数の集合です。従って\link{Section 2.3.3}にて議論したような
任意の集合表現の手法がこのタスクに適用可能です。一方で手続\code{add\-/terms}と\code{mul\-/terms}は
常に高い次数から低い次数へと連続して項リストを常にアクセスします。従って何らかの
順序付きリスト表現を用いることにしましょう。



項リストを表現するリストをどのように構造化するべきでしょうか。1つの考慮点は
私達が操作しようとする多項式の``濃度''です。多項式は多くの次数に関して0でない
係数を持つ場合\newterm{dense}(\jnewterm{密})と呼ばれます。もし多くの0の項を持つ場合には
\newterm{sparse}(\jnewterm{疎})と呼ばれます。例えば、
\begin{comment}

\begin{example}
A : x^5 + 2x^4 + 3x^2 - 2x - 5
\end{example}

\end{comment}
\begin{displaymath}
 A: \quad x^5 + 2x^4 + 3x^2 - 2x - 5 
\end{displaymath}
\noindent
は密多項式です。
\begin{comment}

\begin{example}
B : x^100 + 2x^2 + 1
\end{example}

\end{comment}
\begin{displaymath}
 B: \quad x^{100} + 2x^2 + 1 
\end{displaymath}
\noindent
は疎です。

密多項式の項リストは係数のリストとして最も効率良く表現されます。例えば上の\( A \)は
\code{(1 2 0 3 -2 -5)}としてうまく表わされます。この表現の項の次数は
その項の係数で始まるサブリストの長さから1を引いた数です。\footnote{
これらの多項式の例では\link{Exercise 2.78}で提案された型メカニズムを用いてジェネリック
数値演算システムを実装したと前提しています。従って普通の数値の係数は数値それ自身で表現され、
\code{car}がシンボル\code{scheme\-/number}のペアではありません。}
これは\( B \)のような疎多項式には酷い表現に成り得ます。少なく孤立した非ゼロな項により
中断される巨大なゼロのリストになるでしょう。疎多項式のより適切な項リストの表現は各項が
項の次数とその次数に対する係数を含むリストである非ゼロ項のリストです。
そのような仕組みでは多項式\( B \)は効率的に\code{((100 1) (2 2) (0 1))}として表現されます。
多くの多項式操作が疎多項式上にて実行されるため、私達はこちらの手法を用います。
項リストは項のリストとして表現され高次から低次の項へと並べられます。これを決定すれば
項と項リストに対するセレクタとコンストラクタの実装は簡単です。\footnote{
項リストが順序有りだと想定していますが、\code{adjoin\-/term}を単純に新しい項を既存の項リスト上に
\code{cons}するように実装しました。\code{adjoin\-/term}を用いる(\code{add\-/terms}のような)手続が
常にリスト内の物より高次な項と共にそれを呼ぶことを保証するならばこのままにしておくことが
できます。もしそのような保証を行うことが望ましくなかったならば\code{adjoin\-/term}を集合の
順序付きリスト表現のための\code{adjoin\-/set}(\link{Exercise 2.61})と同様に実装しておくべきだったでしょう。}
\begin{scheme}
(define (adjoin-term term term-list)
  (if (=zero? (coeff term))
      term-list
      (cons term term-list)))

(define (the-empty-termlist) '())
(define (first-term term-list) (car term-list))
(define (rest-terms term-list) (cdr term-list))
(define (empty-termlist? term-list) (null? term-list))

(define (make-term order coeff) (list order coeff))
(define (order term) (car term))
(define (coeff term) (cadr term))
\end{scheme}

\noindent
\code{=zero?}は\link{Exercise 2.80}で定義されています。(下の\link{Exercise 2.87}も参照して下さい)。


多項式パッケージのユーザは(タグ付き)多項式を以下の手続で作成します。

\begin{scheme}
(define (make-polynomial var terms)
  ((get 'make 'polynomial) var terms))
\end{scheme}

\begin{quote}
\heading{\phantomsection\label{Exercise 2.87}Exercise 2.87:} 
対抗式に対する\code{=zero?}をジェネリック数値演算パッケージにインストールせよ。
これは\code{adjoin\-/term}に係数それ自身が多項式である多項式に対して動作を可能にする。
\end{quote}

\begin{quote}
\heading{\phantomsection\label{Exercise 2.88}Exercise 2.88:} 
多項式システムを拡張し多項式の減算を含めよ。(ヒント:ジェネリックな単項算術否定演算子
を定義することが手助けとなるだろう。)
\end{quote}

\begin{quote}
\heading{\phantomsection\label{Exercise 2.89}Exercise 2.89:} 
密多項式に対して適切だと上で説明された項リスト表現を実装する手続を定義せよ。
\end{quote}

\begin{quote}
\heading{\phantomsection\label{Exercise 2.90}Exercise 2.90:} 
疎と密、両方の多項式に対して効率の良い多項式システムを得たいとする。
これを行う1つの方法は両方の種類の項リスト表現をシステム内にて許可することである。
状況は\link{Section 2.4}の複素数の例と同様で、そこでは直行形式と極形式の両表現を許可した。
これを行うため、異なる型の項リストを識別し、項リスト上の命令をジェネリックにせね
ばならない。多項式システムをこの汎化を行うために再設計せよ。これは局所的な変更ではなく
大域的な変更になる。

\end{quote}

\begin{quote}
\heading{\phantomsection\label{Exercise 2.91}Exercise 2.91:} 
1変数多項式は別の1変数多項式により割ることができ、多項式の商と多項式の剰余を
算出する。例えば、
\begin{comment}

\begin{example}
x^5 - 1
------- = x^3 + x, remainder x - 1
x^2 - 1
\end{example}

\end{comment}
\begin{displaymath}
 {x^5 - 1 \over x^2 - 1} = x^3 + x, \hbox{  remainder  } x - 1. 
\end{displaymath}

除算は長除法を通して行うことができる。これは被除数の最高次の項を除数の最高次の項
で割る。結果は商の最初の項である。次に結果に除数を掛け、被除数からその結果を引く。
そして残りの答を再帰的に差を除数で割ることにより求める。除数の次数が被除数の次数を
越えた時に停止し、その時の被除数を剰余であると宣言する。またもし被除数がゼロになった
場合には商と剰余の両者をゼロとして返す。


\code{add\-/poly}と\code{mul\-/poly}のモデルの上に\code{div\-/poly}手続を設計することが可能だ。
この手続は2つの多項式が同じ変数を持つかチェックする。そうであれば\code{div\-/poly}は
変数を取り去りその問題を\code{div\-/terms}に渡す。\code{div\-/terms}は除算命令を項リスト上にて
実行する。\code{div\-/poly}は最終的に変数を再度\code{div\-/terms}の結果に取り付ける。
除算の商と剰余の両者を求める\code{div\-/terms}を設計することは便利だ。\code{div\-/terms}は
2つの項リストを引数として取り商の項リストと剰余の項リストのリストを返す。


以下の\code{div\-/terms}の定義を欠けた式を埋めることにより完成させよ。これを用いて
\code{div\-/poly}を実装せよ。\code{div\-/poly}は2つの多項式を引数として取り商と剰余の
多項式のリストを返す。

\begin{smallscheme}
(define (div-terms L1 L2)
  (if (empty-termlist? L1)
      (list (the-empty-termlist) (the-empty-termlist))
      (let ((t1 (first-term L1))
            (t2 (first-term L2)))
        (if (> (order t2) (order t1))
            (list (the-empty-termlist) L1)
            (let ((new-c (div (coeff t1) (coeff t2)))
                  (new-o (- (order t1) (order t2))))
              (let ((rest-of-result
                     ~\( \langle \)~~\var{compute rest of result recursively}~~\( \rangle \)~
                     ))
                ~\( \langle \)~~\var{form complete result}~~\( \rangle \)~
                ))))))
\end{smallscheme}
\end{quote}

\subsubsection*{記号代数の型の階層}

私達の多項式システムはある型(polynomials)のオブジェクトがどのようにして事実上
多くの異なる型のオブジェクトをその部分として持つ複雑なオブジェクトになり得るのか
について説明しました。これはジェネリックな命令を定義する場合の実際の困難さは
何も引き起していません。複合型の部分の必要な操作を実行するために適切なジェネリック命令を
インストールすることのみが必要です。実際に
多項式がある種の``再帰的データ抽象化''を形成し、多項式のその部分においてそれ自身が
多項式である場合があることを学びました。私達のジェネリック命令とデータ適従プログラミング
スタイルはこの複雑さを大した問題無しに扱うことができます。


一方で多項式代数はデータ型が自然にタワーに配置できないシステムです。例えば
係数が\( y \)の多項式である\( x \)の多項式を持つことができます。また
係数が\( x \)の多項式である\( y \)の多項式を持つことも可能です。
これらの型のどちらももう一方の``上''には自然には成り得ません。その上
各集合から両者の要素を足す必要は良くあります。
これを行う方法はいくつか存在します。1つの可能性としてはある多項式をもう一方の多項式の
型に項の展開と再配置を行うことで両者の多項式が同じ主な変数を持つように変換する
方法が考えられます。この上に変数で順序付けるタワーの様な構造を強制することで、常に
任意の多項式を最優先の変数が主で低優先度の変数が係数に埋め込まれた``基底形式''
に変換することができます。この戦略はとても良く行きます。ただし変換が多項式を
不必要に展開するかもしれないため、読み難くそして恐らく非効率にしてしまいます。
タワーの戦略は全くこの領域では全く自然ではありません。またはユーザが新しい型を
古い型を用いて種々の接続形式にて動的に創作する領域、例えば三角関数、羃級数、積分等
の任意の領域には自然ではないでしょう。



強制をコントロールすることが巨大スケールの代数操作システムの設計において深刻な問題
であることは驚くべきことではありません。そのようなシステムの多くの複雑性は様々な型
の間の関係性に携わっています。私達はまだ完全には強制を理解していないと言うこ
とは本当に公正でしょう。実際に私達はまだデータ型の概念を完全には理解していません。
それでもなお、私達が知っていることは強力な構造化とモジュラー方式の原則を伴ない
巨大システムの設計の支援を与えてくれます。

\begin{quote}
\heading{\phantomsection\label{Exercise 2.92}Exercise 2.92:} 
変数の順序付けを強要することで多項式パッケージを拡張し多項式の加算と乗算が異なる
変数の多項式に対しても働くようにせよ。(これは簡単ではない!)
\end{quote}

\subsubsection*{延長課題: 分数関数}


私達のジェネリック数値演算システムを拡張し\newterm{rational functions}(\jnewterm{分数関数})を
含むようにすることができます。分子と分母が多項式である以下の様な``分数''が存在します。
\begin{comment}

\begin{example}
 x + 1
-------
x^3 - 1
\end{example}

\end{comment}
\begin{displaymath}
 {x + 1 \over x^3 - 1}\,. 
\end{displaymath}

システムは分数関数の加算、減算、乗算、除算をできなければなりません。そして以下の様な
計算を行うために、
\begin{comment}

\begin{example}
 x + 1       x      x^3 + 2x^2 + 3x + 1
------- + ------- = -------------------
x^3 - 1   x^2 - 1    x^4 + x^3 - x - 1
\end{example}

\end{comment}
\begin{displaymath}
 {x + 1 \over x^3 - 1} + {x \over x^2 - 1} = 
	{x^3 + 2x^2 + 3x + 1 \over x^4 + x^3 - x - 1}\,. 
\end{displaymath}
\noindent

(ここでは加算は共通因数を取り除くことで簡約されています。通常の``たすき掛け''なら
5次多項式分の4次多項式の分数を生成しているでしょう。)


私達の分数演算パッケージを変更することでジェネリック命令を用いるようにすると
分数を最小の項に簡約する問題を除いて望むことができます。

\begin{quote}
\heading{\phantomsection\label{Exercise 2.93}Exercise 2.93:} 
分数演算パッケージを変更しジェネリック命令を使用するようにせよ。ただし\code{make\-/rat}
を変更し分数を最小の項に簡約することは試行しないようにせよ。あなたのシステムを
\code{make\-/rational}を2つの多項式上にて呼び出し分数関数を生成することでテストせよ。

\begin{scheme}
(define p1 (make-polynomial 'x '((2 1)(0 1))))
(define p2 (make-polynomial 'x '((3 1)(0 1))))
(define rf (make-rational p2 p1))
\end{scheme}

ここで\code{rf}を自身に\code{add}を用いて足せ。この加算手続が分数を最小項に簡約しない
ことを確認するだろう。
\end{quote}

\noindent
多項式の分数を整数で用いたのと同じ考えを用いて最小の項に簡約することができます。
\code{make\-/rat}を変更し分子と分母の両方を最大公約数で割ります。``Greatest Common Denominator''
(GCD:最大公約数)の概念は多項式に対しても意味を成します。実際に2つの多項式の\acronym{GCD}を
整数に対して働く、基本的に同じユークリッドのアルゴリズムを用いて求めることができます。\footnote{
ユークリッドのアルゴリズムが多項式に対して働くという事実は代数学において多項式が
\newterm{Euclidean ring}(\jnewterm{ユークリッド環})と呼ばれるある種の代数の定義域を形成すると述べることにより
形式化されます。ユークリッド環とは加算、減算、そして可換な乗算を許す定義域であり、
環の各元\( x \)に対する正の整数の``大きさ''\( m(x) \)の割り当て方法とそれに対する性質として
任意の非ゼロな\( x \)と\( y \)に対し\( m(xy) \ge m(x) \)であると共に、
与えられた任意の\( x \)と\( y \)に対し\( y = qx + r \)となる\( q \)が存在し、
\( r = 0 \)または\( m(r) < m(x) \)であることが言えます。
抽象化の視点からこれがユークリッドのアルゴリズムがうまく行くのに必要な条件です。
整数の定義域に対して、整数の大きさ\( m \)はその整数の絶対値です。多項式の定義域においては
多項式の大きさはその次数です。}整数版は以下のとおりです。


\begin{scheme}
(define (gcd a b)
  (if (= b 0)
      a
      (gcd b (remainder a b))))
\end{scheme}

\noindent
これを用いて、項リスト上で働く\acronym{GCD}命令を定義するための明かな変更を行うことができます。

\begin{scheme}
(define (gcd-terms a b)
  (if (empty-termlist? b)
      a
      (gcd-terms b (remainder-terms a b))))
\end{scheme}

\noindent
ここで\code{remainder\-/terms}は\link{Exercise 2.91}で実装された項リストの除算命令
\code{div\-/terms}により返されるリストの剰余部を取り出します。

\begin{quote}
\heading{\phantomsection\label{Exercise 2.94}Exercise 2.94:} 
\code{div\-/terms}を用いて手続\code{remainder\-/terms}を実装し、それを用いて\code{gcd\-/terms}を
上記のように定義せよ。次に2つの多項式の多項式\acronym{GCD}を求める手続\code{gcd\-/poly}
を書け。(この手続は2つの多項式が同じ変数でなければエラーを発しなければならない)。
多項式に対しては\code{gcd\-/poly}を簡約し、通常の数値に対しては通常の\code{gcd}に簡約する
ジェネリック命令\code{greatest\-/common\-/divisor}をシステムにインストールせよ。テストとして以下を試せ。

\begin{scheme}
(define p1 (make-polynomial 
            'x '((4 1) (3 -1) (2 -2) (1 2))))
(define p2 (make-polynomial 'x '((3 1) (1 -1))))
(greatest-common-divisor p1 p2)
\end{scheme}

\noindent

次にその結果を手でチェックせよ。
\end{quote}

\begin{quote}
\heading{\phantomsection\label{Exercise 2.95}Exercise 2.95:} 
以下の多項式\( P_1 \), \( P_2 \), \( P_3 \)を定義せよ。
\begin{comment}

\begin{example}
P_1 : x^2 - 2x + 1

P_2 : 11x^2 + 7

P_3 : 13x + 5
\end{example}

\end{comment}
\begin{displaymath}
%  \eqalign{
% 	P_1 	&: \quad x^2 - 2x + 1, \cr
% 	P_2 	&: \quad 11x^2 + 7, \cr
% 	P_3 	&: \quad 13x + 5. \cr
% } 
\begin{array}{l@{{}:}l}
	P_1 	& \quad x^2 - 2x + 1, \\
	P_2 	& \quad 11x^2 + 7, \\
	P_3 	& \quad 13x + 5. 
\end{array}
\end{displaymath}


次に\( P_1 \)と\( P_2 \)の積\( Q_1 \)、\( P_1 \)と\( P_3 \)の積\( Q_2 \)を定義し、
\code{greatest\-/common\-/divisor}(\link{Exercise 2.94})を用いて\( Q_1 \)と\( Q_2 \)
の\acronym{GCD}を求めよ。答が\( P_1 \)と同じにならないことに注意せよ。これが
非整数命令の演算が\acronym{GCD}に伴なう困難さを生じさせることの例を示している。\footnote{
\acronym{MIT} Schemeの様な実装ではこの問題は\( Q_1 \)と\( Q_2 \)の実際の約数を分数係数を
伴なって生成します。多くのSchemeシステムでは整数の除算が精度に限界のある小数を生成するため、
正しい約数を得るのに失敗します。}
何が起こったのか正しく理解するため、\acronym{GCD}を求める間\code{gcd\-/terms}をトレースするか
この除算を手で試行してみよ。
\end{quote}

\noindent
\link{Exercise 2.95}で示された問題を以下に示す(整数係数の多項式の場合のみ実際には動作する)
\acronym{GCD}アルゴリズムの変更を用いることで解決することができます。
\acronym{GCD}の演算中の一切の多項式の除算の前に、被除数を一切の分数が除算処理の間に現れない
よう保証するために選ばれた整数定数因数を掛けます。答は従って実際の\acronym{GCD}より整数定数
因数の分異なります。しかしこれは分数関数を最小の項に簡約する場合には問題になりません。
\acronym{GCD}は分子と分母の両方を割るために利用されるため、整数定数因数は相殺されます。



より正確に述べれば、もし\( P \)と\( Q \)が多項式である場合、\( O_1 \)を\( P \)の次数とし
(つまり\( P \)の最大項の次数とし)、\( O_2 \)を\( Q \)の次数とします。\( c \)を\( Q \)の
第一の係数とします。すると\( P \)を\newterm{integerizing factor}(\jnewterm{整数化因数})\( c^{1 + O_1 - O_2} \)
で掛けると、結果の多項式は\code{div\-/terms}アルゴリズムを用いて一切の分数を生じずに\( Q \)で割る
ことができます。
被除数をこの定数で乗算した後に割る命令は時々\( P \)の\( Q \)による\newterm{pseudodivision}(\jnewterm{擬除算})と呼ばれます。
除算の剰余は\newterm{pseudoremainder}(\jnewterm{擬剰余})と呼ばれます。

\begin{quote}
\heading{\phantomsection\label{Exercise 2.96}Exercise 2.96:}
\begin{enumerate}[a]

\item
手続\code{pseudoremainder\-/terms}を実装せよ。これは\code{remainder\-/terms}と同様であるが
\code{div\-/terms}を呼ぶ前に被除数を上で説明した整数化因数で掛ける。
\code{gcd\-/terms}を変更し\code{pseudoremainder\-/terms}を用いるようにし、\code{greatest\-/common\-/divisor}が
整数係数の答を\link{Exercise 2.95}の例にて生ずることを確認せよ。

\item
\acronym{GCD}はこれで整数係数を得る。しかしそれらは\( P_1 \)の物よりも大きい。
\code{gcd\-/terms}を変更し解の係数から全ての係数をそれらの(整数)の最大公約数により割ることで共通因数を取り除く
ようにせよ。
\end{enumerate}
\end{quote}

\noindent
従って、以下に分数関数をどのようにして規約分数に簡約するかを説明します。

\begin{itemize}

\item
\link{Exercise 2.96}の\code{gcd\-/terms}の版を用いて、分子と分母の\acronym{GCD}を求める

\item
\acronym{GCD}を得たら分子と分母の両方に同じ整数化因数を\acronym{GCD}で割る前に
掛けることで\acronym{GCD}による除算が非整数な係数を生じないようにする。
因数として\acronym{GCD}の最初の係数を\( 1 + O_1 - O_2 \)乗した物を用いることができ、
この時\( O_2 \)は\acronym{GCD}の次数であり、\( O_1 \)は分子と分母の最大次数である。
こうすることで分子と分母を\acronym{GCD}で割っても分数を生じない。

\item
この操作の結果は分子と分母が整数係数になる。係数は通常とても巨大になる。理由の
全ては整数化因数のせいだ。そのため最終ステップは分子と分母の全ての係数の
(整数の)最大公約数を求めてこの約数で割ることで冗長な因数を取り除くことである。

\end{itemize}

\begin{quote}
\heading{\phantomsection\label{Exercise 2.97}Exercise 2.97:}
\begin{enumerate}[a]

\item
このアルゴリズムを、2つの項リスト\code{n}と\code{d}を引数として取り上で説明された
アルゴリズムにて\code{n}と\code{d}を最小の項に簡約したリスト\code{nn}と\code{dd}を返す
手続\code{reduce\-/terms}として実装せよ。また\code{add\-/poly}と同様に2つの多項式が同じ変数を持つか
チェックする手続\code{reduce\-/poly}も書け。もしそうである場合\code{reduce\-/poly}は変数を取り去り
問題を\code{reduce\-/terms}に渡す。そして\code{reduce\-/terms}により与えられた2つの項リストに
再び変数を取り付ける。

\item
元の\code{make\-/rat}が整数に対して行ったことを行う\code{reduce\-/terms}と同様の手続を定義せよ。

\begin{scheme}
(define (reduce-integers n d)
  (let ((g (gcd n d))) (list (/ n g) (/ d g))))
\end{scheme}

\noindent
次に\code{reduce}をジェネリック命令として定義する。
これは\code{apply\-/generic}を呼び、(\code{polynomial}型引数に対しては)\code{reduce\-/poly}を呼び出し、
(\code{scheme\-/number}型引数に対しては)\code{reduce\-/integers}を呼び出す。
これで\code{make\-/rat}に与えられた分子と分母を接続して分数を形成する前に\code{reduce}を
呼ばせることで、簡単に分数数値演算パッケージに分数を最小の項に約分させることができる。

\begin{scheme}
(define  p1 (make-polynomial 'x '((1 1) (0  1))))
(define  p2 (make-polynomial 'x '((3 1) (0 -1))))
(define  p3 (make-polynomial 'x '((1 1))))
(define  p4 (make-polynomial 'x '((2 1) (0 -1))))
(define rf1 (make-rational p1 p2))
(define rf2 (make-rational p3 p4))
(add rf1 rf2)
\end{scheme}

正しい答を得るかどうか、正しく最小の項に簡約されるかどうか確認せよ。
\end{enumerate}
\end{quote}

\noindent
\acronym{GCD}の計算は分数関数の操作を行うどんなシステムにおいても心臓部に存在します。
上で用いられたアルゴリズムは数学的には簡単ですが非常に遅いです。。遅さの原因の一部は
除算命令の大きな値であり、他には擬除算により生じる非常に大きな中間時の係数のため
となります。代数操作システムの活発な開発領域の1つは多項式の\acronym{GCD}を求める
より良いアルゴリズムの設計です。\footnote{多項式の\acronym{GCD}を求めるための
1つの著しく効率が良く洗練された手法はRichard \link{Zippel (1979)}により発見されました。
この手法は\link{Chapter 1}にて議論した素数性の高速なテストと同様の乱選アルゴリズムです。
Zippelの本(\link{Zippel 1993})はこの手法を多項式の\acronym{GCD}を求める他の方法と共に
解説しています。}


%=======================================================================================================
%=======================================================================================================
%=======================================================================================================

\chapter{モジュール方式、オブジェクト、状態}
\label{Chapter 3}

\begin{quote}
\begin{greek}
Mεταβάλλον ὰναπαύεται
\end{greek}

(例え変化している間も、それは静止していた)

---Heraclitus
\end{quote}

\begin{quote}
Plus \c{c}a change, plus c'est la m\^{e}me chose.

(より多くが変化する程、より同じであり続ける)

---Alphonse Karr
\end{quote}

\vspace{1.0em}

\noindent
ここまでの章はプログラムが作成される基礎的な要素を紹介しました。どのようにして
プリミティブな手続とプリミティブなデータが接続され複合要素を構築するかについて
学び、また抽象化が巨大システムの複雑さに立ち向かうことを手助けする核心であることを
学習しました。しかしこれらのツールはプログラムを設計するのに十分ではありません。
効果的なプログラム統合はプログラム設計全体の形式化をガイドすることが可能な組織的原則を
必要とします。具体的には巨大システムの構造化を手助けする戦略が必要で、それによりそれらが
\newterm{modular}(\jnewterm{モジュラ})化されるよう、つまり``自然に''分離して開発と保守が可能な
論理的部品に分割されるようにします。



物理システムをモデル化したプログラムの構築に特に適切な1つの強力な設計戦略は
プログラムの構造をモデル化されるシステムの構造を元にすることです。システムの各オブジェクト
に対して対応する演算オブジェクトを構築します。各システムのアクションに対しては演算
モデル内の記号操作を定義します。この戦略を用いる見込は新しいオブジェクトやアクションを
供給するためにモデルを拡張することはプログラムに対する戦略上の変更を必要としないことです。
それらのオブジェクト、またはアクションの新しい記号上の類似物の追加のみ変更が必要です。
システムの組織化にて成功しているのなら、新しい機能の追加や古い物のデバッグにおいては
システムの特定の部分上のみで働く必要があります。



すると大体の場合、巨大プログラムを体系化する方法はモデル化されるシステムの私達の
認知により指示されます。この章では2つの大きく異なるシステム構造の``世界観''から
浮かび上がる2つの顕著な体系化戦略について調査します。最初の体系化戦略は
\newterm{objects}(\jnewterm{オブジェクト})に集中し、巨大システムをその振舞が時間と共に変化する
区別可能なオブジェクトの集合だと見ます。代替となる体系化戦略はシステム内を流れる
情報の\newterm{streams}(\jnewterm{ストリーム})に集中します。これは電子技術者の信号処理システム
の視点と同じです。


オブジェクトベースとストリーム処理の両方のアプローチは共にプログラミングにおける
重大な言語上の問題を浮かび上がらせます。オブジェクトでは演算オブジェクトがどのように
変化可能で、それでもその同一性を維持できるかについて関心を持たなければなりません。
このことがより機械的な、しかし論理的に扱い難い演算の\newterm{environment model}(\jnewterm{環境モデル})
のために、私達の古い演算の置換モデル(\link{Section 1.1.5})を諦めさせることになります。
オブジェクト、変化、同一性の取扱の難しさは私達の計算モデル内で時間に取り組むための
必要性の基本的な結論です。これらの問題はプログラムの並行実行を許可する場合にさらに
大きくなります。ストリームの取り組みは
私達のモデル内でシミュレートされた時間を計算機の中で評価の間に発生したイベントの順から
分断した時に最も全体に利用可能です。
\newterm{delayed evaluation}(\jnewterm{遅延評価})として知られるテクニックを用いてこれを達成します。



\section{代入と局所状態}
\label{Section 3.1}



私達は通常世界を独立したオブジェクトが占める物として見なします。各オブジェクトは
時間に伴ない変化する状態を持ちます。オブジェクトはその過去にその振舞が影響される時、
``状態を持つ''と呼びます。例えば銀行講座は預金と引き出しの取引の記録に依存する
``私は\$100引き出せるか?''という質問の答に状態を持ちます。オブジェクトの状態を1つ以上の
\newterm{state variables}(\jnewterm{状態変数})と見做すことができ、それらの間にオブジェクトの
現在の振舞を決定するための歴史についての十分な情報を保存します。
簡単な銀行システムでは口座の状態を口座の取引履歴全体を記憶するのではなく、現在の差引残高
と見做すことができるでしょう。



多くのオブジェクトから成るシステムではオブジェクトが完全に独立していることは稀です。
あるオブジェクトの状態変数を他のオブジェクトのそれに連結する相互作用を通して
各オブジェクトが他の状態に影響を与えることがあるでしょう。実際に、システムが分離した
オブジェクトから成るという見方は、システムの状態変数が密結合されたサブシステムが、
他のサブシステムとは疎結合であるというグループに分けられる時最も便利です。



このシステムの見方はシステムの演算モデルの体系化に対する強力なフレームワークに成ります。
そのようなモデルをモジュール化するためにはシステム内の実際のオブジェクトをモデル化する
計算オブジェクトに分離せねばなりません。各計算オブジェクトは実際のオブジェクトの状態を
説明するそれ自身の\newterm{local state variables}(\jnewterm{ローカル状態変数})を持たねばなりません。
モデル化されるシステム内のオブジェクトの状態は経時変化するため、計算オブジェクトに相対する
状態変数も変化しなければなりません。もし私達がシステム内の時の流れを計算機内で経過する時で
モデル化することを選択するのならば、振舞がプログラムが実行するにつれ変化する計算オブジェクトを
構築する手段を持たねばなりません。具体的には、もし状態変数をプログラミング言語内の通常の
記号名にてモデル化を行いたいのならば、その言語は名前に関連する値を変化することができる
\newterm{assignment operator}(\jnewterm{代入演算子})を提供せねばなりません。



\subsection{局所状態変数}
\label{Section 3.1.1}



時間的に変化する状態を伴なう計算オブジェクトを持つことにより何を意味するのかを説明するため
に、銀行口座からお金を引き出す状況をモデル化してみましょう。これを引数として引き出される
\code{amount}(金額)を取る手続\code{withdraw}を用いて行います。もし口座の中に引き出しを受け入れるのに
十分なお金があるのならば、\code{withdraw}は引き出しの後に残る差引残高を返さねばなりません。
そうでなければ、\code{withdraw}は\emph{Insufficient funds}(資金不足)というメッセージを返します。
例えば口座を\$100で始めた場合、\code{withdraw}を用いて以下の一連の応答を受け取るはずです。

\begin{scheme}
(withdraw 25)
~\textit{75}~
(withdraw 25)
~\textit{50}~
(withdraw 60)
~\textit{"Insufficient funds"}~
(withdraw 15)
~\textit{35}~
\end{scheme}

\noindent
式\code{(withdraw 25)}が2度評価され異なる値を返していることに注目して下さい。
これは手続にとって新しい種類の振舞です。今までは全ての手続は数学上の関数を
計算する仕様だと見做すことができました。手続の呼出は与えられた引数に適用された
関数の値を計算しました。そして同じ手続に同じ引数を与えば場合の2度の呼出は
常に同じ結果を生じました。\footnote{実際にはこれは全く正しい訳ではありません。
例の1つは\link{Section 1.2.6}の乱数生成です。別の例は\link{Section 2.4.3}で紹介した命令-型テーブルに
に従って生じます。同じ引数を伴なう\code{get}の二度の呼出の値は間に入る\code{put}に
依存します。一方で、代入を紹介するまではそのような手続を自分達で作る方法は
無かった訳です。}



\code{withdraw}を実装するために、口座の差引残高を示す変数\code{balance}を用い、
\code{balance}にアクセスする手続\code{withdraw}を定義します。
\code{withdraw}手続は\code{balance}が少なくとも要求された\code{amount}と同じ大きさであるかを
チェックします。もしそうであれば\code{withdraw}は\code{balance}を\code{amount}分減らし、新しい
\code{balance}の値を返します。そうでなければ\code{withdraw}は残高不足のメッセージを返します。
以下に\code{balance}と\code{withdraw}の定義を示します。

\begin{scheme}
(define balance 100)
(define (withdraw amount)
  (if (>= balance amount)
      (begin (set! balance (- balance amount))
             balance)
      "Insufficient funds"))
\end{scheme}

\noindent
\code{balance}を減らすのは次の式により行われます。

\begin{scheme}
(set! balance (- balance amount))
\end{scheme}

\noindent
これは\code{set!}という特殊形式を用いています。その文法は次のとおりです。

\begin{scheme}
(set! ~\( \dark \langle \)~~\var{\dark name}~~\( \dark \rangle \)~ ~\( \dark \langle \)~~\var{\dark new-value}~~\( \dark \rangle \)~)
\end{scheme}

\noindent
ここでは\( \langle \)\var{name}\( \kern0.08em\rangle \)はシンボルであり、
\( \langle \)\var{new-value}\( \kern0.08em\rangle \)は任意の式です。
\code{set!}は\( \langle \)\var{name}\( \kern0.08em\rangle \)を変更し、その値が
\( \langle \)\var{new-value}\( \kern0.08em\rangle \)を評価して得られた結果に
します。この場合では\code{balance}を変更することでその新しい値が\code{balance}の以前の
値から\code{amount}を引いた結果になります。\footnote{


\code{set!}式の値は実装依存です。\code{set!}はその効果のためのみに用いられ、その値のために
用いられてはなりません。


その名前\code{set!}はSchemeで用いられる名前付けの慣習が反映されています。
変数の値を変更する命令(または\link{Section 3.3}で学ぶデータ構造を変える物)は感嘆符(ビックリマーク)で
終わる名前を与えられます。
これは述語をクエスチョンマークで終わる名前で指定するのと同様です。}



\code{withdraw}はまた\code{begin}という特殊形式も使用しており、これは2つの式を\code{if}のテストが
真の場合に評価されるようにします。最初に\code{balance}を減らし、次に\code{balance}の値を返します。
一般的に以下の式を評価すると

\begin{scheme}
(begin ~\( \dark \langle \)~~\( \dark exp_1 \)~~\( \dark \rangle \)~ ~\( \dark \langle \)~~\( \dark exp_2 \)~~\( \dark \rangle \)~ ~\( \dots \)~ ~\( \dark \langle \)~~\( \dark exp_k \)~~\( \dark \rangle \)~)
\end{scheme}

\noindent
\( \langle \)\( exp_1 \)\( \kern0.08em\rangle \)から
\( \langle \)\( exp_k \)\( \kern0.08em\rangle \)までの式は続けて評価され
最後の式\( \langle \)\( exp_k \)\( \kern0.08em\rangle \)が\code{begin}の形式全体の値として
返ります。\footnote{
私達は既に\code{begin}を暗黙的にプログラムの中で使用しています。Schemeでは手続のボディは連続する式と
なるからです。また\code{cond}式の各節の\( \langle \)\var{consequent}\( \kern0.08em\rangle \)の部分
は単一の式でなく一連の式にすることができます。}



\code{withdraw}は望んだ通りに働きますが、変数\code{balance}が問題を表します。
上で指定されたように、\code{balance}はグローバル環境にて定義された名前であり
自由に検査や変更のために任意の手続からアクセスすることができます。
どうにかして\code{balance}を\code{withdraw}の内在にすることで\code{withdraw}のみが
\code{balance}に直接アクセスでき、他の手続のどれもが\code{balance}には間接的に
(\code{withdraw}の呼出を通して)アクセスするようにできればともて良くなるでしょう。
こうすることが口座の状態を追跡するため\code{balance}が\code{withdraw}により利用される
局所状態変数であるという概念をより正確にモデル化します。



定義を以下のように書き直すことで\code{balance}を\code{withdraw}に内在させることができます。

\begin{scheme}
(define new-withdraw
  (let ((balance 100))
    (lambda (amount)
      (if (>= balance amount)
          (begin (set! balance (- balance amount))
                 balance)
          "Insufficient funds"))))
\end{scheme}

\noindent
ここで起こなったのは\code{let}を用いて初期値100に束縛されたローカル変数\code{balance}を
持つ環境を設置しました。この局所環境の中では\code{lambda}を用いて\code{amount}を引数に取り
以前の\code{withdraw}手続と同様に振る舞う手続を作成しています。この手続---\code{let}式の
評価の結果として返される物---は\code{new\-/withdraw}であり正確に\code{withdraw}と同じ
振舞をしますが、その変数\code{balance}は他のどの手続からもアクセスできません。\footnote{
プログラミング言語の専門語において変数\code{balance}は手続\code{new\-/withdraw}にカプセル化
されたと言います。カプセル化は\newterm{hiding principle}(\jnewterm{隠蔽原則})として知られる
一般的なシステム設計の原則を反映しています。隠蔽原則とはシステムの部分をお互いから守ることで
よりモジュール化の推進と頑強なシステムを作成することができるということです。それはつまり
情報へのアクセスを``知ることを必要とする''システムの部分ににも与えることによります。}



\code{set!}をローカル変数と組み合わせることはローカルな状態を持つ計算オブジェクトを
構築するのに用いる一般的なプログラミングテクニックです。残念なことに、このテクニックを
用いることは深刻な問題をもたらします。私達が最初に手続を紹介した時、評価の置換モデル(\link{Section 1.1.5})
も手続の適用が何を意味するのかの解釈を説明するために提供しました。
手続の適用は手続のボディを、形式パラメータをそれらの値で置換して評価することだと解釈
されるべきだと述べました。問題は言語に代入を紹介すると直ぐに置換は最早手続の適用モデル
として適切ではなくなります(なぜそうなのかについては\link{Section 3.1.3}で学びます)。結果として
技術的に今の時点ではなぜ\code{new\-/withdraw}手続が上で主張された通りに振る舞うのか理解する
手立てがありません。本当に\code{new\-/withdraw}の様な手続を理解するためには、手続適用の
新しいモデルの開発を必要とします。\link{Section 3.2}においてそのようなモデルを \code{set!}とローカル変数の
説明と共に紹介します。しかし最初に\code{new\-/withdraw}により設定される主題上のいくつかの変化に
ついて調査することにします。



以下の手続\code{make\-/withdraw}は``引き出し処理''を作成します。\code{make\-/withdraw}の
形式パラメータ\code{balance}は口座の初期残高を指定します。\footnote{
上の\code{new\-/withdraw}とは逆に、\code{balance}をローカル変数にするために\code{let}を使用する
必要がありません。形式パラメータは既にローカル変数であるためです。
\link{Section 3.2}の環境の評価モデルの議論の後にこのことはより明白になります。
(\link{Exercise 3.10}も参照して下さい)}

\begin{scheme}
(define (make-withdraw balance)
  (lambda (amount)
    (if (>= balance amount)
        (begin (set! balance (- balance amount))
               balance)
        "Insufficient funds")))
\end{scheme}

\noindent
\code{make\-/withdraw}は以下のように2つのオブジェクト\code{W1}と\code{W2}を作るのに使用できます。

\begin{scheme}
(define W1 (make-withdraw 100))
(define W2 (make-withdraw 100))

(W1 50)
~\textit{50}~
(W2 70)
~\textit{30}~
(W2 40)
~\textit{"Insufficient funds"}~
(W1 40)
~\textit{10}~
\end{scheme}

\noindent
\code{W1}と\code{W2}が完全独立したオブジェクトであり、各々がそれ自身のローカル状態変数
\code{balance}を持っていることを観察して下さい。ある口座からの引き出しは別の口座には
影響しません。



引き出し同様に預け入れを扱うオブジェクトを作ることもできます。従って簡単な銀行口座を
表現可能です。以下が指定した初期残高を持つ``銀行口座オブジェクト''を返す手続です。

\begin{scheme}
(define (make-account balance)
  (define (withdraw amount)
    (if (>= balance amount)
        (begin (set! balance (- balance amount))
               balance)
        "Insufficient funds"))
  (define (deposit amount)
    (set! balance (+ balance amount))
    balance)
  (define (dispatch m)
    (cond ((eq? m 'withdraw) withdraw)
          ((eq? m 'deposit) deposit)
          (else (error "Unknown request: MAKE-ACCOUNT"
                       m))))
  dispatch)
\end{scheme}

\noindent
各\code{make\-/account}の呼出はローカル状態変数\code{balance}を持つ環境を構築します。
この環境の中で\code{make\-/account}は\code{balance}にアクセスする手続\code{deposit}と\code{withdraw}を定義します。
また追加の手続\code{dispatch}は``メッセージ''を入力として取り2つのローカル手続の内1つを返します。
\code{dispatch}手続それ自身が銀行口座オブジェクトを表現する値として返されます。
これはまさに\link{Section 2.4.3}で学んだ\newterm{message-passing}(\jnewterm{メッセージパッシング})プログラミングスタイルです。
ただしここではそれをローカル変数を変更する能力と合わせて用いています。


\code{make\-/account}は以下のように使用できます。

\begin{scheme}
(define acc (make-account 100))
((acc 'withdraw) 50)
~\textit{50}~
((acc 'withdraw) 60)
~\textit{"Insufficient funds"}~
((acc 'deposit) 40)
~\textit{90}~
((acc 'withdraw) 60)
~\textit{30}~
\end{scheme}

\noindent
各\code{acc}の呼出は局所的に定義された\code{deposit}か\code{withdraw}手続を返し、指定された\code{amount}に適用されます。
\code{make\-/withdraw}を用いる場合でしたので、別の呼出し、

\begin{scheme}
(define acc2 (make-account 100))
\end{scheme}

\noindent
は完全に分離された口座オブジェクトを生成し、それ自身のローカルな\code{balance}を持ちます。

\begin{quote}
\heading{\phantomsection\label{Exercise 3.1}Exercise 3.1:} 
\newterm{accumulator}は1つの数値引数を持ち繰り返し呼ばれる手続で、引数を合計に
蓄積する。呼び出される度に現在の累積和を返す。アキュムレータ(累算器)を返す手続
\code{make\-/accumulator}を書け。アキュムレータはそれぞれが独立した合計を持つ。
\code{make\-/accumulator}への入力は累計の初期値を指定する。例えば、

\begin{scheme}
(define A (make-accumulator 5))
(A 10)
~\textit{15}~
(A 10)
~\textit{25}~
\end{scheme}
\end{quote}

\begin{quote}
\heading{\phantomsection\label{Exercise 3.2}Exercise 3.2:}
ソフトウェアテストアプリケーションでは演算処理の間に与えらえれた手続が何度呼ばれたか
を数えられると便利である。1引数手続\code{f}を入力として取る手続\code{make\-/monitored}を書け。
\code{make\-/monitored}の返す結果は第三の手続(\code{mf}としよう)は内部カウンタを保持することで
何回呼出されたかを追跡する。もし\code{mf}への入力が特別なシンボル\code{how\-/many\-/calls?}で
あるなら、\code{mf}はカウンタの値を返す。入力が特別なシンボル\code{reset\-/count}であるなら
\code{mf}はカウンタをゼロにリセットする。任意の他の入力に対しては\code{mf}はその入力上の
\code{f}呼出の結果を返しカウンタを1増やす。例えば監視版の\code{sqrt}手続を作ることができる
だろう。

\begin{scheme}
(define s (make-monitored sqrt))
(s 100)
~\textit{10}~
(s 'how-many-calls?)
~\textit{1}~
\end{scheme}
\end{quote}

\begin{quote}
\heading{\phantomsection\label{Exercise 3.3}Exercise 3.3:} 
\code{make\-/account}手続を変更しパスワードで守られた口座を作成するようにせよ。
即ち\code{make\-/account}はシンボルを追加引数として以下のように取得する。

\begin{scheme}
(define acc (make-account 100 'secret-password))
\end{scheme}


結果の口座オブジェクトはリクエストをアカウント作成時のパスワードが付随する場合のみ
処理を行いその他の場合には間違いだと返す。

\begin{scheme}
((acc 'secret-password 'withdraw) 40)
~\textit{60}~
((acc 'some-other-password 'deposit) 50)
~\textit{"Incorrect password"}~
\end{scheme}
\end{quote}

\begin{quote}
\heading{\phantomsection\label{Exercise 3.4}Exercise 3.4:} 
\link{Exercise 3.3}の\code{make\-/account}手続に別のローカル状態変数を追加することで
変更し、口座が7回連続間違ったパスワードでアクセスされた場合に手続\code{call\-/the\-/cops}
(警察を呼ぶ)を実行するようにせよ。
\end{quote}

\subsection{代入導入の利点}
\label{Section 3.1.2}



私達が学ぶに従い、代入を私達のプログラミング言語に導入したことは難しい概念上の
問題の藪の中へと導きます。それでもなおシステムをローカルな状態を持つオブジェクトの集合
として見ることはモジュラな設計を維持する為の強力なテクニックです。
簡単な例として、呼ばれる度に無作為な(ランダムな)整数を返す手続\code{rand}の設計について
考えてみて下さい。



``ランダムに選択''が何を意味するのかは全くわかりません。恐らく私達が欲しい物は
\code{rand}への連続した呼出が統計上の性質として均一な分散を持つ一連の数値を生じて
欲しいのでしょう。ここでは適切な数列を生成する手法については議論しません。
そうでなく、数値\( x_1 \)を与えて開始した場合に以下の数列を生成する性質を持つ
手続\code{rand\-/update}を既に持っていると想定しましょう。

\begin{scheme}
~\( \dark x_2 \)~ = (rand-update ~\( \dark x_1 \)~)
~\( \dark x_3 \)~ = (rand-update ~\( \dark x_2 \)~)
\end{scheme}

\noindent
すると数列\( x_1 \), \( x_2 \), \( x_3 \), \( \dots \)は望まれた統計的性質特性を持つでしょう。\footnote{
\code{rand\-/update}を実装する1つの一般的な方法は\( x \)は\( ax + b \) modulo \( m \)に更新されるとする、
この時\( a \), \( b \), \( m \)は適切に選択された整数であるというルールを用いることです。
\link{Knuth 1981}の3章は広範囲に及ぶ乱数列を生成するためのテクニックの議論を含んでおり、また
それらの統計的性質を規定しています。\code{rand\-/update}手続が数学上の関数を計算していることに注意
して下さい。同じ入力を2回与えられれば同じ出力を生成します。従って\code{rand\-/update}により生成
される数列は``ランダム''が数列のどの数値も以前の数値に関係が無いと主張するのであれば、
明らかに``ランダム''ではありません。``真の無作為性(ランダムネス)''と
上手く決定された計算で生成されるがそれでも適切な統計上特性を持つ\newterm{pseudo-random}(\jnewterm{擬似乱数})列の
間の関係は数学と哲学の難しい問題を巻き込む複雑な質問です。
Kolmogorov, Solomonoff, それにChaitinはこれらの問題の解明において大きな進展を上げました。
これに関する議論は\link{Chaitin 1975}に見つかります。
}


\code{rand}をある固定値\code{random\-/init}で初期化されるローカル状態変数\code{x}を持つ手続として
実装できます。\code{rand}への各呼出は現在の\code{x}の値の\code{rand\-/update}を演算し、
これを乱数として返し、また同時にこの値を\code{x}の新しい値として格納します。

\begin{scheme}
(define rand (let ((x random-init))
               (lambda () 
                 (set! x (rand-update x))
                 x)))
\end{scheme}

\noindent
もちろん、代入を用いずに単純に\code{rand\-/update}を直接呼ぶことで同じ乱数列を
生成することも可能でしたでしょう。しかし、これは私達のプログラムの乱数を用いる
任意の部分が明示的に\code{x}の現在の値を\code{rand\-/update}の引数として渡すために
記憶せねばならないことを意味することになります。
これがどれだけ不快であるかを気付くために、乱数を\newterm{Monte Carlo simulation}
(\jnewterm{モンテカルロシミュレーション})と呼ばれるテクニックを実装するために乱数を用いる
場合について考えてみましょう。


モンテカルロ法は巨大集合から無作為にサンプル試行を選択することと、その次に
それらの試行上の結果の集計から推測された確率を基準にして演繹を行うことから成り立ちます。
例えば\( \pi \)を\( 6/\pi^2 \)は2つの無作為に選択された整数に公約数が無い場合の確率
であるという事実を用いて近似値を求められます。言い換えると、2つの整数の最大公約数が1になる
場合ということです。\footnote{この定理はE. Ces\`aroによるものです。その議論と証明については\link{Knuth 1981}の
節4.5.2を参照して下さい。}
\( \pi \)の近似値を求めるためには数多くの試行を行います。各試行において2つの整数を無作為に
選択し、それらの\acronym{GCD}が1であるかをテストします。テストをパスした回数の割合は
\( 6/\pi^2 \)の近似値を与えてくれます。この値から\( \pi \)の近似値を得ます。



プログラムの心臓部は手続\code{monte\-/carlo}です。これは試行回数と引数が無く実行される度に
真偽値を返す手続として表される試行を引数として取ります。\code{monte\-/carlo}は試行を指定された
回数実行し、試行が真と判定された割合を表す数値を返します。

\begin{scheme}
(define (estimate-pi trials)
  (sqrt (/ 6 (monte-carlo trials cesaro-test))))
(define (cesaro-test)
   (= (gcd (rand) (rand)) 1))

(define (monte-carlo trials experiment)
  (define (iter trials-remaining trials-passed)
    (cond ((= trials-remaining 0)
           (/ trials-passed trials))
          ((experiment)
           (iter (- trials-remaining 1) 
                 (+ trials-passed 1)))
          (else
           (iter (- trials-remaining 1) 
                 trials-passed))))
  (iter trials 0))
\end{scheme}

\noindent
さて同じ計算を\code{rand}の代わりに\code{rand\-/update}を用いてやってみましょう。
局所状態をモデル化するために代入を用いない場合に続行を強制される手法です。

\begin{scheme}
(define (estimate-pi trials)
  (sqrt (/ 6 (random-gcd-test trials random-init))))
(define (random-gcd-test trials initial-x)
  (define (iter trials-remaining trials-passed x)
    (let ((x1 (rand-update x)))
      (let ((x2 (rand-update x1)))
        (cond ((= trials-remaining 0)
               (/ trials-passed trials))
              ((= (gcd x1 x2) 1)
               (iter (- trials-remaining 1)
                     (+ trials-passed 1)
                     x2))
              (else
               (iter (- trials-remaining 1)
                     trials-passed
                     x2))))))
  (iter trials 0 initial-x))
\end{scheme}

\noindent
プログラムは今も単純ではありますが、いくつかモジュール方式に対する苦痛を伴なう
侵害行為があります。\code{rand}を用いる最初の版ではモンテカルロ法を直接、引数として
任意の\code{experiment}手続を取る全体的な\code{monte\-/carlo}手続にて表すことができました。
乱数生成に対する状態変数の無い2つ目の版では\code{random\-/gcd\-/test}が明示的に乱数
\code{x1}と\code{x2}を管理し、\code{x2}を繰り返しのループを通して\code{rand\-/update}に対する
新しい入力としてリサイクルしなければなりません。この明示的な乱数の取扱は
テスト結果の蓄積構造と
私達の試行が2つの乱数を利用するという事実を一緒に密に結合します。例え他のモンテカルロの試行が
1つや3つの乱数を使うにしてもです。トップレベルの手続\code{estimate\-/pi}ですら乱数の初期値を
提供することに関心を持たねばなりません。乱数生成器の内部がプログラムの他の部分に漏れ出す
ことはモンテカルロの考えを分離し他のタスクに適用することを難しくします。
プログラムの最初の版では代入が乱数生成器の状態を\code{rand}手続の中にカプセル化しているため
乱数生成器の詳細はプログラムの他の部分からの独立を維持しています。



モンテカルロの例にて説明された一般的な事象は以下のとおりです。複雑なプロセスの
一部の視点からは他の部分は時間に従い変化するように見えます。それらは時間と共に
変化するローカルな状態を隠しています。もしこの分解を反映する構造を持つ計算機プログラム
を書きたいのならば、振舞が時間と共に変換する(銀行口座と乱数生成器の様な)計算オブジェクトを作成します。
私達は状態をローカル状態変数を用いてモデル化し、状態の変化をそれらの変数への代入にて
モデル化します。



この議論を次のよう述べることで結論付けることは魅力的です。曰く、代入と状態を局所変数に
隠す技術を紹介することで、追加のパラメータを渡すことで全ての状態が明示的に操作されなければ
ならない場合よりも、よりモジュール化を行う方法でシステムの構造化を行えます、と。
残念ながらこれから学ぶように、このお話はそんなに簡単ではありません。

\begin{quote}
\heading{\phantomsection\label{Exercise 3.5}Exercise 3.5:} 
\newterm{Monte Carlo integration}(\jnewterm{モンテカルロ積分})はモンテカルロ・シミュレーションを用いて
定積分を推測する手法だ。述語\( P(x, y) \)で記述される空間の領域の面積を計算する場合について
考えてみる。述語\( P(x, y) \)は点\( (x, y) \)が領域の中であれば真であり、そうでなければ偽である。
例えば中心(5, 7)、半径3の円に含まれる領域は\( (x - 5)^2 + (y - 7)^2 \le 3^2 \)であるかテストする
述語にて記述される。そのよう述語で記述された領域の面積を推測するためにその領域を含む長方形を
選択することから始める。例として対角線上の角を(2, 4)と(8, 10)に持つ長方形は先程の円を含む。
期待される積分はその領域が位置する長方形の一部の面積だ。長方形の中の点\( (x, y) \)を
不作為に選択し、各点に対し\( P(x, y) \)をテストしその点が元の領域の中であるかどうかを決定すること
で積分を推定することができる。もしこの試行を数多くの点で行えば領域の中に落ちる点の割合は
長方形の内のその領域の割合の推定値を与えるはずだ。従ってこの割合に長方形全体の面積を掛けること
で積分の推定値を生成可能である。



モンテカルロ積分を手続\code{estimate\-/integral}として実装せよ。これは引数として述語\code{P}、
長方形の上下界として\code{x1}, \code{x2}, \code{y1}, \code{y2}、そして推定値を生成するため実行する
試行回数を取る。手続は上で\( \pi \)を推測するために使用した\code{monte\-/carlo}手続を同じく
使用せねばならない。\code{estimate\-/integral}を用いて\( \pi \)の推測値を単位円の面積を
測ることで求めよ。


与えられた値域から不作為に選択された数値を返す手続を持つことが便利であると発見するかもしれない。
以下の\code{random\-/in\-/range}手続はこれを\link{Section 1.2.6}で使用した\code{random}手続を用いて実装する。
これは入力より小さな非負数を返す。\footnote{\acronym{MIT} Schemeはそのような手続を提供します。
もし\code{random}が(\link{Section 1.2.6}での様に)整数を渡されれば、整数を返します。しかし(この課題のように)
小数を渡された場合には小数を返します。}

\begin{scheme}
(define (random-in-range low high)
  (let ((range (- high low)))
    (+ low (random range))))
\end{scheme}
\end{quote}

\begin{quote}
\heading{\phantomsection\label{Exercise 3.6}Exercise 3.6:} 
乱数生成器を与えられた値から始まる列を生成するためリセットすることができれば便利である。
シンボル\code{generate}またはシンボル\code{reset}のどちらかを引数として呼び出す新しい\code{rand}手続を設計せよ。
これは次のように振る舞う。\code{(rand 'generate)}は新しい乱数を生成する。
\code{((rand 'reset)}\( \;\langle \)\var{new-value}\( \kern0.11em\rangle \)\code{)}は内部の状態変数を指定された
\( \langle \)\var{new-value}\( \kern0.08em\rangle \)でリセットする。
従って状態をリセットすることで繰り返し可能な列の生成が行える。これは乱数を用いるプログラムのテストや
デバッグにおいてとても役に立つ。

\end{quote}

\subsection{代入導入のコスト}
\label{Section 3.1.3}


ここまで見てきたとおり、\code{set!}命令はローカルな状態を持つオブジェクトのモデル化を可能にします。
しかしこの利点は犠牲を伴ないます。私達のプログラミング言語は\link{Section 1.1.5}で紹介した手続適用の置換モデルを用いて説明
することができません。加えて、プログラミング言語の間にオブジェクトと代入を取り扱うための適切なフレームワークとなる
``良い''数学上の特性を伴なう簡単なモデルが存在しません。



代入を使わない限り、同じ引数を伴なう同じ手続の二度の評価は同じ結果を生じ、手続は数学上の関数の計算と
見ることができます。私達がこの本の最初の二章を通じて行ってきたような代入を使用しないプログラミングは、
それ故に\newterm{functional programming}(\jnewterm{関数型プログラミング})として知られています。



代入が問題をどのように困難にするかを理解するために、\link{Section 3.1.1}の\code{make\-/withdraw}手続を
残額が十分であるかのチェックを行わない様に単純化した版について考えます。

\begin{scheme}
(define (make-simplified-withdraw balance)
  (lambda (amount)
    (set! balance (- balance amount))
    balance))
(define W (make-simplified-withdraw 25))
(W 20)
~\textit{5}~
(W 10)
~\textit{-5}~
\end{scheme}

\noindent
この手続と以下の\code{set!}を使用しない\code{make\-/decrementer}手続とを比べてみて下さい。

\begin{scheme}
(define (make-decrementer balance)
  (lambda (amount)
    (- balance amount)))
\end{scheme}

\noindent
\code{make\-/decrementer}は指定された残高\code{balance}からその入力を引きます。
しかし連続した呼び出しにおいて\code{make\-/simplified\-/withdraw}のような累積効果はありません。

\begin{scheme}
(define D (make-decrementer 25))
(D 20)
~\textit{5}~
(D 10)
~\textit{15}~
\end{scheme}

\noindent
\code{make\-/decrementer}がどのように働くかの説明には置換モデルを使用できます。
例えば以下の式の評価を解析してみましょう。

\begin{scheme}
((make-decrementer 25) 20)
\end{scheme}

\noindent
最初に結合のオペレータを\code{make\-/decrementer}のボディの\code{balance}を25と置き換えることにより
簡約します。式は以下のようになります。

\begin{scheme}
((lambda (amount) (- 25 amount)) 20)
\end{scheme}

\noindent
\code{lambda}式のボディにある\code{amount}を20と置き換えることでオペレータを適用します。

\begin{scheme}
(- 25 20)
\end{scheme}

\noindent
最終的な答は5です。


しかしもし\code{make\-/simplified\-/withdraw}に対しても同様の置換分析を試みれば何が起こるか観察して
みて下さい。

\begin{scheme}
((make-simplified-withdraw 25) 20)
\end{scheme}

\noindent
最初に\code{make\-/simplified\-/withdraw}のボディにある\code{balance}を25に置き換えることでオペレータを簡約します。
これにより式は以下のようになります。\footnote{
\code{set!}式に存在する\code{balance}は置き換えません。なぜなら\code{set!}内の
\( \langle \)\var{name}\( \kern0.08em\rangle \)は評価されないからです。
もしこれを置き換えれば\code{(set! 25 (- 25 amount))}を得ることになりますが、これは意味がありません。}

\begin{scheme}
((lambda (amount) (set! balance (- 25 amount)) 25) 20)
\end{scheme}

\noindent
ここで\code{lambda}式のボディの中の\code{amount}を20に置き換えてオペレータを適用します。

\begin{scheme}
(set! balance (- 25 20)) 25
\end{scheme}

\noindent
もし置換モデルに執着するのであれば、手続の適用の意味は最初に\code{balance}を5に設定し、
次に式の値として25を返すと言わざるを得ません。これは間違った答を得ます。
正しい答を得るためには、どうにかして最初の位置の\code{balance}(\code{set!}の効果以前)を
2つ目の\code{balance}(\code{set!}の効果の後)から区別せねばなりません。そして置換モデルはこれを行うことが
できません。


ここでの問題は置換は詰まるところ、私達の言語のシンボルが本質的に値の名前であるという概念に
基いています。しかし\code{set!}と変数の値が変更できるという考えを紹介してから直ぐに、
変数は最早単純な名前ではあり得ません。今では変数はどうにかして値が格納できる場所を
参照し、その場所に格納された値は変更することが可能です。\link{Section 3.2}にて、環境がどのようにして
この``場所''の役割を演じるのかについて学びます。

\subsubsection*{同一性と変更}



ここで表出した問題は特定の演算モデルが単に崩壊したよりもずっと深淵です。
私達の計算モデルに変更を紹介して直ぐに、以前は簡単であった多くの概念が
難問と化します。2つの物が``同じ''であるという観念について考えてみましょう。



\code{make\-/decrementer}を同じ引数を与えて二度呼び二つの手続を作成したとします。

\begin{scheme}
(define D1 (make-decrementer 25))
(define D2 (make-decrementer 25))
\end{scheme}

\noindent
\code{D1}と\code{D2}は同じでしょうか? 無難な答はYESです。\code{D1}と\code{D2}は同じ
計算上の振舞を持ち、それぞれが入力から25を引く手続です。
実際に\code{D1}は任意の計算において結果を変えることなく\code{D2}の代替にできます。



これと\code{make\-/simplified\-/withdraw}の二度の呼出とを対比します。

\begin{scheme}
(define W1 (make-simplified-withdraw 25))
(define W2 (make-simplified-withdraw 25))
\end{scheme}

\noindent
\code{W1}と\code{W2}は同じでしょうか? もちろん違います。\code{W1}と\code{W2}の呼出は区別可能な効果を持ちます。
以下の応答列によりそれが示されます。

\begin{scheme}
(W1 20)
~\textit{5}~
(W1 20)
~\textit{-15}~
(W2 20)
~\textit{5}~
\end{scheme}

\noindent
例え\code{W1}と\code{W2}が同じ式\code{(make\-/simplified\-/withdraw 25)}を評価することで
作成されたという点で``同じ''であっても、\code{W1}が式の評価の結果を変えずに
任意の式で\code{W2}の代替になるかというのは正しくありません。



式において式の値を変化せずに``等しい物は等しい物で置き換えられる''という観念を支持する
言語は\newterm{referentially transparent}(\jnewterm{参照透明})と呼ばれます。
参照透明は私達の計算機言語に\code{set!}を含めた時侵害されました。これがいつ式を等価な式で
置き換えることで簡約できるかを決定することを扱いにくくします。
結果的に、代入を用いるプログラムについての推測は大幅により難しくなります。


参照透明を無しで済ませば、計算オブジェクトが``同じ''であることを意味する概念が
形式的に捉えることが難しくなります。本当に実際の世界での``等価''の意味は
私達のプログラムモデルはそれ自身において全く明確になりません。一般的に2つの
恐らく同じオブジェクトが本当に``同じ物''であるかは一方のオブジェクトを変更した場合に
もう一方のオブジェクトが同様に変化したかを観察するしか手立てがありません。
しかしオブジェクトが``変更された''ことを``同じ''オブジェクトを2回観察し,オブジェクトのある
属性が1回目の観察から次に対して異なるかどうかを見る以外にどうやって判断できるのでしょうか。
従って``同一性''の何らかの\emph{a priori}(先験的な)概念無しに``変化''を判断することができません。
そして変化の結果を観察せずに同一性を判断することはできないのです。


この問題がプログラミングにおいてどのように発生するかの例として、PeterとPaulが\$100入っている
口座を持っている状態について考えましょう。これをモデル化するに当たって以下の定義と

\begin{scheme}
(define peter-acc (make-account 100))
(define paul-acc (make-account 100))
\end{scheme}

\noindent
以下の定義では大きな違いがあります。

\begin{scheme}
(define peter-acc (make-account 100))
(define paul-acc peter-acc)
\end{scheme}

\noindent
最初の状況では、2つの銀行口座は区別できます。Peterにより行われた取引はPaulの口座には
影響を与えません。逆も同じです。2つ目の状況ではしかし、\code{paul\-/acc}が\code{peter\-/acc}と
同じ物になるよう定義しました。実際にPeterとPaulは今では連結銀行口座を持っており
Peterが\code{peter\-/acc}から引き出しを行えばPaulは\code{paul\-/acc}の残額が減ったことを観察
するでしょう。これらの2つの似ているが区別できる状況は計算モデルの構築において混乱の
元となり得ます。具体的には、共有口座のために1つのオブジェクト(銀行口座)が2つの
異なる名前(\code{peter\-/acc}と\code{paul\-/acc})を持つことは特に混乱します。プログラムの中で
\code{paul\-/acc}を変更することができる箇所を全て探す場合、\code{peter\-/acc}を変更する箇所もまた
探さねばならないことを覚えておかなければなりません。\footnote{
単一の計算オブジェクトが複数の名前によりアクセスされる事象は\newterm{aliasing}(\jnewterm{エイリアシング})
として知られています。連結銀行口座の状況はエイリアスのとても簡単な例を説明します。
\link{Section 3.3}では``識別可能''な複合データ構造が一部を共有するようなさらに複雑な例について学びます。
バグはプログラムの中でオブジェクトに対する変更が``副作用''として``異なる''オブジェクトに
対しても変更を行い得る場合を忘れている時に発生します。2つの``異なる''オブジェクトが実際には
異なるエイリアスの下に現れる単一のオブジェクトであるためです。これらは
\newterm{side\-/effect bugs}(\jnewterm{副作用バグ})と呼ばれる物で位置の特定や分析がとても難しいため
一部の人々はプログラミング言語は副作用やエイリアスを許可しないよう設計されるべきだと
提案しています。(\link{Lampson et al. 1981}; \link{Morris et al. 1980})}

上記の``同一性''と``変更''上の見解への参照と共に、もしPeterとPaulは差引残高を調べられるだけで
差引残高を変更する命令を実行することができない場合、2つの口座が区別できるかどうかという
問題が無意味になるのかについて注意して下さい。一般的に、データオブジェクトを変更しない限り、
複合データオブジェクトをまさにそれの部分の全体であると見做すことができます。例えば、
分数はその分子と分母により決定されます。しかしこの見方は変更が存在する時には、
複合データオブジェクトがそれが組み立てられている部品とは異質の``アイデンティティ(自己同一性)''を持つ場合には
有効ではありません。
銀行口座は例え引き出しを行うことで残高を変更しても依然として``同じ''銀行口座です。反対に、
同じ状態情報を持つ2つの異なる銀行口座を持つこともできるでしょう。この複雑さは私達のプログラミング言語に
よる物ではなく、私達のオブジェクトとしての銀行口座の認知によるものです。
例えば私達は通常分数を同一性を保ちながら変更可能なオブジェクトだとは見做しません。分子を変更
したら``同じ''分数をだとは思いません。


\subsubsection*{命令型プログラミングの落とし穴}



関数型プログラミングとは反対に、代入を広範囲に用いるプログラミングは
\newterm{imperative programming}(\jnewterm{命令型プログラミング})として知られています。
計算モデルに関する複雑さを上げるのに加えて、命令型スタイルで書かれたプログラムは
関数型プログラムでは起こり得ないバグを起こしやすくなります。
例えば\link{Section 1.2.1}の反復指数プログラムを思い出して下さい。

\begin{scheme}
(define (factorial n)
  (define (iter product counter)
    (if (> counter n)
        product
        (iter (* counter product) (+ counter 1))))
  (iter 1 1))
\end{scheme}

\noindent
内部の反復ループ内で引数を渡す代わりに変数\code{product}と\code{counter}の値の明示的な代入を用いることで
より命令型のスタイルを受け入れることができます

\begin{scheme}
(define (factorial n)
  (let ((product 1)
        (counter 1))
    (define (iter)
      (if (> counter n)
          product
          (begin (set! product (* counter product))
                 (set! counter (+ counter 1))
                 (iter))))
    (iter)))
\end{scheme}

\noindent
これはプログラムにより生成される結果に違いがありません。しかし微妙な罠を招いています。
私達は代入の順序をどのように決めたのでしょうか。たまたま上のプログラムは正しく書かれています。
しかし代入を逆順に書くことは

\begin{scheme}
(set! counter (+ counter 1))
(set! product (* counter product))
\end{scheme}

\noindent
異なる間違った結果を生じることでしょう。一般的に代入を伴なうプログラミングは
各命令が変更された変数の正しい版を用いることを確認するために、
私達に注意深く代入の相対順序を考えることを強制します。
この問題は単純に関数型プログラミングでは起こりません。\footnote{
この視点ではプログラミング入門が高度に命令型スタイルを用いながら最も頻繁に教えられている
ことは皮肉な事です。これは1960年代から1970年代までの間中、手続を呼ぶプログラムは本質的に
代入を実行するプログラムよりも非効率であるに違いないという共通の信念の名残でしょう。
(\link{Steele 1977}がこの論争が誤りであることを示しました)。
あるいは行毎の代入を思い浮かべることが初心者にとって手続呼出よりも簡単であるという見方も
あるでしょう。どのような理由しろ、このことは初級プログラマに対し
``私はこの変数をあれより前か後に設定するべきか?''といったプログラミングを複雑にし、重要な
考慮点を不明瞭にする心配事をしばしば負わせることになります。}



命令型プログラムの複雑さは複数のプロセスが並行に実行されるアプリケーションを
考える場合により悪くなります。この点については\link{Section 3.4}にて戻ります。
しかし最初に代入を含む表現のための計算モデルを提供する場合の問題を提示します。
そしてシミュレーションの設計においてローカルな状態を持つオブジェクトの使用を
検討します。

\begin{quote}
\heading{\phantomsection\label{Exercise 3.7}Exercise 3.7:} 
\link{Exercise 3.3}で記述したパスワード変更を用いる\code{make\-/account}により作成された
銀行口座オブジェクトについて考える。私達の銀行システムが連結口座の開設能力を
必要とすると仮定しよう。これを達成する手続\code{make\-/joint}を定義せよ。
\code{make\-/joint}は3つの引数を取らねばならない。第一はパスワードで守られた口座である。
第二引数はパスワードで\code{make\-/joint}命令が成功するためには口座が開設された時点の
パスワードに合致しなければならない。第三引数は新しいパスワードである。
\code{make\-/joint}は元の口座に対して新しいパスワードを用いる追加のアクセスを作成する。
例えば\code{peter\-/acc}がパスワード\code{open\-/sesame}を用いる銀行口座であれば、

\begin{scheme}
(define paul-acc
  (make-joint peter-acc 'open-sesame 'rosebud))
\end{scheme}

\noindent
上記は\code{peter\-/acc}に対し名前\code{paul\-/acc}とパスワード\code{rosebud}を用いて取引することを
可能にする。この新しい機能に対応するためあなたの\link{Exercise 3.3}への解答を
変更したいと思うだろう。
\end{quote}

\begin{quote}
\heading{\phantomsection\label{Exercise 3.8}Exercise 3.8:} 
\link{Section 1.1.3}にて評価モデルを定義した時、式の評価の最初のステップはその部分式を評価することだと
述べました。しかし部分式を評価する順については指定しませんでした。(例えば左から右や右から左です)。
代入を導入する時、手続に対する引数が評価される順は結果に違いを起こせます。以下の式を評価した時に、

\begin{scheme}
(+ (f 0) (f 1))
\end{scheme}



\code{+}の引数が左から右へ評価された場合に0を返し、右から左へ評価された場合に1を返すようにする
簡単な手続\code{f}を定義せよ。
\end{quote}

\section{評価の環境モデル}
\label{Section 3.2}


複合手続を\link{Chapter 1}で紹介した時、手続を引数に適用することが何を意味するか
定義するため評価の置換モデルを使用しました(\link{Section 1.1.5})。

\begin{itemize}

\item
複合手続を引数に適用するため、手続のボディを各形式パラメータを相対する引数で置き換えて評価する。

\end{itemize}

\noindent
一旦代入を私達のプログラミング言語で認めれば、そのような定義は最早適切でありません。
具体的には\link{Section 3.1.3}で議論しましたが、代入の出現により、変数は最早単に値に対する
名前であると考えることができません。そうでなく、変数はどうにかして値が格納できる``場所''を
指定することになります。私達の新しい評価モデルではこれらの場所は\newterm{environments}(\jnewterm{環境})と
呼ばれる構造に保存されます。



環境は\newterm{frames}(\jnewterm{フレーム})の列です。各フレームは\newterm{bindings}(\jnewterm{束縛})の
(空の可能性のある)テーブルで、変数名とそれらが相対する値とを結び付けます。
(単一のフレームは任意の変数に対してたかだか1つの束縛を保持します)。
各フレームはまた議論の目的のためフレームが\newterm{global}(\jnewterm{グローバル}、大域的)だと認識されない限り、
\newterm{enclosing environment}(\jnewterm{外部環境})へのポインタを持ちます。
環境に対して\newterm{value of a variable}(\jnewterm{変数の値})はその変数に対する束縛を持つ
環境内の最初のフレーム内の変数の束縛により与えられる値です。もし列内の全ての
フレームがその変数に対する束縛を指定しない場合、その変数はその環境に\newterm{unbound}(\jnewterm{束縛されない})
と呼びます。



\link{Figure 3.1}はI, II, IIIとラベリングした3つのフレームから成る簡単な環境構造を示しています。
図の中でA, B, C, Dは環境へのポインタです。CとDは同じ環境を差しています。変数\code{z}と\code{x}は
フレームIIに束縛され、一方\code{y}と\code{x}はフレームIに束縛されます。
環境Dの\code{x}の値は3です。環境Bに対する\code{x}の値もまた3です。これは次のように決定されます。
列の最初のフレーム(フレーム III)を調べますが\code{x}に対する束縛を見つけられません。
そのため外部環境Dで続けてフレーム Iの中に束縛を見つけます。一方で環境Aでの\code{x}の値は7です。
列の最初のフレーム(フレーム II)が\code{x}から7への束縛を含んでいるからです。
環境Aに対して、フレーム II内の\code{x}から7への束縛はフレーム Iの\code{x}から3への束縛を
\newterm{shadow}(\jnewterm{隠蔽する})と言われます。

\begin{figure}[tb]
\phantomsection\label{Figure 3.1}
\centering
\begin{comment}
\heading{Figure 3.1:} A simple environment structure.

\begin{example}
           +--------+
           |      I |
           | x: 3   |
           | y: 5   |
           +--------+
              ^  ^
              |  |
            C |  | D
+---------+   |  |   +----------+
|      II |   |  |   |      III |
| z: 6    +---+  +---+ m: 1     |
| x: 7    |          | y: 2     |
+---------+          +----------+
\end{example}
\end{comment}
\includegraphics[width=48mm]{fig/chap3/Fig3.1.pdf}
\par\bigskip
\noindent
\heading{Figure 3.1:} 単純な環境構造
\end{figure}

\noindent
環境は評価プロセスに対し不可欠な存在です。式が評価されるべきコンテキスト(文脈)を決定するためです。
実際にプログラミング言語の式、それ自身は意味を持たないと言えるでしょう。
そうでなく、式はそれが評価されるある環境に対してのみ意味を獲得します。
\code{(+ 1 1)}のような簡単な式の逐次実行でさえ、\code{+}が加算のためのシンボルであるというコンテキスト
のなかで操作しているという合意に依存しています。従って私達の評価モデルにおいて私達は常に
ある環境に対して式を評価すると述べます。インタプリタとの相互作用を説明するために、単一のフレームから
成り立ち(外部環境を持たず)、プリミティブな手続に関連するシステムの値を持つグローバル環境が
存在すると仮定します。例えば\code{+}が加算に対するシンボルであるという考えは、シンボル\code{+}が
グローバル環境においてプリミティブな加算手続に対し束縛されていると言うことで捉えられます。



\subsection{評価のルール}
\label{Section 3.2.1}



インタプリタが組み合わせをどのように評価するかの全体的な仕様は最初に\link{Section 1.1.3}にて紹介した時と
同じに残っています。

\begin{itemize}

\item
組み合わせを評価するために

\end{itemize}

\begin{enumerate}

\item
組み合わせの部分式を評価する。\footnote{
代入は評価ルールのステップ1に微妙さを取り込みます。\link{Exercise 3.8}に示されるように
代入の存在は組み合わせの部分式がどの順で評価されるかに依存して異なる値を生じます。
従って正確に述べればステップ1における評価順を指定せねばなりません。(例えば左から右や
右から左等)。しかしこの順は常に実装上の詳細と考えられねばなりません。
例えば洗練されたコンパイラはどの部分式が評価されるかの順を最適化のために変えるかもしれません。}

\item
オペレータ部分式の値をオペランド部分式の値に適用する。

\end{enumerate}

\noindent
評価の環境モデルは置換モデルを複合手続を引数に適用することの意味を指定することで
置き換えます。


評価の環境モデルでは手続は常にあるコードと環境へのポインタのペアから成り立ちます。
手続はただ1つの方法で作成されます。それは\( \lambda \)式を評価することです。
これによりコードが\( \lambda \)式のテキストから得られる手続が生成され、その環境は
\( \lambda \)式が手続を生成するために評価された環境になります。例えば以下の手続
定義について考えてみましょう。

\begin{scheme}
(define (square x)
  (* x x))
\end{scheme}

\noindent
この式はグローバル環境で評価されました。この手続定義の文法は根底にある暗黙的な
\( \lambda \)式のための糖衣構文です。これは次を行った場合と等価です。

\begin{scheme}
(define square
  (lambda (x) (* x x)))
\end{scheme}

\noindent
これは\code{(lambda (x) (* x x))}を評価し、全てグローバル環境において\code{square}をその結果に束縛します。



\link{Figure 3.2}はこの\code{define}式の評価結果を示します。手続オブジェクトは手続が1つの
形式パラメータ\code{x}を持ち手続のボディが\code{(* x x)}ことをコードが指定するペアです。
手続の環境部分はグローバル環境へのポインタです。それが\( \lambda \)式が手続を生じるため
評価される環境なためです。シンボル\code{square}と手続オブジェクトを関連付ける新しい束縛は
グローバルなフレームに追加されます。一般的に\code{define}はフレームに束縛を追加することで
定義を作成します。

\begin{figure}[tb]
\phantomsection\label{Figure 3.2}
\centering
\begin{comment}
\heading{Figure 3.2:} Environment structure produced by
evaluating \code{(define (square x) (* x x))} in the global environment.

\begin{example}
           +----------------------+
           | other variables      |
global --->|                      |
env        | square: --+          |
           +-----------|----------+
                       |       ^
(define (square x)     |       |
  (* x x))             V       |
                   .---.---.   |
                   | O | O-+---+
                   `-|-^---'
                     |
                     V
                   parameters: x
                   body: (* x x)
\end{example}
\end{comment}
\includegraphics[width=49mm]{fig/chap3/Fig3.2b.pdf}
\begin{quote}
\heading{Figure 3.2:} 大域環境内にて\code{(define (square x) (* x x))}を評価することにより生成された環境構造
\end{quote}
\end{figure}

\noindent
これで手続がどのように作成されるのか学んだので手続がどのように適用されるのかを
説明することができます。環境モデルは以下のことを指定します。手続を引数に適用するために、
パラメタを引数の値に束縛するフレームを含む新しい環境を作成します。このフレームの
外部環境は手続により指定された環境です。さて、この新しい環境で手続のボディを評価します。



このルールがどのように従われるかについて示すため、\link{Figure 3.3}は式\code{(square 5)}を
グローバル環境にて\code{square}が\link{Figure 3.2}にて生成された手続ある場合に評価することで
作成された環境構造を図示しています。この手続の適用は図でE1と示される新しい環境の作成に帰着し、
手続の形式パラメタ\code{x}が引数5に束縛されているフレームで始まっています。
このフレームから情報へ向かうポインタはフレームの外部環境がグローバル環境であることを示します。
\code{square}手続オブジェクトの一部として示される環境であるためここでグローバル環境が選択されます。
E1の中では手続のボディ\code{(* x x)}を評価します。E1中の\code{x}の値は5であるため結果は\code{(* 5 5)}、
つまり25です。

\begin{figure}[tb]
\phantomsection\label{Figure 3.3}
\centering
\begin{comment}
\heading{Figure 3.3:} Environment created by evaluating
\code{(square 5)} in the global environment.

\begin{example}
          +------------------------------------+
          | other variables                    |
global -->|                                    |
env       | square: --+                        |
          +-----------|---------------------+--+
                      |       ^             ^
(square 5)            |       |             |
                      V       |             |
                  .---.---.   |         +---+--+
                  | O | O-+---+   E1 -->| x: 5 |
                  `-|-^---'             +------+
                    |
                    V
                  parameters: x
                  body: (* x x)
\end{example}
\end{comment}
\includegraphics[width=78mm]{fig/chap3/Fig3.3b.pdf}
\begin{quote}
\heading{Figure 3.3:} 大域環境内にて\code{(square 5)}を評価することにより作られた環境
\end{quote}
\end{figure}

\noindent
手続適用の環境モデルは2つのルールでまとめられます。

\begin{itemize}

\item
手続オブジェクトはフレームを構築、手続の形式パラメタを呼出の引数へ束縛し、
新しく構築された環境のコンテキストにて手続のボディを評価することで引数の集合に
手続を適用することができる。

\item
手続は与えられた環境に関連する\( \lambda \)式を評価することで作成される。
結果としての手続オブジェクトは\( \lambda \)式のテキストと手続が作成された環境への
ポインタから成るペアである。

\end{itemize}

\noindent
\code{define}を用いてのシンボルの定義は現在の環境フレームに束縛を作成し、そのシンボルに
指示された値を束縛することもまた指摘します。\footnote{
もし既にその変数への束縛が現在のフレームに存在する場合、束縛は変更されます。
これはシンボルの再定義を可能にするため便利です。しかし\code{define}が値の変更に使用できる
こと、そしてこれが明示的に\code{set!}を使用せずとも代入の問題を持ち出すことを意味します。
このため既存のシンボルの再定義に対しエラーや警告を発することを好む人達もいます。}
最後に、\code{set!}の振舞を指定します。私達にそもそも環境モデルの導入を強いた命令です。
ある環境で式
\code{(set!}\( \;\langle \)\var{variable}\( \kern0.08em\rangle \)\( \;\langle \)\var{value}\( \kern0.08em\rangle \)\code{)}
を評価することはその環境に束縛を位置付け、その束縛を新しい値を示すよう変更します。
つまり\code{set!}は環境でその変数の束縛を持つ最初のフレームを探しそのフレームを変更します。
もし変数がその環境では束縛されていないのであれば\code{set!}はエラーを発します。



これらの評価ルールは置換モデルより大幅により複雑ですが、依然として適度に容易です。
さらに環境モデルは抽象的ですがインタプリタが式をどのように評価するかの正し説明を
与えます。\link{Chapter 4}ではこのモデルがどのようにうまく働くインタプリタの実装のための
設計図としての役を果たすのかについて学ぶことになります。
残りの節ではいくつかの実例となるプログラムを分析することによりこのモデルについての詳細を
述べます。


\subsection{単純な手続の適用}
\label{Section 3.2.2}



\link{Section 1.1.5}にて置換モデルを紹介した時、以下の手続定義を与えられた場合に
合成\code{(f 5)}がどのように136として評価されるかについて説明しました。

\begin{scheme}
(define (square x)
  (* x x))
(define (sum-of-squares x y)
  (+ (square x) (square y)))
(define (f a)
  (sum-of-squares (+ a 1) (* a 2)))
\end{scheme}

\noindent
同じ式を環境モデルを用いて分析できます。\link{Figure 3.4}は3つの手続オブジェクトが
\code{f}, \code{square}, and \code{sum\-/of\-/squares}の定義を評価することでグローバル環境に
作成されたことを示します。各手続オブジェクトはいくつかのコードとグローバル環境への
ポインタから成り立ちます。

\begin{figure}[tb]
\phantomsection\label{Figure 3.4}
\centering
\begin{comment}
\heading{Figure 3.4:} Procedure objects in the global frame.

\begin{example}
          +--------------------------------------------+
          | sum-of-squares:                            |
global -->| square:                                    |
env       | f: --+                                     |
          +------|--------------+--------------+-------+
                 |     ^        |     ^        |     ^
                 |     |        |     |        |     |
                 V     |        V     |        V     |
             .---.---. |    .---.---. |    .---.---. |
             | O | O-+-+    | O | O-+-+    | O | O-+-+
             `-|-^---'      `-|-^---'      `-|-^---'
               |              |              |
               V              V              V
   parameters: a          parameters: x  parameters: x, y
   body: (sum-of-squares  body: (* x x)  body: (+ (square x)
           (+ a 1)                                (square y))
           (* a 2))
\end{example}
\end{comment}
\includegraphics[width=106mm]{fig/chap3/Fig3.4a.pdf}
\par\bigskip
\noindent
\heading{Figure 3.4:} グローバルフレーム内の手続オブジェクト
\end{figure}

\noindent
\link{Figure 3.5}は式\code{(f 5)}を評価することで作成された環境構造です。
\code{f}の呼出により\code{f}の形式パラメタ\code{a}が引数5に束縛されるフレームで始まる新しい環境E1が作成されます。
E1の中で\code{f}のボディを評価します。

\begin{scheme}
(sum-of-squares (+ a 1) (* a 2))
\end{scheme}

\noindent
この合成式を評価するために最初に部分式を評価します。最初の部分式\code{sum\-/of\-/squares}は
手続オブジェクトである値を持っています。(この値がどのように見つけられるかに注意して下さい。
最初にE1の第一フレームを調べますが\code{sum\-/of\-/squares}の束縛はありません。次に外部環境に
進みます。つまりグローバル環境です。そこで\link{Figure 3.4}に示すように束縛を見つけます)。
他の2つの部分式はプリミティブな命令\code{+}と\code{*}を、2つの合成式\code{(+ a 1)}と\code{(* a 2)}を
評価しそれぞれ6と10を得るために適用することで評価されます。



これで手続オブジェクト\code{sum\-/of\-/squares}を引数6と10に適用します。結果は形式パラメタ
\code{x}と\code{y}が引数に束縛される新しい環境E2へ帰着します。E2内では合成\code{(+ (square x) (square y))}
を評価します。これが\code{(square x)}の評価へと移り、\code{square}はグローバルフレームで見つかり、\code{x}は6です。
もう一度、新しい環境E3を立ち上げ、\code{x}は6に束縛されE3の中で\code{square}のボディ\code{(* x x)}が評価されます。
また\code{sum\-/of\-/squares}の適用の一部として部分式\code{(square y)}も評価さねばならずそこでは\code{y}は10です。
この2つ目の\code{square}の呼出がまた別の環境E4を作成し、そこでは\code{square}の形式パラメタ\code{x}は10に束縛されます。
そしてE4の中では\code{(* x x)}を評価せねばなりません。

\begin{figure}[tb]
\phantomsection\label{Figure 3.5}
\centering
\begin{comment}
\heading{Figure 3.5:} Environments created by evaluating
\code{(f 5)} using the procedures in \link{Figure 3.4}.

\begin{example}
          +-----------------------------------------------------+
global -->|                                                     |
env       +-----------------------------------------------------+
            ^              ^                ^               ^
(f 5)       |              |                |               |
        +------+       +-------+        +------+        +-------+
  E1 -->| a: 5 |  E2 ->| x: 6  |  E3 -->| x: 6 |  E4 -->| x: 10 |
        |      |       | y: 10 |        |      |        |       |
        +------+       +-------+        +------+        +-------+
   (sum-of-squares   (+ (square x)       (* x x)         (* x x)
     (+ a 1)            (square u))
     (+ a 2))
\end{example}
\end{comment}
\includegraphics[width=100mm]{fig/chap3/Fig3.5a.pdf}
\begin{quote}
\heading{Figure 3.5:} \link{Figure 3.4}内の手続を用いて\code{(f 5)}を評価することで作られた環境
\end{quote}
\end{figure}

\noindent
確認すべき重要な点は\code{square}の各呼出が\code{x}の束縛を持つ新しい環境を構築することです。
ここで私達は異なるフレームがどのようにして全て\code{x}と名付けられた異なるローカル変数の
独立を保つのかについて見ることができるます。\code{square}により作られた各フレームがグローバル
環境を差していることに注意して下さい。これは\code{square}手続オブジェクトが指す環境であるためです。



部分式が評価された後に結果が返されます。2つの\code{square}の呼出により作成された値は
\code{sum\-/of\-/squares}により加算され、この結果が\code{f}により返されます。ここでの私達の
焦点は環境構造上にありますのでこれらの返された値が呼出から呼出へどのように渡されるか
については長々と説明は致しません。しかし、これはまた評価処理の重要な側面であり、
\link{Chapter 5}にてこれの詳細に戻ります。

\begin{quote}
\heading{\phantomsection\label{Exercise 3.9}Exercise 3.9:} 
\link{Section 1.2.1}にて指数演算のための2つの手続を解析するために置換モデルを使用した。
以下が再帰版であり、

\begin{scheme}
(define (factorial n)
  (if (= n 1) 1 (* n (factorial (- n 1)))))
\end{scheme}

\noindent
以下は反復版である。

\begin{scheme}
(define (factorial n) (fact-iter 1 1 n))
(define (fact-iter product counter max-count)
  (if (> counter max-count)
      product
      (fact-iter (* counter product)
                 (+ counter 1)
                 max-count)))
\end{scheme}




各版の\code{factorial}手続を用いて\code{(factorial 6)}を評価した場合に作成される環境構造を示せ。\footnote{
環境モデルはインタプリタは\code{fact\-/iter}のような手続を末尾再帰を用いることで一定量の記憶域にて実行できる
という\link{Section 1.2.1}での私達の主張を明確にはしません。末尾再帰については\link{Section 5.4}にてインタプリタの
コントロール構造を取り扱う時に議論します。}
\end{quote}


\subsection{局所状態のレポジトリとしてのフレーム}
\label{Section 3.2.3}


手続と代入がどのようにしてローカルな状態を持つオブジェクトを表現するために利用できるかを知るために
環境モデルに助けを求めることができます。

\begin{scheme}
(define (make-withdraw balance)
  (lambda (amount)
    (if (>= balance amount)
        (begin (set! balance (- balance amount))
               balance)
        "Insufficient funds")))
\end{scheme}

\noindent
次の定義の評価について説明してみましょう。

\begin{scheme}
(define W1 (make-withdraw 100))
\end{scheme}

\noindent
以下のように用いたとします。

\begin{scheme}
(W1 50)
~\textit{50}~
\end{scheme}

\noindent
\link{Figure 3.6}はグローバル環境における\code{make\-/withdraw}手続の定義の結果を示します。
グローバル環境へのポインタを持つ手続オブジェクトを作成します。今の所、これは今までに
見た例から異なる点はありません。ただし手続のボディそれ自身が\( \lambda \)式であることが
異なります。

\begin{figure}[tb]
\phantomsection\label{Figure 3.6}
\centering
\begin{comment}
\begin{quote}
\heading{Figure 3.6:} Result of defining \code{make\-/withdraw} in the global environment.

\begin{example}
          +---------------------------+
global -->| make-withdraw: --+        |
env       +------------------|--------+
                             |      ^
                             V      |
                         .---.---.  |
                         | O | O-+--+
                         `-|-^---'
                           |
                           V
         parameters: balance
         body: (lambda (amount)
                 (if (>= balance amount)
                     (begin (set! balance
                                  (- balance amount))
                            balance)
                     "Insufficient funds"))
\end{example}
\end{quote}
\end{comment}
\includegraphics[width=91mm]{fig/chap3/Fig3.6b.pdf}
\begin{quote}
\heading{Figure 3.6:} グローバル環境にて\code{make\-/withdraw}を定義した結果
\end{quote}
\end{figure}

\noindent
演算の面白い部分は手続\code{make\-/withdraw}を引数に適用した時に起こります。

\begin{scheme}
(define W1 (make-withdraw 100))
\end{scheme}

\noindent
通常通りに形式パラメタ\code{balance}が引数100に束縛される環境E1を設定することから始まります。
この環境の中で\code{make\-/withdraw}のボディ、即ち\( \lambda \)式を評価します。
これがコードは\code{lambda}で指定され、環境がE1である新しい手続オブジェクトが構築されます。
そのE1の中で\code{lambda}が手続を生成するため評価されています。
結果の手続オブジェクトは\code{make\-/withdraw}を呼び出して返された値です。
これはグローバル環境にて\code{W1}に束縛されます。\code{define}自身がグローバル環境にて評価されたためです。
\link{Figure 3.7}は結果の環境構造を示します。

これで\code{W1}が引数に適用された時に何が起こるかを解析できます。

\begin{scheme}
(W1 50)
~\textit{50}~
\end{scheme}

\begin{figure}[tb]
\phantomsection\label{Figure 3.7}
\centering
\begin{comment}
\heading{Figure 3.7:} Result of evaluating \code{(define W1 (make\-/withdraw 100))}.

\begin{example}
          +-----------------------------------------------+
          | make-withdraw: -----------------------+       |
global -->|                                       |       |
          | W1: --+                               |       |
          +-------|-------------------------------|-------+
                  |                ^              |     ^
                  |                |              V     |
                  |        +-------+------+   .---.---. |
                  |  E1 -->| balance: 100 |   | O | O-+-+
                  |        +--------------+   `-|-^---'
                  V                ^            |
              .---.---.            |            V
            +-+-O | O-+------------+    parameters: balance
            | `---^---'                 body: ...
            V
    parameters: amount
    body: (if (>= balance amount)
              (begin (set! balance (- balance amount))
                     balance)
              "Insufficient funds")
\end{example}
\end{comment}
\includegraphics[width=100mm]{fig/chap3/Fig3.7a.pdf}
\par\bigskip
\noindent
\heading{Figure 3.7:} \code{(define W1 (make\-/withdraw 100))}を評価した結果
\end{figure}

\noindent
\code{W1}の形式パラメタ\code{amount}が引数50に束縛されるフレームを構築することから始めます。
観察すべき重大な点はこのフレームがその外部環境としてグローバル環境ではなく環境E1を持って
いる点です。これが\code{W1}手続オブジェクトにより指示される環境だからです。
この新しい環境の中で手続のボディを評価します。

\begin{scheme}
(if (>= balance amount)
    (begin (set! balance (- balance amount))
           balance)
    "Insufficient funds")
\end{scheme}

\noindent
結果の環境構造は\link{Figure 3.8}に示されます。評価された式は\code{amount}と\code{balance}の両方を
参照します。\code{amount}は環境の最初のフレームに見つかりますが、\code{balance}は外部環境ポインタに
従ってE1にて見つかります。

\begin{figure}[tb]
\phantomsection\label{Figure 3.8}
\centering
\begin{comment}
\heading{Figure 3.8:} Environments created by applying the procedure object \code{W1}.

\begin{example}
          +---------------------------------------------------+
          | make-withdraw: ...                                |
global -->|                                                   |
env       | W1: --+                                           |
          +-------|-------------------------------------------+
                  |               ^
                  |               |
                  |       +-------+------+ Here is the balance
                  | E1 -->| balance: 100 | that will be changed
                  |       +--------------+ by the set!.
                  V               ^   ^
              .---.---.           |   +----+
              | O | O-+-----------+        |
              `-|-^---'             +------+-----+
                |                   | amount: 50 |
                V                   +------------+
      parameters: amount   (if (>= balance amount)
      body: ...                (begin (set! balance
                                            (- balance amount))
                                      balance)
                               "Insufficient funds")
\end{example}
\end{comment}
\includegraphics[width=99mm]{fig/chap3/Fig3.8c.pdf}
\par\bigskip
\noindent
\heading{Figure 3.8:} 手続オブジェクト\code{W1}を適用したことにより作成された環境
\end{figure}

\noindent
\code{set!}が実行された時、E1中の\code{balance}の束縛は変更されます。\code{W1}の呼出が終了する時
\code{balance}は50で、\code{balance}を含むフレームは依然手続オブジェクト\code{W1}から指されています。
\code{amount}を束縛する(その中で\code{balance}を変更するコードを実行した)フレームは最早関係が存在しません。
それを構築した手続呼出は停止したためです。そしてその環境の他の部分からそのフレームを指す
ポインタは存在しません。次回\code{W1}が呼ばれた時、\code{amount}を束縛する新しいフレームが構築され
その外部環境はE1になります。私達はE1が手続オブジェクト\code{W1}のためのローカル状態を持つ``場所''の
役割を果たすのを見ました。\link{Figure 3.9}は\code{W1}を呼び出した後の状況を示します。



二つ目の``withdraw''オブジェクトを別の\code{make\-/withdraw}呼出を行うことで作成した時に何が起こるかについて
観察して下さい。

\begin{scheme}
(define W2 (make-withdraw 100))
\end{scheme}

\begin{figure}[tb]
\phantomsection\label{Figure 3.9}
\centering
\begin{comment}
\heading{Figure 3.9:} Environments after the call to \code{W1}.

\begin{example}
           +------------------------------------+
           | make-withdraw: ...                 |
global --->|                                    |
env        | W1: --+                            |
           +-------|----------------------------+
                   |                   ^
                   |                   |
                   |            +------+------+
                   |     E1 --->| balance: 50 |
                   |            +-------------+
                   V                   ^
               .---.---.               |
               | O | O-+---------------+
               `-|-^---'
                 |
                 V
          parameters: amount
          body: ...
\end{example}
\end{comment}
\includegraphics[width=96mm]{fig/chap3/Fig3.9a.pdf}
\par\bigskip
\noindent
\heading{Figure 3.9:} \code{W1}呼出後の環境
\end{figure}

\begin{figure}[tb]
\phantomsection\label{Figure 3.10}
\centering
\begin{comment}
\heading{Figure 3.10:} Using \code{(define W2 (make\-/withdraw 100))} to create a second object.

\begin{example}
         +-------------------------------------------------+
         | make-withdraw: ...                              |
global ->| W2: ---------------------------+                |
env      | W1: --+                        |                |
         +-------|------------------------|----------------+
                 |              ^         |              ^
                 |              |         |              |
                 |       +------+------+  |       +------+-------+
                 |  E1 ->| balance: 50 |  |  E2 ->| balance: 100 |
                 |       +-------------+  |       +--------------+
                 V              ^         V              ^
             .---.---.          |     .---.---.          |
             | O | O-+----------+     | O | O-+----------+
             `-|-^---'                `-|-^---'
               | +----------------------+
               V V
        parameters: amount
        body: ...
\end{example}
\end{comment}
\includegraphics[width=108mm]{fig/chap3/Fig3.10a.pdf}
\begin{quote}
\heading{Figure 3.10:} \code{(define W2 (make\-/withdraw 100))}を用いて2つ目のオブジェクトを作成
\end{quote}
\end{figure}

\noindent
これにより\link{Figure 3.10}の環境構造が生成され\code{W2}が手続オブジェクトであり、ある程度のコードと
環境によるペアであることを示しています。\code{W2}のための環境E2は\code{make\-/withdraw}の呼出により作成されます。
それ専用の\code{balance}のためのローカルな状態を持つフレームを含みます。一方で\code{W1}と\code{W2}は同じ
コードを持ちます。\code{make\-/withdraw}のボディ内の\( \lambda \)式によりコードは指定されています。\footnote{
\code{W1}と\code{W2}が計算機内の同じ物理コードを共有しているかどうか、またはそれぞれがコードのコピーを持っているのか
は実装上の詳細です。\link{Chapter 4}で実装するインタプリタではコードは実際に共有されます。}
なぜ\code{W1}と\code{W2}が独立したオブジェクトとして振る舞うのかをここで見ました。
\code{W1}の呼出はE1に格納された状態変数\code{balance}を参照し、一方\code{W2}の呼出はE2に格納された\code{balance}を
参照します。従って一方のオブジェクトのローカル状態への変更は他方のオブジェクトに影響を与えません。

\begin{quote}
\heading{\phantomsection\label{Exercise 3.10}Exercise 3.10:} 
\code{make\-/withdraw}手続ではローカル変数\code{balance}は\code{make\-/withdraw}のパラメタと
して作成される。ローカル状態変数を明示的に\code{let}を使って以下の様に作成することもできる。

\begin{scheme}
(define (make-withdraw initial-amount)
  (let ((balance initial-amount))
    (lambda (amount)
      (if (>= balance amount)
          (begin (set! balance (- balance amount))
                 balance)
          "Insufficient funds"))))
\end{scheme}



\link{Section 1.3.2}で\code{let}は手続呼出のための単純な糖衣構文であったことを思い出そう。

\begin{scheme}
(let ((~\( \dark \langle \)~~\var{\dark var}~~\( \dark \rangle \)~ ~\( \dark \langle \)~~\var{\dark exp}~~\( \dark \rangle \)~)) ~\( \dark \langle \)~~\var{\dark body}~~\( \dark \rangle \)~)
\end{scheme}

\noindent
上記は代替的な文法として以下に翻訳される。

\begin{scheme}
((lambda (~\( \dark \langle \)~~\var{\dark var}~~\( \dark \rangle \)~) ~\( \dark \langle \)~~\var{\dark body}~~\( \dark \rangle \)~) ~\( \dark \langle \)~~\var{\dark exp}~~\( \dark \rangle \)~)
\end{scheme}


環境モデルを用いてこの\code{make\-/withdraw}の代替版を解析し、先に記述したような図を描き
相互作用を説明せよ。

\begin{scheme}
(define W1 (make-withdraw 100))
(W1 50)
(define W2 (make-withdraw 100))
\end{scheme}



\code{make\-/withdraw}の2つの版が同じ振舞を持つオブジェクトを作成することを示せ。
環境構造は2つの版でどのように違うか?
\end{quote}


\subsection{内部定義}
\label{Section 3.2.4}


\link{Section 1.1.8}では手続が内部定義を持つこことができ、結果としてブロック構造へと導くことを
説明しました。以下の平方根を求める手続がその例です。

\begin{scheme}
(define (sqrt x)
  (define (good-enough? guess)
    (< (abs (- (square guess) x)) 0.001))
  (define (improve guess)
    (average guess (/ x guess)))
  (define (sqrt-iter guess)
    (if (good-enough? guess)
        guess
        (sqrt-iter (improve guess))))
  (sqrt-iter 1.0))
\end{scheme}

\noindent
さて環境モデルを用いてなぜこれらの内部定義が希望通りに振る舞うのかを調べることができます。
\link{Figure 3.11}は内部手続\code{good\-/enough?}が\code{guess}が1に等しい場合に
最初に呼ばれた状態で式\code{(sqrt 2)}を評価した時点を示しています。



環境構造を観察して下さい。\code{sqrt}はグローバル環境におけるシンボルであり
手続オブジェクトに束縛され、その関連する環境はグローバル環境です。
\code{sqrt}が呼ばれた時、新しい環境E1が形成されグローバル環境の下位に置かれ、
その中ではパラメタ\code{x}が2に束縛されます。\code{sqrt}のボディが次にE1の中で
評価されます。\code{sqrt}のボディの最初の式は以下であり、

\begin{scheme}
(define (good-enough? guess)
  (< (abs (- (square guess) x)) 0.001))
\end{scheme}

\noindent
この式を評価すると手続\code{good\-/enough?}が環境E1の中に定義されます。具体的には、
シンボル\code{good\-/enough?}がE1の最初のフレームに追加され環境E1を指す手続オブジェクトに束縛
されます。同様に\code{improve}と\code{sqrt\-/iter}がE1の中に手続として定義されます。
簡潔さのために、\link{Figure 3.11}は\code{good\-/enough?}に対する手続オブジェクトのみを
示しています。

\begin{figure}[tb]
\phantomsection\label{Figure 3.11}
\centering
\begin{comment}
\heading{Figure 3.11:} \code{Sqrt} procedure with internal definitions.

\begin{example}
          +--------------------------------------------------+
global -->| sqrt: --+                                        |
env       |         |                                        |
          +---------|----------------------------------------+
                    V       ^                   ^
                .---.---.   |                   |
     +----------+-O | O-+---+        +----------+------------+
     |          `---^---'            | x: 2                  |
     V                         E1 -->| good-enough?: -+      |
parameters: x                        | improve: ...   |      |
body: (define good-enough? ...)      | sqrt-iter: ... |      |
      (define improve ...)           +----------------|------+
      (define sqrt-iter ...)          ^  ^            |     ^
      (sqrt-iter 1.0)                 |  |            V     |
                            +---------++ |        .---.---. |
                      E2 -->| guess: 1 | |        | O | O-+-+
                            +----------+ |        `-|-^---'
                      call to sqrt-iter  |          |
                                         |          V
                               +---------++    parameters: guess
                         E3 -->| guess: 1 |    body: (< (abs ...)
                               +----------+             ...)
                         call to good-enough?
\end{example}
\end{comment}
\includegraphics[width=107mm]{fig/chap3/Fig3.11a.pdf}
\begin{quote}
\heading{Figure 3.11:} 内部定義を持つ\code{sqrt}手続
\end{quote}
\end{figure}


\noindent
ローカル手続が定義された後に、式\code{(sqrt\-/iter 1.0)}がまた環境E1の中で評価されます。
そのためE1の中で\code{sqrt\-/iter}に束縛された手続オブジェクトが引数1にて呼ばれます。
これが環境E2を作成し\code{sqrt\-/iter}のパラメタである\code{guess}が1に束縛される。
\code{sqrt\-/iter}は次に\code{good\-/enough?}を(E2の)\code{guess}の値を引数として呼びます。
これが別の環境E3を構築し(\code{good\-/enough?}の引数である)\code{guess}が1に束縛されます。
\code{sqrt\-/iter}と\code{good\-/enough?}の両方が\code{guess}という名前のパラメタを持ちますが、
2つの区別可能なローカル変数が異なるフレームの中に存在します。またE2とE3の両方が
E1を外部環境として持ちます。手続\code{sqrt\-/iter}と\code{good\-/enough?}の両方がE1をそれらの
環境部分として持つためです。これの結果の1つとして\code{good\-/enough?}のボディ内のシンボル\code{x}は
E1内に存在する\code{x}の束縛を参照します。即ち元の\code{sqrt}手続が呼ばれた時の\code{x}の値です。


環境モデルは従ってローカル手続定義をプログラムのモジュール化するための便利なテクニックとする
2つの鍵となる性質を説明します。

\begin{itemize}

\item
ローカル手続の名前は(直の)外部手続の外側の名前と衝突しない。ローカル手続の名前は
手続が実行される時に作成したフレーム内にて束縛されるのであり、グローバル環境内で
束縛される訳でありません。

\item
ローカルな手続はそれを内包する外部手続の引数にアクセスすることができます。
単純にパラメタの名前を自由変数として用いるだけです。これはローカル手続のボディは
外部手続のための評価環境の下位に置かれる環境内で評価されるためです。

\end{itemize}

\begin{quote}
\heading{\phantomsection\label{Exercise 3.11}Exercise 3.11:} 
\link{Section 3.2.3}では環境モデルがどのようにローカルな状態を持つ手続の振舞を説明するかについて
学んだ。ここまでで内部定義がどのように働くかについて理解した。典型的なメッセージパッシング
手続はこれらの側面の両方を持っている。\link{Section 3.1.1}の銀行口座について考えよう。

\begin{scheme}
(define (make-account balance)
  (define (withdraw amount)
    (if (>= balance amount)
        (begin (set! balance (- balance amount))
               balance)
        "Insufficient funds"))
  (define (deposit amount)
    (set! balance (+ balance amount))
    balance)
  (define (dispatch m)
    (cond ((eq? m 'withdraw) withdraw)
          ((eq? m 'deposit) deposit)
          (else 
           (error "Unknown request: 
                   MAKE-ACCOUNT"
                  m))))
  dispatch)
\end{scheme}



以下の応答により生成される環境構造を示せ。

\begin{scheme}
(define acc (make-account 50))
((acc 'deposit) 40)
~\textit{90}~
((acc 'withdraw) 60)
~\textit{30}~
\end{scheme}



\code{acc}の局所状態はどこにあるか? 別の口座を定義したとする。

\begin{scheme}
(define acc2 (make-account 100))
\end{scheme}


2つの口座の局所状態はどのように区別されるか? 環境構造のどの部分が\code{acc}と\code{acc2}にて
共有されるか?
\end{quote}

\section{ミュータブルデータによるモデリング}
\label{Section 3.3}



\link{Chapter 2}では複合データを計算オブジェクトを構築する手段として扱いました。これは
複数の側面を持つ実際の世界のオブジェクトをモデル化するためにいくつかの
部品を持ちます。また\link{Chapter 2}では
データオブジェクトを作成するコンストラクタと、
複合データオブジェクトの部品にアクセスするセレクタを用いてどのデータ構造が指定されるか
に準ずるデータ抽象化の規律についても紹介しました。
しかし今では\link{Chapter 2}が解決しなかったデータの別の側面があることを私達は知りました。
状態が変化するオブジェクトにより成るシステムをモデル化したいという欲求複合データオブジェクトを
構築することやそれらから選択することと同様に変更することの必要性へと導きます。
変換する状態を持つ複合オブジェクトをモデル化するために、セレクタやコンストラクタ
に追加して、データオブジェクトを変更する\newterm{mutators}(\jnewterm{ミューテータ}、変化させる物)と呼ばれる
命令を含むようにデータ抽象化を設計することにします。
例えば、銀行システムのモデル化は口座の差引残高を変更する必要があります。従って銀行口座を表現する
データ構造は以下の命令を許可するでしょう。

\begin{scheme}
(setbalance! ~\( \dark \langle \)~~\var{\dark account}~~\( \dark \rangle \)~ ~\( \dark \langle \)~~\var{\dark newvalue}~~\( \dark \rangle \)~)
\end{scheme}

\noindent
これは指定した口座の差引残高を指定した新しい値に変更します。ミューテータが定義された
データオブジェクトは\newterm{mutable data objects}(\jnewterm{ミュータブルデータオブジェクト}、
変更可能なオブジェクト)として知られます。


\link{Chapter 2}は複合データを合成するため汎用目的の``糊''としてのペアを紹介しました。
この節はペアのための基本的なミューテータを定義することから始め、ペアが変更可能な
データオブジェクトを構築するための架設ブロックとして供給できるようにします。
これらのミューテータはペアの表現力を大きく拡張し、\link{Section 2.2}で用いた列と木以外のデータ構造を
構築することを可能にします。複雑なシステムが局所状態を持つオブジェクトの集合としてモデル化される
シミュレーションのいくつかの例も紹介します。



\subsection{ミュータブルなリスト構造}
\label{Section 3.3.1}



ペア上の基本的な命令---\code{cons}, \code{car}, \code{cdr}---はリスト構造の構築と
リスト構造からの部品の選択に用いることができます。しかしそれらはリスト構造を変更する
能力はありませんでした。同じことが今までに使用した\code{append}や\code{list}の様な
リスト命令にも正しいと言えます。これらが\code{cons}, \code{car}, \code{cdr}を用いて定義できる
ためです。リスト構造を変更するため新しい命令が必要です。



ペアのプリミティブなミューテータは\code{set\-/car!}と\code{set\-/cdr!}です。
\code{set\-/car!}は2つの引数を取り、第一引数はペアでなければなりません。
このペアの\code{car}ポインタを\code{set\-/car!}の第二引数へのポインタで置き換えることでペアを変更します。\footnote{
\code{set\-/car!}と\code{set\-/cdr!}は実装依存な値を返します。\code{set!}と同様にそれらはそれらの効果のためだけに
使用されるべきです。}

\begin{figure}[tp]
\phantomsection\label{Figure 3.12}
\centering
\begin{comment}
\heading{Figure 3.12:} Lists \code{x}: \code{((a b) c d)} and \code{y}: \code{(e f)}.

\begin{example}
     +---+---+     +---+---+     +---+---+
x -->| * | *-+---->| * | *-+---->| * | / |
     +-|-+---+     +-|-+---+     +-|-+---+
       |             V             V
       |           +---+         +---+
       |           | c |         | d |
       |           +---+         +---+
       |           +---+---+     +---+---+
       +---------->| * | *-+---->| * | / |
                   +-|-+---+     +-|-+---+
                     V             V
                   +---+         +---+
                   | a |         | b |
                   +---+         +---+
                   +---+---+     +---+---+
              y -->| * | *-+---->| * | / |
                   +-|-+---+     +-|-+---+
                     V             V
                   +---+         +---+
                   | e |         | f |
                   +---+         +---+
\end{example}
\end{comment}
\includegraphics[width=72mm]{fig/chap3/Fig3.12b.pdf}
\begin{quote}
\heading{Figure 3.12:} リスト \code{x}: \code{((a b) c d)}と\code{y}: \code{(e f)}
\end{quote}
\end{figure}




例として\link{Figure 3.12}に示すように\code{x}がリスト\code{((a b) c d)}に、\code{y}がリスト\code{(e f)}に
束縛されているとします。式\code{ (set\-/car!  x y)}の評価は\code{x}が束縛されているペアを変更し、
その\code{car}を\code{y}の値で置き換えます。命令の結果は\link{Figure 3.13}に示されています。
構造\code{x}が変更され\code{((e f) c d)}となりました。リスト\code{(a b)}を表すペアは、置き換えられた
ポインタにより特定されていましが、元の構造から取り外されました。\footnote{
この点からリストの変更命令はどのアクセス可能な構造の部分でもない``garbage''(ゴミ)を作り得ることが
わかります。\link{Section 5.3.2}にてLispのメモリ管理システムが\newterm{garbage collector}(\jnewterm{ガベージコレクタ}、
清掃局員)を持ち、それにより必要の無いペアにより使用されているメモリ空間を判断しリサイクルを行います。}



\link{Figure 3.13}と\link{Figure 3.14}を比べてください。これは\code{x}と\code{y}が\link{Figure 3.12}の元のリストに
束縛されている時に\code{(define z (cons y (cdr x)))}を実行した結果を図示しています。
変数\code{z}はこれで\code{cons}命令により作成された新しいペアに束縛されます。\code{x}が束縛される
リストは変更されません。


\begin{figure}[bp]
\phantomsection\label{Figure 3.13}
\centering
\begin{comment}
\heading{Figure 3.13:} Effect of \code{(set\-/car! x y)} on the lists in \link{Figure 3.12}.

\begin{example}
     +---+---+     +---+---+     +---+---+
x -->| * | *-+---->| * | *-+---->| * | / |
     +-|-+---+     +-|-+---+     +-|-+---+
       |             V             V
       |           +---+         +---+
       |           | c |         | d |
       |           +---+         +---+
       |           +---+---+     +---+---+
       |           | * | *-+---->| * | / |
       |           +-|-+---+     +-|-+---+
       |             V             V
       |           +---+         +---+
       |           | a |         | b |
       |           +---+         +---+
       +---------->+---+---+     +---+---+
                   | * | *-+---->| * | / |
              y -->+-|-+---+     +-|-+---+
                     V             V
                   +---+         +---+
                   | e |         | f |
                   +---+         +---+
\end{example}
\end{comment}
\includegraphics[width=72mm]{fig/chap3/Fig3.13b.pdf}
\par\bigskip
\noindent
\heading{Figure 3.13:} \link{Figure 3.12}のリスト上での\code{(set\-/car! x y)}の効果
\end{figure}


\begin{figure}[tp]
\phantomsection\label{Figure 3.14}
\centering
\begin{comment}
\heading{Figure 3.14:} Effect of \code{(define z (cons y (cdr x)))} on the lists in \link{Figure 3.12}.

\begin{example}
     +---+---+     +---+---+     +---+---+
x -->| * | *-+---->| * | *-+---->| * | / |
     +-|-+---+ +-->+-|-+---+     +-|-+---+
       |       |     V             V
       |       |   +---+         +---+
       |       |   | c |         | d |
       |       |   +---+         +---+
       |       |   +---+---+     +---+---+
       +-------+-->| * | *-+---->| * | / |
               |   +-|-+---+     +-|-+---+
     +---+---+ |     V             V
z -->| * | *-+-+   +---+         +---+
     +-|-+---+     | a |         | b |
       |           +---+         +---+
       +---------->+---+---+     +---+---+
                   | * | *-+---->| * | / |
              y -->+-|-+---+     +-|-+---+
                     V             V
                   +---+         +---+
                   | e |         | f |
                   +---+         +---+
\end{example}
\end{comment}
\includegraphics[width=72mm]{fig/chap3/Fig3.14b.pdf}
\begin{quote}
\heading{Figure 3.14:} \link{Figure 3.12}のリスト上での\code{(define z (cons y (cdr x)))}の効果
\end{quote}
\end{figure}

\clearpage


\noindent
\code{set\-/cdr!}命令は\code{set\-/car!}と同様です。違いは\code{car}ポインタでなく、\code{cdr}ポインタが置き換えられます。
\link{Figure 3.12}のリスト上での\code{(set\-/cdr! x y)}の実行の結果は\link{Figure 3.15}に示されます。
ここでは\code{x}の\code{cdr}ポインタは\code{(e f)}へのポインタにて置き換えられます。また\code{x}の\code{cdr}として用いられる
リスト\\\code{(c d)}はこれで構造から取り外されます。

\enlargethispage{\baselineskip}

\begin{figure}[bp]
\phantomsection\label{Figure 3.15}
\centering
\begin{comment}
\heading{Figure 3.15:} Effect of \code{(set\-/cdr! x y)} on the lists in \link{Figure 3.12}.

\begin{example}
     +---+---+     +---+---+     +---+---+
x -->| * | * |     | * | *-+---->| * | / |
     +-|-+-|-+     +-|-+---+     +-|-+---+
       |   |         V             V
       |   |       +---+         +---+
       |   |       | c |         | d |
       |   |       +---+         +---+
       |   |       +---+---+     +---+---+
       +---+------>| * | *-+---->| * | / |
           |       +-|-+---+     +-|-+---+
           |         V             V
           |       +---+         +---+
           |       | a |         | b |
           |       +---+         +---+
           +------>+---+---+     +---+---+
                   | * | *-+---->| * | / |
              y -->+-|-+---+     +-|-+---+
                     V             V
                   +---+         +---+
                   | e |         | f |
                   +---+         +---+
\end{example}
\end{comment}
\includegraphics[width=72mm]{fig/chap3/Fig3.15b.pdf}
\par\bigskip
\noindent
\heading{Figure 3.15:} \link{Figure 3.12}のリスト上での\code{(set\-/cdr! x y)}の効果
\end{figure}

\code{cons}は新しいリスト構造を新しいペアを作成することで構築します。一方、\code{set\-/car!}と\code{set\-/cdr!}は
既存のペアを変更します。実際に2つのミューテータと既存のリスト構造の一部ではない新しいペアを返す
\code{get\-/new\-/pair}を一緒に用いて\code{cons}を実装することができます。
新しいペアを得てからその\code{car}と\code{cdr}ポインタに指定されたオブジェクトを設定し、\code{cons}の結果として
返します。\footnote{
\code{get\-/new\-/pair}はLisp実装にて必要とされるメモリ管理の一部として実装されなければならない
命令の1つです。これについては \link{Section 5.3.1}にて議論します。}

\begin{scheme}
(define (cons x y)
  (let ((new (get-new-pair)))
    (set-car! new x)
    (set-cdr! new y)
    new))
\end{scheme}

\clearpage

\begin{quote}
\heading{\phantomsection\label{Exercise 3.12}Exercise 3.12:} 
リストを接続するための以下の手続は\link{Section 2.2.1}で紹介した。

\begin{scheme}
(define (append x y)
  (if (null? x)
      y
      (cons (car x) (append (cdr x) y))))
\end{scheme}



\code{append}は\code{y}に連続して\code{x}の要素を\code{cons}することで新しいリストを作る。
手続\code{append!}は\code{append}と同様だが、コンストラクタではなくミューテータである。
これは\code{x}の最後のペアを変更しその\code{cdr}を\code{y}にし両者を繋ぎ合わせることで
append(付け加え)する。
(\code{append!}を空の\code{x}にて呼ぶのはエラーとなる)。

\begin{scheme}
(define (append! x y)
  (set-cdr! (last-pair x) y)
  x)
\end{scheme}


ここで\code{last\-/pair}はその引数の最後のペアを返す手続である。

\begin{scheme}
(define (last-pair x)
  (if (null? (cdr x)) x (last-pair (cdr x))))
\end{scheme}



以下の応答について考えよ。

\begin{scheme}
(define x (list 'a 'b))
(define y (list 'c 'd))
(define z (append x y))
z
~\textit{(a b c d)}~
(cdr x)
~\( \dark \langle \)~~\var{\dark response}~~\( \dark \rangle \)~
(define w (append! x y))
w
~\textit{(a b c d)}~
(cdr x)
~\( \dark \langle \)~~\var{\dark response}~~\( \dark \rangle \)~
\end{scheme}



欠けている\( \langle \)\var{response}\( \rangle \)は何か?
箱とポインタの図をあなたの答を説明するために描け。
\end{quote}

\begin{quote}
\heading{\phantomsection\label{Exercise 3.13}Exercise 3.13:}
次の\code{make\-/cycle}手続について考えよ。これは\link{Exercise 3.12}で定義した
\code{last\-/pair}手続を用いる。

\begin{scheme}
(define (make-cycle x)
  (set-cdr! (last-pair x) x)
  x)
\end{scheme}


以下の様に作成される\code{z}を表す箱とポインタの図を描け。

\begin{scheme}
(define z (make-cycle (list 'a 'b 'c)))
\end{scheme}


\code{(last\-/pair z)}を演算すると何が置こるか?
\end{quote}

\begin{quote}
\heading{\phantomsection\label{Exercise 3.14}Exercise 3.14:} 
以下の手続はとても便利であるが不明瞭である。

\begin{scheme}
(define (mystery x)
  (define (loop x y)
    (if (null? x)
        y
        (let ((temp (cdr x)))
          (set-cdr! x y)
          (loop temp x))))
  (loop x '()))
\end{scheme}



\code{loop}は``temporary''(一時的)な変数\code{temp}を用いて\code{x}の\code{cdr}を保存する。
次の行の\code{set\-/cdr!}が\code{cdr}を破壊するためである。\code{mystery}が通常何を行うのか
説明せよ。\code{v}が\code{(define v (list 'a 'b 'c 'd))}で定義されているとする。
\code{v}が束縛されるリストを表す箱とポインタの図を描け。次に\code{(define w (mystery v))}を
評価したとする。この式を評価した後の\code{v}と\code{w}の構造を表す箱とポインタの図を描け。
\code{v}と\code{w}の値として何が表示されるか?
\end{quote}

\subsubsection*{共有と自己同一性}



\link{Section 3.1.3}で代入の導入に伴う``同一性''と``変更''という論理的な問題について記述しました。
これらの問題は実際の所個々のペアが異なるデータオブジェクトの間で\newterm{shared}(\jnewterm{共有})
されている時に問題となります。例えば、以下の様に形成される構造について考えてみて下さい。

\begin{scheme}
(define x (list 'a 'b))
(define z1 (cons x x))
\end{scheme}

\noindent
\link{Figure 3.16}で示されるように、\code{z1}はその\code{car}と\code{cdr}の両者が同じペア\code{x}を
指している。この\code{z1}の\code{car}と\code{cdr}による\code{x}の共有は\code{cons}が
直接的な方法で実装されていることによる結果です。一般的に\code{cons}を用いてリストを構築する
ことは多くの個別のペアが多くの異なる構造において共有される、ペアの連結構造に帰着します。


\link{Figure 3.16}とは対照的に、\link{Figure 3.17}は以下の式で作成された構造を示します。

\begin{scheme}
(define z2 (cons (list 'a 'b) (list 'a 'b)))
\end{scheme}

\noindent
この構造においては、2つの\code{(a b)}リスト内のペアは実際のシンボルが共有されていても区別可能です。\footnote{
2つのペアは各\code{cons}呼出が新しいペアを返すため区別可能です。シンボルは共有されています。
Schemeではどの与えられた名前にも固有のシンボルが存在します。Schemeがシンボルを変更する手段を
全く提供しないため、この共有は判別不可能です。共有が単純にポインタの等価性をチェックする\code{eq?}を用いて
シンボルで比較することを可能にする物であることにも注意して下さい。}

\begin{figure}[tb]
\phantomsection\label{Figure 3.16}
\centering
\begin{comment}
\heading{Figure 3.16:} The list \code{z1} formed by \code{(cons x x)}.

\begin{example}
      +---+---+
z1 -->| * | * |
      +-|-+-|-+
        V   V
      +---+---+     +---+---+
 x -->| * | *-+---->| * | / |
      +-|-+---+     +-|-+---+
        V             V
      +---+         +---+
      | a |         | b |
      +---+         +---+
\end{example}
\end{comment}
\includegraphics[width=46mm]{fig/chap3/Fig3.16b.pdf}
\begin{quote}
\heading{Figure 3.16:} \code{(cons x x)}.で形成されたリスト\code{z1}
\end{quote}
\end{figure}

\begin{figure}[tb]
\phantomsection\label{Figure 3.17}
\centering
\begin{comment}
\begin{quote}
\heading{Figure 3.17:} The list \code{z2} formed by \code{(cons (list 'a 'b) (list 'a 'b))}.

\begin{example}
      +---+---+     +---+---+     +---+---+
z2 -->| * | *-+---->| * | *-+---->| * | / |
      +-|-+---+     +-|-+---+     +-|-+---+
        |             V             V
        |           +---+         +---+
        |           | a |         | b |
        |           +---+         +---+
        |             ^             ^
        |             |             |
        |           +-|-+---+     +-|-+---+
        +---------->| * | *-+---->| * | / |
                    +---+---+     +---+---+
\end{example}
\end{quote}
\end{comment}
\includegraphics[width=71mm]{fig/chap3/Fig3.17b.pdf}
\begin{quote}
\heading{Figure 3.17:} \code{(cons (list 'a 'b) (list 'a 'b))}により形成されたリスト\code{z2}
\end{quote}
\end{figure}

\noindent
リストとして考えた時、\code{z1}と\code{z2}の両方が``同じ''リスト\code{((a b) a b)}を表現します。
一般的に共有はリスト上で用いる命令が\code{cons}, \code{car}, \code{cdr}だけならば完全に検出不可能です。
しかしリスト構造上で変更を許可するのであれば、共有に気付くことができます。
共有が作成できる違いの例として、適用された引数の構造の\code{car}を変更する以下の手続について
考えてみましょう。

\begin{scheme}
(define (set-to-wow! x) (set-car! (car x) 'wow) x)
\end{scheme}

\noindent
例え\code{z1}と\code{z2}が``同じ''構造だとしても、\code{set\-/to\-/wow!}をそれらに適用すると異なる結果を返します。
\code{z1}では\code{car}の変更は\code{cdr}も変更します。\code{z1}では\code{car}と\code{cdr}が同じペアであるためです。
\code{z2}では\code{car}と\code{cdr}は区別可能なので\code{set\-/to\-/wow!}は\code{car}のみを変更します。

\begin{scheme}
z1
~\textit{((a b) a b)}~
(set-to-wow! z1)
~\textit{((wow b) wow b)}~
z2
~\textit{((a b) a b)}~
(set-to-wow! z2)
~\textit{((wow b) a b)}~
\end{scheme}

\noindent
リスト構造中の共有を見つける一つの方法は\link{Section 2.3.1}で2つのシンボルが等しいかテストする方法として
紹介した述語\code{eq?}を用います。より一般的には\code{(eq?  x y)}は\code{x}と\code{y}が同じオブジェクトであるか
をテストします(これはつまり\code{x}と\code{y}はポインタとして等しいかです)。従って\link{Figure 3.16}と
\link{Figure 3.17}で示すよう定義された\code{z1}と\code{z2}


以降の節で示されるように、ペアで表現可能なデータ構造のレパートリを大きく拡張することが共有を用いてできます。
一方で、共有はまた危険であり構造に対して行われる変更がたまたま部品を共有する他の構造に対しても影響を与えます。
ミューテータである\code{set\-/car!}と\code{set\-/cdr!}は注意深く利用せねばなりません。
データオブジェクトがどのように共有されているかを良く理解しなければ変更は予期しない結果を引き起します。\footnote{
ミュータブルなデータオブジェクトの共有の取扱の微妙な部分は\link{Section 3.1.3}で取り上げられた``等価性''と``変更''の
根底に横たわる問題を反映しています。そこでは私達の言語に変更を許すことは複合データがそれを構成する部分から何かが
異なるという``自己同一性''を持たねばならないことを述べました。
Lispではこの``自己同一性''を\code{eq?}にてテストされる性質だと考えます。即ち、ポインタの等価性です。
多くのLisp実装ではポインタが本質的にはメモリアドレスですので、オブジェクトの自己同一性を定義することの
``問題の解決''はデータオブジェクト``それ自身''がいくつかの特定の計算機内のメモリ上の場所の集合に格納された
情報であることを要求することにより解決します。これは単純なLispプログラムには十分ですが、
計算モデルの``同一性''の問題を解決する一般的な方法ではありません。}

\begin{quote}
\heading{\phantomsection\label{Exercise 3.15}Exercise 3.15:} 
上記の構造\code{z1}と\code{z2}上での\code{set\-/to\-/wow!}の効果を説明する箱とポインタの図を描け。
\end{quote}

\begin{quote}
\heading{\phantomsection\label{Exercise 3.16}Exercise 3.16:} 
Ben Bitdiddleは任意のリスト構造内のペアの数を数える手続を書くことに決めた。``簡単だよね''と彼は思った。
``任意の構造内のペアの数は\code{car}の中の数と\code{cdr}の中の数の和に現在のペアを数えるために1を足した物''。
だからBenは以下の手続を書いた。

\begin{scheme}
(define (count-pairs x)
  (if (not (pair? x))
      0
      (+ (count-pairs (car x))
         (count-pairs (cdr x))
         1)))
\end{scheme}



この手続が正しくないことを示せ。具体的にはきっちり3つのペアにより作られBenの手続が3, 4, 7を返すだろう、
またBenの手続が絶対に終了しないリスト構造を表現する箱とポインタの図を描け。
\end{quote}

\begin{quote}
\heading{\phantomsection\label{Exercise 3.17}Exercise 3.17:} 
\link{Exercise 3.16}の\code{count\-/pairs}手続の正しい版を考案せよ。これは任意の構造の
中の固有のペアの数を返す。(ヒント:構造を横断しながらどのペアが既に数えられたかを
追跡するために使用する補助的なデータ構造を保存する)。
\end{quote}

\begin{quote}
\heading{\phantomsection\label{Exercise 3.18}Exercise 3.18:}
リストを検査しそれが循環を持つかどうか判断せよ。つまりリストの最後を見つけようとした
プログラムが連続して\code{cdr}を取ることで無限ループに入るかどうかを判定せよ。
\link{Exercise 3.13}にてそのようなリストを構築した。
\end{quote}

\begin{quote}
\heading{\phantomsection\label{Exercise 3.19}Exercise 3.19:} Redo \link{Exercise 3.18} using an
algorithm that takes only a constant amount of space.  (This requires a very
clever idea.)

\link{Exercise 3.18}を一定の容量のメモリのみを用いるアルゴリズムを用いて再度行え。
(これはとても巧妙なアイデアを必要とする)。
\end{quote}

\subsubsection*{変更とは代入のこと}



複合データを紹介した時、\link{Section 2.1.3}にてペアが手続をのみを利用することで表現できることを観察しました。

\begin{scheme}
(define (cons x y)
  (define (dispatch m)
    (cond ((eq? m 'car) x)
          ((eq? m 'cdr) y)
          (else (error "Undefined operation: CONS" m))))
  dispatch)
(define (car z) (z 'car))
(define (cdr z) (z 'cdr))
\end{scheme}

\noindent
同じ観察結果がミュータブルなデータに対しても正しいと言えます。ミュータブル(可変)な
データオブジェクトを代入と局所状態を用いることで手続として実装可能です。
例として上のペアの実装を拡張し、\link{Section 3.1.1}で\code{make\-/account}を用いて銀行口座を
実装した方法とある程度類似して、\code{set\-/car!}と\code{set\-/cdr!}を扱うことができます。

\begin{scheme}
(define (cons x y)
  (define (set-x! v) (set! x v))
  (define (set-y! v) (set! y v))
  (define (dispatch m)
    (cond ((eq? m 'car) x)
          ((eq? m 'cdr) y)
          ((eq? m 'set-car!) set-x!)
          ((eq? m 'set-cdr!) set-y!)
          (else 
           (error "Undefined operation: CONS" m))))
  dispatch)
(define (car z) (z 'car))
(define (cdr z) (z 'cdr))
(define (set-car! z new-value) 
  ((z 'set-car!) new-value) z)
(define (set-/cdr! z new-value)
  ((z 'set-cdr!) new-value) z)
\end{scheme}

\noindent
代入が可変データの振舞を説明するために論理上必要な物全てです。
私達の言語に\code{set!}を認めると直ぐに、代入の問題のみでなく、一般的な可変データの全ての問題を
引き起しました。\footnote{
一方で、実装上の視点からは代入は環境を変更することを必要とし、環境はそれ自身が可変な
データ構造です。従って代入と変更は等位です。つまり一方は他方を用いることで実装可能です。}
\begin{quote}
\heading{\phantomsection\label{Exercise 3.20}Exercise 3.20:} 
以下の連続した式の評価を説明する環境の図を描け。

\begin{scheme}
(define x (cons 1 2))
(define z (cons x x))
(set-car! (cdr z) 17)
(car x)
~\textit{17}~
\end{scheme}

\noindent
上で与えられた手続型の実装を用いよ。(\link{Exercise 3.11}と比較せよ)。
\end{quote}

\subsection{キューの表現}
\label{Section 3.3.2}



ミューテータの\code{set\-/car!}と\code{set\-/cdr!}はペアを用いて\code{cons}, \code{car}, \code{cdr}のみでは
不可能なデータ構造を構築可能です。この節ではキューと呼ばれるデータ構造を表現するために
どのようにペアを用いるかについて示します。\link{Section 3.3.3}ではテーブル(表)と呼ばれるデータ構造
の表現方法について学びます。


\newterm{queue}(\jnewterm{キュー})はアイテムが一方の端(キューの\newterm{rear}(リア、\jnewterm{終端}))に挿入され、
他方の端(\newterm{front}(フロント、\jnewterm{先端}))から削除される列です。
\link{Figure 3.18}は初期化時に空のキューにアイテム\code{a}と\code{b}が挿入された状態を示しています。
次に\code{a}が削除され、\code{c}と\code{d}が挿入され、\code{b}が削除されます。
アイテムは常に挿入順に削除されるためキューは時々\newterm{FIFO}(first in, first out)(\jnewterm{先入れ先出し})バッファ
と呼ばれます。

\begin{figure}[tb]
\phantomsection\label{Figure 3.18}
\centering
\begin{comment}
\heading{Figure 3.18:} Queue operations.

\begin{example}
Operation                Resulting Queue
(define q (make-queue))
(insert-queue! q 'a)     a
(insert-queue! q 'b)     a b
(delete-queue! q)        b
(insert-queue! q 'c)     b c
(insert-queue! q 'd)     b c d
(delete-queue! q)        c d
\end{example}
\end{comment}
\includegraphics[width=70mm]{fig/chap3/Fig3.18a.pdf}
\par\bigskip
\noindent
\heading{Figure 3.18:} キュー命令
\end{figure}

\noindent
データ抽象化の観点ではキューを以下の操作の集合であると見做すことができます。

\begin{itemize}

\item
コンストラクタ:\code{(make\-/queue)}は空のキュー(アイテムを全く持たないキュー)を返す

\item
2つのセレクタ:

\begin{scheme}
(empty-queue? ~\( \dark \langle \)~~\var{\dark queue}~~\( \dark \rangle \)~)
\end{scheme}

\noindent
キューが空であるかテストする

\begin{scheme}
(front-queue ~\( \dark \langle \)~~\var{\dark queue}~~\( \dark \rangle \)~)
\end{scheme}

\noindent
キューの先頭のオブジェクトを返す。もしキューが空ならエラーを発す。キューを変更しない。

\item
2つのミューテータ:

\begin{scheme}
(insert-queue! ~\( \dark \langle \)~~\var{\dark queue}~~\( \dark \rangle \)~ ~\( \dark \langle \)~~\var{\dark item}~~\( \dark \rangle \)~)
\end{scheme}

\noindent
キューの最後尾にアイテムを挿入し、変更されたキューをその値として返す。

\begin{scheme}
(delete-queue! ~\( \dark \langle \)~~\var{\dark queue}~~\( \dark \rangle \)~)
\end{scheme}

\noindent
キューの先頭のアイテムを削除し、その値として変更されたキューを返す。もしキューが削除前に空であればエラーを発す。

\end{itemize}

\noindent
キューはアイテムの列であるため確かに順序有りリストであると表現できます。キューの先頭はリストの\code{car}であり、
キューにアイテムを挿入するのは新しい要素をリストの最後に追加することで、キューからのアイテムの削除は
ただリストの\code{cdr}を得ることと言えるでしょう。しかしこの表現は非効率です。
なぜならアイテムを挿入するためにはリストを終端まで走査しなければなりません。
リストの走査のための手段は\code{cdr}命令を連続して用いるしかなく、この走査は\( n \)アイテムのリストに対し
\( \Theta(n) \)ステップを必要とします。
リスト表現に対する簡単な変更がこの欠点を克服し\( \Theta \)(1)ステップを必要とするキュー命令の実装を可能にします。
これはつまり必要なステップ数がキューの長さから独立するということです。



リスト表現による困難はリストの終端を見つけるための走査が必要である点から生じています。
走査が必要な理由はリストをペアの鎖として表現する標準的な方法が、事前にリストの先頭への
ポインタを提供するのに対し、終端を指す簡単にアクセス可能なポインタを提供しないためです。
欠点を避けるための変更としてキューをリストとしながらリストの最終ペアを示す追加のポインタ
をも用いて表現します。この方法ではアイテムを挿入する場合に終端ポインタを調べることで
リストの走査を避けることができます。



するとキューはポインタのペア、\code{front\-/ptr}と\code{rear\-/ptr}として表現されます。
それぞれが通常のリストの先頭と最後のペアを指します。キューを識別可能なオブジェクトにするために
2つのポインタを接続するのに\code{cons}を用います。従ってキューそれ自身が2つのポインタの\code{cons}に
なります。\link{Figure 3.19}はこの表現を図示します。

\begin{figure}[tb]
\phantomsection\label{Figure 3.19}
\centering
\begin{comment}
\begin{quote}
\heading{Figure 3.19:} Implementation of a queue as a list with front and rear pointers.

\begin{example}
       +---+---+
  q -->| * | *-+-------------------+
       +-|-+---+                   |
         |                         |
         | front-ptr               | rear-ptr
         V                         V
     +---+---+    +---+---+    +---+---+
     | * | *-+--->| * | *-+--->| * | / |
     +-|-+---+    +-|-+---+    +-|-+---+
       V            V            V
     +---+        +---+        +---+
     | a |        | b |        | c |
     +---+        +---+        +---+
\end{example}
\end{quote}
\end{comment}
\includegraphics[width=69mm]{fig/chap3/Fig3.19b.pdf}
\begin{quote}
\heading{Figure 3.19:} 先端と終端のポインタを持つリストとしてのキューの実装
\end{quote}
\end{figure}


\noindent
キューの命令を定義するために以下の手続を用います。これはキューの先端と終端のポインタの選択、変更を
可能にします。

\begin{scheme}
(define (front-ptr queue) (car queue))
(define (rear-ptr queue) (cdr queue))
(define (set-front-ptr! queue item) (set-car! queue item))
(define (set-rear-ptr! queue item) (set-cdr! queue item))
\end{scheme}

\noindent
これで実際のキューの命令を実装できます。もし先端のポインタが空リストならばキューは空であると
考えることにします。

\begin{scheme}
(define (empty-queue? queue) (null? (front-ptr queue)))
\end{scheme}

\noindent
\code{make\-/queue}コンストラクタは初期値として空キューを意味する\code{car}と\code{cdr}の両方が空リストの
ペアを返します。

\begin{scheme}
(define (make-queue) (cons '() '()))
\end{scheme}

\noindent
キューの頭のアイテムを選択するために先端ポインタが指すペアの\code{car}を返します。

\begin{scheme}
(define (front-queue queue)
  (if (empty-queue? queue)
      (error "FRONT called with an empty queue" queue)
      (car (front-ptr queue))))
\end{scheme}

\noindent
キューにアイテムを挿入するために、\link{Figure 3.20}が示す結果を成す手法に従います。
最初に\code{car}が挿入するアイテムであり\code{cdr}が空リストである新しいペアを作成します。
もしキューが空であるならキューの先端と終端のポインタにこの新しいペアを設定します。
そうでなければキューの最終ペアを新しいペアを指すように変更し、また終端ポインタを新しい
ペアを指すようにします。

\begin{scheme}
(define (insert-queue! queue item)
  (let ((new-pair (cons item '())))
    (cond ((empty-queue? queue)
           (set-front-ptr! queue new-pair)
           (set-rear-ptr! queue new-pair)
           queue)
          (else
           (set-cdr! (rear-ptr queue) new-pair)
           (set-rear-ptr! queue new-pair)
           queue))))
\end{scheme}

\begin{figure}[tb]
\phantomsection\label{Figure 3.20}
\centering
\begin{comment}
\heading{Figure 3.20:} Result of using \code{(insert\-/queue! q 'd)} on the queue of \link{Figure 3.19}.

\begin{example}
       +---+---+
  q -->| * | *-+--------------------------------+
       +-|-+---+                                |
         |                                      |
         | front-ptr                            | rear-ptr
         V                                      V
     +---+---+    +---+---+    +---+---+    +---+---+
     | * | *-+--->| * | *-+--->| * | *-+--->| * | / |
     +-|-+---+    +-|-+---+    +-|-+---+    +-|-+---+
       V            V            V            V
     +---+        +---+        +---+        +---+
     | a |        | b |        | c |        | d |
     +---+        +---+        +---+        +---+
\end{example}
\end{comment}
\includegraphics[width=88mm]{fig/chap3/Fig3.20b.pdf}
\begin{quote}
\heading{Figure 3.20:} \link{Figure 3.19}のキューに\code{(insert\-/queue! q 'd)}を用いた結果 
\end{quote}
\end{figure}

\noindent
キューの頭のアイテムを削除するために、ただ単に先端ポインタを変更しキューの二つ目のアイテムを
指すようにします。これは最初のアイテムの\code{cdr}ポインタに従うだけで見つけられます。(\link{Figure 3.21}参照)\footnote{
もし最初のアイテムがキューの最終アイテムでもある場合、先端ポインタは削除後に空リストになるでしょう。
これはキューを空の状態にします。終端ポインタの更新を心配する必要はありません。これは依然として
削除されたアイテムを指しますが、\code{empty\-/queue?}は先端ポインタしか見ません。}

\begin{scheme}
(define (delete-queue! queue)
  (cond ((empty-queue? queue)
         (error "DELETE! called with an empty queue" queue))
        (else (set-front-ptr! queue (cdr (front-ptr queue)))
              queue)))
\end{scheme}

\begin{figure}[tb]
\phantomsection\label{Figure 3.21}
\centering
\begin{comment}
\heading{Figure 3.21:} Result of using \code{(delete\-/queue!  q)} on the queue of \link{Figure 3.20}.

\begin{example}
       +---+---+
  q -->| * | *-+--------------------------------+
       +-|-+---+                                |
         +------------+                         |
            front-ptr |                         | rear-ptr
                      V                         V
     +---+---+    +---+---+    +---+---+    +---+---+
     | * | *-+--->| * | *-+--->| * | *-+--->| * | / |
     +-|-+---+    +-|-+---+    +-|-+---+    +-|-+---+
       V            V            V            V
     +---+        +---+        +---+        +---+
     | a |        | b |        | c |        | d |
     +---+        +---+        +---+        +---+
\end{example}
\end{comment}
\includegraphics[width=88mm]{fig/chap3/Fig3.21b.pdf}
\begin{quote}
\heading{Figure 3.21:} \link{Figure 3.20}のキューに\code{(delete\-/queue!  q)}を用いた結果
\end{quote}
\end{figure}


\begin{quote}
\heading{\phantomsection\label{Exercise 3.21}Exercise 3.21:} 
Ben Bitdiddleは上で説明されたキューの実装をテストすることに決めた。
彼はLispインタプリタに対し手続を入力し、続いて以下のように試行を行った。

\begin{scheme}
(define q1 (make-queue))
\end{scheme}

\begin{scheme}
(insert-queue! q1 'a)
~\textit{((a) a)}~
\end{scheme}

\begin{scheme}
(insert-queue! q1 'b)
~\textit{((a b) b)}~
\end{scheme}

\begin{scheme}
(delete-queue! q1)
~\textit{((b) b)}~
\end{scheme}

\begin{scheme}
(delete-queue! q1)
~\textit{(() b)}~
\end{scheme}




``間違っている!''と彼は文句を言った。``インタプリタの応答は最後のアイテムがキューに二回挿入されている
ことを示している。そして僕が両方のアイテムを消しても二つ目の\code{b}がまだそこにある。だからキューは
空になるべきなのにそうならない''。Eva Lu AtorはBenが何が起こったのか間違って理解していると示唆した。
``アイテムはキューに二回入ってはいないわ''と彼女は説明した。
``Lisp標準の応答がキュー表現の意味をどのように理解するのか知らないだけ。もしあなたがキューが正しく
表示されるのを見たいなら自分でキューを表示する手続を定義する必要があるわ''。
Eva Luが話していることを説明せよ。具体的にはなぜBenの例がそのような表示の結果になるのか示せ。
キューを入力に取りキュー内のアイテムの列を表示する手続\code{print\-/queue}を定義せよ。
\end{quote}

\begin{quote}
\heading{\phantomsection\label{Exercise 3.22}Exercise 3.22:} 
キューをポインタのペアとして表現する代わりに、キューを局所状態を持つ手続として構築することができる。
局所状態は通常のリストの先端と終端へのポインタから成る。従って\code{make\-/queue}手続は以下の形式となる。

\begin{scheme}
(define (make-queue)
  (let ((front-ptr ~\( \dots \)~ )
        (rear-ptr ~\( \dots \)~ ))
    ~\( \dark \langle \)~~\var{\dark definitions of internal procedures}~~\( \dark \rangle \)~
    (define (dispatch m) ~\( \dots \)~)
    dispatch))
\end{scheme}

\code{make\-/queue}の定義を完成させ、この表現を用いたキューの命令を実装せよ。
\end{quote}



\begin{quote}
\heading{\phantomsection\label{Exercise 3.23}Exercise 3.23:} 
\newterm{deque}(``double\-/ended queue''、\jnewterm{両頭キュー})はアイテムの挿入と消去が先端と終端の
両方に対して行える列である。deque上の命令はコンストラクタ\code{make\-/deque}、述語\code{empty\-/deque?}、
セレクタ\code{front\-/deque}と\code{rear\-/deque}、ミューテータ\code{front\-/insert\-/deque!},
\code{rear\-/insert\-/deque!}, \code{front\-/delete\-/deque!}, \code{rear\-/delete\-/deque!}である。
ペアを用いてどのようにdequeを表現するか示せ。また命令の実装を提供せよ。\footnote{
インタプリタに循環を含む構造を表示させないように注意せよ。(\link{Exercise 3.13}参照)。}
全ての命令は\( \Theta \)(1)ステップで達成すること。
\end{quote}

\subsection{テーブルの表現}
\label{Section 3.3.3}



\link{Chapter 2}で種々の集合の表現について学んだ時、\link{Section 2.3.3}にてキーで同定する
索引を持つレコードの表を保存する作業について述べました。\link{Section 2.4.3}での
データ適従プログラミングの実装において二次元テーブルの広範な使用を行い、情報は
2つのキーを用いて格納と取り出しされました。ここではどのように表をミュータブルなリスト構造として
構築するかについて学びます。

\begin{figure}[tb]
\phantomsection\label{Figure 3.22}
\centering
\begin{comment}
\heading{Figure 3.22:} A table represented as a headed list.

\begin{example}
 +---+---+    +---+---+    +---+---+    +---+---+
 | * | *-+--->| * | *-+--->| * | *-+--->| * | / |
 +-|-+---+    +-|-+---+    +-|-+---+    +-|-+---+
   |            |            |            |
   V            V            V            V
+---------+   +---+---+   +---+---+   +---+---+
| *table* |   | * | * |   | * | * |   | * | * |
+---------+   +-|-+-|-+   +-|-+-|-+   +-|-+-|-+
                |   |       |   |       |   |
                V   V       V   V       V   V
             +---+ +---+ +---+ +---+ +---+ +---+
             | a | | 1 | | b | | 2 | | c | | 3 |
             +---+ +---+ +---+ +---+ +---+ +---+
\end{example}
\end{comment}
\includegraphics[width=81mm]{fig/chap3/Fig3.22c.pdf}
\par\bigskip
\noindent
\heading{Figure 3.22:} 頭出しリストとして表現されたテーブル
\end{figure}

\noindent
最初は一次元の表について考えます。各値が単一のキーの下に格納されます。
テーブルをレコードのリストとして実装し、各レコードははキーと関連する値から成るペアとして実装します。
レコードは\code{car}が次のレコードを指すペアによりリストを形成する様に連結されます。
これらの連結されたペアは表の\newterm{backbone}(\jnewterm{バックボーン}、基幹)と呼ばれます。
テーブルに新しいレコードを追加する時に変更可能な場所を得るために、テーブルを
\newterm{headed list}(\jnewterm{頭出しリスト})として構築します。頭出しリストは特別なバックボーンペアを
最初に持ちます。これはダミーの``レコード''---今回の場合、自由裁量で選択したシンボル\code{*table*}---
を持っています。\link{Figure 3.22}は以下のテーブルの箱とポインタの図を示しています。

\begin{scheme}
a:  1
b:  2
c:  3
\end{scheme}

\noindent
テーブルから情報を抽出するには鍵を引数として取り相対する値(またはそのキーの下に
値が格納されていない場合にはfalse)を返す\code{lookup}手続を用います。
\code{lookup}はキーとレコードのリストを引数として期待する\code{assoc}命令を用いて定義します。
\code{assoc}がダミーレコードを絶対に参照しないことに注意して下さい。\code{assoc}は与えられた
キーを\code{car}として持つレコードを返します。\footnote{\code{assoc}が\code{equal?}を用いるため、
シンボル、数値、リスト構造であるキーを認識可能です。}
すると\code{lookup}は\code{assoc}が返した結果のレコードがfalseでないかチェックし、そのレコードの
値(\code{cdr})を返します。

\begin{scheme}
(define (lookup key table)
  (let ((record (assoc key (cdr table))))
    (if record
        (cdr record)
        false)))
(define (assoc key records)
  (cond ((null? records) false)
        ((equal? key (caar records)) (car records))
        (else (assoc key (cdr records)))))
\end{scheme}

\noindent
値をテーブルに指定したキーの下に挿入するために、最初に\code{assoc}を用いて既にテーブルの中に
このキーを持つレコードが存在しないか確認します。もし無ければ鍵と値を\code{cons}することで
新しいレコードを作成しこれをテーブルのレコードリストの先頭のダミーレコードの後ろに挿入します。
もし既にこのキーのレコードが存在する場合にはそのレコードの\code{cdr}に新しい値を設定します。
テーブルのヘッダは新しいレコードを挿入するために変更する固定位置を与えます。\footnote{従って
最初のバックボーンペアはテーブル``それ自身''を表現するオブジェクトです。テーブルを指す
ポインタはこのペアを指すポインタです。この同じバックボーンペアが常にテーブルを始めます。
もしこのようにしなければ\code{insert!}は新しいレコードを追加した時にテーブルの新しい開始地点を
返さなければならなくなるでしょう。}

\begin{scheme}
(define (insert! key value table)
  (let ((record (assoc key (cdr table))))
    (if record
        (set-cdr! record value)
        (set-cdr! table
                  (cons (cons key value)
                        (cdr table)))))
  'ok)
\end{scheme}

\noindent
新しいテーブルを構築するためには単純にシンボル\code{*table*}を持つリストを作成します。

\begin{scheme}
(define (make-table)
  (list '*table*))
\end{scheme}

\subsubsection*{二次元テーブル}

\noindent
二次元テーブルでは各値は2つのキーにより索引付けられます。そのようなテーブルを
各キーが部分テーブルを特定する1次元テーブルとして構築することができます。
\link{Figure 3.23}は以下のテーブルを箱とポインタの図で示しています。

\begin{example}
math:    +:  43        letters:    a:  97
         -:  45                    b:  98
         *:  42
\end{example}

\noindent
これは2つ部分テーブルを持ちます。(部分テーブルは特別なヘッダシンボルを必要としません。
部分テーブルを特定するキーがこの目的を果たすためです)。



アイテムを探す時、最初のキーを用いて正しい部分テーブルを特定します。
次に二つ目のキーを用いて部分テーブル中のレコードを特定します。

\begin{scheme}
(define (lookup key-1 key-2 table)
  (let ((subtable 
         (assoc key-1 (cdr table))))
    (if subtable
        (let ((record 
               (assoc key-2 (cdr subtable))))
          (if record
              (cdr record)
              false))
        false)))
\end{scheme}

\begin{figure}[tb]
\phantomsection\label{Figure 3.23}
\centering
\begin{comment}
\heading{Figure 3.23:} A two-dimensional table.

\begin{example}
table
  |
  V
+---+---+   +---+---+   +---+---+
| * | *-+-->| * | *-+-->| * | / |
+-|-+---+   +-|-+---+   +-|-+---+
  V           |           V
+-------+     |         +---+---+   +---+---+   +---+---+
|*table*|     |         | * | *-+-->| * | *-+-->| * | / |
+-------+     |         +-|-+---+   +-|-+---+   +-|-+---+
              |           V           V           V
              |       +-------+     +---+---+   +---+---+
              |       |letters|     | * | * |   | * | * |
              |       +-------+     +-|-+-|-+   +-|-+-|-+
              |                       V   V       V   V
              |                    +---+ +---+ +---+ +---+
              |                    | a | | 97| | b | | 98|
              |                    +---+ +---+ +---+ +---+
              V
            +---+---+   +---+---+   +---+---+   +---+---+
            | * | *-+-->| * | *-+-->| * | *-+-->| * | / |
            +-|-+---+   +-|-+---+   +-|-+---+   +-|-+---+
              V           V           V           V
          +------+      +---+---+   +---+---+   +---+---+
          | math |      | * | * |   | * | * |   | * | * |
          +------+      +-|-+-|-+   +-|-+-|-+   +-|-+-|-+
                          V   V       V   V       V   V
                       +---+ +---+ +---+ +---+ +---+ +---+
                       | + | | 43| | - | | 45| | * | | 42|
                       +---+ +---+ +---+ +---+ +---+ +---+
\end{example}
\end{comment}
\includegraphics[width=103mm]{fig/chap3/Fig3.23a.pdf}
\par\bigskip
\noindent
\heading{Figure 3.23:} 二次元テーブル
\end{figure}

\noindent
キーのペアの下に新しいアイテムを挿入するには\code{assoc}を用いて最初のキーの下に
部分テーブルが存在するかどうかを確認します。もし無ければ単一のレコード(\code{key\-/2}, \code{value})を含む
新しい部分テーブルを構築し、それを最初のキーの下にテーブルに挿入します。
もし最初のキーに対する部分テーブルが既に存在する場合新しいレコードをこの部分テーブルに先に説明した
一次元テーブルに対する挿入方法を用いて挿入します。

\begin{scheme}
(define (insert! key-1 key-2 value table)
  (let ((subtable (assoc key-1 (cdr table))))
    (if subtable
        (let ((record (assoc key-2 (cdr subtable))))
          (if record
              (set-cdr! record value)
              (set-cdr! subtable
                        (cons (cons key-2 value)
                              (cdr subtable)))))
        (set-cdr! table
                  (cons (list key-1
                              (cons key-2 value))
                        (cdr table)))))
  'ok)
\end{scheme}

\subsubsection*{ローカルなテーブルの作成}



上で定義された\code{lookup}と\code{insert!}命令はテーブルを引数として取ります。
これが複数のテーブルにアクセスするプログラムを許可します。
複数のテーブルを扱う他の方法には各テーブルに対し分離された\code{lookup}と\code{insert!}手続を
持つ方法があります。これはテーブルを手続的に、その局所状態の一部に内部テーブルを持つ
オブジェクトとして表現することにより可能となります。
適切なメッセージを送った時に、この``テーブルオブジェクト''は内部テーブルを操作する手続を
提供します。以下にこの様式で表現された二次元テーブルのためのジェネレータ(生成器)を示します。

\begin{scheme}
(define (make-table)
  (let ((local-table (list '*table*)))
    (define (lookup key-1 key-2)
      (let ((subtable 
             (assoc key-1 (cdr local-table))))
        (if subtable
            (let ((record 
                   (assoc key-2 (cdr subtable))))
              (if record (cdr record) false))
            false)))
    (define (insert! key-1 key-2 value)
      (let ((subtable 
             (assoc key-1 (cdr local-table))))
        (if subtable
            (let ((record 
                   (assoc key-2 (cdr subtable))))
              (if record
                  (set-cdr! record value)
                  (set-cdr! subtable
                            (cons (cons key-2 value)
                                  (cdr subtable)))))
            (set-cdr! local-table
                      (cons (list key-1 (cons key-2 value))
                            (cdr local-table)))))
      'ok)
    (define (dispatch m)
      (cond ((eq? m 'lookup-proc) lookup)
            ((eq? m 'insert-proc!) insert!)
            (else (error "Unknown operation: TABLE" m))))
    dispatch))
\end{scheme}

\noindent
\code{make\-/table}を用いることで\link{Section 2.4.3}で用いたデータ適従プログラミングのための
\code{get}と\code{put}を以下のように実装することができます。

\begin{scheme}
(define operation-table (make-table))
(define get (operation-table 'lookup-proc))
(define put (operation-table 'insert-proc!))
\end{scheme}

\noindent
\code{get}は引数として2つのキーを取り、\code{put}は引数として2つのキーと値を取ります。
両方の命令共に同じ局所テーブルをアクセスします。局所テーブルは\code{make\-/table}の
呼出により作成されたオブジェクトの中にカブセル化されます。

\begin{quote}
\heading{\phantomsection\label{Exercise 3.24}Exercise 3.24:} 
上記のテーブル実装において、キーは\code{equal?}を用いて等価試験を行う。
(\code{assoc}により呼び出される)。これは常に適切な試験ではない。
例として数値キーを用いるテーブルを用いる場合に、検索時に
厳密に等しい必要が無く、ある許容範囲で数値を探したいかもしれない。
キーの``等価性''を試験するのに用いられる\code{same\-/key?}手続を引数として取る
テーブルコンストラクタ\code{make\-/table}を定義せよ。
\code{make\-/table}は内部テーブルに対して適切な手続\code{lookup}と\code{insert!}に
アクセスするのに使用可能な\code{dispatch}手続を返さねばならない。
\end{quote}

\begin{quote}
\heading{\phantomsection\label{Exercise 3.25}Exercise 3.25:} 
1次元と二次元のテーブルを一般化せよ。任意の数のキーの下で値を格納し、異なる値を
異なる数のキーの下格納できるテーブルをどのように実装するか示せ。
\code{lookup}と\code{insert!}手続は入力としてキーのリストを取りテーブルにアクセスする。
\end{quote}

\begin{quote}
\heading{\phantomsection\label{Exercise 3.26}Exercise 3.26:} 
上で実装されたテーブルを検索するにはレコードのリストを走査しなければならない。
これは基本的に\link{Section 2.3.3}の順序無しリスト表現である。大きなテーブルに対しては
異なる様式でテーブルを構造化するほうが効率が良い。(キー, 値)のレコードが二分木を用いて
体系化されるテーブルの実装を説明せよ。キーは何らかの方法にて順序付可能であると
想定する。(\link{Chapter 2}の\link{Exercise 2.66}と比較せよ)。
\end{quote}

\begin{quote}
\heading{\phantomsection\label{Exercise 3.27}Exercise 3.27:} 
\newterm{memoization}(\jnewterm{メモ化})(\newterm{tabulation}(\jnewterm{表形式化})とも呼ばれる)とは
手続の局所テーブルに事前に計算した値を記録することを可能するテクニックである。
このテクニックはプログラムのパフォーマンスに大幅な違いを与えることができる。
メモ化された手続は以前の呼出の値がその値を生成した引数をキーとして格納する
テーブルを持つ。メモ化された手続が値を計算するよう命じられた時、最初に値が既に
テーブルにないかチェックを行い、もし存在すれば単にその値を返す。そうでなければ
新しい値を通常の方法で計算しテーブルに保存する。メモ化の例として\link{Section 1.2.2}から
フィボナッチ数を演算するための指数関数処理を思い出せ。

\begin{scheme}
(define (fib n)
  (cond ((= n 0) 0)
        ((= n 1) 1)
        (else (+ (fib (- n 1)) (fib (- n 2))))))
\end{scheme}

同じ手続のメモ化版は以下である。

\begin{scheme}
(define memo-fib
  (memoize
   (lambda (n)
     (cond ((= n 0) 0)
           ((= n 1) 1)
           (else (+ (memo-fib (- n 1))
                    (memo-fib (- n 2))))))))
\end{scheme}

\noindent
この時、memoizeは以下の様に定義される。

\begin{scheme}
(define (memoize f)
  (let ((table (make-table)))
    (lambda (x)
      (let ((previously-computed-result 
             (lookup x table)))
        (or previously-computed-result
            (let ((result (f x)))
              (insert! x result table)
              result))))))
\end{scheme}

\code{(memo\-/fib 3)}の演算を分析するための環境図を描け。なぜ\code{memo\-/fib}が\( n \)番目のフィボナッチ数
を\( n \)に比例するステップ数で演算するのか説明せよ。単に\code{memo\-/fib}を\code{(memoize fib)}と定義した
場合にもSchemeは正しく処理できるだろうか?
\end{quote}

\subsection{デジタル回路のシミュレータ}
\label{Section 3.3.4}

コンピュータのような複雑なデジタルシステムの設計は重要な工学の活動領域です。
デジタルシステムは簡単な要素を相互接続することで構築されます。これらの個々の要素の
振舞は単純ですが、それらのネットワークはとても複雑な振舞をします。
提案された回路設計のコンピュータシミュレーションはデジタルシステムエンジニアにより使用される
重要なツールです。この節ではデジタル論理シミュレーションを実行するためのシステムを
設計します。このシステムは\newterm{event-driven simulation}(\jnewterm{イベント駆動シミュレーション})
と呼ばれる種類の典型であり、その行動(``イベント'')は後に起こるさらなるイベントを
引き起こし、順により多くのイベントを引き起します。



私達の回路の計算モデルは回路を構築する基本となるコンポーネントに対応するオブジェクトに
より成ります。\newterm{digital signals}(\jnewterm{デジタル信号})を運ぶ\newterm{wires}(\jnewterm{回路})が存在します。
デジタル信号は任意の瞬間に可能な2つの値、0と1の内1つを取ります。また多様なタイプのデジタル
\newterm{function boxes}(\jnewterm{関数箱})が存在し、入力信号を運ぶ回路と別の出力回路を接続します。
そのような箱は入力信号から計算された信号を出力します。出力信号は関数箱のタイプにより時間的に
遅れを生じさせます。例えば\newterm{inverter}(\jnewterm{逆変換器})は入力を反転するプリミティブな関数箱です。
もし逆変換器への入力信号が0に変化したなら、ある逆変換器による遅延の後、逆変換器はその出力信号を
1に変更します。もし逆変換器への入力信号が1に変化したならば、ある逆変換器による遅延の後、
逆変換器は出力信号を0にします。逆変換器を記号として\link{Figure 3.24}に示すように描きます。
\link{Figure 3.24}に示される\newterm{and\-/gate}(\jnewterm{ANDゲート})も2つの入力と1つの出力を持つプリミティブな関数箱です。
入力の\newterm{logical and}(\jnewterm{論理積})の値にその出力の値を駆動します。言い替えれば、もし入力信号の
両方が1になればあるANDゲートによる遅延の後にANDゲートはその出力信号を1にします。そうでなければ
出力は0です。\newterm{or\-/gate}(\jnewterm{ORゲート})も同様の2つの入力を持つプリミティブな関数箱であり
その出力信号は入力に対する\newterm{logical or}(\jnewterm{論理和})の値になります。言い替えれば出力はもし少くとも
1つの入力信号が1であれば1になり、そうでなければ出力は0になります。

\begin{figure}[tb]
\phantomsection\label{Figure 3.24}
\centering
\begin{comment}
\heading{Figure 3.24:} Primitive functions in the digital logic simulator.

\begin{example}
               __          ___
  |\        --|  \       --\  \
--| >o--      |   )--       )  >--
  |/        --|__/       --/__/

Inverter    And-gate     Or-gate
\end{example}
\end{comment}
\includegraphics[width=74mm]{fig/chap3/Fig3.24b.pdf}
\par\bigskip
\noindent
\heading{Figure 3.24:} デジタル論理回路シミュレータにおけるプリミティブな関数
\end{figure}

\noindent
プリミティブな関数を一緒に接続してより複雑な関数を構築できます。これを達成するために
ある関数箱の出力から他の関数箱の入力へと回路を引きます。例えば\link{Figure 3.25}に示す
\newterm{half\-/adder}(\jnewterm{半加算器})はORゲート、2つのANDゲート、逆変換器から成り立ちます。
これは2つの入力信号、AとBを取り2つの出力信号SとCがあります。Sは正確にAとBの内1つが1で
あるならば1になり、CはAとBの両方が1の場合に1になります。遅延が生じるため出力が異なる
時刻に生成されることが図から見てわかります。デジタル回路設計の困難の多くはこの事実から
生じます。

\begin{figure}[tb]
\phantomsection\label{Figure 3.25}
\centering
\begin{comment}
\heading{Figure 3.25:} A half-adder circuit.

\begin{example}
    +--------------------------------------+
    |         ____                         |
A --------*---\   \ D               ___    |
    |     |    >   >---------------|   \   |
    |  +--|---/___/                |    )----- S
    |  |  |              |\  E  +--|___/   |
    |  |  |           +--| >o---+          |
    |  |  |    ___    |  |/                |
    |  |  +---|   \   |                    |
    |  |      |    )--*----------------------- C
B -----*------|___/                        |
    |                                      |
    +--------------------------------------+
\end{example}
\end{comment}
\includegraphics[width=72mm]{fig/chap3/Fig3.25c.pdf}
\par\bigskip
\noindent
\heading{Figure 3.25:} 半加算器回路
\end{figure}

\noindent
今から私達が学習を望むデジタル論理回路をモデル化するためのプログラムを構築します。
プログラムは回路をモデル化する計算モデルを構築します。これは信号を``保持''します。
関数箱は信号間の正しい関係を強制する手続によりモデル化されます。

私達のシミュレーションの基本的要素の1つは手続\code{make\-/wire}であり回路を構築します。
例として6つの回路を以下のように構築できます。

\begin{scheme}
(define a (make-wire))
(define b (make-wire))
(define c (make-wire))
(define d (make-wire))
(define e (make-wire))
(define s (make-wire))
\end{scheme}

\noindent
ある関数箱を回路の集合に対してその種類の箱を構築する手続を呼ぶことにより取り付けることができます。
コンストラクタ手続への引数は箱に取り付けられる回路です。例えばANDゲート、ORゲート、逆変換器を構築
できる場合、\link{Figure 3.25}に示す半加算器を配線することができます。

\begin{scheme}
(or-gate a b d)
~\textit{ok}~
(and-gate a b c)
~\textit{ok}~
(inverter c e)
~\textit{ok}~
(and-gate d e s)
~\textit{ok}~
\end{scheme}

\noindent
もっと良いことには、半加算器に取り付けられる4つの外部回路を与えられた時、
この回路を構築する手続\code{half\-/adder}を定義することでこの操作に明示的に名前を
付けることができます。

\begin{scheme}
(define (half-adder a b s c)
  (let ((d (make-wire)) (e (make-wire)))
    (or-gate a b d)
    (and-gate a b c)
    (inverter c e)
    (and-gate d e s)
    'ok))
\end{scheme}

\noindent
この定義を作ることの利点は\code{half\-/adder}それ自身をより複雑な回路を作成する時に
建築用ブロックとして使用することができることです。例えば\link{Figure 3.26}は
2つの半加算器と1つのORゲートより組み立てられる\newterm{full\-/adder}(\jnewterm{全加算器})を
示しています。\footnote{全加算器は2つの二進数の加算に用いられる基本的な回路要素です。
ここでAとBは加算される2つの数の対応する位置のビットで、\( \rm C_{in} \)は
1つ右の加算からのキャリービット(桁上げビット)です。この回路は対応する位置の合計の
ビットであるSUMと左に伝播されるキャリービットである\( \rm C_{out} \)を算出します。}
全加算器を以下のように構築できます。

\begin{scheme}
(define (full-adder a b c-in sum c-out)
  (let ((s (make-wire)) (c1 (make-wire)) (c2 (make-wire)))
    (half-adder b c-in s c1)
    (half-adder a s sum c2)
    (or-gate c1 c2 c-out)
    'ok))
\end{scheme}

\begin{figure}[tb]
\phantomsection\label{Figure 3.26}
\centering
\begin{comment}
\heading{Figure 3.26:} A full-adder circuit.

\begin{example}
    +----------------------------------+
    |              +-------+           |
A -----------------+ half- +-------------- SUM
    |  +-------+   | adder |   ____    |
B -----+ half- +---+       +---\   \   |
    |  | adder |   +-------+    >or >----- Cout
C -----+       +---------------/___/   |
    |  +-------+                       |
    +----------------------------------+
\end{example}
\end{comment}
\includegraphics[width=74mm]{fig/chap3/Fig3.26a.pdf}
\par\bigskip
\noindent
\heading{Figure 3.26:} 全加算器回路
\end{figure}

\noindent
手続として定義された\code{full\-/adder}を持つことでさらにより複雑な回路を作成するための
建築ブロックとして利用することが可能です。(例えば\link{Exercise 3.30}を参照)。


実質的に、私達のシミュレーターは回路の言語を構築するツールを提供します。
もし\link{Section 1.1}におけるLispの学習への取り組みに用いた言語上の一般的な観点を受け入れれば、
プリミティブな関数箱はプリミティブな言語の要素を形成し、箱の間に回路を引くことは組み合わせの
手段を提供し、手続として回線を引くパターンを指定することは抽象化の手段としての
役割を果たすということが言えます。

\subsubsection*{プリミティブな関数箱}




プリミティブな関数箱はある回路上の信号の変化が他の配線上の信号に影響を与える``力''を実装します。
関数箱を構築するため以下の回路上の命令を用います。

\begin{itemize}

\item \code{(get\-/signal}\( \;\;\langle\kern0.06em\hbox{\ttfamily\slshape wire}\kern0.08em\rangle \)\code{)}

\noindent
回線上の信号の現在地を返す

\item \code{(set\-/signal!}\( \;\;\langle\kern0.08em\hbox{\ttfamily\slshape wire}\kern0.08em\rangle\;\;\langle\kern0.08em\hbox{\ttfamily\slshape new value}\kern0.08em\rangle \)\code{)}

\noindent
回路上の信号の値を新しい値に変更する

\item \code{(add\-/action!}\( \;\;\langle\kern0.08em\hbox{\ttfamily\slshape wire}\kern0.08em\rangle\;\;\langle\kern0.08em\hbox{\ttfamily\slshape procedure of no arguments}\kern0.02em\rangle \)\code{)}

\noindent
指定された手続が回路上の信号が値を変化した場合常に実行される様に宣言する。
そのような手続は、回路上の信号の値の変化が他の回路と通信を行うための伝達手段である。
\end{itemize}

\noindent
さらに手続\code{after\-/delay}を使用し遅延時間と実行される手続を取得し、与えられた手続を遅延時間後に
実行します。


これらの手続を用いてプリミティブなデジタル論理関数を定義できます。
入力を逆変換器を通して出力に接続するために\code{add\-/action!}を用いて入力回路と入力回路上
の信号が値を変化する度に実行される手続を関連付けます。その手続は入力信号の
\code{logical\-/not}(論理否定)を計算し、そして\code{inverter\-/delay}後に出力信号に
この新しい値を設定します。

\begin{scheme}
(define (inverter input output)
  (define (invert-input)
    (let ((new-value (logical-not (get-signal input))))
      (after-delay inverter-delay
                   (lambda ()
                     (set-signal! output new-value)))))
  (add-action! input invert-input) 'ok)
(define (logical-not s)
  (cond ((= s 0) 1)
        ((= s 1) 0)
        (else (error "Invalid signal" s))))
\end{scheme}

\noindent
ANDゲートはより少しだけ複雑です。アクション手続はゲートへの入力のどちらかが
変化した場合に実行されねばなりません。それが入力回路上の信号の値の\code{logical\-/and}(論理積)を
(\code{logical\-/not}と類似の手続を用いて)求め、出力回路上に起こる新しい値への変更を
\code{and\-/gate\-/delay}後に設定します。

\begin{scheme}
(define (and-gate a1 a2 output)
  (define (and-action-procedure)
    (let ((new-value
           (logical-and (get-signal a1) (get-signal a2))))
      (after-delay 
       and-gate-delay
       (lambda () (set-signal! output new-value)))))
  (add-action! a1 and-action-procedure)
  (add-action! a2 and-action-procedure)
  'ok)
\end{scheme}

\begin{quote}
\heading{\phantomsection\label{Exercise 3.28}Exercise 3.28:} 
ORゲートをプリミティブな関数箱として定義せよ。あなたの\code{or\-/gate}コンストラクタは
\code{and\-/gate}と同様でなければならない。
\end{quote}

\begin{quote}
\heading{\phantomsection\label{Exercise 3.29}Exercise 3.29:} 
ORゲートを構築する別の方法は複合デジタル論理デバイスとしてANDゲートと逆変換器から
構築するものである。これを達成する手続\code{or\-/gate}を定義せよ。\code{and\-/gate\-/delay}と
\code{inverter\-/delay}を用いた遅延時間はどのようになるか?
\end{quote}

\begin{quote}
\heading{\phantomsection\label{Exercise 3.30}Exercise 3.30:} 
\link{Figure 3.27}は\( n \)個の全加算器を繋げた\newterm{ripple-carry adder}(\jnewterm{桁上げ伝播加算器})を示している。
これは2つの\( n \)ビット二進数を足すための最も簡単な形式の並列加算器である。
入力\( A_1 \), \( A_2 \), \( A_3 \), \( \dots \), \( A_n \)と\( B_1 \), \( B_2 \), \( B_3 \), \( \dots \), \( B_n \)
は足すべき2つの二進数(各\( A_k \)と\( B_k \)は0か1)である。
回路は\( S_1 \), \( S_2 \), \( S_3 \), \( \dots \), \( S_n \)の\( n \)ビットの和と、
和算の桁上がりである\( C \)を生成する。
この回路を生成する手続\code{ripple\-/carry\-/adder}を書け。この手続は引数としてそれぞれ
\( n \)個の配線を持つ3つのリスト---\( A_k \), \( B_k \), \( S_k \)---と別の配線\( C \)を取る。
桁上げ伝播加算器の主な欠点はキャリー信号の伝播を待つ必要があることである。\( n \)ビットの桁上げ伝播加算器
における完全な出力を得るのに必要な遅延時間はいくらか? ANDゲート、ORゲート、逆変換器の遅延時間から表現せよ。
\end{quote}

\begin{figure}[tb]
\phantomsection\label{Figure 3.27}
\centering
\begin{comment}
\heading{Figure 3.27:} A ripple-carry adder for \( n \)-bit numbers.

\begin{example}
   :                                              :   :
   : A_1 B_1   C_1   A_2 B_2   C_2   A_3 B_3   C_3:   : A_n B_n C_n=0
   :  |   |   +---+   |   |   +---+   |   |   +-----  :  |   |   +-
   |  |   |   |   |   |   |   |   |   |   |   |   :   :  |   |   | 
   : ++---+---++  |  ++---+---++  |  ++---+---++  :   : ++---+---++
   : |   FA    |  |  |   FA    |  |  |   FA    |  :   : |   FA    |
   : +--+---+--+  |  +--+---+--+  |  +--+---+--+  :   : +--+---+--+
   :    |   |     |     |   |     |     |   |     :   :    |   |   
C ------+   |     +-----+   |     +-----+   |     :  ------+   |   
   :        |       C_1     |       C_2     |     :   :C_(n-1) |   
   :        |               |               |     :   :        |   
           S_1             S_2             S_3                S_n
\end{example}
\end{comment}
\includegraphics[width=96mm]{fig/chap3/Fig3.27a.pdf}
\par\bigskip
\noindent
\heading{Figure 3.27:} \( n \)-bit数の桁上げ伝播加算器
\end{figure}

\subsubsection*{回路の表現}



私達のシミュレーションにおけるワイヤ(wire、配線、回路)は2つのローカルな状態変数を持つ
計算オブジェクトになります。その2つは\code{signal\-/value}(信号値)(初期値は0)と信号が値を変えた時に実行される
\code{action\-/procedures}(行動手続)の集合です。
メッセージパッシングスタイルを用いてワイヤを局所手続の集合として適切な局所命令を選択する手続\code{dispatch}と
共に実装します。\link{Section 3.1.1}での簡単な銀行口座オブジェクトと同様に行います。

\begin{scheme}
(define (make-wire)
  (let ((signal-value 0) (action-procedures '()))
    (define (set-my-signal! new-value)
      (if (not (= signal-value new-value))
          (begin (set! signal-value new-value)
                 (call-each action-procedures))
          'done))
    (define (accept-action-procedure! proc)
      (set! action-procedures 
            (cons proc action-procedures))
      (proc))
    (define (dispatch m)
      (cond ((eq? m 'get-signal) signal-value)
            ((eq? m 'set-signal!) set-my-signal!)
            ((eq? m 'add-action!) accept-action-procedure!)
            (else (error "Unknown operation: WIRE" m))))
    dispatch))
\end{scheme}

\noindent
局所手続\code{set\-/my\-/signal!}は新しい信号値が配線上の信号を変えるかチェックします。
もしそうであれば全ての行動手続を以下の手続\code{call\-/each}を用いて実行します。
\code{call\-/each}は引数無し手続のリスト内の全てのアイテムを呼び出します。

\begin{scheme}
(define (call-each procedures)
  (if (null? procedures)
      'done
      (begin ((car procedures))
             (call-each (cdr procedures)))))
\end{scheme}

\noindent
局所手続\code{accept\-/action\-/procedure!}は与えられた手続を実行対象手続リストに追加します。
次に新しい手続を一度実行します。(\link{Exercise 3.31}参照)



ローカルの\code{dispatch}手続が指定通りに設定されていることから、以下の手続を与えて
配線上の局所命令にアクセスすることができます。\footnote{\label{Footnote 27}これらの
手続は単純にオブジェクトの局所手続にアクセスするために通常の手続的な文法を使用する
ことを許可する糖衣構文に過ぎません。``手続''と``データ''の役割をそのような簡単な方法で
交換できることは印象的です。例えばもし\code{(wire 'get\-/signal)}と書いた場合、私達は
\code{wire}をメッセージ\code{get\-/signal}を入力として呼び出される手続だと考えるでしょう。
その代わりに\code{(get\-/signal wire)}と書くことは私達に\code{wire}を手続\code{get\-/signal}に対する
入力としてのデータオブジェクトだと考えることを促します。この問題の真実は私達が手続を
オブジェクトとして扱う言語には``手続''と``データ''の間に基本的な違いが存在せず、
私達はどんなスタイルを選択してもプログラミングを可能にする糖衣構文を選択することができる
ということです。}

\begin{scheme}
(define (get-signal wire) (wire 'get-signal))
(define (set-signal! wire new-value)
  ((wire 'set-signal!) new-value))
(define (add-action! wire action-procedure)
  ((wire 'add-action!) action-procedure))
\end{scheme}

\noindent
時間的に変化する信号を持ち付加的に装置に取り付けられる配線はミュータブルなオブジェクトの
特性を良く示しています。私達はそれを代入により変化するローカル状態変数を持つ手続として
モデル化しました。新しい配線が作成された時、新しい状態変数の信号は(\code{make\-/wire}中の\code{let}式により)
確保され、新しい\code{dispatch}手続が構築され返され、新しい状態変数を持つ環境が確保されます。



配線は様々なデバイスの間で共有され、それらに対して接続されます。従ってあるデバイスとの応答により
起こった変化はその配線に取り付けられた全ての他のデバイスに影響を与えます。配線は
接続が開設された時に提供された行動手続を呼ぶことによりその近傍に対し変化を通知します。

\subsubsection*{予定表}



シミュレータを完成させるために必要な物は\code{after\-/delay}のみです。ここでのアイデアは
\newterm{agenda}(\jnewterm{予定表})と呼ばれるデータ構造を保持し、それに行うべき予定を保存します。
以下の命令は予定表のために定義されます。

\begin{itemize}

\item
\code{(make\-/agenda)}は新しい空の予定表を返す。

\item
\code{(empty\-/agenda?}\( \;\;\langle\kern0.08em\hbox{\ttfamily\slshape agenda}\kern0.06em\rangle\hbox{\tt)} \)は
指定した予定表が空であるなら真である。

\item
\code{(first\-/agenda\-/item}\( \;\;\langle\kern0.08em\hbox{\ttfamily\slshape agenda}\kern0.06em\rangle\hbox{\tt)} \)は
予定表の最初のアイテムを返す。

\item
\code{(remove\-/first\-/agenda\-/item!}\( \;\langle\kern0.08em\hbox{\ttfamily\slshape agenda}\kern0.06em\rangle\hbox{\tt)} \) は
予定表から最初のアイテムを削除する。

\item
\code{(add\-/to\-/agenda!}\( \;\;\langle\kern0.03em\hbox{\ttfamily\slshape time}\kern0.06em\rangle\;\;\langle\kern0.08em\hbox{\ttfamily\slshape action}\kern0.06em\rangle\;\;\langle\kern0.08em\hbox{\ttfamily\slshape agenda}\kern0.06em\rangle\hbox{\tt)} \)は指定された時間後に実行される行動手続を追加する。

\item
\code{(current\-/time}\( \;\;\langle\kern0.08em\hbox{\ttfamily\slshape agenda}\kern0.04em\rangle\hbox{\tt)} \)は現在のシミュレーション時間を返す。
\end{itemize}

\noindent
使用する予定表は\code{the\-/agenda}により指定されます。手続\code{after\-/delay}は
新しい要素を\code{the\-/agenda}に追加します。

\begin{scheme}
(define (after-delay delay action)
  (add-to-agenda! (+ delay (current-time the-agenda))
                  action
                  the-agenda))
\end{scheme}

\noindent
シミュレーションは手続\code{propagate}(伝播)により駆動され、\code{the\-/agenda}上で操作を行い、
予定表上の各手続を順に実行します。一般的にシミュレータが実行されるにつれ、新しいアイテムが
予定表に追加され、\code{propagate}はシミュレーションを予定表にアイテムが存在する間は続けます。

\begin{scheme}
(define (propagate)
  (if (empty-agenda? the-agenda)
      'done
      (let ((first-item (first-agenda-item the-agenda)))
        (first-item)
        (remove-first-agenda-item! the-agenda)
        (propagate))))
\end{scheme}

\subsubsection*{サンプルシミュレーション}



回路上に``probe''(プローブ、探針)を置く以下の手続は実行中のシミュレータを表示します。
プローブは配線に対し信号値が変わる度に新しい信号値を現在に時刻と配線を識別する名前を一緒に表示せよと命じます。

\begin{scheme}
(define (probe name wire)
  (add-action! wire
               (lambda ()
                 (newline)
                 (display name)
                 (display " ")
                 (display (current-time the-agenda))
                 (display "  New-value = ")
                 (display (get-signal wire)))))
\end{scheme}

\noindent
予定表の初期化とプリミティブな関数箱に対し遅延時間を指定することから始めます。

\begin{scheme}
(define the-agenda (make-agenda))
(define inverter-delay 2)
(define and-gate-delay 3)
(define or-gate-delay 5)
\end{scheme}

\noindent
ここで4つの配線を定義し、その内2つにプローブを仕込みます。

\begin{scheme}
(define input-1 (make-wire))
(define input-2 (make-wire))
(define sum (make-wire))
(define carry (make-wire))

(probe 'sum sum)
~\textit{sum 0  New-value = 0}~
(probe 'carry carry)
~\textit{carry 0  New-value = 0}~
\end{scheme}

\noindent
次に配線を(\link{Figure 3.25}の様に)半加算器回路に接続し、\code{input\-/1}上の信号を
1に設定し、シミュレーションを実行します。

\begin{scheme}
(half-adder input-1 input-2 sum carry)
~\textit{ok}~
\end{scheme}

\begin{scheme}
(set-signal! input-1 1)
~\textit{done}~
\end{scheme}

\begin{scheme}
(propagate)
~\textit{sum 8  New-value = 1}~
~\textit{done}~
\end{scheme}

\noindent
\code{sum}の進行は時刻8において1に変化しました。シミュレーションの開始から8単位時間が経過しました。
この時点で\code{input\-/2}上の信号を1に設定し値の伝播を許可します。

\begin{scheme}
(set-signal! input-2 1)
~\textit{done}~
\end{scheme}

\begin{scheme}
(propagate)
~\textit{carry 11  New-value = 1}~
~\textit{sum 16  New-value = 0}~
~\textit{done}~
\end{scheme}

\noindent
\code{carry}は時刻11にて1に変化し、\code{sum}は時刻16において0に変化しました。

\begin{quote}
\heading{\phantomsection\label{Exercise 3.31}Exercise 3.31:} 
\code{make\-/wire}内で定義された内部手続\code{accept\-/action\-/procedure!}は新しい行動手続が配線に
追加された時に、その手続が即座に実行された。この初期化がなぜ必要であるのか説明せよ。
具体的には、上の段落の半加算器の例をトレースし、システムの応答が、\code{accept\-/action\-/procedure!}が
以下のように定義されていた場合にどのように異なるかについて述べよ。

\begin{scheme}
(define (accept-action-procedure! proc)
  (set! action-procedures 
        (cons proc action-procedures)))
\end{scheme}
\end{quote}

\subsubsection*{予定表の実装}



最後に将来に実行される予定の手続を保存する予定表データ構造の詳細について説明します。


予定表は\newterm{time segments}(\jnewterm{タイムセグメント}、時間区分)により構成されています。
各タイムセグメントは数値(時刻)と、
そのタイムセグメントの間に実行されるよう予定された手続を持つキュー (\link{Exercise 3.32}参照)から
成るペアです。

\begin{scheme}
(define (make-time-segment time queue)
  (cons time queue))
(define (segment-time s) (car s))
(define (segment-queue s) (cdr s))
\end{scheme}

\noindent
タイムセグメントのキューは\link{Section 3.3.2}.で説明したキューの命令を用いて操作します。



予定表自身は1次元のタイムセグメントの表です。\link{Section 3.3.3}で説明された表との違いは
セグメントが時間の増す順にソートされることです。加えて\newterm{current time}(\jnewterm{現在時刻})
(言い換えると最後に処理された行動の時刻)を予定表の頭に保存します。新しく構築された予定表は
タイムセグメントを持っておらず現在時刻として0を持ちます。\footnote{予定表は\link{Section 3.3.3}のような
頭出しリストですが、このリストは時刻による頭出しですので追加のダミーヘッダ(テーブルにて
用いられた\code{*table*}シンボルのような物)を必要としません。}

\begin{scheme}
(define (make-agenda) (list 0))
(define (current-time agenda) (car agenda))
(define (set-current-time! agenda time)
  (set-car! agenda time))
(define (segments agenda) (cdr agenda))
(define (set-segments! agenda segments)
  (set-cdr! agenda segments))
(define (first-segment agenda) 
  (car (segments agenda)))
(define (rest-segments agenda) 
  (cdr (segments agenda)))
\end{scheme}

\noindent
予定表はタイムセグメントを持っていなければ空です。

\begin{scheme}
(define (empty-agenda? agenda)
  (null? (segments agenda)))
\end{scheme}

\noindent
予定表に行動(アクション)を追加するために、最初に予定表が空であるか確認します。
もしそうならばアクションのためのタイムセグメントを作成し、それを予定表にインストールします。
そうでなければ予定表を走査し、各セグメントの時刻を調べます。もし指定時刻が存在するならば
対応するキューにアクションを追加します。もし指定時刻よりも後の時間に辿り着いたならば、
新しいタイムセグメントを予定表のその時間の前に挿入します。もし予定表の最後まで辿り着いたならば
新しいタイムセグメントを最後に作らねばなりません。

\begin{scheme}
(define (add-to-agenda! time action agenda)
  (define (belongs-before? segments)
    (or (null? segments)
        (< time (segment-time (car segments)))))
  (define (make-new-time-segment time action)
    (let ((q (make-queue)))
      (insert-queue! q action)
      (make-time-segment time q)))
  (define (add-to-segments! segments)
    (if (= (segment-time (car segments)) time)
        (insert-queue! (segment-queue (car segments))
                       action)
        (let ((rest (cdr segments)))
          (if (belongs-before? rest)
              (set-cdr!
               segments
               (cons (make-new-time-segment time action)
                     (cdr segments)))
              (add-to-segments! rest)))))
  (let ((segments (segments agenda)))
    (if (belongs-before? segments)
        (set-segments!
         agenda
         (cons (make-new-time-segment time action)
               segments))
        (add-to-segments! segments))))
\end{scheme}

\noindent
予定表から最初のアイテムを削除する手続は最初のタイムセグメント中のキューの先頭の
アイテムを削除します。もしこの削除がタイムセグメントを空にするのであれば、
セグメントのリストからそれを削除します。\footnote{この手続の中の\code{if}式が
\( \langle \)\var{alternative}\( \kern0.08em\rangle \)式を持っていないことに
注意して下さい。このような``片腕の\code{if}文''は2つの式の間から選択するのではなく
何かをするかどうかを決定するのに使用されます。\code{if}式は述語が偽になった場合に
未定義の値を返し、\( \langle \)\var{alternative}\( \kern0.08em\rangle \)は有りません。}

\begin{scheme}
(define (remove-first-agenda-item! agenda)
  (let ((q (segment-queue (first-segment agenda))))
    (delete-queue! q)
    (if (empty-queue? q)
        (set-segments! agenda (rest-segments agenda)))))
\end{scheme}

\noindent
最初の予定表のアイテムは最初のタイムセグメントのキューの頭に見つかります。
アイテムを抽出する度に現在時刻の更新も行います。\footnote{このようにして、
現在時刻は常に最も最近に処理されたアクションの時刻になります。この時刻を
予定表の頭に格納することで例え関連するタイムセグメントが削除されても依然として
有効であることを確約します。}

\begin{scheme}
(define (first-agenda-item agenda)
  (if (empty-agenda? agenda)
      (error "Agenda is empty: FIRST-AGENDA-ITEM")
      (let ((first-seg (first-segment agenda)))
        (set-current-time! agenda 
                           (segment-time first-seg))
        (front-queue (segment-queue first-seg)))))
\end{scheme}

\begin{quote}
\heading{\phantomsection\label{Exercise 3.32}Exercise 3.32:} 
予定表の各タイムセグメントの間に実行される手続はキューに保存される。従って各セグメントの
手続は予定表に追加された順に呼び出される(FIFO)。なぜこの順が使用されるべきか説明せよ。
具体的には入力が0,1から1,0に同じセグメントにて変化した時のANDゲートの振舞をトレースし、
もしセグメントの手続を通常の順に格納し、手続の追加と削除を先頭でのみ行った場合(LIFO)に
振舞がどのように異なるかについて述べよ。
\end{quote}


\subsection{制約伝播}
\label{Section 3.3.5}



コンピュータプログラムは伝統的に一方向の演算として体系化されます。
これは事前に指定した引数上で命令を実行し、望んだ出力を生成します。
一方で私達は時折、量の間の関係を用いてシステムをモデル化します。
例えば機械構造の数理的モデルは金属棒の偏差\( d \)が棒上の力\( F \)、棒の長さ\( L \)、
断面積\( A \)、弾性率\( E \)に方程式を通して関連するという情報を含むでしょう。

\begin{comment}

\begin{example}
dAE = FL
\end{example}

\end{comment}
\begin{displaymath}
 dAE = FL. 
\end{displaymath}


そのような方程式は一方向ではありません。任意の4つの量を与えられることで、5つ目を
計算することができます。けれども方程式を伝統的なコンピュータ言語へと翻訳することは
1つの量を選択し他の4つを用いて求めることを私達は強制されます。従って断面積\( A \)を
求める手続は偏差\( d \)を求めることには、例え\( A \)と\( d \)の演算が同じ方程式から
起こっても使用できません。\footnote{制約伝播は最初に信じられない程先進的であった
Ivan \link{Sutherland (1963)}による\acronym{SKETCHPAD}システムに現れました。
Smalltalkをベースにした美しい制約伝播システムはAlan \link{Borning (1977)}により
Xeroxパロアルト研究センタにて開発されました。Sussman, Stallman, Steeleの3人は
制約伝播を電子回路分析に応用しました(\link{Sussman and Stallman 1975}; \link{Sussman and Steele 1980})。
TK!Solver(\link{Konopasek and Jayaraman 1984})は制約をベースにした大規模モデリング環境です。}



この節では関係性自身を用いて働くことが可能な言語の設計を描きます。言語のプリミティブな要素は
\newterm{primitive constraints}(\jnewterm{プリミティブ制約})であり、幾らかの関係性が数量の間に保存される
ことを示します。例えば\( c \)は方程式\( a + b = c \)から参照されねばならず、\code{(multiplier x y z)}
は制約\( xy = z \)を表し、\code{(constant 3.14 x)}は\( x \)の値が3.14に違いないと述べています。



私達の言語はプリミティブ制約をより複雑な関係を表明するために接続する手段を提供します。
制約を\newterm{constraint networks}(\jnewterm{制約ネットワーク})を構築することで接続し、
その中で制約は\newterm{connectors}(\jnewterm{コネクタ})を用いて結合されます。コネクタは値を持つ
オブジェクトであり、1つ以上の制約に加わります。例えば華氏と摂氏の気温の間の関係が
以下であることを知っています。
\begin{comment}

\begin{example}
9C = 5(F - 32)
\end{example}

\end{comment}
\begin{displaymath}
 9C = 5(F - 32). 
\end{displaymath}

そのような制約はプリミティブな加算器、乗算器、不変制約(\link{Figure 3.28})より成り立つ
ネットワークとして考えることができます。図の中で左手に\( m \)1, \( m \)2, \( p \)の
3つの端子を持つ乗算の箱を見ることができます。これらは乗算器を以下のネットワークの残りに
接続します。\( m \)1端子は摂氏の気温を保持するコネクタ\( C \)にリンクされます。\( m \)2端子も
9を持つ整数箱にリンクされます。乗算器の箱が\( m \)1と\( m \)2の積に制約を行う\( p \)端子は
別の乗算器の箱の\( p \)端子に接続され、その箱の\( m \)2は整数5に、\( m \)1は合計の1つの
端子に接続されます。

\begin{figure}[tb]
\phantomsection\label{Figure 3.28}
\centering
\begin{comment}
\begin{quote}
\heading{Figure 3.28:} The relation \( 9C = 5(F - 32) \) expressed as a constraint network.

\begin{example}
       +---------+     +---------+   v   +---------+
C -----+ m1      |  u  |      m1 +-------+ a1      |
       |    *  p +-----+ p  *    |       |    *  s +---- F
    +--+ m2      |     |      m2 +--+ +--+ a2      |
    |  +---------+     +---------+  | |  +---------+
  w |                              x| |y
    |    +-----+        +-----+     | |     +-----+
    +----+  9  |        |  5  +-----+ +-----+  32 |
         +-----+        +-----+             +-----+
\end{example}
\end{quote}
\end{comment}
\includegraphics[width=87mm]{fig/chap3/Fig3.28.pdf}
\begin{quote}
\heading{Figure 3.28:} 制約ネットワークとして表した関係\\
\( 9C = 5(F - 32) \)
\end{quote}
\end{figure}

\noindent
このようなネットワークによる計算は以下の様に進行されます。コネクタに値が(ユーザ、またはリンクされた制約箱により)
与えられた時、その関連する制約全てを(それを起こした制約を除いて)起こし、それらに値を得たことを伝えます。
起きた制約箱は全て次にコネクタに対しコネクタの値を決定するのに十分な情報が存在するかを調査(poll)します。
もしそうであれば、制約箱はコネクタに値を設定し、コネクタはすると関係する制約を全て起こします。これが
繰り返されます。例として摂氏と華氏の間の換算では\( w \), \( x \), \( y \)は整数箱9, 5, 32それぞれにより
直ぐに設定されます。コネクタは乗算器と加算器を起動し、それらは続行に必要な情報が十分ではないことを判断します。
もしユーザ(またはネットワークの何らかの他の部分が)\( C \)に値(例えば25)を設定すると最も左の乗算器が起動され、
\( u \)に\( 25 \cdot 9 = 225 \)を設定します。すると\( u \)が2つ目の乗算器を起動し、それが\( v \)に
45を設定します。そして\( v \)が加算器を起動し、加算器は\( f \)を77に設定します。

\subsubsection*{制約システムの利用}



制約システムを用いて上で説明された気温の計算を実行するには最初に2つのコネクタ、\code{C}と\code{F}を
コンストラクタ\code{make\-/connector}を呼ぶことで作成し、\code{C}と\code{F}をあるべきネットワークにリンクします。

\begin{scheme}
(define C (make-connector))
(define F (make-connector))
(celsius-fahrenheit-converter C F)
~\textit{ok}~
\end{scheme}

\noindent
ネットワークを作成する手続は以下のように定義されます。

\begin{scheme}
(define (celsius-fahrenheit-converter c f)
  (let ((u (make-connector))
        (v (make-connector))
        (w (make-connector))
        (x (make-connector))
        (y (make-connector)))
    (multiplier c w u)
    (multiplier v x u)
    (adder v y f)
    (constant 9 w)
    (constant 5 x)
    (constant 32 y)
    'ok))
\end{scheme}

\noindent
この手続は内部コネクタ\code{u}, \code{v}, \code{w}, \code{x}, \code{y}を作成し、それらを\link{Figure 3.28}
に示されるようにプリミティブな制約コンストラクタ\code{adder}, \code{multiplier}, \code{constant}を
用いてリンクします。



実行中のネットワークを見るために、コネクタ\code{C}と\code{F}にプローブ(探針)を\link{Section 3.3.4}で配線の監視に
用いた物と同様な\code{probe}手続を用いて設置します。プローブのコネクタ上への設置はコネクタに値が
与えられる度にメッセージが表示されるようにします。

\begin{scheme}
(probe "Celsius temp" C)
(probe "Fahrenheit temp" F)
\end{scheme}

\noindent
次に\code{C}の値を25に設定します。(\code{set\-/value!}への3つ目の引数は\code{C}にこの指示が\code{user}による物
であることを伝えています)。

\begin{scheme}
(set-value! C 25 'user)
~\textit{Probe: Celsius temp = 25}~
~\textit{Probe: Fahrenheit temp = 77}~
~\textit{done}~
\end{scheme}

\noindent
\code{C}上のプローブが起動され値を報告します。\code{C}はまたその値を上で説明されたネットワークを通して
伝播させます。これが\code{F}に77を設定し、\code{F}上のプローブにより報告されます。


ここで\code{F}に新しい値、例えば212を設定してみましょう。

\begin{scheme}
(set-value! F 212 'user)
~\textit{Error! Contradiction (77 212)}~
\end{scheme}

\noindent
コネクタが矛盾に気付いたと訴えています。その値は77の時、誰かが212を設定しようとしているのです。
もし本当にネットワークを新しい値にて再利用したいのであれば\code{C}に古い値を忘れるように指示できます。

\begin{scheme}
(forget-value! C 'user)
~\textit{Probe: Celsius temp = ?}~
~\textit{Probe: Fahrenheit temp = ?}~
~\textit{done}~
\end{scheme}

\noindent
\code{C}は元の値を設定した\code{user}が今撤回しているのに気付き、\code{C}はその値をなくすことに
プローブが示すように同意し、ネットワークの残りにこの結果について伝えます。
この情報が結果的に\code{F}に伝播し、\code{F}は今となってはそれ自身の値が77であると信じ続けるための
理由が無いことに気付きます。従って\code{F}もまたその値を諦めプローブにより表示されます。


これで\code{F}は値を持たず、私達は\code{F}に212を設定できます。

\begin{scheme}
(set-value! F 212 'user)
~\textit{Probe: Fahrenheit temp = 212}~
~\textit{Probe: Celsius temp = 100}~
~\textit{done}~
\end{scheme}

\noindent
この新しい値がネットワーク中に伝播された時、\code{C}に100の値を持つことを強制し、
\code{C}上のプローブによりこのことが表されます。全く同じネットワークが\code{F}を
与えて\code{C}を計算するのと、\code{C}を与えて\code{F}を計算することに用いられている
ことに注意して下さい。この方向性の無い演算が制約ベースシステムの特徴的な
機能です。

\subsubsection*{制約システムの実装}



制約システムは局所状態を持つ手続き型のオブジェクトにより、\link{Section 3.3.4}のデジタル回路シミュレータに
良く似た作法で実装されます。制約システムのプリミティブなオブジェクトはいくらかより複雑ではあるものの、
システム全体は予定表や論理遅延時間についての考慮が不要な分、よりシンプルです。


コネクタ上の基本的な命令は次のとおりです。

\begin{itemize}

\item
\code{(has\-/value? <\var{connector}>)}はコネクタが値を持つかどうか判断する

\item
\code{(get\-/value <\var{connector}>)}はコネクタの現在地を返す

\item
\code{(set\-/value! <\var{connector}> <\var{new\-/value}> <\var{informant}>)}は
情報がコネクタに対しその値を新しい値に設定するよう要求することを示す

\item
\code{(forget\-/value! <\var{connector}> <\var{retractor}>)}はコネクタに対し撤回を望む者が値を忘れることを
要求していると伝える

\item
\code{(connect <\var{connector}> <\var{new\-/constraint}>)}はコネクタに対し新しい制約への参加を指示する

\end{itemize}

\noindent
コネクタは与えられた制約にコネクタが値を持っていると伝える手続\code{inform\-/about\-/value}と
制約にコネクタが値を失ったと伝える手続\code{inform\-/about\-/no\-/value}を用いて制約と通信を行います。


\code{adder}は加数コネクタ\code{a1}と\code{a2}と\code{sum}コネクタの間に加算器制約を構築するコンストラクタです。
加算器は局所状態を持つ手続(下記の手続\code{me})として実装されます。

\begin{scheme}
(define (adder a1 a2 sum)
  (define (process-new-value)
    (cond ((and (has-value? a1) (has-value? a2))
           (set-value! sum
                       (+ (get-value a1) (get-value a2))
                       me))
          ((and (has-value? a1) (has-value? sum))
           (set-value! a2
                       (- (get-value sum) (get-value a1))
                       me))
          ((and (has-value? a2) (has-value? sum))
           (set-value! a1
                       (- (get-value sum) (get-value a2))
                       me))))
  (define (process-forget-value)
    (forget-value! sum me)
    (forget-value! a1 me)
    (forget-value! a2 me)
    (process-new-value))
  (define (me request)
    (cond ((eq? request 'I-have-a-value)
           (process-new-value))
          ((eq? request 'I-lost-my-value)
           (process-forget-value))
          (else (error "Unknown request: ADDER" request))))
  (connect a1 me)
  (connect a2 me)
  (connect sum me)
  me)
\end{scheme}

\noindent
\code{adder}は新しい加算器を指定されたコネクタに接続し自身をその値として返します。
手続\code{me}は加算器を表現し、ローカル手続を起動する者の役割を果たします。

\begin{scheme}
(define (inform-about-value constraint)
  (constraint 'I-have-a-value))
(define (inform-about-no-value constraint)
  (constraint 'I-lost-my-value))
\end{scheme}

\noindent
加算器のローカル手続\code{process\-/new\-/value}はその加算器が繋るコネクタの内1つが値を得た事を報された時に
呼び出されます。加算器は最初に\code{a1}と\code{a2}の両方が値を持っているか確認します。
もしそうならば\code{sum}に2つの加数の和をその値として設定するように指示します。
\code{set\-/value!}の\code{informant}(情報提供者)引数は加算器オブジェクト自身である\code{me}です。
もし\code{a1}と\code{a2}の両方が値を持っていない場合、加算器はひょっとしたら\code{a1}と\code{sum}が
値を持っていないか確認します。もしそうならば\code{a2}にその2つの差を設定します。最後に
\code{a2}と\code{sum}が値を持っているのならば加算器に\code{a1}を接待させるために十分な情報を
持っていることになります。もし加算器がコネクタの1つが値を失なったと報された場合、全ての
コネクタに対しその値を捨てるよう指示します。(この加算器により設定された値のみが
実際には失なわれます)。次に加算器は\code{process\-/new\-/value}を実行します。この理由は
1つ、またはそれ以上のコネクタが依然として値を持っている可能性があり(つまり、コネクタが
元々その加算器により設定されたのではない値を持っている)、これらの値は加算器を通して
伝播し返す必要があります。


乗算器は加算器にとても良く似ています。因数のどちらかが0なら例え他方の値がわからなくても
\code{product}を0にします。

\begin{scheme}
(define (multiplier m1 m2 product)
  (define (process-new-value)
    (cond ((or (and (has-value? m1) (= (get-value m1) 0))
               (and (has-value? m2) (= (get-value m2) 0)))
           (set-value! product 0 me))
          ((and (has-value? m1) (has-value? m2))
           (set-value! product
                       (* (get-value m1) (get-value m2))
                       me))
          ((and (has-value? product) (has-value? m1))
           (set-value! m2
                       (/ (get-value product) 
                          (get-value m1))
                       me))
          ((and (has-value? product) (has-value? m2))
           (set-value! m1
                       (/ (get-value product) 
                          (get-value m2))
                       me))))
  (define (process-forget-value)
    (forget-value! product me)
    (forget-value! m1 me)
    (forget-value! m2 me)
    (process-new-value))
  (define (me request)
    (cond ((eq? request 'I-have-a-value)
           (process-new-value))
          ((eq? request 'I-lost-my-value)
           (process-forget-value))
          (else (error "Unknown request: 
                        MULTIPLIER" request))))
  (connect m1 me)
  (connect m2 me)
  (connect product me)
  me)
\end{scheme}

\noindent
\code{constant}コンストラクタは単純に指定されたコネクタの値を設定します。
\code{I\-/have\-/a\-/value}と\code{I\-/lost\-/my\-/value}のどちらのメッセージが定数箱に送られてもエラーを発します。

\begin{scheme}
(define (constant value connector)
  (define (me request)
    (error "Unknown request: CONSTANT" request))
  (connect connector me)
  (set-value! connector value me)
  me)
\end{scheme}

\noindent
最後にプローブは指定されたコネクタの設定、設定解除のメッセージを表示します。

\begin{scheme}
(define (probe name connector)
  (define (print-probe value)
    (newline) (display "Probe: ") (display name)
    (display " = ") (display value))
  (define (process-new-value)
    (print-probe (get-value connector)))
  (define (process-forget-value) (print-probe "?"))
  (define (me request)
    (cond ((eq? request 'I-have-a-value)
           (process-new-value))
          ((eq? request 'I-lost-my-value)
           (process-forget-value))
          (else (error "Unknown request: PROBE" request))))
  (connect connector me)
  me)
\end{scheme}

\subsubsection*{コネクタの表現}


コネクタは局所状態変数を持つ手続型のオブジェクトとして表現され、
\code{value}はコネクタの現在地、\code{informant}はコネクタの値を設定したオブジェクト、
そして\code{constraints}はコネクタが参加する制約のリストです。


\begin{scheme}
(define (make-connector)
  (let ((value false) (informant false) (constraints '()))
    (define (set-my-value newval setter)
      (cond ((not (has-value? me))
             (set! value newval)
             (set! informant setter)
             (for-each-except setter
                              inform-about-value
                              constraints))
            ((not (= value newval))
             (error "Contradiction" (list value newval)))
            (else 'ignored)))
    (define (forget-my-value retractor)
      (if (eq? retractor informant)
          (begin (set! informant false)
                 (for-each-except retractor
                                  inform-about-no-value
                                  constraints))
          'ignored))
    (define (connect new-constraint)
      (if (not (memq new-constraint constraints))
          (set! constraints
                (cons new-constraint constraints)))
      (if (has-value? me)
          (inform-about-value new-constraint))
      'done)
    (define (me request)
      (cond ((eq? request 'has-value?)
             (if informant true false))
            ((eq? request 'value) value)
            ((eq? request 'set-value!) set-my-value)
            ((eq? request 'forget) forget-my-value)
            ((eq? request 'connect) connect)
            (else (error "Unknown operation: CONNECTOR"
                         request))))
    me))
\end{scheme}

\noindent
コネクタの局所手続\code{set\-/my\-/value}はコネクタの値を設定する要求が存在した時に呼ばれます。
もしコネクタが現在値を持っていない場合、その値を設定し、値の設定を要求した制約を\code{informant}として
記憶します。\footnote{\code{setter}は制約ではないかもしれません。気温の例では\code{user}を
\code{setter}として使用しました。}次にコネクタは参加している制約全てに対し値の設定を要求した
制約を除いて通知します。これは以下のiterator(イテレータ、繰り返す者)を用いて達成されます。
イテレータは指定された手続を与えられた1つを除いたリスト中の全てのアイテムに対して適用します。

\begin{scheme}
(define (for-each-except exception procedure list)
  (define (loop items)
    (cond ((null? items) 'done)
          ((eq? (car items) exception) (loop (cdr items)))
          (else (procedure (car items))
                (loop (cdr items)))))
  (loop list))
\end{scheme}

\noindent
もしコネクタがその値を忘れるよう指示されたなら、局所手続\code{forget\-/my\-/value}を
実行し、最初に要求が元々値を設定した同じオブジェクトからであるかを確認します。
もしそうならばコネクタは関連する制約に値の喪失について伝えます。



局所手続\code{connect}は指定された新しい制約を制約リストに、既に存在しない場合には追加します。
次にもしコネクタが値を持っているのならば、新しい制約にその事実を伝えます。



コネクタの手続\code{me}は他の内部手続を実行する役割を果たし、またコネクタをオブジェクトとして
表現します。以下の手続は起動のための文法上のインターフェイスを提供します。

\begin{scheme}
(define (has-value? connector) (connector 'has-value?))
(define (get-value connector) (connector 'value))
(define (set-value! connector new-value informant)
  ((connector 'set-value!) new-value informant))
(define (forget-value! connector retractor)
  ((connector 'forget) retractor))
(define (connect connector new-constraint)
  ((connector 'connect) new-constraint))
\end{scheme}

\begin{quote}
\heading{\phantomsection\label{Exercise 3.33}Exercise 3.33:} 
プリミティブな乗算器、加算器、定数の制約を用いて、3つのコネクタ\code{a}, \code{b}, \code{c}を
入力として取り、\code{c}の値が\code{a}と\code{b}の値の平均を見出す手続\code{averager}を定義せよ。
\end{quote}

\begin{quote}
\heading{\phantomsection\label{Exercise 3.34}Exercise 3.34:} 
Louis Reasonerは2つの端子を持ち、2つ目の端子上のコネクタ\code{b}が常に1つ目の端子上の値\code{a}
の二乗である制約端末squarerを構築したいと考えた。彼は以下の簡単な乗算から作られた
端末を提案した。

\begin{scheme}
(define (squarer a b) (multiplier a a b))
\end{scheme}



このアイデアには致命的な問題がある。説明せよ。
\end{quote}

\begin{quote}
\heading{\phantomsection\label{Exercise 3.35}Exercise 3.35:} 
Ben BitdiddleはLouisに\link{Exercise 3.34}の問題を避ける1つの方法としてsquarerを
新しいプリミティブな制約として定義することを伝えた。Benの新しい制約の輪郭の欠けて
いる部分を埋めそのような文脈での実装を行え。

\begin{scheme}
(define (squarer a b)
  (define (process-new-value)
    (if (has-value? b)
        (if (< (get-value b) 0)
            (error "square less than 0: SQUARER" 
                   (get-value b))
            ~\( \dark \langle \)~~\var{\dark alternative1}~~\( \dark \rangle \)~)
        ~\( \dark \langle \)~~\var{\dark alternative2}~~\( \dark \rangle \)~))
  (define (process-forget-value) ~\( \dark \langle \)~~\var{\dark body1}~~\( \dark \rangle \)~)
  (define (me request) ~\( \dark \langle \)~~\var{\dark body2}~~\( \dark \rangle \)~)
  ~\( \dark \langle \)~~\var{\dark rest of definition}~~\( \dark \rangle \)~
  me)
\end{scheme}
\end{quote}

\begin{quote}
\heading{\phantomsection\label{Exercise 3.36}Exercise 3.36:} 
以下のグローバル環境内の式の列を評価したとする。

\begin{scheme}
(define a (make-connector))
(define b (make-connector))
(set-value! a 10 'user)
\end{scheme}



\code{set\-/value!}の評価の間のある時点で、コネクタのローカル手続から以下の式が評価される。

\begin{scheme}
(for-each-except 
  setter inform-about-value constraints)
\end{scheme}



上の式が評価される環境を示す環境の図を描け。
\end{quote}

\begin{quote}
\heading{\phantomsection\label{Exercise 3.37}Exercise 3.37:} 
\code{celsius\-/fahrenheit\-/converter}(摂氏華氏変換器)手続は以下のような式指向なスタイルと比べた時に
煩わしい。

\begin{scheme}
(define (celsius-fahrenheit-converter x)
  (c+ (c* (c/ (cv 9) (cv 5))
          x)
      (cv 32)))
(define C (make-connector))
(define F (celsius-fahrenheit-converter C))
\end{scheme}



ここで\code{c+}, \code{c*}等は数値演算命令の``制約''版である。例えば\code{c+}は
2つのコネクタを引数として取り、これらに関係するコネクタを加算器制約にて
返す。

\begin{scheme}
(define (c+ x y)
  (let ((z (make-connector)))
    (adder x y z)
    z))
\end{scheme}


同様の手続\code{c\-/}, \code{c*}, \code{c/}, \code{cv}(定数)を定義し、複合制約を上記の変換器の
例の様に定義できるようにせよ。\footnote{
式指向形式は便利です。それは演算の中間式に
名前を付ける必要性を回避できるためです。私達の元々の制約言語の形式は多くの言語が複合データを
取り扱う場合と同様に面倒でした。例として、変数がベクトルを表現する場合に
積\( (a + b) \cdot (c + d) \)を求めたい時、``命令型スタイル''で指定されたベクトルの
値を設定するけれどもそれ自身はベクトルを値として返さない手続を用いて行うことは可能です。

\begin{smallscheme}
(v-sum a b temp1)
(v-sum c d temp2)
(v-prod temp1 temp2 answer)
\end{smallscheme}



代替法として、ベクトルを値として返す手続を用いて式を用いて行うことも可能です。
その場合、明示的に\code{temp1}と\code{temp2}を記述する必要を避けることができます。

\begin{smallscheme}
(define answer (v-prod (v-sum a b) (v-sum c d)))
\end{smallscheme}



Lispは手続の値として複合オブジェクトを返すことができるため、命令型スタイル制約言語を
式指向スタイルに課題で示されたように変形することができます。複合データの扱いが乏しい
言語、例えばAlgol、Basic、Pascal(明示的にPascalのポインタ変数を用いる場合は除く)では
通常複合オブジェクトを操作する場合に命令型スタイルに行き詰まります。式指向形式の利点を
与えられるとある人はシステムを私達がこの節で行ったように命令型スタイルで実装することに
何らかの意味があるのかと尋ねるかもしれません。1つの理由は非式指向の制約言語は
制約オブジェクト上に、コネクタオブジェクト上と同様にハンドルを提供します(例えば\code{adder}手続の値)。
これはもし我々がシステムをコネクタ上の命令を通して間接的に通信するだけでなく、
制約と直接通信する新しい命令を用いてシステムを拡張したい場合にはとても便利です。
式指向スタイルを命令型の実装を用いて実装するのは簡単ですが、逆はとても難しいのです。}
\end{quote}

\section{並行性: 時間が本質}
\label{Section 3.4}



私達はここまで局所状態をモデリングのためのツールとして持つ計算オブジェクトの力を学びました。
それにもかかわらず、\link{Section 3.1.3}で警告したように、この力にはコストが伴います。
参照等価性を失なうことは等価性と変更に関する問題のチケットを増加し、評価の置換モデルを
断念し、より何回な環境モデルの支持を必要とします。


状態、等価性、変更の複雑さの下に潜んでいる中心的課題は、代入を導入することにより
私達は計算モデルの中に\newterm{time}(\jnewterm{時間})の存在を認めることを強制されることです。
代入の導入前は私達のプログラム全ては値を持つ任意の式が常に同じ値を持つという意味に
おいて恒久的でした。対照的に、\link{Section 3.1.1}で紹介した銀行口座からの引き出しと差引残高の返却
のモデル化の例を思い出して下さい。

\begin{scheme}
(withdraw 25)
~\textit{75}~
(withdraw 25)
~\textit{50}~
\end{scheme}

\noindent
ここでは同じ式の一連の評価が異なる値を生じています。この振舞は代入文の実行
(この場合では変数\code{balance}への代入)が値が変化した\newterm{moments in time}(\jnewterm{時間の瞬間})
を描いています。式の評価の結果は式自身だけではなく、これらの瞬間の前か後に評価が
行われたかにも依存します。局所状態を持つ計算モデルを用いたモデルの構築は私達に
プログラミングにおける本質的な概念としての時間に直面することを強います。



計算モデルの構造化において物理世界の私達の認知を一致させることをより進めることは可能です。
世界の中のオブジェクトは一時に1つが順に変わることはありません。そうではなく、
私達はそれらが\newterm{concurrently}(\jnewterm{並行})に---同時に---行動することを知覚します。
そのためシステムを並行に実行する計算処理の集合であるとモデル化することは多くにおいて自然です。
分離された局所状態を持つオブジェクトを用いてモデルを体系化することにより
私達のプログラムをモジュラー化するのと同様に、計算モデルを別々に、並行に発展する部分に
分割することは多くの場合に適切です。例えプログラムが逐次的な計算機により実行される
としてもプログラムを並行に実行される前提で書くことを練習することはプログラマに
不必要な制約を防ぐことを強いるため、プログラムをよりモジュール式にします。



プログラムをよりモジュール式にするのに加えて、並行演算は逐次的演算に対し速度上の
利点を与えることが可能です。逐次的演算は一時に1つの命令のみを実行するためタスクの
実行にかかる時間量は実行される命令の総量に比例します。\footnote{本物のCPUの
多くは実際にはいくつかの命令を同時に、\newterm{pipelining}(\jnewterm{パイプライン})と
呼ばれる戦略に従い実行します。このテクニックは大きくハードウェアの実行効率を
改善しますが、これは一連の命令ストリームの実行を、逐次的プログラムの振舞を保ち
ながら高速化するためのみに利用されます。}
しかしもし問題を相対的に独立した部分に分割することが可能で、稀にしか通信を行う必要が無ければ、
それらの部分を異なる計算機に配置し、存在する計算機の数に比例した速さの利点を生じることが
可能となるでしょう。



残念なことに、代入により持ち込まれた複雑性は並行性の出現により、より一層難しくなります。
並行実行の結果は世界が並列に作動するためか計算機がそれを行うためかによらず、
私達の時間の理解にさらなる複雑性をもたらします。



\subsection{並行システム内の時間の性質}
\label{Section 3.4.1}



表面上は時間は簡単に見えます。時間はイベントに課される順序付けです。\footnote{ケンブリッジの
ビルの壁上のある落書きを引用すれば``時間とは全てが同時に起こることを防ぐために発明された仕掛けだ''}
任意のイベント\( A \)と\( B \)に対し、\( A \)が\( B \)の前に起こるか、\( A \)と\( B \)が同時か、
\( A \)が\( B \)の後に起こるかです。例えば、銀行口座の例に戻れば、最初に\$100を持つ連結口座から
Peterが\$10を引き出し、Paulが\$25を引き出した場合、口座には\$65が残ります。二人の引き出し順により、
口座の残高の列は\( \,\$100 \to \$90 \to \$65\, \)か\( \,\$100 \to \$75 \to \$65\, \)です。
銀行システムの計算機実装においてこの口座の列の変化は連続した変数\code{balance}への代入として
モデル化できます。


複雑な状況ではしかし、そのような見方は問題となりえます。PeterとPaulに加えて他の人々が
同じ銀行口座に世界中に分散された現金自動預け払い機のネットワークを通してアクセスすると
します。実際の口座の残高の列は大きく、アクセスタイミングの詳細と機械の間の通信の詳細に
依存します。


このイベント順の非決定性は並行システムの設計において深刻な問題を提起します。
例えばPeterとPaulの引き出しが共通の変数\code{balance}を共有する2つの分離した処理
だとします。各処理は\link{Section 3.1.1}にて与えられた手続により指定されます。

\begin{scheme}
(define (withdraw amount)
  (if (>= balance amount)
      (begin
        (set! balance (- balance amount)) balance)
      "Insufficient funds"))
\end{scheme}

\noindent
もし2つの処理が独立に動作するなら、Peterは残高を確認し、正当な額面を
引き出そうとします。しかしPaulがPeterが残高を確認した時点とPeterが引き出しを
完了する時点の間にいくらかの資金を引き出すかもしれません。従ってPeterの
確認を無効にするかもしれません。



さらに悪くなりえます。以下の式について考えてみましょう。

\begin{scheme}
(set! balance (- balance amount))
\end{scheme}

\noindent
この式は各引き出し処理の部分として実行されます。これは3つのステップから成り立ちます。
(1) 変数\code{balance}の値にアクセスする。(2) 新しい残高を計算する。(3) \code{balance}に
新しい値を設定する。もしPeterとPaulの引き出しがこの命令を並行に実行した場合、
二人の引き出しは\code{balance}にアクセスし、それに新しい値を設定する順を交互に配置
するかもしれません。



\link{Figure 3.29}のタイミング図は\code{balance}が100で開始し、Peterが10を引き出し、Paulが
25を引き出し、それでも\code{balance}の最終の値が75である場合のイベントの順を描写しています。
図に示されるとおり、この異例の理由はPaulの\code{balance}への75の代入が減算されるべき\code{balance}
の値が100であるという前提の下で行われているためです。しかしこの前提はPeterが\code{balance}を
90に変更した時に無効になります。これは銀行システムにとって最悪な失敗です。なぜなら
システム中のお金の総量が保存されていません。取引前にお金の総額は100でした。その後、Peterは
\$10を持ち、Paulは\$25を持ち、銀行は\$75を持っています。\footnote{このシステムでより悪い失敗が
2つの\code{set!}命令が残高を同時に変更しようとした場合に起こり得ます。このような場合には
メモリ中に現れる実際のデータは2つの処理により書かれる情報の不作為な組み合わせに最後には
なるかもしれません。多くのコンピュータはプリミティブなメモリライト命令上に内部ロックを持つため、
そのような同時アクセスを防ぎます。しかし、この見たところ簡単な種類のプロテクトでさえ
マルチプロセスのコンピュータの設計においては実装上の課題を提起します。
多様なプロセッサが、データが異なるプロセッサの間でメモリアクセスのスピードを向上するために
レプリケート(``キャッシュとして保存'')が行われるかもしれないという事実に係らず、
静的なメモリ内容の見かけを得ることを保証するには、複雑な\newterm{cache\-/coherence}(\jnewterm{キャッシュ一貫性})
プロトコルが必要となります。}

\begin{figure}[tp]
\phantomsection\label{Figure 3.29}
\centering
\begin{comment}
\heading{Figure 3.29:} Timing diagram showing how interleaving the order of events 
in two banking withdrawals can lead to an incorrect final balance.

\begin{example}
 |           Peter              Bank              Paul
 |                              ____
 |                             /    \
 |             .--------------| \$100 |-------------.
 |             |               \____/              |
 |             V                                   V
 |  .----------------------.            .----------------------.           
 |  | Access balance: \$100 |            | Access balance: \$100 |
 |  `----------+-----------'            `----------+-----------'
 |             V                                   V
 |  .----------------------.            .----------------------.           
 |  | new value: 100-10=90 |            | new value: 100-25=75 |
 |  `----------+-----------'            `----------+-----------'
 |             V                                   |
 |  .----------------------.                       |
 |  | set! balance to \$90  |                       |
 |  `----------+-----------'    ____               |
 |             |               /    \              |
 |             `------------->| \$ 90 |             V
 |                             \____/   .----------------------.
 |                                      | new value: 100-25=75 |
 |                              ____    `----------+-----------'
 |                             /    \              |
 |                            | \$ 90 |<------------'
 V                             \____/
time
\end{example}
\end{comment}
\includegraphics[width=109mm]{fig/chap3/Fig3.29b.pdf}
\begin{quote}
\heading{Figure 3.29:} 2回の引き出しのイベント順の相互配置が不正確な最終残高へどのように導くかを
示したタイミング図
\end{quote}
\end{figure}

\noindent
ここに描かれた一般的な現象は、いくつかのプロセスが共通な状態変数を共有していることです。
このことを複雑にしているのは複数のプロセスが共有された状態を同時に操作しようと
試みていることです。銀行口座の例では、各取引の間に、各顧客は他の顧客が存在しないかのように
行動できなければなりません。顧客が口座を残高に依存した形で更新する時、その顧客は、変更の
瞬間の前に、残高が依然として彼が考えた状態であることを前提とできなければなりません。

\subsubsection*{並行プログラムの正しい振舞}



先の例は並行プログラムに潜みがちな微妙なバグの類型です。この複雑性の根本は
異なるプロセスの間で共有される変数への代入に横たわっています。
私達は既に\code{set!}を用いるプログラムを書く場合には気をつけねばならないことを
知っています。計算の結果が代入の起こる順に依存するためです。\footnote{\link{Section 3.1.3}に
おける指数プログラムはこのことを単一の逐次処理にて説明しました。}並行プロセスでは
特に代入に気をつけねばなりません。異なるプロセスにより作られる代入の順を
コントロールできないかもしれないためです。もしいくつかのそのような変更が
(二人の預金者が連結口座にアクセスするように)並行に行われるのであれば私達のシステムの
振舞が正しいことを確認するための何らかの方法を必要とします。例えば、連結口座からの
引き出しの場合、お金が保管されていることを確認しなければなりません。並行プログラムの
振舞を正しくするために、並行実行に何らかの制限を置かねばなりません。



並行性への可能な1つの制限は、任意の共有状態変数を変更するどの2つの命令もどうじ には起こり
得ないことです。これはとても厳しい制限です。分散銀行システムではシステム設計に対し
ただ1つの取引だけが一時に手続できることを保証することを要求します。これは非効率であり、
かつ過度に保守的です。\link{Figure 3.30}はPeterとPaulが銀行口座を共有し、Paulはまたプライベートな
口座を持っていることを示しています。共有口座からの2つの引き出し(1つはPeterによる、もう1つはPaul
によるもの)とPaulのプライベート口座への預金を図示しています。\footnote{列はPeterの財布、
(Bank1内の)共有口座、Paulの財布、(Bank2内の)Paulのプライベート口座の中身を各引き出し(W)と
預金(D)の前後にて示しています。Peterは\$10をBank1から引き出し、Paulは\$5をBank2に預金し、
次にBank1から\$25を引き出しています。}共有口座からの2つの引き出しは並行であってはなりません
(両方が同じ口座にアクセスと更新を行うため)。またPaulの預金と引き出しは並行であってはなりません
(両方がPaulの財布にアクセスと更新を行うため)。しかしPaulによる彼のプライベート口座への預金を
Peterの共有アカウントからの引き出しと並行に進行することを許すことは何の問題も起こさないはずです。


\begin{figure}[tb]
\phantomsection\label{Figure 3.30}
\centering
\begin{comment}
\heading{Figure 3.30:} Concurrent deposits and withdrawals from a joint account 
in Bank1 and a private account in Bank2.

\begin{example}
 |    Peter          Bank1          Paul           Bank2
 |    ____           ____           ____           ____  
 |   /    \         /    \         /    \         /    \ 
 |  |  \$7  |--. .--| \$100 |       |  \$5  |--. .--| \$300 |
 |   \____/   V V   \____/         \____/   V V   \____/ 
 |           +---+                         +---+         
 |           | W |                         | D |         
 |    ____   ++-++   ____           ____   ++-++   ____  
 |   /    \   | |   /    \         /    \   | |   /    \ 
 |  | \$17  |<-' `->| \$90  |--. .--|  \$0  |<-' `->| \$305 |
 |   \____/         \____/   V V   \____/         \____/ 
 |                          +---+
 |                          | W |
 |    ____           ____   ++-++   ____           ____  
 |   /    \         /    \   | |   /    \         /    \ 
 |  | \$17  |       | \$65  |<-' `->| \$25  |       | \$305 |
 |   \____/         \____/         \____/         \____/ 
 V
time
\end{example}
\end{comment}
\includegraphics[width=94mm]{fig/chap3/Fig3.30b.pdf}
\begin{quote}
\heading{Figure 3.30:} 銀行1の連結口座と銀行2の個人口座への並行な預け入れと引き出し
\end{quote}
\end{figure}

\noindent
並行性上の比較的厳しくない制限は並行システムがまるでプロセスが同じ
順に逐次的に実行されたかのように同じ結果を生成することを保証します。
2つの重要な側面がこの制限にはあります。第一にプロセスに対し実際に
逐次的に実行することを要求はしませんが、\emph{あたかも}逐次的に実行された
場合と同じ結果を生成することを要求します。\link{Figure 3.30}の例に対して
銀行口座システムの設計者は安全にPaulの預金とPeterの引き出しを並行に起こす
ことを許可できます。なぜなら2つの命令が逐次的に起こったのと最終結果が同じに
なるためです。第二に、複数の可能な``正しい''結果が並行プログラムにより生成
されるでしょう。なぜなら結果が\emph{ある}逐次的順序と同じ結果であることのみを
要求しているためです。例えばPeterとPaulの連結口座に\$100が初めにあるとし、
Peterが\$40を預金し、Paulが並行に口座の半分のお金を引き出したとします。
すると逐次的実行の口座残高は\$70か\$90のどちらかになります
(\link{Exercise 3.38}参照)。\footnote{\label{Footnote 39}この考えをより形式的に
表す方法は、並行プログラムは本質的に\newterm{nondeterministic}(\jnewterm{非決定的})である
と述べることです。つまり、それらは単一の値を持つ関数ではなく、結果が起こり得る値の
集合となる関数により説明されます。\link{Section 4.3}では非決定的演算について学びます。}



並行プログラムの正しい実行のためのより弱い要件はまだ有ります。拡散のシミュレーションの
プログラム(例えば物質内の熱の流れ)は巨大な数のプロセスから成り、各プロセスは
小容量の空間を表し、その値を並行に更新します。各プロセスはその値を、その値と近傍の値
の平均へと繰り返し変更します。このアルゴリズムは命令が行われる順から独立して正しい
答に収束します。共有値の並行な使用上にどんな制限も必要としません。

\begin{quote}
\heading{\phantomsection\label{Exercise 3.38}Exercise 3.38:} 
Peter, Paul, Maryが初めに\$100を持つ連結銀行口座を共有すると仮定する。。
並行に、Peterが\$10の預金、Paulが\$20の引き出し、Maryは口座の半分のお金の引き出しを
以下のコマンドにより実行した。

\begin{example}
Peter: (set! balance (+ balance 10))
Paul:  (set! balance (- balance 20))
Mary:  (set! balance (- balance (/ balance 2)))
\end{example}

\begin{enumerate}[a]

\item
これらの3つの取引が完了した後に、全ての異なる\code{balance}の起こり得る値を並べよ。
ただし銀行システムはこの3つのプロセスが何らかの順にて逐次的に実行する前提とする。

\item
もしシステムがプロセスにインターリーブ(相互配置)を認めた場合に生成される他の値は何か?
\link{Figure 3.29}の様なタイミング図を描きこれらの値がどのように起こり得るのか説明せよ。
\end{enumerate}
\end{quote}

\subsection{並行性制御のための仕組み}
\label{Section 3.4.2}


並行プロセスの取扱における困難は異なるプロセスのイベント順の交互配置について考える
必要性に原因があることを学びました。例えば2つのプロセスがあり1つは3つの順序付けられた
イベント\( (a, b, c) \)で、もう1つは3つの順序付けられたイベント\( (x, y, z) \)で
あるとします。もし2つのプロセスが、それらの実行がどのように相互配置されるのかについて
制約無しで並行に実行された時、2つのプロセスの個々の順は変わらないとしても、
20の異なる起こり得るイベントの順が存在します。

\begin{example}
(a,b,c,x,y,z)  (a,x,b,y,c,z)  (x,a,b,c,y,z)  (x,a,y,z,b,c)
(a,b,x,c,y,z)  (a,x,b,y,z,c)  (x,a,b,y,c,z)  (x,y,a,b,c,z)
(a,b,x,y,c,z)  (a,x,y,b,c,z)  (x,a,b,y,z,c)  (x,y,a,b,z,c)
(a,b,x,y,z,c)  (a,x,y,b,z,c)  (x,a,y,b,c,z)  (x,y,a,z,b,c)
(a,x,b,c,y,z)  (a,x,y,z,b,c)  (x,a,y,b,z,c)  (x,y,z,a,b,c)
\end{example}

\noindent
プログラマがこのシステムを設計するにつれ、これらの20種の順序のそれぞれの結果について
考慮して、各振舞が受け入れられるか確認する必要があるでしょう。そのような取り組み方は
プロセスとイベントの数が増加するにつれ、急速に手に負えない物となるでしょう。


並行システムの設計に対するより現実的なアプローチは
プログラムの振舞が正しいことを確認できるよう並行プロセスのインターリーブを制約できる
一般的な仕組みを工夫することです。
多くの仕組みがこの目的のため開発されてきました。この節ではそれらの1つ、
\newterm{serializer}(\jnewterm{シリアライザ}、並列直列変換器)について学びます。

\subsubsection*{共有状態へのアクセスの直列化}



直列化(serialization)は次の考えを実装します。プロセスは並行に実行します。しかし幾つかの手続の
集合が存在し、それらは並行には実行できません。もっと正確に言えば直列化は
各直列化された集合内のただ1つの手続の実行が一時に許されるような複数の区別された
手続の集合を作成します。もし1つの集合内のいくつかの手続が実行されるなら、
集合内の任意の手続を実行しようとするプロセスは最初の実行が完了するまで待つことを
強制されます。



直列化を用いて共有変数へのアクセスをコントロールできます。例えばもし共有変数を
その変数の前の値に応じて変更したい時、同じ手続内でその変数の以前の値にアクセスし、
その変数に新しい値を代入します。それからその変数に代入するどの他の手続もこの
手続とは並行には実行でなきないことを、同じシリアライザを持つこれらの手続の全てを
直列化することにより確実にします。これはその変数の値がアクセスとそれに対応する代入の
間に変更されることができないことを保証します。

\subsubsection*{Schemeのシリアライザ}


上記の仕組みをより確実に行うために、\code{parallel\-/execute}(並列実行)と呼ばれる手続を含む
拡張Schemeを持っていると仮定しましょう。

\begin{scheme}
(parallel-execute ~\( \dark \langle \)~~\( \dark p_1 \)~~\( \dark \rangle \)~ ~\( \dark \langle \)~~\( \dark p_2 \)~~\( \dark \rangle \)~ ~\( \dots \)~ ~\( \dark \langle \)~~\( \dark p_k \)~~\( \dark \rangle \)~)
\end{scheme}

\noindent
各\( \langle \)\( p \)\( \kern0.08em\rangle \)は引数無しの手続でなければなりません。
\code{parallel\-/execute}は分離されたプロセスを各\( \langle \)\( p \)\( \kern0.08em\rangle \)に対し作り、
それらのプロセスは\( \langle \)\( p \)\( \kern0.08em\rangle \)を(引数無しで)適用します。
これらのプロセスは全て並行に実行されます。\footnote{\code{parallel\-/execute}は標準Schemeの一部ではありません。
しかし\acronym{MIT} Schemeで実装することが可能です。私達の実装においては新しい並行プロセスはまた
オリジナルのSchemeプロセスと共に並行に実行できます。また私達の実装では\code{parallel\-/execute}により
返される値は特別なコントロールオブジェクトであり新しく作成されたプロセスを停止させるために使用できます。}



これがどのように利用されるかの例として、以下について考えてみて下さい。

\begin{scheme}
(define x 10)
(parallel-execute
 (lambda () (set! x (* x x)))
 (lambda () (set! x (+ x 1))))
\end{scheme}

\noindent
これ2つの並行プロセス---\code{x}に\code{x}かける\code{x}を設定する\( P_1 \)と、\code{x}に1を足す
\( P_2 \)を作成します。実行完了後に、\( P_1 \)と\( P_2 \)の
イベントのインターリーブに依存するため、\code{x}は5つの起こり得る値の内1つに成ります。

\begin{scheme}
101: ~\textrm{\( P_1 \)が\code{x}に100を設定し、次に\( P_2 \)が\code{x}を101に増やす}~
121: ~\textrm{\( P_2 \)が\code{x}を11に増やし、次に\( P_1 \)が\code{x}を\code{x} \code{*} \code{x}に設定}~
110: ~\textrm{\( P_2 \)が\code{x}を10から11に以下の2度のアクセスの間に変化させる} ~
     ~\textrm{\( P_1 \)が\code{x}の値に\code{(* x x)}の評価の間にアクセスする}~
 11: ~\textrm{\( P_2 \)が\code{x}にアクセスし, 次に\( P_1 \)が\code{x}に100を設定し、\( P_2 \)が\code{x}を設定}~
100: ~\textrm{\( P_1 \)が\code{x}に(二度)アクセスし、次に\( P_2 \)が\code{x}を11に設定、次に\( P_1 \)が\code{x}を設定}~
\end{scheme}

\noindent
並行性を\newterm{serializers}(\jnewterm{シリアライザ})により作成された直列化された手続を用いることで抑制
することができます。シリアライザは\code{make\-/serializer}により構築され、この実装は後程与えられます。
シリアライザは手続を引数として取り、元の手続の様に振る舞うserialized(被直列化)手続を返します。
与えられたシリアライザへの全ての呼出は同じ集合に属する被直列化手続を返します。


従って上の例とは異なり、以下の実行は

\begin{scheme}
(define x 10)
(define s (make-serializer))
(parallel-execute
 (s (lambda () (set! x (* x x))))
 (s (lambda () (set! x (+ x 1)))))
\end{scheme}

\noindent
\code{x}に対しただ2つの起こり得る値、101と121を返します。他の可能性は\( P_1 \)と\( P_2 \)の
実行がインターリーブ(相互配置)されないため排除されました。


以下に\link{Section 3.1.1}の\code{make\-/account}手続を預け入れと引き出しが直列化された版を示します。

\begin{scheme}
(define (make-account balance)
  (define (withdraw amount)
    (if (>= balance amount)
        (begin (set! balance (- balance amount))
               balance)
        "Insufficient funds"))
  (define (deposit amount)
    (set! balance (+ balance amount))
    balance)
  (let ((protected (make-serializer)))
    (define (dispatch m)
      (cond ((eq? m 'withdraw) (protected withdraw))
            ((eq? m 'deposit) (protected deposit))
            ((eq? m 'balance) balance)
            (else (error "Unknown request: MAKE-ACCOUNT"
                         m))))
    dispatch))
\end{scheme}

\noindent
この実装により、2つのプロセスは単一の口座に並行に預け入れと引き出しを行うことはできなく
なりました。これにより\link{Figure 3.29}で図示されたエラーの原因、
Paulの新しい値を求めるための残高へのアクセスと、Paulが実際に代入を行う時の間に、
Peterが口座残高を変更する場合は排除されます。
一方で、各口座はそれ自身のシリアライザを持つので、異なる口座への預金と引き出しは並行
に行うことができます。

\begin{quote}
\heading{\phantomsection\label{Exercise 3.39}Exercise 3.39:} 
上で示された並行実行における5つの可能性の内、もし変わりに以下のような実行を起こなった
場合にどれが残るか?

\begin{scheme}
(define x 10)
(define s (make-serializer))
(parallel-execute 
 (lambda () (set! x ((s (lambda () (* x x))))))
 (s (lambda () (set! x (+ x 1)))))
\end{scheme}
\end{quote}

\begin{quote}
\heading{\phantomsection\label{Exercise 3.40}Exercise 3.40:}
以下を実行した場合に\code{x}の起こり得る値の全てを上げよ。

\begin{scheme}
(define x 10)
(parallel-execute (lambda () (set! x (* x x)))
                  (lambda () (set! x (* x x x))))
\end{scheme}

これらの内もし代わりに以下の直列化手続を用いた場合どれが残るか?

\begin{scheme}
(define x 10)
(define s (make-serializer))
(parallel-execute (s (lambda () (set! x (* x x))))
                  (s (lambda () (set! x (* x x x)))))
\end{scheme}
\end{quote}

\begin{quote}
\heading{\phantomsection\label{Exercise 3.41}Exercise 3.41:} 
Ben Bitdiddleは以下のように銀行口座を実行すればより良くなるのではないかと心配している。
(コメントの有る行が変更されている)。

\begin{scheme}
(define (make-account balance)
  (define (withdraw amount)
    (if (>= balance amount)
        (begin (set! balance 
                     (- balance amount))
               balance)
        "Insufficient funds"))
  (define (deposit amount)
    (set! balance (+ balance amount))
    balance)
  (let ((protected (make-serializer)))
    (define (dispatch m)
      (cond ((eq? m 'withdraw) (protected withdraw))
            ((eq? m 'deposit) (protected deposit))
            ((eq? m 'balance)
             ((protected 
               (lambda () balance)))) ~\textrm{; serialized}~
            (else
             (error "Unknown request: MAKE-ACCOUNT"
                    m))))
    dispatch))
\end{scheme}

\noindent
心配の理由は非直列化アクセスを銀行口座に許すと得意な振舞が起こり得るためだ。
同意するか? Benの懸念を実演するシナリオは存在するか?
\end{quote}

\begin{quote}
\heading{\phantomsection\label{Exercise 3.42}Exercise 3.42:} 
Ben Bitdiddleは全ての\code{withdraw}と\code{deposit}メッセージに対して新しい被直列化手続を
作成することは時間の無駄であると提案した。彼は\code{protected}への呼出が
\code{dispatch}手続の外で行われるよう\code{make\-/account}を変更することができると述べた。
つまりwithdrawal手続が呼ばれる度に、口座が(口座が作成されたと同時に作成された)同じ
被直列化手続を返すことになるだろう。

\begin{scheme}
(define (make-account balance)
  (define (withdraw amount)
    (if (>= balance amount)
        (begin (set! balance (- balance amount))
               balance)
        "Insufficient funds"))
  (define (deposit amount)
    (set! balance (+ balance amount))
    balance)
  (let ((protected (make-serializer)))
    (let ((protected-withdraw (protected withdraw))
          (protected-deposit (protected deposit)))
      (define (dispatch m)
        (cond ((eq? m 'withdraw) protected-withdraw)
              ((eq? m 'deposit) protected-deposit)
              ((eq? m 'balance) balance)
              (else 
               (error "Unknown request: MAKE-ACCOUNT"
                      m))))
      dispatch)))
\end{scheme}



これは行うことが安全な変更だろうか? 具体的には、これらの2つの版の\code{make\-/account}により
許される並行性に違いは存在するだろうか?
\end{quote}

\subsubsection*{複数共有リソース使用の複雑さ}



シリアライザは並行プログラムの複雑性の分離を手助けすることで、注意深く(願わくは)正しく
取り扱えるようにする強力な抽象化を与えます。しかしシリアライザの使用は
(単一の銀行口座のような)ただ単一の共有リソースが存在する場合には相対的に簡単ですが、
並行プログラミングは複数の共有リソースがある場合に、裏切るかのように難しくなります。



提起できる困難さの内1つを説明するために、2つの銀行口座の残高を交換したいと
考えます。各口座にアクセスし残高を見つけ、残高間の差を計算し、一方の口座から
この差を引き出し、もう一方の口座へ預け入れます。これを以下のように実装することが
できます。\footnote{\code{deposit}メッセージが負の額面を受け入れるという事実を利用する
ことで\code{exchange}を簡略化しました。(これは私達の銀行システムの深刻なバグです!)}

\begin{scheme}
(define (exchange account1 account2)
  (let ((difference (- (account1 'balance)
                       (account2 'balance))))
    ((account1 'withdraw) difference)
    ((account2 'deposit) difference)))
\end{scheme}

\noindent
この手続は単一のプロセスのみが交換を試みる場合にはうまく働きます。しかしPeterとPaulが
二人共口座\( a \)1, \( a \)2, \( a \)3にアクセスし、そしてPeterが\( a \)1と\( a \)2を
交換している間にPaulが並行に\( a \)1と\( a \)3を交換している場合を考えてみて下さい。
例え口座の預け入れと引き出しが個別の口座に対して(この節の上で示された\code{make\-/account}手続の
ように)直列化されたとしても、\code{exchange}は依然として不正確な結果を生じることができます。
例えばPeterが\( a \)1と\( a \)2の残高の差を求める時、PaulがPeterが交換を完了する前に
\( a \)1の残高を変更するかもしれません。\footnote{もし口座残高が\$10, \$20, \$30で始めた
場合、任意の回数の交換の後に、残高は何らかの順にて依然として\$10, \$20, \$30にならねばなりません。
個別の口座への預け入れの直列化はこれを保証するのに十分ではありません。\link{Exercise 3.43}を
参照して下さい。}正しい振舞のためには、\code{exchange}手続を、交換の全体の時間の間、口座への
どの他の並行アクセスもロックアウト(締め出し)するように準備をしなければなりません。


これを達成する1つの方法は両方の口座のシリアライザを用いて\code{exchange}手続全体を
直列化します。これを行うためには、口座のシリアライザへのアクセスに準備を行います。
シリアライザを露出することで、銀行口座オブジェクトのモジュール化を意図的に破っている
ことに注意して下さい。\code{make\-/account}の以下の版は\link{Section 3.1.1}で与えられた元の版と
シリアライザがbalance変数を守るため提供されていることを除けば同じです。そして
シリアライザはメッセージパッシングを通して転送されます。

\begin{scheme}
(define (make-account-and-serializer balance)
  (define (withdraw amount)
    (if (>= balance amount)
        (begin (set! balance (- balance amount))
               balance)
        "Insufficient funds"))
  (define (deposit amount)
    (set! balance (+ balance amount))
    balance)
  (let ((balance-serializer (make-serializer)))
    (define (dispatch m)
      (cond ((eq? m 'withdraw) withdraw)
            ((eq? m 'deposit) deposit)
            ((eq? m 'balance) balance)
            ((eq? m 'serializer) balance-serializer)
            (else 
             (error "Unknown request: MAKE-ACCOUNT"
                    m))))
    dispatch))
\end{scheme}

\noindent
これを用いて直列化された預け入れと引き出しを行うことができます。しかし最初の直列化された
口座とは異なり、直列化を明示的に管理することは銀行口座オブジェクトの各ユーザの責任です。
例えば以下の様にです。\footnote{\link{Exercise 3.45}にてなぜ預け入れと引き出しがもはや
自動的に口座により直列化されないのかについて調査します。}

\begin{scheme}
(define (deposit account amount)
  (let ((s (account 'serializer))
        (d (account 'deposit)))
    ((s d) amount)))
\end{scheme}

\noindent
シリアライザをこの方法で外出しすることは私達に直列化された交換プログラムを実装するのに
十分な柔軟性を与えます。単純に元の\code{exchange}手続を両方の口座のシリアライザにて
直列化します。

\begin{scheme}
(define (serialized-exchange account1 account2)
  (let ((serializer1 (account1 'serializer))
        (serializer2 (account2 'serializer)))
    ((serializer1 (serializer2 exchange))
     account1
     account2)))
\end{scheme}

\begin{quote}
\heading{\phantomsection\label{Exercise 3.43}Exercise 3.43:} 
3つの口座の残高が\$10, \$20, \$30で始まり、複数のプロセスが実行され口座の残高を
交換すると考える。プロセスが逐次的に実行されるなら、任意の数の並行な交換の後に、
口座残高がある順序において\$10, \$20, \$30になると主張する。\link{Figure 3.29}の
ようなタイミング図を描き、交換がこの節のaccount-exchangeの最初の版を
用いて実装された場合にこの前提がどのように破られるかについて示せ。一方で、
例えこの\code{exchange}プログラムを用いても口座の残高の合計は保存されると主張する。
タイミング図を描き、個別の口座上の取引を直列化しない場合には
例えこの前提でもどのように破られるかについて示せ。
\end{quote}

\begin{quote}
\heading{\phantomsection\label{Exercise 3.44}Exercise 3.44:} 
ある口座から別の口座への振込の問題について考える。Ben Bitdiddleは
例え複数の人々が並行にお金を複数の口座間にて転送をしても、以下の手続を用いることで、
預金と引き出しの取引を直列化する任意の口座の仕組み、例えば上のテキストの\code{make\-/account}の版を
用いながら振込を達成できると主張する。

\begin{scheme}
(define (transfer from-account to-account amount)
  ((from-account 'withdraw) amount)
  ((to-account 'deposit) amount))
\end{scheme}



Louis Reasonerはここにも問題があると主張した。交換問題を取り扱うのに必要とされた様な
より洗練された手法が必要であるとも述べた。Louisは正しいだろうか? もし正しくないのならば
振込問題と交換問題の間の本質的な違いは何か? (\code{from\-/account}の残高は少くとも\code{amount}で
あると考えること)。
\end{quote}

\begin{quote}
\heading{\phantomsection\label{Exercise 3.45}Exercise 3.45:} 
Louis Reasonerは私達の銀行口座システムは不必要に複雑、かつエラーを起こしやすく、
預け入れと引き出しも自動的に直列化されないと考えた。彼は
\code{make\-/account}が行ったように口座と預け入れを直列化するためにそれを用いることに加えて
(その代わりにではなく)
\code{make\-/account\-/and\-/serializer}はシリアライザを(\code{serialized\-/exchange}のような
手続にて利用するために)露出させるべきだったと主張した。
彼は口座を以下のように再定義することを提案した。

\begin{smallscheme}
(define (make-account-and-serializer balance)
  (define (withdraw amount)
    (if (>= balance amount)
        (begin (set! balance (- balance amount)) balance)
        "Insufficient funds"))
  (define (deposit amount)
    (set! balance (+ balance amount)) balance)
  (let ((balance-serializer (make-serializer)))
    (define (dispatch m)
      (cond ((eq? m 'withdraw) (balance-serializer withdraw))
            ((eq? m 'deposit) (balance-serializer deposit))
            ((eq? m 'balance) balance)
            ((eq? m 'serializer) balance-serializer)
            (else (error "Unknown request: MAKE-ACCOUNT" m))))
    dispatch))
\end{smallscheme}



するとdeposit(預け入れ)は元々の\code{make\-/account}で用いたように扱われる。

\begin{scheme}
(define (deposit account amount)
  ((account 'deposit) amount))
\end{scheme}


Louisの推論の何が間違っているか説明せよ。具体的には\code{serialized\-/exchange}が
呼ばれた時に何が起こるかについて考えよ。
\end{quote}

\subsubsection*{シリアライザの実装}



私達はシリアライザを\newterm{mutex}(\jnewterm{ミューテックス}、相互排除)と呼ばれるよりプリミティブな
同期の仕組みを用いて実装します。mutexは2つの命令をサポートするオブジェクトです。1つはmutex
が\newterm{acquired}(\jnewterm{獲得})でき、もう1つはmutexが\newterm{released}(\jnewterm{解放})できます。
一度mutexが獲得されれば、他のそのmutexに対する獲得命令はそのmutexが解放されるまで
続行することができません。\footnote{``mutex''という用語は\newterm{mutual exclusion}(\jnewterm{相互排除})
の省略形です。並行処理が安全に資源を共有することを可能にする仕組みの準備における一般的な問題は
相互排除問題と呼ばれます。私達のmutexは\newterm{semaphore}(\jnewterm{セマフォ})という仕組みの簡単な改良型
です。(\link{Exercise 3.47}参照)。これはアイントホーフェン技術大学にて開発された
``THE'' Multiprogramming system(訳注: THEは究極のとかこれぞとか唯一の等の意味になる)
にて導入され、大学のオランダ語でのイニシャルから名付けられました(\link{Dijkstra 1968a})。
acquireとreleaseの命令は元々はオランダ語の単語\emph{passeren}(渡す)と\emph{vrijgeven}(解放する)から
PとVと呼ばれ、鉄道システムにて用いられたsemaphores(信号装置)を参照しています。
Dijkstra(ダイクストラ)の古典的解説(\link{Dijkstra 1968b})は明確に並行コントロールの問題を表した最も初期の
1つであり、多様な並行問題をどのようにセマフォを用いて扱うかについて示しました。}
私たちの実装では、各シリアライザは関連付けられたmutexを持ちます。手続\code{p}を与えられた場合、
シリアライザはmutexを獲得する手続を返し、\code{p}を実行し、それからmutexを解放します。
これがシリアライザにより生成された手続の1つのみが一度に実行できることを保証します。これが
まさに私たちが保証する必要のある、直列化の特性です。

\begin{scheme}
(define (make-serializer)
  (let ((mutex (make-mutex)))
    (lambda (p)
      (define (serialized-p . args)
        (mutex 'acquire)
        (let ((val (apply p args)))
          (mutex 'release)
          val))
      serialized-p)))
\end{scheme}

\noindent
mutexはミュータブルなオブジェクト(ここで私達は1要素のリストを使用し、\newterm{cell}(\jnewterm{セル})
と参照します。)であり、trueかfalseの値を保持します。値がfalseの時、mutexは獲得可能です。
値がtrueの時、mutexは使用不可であり、このmutexを獲得しようと試みるプロセスは待たなければ
いけません。



私達のmutexコンストラクタ\code{make-mutex}はセルの中身をfalseに初期化することから始めます。
mutexを獲得するためにはセルを確認します。もしmutexが使用可能であれば、セルの中身をtrueに
して続行します。そうでなければループの中で待ち、mutexが使用可能になるまで何度も獲得を
試みます。\footnote{多くの時分割OSではmutexでブロックされるプロセスは上記のように
``busy-waiting''(占有待ち)にて時間を無駄にはしません。その代わりにシステムは他のプロセスを
最初のプロセスが待っている間に実行するようにスケジュールし、ブロックされたプロセスはmutexが使用可能に
なると起こされます。}mutexを解放するためにはセルの中身にfalseを設定します。

\begin{scheme}
(define (make-mutex)
  (let ((cell (list false)))
    (define (the-mutex m)
      (cond ((eq? m 'acquire)
             (if (test-and-set! cell)
                 (the-mutex 'acquire))) ~\textrm{; retry}~
            ((eq? m 'release) (clear! cell))))
    the-mutex))
(define (clear! cell) (set-car! cell false))
\end{scheme}

\noindent
\code{test\-/and\-/set!}はセルをテストし、テストの結果を返します。さらに、もしテストが
falseであれば\code{test\-/and\-/set!}はセルの中身にfalseを返す前にtrueを設定します。
この振舞は以下の手続のように表現できます。

\begin{scheme}
(define (test-and-set! cell)
  (if (car cell) true (begin (set-car! cell true) false)))
\end{scheme}

\noindent
しかし、この\code{test\-/and\-/set!}の実装は現状では十分ではありません。
致命的な機微がここに存在し、ここが並行性コントロールがシステムに入る
本質的な場所です。\code{test\-/and\-/set!}命令は\newterm{atomically}(不可分に、\jnewterm{アトミック}に)
実行されなければなりません。つまり、一度プロセスがセルをテストしfalseであると
知ったならば、セルの中身が実際にセルをテストできるどの他のプロセスよりも先に
trueと設定されることを保証せねばなりません。もしこの保証をしなければmutexは
\link{Figure 3.29}における銀行口座の失敗と似た失敗をします。(\link{Exercise 3.46}参照)。



\code{test\-/and\-/set!}の実際の実装は私達のシステムが並行プロセスをどのように実行するかの
詳細に依存します。例えば私達は並行プロセスを逐次的なプロセッサ上に時分割のメカニズムを
用いて複数のプロセスを循環させることで実行するかもしれません。各プロセスに少ない時間の
間割り込みが発生するまで実行することを許し次のプロセスを開始します。このような場合には
\code{test\-/and\-/set!}はテストと設定の間は時分割を停止することでうまく行きます。\footnote{
シングルプロセッサ向けの\acronym{MIT} Schemeは時分割モデルを使うので\code{test\-/and\-/set!}は以下の様に
実装できます。

\begin{smallscheme}
(define (test-and-set! cell)
  (without-interrupts
   (lambda ()
     (if (car cell)
         true
         (begin (set-car! cell true)
                false)))))
\end{smallscheme}

\noindent
\code{without\-/interrupts}は時分割割り込みをその引数である手続が実行されている間、無効にします。
}
代替法として、マルチプロセスのコンピュータはアトミックな命令を直接ハードウェアにて
サポートします。\footnote{そのような命令にはtest\-/and-set,
test-and-clear, swap, compare-and-exchange, load-reserve, store-conditional等
様々なものが存在し、その設計は注意深くマシンのプロセッサ-メモリ間インターフェイスに
合わせなければいけません。ここで起こる1つの問題にはそのような命令を用いて
完全に同時に同じリソースを2つのプロセスが獲得しようと試みた場合に何が起こるかを
決定することです。これはどのプロセスがコントロールを握るのかについて決定する
ための何らかの仕組みを要求します。そのような仕組みは\newterm{arbiter}(\jnewterm{アービタ}、調停者)
と呼ばれます。アービタは通常ある種のハードウェアデバイスにまとめられます。
残念なことに、アービタに対し自由裁量の長さの時間を決定を行うのに許さない限り
100\%の時間を働く公平なアービタを構築することは物理的に不可能であることが証明できます。
ここでの根本的な現象は元々14世紀のフランス人哲学者Jean Buridan(ジャン ビュリダン)により
Aristotle(アリストテレス)の\textit{De caelo}(天体論)への注釈において観察されています。
ビュリダンは2つの等しく魅力的な食事の情報源の間に置かれた完全に理性的な犬は
飢えて死ぬと主張しました。最初にどちらに行くのか決めることが不可能なためです。
}

\begin{quote}
\heading{\phantomsection\label{Exercise 3.46}Exercise 3.46:} 
\code{test\-/and\-/set!}をテキストに示される通常の手続を用い、命令をアトミックにする試み無しで
実装すると仮定する。\link{Figure 3.29}の様なタイミング図を描き、2つのプロセスが同時に
mutexを獲得するのを許可した場合にmutexの実装がどのように失敗するのか説明せよ。
\end{quote}

\begin{quote}
\heading{\phantomsection\label{Exercise 3.47}Exercise 3.47:} 
(サイズ\( n \)の)セマフォはmutexの一般化である。mutexのように、セマフォはacquireと
release命令をサポートするが、最大\( n \)プロセスまでが並行に獲得できることでは
より一般的である。セマフォを獲得しようとする追加のプロセスは解放命令を待たなければ
ならない。セマフォの実装を以下の条件で行え。

\begin{enumerate}[a]

\item
mutexを用いる

\item
アトミックな\code{test\-/and\-/set!}命令を用いる

\end{enumerate}
\end{quote}

\subsubsection*{デッドロック}



シリアライザをどのように実装するべきかについて学習したため、
例え上記の\code{serialized\-/exchange}を用いても口座の交換が依然として
問題を持つことを理解することができます。Peterが\( a \)1と\( a \)2を交換しようとした時、
Paulが並行に\( a \)2を\( a \)1と交換しようと試みていると想像してみて下さい。
Peterのプロセスが\( a \)1を守る直列化された手続に入った時点に届いたとします。
その直後に、Paulのプロセスが\( a \)2を守る直列化された手続に入りました。
さてPeterは(\( a \)2を守っている直列化された手続に入ること)を進めることはPaulが
\( a \)2を守る直列化された手続から抜けるまでできません。同様に、Paulもまた
Peterが\( a \)1を守る直列化された手続を抜けるまで進めることができません。
この状況は\newterm{deadlock}(\jnewterm{デッドロック})と呼ばれます。デッドロックは並行なアクセスを
複数の共有リソースに対し提供するシステムでは常に存在する危険性です。



この状況におけるデッドロックを防ぐ1つの方法は各口座に固有の識別番号を与え、
\code{serialized\-/exchange}を書き換えることでプロセスが常に最も小さな番号の
口座を守る手続を最初に入るよう試みるようにします。この方法は交換問題に対して
うまく行きますが、より洗練されたデッドロック防止技術を必要とする他の状況が
存在します。またはデッドロックが全く防げない状況も存在します。(\link{Exercise 3.48} 
と\link{Exercise 3.49}を参照)\footnote{デッドロックを共有リソースに番号を付け、順に
獲得する一般的なテクニックは\link{Havender (1968)}によります。デッドロックが防げない
状況では\newterm{deadlock\-/recovery}(\jnewterm{デッドロックリカバリ}(復帰))手法を必要とし、それはプロセスに
デッドロック状態の``back out''(取消)と再試行を引き起します。デッドロックリカバリの
仕組みは広くデータベース管理システムにて使用され、\link{Gray and Reuter 1993}に
詳細が取り上げられています。}

\begin{quote}
\heading{\phantomsection\label{Exercise 3.48}Exercise 3.48:} 
なぜ上で説明されたデッドロック防止手法(即ち口座に番号を付け各プロセスが最も小さな
番号の口座を最初に獲得する)が交換問題のデッドロックを防ぐのか詳細に説明せよ。
\code{serialized\-/exchange}をこの考えを組込むように書き直せ。(\code{make\-/account}も
変更する必要があり、そうすることで各口座が番号と共に作られ、その番号が適切な
メッセージを送ることによりアクセスできるようにしなければならない。)
\end{quote}

\begin{quote}
\heading{\phantomsection\label{Exercise 3.49}Exercise 3.49:} 
上で説明されたデッドロック防止の仕組みがうまく行かない場合のシナリオを示せ。

\end{quote}

\subsubsection*{並行性、時間と通信}



並行システムのプログラミングが異なるプロセスが共有状態にアクセスする時に
イベントの順序をコントロールすることをどうして必要とするかについて学びました。
そして賢明なシリアライザの使用を通してこのコントロールをどのように達成する
かについても学びました。しかし根本的な視点から、常に``共有状態''が
何を意味するのかが明らかでないために、並行性の問題はそれよりも深く位置します。



\code{test\-/and\-/set!}のような仕組みはプロセスに対し任意の時間にグローバルな共有フラグ
の試験を要求します。これは解決が難しく、現在の高速なCPUにおいて実装するのに非効率的です。
パイプラインやキャッシュメモリの様な最適化の仕組みのためメモリの中身は各瞬間において
静的な状態にはありません。現代のマルチプロセスシステムにおいては、従ってシリアライザの
パラダイムは並行性コントロールの新しい取り組みにより取って代わられてきています。\footnote{そのような
直列化の代替法の1つは\newterm{barrier synchronization}(\jnewterm{バリア同期})と呼ばれます。
プログラマは並行プロセスにそれらが気に入るように実行することを許可します。しかし
どのプロセスも全てのプロセスがバリアに着くまでは先に進むことができないいくつかの
同期点(``バリア'')を設置します。現代のプロセッサはプログラマに一貫性が要求される場所に
同期点を設置することを可能にする機械語命令を提供します。例えば\acronym{PowerPC}は
この目的のため\acronym{SYNC}(同期)と
\acronym{EIEIO}(Enforced In\-/order Execution of Input/Output, I/Oの強制順序実行)と
呼ばれる2つの命令を含んでいます。}



共有状態の問題となる側面は巨大な分散システムにおいても生じます。例として、
分散銀行システムを想像して下さい。個別の銀行支店は銀行残高のローカル値を保持し
繰り返しそれらを他の支店により保存されている値と比較します。そのようなシステムにおいては
``口座残高''は同期直後を除いて不確定になるでしょう。もしPeterがお金をPaulと連結する
口座に預け入れした時に、いつ口座残高が変更されたと言うべきか\-/\-/\-/地元の支店が残高を変更した
時か、または同期の後までは言えないのか? そしてもしPaulが異なる支店から口座にアクセスした場合、
振舞が``正しい''銀行システム上に設置する妥当な制約とは何か? 正確性に対し問題となるものは
PeterとPaulが独立して観察する振舞と同期直後の口座の``状態''のみでしょう。``本当の''口座残高に
関わる質問や同期の間のイベントの順は準用ではないか、意味がないでしょう。\footnote{これは
おかしな見方のように見えるかもしれません。しかしこのように動くシステムは存在します。
例えばクレジットカードの口座への国際課金は通常国毎の拠点上で精算され異なる国での
課金は繰り返し消し込みされます。従って口座残高は異なる国では異なります。}



ここでの基本的な現象は異なるプロセスの同期、共有状態の設置、またはイベントの順を
強いることはプロセス間通信を必要とします。本質的に、並行性コントロールにおける
任意の時間の概念は緊密に通信に結びつけられねばなりません。\footnote{分散システムに
対するこの視点は\link{Lamport (1978)}により追求されました。彼は
分散システムにおいてイベントの順序付けを成立させるのに使用できる``グローバルな時計''
を設立するためにどのように通信を用いるかについて示しました。}
面白いことに似たような時間と通信の間の繋がりが相対論にも生じています。光速(イベントの
同期に使用可能な最も高速な信号)は基本的に時間と空間に関連して一定です。
私達の計算モデルの時間と状態を取り扱うために遭遇した複雑性は実際に物理的宇宙の
根本的な複雑性を映しているのかもしれません。

\section{ストリーム}
\label{Section 3.5}


モデリングにおけるツールとしての代入について、また代入が生じる複雑な問題の認識に
ついても良い理解を得ることができました。次は我々が行ってきたことを異なる方法で
行えたのか、そうすることでこれらの問題を回避できたのかについて尋ねる番です。
この節では状態をモデル化する代替となる取り組み方について、\newterm{streams}(\jnewterm{ストリーム})
と呼ばれるデータ構造を基にして探求します。私達が学ぶにつれて、ストリームは状態の
モデル化の複雑性のいくらかを和らげることができます。



一旦戻って、この複雑性がどこから来たのか再検討してみましょう。実際の世界の現象を
モデル化する試みにおいて、私達は幾らかの恐らく適切な決定をしました。私達は
実際の世界のオブジェクトを局所状態を用いて、ローカル変数を持つ計算オブジェクトによって
モデル化しました。私達はコンピュータ内の時間変化により実際の世界の時間変化を
判断しました。私達はコンピュータ内のモデルオブジェクトの状態の時系列変化をモデル
オブジェクトのローカル変数への代入を用いて実装しました。



他に取り組み方があるでしょうか? コンピュータ内の時間をモデル化された世界の時間を用いて
判断することを避けられるでしょうか? 変わり行く世界の事象をモデル化するために
モデルを時間と共に変化させなければならないのでしょうか?
問題を数学の関数を用いて考えましょう。数量\( x \)の時間的に変化する振舞を時間の関数
\( x(t) \)として説明できます。もし瞬間毎に\( x \)に集中すれば変化する数量だと考え
ることができます。けれどももし値の歴史全体の時間集中すれば私達は変化を重要視しません。
関数それ自体は変化しません。\footnote{物理学者は時折粒子の``world lines''(世界線)を
運動に関する推測のための手段として導入することでこの見方を受け入れます。私達もまた既に
(\link{Section 2.2.3})においてこれが信号処理システムについて考える自然な方法であると説明しました。
\link{Section 3.5.3}にて信号処理に対するストリームの適用について探求します。}


もし時間が不連続なステップにより測られるのであれば、(無限に成りうる)列として時間関数を
モデル化できます。この節では変化をモデル化されたシステムの時刻歴(time history)を表す
列を用いてどのように変化をモデル化するかについて学びます。
これを達成するために、\newterm{streams}(\jnewterm{ストリーム})と呼ばれる新しいデータ構造を導入します。
抽象的な視点からはストリームは単に列です。しかし私達はストリームの(\link{Section 2.2.1}にあるような)
リストによる簡単な実装はストリーム処理の力を完全に明かすことができないことを知るでしょう。
代替法として、\newterm{delayed evaluation}(\jnewterm{遅延評価})のテクニックを導入します。
遅延評価は巨大な(例え無限でも)列をストリームして表現することを可能にします。



ストリーム処理は状態を持つシステムを代入やミュータブルなデータを用いずにモデル化することを
可能にします。これは重要な意味合いを倫理的、物理的両方で持ちます。なぜなら代入の導入による
固有の欠陥を防ぐモデルを構築できるためです。一方で、ストリームフレームワークはそれ自身の
困難を持ちます。そしてどのモデリングテクニックがよりモジュラでより簡単にシステムを保守できるか
の疑問が残ります。



\subsection{ストリームとは遅延化リスト}
\label{Section 3.5.1}



\link{Section 2.2.3}で学んだように、列はプログラムモジュールを組み合わせるための標準的な
インターフェイスの役割を果たすことができます。列を操作するための強力な抽象化を
形式化しました。例えば\code{map}, \code{filter}, \code{accumulate}であり、簡潔であり、かつ
洗練された作法にて広範囲の操作を獲得します。



残念なことに、列をリストとして表現するとこの洗練さは演算により必要とされる時間と記憶域に関する
深刻な非効率性を犠牲にして得ることになります。列上の操作をリストの変形として表現した時、
私達のプログラムは(大きくなりえる)データ構造を処理の各ステップにおいて構築とコピーをせねば
なりません。


なぜこれが正しいのか知るために、ある区間の全ての素数の和を求めるための2つのプログラムを
比較してみましょう。最初のプログラムは標準的な繰り返しのスタイルを用います。\footnote{素数性を
テストする(\link{Section 1.2.6}のような)述語\code{prime?}を持っていると仮定します。}

\begin{scheme}
(define (sum-primes a b)
  (define (iter count accum)
    (cond ((> count b) accum)
          ((prime? count) 
             (iter (+ count 1) (+ count accum)))
          (else (iter (+ count 1) accum))))
  (iter a 0))
\end{scheme}

\noindent
2つ目のプログラムは同じ演算を\link{Section 2.2.3}の列命令を用いて実行します。

\begin{scheme}
(define (sum-primes a b)
  (accumulate +
              0
              (filter prime? 
                      (enumerate-interval a b))))
\end{scheme}

\noindent
演算の実行において、最初のプログラムは蓄積される合計のみを格納する必要があります。
逆に、2つ目のプログラムのフィルタは\code{enumerate\-/interval}が区間の数の完全なリスト
を構築するまで一度もテストを行うことができません。フィルタは別のリストを生成し、
合計を形成するため畳み込まれる前に順に\code{accumulate}に渡されます。そのような大きな
中間ストレージは最初のプログラムでは必要ありません。最初のプログラムは区間を昇順に
列挙し、各素数が生成されるにつれ合計に足していくと考えることができます。



リスト使用における非効率性は、以下の式を評価して10,000から1,000,000の区間にて
2つ目の素数を求めるのに列パラダイムを用いると、悲痛な程、明らかです。

\begin{scheme}
(car (cdr (filter prime?
                  (enumerate-interval 10000 1000000))))
\end{scheme}

\noindent
この式は2つ目の素数を確かに見つけました。しかし計算上のコストは酷過ぎます。
ほとんど百万の整数のリストを構築し、このリストを各要素の素数性をテストすることで
選別し、ほとんど全ての結果を無視します。より伝統的なプログラミングスタイルにおいては
列挙とフィルタリングを交互に配置し、2つ目の素数を見つけたら停止します。



ストリームは列をリストとして扱うコストを負担することなく列操作を用いることが可能な
賢明な考えです。ストリームを用いると2つの世界の良い所取りができます。プログラムを
列操作のように優雅に定式化できます。繰り返し演算の効率も獲得できます。
基本的なアイデアはストリームを部分的にのみ構築する準備を行い、部分的な構築物を
ストリームを消費するプログラムに渡します。もし消費プログラムがまだ構築されていない
ストリームの部分にアクセスしようと試みた場合、ストリームは要求された部分を生成するために
自動的にそれ自身の十分な追加を構築します。従ってストリーム全体が存在するという
錯覚を維持することができます。言い替えれば、私達は完全な列を処理するようなプログラムを
書きますが、私達のストリーム実装に自動的に、透過的にストリームの構築とその使用を
相互配置するように設計します。



表面上では、ストリームはそれを操作するための異なった名前を持つただのリストです。
コンストラクタ\code{cons\-/stream}と以下の制約を満たす2つのセレクタ\code{stream\-/car}と\code{stream\-/cdr}
が存在します。

\begin{example}
(stream-car (cons-stream x y)) = x
(stream-cdr (cons-stream x y)) = y
\end{example}

\noindent
判別可能なオブジェクト\code{the\-/empty\-/stream}が存在し、これはどんな\code{cons\-/stream}命令の
結果にはなりえず、述語\code{stream\-/null?}にて識別できます。\footnote{\acronym{MIT}の実装では
\code{the\-/empty\-/stream}は空のリスト\code{'()}と同じで、\code{stream\-/null?}は\code{null?}と同じです。}
従ってストリームを作成し、使用して、リストの作成と使用と同様に、準備された列のデータの
集約を表現することができます。具体的には、\link{Chapter 2}からストリーム用のリスト命令の類似手続、
例えば\code{list\-/ref}, \code{map}, \code{for\-/each}を構築できます。\footnote{これはあなたを困惑させ
るでしょう。そのような似た手続をストリームとリストに定義するという事実は、私達がその根底にある
抽象を見逃していることを示します。残念なことに、この抽象を利用するためには、現在可能なものより
より細かな評価過程に対するコントロールを行使する必要があります。この点については\link{Section 3.5.4}の
終わりにてより詳細に議論します。\link{Section 4.2}ではリストとストリームを統合するフレームワークを
開発します。}

\begin{scheme}
(define (stream-ref s n)
  (if (= n 0)
      (stream-car s)
      (stream-ref (stream-cdr s) (- n 1))))
(define (stream-map proc s)
  (if (stream-null? s)
      the-empty-stream
      (cons-stream (proc (stream-car s))
                   (stream-map proc (stream-cdr s)))))
(define (stream-for-each proc s)
  (if (stream-null? s)
      'done
      (begin (proc (stream-car s))
             (stream-for-each proc (stream-cdr s)))))
\end{scheme}

\noindent
\code{stream\-/for\-/each}はストリームを見るのに便利です。

\begin{scheme}
(define (display-stream s) 
  (stream-for-each display-line s))
(define (display-line x) (newline) (display x))
\end{scheme}

\noindent
ストリームの実装に自動的、かつ透過的にストリームの構築とその使用を相互配置させる
ためには、ストリームの\code{cdr}が、ストリームが\code{cons\-/stream}により構築された時でなく、
\code{stream\-/cdr}手続によりアクセスされた時に評価されるように手筈を整えます。
この実装の選択は\link{Section 2.1.2}での分数の議論を思い出させます。その場合は分子と分母の
最小の項への約分を構築時または選択時に実行されるよう実装を選択できることを
学びました。2つの分数実装は同じデータ抽象化を生成しますが、選択が効率に影響を与えました。
似た関係がストリームと通常のリストの間にも存在します。データ抽象化としては、ストリームは
リストと同じです。違いは要素が評価されるタイミングです。通常のリストでは\code{car}と\code{cdr}の両方は
構築時に評価されます。ストリームでは\code{cdr}は選択時に評価されます。


私達のストリームの実装は\code{delay}(遅延)と呼ばれる特殊形式を基にします。
\code{(delay <\var{exp}>)}の評価は式\( \langle \)\var{exp}\( \kern0.08em\rangle \)を評価しません。
しかしその代わりに所謂\newterm{delayed object}(\jnewterm{遅延オブジェクト})を返します。これは
ある将来の時点で\( \langle \)\var{exp}\( \kern0.08em\rangle \)を評価する``promise''(プロミス、約束)として
考えることができます。\code{delay}の相方として\code{force}(強いる)と呼ばれる手続が存在し、
遅延オブジェクトを引数として取り、評価を実行します。実際に\code{delay}にその約束を果たさせる
ことを強要します。以下で\code{delay}と\code{force}がどのように実装できるかについて学びますが、
最初にこれらを用いてストリームを構築しましょう。


\noindent
\code{cons\-/stream}は特殊形式で、

\begin{scheme}
(cons-stream ~\( \dark \langle \)~~\var{\dark a}~~\( \dark \rangle \)~ ~\( \dark \langle \)~~\var{\dark b}~~\( \dark \rangle \)~)
\end{scheme}

\noindent
上が以下と同じになるよう設計されています。

\begin{scheme}
(cons ~\( \dark \langle \)~~\var{\dark a}~~\( \dark \rangle \)~ (delay ~\( \dark \langle \)~~\var{\dark b}~~\( \dark \rangle \)~))
\end{scheme}

\noindent
これの意味する所は、私達はペアを用いてストリームを構築します。しかし、ペアの
\code{cdr}にストリームの残りの値を置くのではなく、そこにプロミスを置き要求された
時点で残りを計算します。これで\code{stream\-/car}と\code{stream\-/cdr}が手続として定義できます。

\begin{scheme}
(define (stream-car stream) (car stream))
(define (stream-cdr stream) (force (cdr stream)))
\end{scheme}

\noindent
\code{stream\-/car}はペアの\code{car}を選択します。\code{stream\-/cdr}はペアの\code{cdr}を選択し、
そこに見つかった遅延表現を評価し、ストリームの残りを得ます。\footnote{\code{stream\-/car}と
\code{stream\-/cdr}が手続として定義できるにも係わらず、\code{cons\-/stream}は特殊形式で
なければなりません。もし\code{cons\-/stream}が手続であるのならば、私達の評価モデルに従い、
\code{(cons\-/stream <\var{a}> <\var{b}>)}の評価は自動的に\( \langle \)\var{b}\( \kern0.08em\rangle \)の
評価を起こします。これは明らかに私たちにとって起こって欲しくないことです。
同じ理由から\code{delay}も特殊形式でなければなりません。しかし\code{force}は
通常の手続になります。}

\subsubsection*{ストリーム実装の実践}



この実装がどのように振る舞うのかを見るために、先に見た``法外な''素数演算をストリームを
用いて再定式化したものを分析してみましょう。

\begin{scheme}
(stream-car
 (stream-cdr
  (stream-filter prime? 
                 (stream-enumerate-interval 
                  10000 1000000))))
\end{scheme}

\noindent
これが本当に効率的に働くことを見るでしょう。


\code{stream\-/enumerate\-/interval}を引数10,000と1,000,000と共に呼び出すことから始めます。
\code{Stream\-/enumerate\-/interval}は\code{enumerate\-/interval} (\link{Section 2.2.3})の
ストリーム版同等品です。

\begin{scheme}
(define (stream-enumerate-interval low high)
  (if (> low high)
      the-empty-stream
      (cons-stream
       low
       (stream-enumerate-interval (+ low 1) high))))
\end{scheme}

\noindent
従って\code{stream\-/enumerate\-/interval}で返される結果は以下のようになります。
\code{cons\-/stream}で形成された\footnote{ここで示されている数値は遅延オブジェクトの
中には実際には現れません。実際に現れるのは元の式であり、環境の中で変数は適切な
数値に束縛されています。例えば\code{low}が10,000に束縛されながら\code{(+ low 1)}が
\code{10001}が表示されている場所に現れます。}

\begin{scheme}
(cons 10000
      (delay (stream-enumerate-interval 10001 1000000)))
\end{scheme}

\noindent
つまり\code{stream\-/enumerate\-/interval}はペアとして表現されたストリームを返しその\code{car}は
10,000で、その\code{cdr}はプロミスでありもし要求されれば区間のより多くを列挙します。
このストリームはここでフィルタをかけ素数を残します。
\code{filter}手続(\link{Section 2.2.3})のストリーム版同等品を用います。

\begin{scheme}
(define (stream-filter pred stream)
  (cond ((stream-null? stream) the-empty-stream)
        ((pred (stream-car stream))
         (cons-stream (stream-car stream)
                      (stream-filter 
                       pred
                       (stream-cdr stream))))
        (else (stream-filter pred (stream-cdr stream)))))
\end{scheme}

\noindent
\code{stream\-/filter}はストリームの\code{stream\-/car}(ペアの\code{car}であり10,000)を
テストします。これは素数ではないので\code{stream\-/filter}は入力ストリームの\code{stream\-/cdr}を
調査します。\code{stream\-/cdr}の呼出は遅延化された\code{stream\-/enumerate\-/interval}の
評価を強制します。これは今、以下を返します。

\begin{scheme}
(cons 10001
      (delay (stream-enumerate-interval 10002 1000000)))
\end{scheme}

\noindent
\code{stream\-/filter}は今このストリームの\code{stream\-/car}である10,001を見て、
これもまた素数ではないことを確認し、もう一度\code{stream\-/cdr}を強制します。
これを\code{stream\-/enumerate\-/interval}が素数10,007を生じるまで繰り返し、すると
直ぐに\code{stream\-/filter}はその定義に従い以下を返します。

\begin{scheme}
(cons-stream (stream-car stream)
             (stream-filter pred (stream-cdr stream)))
\end{scheme}

\noindent
これはこの場合以下のようになります。

\begin{scheme}
(cons 10007
      (delay (stream-filter
              prime?
              (cons 10008
                    (delay (stream-enumerate-interval
                            10009
                            1000000))))))
\end{scheme}

\noindent
これでこの結果は元の式の\code{stream\-/cdr}に渡されます。これにより遅延された
\code{stream\-/filter}が強制され、それが順に遅延された\code{stream\-/enumerate\-/interval}を
次の素数、10,009を見つけるまで強制します。最終的に、結果が私達の元の式の
\code{stream\-/car}に渡された物が以下です。

\begin{scheme}
(cons 10009
      (delay (stream-filter
              prime?
              (cons 10010
                    (delay (stream-enumerate-interval
                            10011
                            1000000))))))
\end{scheme}

\noindent
\code{stream\-/car}が10,009を返し計算が完了します。2つ目の素数を見つけるのに必要なだけの
整数が素数性のテストを受け、区間は素数フィルタに入力するのに必要なだけ列挙されました。



一般的に、遅延評価は``demand-driven''(要求駆動)プログラミングだと考えることができ、
ストリーム処理の各ステージは次のステージを満たすのに十分な場合にのみ稼動されます。
私達がここで行ったことは手続の見掛け上の構造から実際のイベントの順を分断することです。
手続をストリームが``一度に揃って''存在するかのように書くが、実際には演算は漸増的に
伝統的なプログラミングスタイルのように実行されます。


\subsubsection*{\code{delay}と\code{force}の実装}


\code{delay}と\code{force}はミステリアスな命令に見えるかもしれませんが、それらの実装は
本当にとても簡単です。\code{delay}は式を梱包して要求に応じて評価できるようにせねば
なりません。私達はこれを手続のボディのように式を扱うことで簡単に達成できます。
\code{delay}は以下のような特殊形式です。

\begin{scheme}
(delay ~\( \dark \langle \)~~\var{\dark exp}~~\( \dark \rangle \)~)
\end{scheme}

\noindent
これは以下の糖衣構文になります。

\begin{scheme}
(lambda () ~\( \dark \langle \)~~\var{\dark exp}~~\( \dark \rangle \)~)
\end{scheme}

\noindent
\code{force}は単純に\code{delay}により生成された(引数無しの)手続を呼び出します。
従って\code{force}は手続として実装可能です。

\begin{scheme}
(define (force delayed-object) (delayed-object))
\end{scheme}

\noindent
この実装は\code{delay}と\code{force}が期待通りに動く程度には十分です。しかし導入可能な
重要な最適化が存在します。多くのアプリケーションにおいては同じ遅延オブジェクトを
何度も強制することになります。これがストリームを利用する再帰プログラムにおいて深刻な
非効率の原因となります(\link{Exercise 3.57}参照)。解決方法は遅延オブジェクトが初めて
強制された時に計算された値を保存するように遅延オブジェクトを構築します。
続く強制は格納された値の計算を繰り返さずに、単純に格納された値を返します。
言い替えれば、\code{delay}を特別な目的のメモ化手続として\link{Exercise 3.27}にて説明された
物と同様に実装します。これを達成する1つの方法は以下の手続を用います。これは引数として
(引数の無い)手続を取りその手続のメモ化された版を返します。メモ化された手続が最初に
実行される時、計算結果を格納します。以降の評価では単純に結果を返します。

\begin{scheme}
(define (memo-proc proc)
  (let ((already-run? false) (result false))
    (lambda ()
      (if (not already-run?)
          (begin (set! result (proc))
                 (set! already-run? true)
                 result)
          result))))
\end{scheme}

\noindent
\code{delay}はすると\code{(delay <\var{exp}>)}が以下と等価になるように定義されます。

\begin{scheme}
(memo-proc (lambda () ~\( \dark \langle \)~~\var{\dark exp}~~\( \dark \rangle \)~))
\end{scheme}

\noindent
そして\code{force}は以前の定義と同じです。\footnote{この節で説明された物以外にも
多くのストリームの実装が存在します。遅延評価はストリームを現実的にする鍵ですが、
Algol 60の\newterm{call\-/by\-/name}(\jnewterm{コールバイネーム}、名前による呼出)パラメタパッシング法
固有の物でした。ストリームの実装にこの仕組みを使用することは最初に\link{Landin (1965)}により
説明されました。ストリームに対する遅延評価は\link{Friedman and Wise (1976)}により
Lispに導入されました。彼等の実装では\code{cons}は常にその引数の評価を遅延するので、
リストは自動的にストリームとして振舞いました。メモ化最適化は\newterm{call\-/by\-/need}
(\jnewterm{コールバイニード}、必要による呼出)
としても知られています。Alogolコミュニティは私達の元の遅延オブジェクトを
\newterm{call\-/by\-/name thunks}(\jnewterm{コールバイネーム サンク})と呼び最適化された版を
\newterm{call\-/by\-/need thunks}(\jnewterm{コールバイニード サンク})と呼ぶでしょう。}

\begin{quote}
\heading{\phantomsection\label{Exercise 3.50}Exercise 3.50:} 
以下の定義を完成させよ。これは\code{stream\-/map}を複数の引数を取ることができる
ようにする\link{Section 2.2.1}の\code{map}, \link{Footnote 12}の同等品である

\begin{scheme}
(define (stream-map proc . argstreams)
  (if (~\( \dark \langle \)~~\var{\dark ??}~~\( \dark \rangle \)~ (car argstreams))
      the-empty-stream
      (~\( \dark \langle \)~~\var{\dark ??}~~\( \dark \rangle \)~
       (apply proc (map ~\( \dark \langle \)~~\var{\dark ??}~~\( \dark \rangle \)~ argstreams))
       (apply stream-map
              (cons proc (map ~\( \dark \langle \)~~\var{\dark ??}~~\( \dark \rangle \)~ argstreams))))))
\end{scheme}
\end{quote}

\begin{quote}
\heading{\phantomsection\label{Exercise 3.51}Exercise 3.51:} 
遅延評価のより詳細を見るために、単純に引数を表示した後に引数を返すだけの以下の手続を使用する。

\begin{scheme}
(define (show x)
  (display-line x)
  x)
\end{scheme}



インタプリタが以下の一連の式のそれぞれを評価した時に何を
表示するだろうか?\footnote{\link{Exercise 3.51}や\link{Exercise 3.52}のような課題は
\code{delay}がどのように働くかについての私達の理解を試すために価値有るものです。
一方で、遅延評価を表示---そしてさらに悪いことに代入と---混ぜることは大きな
混乱要因であり、コンピュータ言語の授業のインストラクタ達はこの節にあるような
試験問題で学生達を苦しめてきました。言うまでもありませんが、そのような微妙さ
に依存するプログラムを書くことは醜悪なプログラミングスタイルです。ストリーム処理の
力の一部は私達にイベントが実際にプログラムの中で起こる順について忘れさせてくれる
ことです。残念なことにこれは明らかに代入が存在する場合にはできない事です。
代入は私達に時間と変更に関して心配することを強いるのです。}

\begin{scheme}
(define x 
  (stream-map show 
              (stream-enumerate-interval 0 10)))
(stream-ref x 5)
(stream-ref x 7)
\end{scheme}
\end{quote}

\begin{quote}
\heading{\phantomsection\label{Exercise 3.52}Exercise 3.52:} 
以下の一連の式について考える。

\begin{scheme}
(define sum 0)
(define (accum x) (set! sum (+ x sum)) sum)
(define seq 
  (stream-map accum
              (stream-enumerate-interval 1 20)))
(define y (stream-filter even? seq))
(define z 
  (stream-filter (lambda (x) (= (remainder x 5) 0))
                 seq))
(stream-ref y 7)
(display-stream z)
\end{scheme}



上記の各式が評価された後の\code{sum}の値はいくつか? 式\code{stream\-/ref}と\code{display\-/stream}を
評価した時表示される応答は何か? これらの応答はもし\code{(delay <\var{exp}>)}を単純に
\code{(lambda () <\var{exp}>)}と実装し\code{memo\-/proc}により提供される最適化を使用しな
かった場合に異なるだろうか? 説明せよ。
\end{quote}

\subsection{無限ストリーム}
\label{Section 3.5.2}



実際にはアクセスに必要な分のストリームしか計算していないのに
ストリームを完全な要素の集合として扱うイリュージョンをどのようにサポートするのかについて
学びました。このテクニックを利用して例え列が実際にはとても長くても列を効率的にストリーム
として表現することができます。より印象的なことに、ストリームを無限に長い列を表現するために
使用することができます。例として以下の正の整数のストリームの定義について考えてみましょう。
\begin{scheme}
(define (integers-starting-from n)
  (cons-stream n (integers-starting-from (+ n 1))))
(define integers (integers-starting-from 1))
\end{scheme}

\noindent
\code{integers}が\code{car}が1で\code{cdr}が2で始まる整数を生成するプロミスになるため
これは理にかなっています。これは無限に長いストリームです。しかし任意の与えれた
時間にはその有限な一部しか検討することはできません。従って私達のプログラムは
無限のストリーム全体がそこにあることを知ることはできません。



\code{integers}を用いて他の無限のストリームを定義できます。例えば7で割ることのできない
整数のストリームです。

\begin{scheme}
(define (divisible? x y) (= (remainder x y) 0))
(define no-sevens
  (stream-filter (lambda (x) (not (divisible? x 7)))
                 integers))
\end{scheme}

\noindent
すると7で割り切れない整数をこのストリームの要素にアクセスするだけで見つけることができます。

\begin{scheme}
(stream-ref no-sevens 100)
~\textit{117}~
\end{scheme}

\noindent
\code{integers}と同様に、フィボナッチ数の無限ストリームも定義できます。

\begin{scheme}
(define (fibgen a b) (cons-stream a (fibgen b (+ a b))))
(define fibs (fibgen 0 1))
\end{scheme}

\noindent
\code{fibs}はその\code{car}が0で、その\code{cdr}は\code{(fibgen 1 1)}を評価するプロミスであるペアです。
この遅延化した\code{(fibgen 1 1)}を評価すると、\code{car}が1で\code{cdr}が
\code{(fibgen 1 2)}を評価するプロミスであるペアを生成します。以下、その繰り返しです。



より刺激的な無限ストリームの調査のために、\code{no\-/sevens}の例を一般化し、素数の
無限ストリームを\newterm{sieve of Eratosthenes}(\jnewterm{エラトステネスの篩})として
知られる手法を用いて構築します。\footnote{エラトステネスは紀元前3世紀の
アレキサンドリア学派のギリシャ人哲学者で、地球の外周を最初に正しく推測した
として有名です。彼は夏至の日の正午の影を観察することでこれを求めました。
エラトステネスの篩は古典ですが、特殊用途のハードウェア``篩''の基礎をなしており、
最近まで巨大な素数を突き止める最も強力なツールでした。しかし70年代からこれらの
手法は\link{Section 1.2.6}で議論された確率的な技術の成長により取って代わられました。}
私達は整数を最初の素数である2で始めます。残りの素数を得るために、整数の残りから
2の倍数をフィルタリングすることから始めます。これは3で始まるストリームを残し、
3は次の素数です。ここで3の倍数をこのストリームの残りからフィルタリングします。
これは5で始まるストリームを残し、5は次の素数です。以下これを繰り返します。
言い換えれば、素数を次の様に説明する篩にかける処理により構築します。まずストリーム\code{S}に
篩をかけるために、最初の要素が\code{S}の最初の要素であり、残りは\code{S}の残りから\code{S}の
最初の要素の倍数をフィルタリングすることで得られるストリームを形成します。
そして結果をさらに篩にかけます。この処理は容易にストリーム命令を用いて記述できます。

\begin{scheme}
(define (sieve stream)
  (cons-stream
   (stream-car stream)
   (sieve (stream-filter
           (lambda (x)
             (not (divisible? x (stream-car stream))))
           (stream-cdr stream)))))
(define primes (sieve (integers-starting-from 2)))
\end{scheme}

\noindent
これで特定の素数を見つけるのには以下のように尋ねるだけです。

\begin{scheme}
(stream-ref primes 50)
~\textit{233}~
\end{scheme}

\noindent
\link{Figure 3.31}の``ヘンダーソン図''に示されるように\code{sieve}により設定
された信号処理システムを熟考することは面白いです。\footnote{私達はこれらの図をPeter Henderson
にちなんで名付けました。彼はこの種の図をストリーム処理について考える方法として示した最初の
人物です。各実線は送信される値のストリームを表しています。\code{car}から\code{cons}と\code{filter}への点線は
これがストリームではなく単一の値であることを示します。}
入力ストリームは``un\code{cons}er''に流し込まれ、ストリームの最初の要素をストリームの残りから分離します。
最初の要素は可分性フィルタを構築するのに用いられ、残りはそれに渡され通ります。フィルタの出力は
もう1つの篩の箱に流し込まれます。次に元の最初の用途は内側の篩の出力上に\code{cons}され出力ストリーム
を形成します。従ってストリームのみが無限ではなく、信号処理器もまた無限です。
なぜなら篩がその中に篩を持っているからです。

\begin{figure}[tb]
\phantomsection\label{Figure 3.31}
\centering
\begin{comment}
\heading{Figure 3.31:} The prime sieve viewed as a signal-processing system.

\begin{example}
  +---------------------------------------------------------------+
  | sieve                                                         |
  |                                                               |
  |        __/|                                        |\__       |
  |     __/car|........................................|   \__    |
  |   _/      |           :                            |      \_  |
----><_       |           V                            |  cons _>---->
  |    \__    |    +------------+    +------------+    |    __/   |
  |       \cdr|--->| filter:    |    | sieve      |--->| __/      |
  |          \|    |            |--->|            |    |/         |
  |                | not        |    |            |               |
  |                | divisible? |    |            |               |
  |                +------------+    +------------+               |
  +---------------------------------------------------------------+
\end{example}
\end{comment}
\includegraphics[width=111mm]{fig/chap3/Fig3.31.pdf}
\begin{quote}
\heading{Figure 3.31:} 信号処理システムとして見た素数の篩
\end{quote}
\end{figure}

\subsubsection*{暗黙的ストリーム定義}


上記の\code{integers}と\code{fibs}のストリームは明示的にストリーム要素を1つづつ計算する
``生成''手続を指定することにより定義されました。ストリームを指定する代替法として
遅延評価の利点を用いて暗黙的にストリームを定義することが上げられます。
例えば以下の式はストリーム\code{ones}を1の無限ストリームとして定義します。

\begin{scheme}
(define ones (cons-stream 1 ones))
\end{scheme}

\noindent
これは再帰手続の定義そっくりに動きます。\code{ones}はペアでその\code{car}は1でその\code{cdr}は
\code{ones}を評価するプロミスです。\code{cdr}の評価は再び1と\code{ones}を評価するプロミスを
与えます。以下、繰り返しです。



より面白いこととしてストリームを\code{add\-/streams}のような命令で操作することができます。
\code{add\-/streams}は2つの与えらえたストリームのエレメント同士の和を生成します。

\begin{scheme}
(define (add-streams s1 s2) (stream-map + s1 s2))
\end{scheme}

\noindent
これで整数を以下のように定義できます。

\begin{scheme}
(define integers 
  (cons-stream 1 (add-streams ones integers)))
\end{scheme}

\noindent
これは\code{integers}が最初の要素は1で残りは\code{ones}と\code{integers}の和になります。
従って\code{integers}の2つ目の要素は1 足す \code{integers}の最初の要素、つまり2になります。
\code{integers}の3つ目の要素は1 足す \code{integers}の2つ目の要素、つまり3です。以下繰り返しです。
この定義は任意の時点で十分な\code{integers}ストリームが生成されているので次の整数を生成する
ために定義にフィードバックすることができるためうまく行くのです。


フィボナッチ数も同じスタイルで定義できます。

\begin{scheme}
(define fibs 
  (cons-stream
   0
   (cons-stream 1 (add-streams (stream-cdr fibs) fibs))))
\end{scheme}

\noindent
この定義は\code{fibs}は0と1で始まるストリームであり、残りのストリームは\code{fibs}を
自身に1つずらして足すことで生成することができると述べています。

\begin{scheme}
      1  1  2  3  5  8   13  21  ~\( \dots \)~  =  ~\code{(stream\-/cdr fibs)}~
      0  1  1  2  3  5   8   13  ~\( \dots \)~  =  ~\code{fibs}~
0  1  1  2  3  5  8  13  21  34  ~\( \dots \)~  =  ~\code{fibs}~
\end{scheme}

\noindent
\code{scale\-/stream}はまた別の、そのようなストリーム定義を形成するのに便利な手続です。
これはストリームの各要素に与えられた定数を掛けます。

\begin{scheme}
(define (scale-stream stream factor)
  (stream-map (lambda (x) (* x factor))
              stream))
\end{scheme}

\noindent
例として

\begin{scheme}
(define double (cons-stream 1 (scale-stream double 2)))
\end{scheme}

\noindent
は2の冪乗のストリームを生成します:1, 2, 4, 8, 16, 32, \( \dots \)



素数ストリームの代替定義は整数で始まり、それらの素数性をテストすることでフィルタリング
することでも与えられます。最初の素数、2を開始に必要とします。

\begin{scheme}
(define primes
  (cons-stream
   2
   (stream-filter prime? (integers-starting-from 3))))
\end{scheme}

\noindent
この定義はあまり見かけほど簡単ではありません。\( n \)が(任意の整数でなく)\( \sqrt{n} \)以下の
素数で割り切れるかどうかをチェックすることにより数\( n \)が素数であるかを決めるためです。

\begin{scheme}
(define (prime? n)
  (define (iter ps)
    (cond ((> (square (stream-car ps)) n) true)
          ((divisible? n (stream-car ps)) false)
          (else (iter (stream-cdr ps)))))
  (iter primes))
\end{scheme}

\noindent
これは再帰定義であり、\code{primes}が\code{primes}を用いる\code{prime?}述語を用いて定義されています。
この手続がうまく行く理由は、任意の時点で、十分な\code{primes}ストリームが生成されており、
次にチェックするのに必要な数の素数性をテストできるからです。全ての\( n \)に対して素数性を
テストします。例え\( n \)が素数でなくても(この場合、それを割り切れる素数が既に生成されています。),
例え\( n \)が素数(この場合、素数が既に生成されています---言い換えれば、\( \sqrt{n} \)より大きく\( n \)未満の
素数)であってもです。\footnote{この最後の点はとても微妙で\( p_{n+1} \le p_n^2 \)という事実に依存して
います。(ここで\( p_k \)は\( k \)番目の素数を示します)。このような予測を立証するのはとても難しいです。
ユークリッドによる太古の証明による、ある素数が無限に存在することが\( p_{n+1} \le p_1 p_2 \ldots p_n + 1 \)を
示しています。そして実質的にはより良い結果が証明されることは1851年までありませんでした。
この年、ロシア人の数学者P. L. Chebyshev(パフヌティ・チェビシェフ)は全ての\( n \)に対し\( p_{n+1} \le 2p_n \)で
あることを証明しました。最初に1845年に予想されたこの結果は\newterm{Bertrand's hypothesis}(\jnewterm{ベルトランの
仮説})として知られています。証明は\link{Hardy and Wright 1960}の節22.3に見つかります。}

\begin{quote}
\heading{\phantomsection\label{Exercise 3.53}Exercise 3.53:} 
プログラムを実行すること無しに以下により定義されたストリームの要素について
説明せよ。

\begin{scheme}
(define s (cons-stream 1 (add-streams s s)))
\end{scheme}
\end{quote}

\begin{quote}
\heading{\phantomsection\label{Exercise 3.54}Exercise 3.54:} 
\code{add\-/streams}と類似の手続\code{mul\-/streams}を定義せよ。これは2つの入力ストリームの
要素同士の積を生成する。また、以下を\code{integers}ストリームと共に用いて以下のストリームの
定義を完成させよ。これの\( n \)番目の要素(0で開始)は\( n + 1 \)の階乗である。

% \begin{scheme}
~%~ (define factorials (cons\-/stream 1 (mul\-/streams ~\( \dark \langle \)~~\var{\dark ??}~~\( \dark \rangle \)~ ~\( \dark \langle \)~~\var{\dark ??}~~\( \dark \rangle \)~)))
~%~ \end{scheme}
\noindent
\code{(define factorials (cons\-/stream 1 (mul\-/streams}\( \kern0.7ex\langle \)\code{?}\( \rangle \)\( \kern0.7ex\langle \)\code{?}\( \rangle \)\code{)))}
\end{quote}

\begin{quote}
\heading{\phantomsection\label{Exercise 3.55}Exercise 3.55:} 
手続\code{partial\-/sums}を定義せよ。これはストリーム\( S \)を引数として取り、要素が
\( S_0 \), \( S_0 + S_1 \), \( S_0 + S_1 + S_2, \dots \)であるストリームを返す。
例えば\code{(partial\-/sums integers)}は1, 3, 6, 10, 15, \( \dots \)のストリームにならねば
ならない。
\end{quote}

\begin{quote}
\heading{\phantomsection\label{Exercise 3.56}Exercise 3.56:} 
R. Hamming(リチャード ハミング)により取り上げられた有名な問題に、昇順に、重複無く、
2, 3, 5以外の素因数を持たない正の整数を列挙せよというものがある。これを行う
1つの明らかな方法は単純に各整数を順に2, 3, 5以外の素因数を持つかどうかテストする
方法です。しかしこれはとても非効率です。整数が大きくなる程に要求に合う数はより
少なくなるためです。代替法として、要求された数のストリームを\code{S}と呼び、以下の
事実について注目してみましょう。

\begin{itemize}

\item
\code{S}は1で始まる。

\item
\code{(scale\-/stream S 2)}の要素もまた\code{S}の要素である。

\item
同じことが\code{(scale\-/stream S 3)}と\code{(scale\-/stream 5 S)}に対しても真である。

\item
これらは全て\code{S}の要素である。

\end{itemize}

さて私達が行わなければならないこと全てはこれらの情報から要素を結合することである。
このために2つの順序有りストリームを重複を省き1つの順序付けられた結果のストリームに結合する
手続\code{merge}を定義する。

\begin{scheme}
(define (merge s1 s2)
  (cond ((stream-null? s1) s2)
        ((stream-null? s2) s1)
        (else
         (let ((s1car (stream-car s1))
               (s2car (stream-car s2)))
           (cond ((< s1car s2car)
                  (cons-stream 
                   s1car 
                   (merge (stream-cdr s1) s2)))
                 ((> s1car s2car)
                  (cons-stream 
                   s2car 
                   (merge s1 (stream-cdr s2))))
                 (else
                  (cons-stream 
                   s1car
                   (merge (stream-cdr s1)
                          (stream-cdr s2)))))))))
\end{scheme}

次に要求されたストリームが\code{merge}を用いて以下のよう構築されるだろう。

\begin{scheme}
(define S (cons-stream 1 (merge ~\( \dark \langle \)~~\var{\dark ??}~~\( \dark \rangle \)~ ~\( \dark \langle \)~~\var{\dark ??}~~\( \dark \rangle \)~)))
\end{scheme}

上で\( \langle \)\var{??}\( \kern0.08em\rangle \)とマークされた箇所の欠けた式を埋めよ。
\end{quote}

\begin{quote}
\heading{\phantomsection\label{Exercise 3.57}Exercise 3.57:} 
\( n \)番目のフィボナッチ数を\code{add\-/streams}手続を基にした\code{fibs}の定義を用いて
計算した場合に加算は何回実行されるか? 加算回数が\code{(delay <\var{exp}>)}を単純に
\link{Section 3.5.1}で説明した\code{memo\-/proc}手続により提供される最適化を用いずに、
\code{(lambda () <\var{exp}>)}として実装した場合に指数関数的に増加することを示せ。\footnote{この課題は
call\-/by\-/needが\link{Exercise 3.27}で説明された通常のメモ化に密接に関係していることを示します。
その課題では代入を明示的にローカルの表の構築に用いました。私達のcall-by-needストリームの
最適化は効果的にそのようなテーブルを自動的に構築し、ストリームの以前に強制された部分の値を
格納します。}
\end{quote}

\begin{quote}
\heading{\phantomsection\label{Exercise 3.58}Exercise 3.58:} 
以下の手続により計算されるストリームの解説を与えよ。

\begin{scheme}
(define (expand num den radix)
  (cons-stream
   (quotient (* num radix) den)
   (expand (remainder (* num radix) den) den radix)))
\end{scheme}


(\code{quotient}はプリミティブであり、2つの整数の、整数の商を返す)。
\code{(expand 1 7 10)}により生成される一連の要素は何か?
\code{(expand 3 8 10)}では何が生成されるか?
\end{quote}

\begin{quote}
\heading{\phantomsection\label{Exercise 3.59}Exercise 3.59:} 
\link{Section 2.5.3}にて多項式を項のリストとして表現する多項式数値演算システムをどのように
実装するかについて学んだ。同様な方法で以下のような\newterm{power series}(\jnewterm{べき級数})に
ついても扱うことができる。
\begin{comment}

\begin{example}
               x^2     x^3       x^4
e^x = 1 + x + ----- + ----- + --------- + ...
                2     3 * 2   4 * 3 * 2

             x^2       x^4
cos x = 1 - ----- + --------- - ...
              2     4 * 3 * 2

             x^3         x^5
sin x = x - ----- + ------------- - ...
            3 * 2   5 * 4 * 3 * 2
\end{example}

\end{comment}
\begin{displaymath}
e^x = 1 + x + \displaystyle\frac{x^2}{2} + \displaystyle\frac{x^3}{3 \cdot 2} + \displaystyle\frac{x^4}{4 \cdot 3 \cdot 2} + \dots, 
\end{displaymath}
\begin{displaymath}
\cos x = 1 - \displaystyle\frac{x^2}{2} + \displaystyle\frac{x^4}{4 \cdot 3 \cdot 2} - \dots, 
\end{displaymath}
\begin{displaymath}
\sin x = x - \displaystyle\frac{x^3}{3 \cdot 2} + \displaystyle\frac{x^5}{5 \cdot 4 \cdot 3 \cdot 2} - \dots 
\end{displaymath}
\noindent
これらは無限ストリームとして表現されている。数列\( a_0 +
a_1 x + a_2 x^2 + a_3 x^3 + \dots \)を要素が係数\( a_0 \), \( a_1 \), \( a_2 \), \( a_3 \), \( \dots \)
のストリームとして表すことにする。

\begin{enumerate}[a]

\item
級数\( a_0 + a_1 x + a_2 x^2 + a_3 x^3 + \dots \)の積分は次の級数になる。
\begin{comment}

\begin{example}
             1             1             1
c + a_0 x + --- a_1 x^2 + --- a_2 x^3 + --- a_3 x^4 + ...
             2             3             4
\end{example}

\end{comment}
\begin{displaymath}
 c + a_0 x + {1\over2} a_1 x^2 + {1\over3} a_2 x^3 + {1\over4} a_3 x^4 + \dots, 
\end{displaymath}
\noindent
ここで\( c \)は任意の定数である。羃級数を表すストリーム\( a_0 \), \( a_1 \), \( a_2 \), \( \dots \)を
入力として取り、その級数の積分の非定数項の係数のストリーム\( a_0 \), \( {1\over2}a_1 \), \( {1\over3}a_2 \), \( \dots \)を
返す手続\code{integrate\-/series}を定義せよ。(結果が定数項を持たないため、それは羃級数では無い。
\code{integrate\-/series}を使う時、後で適切な定数を\code{cons}する。)

\item
関数\( x \mapsto e^x \)はそれ自身が導関数である。これは\( e^x \)と\( e^x \)の不定積分が
定数項を除いて同じ級数になることを意味する。定数項は\( e^0 = 1 \)である。結果的に、
\( e^x \)の級数を次のように生成できる。

\begin{scheme}
(define exp-series
  (cons-stream 1 (integrate-series exp-series)))
\end{scheme}



sinとcosの級数をどのように生成するか示せ。sinの導関数がcosであり、cosの導関数が
負のsinであることから始めよ。

\begin{scheme}
(define cosine-series (cons-stream 1 ~\( \dark \langle \)~~\var{\dark ??}~~\( \dark \rangle \)~))
(define sine-series (cons-stream 0 ~\( \dark \langle \)~~\var{\dark ??}~~\( \dark \rangle \)~))
\end{scheme}
\end{enumerate}
\end{quote}

\begin{quote}
\heading{\phantomsection\label{Exercise 3.60}Exercise 3.60:}
\link{Exercise 3.59}における係数ストリームとして表現された羃級数を用いて、級数の加算
は\code{add\-/streams}により実装される。級数を乗算するための以下の手続の定義を完成させよ。

\begin{scheme}
(define (mul-series s1 s2)
  (cons-stream ~\( \dark \langle \)~~\var{\dark ??}~~\( \dark \rangle \)~ (add-streams ~\( \dark \langle \)~~\var{\dark ??}~~\( \dark \rangle \)~ ~\( \dark \langle \)~~\var{\dark ??}~~\( \dark \rangle \)~)))
\end{scheme}



手続ができたら\link{Exercise 3.59}の級数を用いて\( \sin^2 x +
\cos^2 x = 1 \)を確認せよ。
\end{quote}

\begin{quote}
\heading{\phantomsection\label{Exercise 3.61}Exercise 3.61:}
\( S \)が定数項が1の羃級数(\link{Exercise 3.59})であるとする。
羃級数\( 1 / S \)を見つけたいとする。つまり\( SX = 1 \)となるような級数\( X \)である。
\( S_R \)が\( S \)の定数項の後の部分である場合に\( S = 1 + S_R \)を書け。
そうすれば\( X \)を以下のようにして求めることができる。
\begin{comment}

\begin{example}
        S * X = 1
(1 + S_R) * X = 1
  X + S_R * X = 1
            X = 1 - S_R * X
\end{example}

\end{comment}
\begin{displaymath}
%  \eqalign{
% 	        S \cdot X 	&= 1, \cr
% 	(1 + S_R) \cdot X 	&= 1, \cr
% 	  X + S_R \cdot X 	&= 1, \cr
% 	            	X 	&= 1 - S_R \cdot X. \cr
% } 
\begin{array}{r@{{}={}}l}
	        S \cdot X 	& 1, \\
	(1 + S_R) \cdot X 	& 1, \\
	  X + S_R \cdot X 	& 1, \\
	            	X 	& 1 - S_R \cdot X. 
\end{array}
\end{displaymath}


言い換えれば、\( X \)は定数項が1であり高次項が負の\( S_R \)と\( X \)の積により与え
られる羃級数である。この考えを用いて定数項1を持つ羃級数\( S \)に対する
\( 1 / S \)を求める手続\code{invert\-/unit\-/series}を書け。\link{Exercise 3.60}の
\code{mul\-/series}を用いる必要がある。
\end{quote}

\begin{quote}
\heading{\phantomsection\label{Exercise 3.62}Exercise 3.62:} 
\link{Exercise 3.60}と\link{Exercise 3.61}の結果を用いて2つの羃級数を割る
手続\code{div\-/series}を定義せよ。\code{div\-/series}は任意の2つの級数に対して利用できねば
ならず分母の級数は非ゼロな定数項で始まらねばならない。(もし分母がゼロの定数項を
持つならば\code{div\-/series}はエラーを発すこと)。\code{div\-/series}を\link{Exercise 3.59}の
結果と一緒にどのように用いてtanの羃級数を生成するか示せ。
\end{quote}

\subsection{ストリームパラダイムの利用}
\label{Section 3.5.3}


遅延評価を伴なうストリームは強力なモデリングツールにすることができ、局所状態と
代入の利点の多くを提供する。さらにプログラミング言語への代入の導入に伴う、
いくつかの理論的な混乱を防ぎます。



ストリームのアプローチは私達に、状態変数への代入の周りに体系化されたシステムよりも、
異なるモジュール境界を伴なうシステムを構築することを可能にするため、啓発的です。
例えば私達は個別の瞬間における状態変数の値としてではなく、
時系列(または信号)全体を興味の中心として考えることができます。
このことが異なる瞬間の状態のコンポーネントの比較と接続を行うのにより便利にします。

\subsubsection*{反復をストリームプロセスとして定式化する}



\link{Section 1.2.1}において、反復プロセスを紹介しました。これは状態変数を更新することで
進行されます。私達は今、状態を更新される変数の集合としてではなく、
``永遠''の値のストリームとして表すことができます。\link{Section 1.1.7}の平方根手続への
再訪問にこの視点を導入しましょう。考え方は推測値を改善する手続を何度も適用することで
\( x \)の平方根の推測値をより良い値の列を生成するということを思い出して下さい。

\begin{scheme}
(define (sqrt-improve guess x)
  (average guess (/ x guess)))
\end{scheme}

\noindent
私達の元の\code{sqrt}手続では、これらの推測値を状態変数の一連の値にしました。
代わりに推測値の無限ストリームを作ることができます。推測値の初期値は1で
始めます。\footnote{\code{let}をローカル変数\code{guesses}を束縛するのに使うことは
できません。\code{guesses}の値は\code{guesses}自身に依存するためです。
\link{Exercise 3.63}はなぜここで局所変数を欲しがるのかを扱います。}

\begin{scheme}
(define (sqrt-stream x)
  (define guesses
    (cons-stream
     1.0
     (stream-map (lambda (guess) (sqrt-improve guess x))
                 guesses)))
  guesses)

(display-stream (sqrt-stream 2))
~\textit{1.}~
~\textit{1.5}~
~\textit{1.4166666666666665}~
~\textit{1.4142156862745097}~
~\textit{1.4142135623746899}~
~\( \dots \)~
\end{scheme}

\noindent
より多くのストリームの項を生成することでより良い推測値を得ることができます。
もし望むなら、解答が十分に良くなるまで項の生成を続ける手続を書くことも可能です。
(\link{Exercise 3.64}参照)。


同じ方法で扱えるもう1つの反復は\( \pi \)の近似値を\link{Section 1.3.1}で見た
交項級数(交代級数)を基にして生成することが可能です。
\begin{comment}

\begin{example}
[pi]        1     1     1
-\-/\-/- = 1 - -\-/- + -\-/- - -\-/- + ...
  4         3     5     7
\end{example}

\end{comment}
\begin{displaymath}
 {\pi\over4} = 1 - {1\over3} + {1\over5} - {1\over7} + \dots. 
\end{displaymath}

最初に級数の加数(符号が交互に代わる奇数の逆数)のストリームを生成します。次に
(\link{Exercise 3.55}の\code{partial\-/sums}手続を用いてより多くの項の
和のストリームを取り、結果を4倍します。

\begin{scheme}
(define (pi-summands n)
  (cons-stream (/ 1.0 n)
               (stream-map - (pi-summands (+ n 2)))))
(define pi-stream
  (scale-stream (partial-sums (pi-summands 1)) 4))

(display-stream pi-stream)
~\textit{4.}~
~\textit{2.666666666666667}~
~\textit{3.466666666666667}~
~\textit{2.8952380952380956}~
~\textit{3.3396825396825403}~
~\textit{2.9760461760461765}~
~\textit{3.2837384837384844}~
~\textit{3.017071817071818}~
~\( \dots \)~
\end{scheme}

\noindent
これはより良い\( \pi \)の近似値のストリームを提供します。しかし、近似値の
収束はとても遅いです。列の8個の項は\( \pi \)の値を3.284から3.017の間に
束縛されます。


今の所、状態のストリームを使用する取り組みは状態変数を更新する物から大きくは
異なりません。しかしストリームはある面白いトリックを行う機会を提供します。
例えば、近似値の列を同じ値に、ただし、より速く収束する列に変換する
\newterm{sequence accelerator}(\jnewterm{列アクセラレータ})を用いてストリームを
変換することができます。


18世紀のスイスの数学者Leonhard Euler(レオンハルト オイラー)によるそのようなアクセラレータの1つは、
交項級数(符号を互い違いにする項の列)の部分和である列にうまく働きます。
オイラーの手法においては、もし\( S_n \)が元の和の列の\( n \)番目の項であるなら、
加速された列は以下の項を持ちます。
\begin{comment}

\begin{example}
             (S_(n+1) - S_n)^2
S_(n+1) - ------------------------
          S_(n-1) - 2S_n + S_(n+1)
\end{example}

\end{comment}
\begin{displaymath}
 S_{n+1} - {(S_{n+1} - S_n)^2 \over S_{n-1} - 2S_n + S_{n+1}}\,. 
\end{displaymath}
従って元の列が値のストリームとして表現されるならば、変換された列は以下により与えられる。

\begin{scheme}
(define (euler-transform s)
  (let ((s0 (stream-ref s 0))     ~\textrm{; \( S_{n-1} \)}~
        (s1 (stream-ref s 1))     ~\textrm{; \( S_n \)}~
        (s2 (stream-ref s 2)))    ~\textrm{; \( S_{n+1} \)}~
    (cons-stream (- s2 (/ (square (- s2 s1))
                          (+ s0 (* -2 s1) s2)))
                 (euler-transform (stream-cdr s)))))
\end{scheme}

\noindent
オイラーによる加速を私達の\( \pi \)の近似値の列を用いて実演できます。

\begin{scheme}
(display-stream (euler-transform pi-stream))
~\textit{3.166666666666667}~
~\textit{3.1333333333333337}~
~\textit{3.1452380952380956}~
~\textit{3.13968253968254}~
~\textit{3.1427128427128435}~
~\textit{3.1408813408813416}~
~\textit{3.142071817071818}~
~\textit{3.1412548236077655}~
~\( \dots \)~
\end{scheme}

\noindent
さらに良くなるよう、加速された列を加速でき、そして再帰的にそれの加速を繰り返すことが可能です。
すなわち、ストリームのストリーム(\newterm{tableau}(\jnewterm{タブロー})と呼ぶ構造)を作り、
その中では各ストリームは1つ前の変換です。

\begin{scheme}
(define (make-tableau transform s)
  (cons-stream s (make-tableau transform (transform s))))
\end{scheme}

\noindent
タブローは以下の形を取ります。
\begin{comment}

\begin{example}
s_00   s_01   s_02   s_03   s_04   ...
       s_10   s_11   s_12   s_13   ...
              s_20   s_21   s_22   ...
                            ...
\end{example}

\end{comment}
\begin{displaymath}
 \vbox{
\offinterlineskip
\halign{
\strut 	\hfil \  #\  \hfil & 
	\hfil \  #\  \hfil &
	\hfil \  #\  \hfil &
	\hfil \  #\  \hfil &
	\hfil \  #\  \hfil &
	\hfil \  #\  \hfil \cr

$ s_{00} $ 	&  $ s_{01} $ 	&  $ s_{02} $ 	&  $ s_{03} $ 	&  $ s_{04} $ 	&  $ \dots $ \cr
		&  $ s_{10} $ 	&  $ s_{11} $ 	&  $ s_{12} $ 	&  $ s_{13} $ 	&  $ \dots $ \cr
		& 		&  $ s_{20} $ 	&  $ s_{21} $ 	&  $ s_{22} $ 	&  $ \dots $ \cr
		& 		& 		&  $ \dots $ 	& 		&  \cr }
} 
\end{displaymath}


最後にタブローの各行の最初の項を取ることで列を形成します。

\begin{scheme}
(define (accelerated-sequence transform s)
  (stream-map stream-car (make-tableau transform s)))
\end{scheme}

\noindent
\( \pi \)の列のこの種の``超加速''を実演することができます。

\begin{scheme}
(display-stream
 (accelerated-sequence euler-transform pi-stream))
~\textit{4.}~
~\textit{3.166666666666667}~
~\textit{3.142105263157895}~
~\textit{3.141599357319005}~
~\textit{3.1415927140337785}~
~\textit{3.1415926539752927}~
~\textit{3.1415926535911765}~
~\textit{3.141592653589778}~
~\( \dots \)~
\end{scheme}

\noindent
結果は感動的です。列の8つの項を得ることで\( \pi \)の小数点以下14桁の正しい値が
もたらせられます。もし元の\( \pi \)の列のみを使用したなら、\( 10^{13} \)のオーダーの
演算をする必要が(すなわち列の個々の項が\( 10^{-13} \)よりも小さくなるまで十分に長く
展開する必要が)同じ程度の正確さを得るためには必要です!



これらの加速テクニックをストリームを用いずに実装することもできました。
しかしストリームによる定式化はとりわけエレガントで便利です。状態の列全体が
統一された命令の集合により操作可能なデータ構造として使用できるからです。

\begin{quote}
\heading{\phantomsection\label{Exercise 3.63}Exercise 3.63:} 
Louis Reasonerはなぜ\code{sqrt\-/stream}手続が以下のより簡単な方法で、局所変数
\code{guesses}無しで実装されていないのか尋ねた。

\begin{scheme}
(define (sqrt-stream x)
  (cons-stream 1.0 (stream-map
                    (lambda (guess)
                      (sqrt-improve guess x))
                    (sqrt-stream x))))
\end{scheme}



Alyssa P. Hackerが問題の手続のこの版は冗長な演算を行うため、かなり非効率であるからと答えた。
Alyssaの答を説明せよ。もし\code{delay}の実装が\code{memo\-/proc} (\link{Section 3.5.1})で提供された最適化を
用いずに\code{(lambda () <\var{exp}>)}のみを使用したならば2つの版の間に依然として効率上の
違いは存在するだろうか?
\end{quote}

\begin{quote}
\heading{\phantomsection\label{Exercise 3.64}Exercise 3.64:} 
引数としてストリームと許容値の数値を取る手続\code{stream\-/limit}を書け。
差の絶対値が許容値未満である2つの連続する要素を見つけるまでストリームを検査し、
その2つの要素の2番目を返す。これを用いて与えられた許容誤差以内の平方根を求める
ことができるだろう。

\begin{scheme}
(define (sqrt x tolerance)
  (stream-limit (sqrt-stream x) tolerance))
\end{scheme}
\end{quote}

\begin{quote}
\heading{\phantomsection\label{Exercise 3.65}Exercise 3.65:} 
以下の級数を用いて、
\begin{comment}

\begin{example}
            1     1     1
ln 2 = 1 - --- + --- - --- + ...
            2     3     4
\end{example}

\end{comment}
\begin{displaymath}
 \ln 2 = 1 - {1\over2} + {1\over3} - {1\over4} + \dots 
\end{displaymath}
\noindent
\( \pi \)に対して上で行ったのと同様に、2の自然対数の近似値を3種の近似値の列を
求めよ。これらの列はどれだけ早く収束するか?

\end{quote}

\subsubsection*{ペアの無限ストリーム}


\link{Section 2.2.3}において列パラダイムがどのように伝統的な入れ子ループをペアの列上に
定義された手続として扱うかについて学びました。もしこのテクニックを無限ストリームに
対しても一般化すれば簡単には繰り返しとしては表現されないプログラムを書くことが
できます。なぜなら``ループ''を無限集合の範囲にも渡らせなければなりません。



例えば\link{Section 2.2.3}の\code{prime\-/sum\-/pairs}手続を一般化して、整数\emph{全て}の
ペア\( (i, j) \)、但し\( i \le j \)で\( i + j \)が素数である場合の
ストリームを生成します。もし\code{int\-/pairs}が\( i \le j \)における全ての整数のペア
\( (i, j) \)の列であるのならば、私達が必要とするストリームは単純に以下のように
定義されます。\footnote{\link{Section 2.2.3}にもある通り、私達は整数のペアをLispのペアでは
なく、リストにて表現します。}

\begin{scheme}
(stream-filter
 (lambda (pair) (prime? (+ (car pair) (cadr pair))))
 int-pairs)
\end{scheme}

\noindent
すると問題は\code{int\-/pairs}ストリームを生成することになります。より一般的には、
2つのストリーム、\( S = (S_i) \)と\( T = (T_j) \)を持っているとした場合に、
無限の長方形の配列を想像してみて下さい。
\begin{comment}

\begin{example}
(S_0, T_0)  (S_0, T_1)  (S_0, T_2)  ...
(S_1, T_0)  (S_1, T_1)  (S_1, T_2)  ...
(S_2, T_0)  (S_2, T_1)  (S_2, T_2)  ...
   ...
\end{example}

\end{comment}
\begin{displaymath}
 \vbox{
\offinterlineskip
\halign{
\strut 	\hfil \  #\  \hfil & 
	\hfil \  #\  \hfil &
	\hfil \  #\  \hfil &
	\hfil \  #\  \hfil \cr

	$ (S_0, T_0) $ & $ (S_0, T_1) $ & $ (S_0, T_2) $ & $ \dots $ \cr
	$ (S_1, T_0) $ & $ (S_1, T_1) $ & $ (S_1, T_2) $ & $ \dots $ \cr
	$ (S_2, T_0) $ & $ (S_2, T_1) $ & $ (S_2, T_2) $ & $ \dots $ \cr
	$ \dots $ & & & \cr }
} 
\end{displaymath}
配列内の、対角線上かその上部の全てのペアを含むストリームを生成したいと考えます。
つまり、以下のペアです。
\begin{comment}

\begin{example}
(S_0, T_0)  (S_0, T_1)  (S_0, T_2)  ...
            (S_1, T_1)  (S_1, T_2)  ...
                        (S_2, T_2)  ...
                                    ...
\end{example}

\end{comment}
\begin{displaymath}
 \vbox{
\offinterlineskip
\halign{
\strut 	\hfil \  #\  \hfil & 
	\hfil \  #\  \hfil &
	\hfil \  #\  \hfil &
	\hfil \  #\  \hfil \cr

$ (S_0, T_0) $ 	& $ (S_0, T_1) $ 	& $ (S_0, T_2) $ 	& $ \dots $ \cr
		& $ (S_1, T_1) $ 	& $ (S_1, T_2) $ 	& $ \dots $ \cr
		& 			& $ (S_2, T_2) $ 	& $ \dots $ \cr
		& 			& 			& $ \dots $ \cr }
} 
\end{displaymath}
\noindent
(もし\( S \)と\( T \)の両方を整数のストリームとして取るなら、これが望んだストリーム\code{int\-/pairs}です。)


一般的なペアのストリームを\code{(pairs S T)}と呼び、それが3つの部分から組み立てられていると考えます。
ペア\( (S_0, T_0) \)、最初の行の残りのペア、残りのペアです。\footnote{なぜ私達が分解を選ぶのかについての
見識については\link{Exercise 3.68}を参照して下さい。}
\begin{comment}

\begin{example}
(S_0, T_0) | (S_0, T_1)  (S_0, T_2)  ...
-----------+-----------------------------
           | (S_1, T_1)  (S_1, T_2)  ...
           |             (S_2, T_2)  ...
           |                         ...
\end{example}

\end{comment}
\begin{displaymath}
 \vbox{
\offinterlineskip
\halign{
\strut 	\hfil \  #\  \hfil & \vrule
	\hfil \  #\  \hfil &
	\hfil \  #\  \hfil &
	\hfil \  #\  \hfil \cr

$ (S_0, T_0) $ 	& $ (S_0, T_1) $ 	& $ (S_0, T_2) $ 	& $ \dots $ \cr
\noalign{\hrule}
		& $ (S_1, T_1) $ 	& $ (S_1, T_2) $ 	& $ \dots $ \cr
		& 			& $ (S_2, T_2) $ 	& $ \dots $ \cr
		& 			& 			& $ \dots $ \cr }
} 
\end{displaymath}

この分解の3つ目の断片(最初の行にないペア)は(再帰的に)\code{(stream\-/cdr S)}と\code{(stream\-/cdr T)}から
形成されることに注意して下さい。また2番目の断片(最初の行の残り)は以下により求められます。

\begin{scheme}
(stream-map (lambda (x) (list (stream-car s) x))
            (stream-cdr t))
\end{scheme}

\noindent
従って私達のペアのストリームは以下により形成できます。

\begin{scheme}
(define (pairs s t)
  (cons-stream
   (list (stream-car s) (stream-car t))
   (~\( \dark \langle \)~~\var{\dark combine-in-some-way}~~\( \dark \rangle \)~
     (stream-map (lambda (x) (list (stream-car s) x))
                 (stream-cdr t))
     (pairs (stream-cdr s) (stream-cdr t)))))
\end{scheme}

\noindent
手続を完成させるためには、2つの内部ストリームを接続する何らかの方法を選択せねばなりません。
アイデアの1つは\link{Section 2.2.1}の\code{append}手続の類似ストリームを用いる方法です。

\begin{scheme}
(define (stream-append s1 s2)
  (if (stream-null? s1)
      s2
      (cons-stream (stream-car s1)
                   (stream-append (stream-cdr s1) s2))))
\end{scheme}

\noindent
しかし、これは無限ストリームには不適切です。なぜならばこれは最初のストリームからの
要素を全て、2つ目のストリームとの合併前に取ります。具体的には、もし全ての正の整数
のペアを以下のようにして生成しようとすると、

\begin{scheme}
(pairs integers integers)
\end{scheme}

\noindent
結果のストリームは最初に1番目の整数が1の場合の全てのペアを通して実行しようとします。
そしてそれ故に1番目の整数が他の値のペアを全く生成することができません。


無限ストリームを扱うためには、プログラムを十分に長く実行したならば全ての要素がいつかは
得られることを保証する組み合わせの順を工夫する必要があります。これを達成する洗練された
方法は以下の\code{interleave}(相互配置)手続を用います。\footnote{組み合わせの順に要求された属性
を正しく上げると次にようになります。2つの引数を取る関数が必須であり、最初のストリームの要素
\( i \)と2つ目のストリームの要素\( j \)に対応するペアは出力ストリームの\( f(i, j) \)番目として
現れます。\code{interleave}を用いてこれを達成するトリックは、KRC言語にこれを採用したDavid Turnerにより
示されました。(\link{Turner 1981})}

\begin{scheme}
(define (interleave s1 s2)
  (if (stream-null? s1)
      s2
      (cons-stream (stream-car s1)
                   (interleave s2 (stream-cdr s1)))))
\end{scheme}

\noindent
\code{interleave}は2つのストリームから交代に要素を得るため、2つ目のストリームの各要素が
いつかは相互配置ストリームへ入ることが、例え最初のストリームが無限でもわかります。



従って要求されたペアのストリームを以下のように生成できます。

\begin{scheme}
(define (pairs s t)
  (cons-stream
   (list (stream-car s) (stream-car t))
   (interleave
    (stream-map (lambda (x) (list (stream-car s) x))
                (stream-cdr t))
    (pairs (stream-cdr s) (stream-cdr t)))))
\end{scheme}

\begin{quote}
\heading{\phantomsection\label{Exercise 3.66}Exercise 3.66:} 
ストリーム\code{(pairs integers integers)}を試験せよ。pairsがストリーム内に配置する順に
ついて全体的なコメントを行え。例えばペア(1, 100)の前にはおよそどれだけの数のペアが
先行するか? (99, 100)と(100, 100)の場合についても答えよ。(もし正確な数学上の説明が
できるなら、なおさら良い。しかし行き詰まったと感じるのならばより程度的な回答を
気楽に上げて欲しい。)
\end{quote}

\begin{quote}
\heading{\phantomsection\label{Exercise 3.67}Exercise 3.67:} 
\code{pairs}手続を変更し、\code{(pairs integers integers)}が\emph{全て}の整数のペア\( (i, j) \)を
(\( i \le j \)という条件無しで)生成するようにせよ。ヒント:追加のストリームを混ぜ合わせる
必要がある。
\end{quote}

\begin{quote}
\heading{\phantomsection\label{Exercise 3.68}Exercise 3.68:} 
Louis Reasonerは3つの部分からペアストリームを構築することは不必要に複雑なのではないかと
考えた。最初の行のペア\( (S_0, T_0) \)を残りのペアから分離する代わりに、以下のように
最初の行全体を用いて行うことを提案した。

\begin{scheme}
(define (pairs s t)
  (interleave
   (stream-map (lambda (x) (list (stream-car s) x))
               t)
   (pairs (stream-cdr s) (stream-cdr t))))
\end{scheme}


これはうまく行くだろうか?  \code{(pairs integers integers)}をLouisの\code{pairs}の
定義を用いて評価した場合に何が起こるか考えよ。
\end{quote}

\begin{quote}
\heading{\phantomsection\label{Exercise 3.69}Exercise 3.69:}
3つの無限ストリーム、\( S \), \( T \), \( U \)を取り、三つ組\( (S_i, T_j, U_k) \)の
ストリームを生成する手続\code{triples}を書け。但し\( i \le j \le k \)とする。
\code{triples}を用いて全ての正の整数のピタゴラス数の3つ組のストリームを生成せよ。
すなわち三つ組\( (i, j, k) \)は\( i \le j \)、かつ\( i^2 + j^2 = k^2 \)である。
\end{quote}

\begin{quote}
\heading{\phantomsection\label{Exercise 3.70}Exercise 3.70:} 
\emph{アドホック}(その場その場)な相互配置処理の結果の順ではなく、
ペアが何らかの便利な順で現れるストリームを生成できれば便利だろう。
もし整数の1つのペアが別のペアよりも``小さい''と言える方法を定義できるならば
\link{Exercise 3.56}の\code{merge}手続に似たテクニックを用いることができる。
これを行う1つの方法は``重み関数''\( W(i, j) \)を定義し
\( W(i_1, j_1) < W(i_2, j_2) \)であるなら
\( (i_1, j_1) \)は\( (i_2, j_2) \)未満であると取り決める。
\code{merge}に似た手続\code{merge\-/weighted}を書け。ただし\code{merge\-/weighted}は追加の引数
\code{weight}を取り、\code{weight}はペアの重みを計る手続でありマージされた結果のストリームの中で
どの要素が現れるべきかの順を決定するのに利用される。\footnote{ペアの重みがペアの配列の中で
行に沿って外へ動くか、列に沿って下った場合に増えるように重み関数に対して要求するだろう。}
重み関数を計算する手続と一緒にこれを用いて、
\code{pairs}を2つのストリームを取る手続\code{weighted\-/pairs}に一般化し、
重みに従った順のペアのストリームを生成する。
作成した手続を用いて以下を生成せよ。

\begin{enumerate}[a]

\item
全ての正の整数のペア\( (i, j) \)のストリームを\( i \le j \)の条件で、和\( i + j \)に
従った順で生成する

\item
全ての正の整数のペア\( (i, j) \)のストリームを\( i \le j \)かつ\( i \)とmath{j}が
どちらも2, 3, 5で割り切れない条件で、和\( 2i + 3j + 5i\!j \)に従う順序で生成する

\end{enumerate}
\end{quote}

\begin{quote}
\heading{\phantomsection\label{Exercise 3.71}Exercise 3.71:} 
2つの立方数の和で表す方法が複数ある数は時折\newterm{Ramanujan numbers}(\jnewterm{ラマヌジャン数})と
呼ばれる。これは数学者Srinivasa Ramanujan(シュリニヴァーサ・ラマヌジャン)に敬意を表して
いる。\footnote{G. H. Hardy(ゴッドフレイ・ハロルド・ハーディ)によるラマヌジャンの
死亡告知(\link{Hardy 1921})から引用すれば、``Mr. Littlewoodこそが`全ての自然数が彼の友達であった'と
述べられた(私が信じる)人だ。私は彼がPutneyで病気で倒れた
時に1度会いに行った。その時私は車番1729のタクシーに乗ったのでとてもつまらない数に見えたと
告げ、それが良くない前触れでないことを祈ると伝えた。`いいえ'と彼が答えた。`それはとても
面白い数です。それは2つの立方数の和により表現する方法が2つ有る最小の数です'''
ラマヌジャン数を生成する重み付けられたペアのトリックはCharles Leisersonにより私達に
示されました。}
ペアの順序有りストリームはこれらの数を計算する問題に対し洗練された解法を提供する。
2つの立方数の和として表現する方法が2つある数を見付けるためには、\( i^3 + j^3 \)の和に従い
重み付けられた整数のペア\( (i, j) \)のストリームを生成し(\link{Exercise 3.70}参照)、次に
ストリームから同じ重みを持つ連続したペアを探すだけで良い。ラマヌジャン数を生成する
手続を書け。そのような最初の数は1,729である。次の5つは何か?
\end{quote}

\begin{quote}
\heading{\phantomsection\label{Exercise 3.72}Exercise 3.72:} 
\link{Exercise 3.71}と同様な方法で2つの平方数の和として3つの異なる方法で書ける全ての数の
ストリームを生成せよ。(それらがどのようにして、そう書けるのか示せ)。
\end{quote}

\subsubsection*{信号としてのストリーム}



ストリームの議論を信号処理システムにおける``信号''の計算可能な同類であると説明する
ことから始めました。実際に、ストリームを用いて信号処理システムをとても直接的な方法で
モデル化することができ、連続する時間区間の信号の値をストリームの連続する要素として
表現します。例えば\newterm{integrator}(\jnewterm{積分器})、つまり\newterm{summer}(\jnewterm{アナログ加算器})を実装し、
入力ストリーム\( x = (x_i) \)と初期値\( C \)、小さな増分\( dt \)に対し、以下の
合計を累算し、
\begin{comment}

\begin{example}
           i
          ---
S_i = C + >   x_j dt
          ---
          j=1
\end{example}

\end{comment}
\begin{displaymath}
 S_i = C + \sum_{j=1}^i x_j dt 
\end{displaymath}
\noindent
値\( S = (S_i) \)のストリームを返します。以下の\code{integral}手続は(\link{Section 3.5.2})の
整数ストリームの``暗黙的なスタイル''の定義を思い出させます。

\begin{scheme}
(define (integral integrand initial-value dt)
  (define int
    (cons-stream initial-value
                 (add-streams (scale-stream integrand dt)
                              int)))
  int)
\end{scheme}

\noindent
\link{Figure 3.32}は\code{integral}手続に相当する信号処理システムの絵です。入力ストリームは
\( dt \)によりスケール(拡大)され加算器を通して渡され、その出力は同じ加算器に
戻されます。\code{int}の定義内の自己参照が図では加算器の出力が入力の1つに接続される
フィードバックループにより反映されています。

\begin{figure}[tb]
\phantomsection\label{Figure 3.32}
\centering
\begin{comment}
\heading{Figure 3.32:} The \code{integral} procedure viewed as a signal-processing system.

\begin{example}
                             initial-value
                                  |
       +-----------+              |   |\__
input  |           |      |\__    +-->|   \_  integral
------>| scale: dt +----->|   \_      |cons_>--*------->
       |           |      | add_>---->| __/    |
       +-----------+  +-->| __/       |/       |
                      |   |/                   |
                      |                        |
                      +------------------------+
\end{example}
\end{comment}
\includegraphics[width=102mm]{fig/chap3/Fig3.32.pdf}
\begin{quote}
\heading{Figure 3.32:} 信号処理システムとして見た\code{integral}手続
\end{quote}
\end{figure}

\begin{quote}
\heading{\phantomsection\label{Exercise 3.73}Exercise 3.73:} 
電子回路をストリームを用いて一連の時間の時系列電流や電圧の値を表すことでモデル化できる。
例えば、抵抗値\( R \)の抵抗と静電容量\( C \)のコンデンサから成る\newterm{RC circuit}(\jnewterm{RC回路})を
連続して持っているとする。入力された電流\( i \)に対する回路の電圧レスポンス\( v \)は
\link{Figure 3.33}の式により決定し、その構造は添付の信号フロー図により示される。

\clearpage
\begin{figure}[tb]
\phantomsection\label{Figure 3.33}
\centering
\begin{comment}
\heading{Figure 3.33:} An RC circuit and the associated signal-flow diagram.

\begin{example}
  +        v        -

 ->----'\/\/\,---| |---
  i       R         C


                  / t
               1  |
 v  =  v   +  --- |  i dt  +  R i
        0      C  |
                  / 0

         +--------------+
     +-->|   scale: R   |---------------------+   |\_
     |   +--------------+                     |   |  \_
     |                                        +-->|    \   v
  i  |   +--------------+     +------------+      | add >--->
 ----+-->|  scale: 1/C  |---->|  integral  |----->|   _/
         +--------------+     +------------+      | _/
                                    |             |/
				   v
				    0
\end{example}
\end{comment}
\includegraphics[width=94mm]{fig/chap3/Fig3.33.pdf}
\par\bigskip
\noindent
\heading{Figure 3.33:} RC回路と対応する信号処理図
\end{figure}



この回路をモデル化する手続\code{RC}を書け。\code{RC}は入力として\( R \), \( C \), \( dt \)を
取り、手続を返さねばならない。返り値の手続は入力として電流\( i \)を表すストリームと
コンデンサの初期電圧\( v_0 \)を取り、出力として電圧\( v \)のストリームを生成する。
例えば\code{RC}を用いて\( R \)が5[Ω]、\( C \)が1[F]、タイムステップが0.5秒のRC回路を
\code{(define RC1 (RC 5 1 0.5))}を評価することでモデル化できなければならない。
これは\code{RC1}を電流の時系列を表すストリームとコンデンサの初期電圧を取り電圧の出力ストリーム
を生成する。
\end{quote}

\begin{quote}
\heading{\phantomsection\label{Exercise 3.74}Exercise 3.74:} 
Alyssa P. Hackerは物理センサから来る信号を処理するシステムを設計している。
彼女が作りたい重要な機能は入力信号の\newterm{zero crossings}(\jnewterm{ゼロ交差})を記録する信号である。
結果の信号は入力信号が負から正に変わった時に\( +1 \)、正から負に変わった時に\( -1 \)、
その他の場合は0である。(入力が0の場合の符号は正とする)。例えば典型的な入力信号とその関連する
ゼロ交差信号は以下のようになる。

\begin{scheme}
~\( \dots \)~ 1 2 1.5 1 0.5 -0.1 -2 -3 -2 -0.5 0.2 3 4 ~\( \dots \)~
~\( \dots \)~ 0 0  0  0  0   -1   0  0  0   0   1  0 0 ~\( \dots \)~
\end{scheme}

% \begin{smallscheme}
~%~ ~\( \dots \)~1  2  1.5  1  0.5  -0.1  -2  -3  -2  -0.5  0.2  3  4~\( \dots \)~
~%~ ~\( \dots \)~0  0   0   0   0    -1    0   0   0    0    1   0  0~\( \dots \)~
~%~ \end{smallscheme}


Alyssaのシステムではセンサからの信号はストリーム\code{sense\-/data}で表され、
ストリーム\code{zero\-/crossings}が関連するゼロ交差のストリームである。
Alyssaは最初に手続\code{sign\-/change\-/detector}を書いた。これは2つの値を引数として
取り値の符号を比べ値に対応した0, 1, -1を生成する。次にゼロ交差ストリームを以下の
ように構築した。

\begin{scheme}
(define (make-zero-crossings 
         input-stream last-value)
  (cons-stream
   (sign-change-detector
    (stream-car input-stream) 
    last-value)
   (make-zero-crossings
    (stream-cdr input-stream)
    (stream-car input-stream))))
(define zero-crossings
  (make-zero-crossings sense-data 0))
\end{scheme}


Alyssaの上司、Eva Lu Atorが歩み寄り、このプログラムは以下の、\link{Exercise 3.50}の
\code{stream-map}を一般化した版を使用した物とほぼ同じであると提案した。

\begin{scheme}
(define zero-crossings
  (stream-map sign-change-detector
              sense-data
              ~\( \dark \langle \)~~\var{\dark expression}~~\( \dark \rangle \)~))
\end{scheme}



\( \langle \)\var{expression}\( \rangle \)で示された部分を与えてプログラムを完成させよ。
\end{quote}

\begin{quote}
\heading{\phantomsection\label{Exercise 3.75}Exercise 3.75:} 
残念なことに、\link{Exercise 3.74}のAlyssaのゼロ交差判別器は十分でないことが証明された。
センサからのノイズの多い信号が誤ったゼロ交差へと導くためである。ハードウェアのスペシャリスト、
Lem E. TweakitはAlyssaにゼロ交差を試験する前にノイズを排除するために信号を滑らかに
することを提案した。Alyssaは彼のアドバイスを受け入れ、センサのデータの各値を前の値との
平均を取ることで構築された信号からゼロ交差を抽出することを決めた。彼女は問題を彼女の
アシスタント、Louis Reasonerに伝えた。彼はその考えを実装しようと試み、Alyssaのプログラムを
以下のように変更した。

\begin{scheme}
(define (make-zero-crossings
         input-stream last-value)
  (let ((avpt (/ (+ (stream-car input-stream) 
                    last-value) 
                 2)))
    (cons-stream
     (sign-change-detector avpt last-value)
     (make-zero-crossings 
      (stream-cdr input-stream) avpt))))
\end{scheme}



これはAlyssaの計画を正しく実装していない。Louisが入れてしまったバグを見つけ
プログラムの構造を変更せずに直せ。(ヒント:\code{make\-/zero\-/crossings}の引数の数を
増やす必要がある。)
\end{quote}

\begin{quote}
\heading{\phantomsection\label{Exercise 3.76}Exercise 3.76:} 
Eva Lu Atorは\link{Exercise 3.75}におけるLouisの取り組み方を批判した。彼が書いた
プログラムはモジュラ化されていない。滑らかにする操作とゼロ交差抽出が混ざってしまって
いるためである。例えば抽出器はAlyssaが入力信号を調整するより良い手段を見つければ
変更する必要が無かった。Louisを手助けし、入力としてストリームを取り、2つの連続する
入力ストリームの要素の平均を要素とするストリームを生成する手続\code{smooth}を書け。
次に\code{smooth}をゼロ交差判定機を実装するためのコンポーネントとしてよりモジュラー化
スタイルにて用いよ。
\end{quote}

\subsection{ストリームと遅延評価}
\label{Section 3.5.4}



先の節の終わりの\code{integral}手続はどのようにストリームを用いてフィードバックループを
持つ信号処理システムをモデル化できるかを示しています。\link{Figure 3.32}に示される加算器の
フィードバックループは\code{integral}の内部ストリーム\code{int}がそれ自身を用いて定義されている
事実によりモデル化されています。

\begin{scheme}
(define int
  (cons-stream
   initial-value
   (add-streams (scale-stream integrand dt)
                int)))
\end{scheme}

\noindent
暗黙的定義のような物を扱うインタプリタの能力は\code{cons\-/stream}に組込まれている
\code{delay}に依存している。この\code{delay}無しではインタプリタは、\code{int}が既に定義されている
ことを要件とする\code{cons\-/stream}への引数両方を評価する前には\code{int}を構築することができませんでした。
一般的に、\code{delay}はストリームを用いてループを含む信号処理システムをモデル化するのに不可欠です。
\code{delay}無しでは任意の信号処理コンポーネントへの入力が、出力を生成する前に
完全に評価されるように、私達のモデルが定式化されなければなりません。

\begin{figure}[tb]
\phantomsection\label{Figure 3.34}
\centering
\begin{comment}
\heading{Figure 3.34:} An ``analog computer circuit'' that solves the equation \( dy / dt = f(y) \).

\begin{example}
                            y_0
                             |
                             V
    +----------+  dy   +----------+     y
+-->|  map: f  +------>| integral +--*----->
|   +----------+       +----------+  |
|                                    |
+------------------------------------+
\end{example}
\end{comment}
\includegraphics[width=67mm]{fig/chap3/Fig3.34.pdf}
\begin{quote}
\heading{Figure 3.34:} 方程式\( dy / dt = f(y) \)を解く``アナログ演算回路''
\end{quote}
\end{figure}

\noindent
残念なことに、ループを伴うシステムのストリームモデルは\code{cons\-/stream}により提供される
``隠れた''\code{delay}を越えて、\code{delay}の使用を要求します。例えば\link{Figure 3.34}は
\( f \)が与えられた関数である場合に微分方程式信号\( dy / dt = f(y) \)を
解く処理システムを示しています。図は\( f \)をその入力信号に適用するマッピング
コンポーネント(map)を示しています。mapはフィードバックループの中に積分器へ向けて
実際にそのような方程式を解くために利用されているアナログ計算機回路にとても似た作法で
接続されています。



\( y \)に対し初期値\( y_0 \)を与えられたとした時、このシステムを以下の手続を用いて
モデル化を試みることができるでしょう。

\begin{scheme}
(define (solve f y0 dt)
  (define y (integral dy y0 dt))
  (define dy (stream-map f y))
  y)
\end{scheme}

\noindent
この手続はうまく行きません。\code{solve}の最初の行にて\code{integral}の呼出は
入力\code{dy}が定義されてることを要求します。これは\code{solve}の二行目までは
起こり得ません。


一方で、私達の定義の意図もつじつまが合いません。原理上は、\code{y}ストリームを
\code{dy}を知らずに生成し始めることができます。再に\code{integral}や他の多くのストリーム
命令は\code{cons\-/stream}に、引数に関する部分的な情報を与えられただけで応答の部分を
生成できるという点で似た性質を持っています。\code{integral}では出力ストリームの
最初の要素は\code{initial\-/value}で与えられます。従って出力ストリームの最初の要素を
被積分関数\code{dy}を評価せずに生成することができます。一度\code{y}の最初の要素を知る
ことができれば、\code{solve}の2行目の\code{stream\-/map}は\code{dy}の最初の要素を生成する
仕事を開始できます。これにより\code{y}の次の要素を生成することもでき、以下繰り返し
となります。


この考えの利点を得るために、\code{integral}を再定義し、被積分関数ストリームが
\newterm{delayed argument}(\jnewterm{遅延引数})を要求するようにします。\code{Integral}は
出力ストリームの最初の要素より多くを生成することを要求された時のみ、被積分関数を
\code{force}し評価させます。

\begin{scheme}
(define (integral delayed-integrand initial-value dt)
  (define int
    (cons-stream
     initial-value
     (let ((integrand (force delayed-integrand)))
       (add-streams (scale-stream integrand dt)
                    int))))
  int)
\end{scheme}

\noindent
これで\code{solve}手続の実装が、\code{y}の定義内で\code{dy}を遅延させればできます。\footnote{この
手続は全てのScheme実装で動くことが保証されていません。とはいえ、任意の実装に対して簡単な
変更で動作します。問題はScheme実装の内部定義の扱い方に関係があります。(\link{Section 4.1.6}参照)}

\begin{scheme}
(define (solve f y0 dt)
  (define y (integral (delay dy) y0 dt))
  (define dy (stream-map f y))
  y)
\end{scheme}

\noindent
全体的に、\code{integral}を呼び出す者は今では被積分関数を\code{delay}しなければなりません。
\( e \approx 2.718 \)の近似値を微分方程式\( dy / dt = y \)に対する
解が\( y = 1 \)の場合の値を初期条件\( y(0) = 1 \)で求めることで\code{solve}手続が
うまく働くことを実演できます。

\begin{scheme}
(stream-ref (solve (lambda (y) y) 1 0.001) 1000)
~\textit{2.716924}~
\end{scheme}

\begin{quote}
\heading{\phantomsection\label{Exercise 3.77}Exercise 3.77:} 
上で使用された\code{integral}手続は\link{Section 3.5.2}の整数無限ストリームの``暗示的''定義に
似ている。代替的に、より\code{integers\-/starting\-/from}に似た\code{integral}の定義を
与えることができない。(これも\link{Section 3.5.2}参照)

\begin{smallscheme}
(define (integral integrand initial-value dt)
  (cons-stream
   initial-value
   (if (stream-null? integrand)
       the-empty-stream
       (integral (stream-cdr integrand)
                 (+ (* dt (stream-car integrand))
                    initial-value)
                 dt))))
\end{smallscheme}


ループを持つシステム内で利用された場合、この手続は\code{integral}の元の版が抱えた
問題と同じ問題を持つ。手続を変更して\code{integrand}に対し遅延された引数を要求する
ようにし、それ故に上で示されたように\code{solve}手続で利用できるようにせよ。
\end{quote}

\begin{quote}
\heading{\phantomsection\label{Exercise 3.78}Exercise 3.78:} 
単項二次線形微分方程式を学ぶための信号処理システムの設計問題について考えよ。
\begin{comment}

\begin{example}
d^2 y        d y
-----  -  a -----  -  by  =  0
d t^2        d t
\end{example}

\end{comment}
\begin{displaymath}
 {d^2y \over dt^2} - a {dy \over dt} - by = 0. 
\end{displaymath}



\( y \)をモデル化する出力ストリームはループを含むネットワークにより生成される。
これは\( d^2y / dt^2 \)の値が\( y \)と\( dy / dt \)の値に依存し、
これらの両方が\( d^2y / dt^2 \)を積分することにより決定されるからである。
\link{Figure 3.35}に示される図の符号化を行いたい。定数\( a \), \( b \), \( dt \)と
初期値、\( y \)に対する\( y_0 \)と\( dy_0 \)を引数として取り、\( y \)の一連の値の
ストリームを生成する手続\code{solve-2nd}を書け。
\end{quote}

\begin{figure}[tb]
\phantomsection\label{Figure 3.35}
\centering
\begin{comment}
\heading{Figure 3.35:} Signal-flow diagram for the solution to a second-order linear differential equation.

\begin{example}
               dy_0                y_0
                |                   |
                V                   V
   ddy     +----------+    dy  +----------+    y
+--------->| integral +-----*--+ integral +--*--->
|          +----------+     |  +----------+  |
|                           |                |
|            +----------+   |                |
|     __/|<--+ scale: a |<--+                |
|   _/   |   +----------+                    |
+--<_add |                                   |
     \__ |   +----------+                    |
        \|<--+ scale: b |<-------------------+
             +----------+
\end{example}
\end{comment}
\includegraphics[width=91mm]{fig/chap3/Fig3.35a.pdf}
\begin{quote}
\heading{Figure 3.35:} 二次線形微分方程式の解のための信号フロー図
\end{quote}
\end{figure}


\begin{quote}
\heading{\phantomsection\label{Exercise 3.79}Exercise 3.79:} 
\link{Exercise 3.78}の\code{solve\-/2nd}手続を一般化し、一般的な二次微分方程式\( d^2y / dt^2 =
f(dy / dt, y) \)を解くのに使用できるようにせよ。
\end{quote}

\begin{quote}
\heading{\phantomsection\label{Exercise 3.80}Exercise 3.80:} 
\newterm{series RLC circuit}(\jnewterm{連続RLC回路})は抵抗、コンデンサ、インダクタンスが\link{Figure 3.36}に
示されるように連結されている。\( R \), \( L \), \( C \)が抵抗、インダクタンス、コンデンサである
場合、3つのコンポーネントに対する電圧(\( v \))と電流(\( i \))の間の関係は以下の方程式により
説明される。
\begin{comment}

\begin{example}
v_R = i_R R

         d i_L
v_L = L -------
          d t

         d v_C
i_C = C -------
          d t
\end{example}

\end{comment}
\begin{displaymath}
 	v_R 	= 	i_R R, \qquad\quad
	v_L 	= 	L {di_L \over dt}\,, \qquad\quad
	i_C 	= 	C {dv_C \over dt}\,, 
\end{displaymath}

そして回路の接続が以下の関係を決定する。
\begin{comment}

\begin{example}
i_R = i_L = -i_C

v_C = v_L + v_R
\end{example}

\end{comment}
\begin{displaymath}
 	i_R 	= 	i_L = -i_C\,, \qquad\quad
	v_C 	= 	v_L +  v_R\,.  
\end{displaymath}


これらの方程式の組み合わせは(コンデンサに渡る電圧\( v_C \)とインダクタンスの
電流\( i_L \)にてまとめれば)回路の状態が以下の微分方程式のペアで説明されることを
示している。
\begin{comment}

\begin{example}
d v_C        i_L
-----  =  -  ---
 d t          C

d i_L      1           R
-----  =  --- v_C  -  --- i_L
 d t       L           L
\end{example}

\end{comment}
\begin{displaymath}
  {dv_C \over dt}  =  -{i_L \over C}\,, \qquad\quad
    {di_L \over dt}  =   {1   \over L} v_C - {R \over L} i_L\,. 
\end{displaymath}

この微分方程式のシステムを表す信号フロー図は\link{Figure 3.37}に示される。
\end{quote}

\begin{figure}[tb]
\phantomsection\label{Figure 3.36}
\centering
\begin{comment}
\heading{Figure 3.36:} A series RLC circuit.

\begin{example}
              + v_R -
        i_R
     +--->----'\/\/\,--------+
     |                       |  i_L
    \|/          R          \|/
  +  |  i_C                  |_   +
    -+-                       _)
v_C -+- C                     _)  v_L
     |                        _)
  -  |                       |    -
     +-----------------------+
\end{example}
\end{comment}
\includegraphics[width=60mm]{fig/chap3/Fig3.36.pdf}
\par\bigskip
\noindent
\heading{Figure 3.36:} 連続RLC回路
\end{figure}

\begin{quote}
引数として回路のパラメタ\( R \), \( L \), \( C \)と時間の増分\( dt \)を取る
手続\code{RLC}を書け。ある意味では\link{Exercise 3.73}の\code{RC}手続のそれに似ているが、
\code{RLC}は状態変数の初期値\( v_{C_0} \)と\( i_{L_0} \)を取り、(\code{cons}を用いて)
\( v_C \)と\( i_L \)の状態のストリームのペアを生成する手続を生成せねばならない。
\code{RLC}を用いて、連結RLC回路の振舞をモデル化するストリームのペアを生成せよ。
ただし\( R \) = 1[Ω], \( C \) = 0.2[F], \( L \) = 1 henry, \( dt \)
= 0.1[s], soreni初期値 \( i_{L_0} \) = 0[A], \( v_{C_0} \) = 10[V]とする。
\end{quote}

\begin{figure}[tb]
\phantomsection\label{Figure 3.37}
\centering
\begin{comment}
\heading{Figure 3.37:} A signal-flow diagram for the solution to a series RLC circuit.

\begin{example}
                 +-------------+
+----------------+  scale: l/L |<--+
|                +-------------+   |
|                                  |
|                +-------------+   |  v_C
|       dv_C +-->|   integral  +---*------>
|            |   +-------------+
|            |        ^
|            |        | v_(C_0)
|            |
|            |   +-------------+
|            +---+ scale: -l/C |<--+
|                +-------------+   |
|  |\__                            |
+->|   \_  di_L  +-------------+   |  i_L
   | add_>------>|   integral  +---*------>
+->| __/         +-------------+   |
|  |/                 ^            |
|                     | i_(L_0)    |
|                                  |
|                +-------------+   |
+----------------+ scale: -R/L |<--+
                 +-------------+
\end{example}
\end{comment}
\includegraphics[width=68mm]{fig/chap3/Fig3.37a.pdf}
\begin{quote}
\heading{Figure 3.37:} 連続RLC回路の解のための信号フロー図
\end{quote}
\end{figure}

\subsubsection*{正規順評価}

この節の例は明示的な\code{delay}と\code{force}の使用がどのようにして大きなプログラミングの
柔軟性を与えるかについて説明します。しかし同じ例がまたこのことがどのようにして私達の
プログラムをより複雑にするかについても示します。例として私達の新しい\code{integral}手続は
ループを伴うシステムをモデル化するための力を与えます。しかし今では\code{integral}が
遅延化された被積分関数と共に呼び出されなければならないことを忘れてはならなくなりました。
そして\code{integral}を使用する全ての手続はこのことについて注意しなければなりません。
実際には、手続の2つの組を作りました。通常の手続と遅延化された引数を取る手続です。
一般的に、分離された手続の組を作ることは私達に分離された高階手続の組もまた作ることを
強います。\footnote{これはPascalの様な旧来の強い方の言語が高階手続をこなす場合に持つ
困難さの(Lispにとっては)小さな反射です。そのような言語ではプログラマは必ず各手続の引数と
結果のデータ型を指定せねばなりません。数値、論理値、配列、等です。その結果として
``与えられた手続\code{proc}を列の全ての要素にmapする''ような抽象化を\code{stream\-/map}のような
単一の高階手続にて表すことができませんでした。それどころか\code{proc}に
対して指定されるかもしれない異なる引数と結果のデータ型の組み合わせ全てに対して
異なるマッピング手続を必要としました。高階手続の存在における``データ型''の実用的な
概念を維持することは多くの困難な問題を提起しました。この問題を処理する1つの方法は
言語ML(\link{Gordon et al. 1979})により説明され、その``多層データ型''はデータ型間の
高階変換のためのテンプレートを含んでいます。さらにMLのほとんどの手続データ型は
明示的にプログラマにより宣言されることはありません。その代わりにMLは
\newterm{type\-/inferencing}(\jnewterm{型推論})メカニズムを含み、環境の情報を用いて新しく定義された
手続のデータ型を推論します。}


2つの異なる手続の組の必要性を防ぐ1つの方法は全ての手続に対し遅延引数を取らせることです。
手続に対する全ての引数が自動的に遅延化され、引数が実際に必要とされる時(例えばプリミティブ
命令に要求された時)強制される評価のモデルを受け入れることができるでしょう。これは
私達の言語を正規順評価を用いるように変形します。これは私達が\link{Section 1.1.5}の評価の置換
モデルを紹介した時、最初に説明しました。正規順評価への変換は統一、洗練された方法で
遅延評価の利用を簡易化します。そしてこれはストリーム処理のみについて考慮するのならば
受け入れるべき自然な戦略です。\link{Section 4.2}では評価機を学んだ後に、私達の言語をどのように
してこのように変形するのかについて学びます。残念なことに手続呼出に遅延を導入することは
イベント順に依存するプログラムを設計する能力に混乱をもたらします。例えば代入を利用する、
データを変更する、入出力を実行するプログラムです。例え1つの\code{cons\-/stream}内の\code{delay}でも
\link{Exercise 3.51}や\link{Exercise 3.52}で説明された大きな混乱を招きます。誰もが知っている
ように、変更可能性と遅延評価はプログラミング言語の中でうまく混ざりません。
そしてこれらの両方を一度に取り扱う方法の発明は活発な研究領域です。

\subsection{関数型プログラムのモジュール化とオブジェクトのモジュール化}
\label{Section 3.5.5}


\link{Section 3.1.2}で学んだように、代入の導入の主な利点の1つは巨大システムの状態の一部を
ローカル変数の中にカプセル化、または``隠す''ことによりシステムのモジュール化の
容易性を増すことができます。ストリームモジュールは同等なモジュールの容易性を
代入の使用成しに提供可能です。例として\( \pi \)のモンテカルロ推定を再実装して
みましょう。\link{Section 3.1.2}にてこれをストリーム処理の視点から試しました。



モジュール化容易性の鍵となる問題は、乱数生成器の内部状態を乱数を使用するプログラムから
隠したいと願ったことです。手続\code{rand\-/update}から始めました。これの連続する値が
私達の乱数を供給し、そしてこれを乱数生成器を作り出すのに使用しました。

\begin{scheme}
(define rand
  (let ((x random-init))
    (lambda ()
      (set! x (rand-update x))
      x)))
\end{scheme}

\noindent
ストリームの定式化においては乱数生成器が単体では存在しません。乱数のストリームがただ
\code{rand\-/update}を連続して呼ぶことで生成されます。

\begin{scheme}
(define random-numbers
  (cons-stream random-init
               (stream-map rand-update random-numbers)))
\end{scheme}

\noindent
これを用いて乱数ストリームにおける連続したペア上で行われたCes\`aro(チェザロ)の実験
の結果のストリームを構築します。

\begin{scheme}
(define cesaro-stream
  (map-successive-pairs
   (lambda (r1 r2) (= (gcd r1 r2) 1))
   random-numbers))
(define (map-successive-pairs f s)
  (cons-stream
   (f (stream-car s) (stream-car (stream-cdr s)))
   (map-successive-pairs f (stream-cdr (stream-cdr s)))))
\end{scheme}

\noindent
\code{cesaro\-/stream}が次に\code{monte\-/carlo}手続に与えられます。これは確率の推測のストリームを
生成します。するとその結果は\( \pi \)の推測値のストリームへと変換されます。
このプログラムのこの版は何回試行を行うかのパラメタが必要ありません。より良い\( \pi \)の
推測値(より多くの試行からの)はより多くの\code{pi}ストリームを見ることで得られます。

\begin{scheme}
(define (monte-carlo experiment-stream passed failed)
  (define (next passed failed)
    (cons-stream
     (/ passed (+ passed failed))
     (monte-carlo
      (stream-cdr experiment-stream) passed failed)))
  (if (stream-car experiment-stream)
      (next (+ passed 1) failed)
      (next passed (+ failed 1))))
(define pi
  (stream-map
   (lambda (p) (sqrt (/ 6 p)))
   (monte-carlo cesaro-stream 0 0)))
\end{scheme}

\noindent
考慮すべきモジュール化容易性がこの取り組み方には存在します。なぜなら依然として
任意の実験を取り扱うことが可能な一般的な\code{monte\-/carlo}手続を定式化できるためです。
その上、代入やローカル変数が存在しません。

\begin{quote}
\heading{\phantomsection\label{Exercise 3.81}Exercise 3.81:} 
\link{Exercise 3.6}は乱数生成器に乱数列のリセットを許可することで``ランダム''な
数の列を繰り返し生成させる一般化について議論した。
入力ストリームの要求に従い操作する これと同じ生成器のストリーム定式化を実現せよ
ストリームの要素が\code{generate}なら新しい乱数を生成し、また\code{reset}なら指定された値に列をリセットすることで
希望の乱数列を生成する。
代入は使用しないこと。
\end{quote}

\begin{quote}
\heading{\phantomsection\label{Exercise 3.82}Exercise 3.82:} 
\link{Exercise 3.5}のモンテカルロ積分をストリームを用いて再度行え。
ストリーム版の\code{estimate\-/integral}は何度試行を行うのか伝える引数は持たない。
その代わり連続するより多くの試行を基に推測値のストリームを生成する。
\end{quote}

\subsubsection*{時間の関数型プログラミング的視点}


さて、この章の始めに提起されたオブジェクトと状態の問題に戻り新しい光の下で調査
しましょう。私達は代入とミュータブルオブジェクトを導入し状態を持つシステムのモデル化
を行うプログラムのモジュラー方式の構築のための仕組みを提供しました。ローカル状態変数
を持つ計算オブジェクトを構築し、代入を用いてこれらの変数を変更しました。
世界のオブジェクトの一時的な振舞を相当する計算オブジェクトの一時的な振舞により
モデル化しました。



今までストリームが局所状態を持つオブジェクトのモデル化する代替法を提供することを学んで
きました。何らかのオブジェクトの局所状態のような変化する数量を、連続する状態の時刻歴を
表現するストリームを用いてモデル化できます。
本質的に、私達はストリームを用いることで時間を明示的に表現しています。そうすることで
私達のシミュレートされた世界の時間を評価の間に現れる一連のイベントから分断しています。
実際に、\code{delay}の存在のため、モデルのシミュレートされた時間と評価中のイベントの順の
間には何の関係も無いでしょう。



これらの2つのモデル化の取り組み方を対比するために、``銀行の引き出し機''の実装について
再考してみましょう。これは銀行口座の残高を監視します。\link{Section 3.1.3}ではそのような処理機
の単純化された版を実装しました。

\begin{scheme}
(define (make-simplified-withdraw balance)
  (lambda (amount)
    (set! balance (- balance amount))
    balance))
\end{scheme}

\noindent
\code{make\-/simplified\-/withdraw}への呼出は計算オブジェクトを生成します。各計算
オブジェクトは局所状態変数\code{balance}を個別に持ち、そのオブジェクトを続けて呼ぶと
\code{balance}は減少します。銀行口座のユーザが連続したそのようなオブジェクトに対する
入力を打ち、ディスプレイの画面に表われる一連の返り値を観察するのを想像できます。


代替的に、引き出し処理機を入力として残高と引き出す金額のストリームを取り口座の一連の
残高のストリームを生成する手続としてモデル化することが可能です。

\begin{scheme}
(define (stream-withdraw balance amount-stream)
  (cons-stream
   balance
   (stream-withdraw (- balance (stream-car amount-stream))
                    (stream-cdr amount-stream))))
\end{scheme}

\noindent
\code{stream\-/withdraw}は明確な数学上の関数を実装し、関数の出力はその入力のみにより
完全に決定します。しかし入力\code{amount\-/stream}がユーザにより打鍵された一連の値の
ストリームであり結果の残高ストリームが表示されたと考えてみて下さい。すると、
値を入力し結果を見ているユーザの視点からはストリーム処理が\code{make\-/simplified\-/withdraw}に
より作成されたかのように、同じ振舞をしています。
しかしストリーム版では代入が無く、局所状態変数が無く、それ故に\link{Section 3.1.3}で遭遇した
論理的な困難に存在しません。それにもかかわらずシステムは状態を持っています!



これは本当に驚くべきことです。\code{stream\-/withdraw}は明確な数学上の関数を実装し
その振舞は代わらないのに、ここでのユーザの知覚はシステムとの相互作用の1つであり変化
する状態を持ちます。このパラドックスを解決する1つの方法はユーザの一時的な存在が
システムに状態を与えていると認識することです。もしユーザが相互作用から一歩離れて
個々の取引でなく、残高のストリームに関して考えれば、システムはステートレス(状態が
無く)として現われるでしょう。\footnote{物理でも同様に、私達が移動点を観察する時、
点の位置(状態)は変化していると言えます。しかし、移動点の時空の世界線の視点からは
何の変化も起こってはいません。}



複雑な処理のある部分の視点からは、他の部分は時間と共に変化するように見えます。
それらは隠された時間と伴に変化する局所状態を持ちます。もし私達がこの種の自然な
分解を私達の世界において(世界の一部である私達の視点から見たままに)計算機内の
モデル化したプログラムを計算機内の構造を用いて書きたいのならば、関数型でない
計算オブジェクトを作成します。それらは時間と共に変化します。状態を局所状態変数で
モデル化し、そして状態の変化をそれらの変数への代入を用いてモデル化します。
これを行うことにより計算の実行時間を、私達がその一部である世界の時間のモデルと
し、従って私達は``オブジェクト''を計算機の中に得ることになります。



オブジェクトを用いるモデリングは強力、かつ直感的です。その理由の多くはこれが
私達がその一部である世界との相互作用の視点に合うためです。
しかしこの章を通して繰り返し学んできた様に、これらのモデルは悩ましい
イベント順の制約とと複数の処理間の同期の問題を提起します。
これらの問題を防ぐ可能性から\newterm{functional programming languages}(\jnewterm{関数型プログラミング言語})の
開発が促進されてきました。これは代入や変更可能なデータを提供しません。そのような言語では
全ての手続は引数の明確な数学の関数を実装し、その振舞は変化しません。関数型の取り組み方は
並行システムを扱うのにとても魅力的です。\footnote{Fortranの開発者であるJohn Backusは
1978年に\acronym{ACM}のチューリング賞を授与された時に関数型プログラミングに高い知名度を
与えました。彼の受賞スピーチ(\link{Backus 1978})は関数型のアプローチを強く支持しました。
関数型プログラミングの良い概観は\link{Henderson 1980}と\link{Darlington et al. 1982}で
与えられます。}


一方で、もしきっちりと見てみれば時間に関係する問題が関数型のモデルにも潜んでいることが見えます。
ある特に厄介な領域がインタラクティブシステム(応答システム)を設計したい時に、特に独立した要素の間で相互作用を
行うシステムにおいて提起されます。例として、もう1度連結銀行口座を許可する銀行システムについて
考えてみましょう。代入とオブジェクトを用いる保守的なシステムでは、PeterとPaulが口座を共有している
という事実をモデル化します。共有は\link{Section 3.1.3}で見たように、PeterとPaulの両者が彼等の取引要求を
同じ銀行口座オブジェクトに送ることにより行われます。ストリームの視点からは、``オブジェクト''それ自身は
無いため、銀行口座を取引要求の操作を行う処理としてモデル化し、応答のストリームを生成できる
ことが既に示されています。従って、PeterとPaulが連結銀行口座を持つことが、\link{Figure 3.38}で示すように
Peterの取引要求ストリームとPaulの取引要求リクエストをマージし、その結果を銀行口座ストリーム処理へ渡すと
いう事実をモデル化することができるでしょう。

\begin{figure}[tb]
\phantomsection\label{Figure 3.38}
\centering
\begin{comment}
\heading{Figure 3.38:} A joint bank account, modeled by merging two streams of transaction requests.

\begin{example}
Peter's requests   +---------+     +---------+
------------------>|         |     |         |
Paul's requests    |  merge  |---->| bank    |---->
------------------>|         |     | account |
                   +---------+     +---------+
\end{example}
\end{comment}
\includegraphics[width=88mm]{fig/chap3/Fig3.38.pdf}
\begin{quote}
\heading{Figure 3.38:} 取引要求リクエストの2つのストリームをマージすることでモデル化した連結銀行口座
\end{quote}
\end{figure}

\noindent
この定式化に伴う問題は\newterm{merge}(\jnewterm{マージ})という概念にあります。これは2つの
ストリームを単純に交互にPeterの要求を1つ、Paulの要求を1つと取りマージすることは
しません。Paulが口座にとても稀にしかアクセスしないと考えてみましょう。
Peterに対し、彼が2つ目の取引を発行できる前にPaulが口座にアクセスするのを待つよう
強いることはできません。しかしそのようなマージが実装された場合、
PeterとPaulにより知覚される``実時間''に制約された何らかの方法で2つの取引ストリームを
相互配置しなければなりません。何らかの方法とはもしPeterとPaulが会えば、いくつかの取引が会う
前に処理され、他の取引が会った後に処理されることに合意できるという意味でです。\footnote{任意の
2つのストリームに対し一般的に複数の受け入れ可能な相互配置の順が存在することに
注意して下さい。従って技術的には``マージ''は関数ではなく関係です。その答は
入力の決定的な関数ではありません。私達は既に(\link{Footnote 39})で非決定論が
並行の扱いの本質であると述べました。マージの関係は同じ本質的な非決定論を関数型の
視点から説明します。\link{Section 4.3}では非決定論をまた別の視点から見ることになります。}
これは正確に、\link{Section 3.4.1}で扱わねばならなかったのと同じ制約です。そこでは状態を
持つオブジェクトの並行処理におけるイベントの``正しい''順を保証する明示的な同期の
導入の必要性が見つかりました。従って、関数型のスタイルをサポートする取り組みにおいて、
異なる要因からの入力のマージの必要性は関数型のスタイルが排除するはずだった同じ問題
を再び導入します。


私達はこの章を、私達がモデル化しようとする実際の世界の私達の視点に合う構造を持つ
計算モデルの構築をゴールとして始めました。
世界を分離した、時間制約のある、相互応答する、状態を持つオブジェクトでモデル化できます。
または世界を単一の、時間制約のない、状態の無い、個体によりモデル化できます。
それぞれの視点が強力な利点を持ちますが、どちらの視点も単独では十分ではありません。
大統一は未だ現われてはいません。\footnote{オブジェクトモデルは世界を分割し分離した
部分にすることで近似します。関数型モデルはオブジェクト境界に従ってモジュラ化はしません。
オブジェクトモデルは``オブジェクト''の非共有下の状態が共有されている状態よりも
とても大きい場合に便利です。オブジェクトの視点が失敗する場合の例は量子力学です。
そこでは物を個別の点として考えることは逆説と混乱を招きます。オブジェクトの視点を
関数型の視点と統一することはプログラミングとはあまり関係が無いかもしれません。
しかしより根本的な認識論の問題と関係するのです。}


\chapter{メタ言語抽象化}
\label{Chapter 4}

\begin{quote}


\( \dots \) 魔法とは言葉の中にある---アブラカダブラ、開けゴマ、その他もろもろ---しかしあるお話の
魔法の言葉は次のお話では魔法ではない。真の魔法とはどの言葉が、いつ、何に対して働くか
を知ることだ。トリックを学ぶことがトリックなんだ。



\( \dots \) そしてそれらの言葉は私達のアルファベットの文字から出来ている。
ペンで書ける2、3ダースの走り書きだ。これが鍵なんだ!
そして宝でもある、もしそれに手をつけることさえできれば!
それはまるで---まるで宝の鍵こそが宝のようだ!

---John Barth, \textit{Chimera}
\end{quote}

\vspace{0.8em}

\noindent
プログラム設計の学習において、エキスパートなプログラマが設計の複雑さを
全ての複雑なシステムの設計者が用いるのと同じ一般的な技術を用いてコントロール
することを学んできました。彼等はプリミティブな要素を接続して複合オブジェクトを
形成し、複合オブジェクトを抽象化することでより高いレベルの建築ブロックを形成し
そして適切な大規模のシステム構造の見方を受け入れることでモジュール化方式は維持しました。
これらのテクニックの説明において、私達はLispをプロセスを記述するための言語として
用い、また計算データオブジェクトと実世界の複雑な現象をモデル化する処理を構築する
ためにも用いてきました。しかし、複雑さを増す問題に取り組むにつれ、Lisp、または
どのような固定されたプログラミング言語も、我々の必要には十分でないことを知る
ことでしょう。私達は、私達の考えをより効果的に表現するために、耐えず新しい言語に
向かわねばなりません。新しい言語を定めることは工学上の設計の複雑さをコントロール
するための強力な戦略です。私達は良く、問題を異なった方法
で記述できる(そしてそれ故に考えることができる)新しい言語を受け入れることで、
複雑な問題への対処能力を拡張することができます。プリミティブな、組み合わせの
手段や抽象化の手段を、目前の問題に特によく合ったものを用います。\footnote{同じ考えが
工学全てに渡り普及しています。例えば電子工学は多くの異なる言語を回路の記述に
用います。これらの内2つは電子ネットワークの言語と電子システムの言語です。
ネットワーク言語は別個の電子素子に関する装置の物理モデリングを重視します。
ネットワーク言語のプリミティブなオブジェクトはプリミティブな抵抗や、キャパシタ(コンデンサ)、
コイルやトランジスタ等の電子コンポーネントであり電圧と電流と呼ばれる物理的変数を
用いて特徴付けられます。回路をネットワーク言語で記述する時、技術者は設計の物理
特性に関心を持ちます。逆に、システム言語のプリミティブなオブジェクトはフィルタや
アンプのような信号処理モジュールです。モジュールの機能上の振舞のみが関係し、信号は
それらの電圧や電流のような物理的な認識に関心を持ちません。
信号処理システムの要素が電子ネットワークから構築
される意味の上ではシステム言語はネットワーク言語の上に組み立てられます。
しかしここでは関心事は与えられた応用問題を解くための大規模な電子装置の編成にあります。
パーツの物理的実現可能性は当然と考えられています。この階層化された言語の集合は
\link{Section 2.2.4}のピクチャー言語にて説明された階層化された設計テクニックのまた別の
例になります。}



プログラミングは数多くの言語により生じます。特定のコンピュータのための
機械語のような物理言語も存在します。これらの言語は個別のストレージの断片と
プリミティブな機械命令を用いてデータとコントロールの表現に関係します。
機械語プログラマは与えられたハードウェアの使用に関心を持つことでリソースに
限りある演算の効率的な実装のためのシステムとユーティリティを組み立てます。
高級言語は機械語の素地の上にありますが、データをビットの集合として表したり、
プログラムをプリミティブな命令の列で表すという懸念を隠します。これらの言語は
手続定義のような組み合わせと抽象化の手段を持ち大規模なシステム構成に適しています。



\newterm{Metalinguistic abstraction}(\jnewterm{メタ言語抽象化})---新しい言語を構築すること---が
工学設計の全ての部門にて重要な役割を果たします。これは計算機プログラミングでとても
重要です。プログラミングでは新しい言語を形成するだけでなく、これらの言語を評価機を
構築することで実装することもできるからです。プログラミング言語の\newterm{evaluator}(\jnewterm{評価機})(または
\newterm{interpreter}(\jnewterm{インタプリタ}))は手続であり、言語の式に対して適用された時、
その式を評価するために要求される行動を実行します。



プログラミングにおける考えで最も根本的な物と見做すことに何の誇張もありません。

\begin{quote}
評価機はプログラミング言語の評価手段を決定するが、それ自体は別のプログラムである。
\end{quote}

\noindent
この点を理解することはプログラマとしての私達自身のイメージを変更することです。
私達は私達自身を、他人が設計した言語のユーザとしてのみではなく、
言語の設計者として見る時点に辿りつきました。


実際に、私達はほとんど全てのプログラムをある言語の評価機だと見做すことができます。
例えば、\link{Section 2.5.3}の多項式操作システムは多項式の数値演算のルールを具象化し、
リスト構造データ上の命令を用いて実装しました。もし私達がこのシステムを多項式を
読み込み、表示する手続と共に拡張したなら、記号数学の問題を扱う特定目的言語の
コア(核)を持つことになります。\link{Section 3.3.4}のデジタル論理シミュレータと\link{Section 3.3.5}の
制約伝播はそれら自身の正しさにおける論理的言語であり、それぞれがそれ自身の
プリミティブと組み合わせの手段、抽象化の手段を持ちます。この視点から見れば
大規模計算機システムをこなす技術は新しい計算機言語を構築する技術と結合し、
計算機科学それ自身が適切な記述言語を構築する分野それ以上でも以下でもなくなります。



私達は今から他言語を用いて言語が構築される技術を巡り始めます。この章では
Lispを基盤として用い、評価機をLispの手続として実装します。Lispはこの任務に
とても良く合います。記号式を表現し、操作する能力がその理由です。
私達はLisp自身の評価機を構築することで言語がどのように実装されているかを
理解することから最初の一歩を踏み出します。私達の評価機により実装される
言語はこの本で用いるLipsのScheme方言の部分集合となります。この章で説明される
評価機がLispの特定の方言に向けて書かれていても、逐次式計算機のプログラムを
書くために設計された任意の式指向言語のための評価機の本質的な構造を含みます。
(実際に、多くの言語処理機がそれらの奥深くに小さな``Lisp''評価機を含んでいるのです。)
評価機は説明と議論のために簡略化されており、製品品質のLispシステム
に含まれるべき重要な機能が省略されています。
それにもかかわらず、この単純な評価機はこの本に現われる多くのプログラムを実行するのに
適しています。\footnote{私達の評価機が取り除いた最も重要な機能はエラーを扱う仕組みと
デバッグのサポートです。評価機のより広範囲の議論については\link{Friedman et al. 1992}を
参照して下さい。これはSchemeで書かれた一連の評価機を通して進められたプログラミング言語
の解説を与えます。}


評価機をLispプログラムとして利用可能にする重要な利点は代替となる評価ルールを
評価機プログラムへの変更として記述することで実装できることです。
この力を良い効果として用いることが可能な箇所として、\link{Chapter 3}の議論の
まさに中心であった、計算モデルが時間の概念を統合する方法に対し特別な
コントロールを得ることです。そこではストリームを用いて世界の時間表現を計算機の
時間から分離することで、状態と代入の複雑さのいくらかを緩和しました。
しかし、私達のストリームプログラムは時々扱いにくい物でした。Schemeの評価の
適用順により制約されていたためです。
\link{Section 4.2}ではより洗練された取り組み方を準備するために、
\newterm{normal-order evaluation}(\jnewterm{正規順評価})に対応する様に評価機を変更することで
基盤となる言語を変更します。


\link{Section 4.3}では式が単一の値のみでなく多くの値を持つ場合において、より野心的な言語の変更を実装します。
この\newterm{nondeterministic computing}(\jnewterm{非決定的演算})の言語においては、式の
全ての可能な値を生成する過程を生成し、次にそれらの値からいくつかの
制約を充足する値を探索することが自然に表現できます。。計算と時間のモデルにを用いれば、これは``可能な未来''の集合
を成す時間の分岐を持ち、次に適切な時系列を探すような物です。私達の
非決定的評価機を用いる複数の値の追跡と探索の実行は、根底に存在する言語の
仕組みにより自動的に取り扱われます。



\link{Section 4.4}では\newterm{logic\-/programming}(\jnewterm{論理プログラミング})言語を実装します。
それにより知識が入出力を伴なう計算を用いてではなく、関係性を用いて表現されます。
これは言語をLispから、または本当に全ての従来の言語から大幅に異なる物にしますが、
論理プログラミング評価機がLisp評価機の本質的な構造を共有することを学びます。



\section{メタ循環評価機}
\label{Section 4.1}



私達のLisp評価機はLispプログラムとして実装されます。LispプログラムをLispで実装
された評価機を用いて評価することについて考えることは循環論に見えるかもしれません。
しかし評価はプロセス(処理、過程)であり、従って評価過程をLispを用いて説明することは適切
です。Lispは結局の所、プロセスを記述するためのツールなのです。\footnote{例え
そうだとしても、私達の評価機により説明されない評価プロセスの重要な側面が残り
ます。これらの最も重要なことは手続が他の手続を呼び出し、そしてそれらを呼び出した
物に値を返す原因となる詳細な仕組みです。これらの問題は\link{Chapter 5}で解明
します。そこで私達は評価機を簡単なレジスタマシンとして実装することで評価プロセスに
より詳細に調べます。}評価する対象と同じ言語で書かれた評価機は
\newterm{metacircular}(\jnewterm{メタ循環})と呼ばれます。


メタ循環評価機は本質的には\link{Section 3.2}で説明された評価の環境モデルのScheme形式化です。
モデルには以下の2つの基本的パーツがあることを思い出して下さい

\begin{enumerate}

\item
組み合わせ(特殊な形式を除く複合式)を評価するためには、部分式を評価し、次に
オペレータ部分式をオペランド部分式の値に適用する。

\item
複合手続を引数の集合に適用するためには、手続のボディを新しい環境で評価する。
この環境を構築するためには、手続オブジェクトの環境部分をフレームにより拡張する。
フレームの中ではその手続の形式パラメタが、その手続が適用される引数に対して
束縛される
\end{enumerate}

\noindent
これら2つのルールが評価プロセスの本質を説明します。環境の中で式が評価される
基本的なサイクルは引数に適用される手続に簡約され、引数は順に新しい環境で評価される
新しい式へと簡約され、以下、値がその環境の中で見つかるシンボルか直接適用される
プリミティブな手続(\link{Figure 4.1}参照)に辿り着くまで繰り返されます。\footnote{
もし私達自身にプリミティブを適用する能力を与えるのであれば、評価機の実装には何が残って
いるのでしょうか? 評価機の仕事は言語のプリミティブを指定することではなく、
結合組織---組み合わせと抽象化の手段---を提供することであり、それがプリミティブの
集合を言語を形成するために束縛します。具体的には、



\( \bullet \) 評価機は入れ子の式の取扱を許可します。例えば単純にプリミティブを適用することは
式\code{(+ 1 6)}を評価するのには十分ですが、\code{(+ 1 (* 2 3))}を取り扱うには十分ではありません。
プリミティブな手続\code{+}が対象である限り、その引数は数値でなければならなず、もし
式\code{(* 2 3)}を引数として渡せば失敗します。評価機の重要な役割の1つは手続合成を演出することで、
\code{(* 2 3)}を\code{+}に引数として渡す前に6に簡約します。



\( \bullet \) 評価機は変数の使用を許可します。例えば加算のためのプリミティブな手続は
\code{(+ x 1)}のような式に対応する手段を持ちません。私達は評価機に変数を追跡しその値を
プリミティブな手続を実行する前に得るようにする必要があります。



\( \bullet \) 評価機は複合手続の定義を許可します。これは手続定義の追跡を含み、
これらの手続を式評価においてどのように使用するかを知っています。そして手続に
引数を受け入れることを許可する仕組みを提供します。


\( \bullet \) 評価機は特殊形式を提供します。これは手続呼出と異なった形で評価されねば
なりません。} 


この評価サイクルは評価機内の2つの重大な手続、\code{eval}と\code{apply}の間の相互作用により
具体化されます。これらの手続は\link{Section 4.1.1}にて説明されます。(\link{Figure 4.1}参照)



評価機の実装は評価される式の\newterm{syntax}(\jnewterm{構文})を定義する手続に依存します。
私達はデータ抽象化を用いて評価機を言語の表現に非依存にします。
例えば代入はシンボル\code{set!}で始まるリストにより表現されるべきという選択に
委ねるのではなく、代入のためのテストに抽象述語\code{assignment?}を用い、そして
代入の部品にアクセスするために抽象セレクタ\code{assignment\-/variable}と\code{assignment\-/value}を
用います。式の実装については\link{Section 4.1.2}で詳細に説明されます。また\link{Section 4.1.3}で
説明される``命令''もあり、これは手続と環境の表現を指定します。例えば\code{make\-/procedure}は
複合手続を構築し、\code{lookup\-/variable\-/value}は変数の値にアクセスし、
\code{apply\-/primitive\-/procedure}はプリミティブな手続を与えられた引数のリストに対し
適用します。

\begin{figure}[tb]
\phantomsection\label{Figure 4.1}
\centering
\begin{comment}
\heading{Figure 4.1:} The \code{eval}-\code{apply} cycle exposes the essence of a computer language.

\begin{example}
                           .,ad88888888baa,
                  _    ,d8P"""        ""9888ba.      _
                 /  .a8"          ,ad88888888888a   |\
               /   aP'          ,88888888888888888a   \
              /  ,8"           ,88888888888888888888,  \
             |  ,8'            (888888888888888888888, |
            /  ,8'             `8888888888888888888888  \
            |  8)               `888888888888888888888, |
Procedure,  |  8                  "88888 Apply 8888888) | Expression
Arguments   |  8     Eval          `888888888888888888) | Environment
            |  8)                    "8888888888888888  |
            \  (b                     "88888888888888'  /
             | `8,                     8888888888888)  |
             \  "8a                   ,888888888888)  /
              \   V8,                 d88888888888"  /
              _\| `8b,             ,d8888888888P' _/
                     `V8a,       ,ad8888888888P'
                        ""88888888888888888P"
                             """"""""""""

                               [graphic by Normand Veillux, modified]
\end{example}
\end{comment}
\includegraphics[width=100mm]{fig/chap4/Fig4.1.pdf}
\begin{quote}
\heading{Figure 4.1:} \code{eval}-\code{apply}サイクルがコンピュータ言語の本質を顕在化させる
\end{quote}
\end{figure}



\subsection{評価機の核}
\label{Sec. 4.1.1}
\label{Section 4.1.1}



評価プロセスは2つの手続\code{eval}と\code{apply}の相互作用であると説明可能です。

\subsubsection*{Eval}



\code{eval}は引数として式と環境を取ります。これは式と分類しその評価を監督します。
\code{eval}は評価される式の構文上の方の事例分析として構造化されます。
手続の一般性を保つため、式の型の決定を抽象的に表現し、多種の式に対するどんな
特定の表現にも委託しません。式の各型はそれをテストする述語と、その部分を選択する
抽象手段を持ちます。この\newterm{abstract syntax}(\jnewterm{抽象構文})は同じ評価機を用いつつ、
異なる構文手続の集合と合わせることで、言語の文法をどのようにして変更できるかについて
知ることを簡単にします。

\noindent
\textbf{プリミティブな式}

\begin{itemize}

\item
数値のような自己評価式に対しては\code{eval}は式それ自身を返す。

\item
\code{eval}は環境の中で変数をその値を見つけるために探さなければならない。

\end{itemize}

\noindent
\textbf{特殊形式}

\begin{itemize}

\item
クォートされた式に対しては\code{eval}はクォートされた式を返す。

\item
変数への代入(または定義)は再帰的に\code{eval}を呼び出し変数に関連付けられる新しい値を
計算しなければならない。環境は変数の束縛を変更(または作成)しなければならない。

\item
\code{if}式はその部品に対し特別な処理を要求する。もし述語が真であれば
consequent(結果)を評価し、そうでなければalternative(代替)を評価するためである。

\item
\code{lambda}(ラムダ)式は適用可能な手続に変形しなければならない。変形はラムダ式により
指定されたパラメタとボディを評価の環境と共にパッケージ化することにより行う。

\item
\code{begin}式はその一連の式をそれらが現れる順で評価する必要がある。

\item
事例分析(\code{cond})は入れ子の\code{if}式に変形し、それから評価する。

\end{itemize}

\noindent
\textbf{組み合わせ}

\begin{itemize}

\item
手続の適用に対して、\code{eval}は再帰的に組み合わせの演算子とオペランドの部分を評価
しなければならない。結果となる手続と引数は\code{apply}に渡す。これは実際の
手続適用を取り扱う。

\end{itemize}

\noindent
以下に\code{eval}の定義を示します。

\begin{scheme}
(define (eval exp env)
  (cond ((self-evaluating? exp) exp)
        ((variable? exp) (lookup-variable-value exp env))
        ((quoted? exp) (text-of-quotation exp))
        ((assignment? exp) (eval-assignment exp env))
        ((definition? exp) (eval-definition exp env))
        ((if? exp) (eval-if exp env))
        ((lambda? exp)
         (make-procedure (lambda-parameters exp)
                         (lambda-body exp)
                         env))
        ((begin? exp)
         (eval-sequence (begin-actions exp) env))
        ((cond? exp) (eval (cond->if exp) env))
        ((application? exp)
         (apply (eval (operator exp) env)
                (list-of-values (operands exp) env)))
        (else
         (error "Unknown expression type: EVAL" exp))))
\end{scheme}

\noindent
明快さのために、\code{eval}は\code{cond}を用いた条件分岐として実装されています。
これの欠点は手続がいくつかの判別可能な式の型のみを取り扱い、\code{eval}の
定義を編集すること無しに新しい式が定義できないことです。多くのLisp実装では
式の型に従う呼出はデータ適従スタイルにより行われています。これはユーザに
\code{eval}が判別可能な新しい式の型の追加を許可します。\code{eval}自身の定義の
変更は必要有りません。(\link{Exercise 4.3}参照)

\subsubsection*{Apply}


\code{apply}は2つの引数、手続と手続が適用されるべき引数のリストを取ります。
\code{apply}は手続を2つ種類に分類します。プリミティブの適用には
\code{apply\-/primitive\-/procedure}を呼びます。複合手続の適用には手続のボディを
作る式を連続して評価することにより行います。複合手続のボディの評価のための
環境は手続により運ばれた基礎環境を拡張することで構築し、手続のパラメタを手続が
適用される引数に束縛するフレームを含めます。以下が\code{apply}の定義です。

\begin{scheme}
(define (apply procedure arguments)
  (cond ((primitive-procedure? procedure)
         (apply-primitive-procedure procedure arguments))
        ((compound-procedure? procedure)
         (eval-sequence
           (procedure-body procedure)
           (extend-environment
             (procedure-parameters procedure)
             arguments
             (procedure-environment procedure))))
        (else
         (error
          "Unknown procedure type: APPLY" procedure))))
\end{scheme}

\subsubsection*{手続の引数}



\code{eval}が手続適用を処理する時、\code{list\-/of\-/values}を用いて手続が適用される引数の
リストを生成します。\code{list\-/of\-/values}は引数として組み合わせのオペランドを取ります。
各オペランドを評価し対応する値のリストを返します。\footnote{\code{eval}の
\code{application?}節は明示的に\code{list\-/of\-/values}手続を書くのではなく、\code{map}を
用いることで(そして\code{operands}がリストを返すよう規定することで)より単純にすることが
できました。ここでは\code{map}を用いないことを選択することで高階手続を用いなくとも、
例え評価機がサポートする言語が高階手続をサポートすることになっても、評価機が
実装できることを強調しました。(従って高階手続を持たない言語で評価機を書くことも可能です)。}

\begin{scheme}
(define (list-of-values exps env)
  (if (no-operands? exps)
      '()
      (cons (eval (first-operand exps) env)
            (list-of-values (rest-operands exps) env))))
\end{scheme}

\subsubsection*{条件文}


\code{eval\-/if}は与えられた環境において\code{if}式の述語部分を評価します。
もし結果が真なら\code{eval\-/if}はconsequent(結果)を評価し、そうでなければ
alternative(代替)を評価します。

\begin{scheme}
(define (eval-if exp env)
  (if (true? (eval (if-predicate exp) env))
      (eval (if-consequent exp) env)
      (eval (if-alternative exp) env)))
\end{scheme}

\noindent
\code{eval\-/if}内での\code{true?}の使用は実装言語と被実装言語の間の接続の問題を強調します。
\code{if\-/predicate}は被実装言語にて評価されるのでその言語の値を生じます。
インタプリタの述語\code{true?}はその値を実装言語の\code{if}でテストできる値に翻訳します。
真実性のメタ循環表現は根底をなすSchemeのそれとは同じではないかもしれません。\footnote{今回は
実装言語と被実装言語は同じです。ここでの\code{true?}の意味に対する熟考は本質を誤解
することなく理解の発展を促します。}

\subsubsection*{列}



\code{eval\-/sequence}は\code{apply}により用いられ手続のボディの中にある連続した式を
評価します。また\code{eval}でも使用され\code{begin}式の中の一連の式を評価します。
引数として一連の式と環境を取り、式が現われる順で評価します。返り値は最後の
式の値です。

\begin{scheme}
(define (eval-sequence exps env)
  (cond ((last-exp? exps)
         (eval (first-exp exps) env))
        (else
         (eval (first-exp exps) env)
         (eval-sequence (rest-exps exps) env))))
\end{scheme}

\subsubsection*{代入と定義}


以下の手続は変数への代入を扱います。\code{eval}を呼び代入される値を見つけ
値と結果となる変数を\code{set\-/variable\-/value!}へ転送することで指定された
環境へ設定されるようにします。

\begin{scheme}
(define (eval-assignment exp env)
  (set-variable-value! (assignment-variable exp)
                       (eval (assignment-value exp) env)
                       env)
  'ok)
\end{scheme}

\noindent
変数の定義は同様の方法で扱われます。\footnote{この\code{define}の実装は内部定義の
扱いの微妙な問題を無視します。しかし多くの場合では正しく動きます。問題が何か、
どのようにして解決するかについては\link{Section 4.1.6}で学びます。}

\begin{scheme}
(define (eval-definition exp env)
  (define-variable! (definition-variable exp)
                    (eval (definition-value exp) env)
                    env)
  'ok)
\end{scheme}

\noindent
ここで代入、または定義の値としてシンボル\code{ok}を返すことを選択しました。\footnote{\code{define}と
\code{set!}を導入した時に述べたように、これらの値はSchemeの実装依存です---つまり、
実装者がどんな値を返すのか選択できます。}

\begin{quote}
\heading{\phantomsection\label{Exercise 4.1}Exercise 4.1:} 
メタ循環評価機がオペランドを左から右へ評価するのか、右から左へなのか
判断が付かないことに注意せよ。評価順は下位に横たわるLispから継承する。もし
\code{list\-/of\-/values}内の\code{cons}の引数が左から右へ評価されるのなら、
\code{list\-/of\-/values}はオペランドを左から右へと評価する。もし\code{cons}の引数が
右から左へ評価されるなら、\code{list\-/of\-/values}は右から左へ評価する。



オペランドを左から右へと下位に横たわるLispの評価順に係らず評価する\code{list\-/of\-/values}の
版を書け。またオペランドを右から左へ評価する\code{list\-/of\-/values}の版も書け。
\end{quote}

\subsection{式の表現}
\label{Sec. 4.1.2}
\label{Section 4.1.2}



評価機は\link{Section 2.3.2}で議論された記号微分プログラムを思い出させます。
双方のプログラムが記号式を操作します。両方のプログラムにおいて、複合式上の操作の
結果は式の断片を再帰的に操作し、式の型に依存した方法で結合することにより決定します。
両方のプログラムにおいて、私達はデータ抽象化を用いて式がどのように表現されるかの
詳細から命令の一般的なルールを分離します。微分プログラムではこのことが、同じ微分手続が
接頭辞形式、接中辞形式、またはいくつかの他の形式の代数式を扱えることを意味しました。
評価機にとっては、これは評価される言語の文法がもっぱら式を分類し、断片を抽出する
手続により決定されることを意味します。


以下に私達の言語の構文の仕様を示します。

\begin{itemize}

\item
自己評価アイテムは数値と文字列のみです。

\begin{scheme}
(define (self-evaluating? exp)
  (cond ((number? exp) true)
        ((string? exp) true)
        (else false)))
\end{scheme}

\item
変数はシンボルにより表現されます。

\begin{scheme}
(define (variable? exp) (symbol? exp))
\end{scheme}

\item
引用は\code{(quote <\var{text\-/of\-/quotation}>)}の形式を持ちます。\footnote{\link{Section 2.3.1}で
述べたとおり、評価機は引用(quote)された式を\code{quote}で始まるリストだと見ます。
例え式がクォーテーションマークで入力されていてもです。例えば式\code{'a}は
この評価機では\code{(quote a)}と見られます。\link{Exercise 2.55}を参照して下さい。}

\begin{scheme}
(define (quoted? exp) (tagged-list? exp 'quote))
(define (text-of-quotation exp) (cadr exp))
\end{scheme}


\code{quoted?}は手続\code{tagged\-/list?}を用いて定義されます。これはリストが指定された
シンボルで開始するかを判断します。

\begin{scheme}
(define (tagged-list? exp tag)
  (if (pair? exp)
      (eq? (car exp) tag)
      false))
\end{scheme}

\item
代入は\code{(set! <\var{var}> <\var{value}>)}の形式を取ります。

\begin{scheme}
(define (assignment? exp) (tagged-list? exp 'set!))
(define (assignment-variable exp) (cadr exp))
(define (assignment-value exp) (caddr exp))
\end{scheme}

\item
定義は以下の形式を取ります。

\begin{scheme}
(define ~\( \dark \langle \)~~\var{\dark var}~~\( \dark \rangle \)~ ~\( \dark \langle \)~~\var{\dark value}~~\( \dark \rangle \)~)
\end{scheme}

\noindent
または以下の形式になります。

\begin{scheme}
(define (~\( \dark \langle \)~~\var{\dark var}~~\( \dark \rangle \)~ ~\( \dark \langle \)~~\( \dark parameter_1 \)~~\( \dark \rangle \)~ ~\( \dots \)~ ~\( \dark \langle \)~~\( \dark parameter_n \)~~\( \dark \rangle \)~)
  ~\( \dark \langle \)~~\var{\dark body}~~\( \dark \rangle \)~)
\end{scheme}


後者の形式(標準手続定義)は以下に対する糖衣構文です。

\begin{scheme}
(define ~\( \dark \langle \)~~\var{\dark var}~~\( \dark \rangle \)~
  (lambda (~\( \dark \langle \)~~\( \dark parameter_1 \)~~\( \dark \rangle \)~ ~\( \dots \)~ ~\( \dark \langle \)~~\( \dark parameter_n \)~~\( \dark \rangle \)~)
    ~\( \dark \langle \)~~\var{\dark body}~~\( \dark \rangle \)~))
\end{scheme}



対応する構文手続は以下となります。

\begin{scheme}
(define (definition? exp) (tagged-list? exp 'define))
(define (definition-variable exp)
  (if (symbol? (cadr exp))
      (cadr exp)
      (caadr exp)))
(define (definition-value exp)
  (if (symbol? (cadr exp))
      (caddr exp)
      (make-lambda (cdadr exp)     ~\textrm{; formal parameters}~
                   (cddr exp))))   ~\textrm{; body}~
\end{scheme}

\item
\code{lambda}式はシンボル\code{lambda}で始まるリストです。

\begin{scheme}
(define (lambda? exp) (tagged-list? exp 'lambda))
(define (lambda-parameters exp) (cadr exp))
(define (lambda-body exp) (cddr exp))
\end{scheme}


また\code{lambda}式に対するコンストラクタも提供します。これは
上記の\code{definition\-/value}で使用されます。

\begin{scheme}
(define (make-lambda parameters body)
  (cons 'lambda (cons parameters body)))
\end{scheme}

\item
条件式は\code{if}で始まり述語、結果式、持ち、(任意で)代替式を持ちます。もし式が代替式
の部分を持たないのであれば代替式として\code{false}を与えます。\footnote{述語がfalseになり
代替式が存在しない場合の\code{if}式の値はSchemeでは未定義です。ここでは私達はfalseにする
ことを選択しました。私達は変数\code{true}と\code{false}の式内での利用をサポートし、グローバル
環境でのそれらの束縛により評価されるようにします。\link{Section 4.1.4}参照。}

\begin{scheme}
(define (if? exp) (tagged-list? exp 'if))
(define (if-predicate exp) (cadr exp))
(define (if-consequent exp) (caddr exp))
(define (if-alternative exp)
  (if (not (null? (cdddr exp)))
      (cadddr exp)
      'false))
\end{scheme}



また\code{if}式に対するコンストラクタも提供します。これは\code{cond\-/>if}により
\code{cond}式を\code{if}式に変換するのに用いられます。

\begin{scheme}
(define (make-if predicate consequent alternative)
  (list 'if predicate consequent alternative))
\end{scheme}

\item
\code{begin}は一連の式を1つの式へとまとめます。\code{begin}式から実際の列を取り出す命令と同時に、
列の最初の式とその残りの式を返すセレクタも含まれます。\footnote{式のリストに対する
これらのセレクタ---それに対応するオペランドのリスト向けのものも含めて---はデータ抽象化を
意図するものではありません。それらは基本的なリスト命令のためのmnemonic(ニーモニック)名として
\link{Section 5.4}にて明示的コントロール評価機を理解することを易しくするために導入されます。}

\begin{scheme}
(define (begin? exp) (tagged-list? exp 'begin))
(define (begin-actions exp) (cdr exp))
(define (last-exp? seq) (null? (cdr seq)))
(define (first-exp seq) (car seq))
(define (rest-exps seq) (cdr seq))
\end{scheme}


また\code{cond\-/>if}で用いるコンストラクタ\code{sequence\-/>exp}も含めます。これは
列を単一の式に、必要ならば\code{begin}を用いて、変換します。

\begin{scheme}
(define (sequence->exp seq)
  (cond ((null? seq) seq)
        ((last-exp? seq) (first-exp seq))
        (else (make-begin seq))))
(define (make-begin seq) (cons 'begin seq))
\end{scheme}

\item
手続の適用は上記の式の型ではない任意の複合式です。その式の\code{car}はオペレータであり、
\code{cdr}はオペランドのリストです。

\begin{scheme}
(define (application? exp) (pair? exp))
(define (operator exp) (car exp))
(define (operands exp) (cdr exp))
(define (no-operands? ops) (null? ops))
(define (first-operand ops) (car ops))
(define (rest-operands ops) (cdr ops))
\end{scheme}

\end{itemize}

\subsubsection*{派生式}


いくつかの私達の言語内での特殊形式は直接実装されるのでなく、他の特殊形式を含む式を用いて定義できます。
例の1つは\code{cond}です。これは入れ子の\code{if}式として実装できます。例えば以下の式の評価上の
問題を、

\begin{scheme}
(cond ((> x 0) x)
      ((= x 0) (display 'zero) 0)
      (else (- x)))
\end{scheme}

\noindent
次の\code{if}と\code{begin}の式を含む式の評価問題へと簡約することができます。

\begin{scheme}
(if (> x 0)
    x
    (if (= x 0)
        (begin (display 'zero) 0)
        (- x)))
\end{scheme}

\noindent
\code{cond}の評価をこのように実装することは評価機を簡略化します。評価過程が明示的に
指定されねばならない特殊形式の数を減らすことができるからです。



\code{cond}式の部分を抽出する構文手続と\code{cond}式を\code{if}式に変形する式\code{cond->if}を
含めます。事例分析は\code{cond}で始まり述語-行動節のリストを持ちます。節はもし
その述語がシンボル\code{else}ならば\code{else}節です。\footnote{全ての述語がfalseで
\code{else}節が存在しない場合の\code{cond}式の値はSchemeでは未定義です。ここでは
それをfalseにしました。}


\begin{scheme}
(define (cond? exp) (tagged-list? exp 'cond))
(define (cond-clauses exp) (cdr exp))
(define (cond-else-clause? clause)
  (eq? (cond-predicate clause) 'else))
(define (cond-predicate clause) (car clause))
(define (cond-actions clause) (cdr clause))
(define (cond->if exp) (expand-clauses (cond-clauses exp)))
(define (expand-clauses clauses)
  (if (null? clauses)
      'false                        ~\textrm{; no \code{else} clause}~
      (let ((first (car clauses))
            (rest (cdr clauses)))
        (if (cond-else-clause? first)
            (if (null? rest)
                (sequence->exp (cond-actions first))
                (error "ELSE clause isn't last: COND->IF"
                       clauses))
            (make-if (cond-predicate first)
                     (sequence->exp (cond-actions first))
                     (expand-clauses rest))))))
\end{scheme}

\noindent
文法上の変形を実装することを選択した\code{cond}のような式は
\newterm{derived expressions}(\jnewterm{派生式})と呼ばれます。\code{let}式もまた派生式です。
(\link{Exercise 4.6}参照)\footnote{実用的なLispシステムはユーザに対し新しい派生式を
追加し、評価機の変更無しに文法上の変形としての実装を指定できる仕組みを提供します。
そのようなユーザ定義変形は\newterm{macro}(\jnewterm{マクロ})と呼ばれます。マクロ定義の初歩的な仕組みを
追加することは簡単なのですが、結果的にその言語は微妙な名前衝突の問題を持ちます。
これらの困難をもたらさないマクロ定義の仕組みに関する多くの研究が存在します。
例えば\link{Kohlbecker 1986}, \link{Clinger and Rees 1991}, \link{Hanson 1991}を
参照して下さい。}

\begin{quote}
\heading{\phantomsection\label{Exercise 4.2}Exercise 4.2:} 
Louis Reasonerは\code{eval}の\code{cond}の順を変えて手続適用の節が代入のための節の前に
なるようにする計画を立てた。彼はこうすることでインタプリタをより効率良くできると
主張した。プログラムは通常代入、定義等より適用を含んでいるためだ、と。
彼の変更した\code{eval}は元の\code{eval}よりも通常より少ない節を式の型が判明する前に
チェックするだろうという主張だ

\begin{enumerate}[a]

\item
Louisの計画の何が間違っているか? (ヒント:Louisの評価機は式\code{(define x 3)}に
対し何を行うか?)

\item
Louisは彼の計画がうまく行かないことに激昂した。彼は他の多くの型の式をチェックする
前に彼の評価機をいくらでも長くして手続適用を認識させようとしている。評価される
言語を変更し手続適用が\code{call}で始まるようにすることで彼の手助けをせよ。
例えば\code{(factorial 3)}の代わりに変更後は\code{(call factorial 3)}と書かねばならず、
\code{(+ 1 2)}は\code{(call + 1 2)}と書かねばならない。

\end{enumerate}
\end{quote}

\begin{quote}
\heading{\phantomsection\label{Exercise 4.3}Exercise 4.3:} 
\code{eval}を書き直し呼出がデータ適従スタイルにて行われるようにせよ。これを
\link{Exercise 2.73}のデータ適従型微分手続と比較せよ。この節で実装された文法
に適切であるとおり、(複合式の\code{car}を式の型として用いてよい。
\end{quote}

\begin{quote}
\heading{\phantomsection\label{Exercise 4.4}Exercise 4.4:} 
\link{Chapter 1}の特殊形式\code{and}と\code{or}の定義を思い出せ。

\begin{itemize}

\item
\code{and}: 式は左から右へと評価される。もし任意の式がfalseと評価されるならfalseが返される。
残りの式全ては評価されない。もし全ての式がtrueの値に評価されるなら最後の式の評価値が
返される。もし式が全く存在しないならtrueが返される。

\item
\code{or}: 式は左から右へと評価される。もし任意の式がtrueと評価されるのならその値が
返される。残りの式全ては評価されない。もし全ての式がfalseと評価されるのなら、または
もし式が全く存在しないなら、falseが返される。

\end{itemize}
\code{and}と\code{or}を評価機に対する新しい特殊形式として適切な構文手続と
評価手続\code{eval\-/and}と\code{eval\-/or}を定義することで導入せよ。
代替法として、\code{and}と\code{or}を派生式として実装する方法を示せ。
\end{quote}

\begin{quote}
\heading{\phantomsection\label{Exercise 4.5}Exercise 4.5:} 
Schemeは\code{cond}の節に追加の文法、\code{(<\var{test}> => <\var{recipient}>)}を認めている。
もし\( \langle \)\var{test}\( \kern0.08em\rangle \)がtrueとして評価されるなら、
\( \langle \)\var{recipient}\( \kern0.08em\rangle \)が評価される。その値は1引数の手続でなければならない。
そしてこの手続が\( \langle \)\var{test}\( \kern0.08em\rangle \)の値で起動され、その結果が\code{cond}式の値と
して返される。例えば、

\begin{scheme}
(cond ((assoc 'b '((a 1) (b 2))) => cadr)
      (else false))
\end{scheme}

\noindent
は2を返す。\code{cond}を変更してこの拡張文法をサポートするようにせよ。
\end{quote}

\begin{quote}
\heading{\phantomsection\label{Exercise 4.6}Exercise 4.6:} 
\code{let}式は派生式である。なぜなら、

\begin{scheme}
(let ((~\( \dark \langle \)~~\( \dark var_1 \)~~\( \dark \rangle \)~ ~\( \dark \langle \)~~\( \dark exp_1 \)~~\( \dark \rangle \)~) ~\( \dots \)~ (~\( \dark \langle \)~~\( \dark var_n \)~~\( \dark \rangle \)~ ~\( \dark \langle \)~~\( \dark exp_n \)~~\( \dark \rangle \)~))
  ~\( \dark \langle \)~~\var{\dark body}~~\( \dark \rangle \)~)
\end{scheme}

\noindent
は以下と等価である。

\begin{scheme}
((lambda (~\( \dark \langle \)~~\( \dark var_1 \)~~\( \dark \rangle \)~ ~\( \dots \)~ ~\( \dark \langle \)~~\( \dark var_n \)~~\( \dark \rangle \)~)
   ~\( \dark \langle \)~~\var{\dark body}~~\( \dark \rangle \)~)
 ~\( \dark \langle \)~~\( \dark exp_1 \)~~\( \dark \rangle \)~
 ~\( \dots \)~
 ~\( \dark \langle \)~~\( \dark exp_n \)~~\( \dark \rangle \)~)
\end{scheme}

文法上の変形\code{let\-/>combination}を実装せよ。これは\code{let}式の評価を上記で示された
型の組み合わせの評価へと簡約する。そして\code{let}式を扱うために\code{eval}に適切な節を
追加する。
\end{quote}

\begin{quote}
\heading{\phantomsection\label{Exercise 4.7}Exercise 4.7:}
\code{let*}は\code{let}に似ているが、\code{let*}の変数の束縛が左から右へと続けて実行され、
全ての先行する束縛が可視となるよう各束縛が環境へ追加されていく。例えば、

\begin{scheme}
(let* ((x 3)  (y (+ x 2))  (z (+ x y 5)))
  (* x z))
\end{scheme}

\noindent
は39を返す。\code{let*}式が入れ子の\code{let}式の集合としてどのように書き直すことができるか
説明せよ。そしてこの変形を実行する手続\code{let*\-/>nested\-/lets}を書け。
もし私達が既に\code{let}を実装していて(\link{Exercise 4.6})、評価機を拡張し\code{let*}を
扱いたいとしたら、以下の処理を行う節を\code{eval}に追加することは十分であろうか?

\begin{scheme}
(eval (let*->nested-lets exp) env)
\end{scheme}

\noindent
または私達は明示的に\code{let*}を非派生式を用いて拡張するべきであろうか?
\end{quote}

\begin{quote}
\heading{\phantomsection\label{Exercise 4.8}Exercise 4.8:} 
``名前付き\code{let}''は\code{let}の変種であり以下の形式を持つ。

\begin{scheme}
(let ~\( \dark \langle \)~~\var{\dark var}~~\( \dark \rangle \)~ ~\( \dark \langle \)~~\var{\dark bindings}~~\( \dark \rangle \)~ ~\( \dark \langle \)~~\var{\dark body}~~\( \dark \rangle \)~)
\end{scheme}


\( \langle \)\var{bindings}\( \kern0.08em\rangle \)と\( \langle \)\var{body}\( \kern0.08em\rangle \)は
通常の\code{let}と同様である。しかし\( \langle \)\var{var}\( \kern0.08em\rangle \)が
\( \langle \)\var{body}\( \kern0.08em\rangle \)内部で束縛される手続であり、
ボディが\( \langle \)\var{body}\( \kern0.08em\rangle \)であり、
かつパラメタが\( \langle \)\var{bindings}\( \kern0.08em\rangle \)の変数である点が異なる。
従って\( \langle \)\var{var}\( \kern0.08em\rangle \)で名付けられた手続を呼び出すことで
繰り返し\( \langle \)\var{body}\( \kern0.08em\rangle \)を実行することができる。例えば、
反復フィボナッチ手続(\link{Section 1.2.2})は名前付き\code{let}を用いて以下のように書き直すことができる。
\begin{scheme}
(define (fib n)
  (let fib-iter ((a 1)
                 (b 0)
                 (count n))
    (if (= count 0)
        b
        (fib-iter (+ a b) a (- count 1)))))
\end{scheme}


\link{Exercise 4.6}の\code{let\-/>combination}を変更して名前付き\code{let}もサポートするようにせよ。
\end{quote}

\begin{quote}
\heading{\phantomsection\label{Exercise 4.9}Exercise 4.9:} 
多くの言語が\code{do}, \code{for}, \code{while}, \code{until}のような多様な反復構造をサポートする。
Schemeでは反復処理が通常の手続呼出を用いて表現できるため特別な反復構造が演算能力に対し
本質的な利益を与えることはない。一方でそのような構造は時折便利でもある。いくつかの
反復構造を設計せよ。それらの使用の例を与えどのように派生式として実装するかについて示せ。
\end{quote}

\begin{quote}
\heading{\phantomsection\label{Exercise 4.10}Exercise 4.10:}
データ抽象化を用いることで、評価されるべき言語の特定の文法から独立した\code{eval}手続を
書くことができる。これを説明するために\code{eval}と\code{apply}を変更すること
なくこの節の手続を変更することでSchemeの新しい文法を設計し、実装せよ。
\end{quote}

\subsection{評価機のデータ構造}
\label{Sec. 4.1.3}
\label{Section 4.1.3}

式の外側の文法を定義するのに加えて、評価機の実装は評価機が内部的に操作するデータ構造も
プログラムの実行の一部として、手続と環境の表現やtrueとfalseの表現を定義しなければなりません。

\subsubsection*{述語のテスト}


条件節に対しては真になるものは全て受け入れます。真とは明示的な\code{false}オブジェクトでは無いものです。

\begin{scheme}
(define (true? x)  (not (eq? x false)))
(define (false? x) (eq? x false))
\end{scheme}

\subsubsection*{手続の表現}


プリミティブを扱うために、以下の手続が利用可能であると仮定します。

\begin{itemize}

\item
\code{(apply\-/primitive\-/procedure <\var{proc}> <\var{args}>)}

\noindent
与えられたプリミティブな手続をリスト\( \langle \)\var{args}\( \kern0.08em\rangle \)中の
引数の値に適用し、適用の結果を返します。

\item
\code{(primitive\-/procedure? <\var{proc}>)}

\noindent
\( \langle \)\var{proc}\( \kern0.08em\rangle \)がプリミティブな手続であるか確認します。

\end{itemize}

\noindent
プリミティブを扱うこれらの仕組みは\link{Section 4.1.4}でさらに説明されます。


複合手続はパラメタ、手続のボディ、環境からコンストラクタ\code{make\-/procedure}を用いて
構築されます。

\begin{scheme}
(define (make-procedure parameters body env)
  (list 'procedure parameters body env))
(define (compound-procedure? p)
  (tagged-list? p 'procedure))
(define (procedure-parameters p) (cadr p))
(define (procedure-body p) (caddr p))
(define (procedure-environment p) (cadddr p))
\end{scheme}

\subsubsection*{環境上の命令}


評価機は環境を操作する命令を必要とします。\link{Section 3.2}で説明された通り、環境は
連続するフレームであり、各フレームは変数をその対応する値に関連付ける束縛のテーブルです。
以下の命令を用いて環境を操作します。

\begin{itemize}

\item
\code{(lookup\-/variable\-/value <\var{var}> <\var{env}>)}


環境\( \langle \)\var{env}\( \kern0.08em\rangle \)内でシンボル\( \langle \)\var{var}\( \kern0.08em\rangle \)に
束縛された値を返します。または変数が束縛されていない場合エラーを発します。

\item
\code{(extend\-/environment <\var{variables}> <\var{values}> <\var{base\-/env}>)}

新しいフレームから成る環境を返します。フレームの中ではリスト\( \langle \)\var{variables}\( \kern0.08em\rangle \)中の
シンボルがリスト中\( \langle \)\var{values}\( \kern0.08em\rangle \)の対応する要素に束縛されます。
取り囲む環境は環境\( \langle \)\var{base\-/env}\( \kern0.08em\rangle \)です。
\item
\code{(define\-/variable! <\var{var}> <\var{value}> <\var{env}>)}


環境\( \langle \)\var{env}\( \kern0.08em\rangle \)の最初のフレームに
変数\( \langle \)\var{var}\( \kern0.08em\rangle \)を
値\( \langle \)\var{value}\( \kern0.08em\rangle \)に関連付ける新しい束縛を追加します。
\item
\code{(set\-/variable\-/value! <\var{var}> <\var{value}> <\var{env}>)}


環境\( \langle \)\var{env}\( \kern0.08em\rangle \)中の変数\( \langle \)\var{var}\( \kern0.08em\rangle \)の束縛を変更し、
その変数が新しく値\( \langle \)\var{value}\( \kern0.08em\rangle \)に束縛されるようにします。
またはもし変数が束縛されていない場合にはエラーを発します。
\end{itemize}

\noindent
これらの命令を実装するためには環境をフレームのリストとして表現します。
環境を内包する環境はリストの\code{cdr}です。空の環境は単純に空リストです。

\begin{scheme}
(define (enclosing-environment env) (cdr env))
(define (first-frame env) (car env))
(define the-empty-environment '())
\end{scheme}

\noindent
環境の各フレームはリストのペアとして表現されます。フレームに束縛される変数のリストと
対応する値のリストです。\footnote{フレームは実際には以下のコードにおいてデータ抽象化されていません。
\code{Set\-/variable\-/value!}と\code{define\-/variable!}は\code{set\-/car!}を用いて直接フレームの値を変更しています。
フレーム手続の目的は環境操作手続を読み易くすることです。}

\begin{scheme}
(define (make-frame variables values)
  (cons variables values))
(define (frame-variables frame) (car frame))
(define (frame-values frame) (cdr frame))
(define (add-binding-to-frame! var val frame)
  (set-car! frame (cons var (car frame)))
  (set-cdr! frame (cons val (cdr frame))))
\end{scheme}

\noindent
環境を変数を値に関連付ける新しいフレームにより拡張するために、変数のリストと値のリストから
成るフレームを作成します。そしてその環境に隣接させます。もし変数の数が値の数に合わない場合には
エラーを発します。

\begin{scheme}
(define (extend-environment vars vals base-env)
  (if (= (length vars) (length vals))
      (cons (make-frame vars vals) base-env)
      (if (< (length vars) (length vals))
          (error "Too many arguments supplied" vars vals)
          (error "Too few arguments supplied" vars vals))))
\end{scheme}

\noindent
環境内の変数を探すためには、最初のフレームの変数のリストを走査します。
希望の変数を見つければ対応する値リスト内の要素を返します。
もし現在のフレーム内にその変数が見つからなければ内包する環境を探します。以下、繰り返しです。
もし空環境まで辿り着いたならば``束縛されていない変数''のエラーを発します。

\begin{scheme}
(define (lookup-variable-value var env)
  (define (env-loop env)
    (define (scan vars vals)
      (cond ((null? vars)
             (env-loop (enclosing-environment env)))
            ((eq? var (car vars)) (car vals))
            (else (scan (cdr vars) (cdr vals)))))
    (if (eq? env the-empty-environment)
        (error "Unbound variable" var)
        (let ((frame (first-frame env)))
          (scan (frame-variables frame)
                (frame-values frame)))))
  (env-loop env))
\end{scheme}

\noindent
変数に新しい値を指定された環境にて設定するには、\code{lookup\-/variable\-/value}と同様に変数を走査し、
対応する見つかった場合には対応する値を変更します。

\begin{scheme}
(define (set-variable-value! var val env)
  (define (env-loop env)
    (define (scan vars vals)
      (cond ((null? vars)
             (env-loop (enclosing-environment env)))
            ((eq? var (car vars)) (set-car! vals val))
            (else (scan (cdr vars) (cdr vals)))))
    (if (eq? env the-empty-environment)
        (error "Unbound variable: SET!" var)
        (let ((frame (first-frame env)))
          (scan (frame-variables frame)
                (frame-values frame)))))
  (env-loop env))
\end{scheme}

\noindent
変数を定義するには、最初にその変数の束縛を最初のフレームにて探します。
束縛が存在すれば変更を行います。(\code{set\-/variable\-/value!}と同様です)。
そのような束縛がなければ最初のフレームに追加します。

\begin{scheme}
(define (define-variable! var val env)
  (let ((frame (first-frame env)))
    (define (scan vars vals)
      (cond ((null? vars)
             (add-binding-to-frame! var val frame))
            ((eq? var (car vars)) (set-car! vals val))
            (else (scan (cdr vars) (cdr vals)))))
    (scan (frame-variables frame) (frame-values frame))))
\end{scheme}

\noindent
ここで記述された手法は環境を表現する多くのもっともな方法の1つでしかありません。
データ抽象化を用いて評価機の他の部分を表現の詳細な選択から分離したので、もし
望めば環境の表現を変更することが可能です。(\link{Exercise 4.11}参照)。
実運用品質のLispシステムでは評価機の環境向け命令のスピードが---特に変数探索の物が---システムの
パフォーマンスに主に影響を与えます。ここで説明された表現は概念上シンプルではありますが、
効率的ではなく通常は実運用システムでは用いられません。\footnote{この
表現の欠点は(\link{Exercise 4.11}の亜種も同様に)評価機が与えられた変数を見つけるために
数多くのフレームを探索しなければならないかもしれない点です。(このような取り組み方は
\newterm{deep binding}(\jnewterm{深い束縛})と参照されます)。この非効率性を防ぐ1つの方法は
\newterm{lexical addressing}(\jnewterm{レキシカルアドレッシング})と呼ばれ\link{Section 5.5.6}にて
議論されます。}

\begin{quote}
\heading{\phantomsection\label{Exercise 4.11}Exercise 4.11:} 
フレームをリストのペアと表現する代わりに、フレームを束縛のリストとして表現可能である。
この場合、各束縛は名前と値のペアだ。環境の命令を書き換えこの代替表現を用いるようにせよ。
\end{quote}

\begin{quote}
\heading{\phantomsection\label{Exercise 4.12}Exercise 4.12:}
手続\code{set\-/variable\-/value!}, \code{define\-/variable!}, \code{lookup\-/variable\-/value}は
環境の構造を縦断するためのより抽象的な手続を用いて表現することができる。
共通なパターンを捕える抽象化を定義し、3つの手続をこられの抽象化を用いて再定義せよ。
\end{quote}

\begin{quote}
\heading{\phantomsection\label{Exercise 4.13}Exercise 4.13:} 
Schemeは\code{define}を用いて新しい束縛を作成することができる。しかし束縛を
取り除く手段は提供しない。評価機に特殊形式\code{make\-/unbound!}を実装せよ。
これは\code{make\-/unbound!}が評価された環境から与えられたシンボルの束縛を
削除する。この問題は完全には指示されていない。例えば環境の最初のフレームの
束縛のみを削除するべきだろうか? 仕様を完成させあなたが行った選択について
理由を述べよ。
\end{quote}

\subsection{評価機をプログラムとして実行する}
\label{Section 4.1.4}


評価機を与えられたことで、Lisp式が評価されるプロセスの(Lispで表現された)記述を
手中にしました。評価機をプログラムとして表現することの利点の1つはプログラムを
実行できることです。これによりLispの中で実行することでLisp自身がどのように式を
評価するのかについての実行モデルを得ることができました。これは評価ルールを検証する
フレームワークの役割を果たします。実際にこの章の後の方で行います。


評価機プログラムは式を究極的にはプリミティブな手続の適用まで簡約します。
従って評価機を実行するのに必要なもの全ては基盤を無すLispシステムを呼び出す
仕組みを作成することでプリミティブ手続の適用をモデル化することです。


各プリミティブな手続の名前の束縛が存在しなければなりません。そのため\code{eval}が
プリミティブの適用の命令を評価する時、\code{apply}に渡すオブジェクトを見つけます。
従って私達は評価しようとする式の中に現れることが可能なプリミティブな手続の名前と
独自のオブジェクトを関連付けするグローバル環境を設定します。グローバル環境は
またシンボル\code{true}と\code{false}のための束縛も含めます。そうすることで
それらが評価される式の中で変数として利用することができます。

\begin{scheme}
(define (setup-environment)
  (let ((initial-env
         (extend-environment (primitive-procedure-names)
                             (primitive-procedure-objects)
                             the-empty-environment)))
    (define-variable! 'true true initial-env)
    (define-variable! 'false false initial-env)
    initial-env))
(define the-global-environment (setup-environment))
\end{scheme}

\noindent
どのようにプリミティブ手続オブジェクトを評価するかは、\code{apply}がそれらを手続
\code{primitive\-/procedure?}と\code{apply\-/primitive\-/procedure}を用いて判別できる限り問題ではありません。
私達はプリミティブな手続をシンボル\code{primitive}で始まり、そのプリミティブを実装する低層のLispの手続を
含むリストとして表現することを選択しました。

\begin{scheme}
(define (primitive-procedure? proc)
  (tagged-list? proc 'primitive))
(define (primitive-implementation proc) (cadr proc))
\end{scheme}

\noindent
\code{setup\-/environment}はプリミティブの名前と実装手続をリストから得ます。\footnote{低層のLispで
定義される任意の手続はメタ循環評価機のプリミティブとして使用できます。評価機にインストールされる
プリミティブの名前は低層のLispにおける実装の名前と同じである必要はありません。
ここで名前が同じなのはメタ循環評価機がSchemeそれ自身を実装するためです。従って例えば
\code{(list 'first car)}や\code{(list 'square (lambda (x) (* x x)))}を
\code{primitive\-/procedures}に入れることもできたでしょう。}

\begin{scheme}
(define primitive-procedures
  (list (list 'car car)
        (list 'cdr cdr)
        (list 'cons cons)
        (list 'null? null?)
        ~\( \dark \langle \)~~\var{\dark more primitives}~~\( \dark \rangle \)~ ))
(define (primitive-procedure-names)
  (map car primitive-procedures))
(define (primitive-procedure-objects)
  (map (lambda (proc) (list 'primitive (cadr proc)))
       primitive-procedures))
\end{scheme}

\noindent
プリミティブ手続を適用するためには単純に実装手続を引数に対して低層のLispシステムを
用いて適用します。\footnote{
\code{apply\-/in\-/underlying\-/scheme}は前の章で使用した
\code{apply}手続です。メタ循環評価機の\code{apply}手続(\link{Section 4.1.1})はこのプリミティブの
動き方をモデルにしています。2つの異なる\code{apply}と呼ばれる物を持つことはメタ循環
評価機を実行するにおいて問題へと導きます。メタ循環評価機の\code{apply}を定義することが
プリミティブの定義を隠してしまうためです。これを回避する1つの方法はメタ循環の
\code{apply}をリネームすることでプリミティブ手続の名前との衝突を避けることです。
私達はその代わりに下層の\code{apply}への参照をメタ循環の\code{apply}を定義する前に
以下のようにすることで保存しました。

\begin{smallscheme}
(define apply-in-underlying-scheme apply)
\end{smallscheme}

\noindent
これで元の版の\code{apply}に異なる名前でアクセスできるようになりました。}

\begin{scheme}
(define (apply-primitive-procedure proc args)
  (apply-in-underlying-scheme
   (primitive-implementation proc) args))
\end{scheme}

\noindent
メタ循環評価機実行時の利便性のために、低層のLispシステムのread-eval-print loop (REPL:レプル)を
モデルにした\newterm{driver loop}(\jnewterm{ドライバループ})を提供します。これは\newterm{prompt}(\jnewterm{プロンプト})を表示し、
入力式を読み込み、この式をグローバル環境の中で評価し、結果を表示します。
私達は各表示された結果の前に\newterm{output prompt}(\jnewterm{出力プロンプト})を置きます。そうすることで
式の値を他の表示されるかもしれない出力から判別するためです。\footnote{プリミティブな手続
\code{read}はユーザからの入力を待ち、次の入力された完全な式を返します。例えばもしユーザが
\code{(+ 23 x)}と入力した場合、\code{read}は3つの要素、シンボル\code{+}、数値23、シンボル\code{x}を
含むリストを返します。もしユーザが\code{'x}と入力したなら\code{read}は2つの要素、
シンボル\code{quote}とシンボル\code{x}を含むリストを返します。}

\begin{scheme}
(define input-prompt  ";;; M-Eval input:")
(define output-prompt ";;; M-Eval value:")
(define (driver-loop)
  (prompt-for-input input-prompt)
  (let ((input (read)))
    (let ((output (eval input the-global-environment)))
      (announce-output output-prompt)
      (user-print output)))
  (driver-loop))
(define (prompt-for-input string)
  (newline) (newline) (display string) (newline))
(define (announce-output string)
  (newline) (display string) (newline))
\end{scheme}

\noindent
私達は特別なプリント手続、\code{user\-/print}を使用します。これは複合手続の環境部分を
表示するのを防ぐためです。これはとても長いリストに成り得ます。(またはさらにループを
含んでいるかもしれません。)

\begin{scheme}
(define (user-print object)
  (if (compound-procedure? object)
      (display (list 'compound-procedure
                     (procedure-parameters object)
                     (procedure-body object)
                     '<procedure-env>))
      (display object)))
\end{scheme}

\noindent
これで評価機を実行するのに必要なことはグローバル環境の初期化と
ドライバーループの開始のみです。以下がサンプルの応答です。

\begin{scheme}
(define the-global-environment (setup-environment))
(driver-loop)
~\textit{;;; M-Eval input:}~
(define (append x y)
  (if (null? x)
      y
      (cons (car x) (append (cdr x) y))))
~\textit{;;; M-Eval value:}~
~\textit{ok}~
~\textit{;;; M-Eval input:}~
(append '(a b c) '(d e f))
~\textit{;;; M-Eval value:}~
~\textit{(a b c d e f)}~
\end{scheme}

% \vspace{1em}
\begin{quote}
\heading{\phantomsection\label{Exercise 4.14}Exercise 4.14:}
Eva Lu AtorとLouis Reasonerはそれぞれ評価機を検証している。Evaは\code{map}の定義を
入力しいくつかそれを用いるテストプログラム実行している。それらはうまく動いた。
Louisは逆に\code{map}のシステム版をメタ循環評価機のプリミティブとして導入した。
彼がそれを確かめた時、全くうまく動かなかった。なぜEvaはうまく行ったのに
Louisの\code{map}は失敗するのか。説明せよ。
\end{quote}

\subsection{プログラムとしてのデータ}


Lisp式を評価するLispプログラムについて考えることにおいて、例えはとても良い
手助けになるでしょう。プログラムの意味についての命令上の視点の1つに、プログラムは
(恐らく無限に大きな)抽象機械の記述であるという物があります。
例えば階乗を計算する親しみのあるプログラムについて考えてみましょう。

\begin{scheme}
(define (factorial n)
  (if (= n 1) 1 (* (factorial (- n 1)) n)))
\end{scheme}

\noindent
私達はこのプログラムを減算、乗算、等価試験の部品と一緒に2つの位置を持つスイッチと
他の階乗機械を含める機械の説明であると見做すことができるでしょう。(階乗機械は
無限です。他の階乗機械をその中に含んでいるためです)。\link{Figure 4.2}は
階乗機械の流れ図でありその部品がどのようにお互いに接続されているかを示しています。


同様な方法で、評価機を入力として機械の説明書を取るとても特殊な機械だと見做すことができます。
この入力を与えられると、評価機はそれ自身を記述された機械を真似るように設定します。
例えばもし評価機に\link{Figure 4.3}で示される\code{factorial}の記述を与えれば、
評価機は階乗の計算ができるようになります。

\begin{figure}[tb]
\phantomsection\label{Figure 4.2}
\centering
\begin{comment}
\heading{Figure 4.2:} The factorial program, viewed as an abstract machine.

\begin{example}
    +-----------------------------------+
    | factorial                   |1    |
    |              |1             V     |
    |              |           +-----+  |
    |              V           | #   |  |
    |           +-----+        |     |  |
6 --------*-----|  =  |------->|   #-+-----> 720
    |     |     +-----+        |  /  |  |
    |     |                    | #   |  |
    |     |                    +-----+  |
    |     |                       ^     |
    |     |                       |     |
    |     |                    +--+--+  |
    |     *------------------->|  *  |  |
    |     |                    +-----+  |
    |     V                       ^     |
    |  +-----+    +-----------+   |     |
    |  |  -  +--->| factorial +---+     |
    |  +-----+    +-----------+         |
    |     ^                             |
    |     |1                            |
    +-----------------------------------+
\end{example}
\end{comment}
\includegraphics[width=84mm]{fig/chap4/Fig4.2.pdf}
\begin{quote}
\heading{Figure 4.2:} 抽象機械として見た階乗プログラム
\end{quote}
\end{figure}

\noindent
この視点からは、私達の評価機は\newterm{universal machine}(\jnewterm{万能機械})であると見えます。
他の機械がLispにて説明される時、それを真似します。\footnote{機械がLispで記述されるという
ことは本質ではありません。もし私達の評価機にC言語の様な他の言語のための評価機として
振る舞うLispプログラムを与えた場合、Lisp評価機はC評価機の真似をします。
それは順に、C言語で記述された任意の機械の真似が可能です。同様にCで書かれたLisp評価機は
任意のLispプログラムを実行できるCのプログラムを生成します。ここでの深い意図は評価機は
任意の他の物を真似できることです。従って``原理上、何が計算できるのか''という概念(必要な
時間とメモリの実現性は無視)は言語や計算機に非依存です。その代わりに根底を成す概念である
\newterm{computability}(\jnewterm{計算可能性})を反映します。これは最初にAlan M. Turing (1912-1954)に
より明確に証明されました。彼の1936年の論文は計算機科学理論の基礎を導きました。この論文で
チューリングは簡素な計算モデル---今日、\newterm{Turing machine}(\jnewterm{チューリングマシン})として
知られる---を公開し、任意の``実効的な処理''はそのような機械のプログラムとして定式化できると
主張しました。(この論拠は\newterm{Church-Turing thesis}(\jnewterm{チャーチ・チューリングのテーゼ}、
または提唱)として知られます)。チューリングは次に万能機械、即ちチューリングマシン向けプログラムの
評価機として振る舞うチューリングマシンを実装しました。彼はこのフレームワークを用いて
チューリングマシンでは計算できない上手く設定された問題が存在することを証明しました。
(\link{Exercise 4.15}参照)。そのため暗に``実効的な処理''として定式化
できない問題の存在も示したのです。チューリングは実用的な計算機科学への基礎的な貢献の
行いも続けました。例えば彼は汎用目的サブルーチンを用いて構造化プログラミングの考えを
発明しました。チューリングの経歴については\link{Hodges 1983}を参照して下さい。}これは
特筆すべきことです。電子回路向けの同等な評価機について想像することを試してみて下さい。
入力としてフィルタのようなある他の回路の計画を符号化した信号を取る回路になるでしょう。
この入力を与えられて、回路評価機はそこで記述と同じフィルタのように振る舞うでしょう。
そのような万能電子回路はほとんど想像不可能なほど複雑です。プログラム評価機がとても
簡単なプログラムであることは特筆に値します。\footnote{ある人々は比較的単純な手続により
実装された評価機が評価機それ自身より複雑なプログラムの真似ができることが直感的でないと
感じました。万能評価機械の存在は深く、そして素晴しい演算処理の特性です。\newterm{Recursion
theory}(\jnewterm{再帰理論})は数理論理学の1部門であり、演算処理の論理上の制約に関係します。
Douglas Hofstadter(ダグラス ホフスタッター)の美しい本
\textit{G\"odel, Escher, Bach}(邦題:ゲーデル エッシャー バッハ)は
これらの考えのいくつかについて探求します。(\link{Hofstadter 1979})}

\begin{figure}[tb]
\phantomsection\label{Figure 4.3}
\centering
\begin{comment}
\heading{Figure 4.3:} The evaluator emulating a factorial machine.

\begin{example}
                   +--------+
            6 ---->|  eval  |----> 720
                   +--------+
                       /
             . . .    /  . . .
       . . .       ../. .      .
     .                           ..
    .   (define (factorial n)      . . .
   .      (if (= n 1)                   . .
    .         1                            .
    .         (* (factorial (- n 1)) n)))   .
      . .                       . .        .
          . .  . .      . . . .     . . . .
                   . ..
\end{example}
\end{comment}
\includegraphics[width=69mm]{fig/chap4/Fig4.3.pdf}
\par\bigskip
\noindent
\heading{Figure 4.3:} 階乗マシンを真似する評価機
\end{figure}

\noindent
もう1つの特筆すべき評価機の側面はそれがプログラミング言語により操作されるデータオブジェクトと
プログラミング言語それ自身との間のブリッジ(橋)として働くことです。(Lispで実装された)
評価機プログラムが実行中であり、ユーザが式をその評価機に入力し結果を観察していると
想像してみて下さい。ユーザの視点からは\code{(* x x)}の様な入力式はプログラミング言語に
よる式であり、評価機が実行すべき物です。しかし、評価機の視点からは式は単純なリスト
(この場合ではシンボル\code{*}, \code{x}, \code{x}のリスト)であり、これは明確なルール集合に
従って操作されねばならぬ物です。

ユーザのプログラムが評価機のデータだということは混乱の元となる必要はありません。
実際に、時々はこの区別は無視したほうが便利です。そしてユーザに対し明示的にデータ
オブジェクトをLispの式として評価する能力を\code{eval}手続をプログラム中で使用できる
ようにすることで与えることもまた便利なことです。多くのLisp方言は引数として
式と環境を取り、その環境に関連してその式を評価するプリミティブな\code{eval}手続を
与えます。\footnote{警告:この\code{eval}プリミティブは私達が\link{Section 4.1.1}で
実装した\code{eval}手続とは異なります。それは私達が\link{Section 4.1.3}で構築した
サンプルの環境構造ではなく、実際のScheme環境を用いるためです。これらの実際の環境は
ユーザにより通常のリストとして操作することはできません。それらは\code{eval}により
アクセスされるか、他の特別な命令を用います。同様に以前に見た\code{apply}プリミティブ
もメタ循環\code{apply}とは異なります。それが私達が\link{Section 4.1.3}と\link{Section 4.1.4}で
構築した手続オブジェクトではなく、実際のScheme手続を用いるからです。}従って、

\begin{scheme}
(eval '(* 5 5) user-initial-environment)
\end{scheme}

\noindent
と

\begin{scheme}
(eval (cons '* (list 5 5)) user-initial-environment)
\end{scheme}

\noindent
の両方は25を返します。\footnote{Schemeの\acronym{MIT}実装は\code{eval}と同様に
ユーザの入力式が評価される初期環境に束縛されるシンボル\code{user\-/initial\-/environment}も
含みます。}

\begin{quote}
\heading{\phantomsection\label{Exercise 4.15}Exercise 4.15:}
1引数手続\code{p}とオブジェクト\code{a}を与えられた時、式\code{(p a)}が(エラーメッセージや
無限に停止しない場合とは対照的に)値を返す場合に、\code{p}は\code{a}に対して``halt''(停止)
すると呼ばれる。\code{p}が\code{a}に対し停止するかどうかを任意の手続\code{p}と任意のオブジェクト\code{a}に
対して正確に決定する手続\code{halts?}を書くことは不可能であることを示せ。以下の推測を
用いろ:もしそのような手続\code{halts?}が存在するなら以下のプログラムを実装できるだろう。

\begin{scheme}
(define (run-forever) (run-forever))
(define (try p)
  (if (halts? p p) (run-forever) 'halted))
\end{scheme}


ここで式\code{(try try)}の評価について考え、どんな可能な結末(停止するか、無限に実行するか)も
\code{halts?}の意図した振舞に違反することを示せ。\footnote{\code{halts?}が手続オブジェクトを
与えらえたと規定したが、この推測が例え\code{halts?}が手続のテキストとその環境への
アクセスを得ることが出来るとしても依然として適用できることに注意せよ。
これはチューリングの著名な\newterm{Halting Theorem}(\jnewterm{停止性問題})であり、
\newterm{non\-/computable}(\jnewterm{計算不可能})な問題の最初の明確な例を与える。
言い換えれば、計算手続として実行不可能なうまく設定された課題である。}
\end{quote}

\subsection{内部定義}
\label{Section 4.1.6}


私達の評価の環境モデルとメタ循環評価機は定義を順に実行し、環境のフレームを1度に
1定義づつ拡張します。これはインタラクティブなプログラム開発に対しては特に便利です。
その場合にはプログラマは自由に手続の適用を新しい手続の定義に混ぜる必要があります。
しかし、(\link{Section 1.1.8}で紹介された)ブロック構造を実装するために用いられた内部定義に
ついて注意深く考えてみれば、環境の名前毎の拡張はローカル変数の定義に最良の方法
ではないのではと気付くのではないでしょうか。


内部定義を伴なう以下のような手続について考えてみます。

\begin{scheme}
(define (f x)
  (define (even? n) (if (= n 0) true  (odd?  (- n 1))))
  (define (odd? n)  (if (= n 0) false (even? (- n 1))))
  ~\( \dark \langle \)~~\var{\dark rest of body of \code{f}}~~\( \dark \rangle \)~)
\end{scheme}

\noindent
ここでの意図は手続\code{even?}のボディ内の名前\code{odd?}は\code{even?}の後に定義された
手続\code{odd?}を参照しなければなりません。名前\code{odd?}のスコープは\code{f}の
ボディ全体であり、\code{odd?}の定義が起こった箇所から始まる\code{f}のボディの
一部分ではありません。実際に\code{odd?}がそれ自身\code{even?}を用いて定義されて
いることについて考えると---\code{even?}と\code{odd?}は相互再帰手続であり---2つの
\code{define}を満足させる解釈はそれらを名前\code{even?}と\code{odd?}が環境に同時に
追加されたと見做すことのみだとわかります。より一般的には、ブロック構造において、
ローカルな名前のスコープは\code{define}が評価された手続のボディ全体だという
ことです。

偶然にも私達のインタプリタは\code{f}の呼出を正確に評価します。しかし``予想外''の
理由のためです。内部手続の定義が最初に来るため、これらの手続への呼出はそれらの
全てが定義されるまで起こりません。従って\code{odd?}は\code{even?}が実行された時に
定義されるのです。
内部定義がボディの最初に来て定義された変数の値の式の評価が実際にはどの定義された
変数も用いない任意の手続に対し直接、同時定義を実装する仕組みと、私達の逐次的な
評価システムの仕組みは実際に同じ結果を与えます。
(これらの制限に従わず、その結果逐次定義が同時定義と等価でない手続の例に対しては
\link{Exercise 4.19}を参照して下さい)。\footnote{プログラムにこの評価の仕組みに依存
して欲しくないというのが\link{Chapter 1}の\link{Footnote 28}での見解、``管理は責任を
取れない''に対する理由です。これを主張することで内部定義は最初に来て、定義中で
お互いを定義が評価されている間に使用はしません。Schemeの\acronym{IEEE}標準は
実装者にこれらの定義の評価に用いられる仕組みについて幾つかの選択を残します。
別のルールではなくある評価ルールを選択することはここでは``悪い形式''のプログラムの
解釈のみに影響する小さな問題に見えるかもしれません。しかし\link{Section 5.5.6}では同時に
内部定義を行うモデルへの移行が、そうしなければコンパイラの実装にて起こり得る意地の悪い
問題を防ぐことを学びます。}


しかし内部定義の名前が真に同時にスコープを持つようになる簡単な定義の扱い方が存在します。
単に現在の環境に入ることになる全てのローカル変数をどの値の式が評価されるよりも早く
作成することです。これを行う1つの方法は\code{lambda}式上の構文変形によります。
\code{lambda}式のボディを評価する前に、ボディの中の全ての内部定義を走査し、削除します。
内部で定義された変数は\code{let}を用いて作成され、次に代入を用いてそれらの値に設定されます。
例えば、以下の手続は、

\begin{scheme}
(lambda ~\( \dark \langle \)~~\var{\dark vars}~~\( \dark \rangle \)~
  (define u ~\( \dark \langle \)~~\var{\dark e1}~~\( \dark \rangle \)~)
  (define v ~\( \dark \langle \)~~\var{\dark e2}~~\( \dark \rangle \)~)
  ~\( \dark \langle \)~~\var{\dark e3}~~\( \dark \rangle \)~)
\end{scheme}

\noindent
以下の形式に変形されます。

\begin{scheme}
(lambda ~\( \dark \langle \)~~\var{\dark vars}~~\( \dark \rangle \)~
  (let ((u '*unassigned*)
        (v '*unassigned*))
    (set! u ~\( \dark \langle \)~~\var{\dark e1}~~\( \dark \rangle \)~)
    (set! v ~\( \dark \langle \)~~\var{\dark e2}~~\( \dark \rangle \)~)
    ~\( \dark \langle \)~~\var{\dark e3}~~\( \dark \rangle \)~))
\end{scheme}

\noindent
ここで\code{*unassigned*}は特別なシンボルであり、変数が調べられた時にもしまだ値が
割り当てられていない変数を使用しようとしたならばエラーを発せさせます。

内部定義を全て走査する方法の代替となる戦略は\link{Exercise 4.18}にて示されます。
上で示された変形とは異なり、これは定義された変数の値がその変数のどんな値も
用いずに評価できるという制約を強制します。\footnote{Schemeの\acronym{IEEE}標準は
この制約を強制する実装にまかせるのではなく、プログラマに対してこの制約に従うかを
まかせると指定することで、異なる実装戦略を許しています。\acronym{MIT} Schemeを含む
いくつかのScheme実装は上で示された変形を用いています。従ってこの制約に従わない
プログラムは実際にはそのような実装の下では動作します。}

\begin{quote}
\heading{\phantomsection\label{Exercise 4.16}Exercise 4.16:}
この課題では内部定義を逐次実行するためについ先程説明された手法を実装する。
評価機は\code{let}をサポートすると仮定する。(\link{Exercise 4.6}参照)

\begin{enumerate}[a]

\item
\code{lookup\-/variable\-/value} (\link{Section 4.1.3})を変更してもし見つけた値が
シンボル\code{*unassigned*}ならエラーを発するようにする。

\item
手続のボディを取り内部手続を持たない同等な手続を返す手続\code{scan\-/out\-/defines}を
上で説明された変形を作成することにより、書け。

\item
\code{scan\-/out\-/defines}をインタプリタの\code{make\-/procedure}または
\code{procedure\-/body} (see \link{Section 4.1.3})の中に導入せよ。
どちらの場所が良いか? それは何故か?

\end{enumerate}
\end{quote}

\begin{quote}
\heading{\phantomsection\label{Exercise 4.17}Exercise 4.17:}
この本の手続の式\( \langle \)\var{e3}\( \kern0.1em\rangle \)の評価を実施している時の
環境図を書くことで、定義が逐次的に翻訳された時にどのように構築されるかと、定義が
説明されたように走査された場合にどのように構築されるかとの違いを比較せよ。
変形されたプログラムにはなぜ余分なフレームが存在するのか? 環境構造内のこの違いが
正しいプログラムの振舞に違いを起こさないのか説明せよ。インタプリタに内部定義の
``同時''スコープのルールを余分なフレームの構築成しに実装させる方法を設計せよ。
\end{quote}

\begin{quote}
\heading{\phantomsection\label{Exercise 4.18}Exercise 4.18:}
テキストの例を以下の様に変形する定義の走査に対する代替となる戦略を考えよ。

\begin{scheme}
(lambda ~\( \dark \langle \)~~\var{\dark vars}~~\( \dark \rangle \)~
  (let ((u '*unassigned*)
        (v '*unassigned*))
    (let ((a ~\( \dark \langle \)~~\var{\dark e1}~~\( \dark \rangle \)~) (b ~\( \dark \langle \)~~\var{\dark e2}~~\( \dark \rangle \)~))
      (set! u a)
      (set! v b))
    ~\( \dark \langle \)~~\var{\dark e3}~~\( \dark \rangle \)~))
\end{scheme}

ここで\code{a}と\code{b}は新しい変数の名前を表現することを意味し、インタプリタにより作成され、ユーザの
プログラムには現れない。\link{Section 3.5.4}の\code{solve}手続について考える。

\begin{scheme}
(define (solve f y0 dt)
  (define  y (integral (delay dy) y0 dt))
  (define dy (stream-map f y))
  y)
\end{scheme}

この手続はこの課題に示されたように内部定義が走査された場合にうまく動くだろうか?
テキストに示されたように走査された場合には動くだろうか? 説明せよ。
\end{quote}

\begin{quote}
\heading{\phantomsection\label{Exercise 4.19}Exercise 4.19:}
Ben Bitdiddle, Alyssa P. Hacker, Eva Lu Atorの3人は以下の式を評価した場合の
望まれた結果について議論している。

\begin{scheme}
(let ((a 1))
  (define (f x)
    (define b (+ a x))
    (define a 5)
    (+ a b))
  (f 10))
\end{scheme}

Benは結果は\code{define}に対する逐次的実行のルールを用いて得られるべきだと主張した。
\code{b}は11に定義され、\code{a}は5に定義される。従って結果は16である。Alyssaは相互再帰は
同時スコープのルールが内部手続定義に要求されるとして異議を唱えた。手続の名前を
他の名前から異なって扱うのは不合理だ。従って彼女は\link{Exercise 4.16}で実装された
仕組みに賛成した。これは\code{a}が\code{b}の値が計算される時点では割り当てられていない
という結論に導くだろう。従ってAlyssaの視点では手続はエラーを生じなければならない。
Evaは3つ目の立ち位置を取る。彼女はもし\code{a}と\code{b}の定義が真に同時であることを意味
するのであれば、\code{a}に対する値5は\code{b}の評価にて用いられるべきであると述べた。
従ってEvaの視点では\code{a}は5でなければならず、\code{b}は15でなければならない。そして
結果は20にならなければならない。(もし同意するなら)3人の視点であなたはどれを支持するのか?
あなたはEvaが好んだように振る舞う内部定義を実装する方法を考案できるか?\footnote{\acronym{MIT} Schemeの
実装者達は次の根拠に従ってAlyssaを支持する。Evaは原理上は正しい。定義は同時だと見做される
べきだ。しかしEvaが要求することを行う一般的でかつ効率的な仕組みを実装することは難しく
見える。そのような仕組みが不足している状況では、同時定義の難しい場合についてはエラーを
生成するほうが(Alyssaの意見)、正しくない答を生成するよりも(Benの様に)、より良いだろう。}
\end{quote}

\begin{quote}
\heading{\phantomsection\label{Exercise 4.20}Exercise 4.20:}
内部定義は逐次的に見えるが実際には同時であるため、いくらかの人々はこれを完全に
回避するほうを好むだろう。そして特殊形式\code{letrec}を代わりに用いる。
\code{letrec}は\code{let}に似ているため、それが束縛する変数が同時に束縛されお互いに
同じスコープを持つことは不思議ではないだろう。上記のサンプル手続\code{f}は内部手続を
用いずに、しかし全く同じ意味を持つように書くことができる。

\begin{scheme}
(define (f x)
  (letrec
    ((even? (lambda (n)
              (if (= n 0) true  (odd?  (- n 1)))))
     (odd?  (lambda (n)
              (if (= n 0) false (even? (- n 1))))))
    ~\( \dark \langle \)~~\var{\dark rest of body of \code{f}}~~\( \dark \rangle \)~))
\end{scheme}

\code{letrec}式は以下の形式を持つ。

\begin{scheme}
(letrec ((~\( \dark \langle \)~~\( \dark var_1 \)~~\( \dark \rangle \)~ ~\( \dark \langle \)~~\( \dark exp_1 \)~~\( \dark \rangle \)~) ~\( \dots \)~ (~\( \dark \langle \)~~\( \dark var_n \)~~\( \dark \rangle \)~ ~\( \dark \langle \)~~\( \dark exp_n \)~~\( \dark \rangle \)~))
  ~\( \dark \langle \)~~\var{\dark body}~~\( \dark \rangle \)~)
\end{scheme}

\noindent
\code{letrec}式は\code{let}の亜種であり、変数\( \langle \)\( var_k \)\( \kern0.08em\rangle \)に
初期値を与える式\( \langle \)\( exp_k \)\( \kern0.08em\rangle \)は、全ての\code{letrec}の束縛を
含む環境にて評価される。これは上の例においての\code{even?}と\code{odd?}の相互再帰のような束縛の中での再帰を許す。
または以下の様な10の階乗の評価も可能である。

\begin{scheme}
(letrec
  ((fact (lambda (n)
           (if (= n 1) 1 (* n (fact (- n 1)))))))
  (fact 10))
\end{scheme}

\begin{enumerate}[a]

\item
\code{letrec}を派生式として実装せよ。\code{letrec}式を\code{let}を上で示したように、
または\link{Exercise 4.18}の様に変形することで行え。
即ち、\code{letrec}の変数は\code{let}を用いて作成しなければならず、そして次に
それらの値を\code{set!}で代入すること。

\item
Louis Reasonerは内部定義に関するこの全ての空騒ぎにより混乱してしまった。
彼の見解は、もし手続の中での\code{define}の使用を好まないのであれば、単に
\code{let}を使えるのではないかである。彼のreasoning(推測)の何が緩いのかを、この課題と同様に
定義された\code{f}を用いて、式\code{(f 5)}の評価の間に
\( \langle \)\var{rest of body of \code{f}}\( \kern0.08em\rangle \)が評価された
環境を示す環境図を書くことによって説明せよ。
同じ環境の、ただし\code{f}の定義中の\code{letrec}の場所に\code{let}を用いた場合の環境図を書け。

\end{enumerate}
\end{quote}

\begin{quote}
\heading{\phantomsection\label{Exercise 4.21}Exercise 4.21:}
驚くべきことに、\link{Exercise 4.20}におけるLouisの直感は正しい。\code{letrec}を(または
\code{define}すらも)用いずに再帰手続を指定することは本当に可能である。しかしこれを
達成する手法はLouisが存在したよりもずっと繊細である。以下の式は10の階乗を再帰
階乗手続を適用することで求めている。\footnote{この例は再帰手続を\code{define}を用いずに
定式化するためのプログラミング上の技を説明しています。最も一般的なこの種の技は
\( Y \) \newterm{operator}(\jnewterm{Yコンビネータ}、不動点演算子)です。これは
``pure \( \lambda \)-calculus''(純粋ラムダ計算)による再帰の実装を与えます。(\( \lambda \)計算の
詳細については\link{Stoy 1977}を参照して下さい。またSchemeによる\( Y \)コンビネータの
解説については\link{Gabriel 1988}を参照して下さい。}

\begin{scheme}
((lambda (n)
   ((lambda (fact) (fact fact n))
    (lambda (ft k)
      (if (= k 1) 1 (* k (ft ft (- k 1)))))))
 10)
\end{scheme}

\begin{enumerate}[a]

\item
(式を評価することで)これが実際に階乗を計算することを確認せよ。
フィボナッチ数を計算する同様な式を工夫せよ。

\item
以下の手続について考える。これは相互再帰内部定義を含む。

\begin{scheme}
(define (f x)
  (define (even? n)
    (if (= n 0) true  (odd?  (- n 1))))
  (define (odd? n)
    (if (= n 0) false (even? (- n 1))))
  (even? x))
\end{scheme}

\code{f}の代替的な定義を完成するために欠けている式を埋めよ。これは内部定義も
\code{letrec}も使用してはいない。

\begin{scheme}
(define (f x)
  ((lambda (even? odd?) (even? even? odd? x))
   (lambda (ev? od? n)
     (if (= n 0) true (od? ~\( \dark \langle \)~??~\( \dark \rangle \)~ ~\( \dark \langle \)~??~\( \dark \rangle \)~ ~\( \dark \langle \)~??~\( \dark \rangle \)~)))
   (lambda (ev? od? n)
     (if (= n 0) false (ev? ~\( \dark \langle \)~??~\( \dark \rangle \)~ ~\( \dark \langle \)~??~\( \dark \rangle \)~ ~\( \dark \langle \)~??~\( \dark \rangle \)~)))))
\end{scheme}
\end{enumerate}
\end{quote}

\subsection{構文分析を実行から分離する}
\label{Section 4.1.7}

上で実装された評価機は簡単ですが、非効率です。式の構文上の分析がその実行と相互
配置されているためです。従ってもしプログラムが何度も実行された場合、その構文は
何度も分析されます。例えば次の\code{factorial}を用いて\code{(factorial 4)}を評価する
ことを考えてみて下さい。

\begin{scheme}
(define (factorial n)
  (if (= n 1) 1 (* (factorial (- n 1)) n)))
\end{scheme}

\noindent
\code{factorial}が呼ばれる度に、評価機はボディが\code{if}式であることを判断せねばならず、
それから述語を取り出します。その後にのみ述語を評価しその値により振り分けが行えます。
式\code{(* (factorial (- n 1)) n)}、または部分式
\code{(factorial (- n 1))}と\code{(- n 1)}を評価する度に、評価機は\code{eval}にて状況分析を
行い式が適用であるかを判断せねばならず、また演算子とオペランドの抽出をせねばなりません。
この分析はコストが高いのです。これを繰り返し実行することは無駄が多いでしょう。

評価機を変形し構文上の分析をたった1度のみ実行されるように準備することで
著しく効率良くすることができます。\footnote{この技はコンパイル過程に
不可欠な要素であり、\link{Chapter 5}で議論します。Jonathan Reesは1982年頃に
このようなSchemeインタプリタをTプロジェクトのために書きました(\link{Rees and Adams 1982})。
Marc \link{Feeley (1986)} (\link{Feeley and Lapalme 1987}も参照)は彼の修士論文にて
独力でこの技を発明しました。}私達は式と環境を取る\code{eval}を2つに分けます。
手続\code{analyze}は式のみを取ります。構文上の分析を行い新しい手続
\newterm{execution procedure}(\jnewterm{実行手続})を返します。この手続は分析された式を
実行するにおいて行われた結果をカプセル化します。実行手続は環境を引数として
取り評価を完了します。これは実行手続が何度も呼ばれるのに対し、\code{analyze}が式に
対して1度しか呼ばれないため作業量を減らせます。


分析と実行への分離に伴ない、\code{eval}は以下の様になります。

\begin{scheme}
(define (eval exp env) ((analyze exp) env))
\end{scheme}

\noindent
\code{analyze}の呼出の結果は環境に適用される実行手続です。\code{analyze}手続は\link{Section 4.1.1}の
元の\code{eval}により実行されたのと同じ状況分析です。ただし私達が呼び出す手続は完全な
評価ではなく分析のみを実行します。

\begin{scheme}
(define (analyze exp)
  (cond ((self-evaluating? exp)
         (analyze-self-evaluating exp))
        ((quoted? exp) (analyze-quoted exp))
        ((variable? exp) (analyze-variable exp))
        ((assignment? exp) (analyze-assignment exp))
        ((definition? exp) (analyze-definition exp))
        ((if? exp) (analyze-if exp))
        ((lambda? exp) (analyze-lambda exp))
        ((begin? exp)
         (analyze-sequence (begin-actions exp)))
        ((cond? exp) (analyze (cond->if exp)))
        ((application? exp) (analyze-application exp))
        (else
         (error "Unknown expression type: ANALYZE" exp))))
\end{scheme}

\noindent
以下に最も簡単な構文分析手続があります。これは自己評価式です。環境引数を無視し、
ただ式を返す実行手続を返します。

\begin{scheme}
(define (analyze-self-evaluating exp)
  (lambda (env) exp))
\end{scheme}

\noindent
クォートされた式に対してはそのテキストの取り出しを実行フェーズでなく、分析フェーズで
1度だけ行うことでほんの少し効率良くすることができます。

\begin{scheme}
(define (analyze-quoted exp)
  (let ((qval (text-of-quotation exp)))
    (lambda (env) qval)))
\end{scheme}

\noindent
変数の値の探索は依然として実行フェーズで行わねばなりません。これは環境を知ることに
依存するためです。\footnote{しかし、構文上の分析の部分にて終わらせられる変数探索の
重要な部分があります。\link{Section 5.5.6}にて示されるように、環境構造の中でどこで変数の値が
見つかるか、その位置を決定することが可能です。従って変数にマッチするエントリのために
環境を走査する必要を防ぐことができます。}

\begin{scheme}
(define (analyze-variable exp)
  (lambda (env) (lookup-variable-value exp env)))
\end{scheme}

\noindent
\code{analyze\-/assignment}もまた実際の変数の設定を環境の供給が完了する実行時まで
遅らせなければなりません。しかし\code{assignment\-/value}式が分析の間に
(再帰的に)分析されることができることは効率を大きく向上します。\code{assignment\-/value}式は
今はただ1度しか分析されないためです。同じことが定義に対しても言えます。

\begin{scheme}
(define (analyze-assignment exp)
  (let ((var (assignment-variable exp))
        (vproc (analyze (assignment-value exp))))
    (lambda (env)
      (set-variable-value! var (vproc env) env)
      'ok)))
(define (analyze-definition exp)
  (let ((var (definition-variable exp))
        (vproc (analyze (definition-value exp))))
    (lambda (env)
      (define-variable! var (vproc env) env)
      'ok)))
\end{scheme}

\noindent
\code{if}式に対しては分析時に述語、結果、代替を取り出し分析します。

\begin{scheme}
(define (analyze-if exp)
  (let ((pproc (analyze (if-predicate exp)))
        (cproc (analyze (if-consequent exp)))
        (aproc (analyze (if-alternative exp))))
    (lambda (env) (if (true? (pproc env))
                      (cproc env)
                      (aproc env)))))
\end{scheme}

\noindent
\code{lambda}式の分析もまた効率が大きく向上します。\code{lambda}のボディは1度しか
分析しません。例え\code{lambda}の評価の結果としての手続が何度適用されてもです。

\begin{scheme}
(define (analyze-lambda exp)
  (let ((vars (lambda-parameters exp))
        (bproc (analyze-sequence (lambda-body exp))))
    (lambda (env) (make-procedure vars bproc env))))
\end{scheme}

\noindent
(\code{begin}や\code{lambda}式のボディの中としての)式の列の評価の分析は
より必要とされます。\footnote{列の処理に関する実態については\link{Exercise 4.23}を
参照して下さい。}列の各式は分析され実行手続を生じます。これらの実行手続は
環境を引数として取り順番に各個別の実行手続を引数としての環境と共に呼び出す
実行手続を生成するために組み合わせれます。

\begin{scheme}
(define (analyze-sequence exps)
  (define (sequentially proc1 proc2)
    (lambda (env) (proc1 env) (proc2 env)))
  (define (loop first-proc rest-procs)
    (if (null? rest-procs)
        first-proc
        (loop (sequentially first-proc (car rest-procs))
              (cdr rest-procs))))
  (let ((procs (map analyze exps)))
    (if (null? procs) (error "Empty sequence: ANALYZE"))
    (loop (car procs) (cdr procs))))
\end{scheme}

\noindent
適用を分析するためには、演算子とオペランドを分析し、演算子の実行手続を(実際に適用される
手続を得るために)呼び出し、オペランドの実行手続を(実際の引数を得るために)呼び出す
実行手続を構築します。次にこれらを\code{execute\-/application}に渡します。これは
\link{Section 4.1.1}の\code{apply}の類似品です。\code{execute\-/application}は\code{apply}とは
複合手続のための手続のボディが既に分析されている点が異なります。そのためさらなる
分析の必要性がありません。その代わりに、ただ拡張された環境上のボディに対して
手続実行を呼び出します。

\begin{scheme}
(define (analyze-application exp)
  (let ((fproc (analyze (operator exp)))
        (aprocs (map analyze (operands exp))))
    (lambda (env)
      (execute-application
       (fproc env) 
       (map (lambda (aproc) (aproc env))
            aprocs)))))
(define (execute-application proc args)
  (cond ((primitive-procedure? proc)
         (apply-primitive-procedure proc args))
        ((compound-procedure? proc)
         ((procedure-body proc)
          (extend-environment
           (procedure-parameters proc)
           args
           (procedure-environment proc))))
        (else
         (error "Unknown procedure type:
                 EXECUTE-APPLICATION"
                proc))))
\end{scheme}

\noindent
私達の新しい評価機は節\link{Section 4.1.2}, \link{Section 4.1.3}, \link{Section 4.1.4}に
あるように、同じデータ構造、構文手続、実行時サポート手続を用います。

\begin{quote}
\heading{\phantomsection\label{Exercise 4.22}Exercise 4.22:}
この節の評価機を特殊形式\code{let}をサポートするように拡張せよ。(\link{Exercise 4.6}参照)
\end{quote}

\begin{quote}
\heading{\phantomsection\label{Exercise 4.23}Exercise 4.23:}
Alyssa P. Hackerはなぜ\code{analyze\-/sequence}がそんなに複雑になるのか理解できなかった。
他の分析手続全ては\link{Section 4.1.1}の対応する評価手続(または\code{eval}節)の簡単な変形である。
彼女は\code{analyze\-/sequence}は以下のようになるのではと予想した。

\begin{scheme}
(define (analyze-sequence exps)
  (define (execute-sequence procs env)
    (cond ((null? (cdr procs))
           ((car procs) env))
          (else
           ((car procs) env)
           (execute-sequence (cdr procs) env))))
  (let ((procs (map analyze exps)))
    (if (null? procs)
        (error "Empty sequence: ANALYZE"))
    (lambda (env) (execute-sequence procs env))))
\end{scheme}


Eva Lu AtorはAlyssaに対し、テキストの版は分析時に列を評価する仕事よりもより多くの
ことを行っていると説明した。Alyssaの逐次実行手続は個別の組み込みの実行手続に
対する呼出を行うのではなく、複数の手続を通してそれらを呼び出すためにループする。
実際に列内の個別の式は分析されるが、列それ自身は分析されない。


2つの版の\code{analyze\-/sequence}を比較せよ。例として、列がただ1つの式を持つ場合に
おいて(手続のボディ特有の)共通な場合について考えよ。Alyssaのプログラムにより生成された
実行手続はどのような行いをするか? 上のテキスト内のプログラムで生成された実行手続に
ついてはどうか? 2つの版は2つの式を持つ列に対してはどのように比較されるか?
\end{quote}

\begin{quote}
\heading{\phantomsection\label{Exercise 4.24}Exercise 4.24:}
元のメタ循環評価機とこの節の版のスピードを比較するためのいくつかの実験を設計し
実行せよ。あなたの結果を用いて種々の手続に対して分析と実行で消費された時間を
概算せよ。
\end{quote}

\section{Scheme上でのバリエーション --- 遅延評価}
\label{Sec. 4.2}
\label{Section 4.2}

今や私達はLispプログラムとして表現された評価機を得ました。これで言語設計上の代替となる
選択を単純に評価機を変更することで試験することができます。実際に新しい言語は良く、
最初に既存の高級言語の中に新しい言語を埋め込む評価機を書くことで開発されます。
例えばもしわたしたちがLispに対する変更の提案のある側面についてLispコミュニティの他の
メンバと議論したい時に、変更を組み込んだ評価機を与えることができます。受け手はすると
新しい評価機を持ちいて実験を行いさらなる変更としてのコメントを返すことができます。
高レベルな実装ベースが評価機のテストとデバッグをより簡単にするだけではありません。
加えて組込むことは設計者に対し下層の言語から機能をsnarfする\footnote{Snarf: ``つかみ
取ること、特に巨大な文書やファイルを持ち主の許可を得ても得なくても使う目的のため''
Snarf Down: ``snarfすること、稀に吸収する、処理する、または理解するの含意を持つ''
(これらの定義は\link{Steele et al. 1983}からsnarfした。\link{Raymond 1993}も参照すること)}
ことを可能にします。これは
私達の組込Lisp評価機が下層のLispからプリミティブやコントロール構造を使用するのと同じです。
設計者は(もし必要があれば)後で低レベル言語やハードウェアにて完全な実装を構築するだけです。
この節と次ではSchemeの、優位な追加の表現力を提供するいくつかのバリエーションについて探求します。



\subsection{正規順と適用順}
\label{Section 4.2.1}

\link{Section 1.1}では評価のモデルについての議論を始めましたが、Schemeは\newterm{applicative-order}(\jnewterm{適用順序})
言語であると記しました。即ち、Schemeの手続への全ての引数は手続が適用される時に評価される、と。
逆に、\newterm{normal-order}(\jnewterm{正規順序})言語は手続引数の評価を実際に引数の値が必要とされるまで
遅らせます。手続引数の評価を可能な限り最後の瞬間まで(例えばプリミティブ命令により必要とされるまで)
遅らせることは\newterm{lazy evaluation}(\jnewterm{遅延評価})と呼ばれます。\footnote{専門用語
``lazy''と``normal-order''の間の違いはいささか曖昧(fuzzy)です。一般的に``lazy''は特定の
評価機の仕組みを参照しますが、一方で``normal-order''は言語の意味を参照し、どんな特定の評価戦術からも
独立しています。しかしこれは確かな区別ではありません。そして2つの専門用語は良く同義的に
用いられています。} 以下の手続について考えてみましょう。

\begin{scheme}
(define (try a b) (if (= a 0) 1 b))
\end{scheme}

\noindent
\code{(try 0 (/ 1 0))}の評価はSchemeではエラーを生成します。遅延評価ではエラーは現れません。
その式の評価は1になります。なぜなら絶対に引数は評価されないためです。


遅延評価を利用した例で、手続\code{unless}の定義です。

\begin{scheme}
(define (unless condition usual-value exceptional-value)
  (if condition exceptional-value usual-value))
\end{scheme}

\noindent
これは以下のような式で使用できます。

\begin{scheme}
(unless (= b 0)
        (/ a b)
        (begin (display "exception: returning 0") 0))
\end{scheme}

\noindent
これは適用順序の言語では動きません。通常値と例外値の両方が\code{unless}が呼ばれる前に
評価されるためです。(\link{Exercise 1.6}と比較してみて下さい)。遅延評価の利点は
\code{unless}のような手続は例えそれらの引数の幾つかの評価がエラーを発したり、
停止しなかったとしても役立つ計算ができます。

引数の評価が完了する前に手続のボディに入ることを、手続がその引数において
\newterm{non\-/strict}(\jnewterm{非厳密})であると呼びます。もし引数が手続のボディに入る前に
評価されたなら手続はその引数に対し\newterm{strict}(\jnewterm{厳密})であると言います。\footnote{``厳密''対
``非厳密''の技術用語は本質的には``適用順序''対``正規順序''と同じことを言っています。
しかし個別の手続と引数を言及しており、言語全体を言及してはいません。
プログラミング言語のカンファレンスでは誰かがこのようなことを言うのを聞くかもしれません。
``正規順序言語のHassleはいくつか厳密なプリミティブを持っている。他の手続はそれらの引数を
遅延評価で取る。''}純粋適用順序言語では全ての手続が全ての引数に対し厳密です。そして
プリミティブな手続は厳密にも非厳密にも成り得ます。またプログラマに彼等が定義する手続の
厳密さに細かなコントロールを提供する言語もあります。(\link{Exercise 4.31}参照)

実用性のため非厳密にすることができる手続の印象的な例には\code{cons}(または一般的に、
ほとんど全てのデータ構造のコンストラクタが)あります。
例えもし要素の値がわからなくても、データ構造を形成するよう要素を組み立て、
結果のデータ構造上で操作する実用的な計算を行えます。例えばリストの長さを
リスト内の個々の要素の値を知ること無しに計算することは完璧に意味があります。
私達はこの考えを\link{Section 4.2.3}で非厳密な\code{cons}ペアにより形成されたリストとして
\link{Chapter 3}のストリームを実装するために利用します。

\begin{quote}
\heading{\phantomsection\label{Exercise 4.25}Exercise 4.25:}
(通常の適用順Schemeにおいて)\code{unless}を上で示されたように定義し、
次に\code{unless}を用いて以下のように\code{factorial}を定義する。

\begin{scheme}
(define (factorial n)
  (unless (= n 1)
          (* n (factorial (- n 1)))
          1))
\end{scheme}


もし\code{(factorial 5)}を評価したら何が起こるか? この定義は正規順序言語では
動くか?
\end{quote}

\begin{quote}
\heading{\phantomsection\label{Exercise 4.26}Exercise 4.26:}
Ben BitdiddleとAlyssa P. Hackerは\code{unless}のような物を実装するための
遅延評価の重要性について意見が分かれた。Benは\code{unless}を適用順序でも特殊形式として
実装可能である点を指摘した。Alyssaはもしそれを行えば\code{unless}はただ単に構文であり
高階手続と連動して使用できる手続ではないと反論した。議論の両サイド上の詳細を埋めよ。
\code{unless}をどのようにして派生手続として(\code{cond}や\code{let}のように)実装するかを示せ。
そして特殊形式ではなく手続として存在する\code{unless}を持つことが有効である状況の
例を与えよ。
\end{quote}

\subsection{遅延評価を持つインタプリタ}
\label{Section 4.2.2}


この節ではSchemeと同じですが、複合手続が全ての引数に対して非厳密であることが異なる
正規順言語を実装します。プリミティブな手続は依然として厳密です。\link{Section 4.1.1}の
評価機を、それが解釈する言語がこのように振る舞うように変更するのは難しくありません。
ほとんど全ての必要な変更は手続適用が中心となります。


基本的な考えは、手続を適用する時、インタプリタはどの引数が評価されるべきかと、
どの引数が遅延されるべきかを決定しなければなりません。遅延化された引数は評価されません。
その代わりにそれらは\newterm{thunks}(\jnewterm{サンク})と呼ばれるオブジェクトに変形されます。\footnote{\newterm{thunk}という
単語は非公式な作業部会により考案されました。彼等はAlgol 60にてcall\-/by\-/nameの
実装について議論していたのです。彼等は式のほとんどの分析は(``式についての考えは'')
コンパイル時に行えることに気付きました。従って実行時には式は既に(\link{Ingerman et al. 1960})に
係わる``サンク''を持っていました。}サンクは引数の値を生成するために必要な情報を必要な時に
含んでなければなりません。それはまるで適用時に評価されたかのようにです。従って、
サンクは引数の式と手続適用がその中で評価される環境を持たなければなりません。


サンク中の式の評価プロセスは\newterm{forcing}(\jnewterm{強制})と呼ばれます。\footnote{これは
\link{Chapter 3}でストリームを表現するのに導入された遅延化オブジェクト上で\code{force}を
用いることに類似しています。ここで行っていることと、\link{Chapter 3}で行ったことの
重大な違いは、ここでは遅延化と強制を評価機の中に構築していることです。従って
これを言語を通して同一化し、自動化しています。}
一般的にはサンクはその値が必要になった時のみ強制されます。サンクの値を使用する
プリミティブな手続に渡された時です。またオペレータの値であり手続として適用される時です。
設計上の1つの選択として可能なこととして、\link{Section 3.5.1}にて遅延化オブジェクトに我々がしたように、
サンクを\newterm{memoize}(\jnewterm{メモ化})するか否かがあります。メモ化を用いれば、サンクが初めて
強制された時、計算された値が格納されます。続く強制は単純に演算を繰り返すことなく単純に
格納された値を返します。私達はインタプリタをメモ化します。これはとても多くのアプリケーションに
対して効率的だからです。しかし、これには用心しなければならない考慮点が存在します。\footnote{メモ化と
組み合わされた遅延評価は時々、\newterm{call-by-need}(\jnewterm{必要時呼出})引数渡しと呼ばれます。
\newterm{call-by-name}(\jnewterm{名前呼出})引数渡しと対照的です。(call-by-nameはAlgol 60で導入されましたが、
メモ化を行わない遅延評価と同類です)。言語設計者として、私達は評価機をメモ化することも、しないことも、
プログラマに任せることも可能です(\link{Exercise 4.31})。\link{Chapter 3}からおわかりかと思いますが、
これらの選択は微妙で、かつ混乱を招く問題を代入の存在において提起します。(\link{Exercise 4.27}と
\link{Exercise 4.29}を参照)。\link{Clinger (1982)}による素晴しい論文がここで提起される混乱の
複数の特徴を明らかにしようと試みています。}

\subsubsection*{評価機を変更する}

遅延評価と\link{Section 4.1}の評価との主な違いは\code{eval}と\code{apply}における手続適用の取扱に存在します。


\code{eval}の\code{application?}節は以下になります。

\begin{scheme}
((application? exp)
 (apply (actual-value (operator exp) env)
        (operands exp)
        env))
\end{scheme}

\noindent
これは\link{Section 4.1.1}の\code{eval}の\code{application?}節とほとんど同じです。
遅延評価のためにはしかし、\code{apply}をオペランド式と共に呼びます。それらを評価することで
生成された引数と共にではありません。もし引数が遅延化されるのであれば環境にサンクを
構築させる必要が出るので、これも渡さなければいけません。依然として演算子は評価します。
\code{apply}は実際の手続が必要です。その型(プリミティブであるか複合であるか)に従って
呼出と適用を行うためです。


式の実際の値が必要になる度に、\code{eval}するだけの代わりに以下を用います。

\begin{scheme}
(define (actual-value exp env)
  (force-it (eval exp env)))
\end{scheme}

\noindent
これでもし式の値がサンクであれば強制されます。


また新しい版の\code{apply}もほとんど\link{Section 4.1.1}の版と同じです。
違いは\code{eval}が未評価のオペランド式を通り直ぎたことです。(厳密である)プリミティブな手続に対しては
プリミティブを適用する前に全ての引数を評価します。(非厳密である)複合手続に対しては
全ての引数を手続に適用する前に遅延化します。

\begin{scheme}
(define (apply procedure arguments env)
  (cond ((primitive-procedure? procedure)
         (apply-primitive-procedure
          procedure
          (list-of-arg-values arguments env)))   ;~\textrm{changed}~
        ((compound-procedure? procedure)
         (eval-sequence
          (procedure-body procedure)
          (extend-environment
           (procedure-parameters procedure)
           (list-of-delayed-args arguments env)  ;~\textrm{changed}~
           (procedure-environment procedure))))
        (else (error "Unknown procedure type: APPLY"
                     procedure))))
\end{scheme}

\noindent
引数を処理する手続は\link{Section 4.1.1}の\code{list\-/of\-/values}そっくりです。しかし
\code{list\-/of\-/delayed\-/args}が引数を評価するのではなく遅延化することと、
\code{list\-/of\-/arg\-/values}が\code{eval}の代わりに\code{actual\-/value}を用いることが違います。

\begin{scheme}
(define (list-of-arg-values exps env)
  (if (no-operands? exps)
      '()
      (cons (actual-value (first-operand exps)
                          env)
            (list-of-arg-values (rest-operands exps)
                                env))))
(define (list-of-delayed-args exps env)
  (if (no-operands? exps)
      '()
      (cons (delay-it (first-operand exps)
                      env)
            (list-of-delayed-args (rest-operands exps)
                                  env))))
\end{scheme}

\noindent
評価機で変更しなければいけない他の場所は\code{if}の取扱の中にあります。
そこでは\code{eval}の代わりに\code{actual\-/value}を使用して述語が真であるか偽で
あるかテストする前に、述語式の値を取らねばなりません。

\begin{scheme}
(define (eval-if exp env)
  (if (true? (actual-value (if-predicate exp) env))
      (eval (if-consequent exp) env)
      (eval (if-alternative exp) env)))
\end{scheme}

\noindent
最後に、\link{Section 4.1.4}の\code{driver\-/loop}手続を変更して、\code{eval}の代わりに
\code{actual\-/value}を使用せねばなりあせん。そうすることでもし遅延化された値が
REPLに伝播して返った場合に、表示される前に強制されます。またプロンプトも
変更してこれが遅延評価であることを示します。

\begin{scheme}
(define input-prompt  ";;; L-Eval input:")
(define output-prompt ";;; L-Eval value:")
(define (driver-loop)
  (prompt-for-input input-prompt)
  (let ((input (read)))
    (let ((output
           (actual-value
            input the-global-environment)))
      (announce-output output-prompt)
      (user-print output)))
  (driver-loop))
\end{scheme}

\noindent
これらの変更を行うことで、評価機を開始しテストすることができます。
\link{Section 4.2.1}で議論された\code{try}式の評価の成功はインタプリタが遅延評価を
実行していることを示しています。

\begin{scheme}
(define the-global-environment (setup-environment))
(driver-loop)
~\textit{;;; L-Eval input:}~
(define (try a b) (if (= a 0) 1 b))
~\textit{;;; L-Eval value:}~
~\textit{ok}~
~\textit{;;; L-Eval input:}~
(try 0 (/ 1 0))
~\textit{;;; L-Eval value:}~
~\textit{1}~
\end{scheme}

\subsubsection*{サンクの表現}


私達の評価機は手続が引数に適用される時にサンクを作成することと、これらのサンクを
後で強制することの準備をせねばなりません。サンクは式を環境と一緒にまとめなければ
なりません。そうすることで引数が後から生成されることができます。
サンクを強制するためには単純に式と環境をサンクから取り出しその環境の中で式を評価します。
\code{eval}ではなく\code{actual\-/value}を用いることで式の値それ自身がサンクである場合にその
強制をサンクでない物に辿り着くまで繰り返します。

\begin{scheme}
(define (force-it obj)
  (if (thunk? obj)
      (actual-value (thunk-exp obj) (thunk-env obj))
      obj))
\end{scheme}

\noindent
式と環境をまとめる1つの簡単な方法はその式と環境を含むリストを作成することです。
従ってサンクを以下のように作成します。

\begin{scheme}
(define (delay-it exp env)
  (list 'thunk exp env))
(define (thunk? obj)
  (tagged-list? obj 'thunk))
(define (thunk-exp thunk) (cadr  thunk))
(define (thunk-env thunk) (caddr thunk))
\end{scheme}

\noindent
実際に私達のインタプリタに欲っしたものはこれではなく、メモ化されたサンクです。
サンクが強制された時に格納された式とその値に置き換え、
\code{thunk}タグを変更することでサンクをサンクが評価された物に変更することで、
既に評価されたと認識させることができます。\footnote{
一度式の値が計算されれば
サンクから\code{env}もまた消去します。これによりインタプリタにより返される値に違いは
生じません。しかしこれはスペースを本当に節約します。サンクから\code{env}への参照を
必要が無くなったら直ぐに削除することで、この構造が
\newterm{garbage-collected}(\jnewterm{ガベージコレクション}、GC、ゴミ集め)され、そのスペースは
\link{Section 5.3}で議論されるようにリサイクルされます。


同様に、\link{Section 3.5.1}のメモ化、遅延化されたオブジェクトの中の不必要な環境をGCされるように
許可することも、\code{memo\-/proc}に\code{(set! proc '())}の様な何かを行わせて
(\code{delay}が中で評価される環境を含む)手続\code{proc}をその値を格納した後に捨てることでできました。}

\begin{scheme}
(define (evaluated-thunk? obj)
  (tagged-list? obj 'evaluated-thunk))
(define (thunk-value evaluated-thunk) 
  (cadr evaluated-thunk))
(define (force-it obj)
  (cond ((thunk? obj)
         (let ((result (actual-value (thunk-exp obj)
                                     (thunk-env obj))))
           (set-car! obj 'evaluated-thunk)
           (set-car! (cdr obj) 
                     result)     ;~\textrm{\code{exp}をその値で置き換える}~
           (set-cdr! (cdr obj) 
                     '())        ;~\textrm{不必要な\code{env}を忘れる}~
           result))
        ((evaluated-thunk? obj) (thunk-value obj))
        (else obj)))
\end{scheme}

\noindent
同じ\code{delay\-/it}手続がメモ化有りでも無しでも動作することに注意して下さい。

\begin{quote}
\heading{\phantomsection\label{Exercise 4.27}Exercise 4.27:}
以下の定義を遅延評価機に入力したとする。

\begin{scheme}
(define count 0)
(define (id x) (set! count (+ count 1)) x)
\end{scheme}


以下の一連の応答の欠けた値を与えよ。そして貴方の回答について説明せよ。\footnote{この
課題は遅延評価と副作用の間の応答が大きな混乱の原因となりえることを実演している。
これこそが\link{Chapter 3}の議論からあなたが予期するかもしれないことである。}

\begin{scheme}
(define w (id (id 10)))
~\textit{;;; L-Eval input:}~
count
~\textit{;;; L-Eval value:}~
~\( \dark \langle \)~~\var{\dark response}~~\( \dark \rangle \)~
~\textit{;;; L-Eval input:}~
w
~\textit{;;; L-Eval value:}~
~\( \dark \langle \)~~\var{\dark response}~~\( \dark \rangle \)~
~\textit{;;; L-Eval input:}~
count
~\textit{;;; L-Eval value:}~
~\( \dark \langle \)~~\var{\dark response}~~\( \dark \rangle \)~
\end{scheme}
\end{quote}

\begin{quote}
\heading{\phantomsection\label{Exercise 4.28}Exercise 4.28:}
\code{eval}は\code{eval}でなく\code{actual\-/value}を用いて\code{apply}に渡す前に
演算子を評価する。演算子の値を強制するためである。この強制の必要性を実演する
例を与えよ。
\end{quote}

\begin{quote}
\heading{\phantomsection\label{Exercise 4.29}Exercise 4.29:}
メモ化しない場合に、メモ化した場合よりも非常に遅く実行されると予測するプログラムを
示せ。また以下の応答について考えよ。\code{id}手続は\link{Exercise 4.27}と同じに定義され
\code{count}は0から始める。

\begin{scheme}
(define (square x) (* x x))
~\textit{;;; L-Eval input:}~
(square (id 10))
~\textit{;;; L-Eval value:}~
~\( \dark \langle \)~~\var{\dark response}~~\( \dark \rangle \)~
~\textit{;;; L-Eval input:}~
count
~\textit{;;; L-Eval value:}~
~\( \dark \langle \)~~\var{\dark response}~~\( \dark \rangle \)~
\end{scheme}


評価機がメモ化された場合とメモ化されない場合の両方について応答を与えよ。
\end{quote}

\begin{quote}
\heading{\phantomsection\label{Exercise 4.30}Exercise 4.30:}
Cy D. Fectは元C言語プログラマである。彼はいくつかのside effects(副作用)が起こらない
のではないかと心配している。遅延評価が列内の式に強制を行わないためである。
最後の1つ以外の列内の式の値は使用されないため(式は変数への代入や表示等の作用のため
だけに存在している)、この値の強制を引き起こす後の使用は存在しない(例えばプリミティブな手続の
引数として)。Cyは従って列を評価する時には列内の最後の1つを除いた全ての式を評価せねば
ならないと考えた。彼は\link{Section 4.1.1}の\code{eval\-/sequence}を変更し\code{eval}でなく
\code{actual\-/value}を使用することを提案した。

\begin{scheme}
(define (eval-sequence exps env)
  (cond ((last-exp? exps) (eval (first-exp exps) env))
        (else (actual-value (first-exp exps) env)
              (eval-sequence (rest-exps exps) env))))
\end{scheme}

\begin{enumerate}[a]

\item
Ben BitdiddleはCyが間違っていると考えた。彼はCyに\link{Exercise 2.23}で説明された
\code{for\-/each}手続を見せた。これは副作用を伴う列の重要な例を与える。 

\begin{scheme}
(define (for-each proc items)
  (if (null? items)
      'done
      (begin (proc (car items))
             (for-each proc (cdr items)))))
\end{scheme}

彼はテキストの評価機(オリジナルの\code{eval\-/sequence}を持つ物)は
これを正しく取り扱うと主張した。

\begin{scheme}
~\textit{;;; L-Eval input:}~
(for-each (lambda (x) (newline) (display x))
          (list 57 321 88))
~\textit{57}~
~\textit{321}~
~\textit{88}~
~\textit{;;; L-Eval value:}~
~\textit{done}~
\end{scheme}

なぜBenが\code{for\-/each}の振舞について正しいのか説明せよ。

\item
CyはBenが\code{for\-/each}について正しいのは同意した。しかし彼が\code{eval\-/sequence}に対する
変更を提案した時、彼が考えていたのはこの種のプログラムではないと言った。
彼は以下の2つの遅延評価の手続を定義した。

\begin{scheme}
(define (p1 x)
  (set! x (cons x '(2)))
  x)
(define (p2 x)
  (define (p e)
    e
    x)
  (p (set! x (cons x '(2)))))
\end{scheme}

オリジナルの\code{eval\-/sequence}を用いた時、\code{(p1 1)}と\code{(p2 1)}の値はいくらか?
Cyが提案した\code{eval\-/sequence}への変更を用いた時には値はどうなるか?

\item
Cyは彼が提案した通りの\code{eval\-/sequence}への変更はaの例の振舞に影響を与えないと
指摘した。なぜこれが正しいのか説明せよ。

\item
遅延評価では列はどのように扱われるべきと考えるか? あなたはCyのアプローチ、テキストの
アプローチ、または他のアプローチのどれを好むか?

\end{enumerate}
\end{quote}

\begin{quote}
\heading{\phantomsection\label{Exercise 4.31}Exercise 4.31:}
この節で取り上げられた取り組み方は少々、不愉快である。Schemeに対して互換性のない
変更を行うためだ。遅延評価を\newterm{upward-compatible extension}(\jnewterm{上位互換性のある拡張})として
実装するほうがより良いだろう。それは通常のSchemeプログラムが依然と同じように
働くということである。これをユーザに引数が遅延されるか、されないかを
コントロールさせるように手続定義の構文を拡張することで可能である。それを行う間、
ユーザに遅延をメモ化させるか、させないかの選択も同様に与えることができるだろう。
例えば、以下の定義は

\begin{scheme}
(define (f a (b lazy) c (d lazy-memo))
  ~\( \dots \)~)
\end{scheme}

\noindent
\code{f}は4つの引数の手続であり、最初と3番目の引数は手続が呼ばれた時に評価され、
2番目は遅延化され、4番目は遅延化とメモ化が行われる。従って通常の手続定義は
通常のSchemeと同じ振舞を行うが、\code{lazy\-/memo}宣言を各複合手続の各パラメタに
追加することでこの節で定義された遅延評価の振舞を行う。この変更の設計と実装は
Schemeに対しそのような拡張の生成を必要とする。あなたは\code{define}に対する新しい構文を
取り扱う新しい構文手続を実装しなければならない。また引数がいつ遅延化されるか、そして
いつ強制するか、またはそれに応じて引数を遅延化するか決定するために\code{eval}や
\code{apply}に対して準備もしなければならない。同時に強制に対してメモ化するか、しないかも
適切に準備すること。
\end{quote}

\subsection{遅延化リストとしてのストリーム}
\label{Section 4.2.3}


\link{Section 3.5.1}ではどのようにストリームを遅延化されたリストとして実装するかについて示しました。
特殊形式\code{delay}と\code{cons\-/stream}導入しました。このことは私達にストリームの\code{cdr}を
求める``promise''(プロミス、約束)を、実際には後になるまではプロミスを実行すること
無しに構築することを可能にしました。

遅延評価ではストリームとリストは同一にできます。そのため特殊形式やリストとストリームの
命令を分ける必要はありません。私達が行わなければならないこと全ては\code{cons}が非厳密になる
よう問題を準備することです。これを達成する1つの方法は遅延評価を拡張しプリミティブにも
非厳密を許し\code{cons}をこれらの内の1つとすることです。より簡単な方法は(\link{Section 2.1.3})の
\code{cons}をプリミティブとして実装する必要性は本質的には全く無いということを思い出すことです。
その代わりに、ペアは手続として表現可能です。\footnote{これが\link{Exercise 2.4}で説明された
手続表現です。本質的にはどんな手続表現(例えばメッセージパッシングによる実装)も同じことが行え
るでしょう。これらの定義を遅延評価に単純にドライバーループにて型を付けることのみでインストール
できることに注意して下さい。もし私達が元々\code{cons}, \code{car}, \code{cdr}をグローバル環境の
プリミティブとして含めていたのならば、それらは再定義されるでしょう。(\link{Exercise 4.33}と
\link{Exercise 4.34}も参照して下さい。}

\begin{scheme}
(define (cons x y) (lambda (m) (m x y)))
(define (car z) (z (lambda (p q) p)))
(define (cdr z) (z (lambda (p q) q)))
\end{scheme}

\noindent
これらの基本的な命令において、リスト命令の標準定義は無限リスト(ストリーム)と同様に
有限な物としても働きます。そしてストリーム命令はリスト命令として実装可能です。
以下にいくつかの例を示します。

\begin{scheme}
(define (list-ref items n)
  (if (= n 0)
      (car items)
      (list-ref (cdr items) (- n 1))))
(define (map proc items)
  (if (null? items)
      '()
      (cons (proc (car items)) (map proc (cdr items)))))
(define (scale-list items factor)
  (map (lambda (x) (* x factor)) items))
(define (add-lists list1 list2)
  (cond ((null? list1) list2)
        ((null? list2) list1)
        (else (cons (+ (car list1) (car list2))
                    (add-lists (cdr list1) (cdr list2))))))
(define ones (cons 1 ones))
(define integers (cons 1 (add-lists ones integers)))
~\textit{;;; L-Eval input:}~
(list-ref integers 17)
~\textit{;;; L-Eval value:}~
~\textit{18}~
\end{scheme}

\noindent
これらの遅延リストは\link{Chapter 3}のストリームよりもさらに遅延化されています。
リストの\code{car}も\code{cdr}と同様に遅延化されます。\footnote{このことはより一般的な
種類の、ただの列ではない、リスト構造の遅延化版を作ることを可能にします。
\link{Hughes 1990}は``遅延化木''のいくつかのアプリケーションについて議論しています。}実際に、
遅延化ペアの\code{car}や\code{cdr}に対するアクセスさえもリスト要素の値を強制する必要がありません。
その値はそれが本当に必要になった場合---例えばプリミティブの引数としてや回答として
表示される場合に---強制されることになります。

遅延化ペアはまた\link{Section 3.5.4}でストリームに対して提起された問題に対する手助けにも
なります。その時はループを伴なうシステムのストリームモデルを定式化することは
明示的な\code{delay}命令を\code{cons\-/stream}で提供されるものを越えてプログラムの中に
撒き散らすことを必要としました。遅延評価では全ての手続の引数は遅延化に統一されています。
例えばリストと統合する手続を実装し微分方程式を\link{Section 3.5.4}で元々意図したように実装する
ことが可能です。

\begin{scheme}
(define (integral integrand initial-value dt)
  (define int
    (cons initial-value
          (add-lists (scale-list integrand dt) int)))
  int)
(define (solve f y0 dt)
  (define  y (integral dy y0 dt))
  (define dy (map f y))
  y)
~\textit{;;; L-Eval input:}~
(list-ref (solve (lambda (x) x) 1 0.001) 1000)
~\textit{;;; L-Eval value:}~
~\textit{2.716924}~
\end{scheme}

\begin{quote}
\heading{\phantomsection\label{Exercise 4.32}Exercise 4.32:}
\link{Chapter 3}のストリームとこの節で説明された``より遅延化された''遅延化リストの間
の違いを説明する例をいくつか上げよ。この拡張された遅延性の利点をどのように活用するか?
\end{quote}

\begin{quote}
\heading{\phantomsection\label{Exercise 4.33}Exercise 4.33:}
Ben Bitdiddleは上で与えられた遅延リスト実装を式\code{(car '(a b c))}を評価することで
テストした。

\begin{scheme}
(car '(a b c))
\end{scheme}

驚いたことにこれはエラーを生じる。幾らか考えた後に、彼はクォートされた
式を読み込むことで得られた``リスト''が\code{cons}, \code{car}, \code{cdr}の新しい定義で
操作されたリストから異なることに気が付いた。評価機のクォートされた式の扱いを
修正しドライバループで型付けされたクォートされたリストが正しい遅延リストを
生成するようにせよ。
\end{quote}

\begin{quote}
\heading{\phantomsection\label{Exercise 4.34}Exercise 4.34:}
評価機のドライバループを変更し遅延化したペアとリストが何らかの妥当な方法で
表示を行うようにせよ。(無限リストに対しては何を行うか?)。遅延化ペアの表現も
変更が必要になるだろう。評価機がそれらを表示するためにそれらを判別することが
できるようにするためである。
\end{quote}

\section{Scheme上でのバリエーション --- 非決定性演算}
\label{Section 4.3}


この節では\newterm{nondeterministic computing}(\jnewterm{非決定性演算})と呼ばれる
プログラミングパラダイムをサポートするよう、評価機の中に自動的な探索をサポートするための
機能を構築することで、Scheme評価機を拡張します。これは\link{Section 4.2}での
遅延評価の導入に比べ、とても深い言語への変更です。


非決定性演算はストリーム処理のように、``生成してテストする''アプリケーションに対して
便利です。正の整数の2つのリストを用いて開始する、整数のペアを見つけるタスクについて考えます
1つは最初のリストから、もう1つは別のリストから取得し、その和は素数となります。
私達はこの問題をどのように扱うかについて、\link{Section 2.2.3}では有限列の命令を用いて、
\link{Section 3.5.3}では無限ストリームを用いる方法について学びました。私達の取り組み方は
全ての可能なペアを生成し、これらから和が素数になるペアを選択するという方法でした。
実際にペアの列全体を最初に生成する\link{Chapter 2}や、生成とフィルタリングを相互配置する
\link{Chapter 3}に係わらず演算がどのように体系化されるかの本質的なイメージに対しては
重要ではありませんでした。


非決定性の取り組み方は異なるイメージを喚起します。単純に(何らかの方法で)最初のリストから
数値を選択し、別の数値を2つ目のリストから選択し、(何らかの仕組みで)それらの和が素数である
ことを要求とすると想像してみてください。これは以下の手続により表現されます。

\begin{scheme}
(define (prime-sum-pair list1 list2)
  (let ((a (an-element-of list1))
        (b (an-element-of list2)))
    (require (prime? (+ a b)))
    (list a b)))
\end{scheme}

\noindent
この手続はただ単に問題を言い直したに過ぎなく、解法を指定したようには見えないかも
しれません。それにもかかわらず、これは正規の非決定性プログラムです。\footnote{以前に
数値が素数であるかをテストする手続\code{prime?}を定義しました。例え\code{prime?}が
定義されていたとしても、\code{prime\-/sum\-/pair}手続は、\link{Section 1.1.7}の最初で説明した
助けにならない``擬似Lisp''による平方根関数の定義の試みのように疑わしく見えるかも
しれません。現実に、あれらの行に沿った平方根手続が実際に非決定性プログラムとして
定式化することができます。評価機に探索の仕組みを合併することで、どのようにして回答を
計算するかについての純粋な宣言型の記述と命令型の仕様の間の区別を侵食していきます。
私達は\link{Section 4.4}にてこの方向へとさらに進みます。}

ここでの鍵となる考えは、非決定性言語における式は1つ以上の可能な値を持つことが
できるということです。例えば\code{an\-/element\-/of}は与えられたリストのどの要素でも
返すことが有り得ます。私達の非決定性プログラム評価機は自動的に可能な値を選択し
その選択を追跡することで働きます。もし続く要件に合わなければ、評価機は異なる
選択を試します。そして評価が成功するまで、または選択肢が無くなるまで、新しい選択を試し続けます。
遅延評価がプログラマを値がどのように遅延化され強制されるかの詳細から解放されたのと
同様に、非決定性プログラムの評価機はプログラマを選択がどのように行われるかの詳細から
解放します。

非決定性評価とストリーム処理により起こった時間の異なるイメージの対比は
示唆的です。ストリーム処理は遅延評価を可能な答のストリームが組まれた時間を
実際のストリーム要素が生成された時間から分離します。評価機は全ての可能な
回答が私達の前に永遠の列の中に横たわっているというイリュージョンを支えます。
非決定性評価機では式は可能な世界の集合の調査を表現します。それぞれは選択の
集合により判断されます。可能なせかい のいくつかは行き止まりへと導き、一方、
他は役立つ値へと導きます。非決定性プログラム評価機は時間の分岐と私達のプログラムが
異なる可能な実行履歴を持つというイリュージョンを支えます。行き止まりに辿り着いた
時には直前の選択地点に戻り、異なる分岐に従って進むことが可能です。

以下で実装される非決定性プログラム評価機は\code{amb}評価機と呼ばれます。
それが\code{amb}と呼ばれる新しい特殊形式に基づくためです。上記の\code{prime\-/sum\-/pair}の
定義を\code{amb}評価機ドライバループにて(\code{prime?}, \code{an\-/element\-/of}, \code{require}と
一緒に)型を付け、以下のように手続を実行することができます。

\begin{scheme}
~\textit{;;; Amb-Eval input:}~
(prime-sum-pair '(1 3 5 8) '(20 35 110))
~\textit{;;; Starting a new problem}~
~\textit{;;; Amb-Eval value:}~
~\textit{(3 20)}~
\end{scheme}

\noindent
返り値は評価機が条件に合う選択が行われるまで繰り返し各リストから要素を選択した
後に取得されました。


\link{Section 4.3.1}は\code{amb}を紹介しそれがどのように非決定性を評価機の自動探索の仕組みを
通してサポートするかについて説明します。\link{Section 4.3.2}は非決定性プログラムの例を与え、
\link{Section 4.3.3}は\code{amb}の実装方法の詳細を通常のScheme評価機を変更することで与えます。



\subsection{ambと検索}
\label{Section 4.3.1}

Schemeに非決定性を対応するよう拡張するために、\code{amb}と呼ばれる新しい特殊形式を
導入します。\footnote{非決定性プログラミングのための\code{amb}の考えは1961年に
最初にJohn McCarthyにより説明されました。(\link{McCarthy 1963}参照)。}

\begin{scheme}
(amb ~\( \dark \langle \)~~\( \dark e_1 \)~~\( \dark \rangle \)~ ~\( \dark \langle \)~~\( \dark e_2 \)~~\( \dark \rangle \)~ ~\( \dots \)~ ~\( \dark \langle \)~~\( \dark e_n \)~~\( \dark \rangle \)~)
\end{scheme}

\noindent
上の式は\( n \)個の式\( \langle \)\( e_i \)\( \kern0.08em\rangle \)の内1つの値を``ambiguously''(曖昧に)
返します。例えば以下の式は

\begin{scheme}
(list (amb 1 2 3) (amb 'a 'b))
\end{scheme}

\noindent
以下の6つの値の可能性があります。

\begin{scheme}
~\code{(1 a)}~ ~\code{(1 b)}~ ~\code{(2 a)}~ ~\code{(2 b)}~ ~\code{(3 a)}~ ~\code{(3 b)}~
\end{scheme}

\noindent
単一の選択を行う\code{amb}は通常の(単一の)値を生成します。

\noindent
選択を行わない\code{amb}---式\code{(amb)}---は受け入れられる値の無い式です。操作上、
\code{(amb)}を評価された時に演算に対し``fail''(失敗)を起こさせると考えることができます。
演算は異常終了し、何の値も生成されません。この考えを用いて、ある特定の述語式\code{p}が
真でなければならないことを以下のように表現可能です。

\begin{scheme}
(define (require p) (if (not p) (amb)))
\end{scheme}

\noindent
\code{amb}と\code{require}を用いて上で使用された\code{an\-/element\-/of}を実装可能です。

\begin{scheme}
(define (an-element-of items)
  (require (not (null? items)))
  (amb (car items) (an-element-of (cdr items))))
\end{scheme}

\noindent
\code{an\-/element\-/of}はリストが空である場合には失敗します。そうでなければリストの
最初の要素か、リストの残りの要素から選択された要素を曖昧に返します。

無限の範囲の選択も表現可能です。以下の手続は可能性としてある与えられた\( n \)に
等しいか大きな任意の整数を返します。

\begin{scheme}
(define (an-integer-starting-from n)
  (amb n (an-integer-starting-from (+ n 1))))
\end{scheme}

\noindent
これはまるで\link{Section 3.5.2}で説明されたストリーム手続\code{integers\-/starting\-/from}の様です。
しかし重要な違いがあります。ストリーム処理は\( n \)で始まる全ての整数の列を表す
オブジェクトを返します。一方、\code{amb}手続は単一の整数を返します。\footnote{本当は
非決定的に単一の選択を返すことと全ての選択を返すことの違いは私達の視点に
幾分、依存します。値を使用するコードの視点からは非決定性による選択は単一の値を返します。
コードを設計するプログラマの視点からは非決定性による選択は潜在的に全ての可能な値を
返します。そして各値が個別に調査されるように演算は分岐するのです。}

抽象的に、\code{amb}式を評価することが時間に対して分岐を起こさせ、演算は各分岐上にて
式の可能な値の1つに上で続行するのだと想像することができます。\code{amb}が
\newterm{nondeterministic choice point}(\jnewterm{非決定性選択点})を表現すると言えます。
もし私達が動的に獲得できる十分な数のプロセッサを持つ計算機を持っているならば、
探索を簡単な方法で実装できるでしょう。実行は\code{amb}式に遭遇するまでは逐次的に
行われます。遭遇した時点では多くのプロセッサが獲得され選択により暗示された
全ての並列実行を続けるために初期化されます。各プロセッサは選択がそれしか無かった
かのように逐次的に続行します。その処理は失敗に遭遇して停止するか、さらなる分岐が
起こるか、完了するまで続けられます。\footnote{これは絶望的に非効率だと異議を唱える
人がいるかもしれません。この方法では簡単に規定された問題を解くのに数百万の
プロセッサを必要とするかもしれません。そしてそれらの多くのプロセッサはほとんどの
時間をアイドル状態(遊休状態)になるでしょう。この異議は歴史の文脈で捕えられるべきです。
メモリはとても高価な消費財だと考えられてきました。1964年にはメガバイトのRAMは\$400,000
の費用が掛かりました。現在では全てのPCが多数のメガバイトのRAMを積んでいます。そして
ほとんどの時間でほとんどのメモリは利用されていません。大量生産された電子製品の
コストを過小評価することは難しいことです。}

一方で、もし私達が1つのプロセス(またはいくつかの並行プロセス)しか実行できない計算機を持っている場合には
逐次的に動作する代替法を考えねばなりません。1つの方法としては評価機を選択点に辿り着いた時に
無作為に分岐を選択するよう変更することが考えられるでしょう。しかし無作為な選択は簡単に
失敗する値へと導きます。評価機を何度も何度も実行し無作為な選択を行い失敗しない値を見つけることを
期待するかもしれません。しかし全ての可能な実行パス(実行経路)を
\newterm{systematically search}(\jnewterm{体系的探索})をしたほうがより良いです。私達がこの節で開発し
働きかける\code{amb}評価機は体系的探索を次のように実装します。評価機が\code{amb}の適用に遭遇した
場合に初期値として最初の選択肢を選択します。この選択それ自身がさらなる選択へと導きます。
評価機は常に初期値として最初の選択肢を各選択点にて選択します。もし選択の結果が失敗となれば
評価機はautomagically\footnote{automagically: ``自動的に、しかし幾つかの理由で(典型的には
複雑過ぎて、または酷すぎて、または恐らくさらにどうでも良すぎて)話者が説明する気にならない''
(\link{Steele et al. 1983}, \link{Raymond 1993})\label{Footnote 4.47}}に最も最近の選択に
\newterm{backtracks}(\jnewterm{バックトラック}、引き返す)し、次の選択肢を試行します。もしどこかの
選択点において、全ての選択肢を使用してしまえば、評価機は以前の選択点へと戻りそこから
再開します。この処理は\newterm{depth-first search}(\jnewterm{深さ優先探索})または\newterm{chronological
backtracking}(クロノロジカルバックトラック、\jnewterm{年代順バックトラック})として知られる
探索戦略へと導きます。\footnote{自動的な探索戦略の
プログラミング言語への統合は長く功罪相半ばする歴史があります。非決定性アルゴリズムが美しく
プログラミング言語へと探索と自動的バックトラックと共に組込まれただろう最初の提案は
Robert \link{Floyd (1967)}により行われました。Carl \link{Hewitt (1969)}はPlannerと呼ばれる
プログラミング言語を発明しましたが、これは明示的に自動的なクロノロジカルバックトラックを
サポートし、組込の深さ優先探索戦略を提供していました。\link{Sussman et al. (1971)}はこの言語の
部分集合であるMicroPlannerを実装しました。これは問題解決とロボット計画の仕事の支持に使用され
ました。同様なアイデアが論理と定理証明から提起され、エディンバラとマルセイユにて洗練された
言語Prologの起源へと導きました。(\link{Section 4.4}にて議論します)。自動探索に対する多くの不満の後に
\link{McDermott and Sussman (1972)}はConniverと呼ばれる言語を開発しました。これはプログラマの
コントロール下に探索戦略を置くための仕組みを含みました。しかしこれは扱い難く、
\link{Sussman and Stallman 1975}はもっと御しやすい取り組み方を電子回路向け記号分析の手法の
研究の間に発見しました。彼等は事実を繋げる論理的依存性の追跡を基にした非クロノロジカルな
バックトラック計画を開発しました。これは
\newterm{dependency-directed backtracking}(\jnewterm{依存型バックトラック})として知られるようになった
技術です。彼等の手法は複雑でしたが、合理的な効率の良いプログラムを生成しました。冗長な
探索をほとんどしなかったためです。\link{Doyle (1979)}と\link{McAllester (1978; 1980)}は
StallmanとSussmanの手法を一般化し、明確にしました。そして探索を定式化するための
新しいパラダイムを開発しました。これは今では\newterm{truth maintenance}(\jnewterm{真理維持})と呼ばれています。
現代の問題解決システムは全て真理維持システムの何らかの形式を素地として使用しています。
真理維持システムと真理維持を用いたアプリケーションを構築するための洗練された方法の
議論については\link{Forbus and deKleer 1993}を参照して下さい。
\link{Zabih et al. 1987}はSchemeに対する\code{amb}を基にした非決定性拡張を説明しています。
これはこの節で説明されるインタプリタと同様です。しかしより高度な物です。それが
クロノロジカルバックトラックではなく依存型バックトラックを使用しているためです。
\link{Winston 1992}は両方の種類のバックトラックに対する入門を提供しています。}

\subsubsection*{ドライバループ}

\code{amb}評価機のドライバループはいくつかの普通ではない性質を持っています。
これは式を読み最初の失敗ではない実行の値を上で示された\code{prime\-/sum\-/pair}の例のように
表示します。もし次の成功する実行の値を見たいのであれば、インタプリタにバックトラックして
二番目の失敗ではない実行を生成する試行を命令します。これはシンボル\code{try\-/again}を入力
することで伝えられます。もし\code{try\-/again}ではない任意の式が与えられたなら、
インタプリタは新しい問題を開始し、直前の問題の調査されていない選択肢を捨てます。
以下にサンプルの応答を示します。

\begin{scheme}
~\textit{;;; Amb-Eval input:}~
(prime-sum-pair '(1 3 5 8) '(20 35 110))
~\textit{;;; Starting a new problem}~
~\textit{;;; Amb-Eval value:}~
~\textit{(3 20)}~

~\textit{;;; Amb-Eval input:}~
try-again
~\textit{;;; Amb-Eval value:}~
~\textit{(3 110)}~

~\textit{;;; Amb-Eval input:}~
try-again
~\textit{;;; Amb-Eval value:}~
~\textit{(8 35)}~

~\textit{;;; Amb-Eval input:}~
try-again
~\textit{;;; There are no more values of}~
~\textit{(prime-sum-pair (quote (1 3 5 8)) (quote (20 35 110)))}~

~\textit{;;; Amb-Eval input:}~
(prime-sum-pair '(19 27 30) '(11 36 58))
~\textit{;;; Starting a new problem}~
~\textit{;;; Amb-Eval value:}~
~\textit{(30 11)}~
\end{scheme}

\begin{quote}
\heading{\phantomsection\label{Exercise 4.35}Exercise 4.35:}
2つの与えられた境界値の間の整数を返す手続\code{an\-/integer\-/between}を書け。
これはピタゴラスの3つ組を求める手続を実装するのに使用できる。例えば与えられた範囲の間の
整数の三つ組\( (i, j, k) \)で\( i \le j \) and \( i^2 + j^2 = k^2 \)の場合は以下の様になる。

\begin{scheme}
(define (a-pythagorean-triple-between low high)
  (let ((i (an-integer-between low high)))
    (let ((j (an-integer-between i high)))
      (let ((k (an-integer-between j high)))
        (require (= (+ (* i i) (* j j)) (* k k)))
        (list i j k)))))
\end{scheme}
\end{quote}

\begin{quote}
\heading{\phantomsection\label{Exercise 4.36}Exercise 4.36:}
\link{Exercise 3.69}ではどのように``全ての''ピタゴラスの三つ組のストリームを探索対象の整数の
サイズに上限無しで生成するかについて議論した。なぜ単純に\code{an\-/inte\-/ger\-/between}を
\link{Exercise 4.35}の手続内の\code{an\-/integer\-/starting\-/from}で置き換えることは、
自由裁量なピタゴラスの三つ組を生成するのに適切でないのか、説明せよ。
これを実際に達成する手続を書け。
(すなわち、原理上は\code{try\-/again}を繰り返し入力することで全てピタゴラスの三つ組を生成する手続を書け)。
\end{quote}

\begin{quote}
\heading{\phantomsection\label{Exercise 4.37}Exercise 4.37:}
Ben Bitdiddleは以下のピタゴラスの3つ組を生成する手法は\link{Exercise 4.35}の手法に比べ
より効率的であると主張した。彼は正しいだろうか? (ヒント:探索しなければならない可能性の
数を考えよ)

\begin{scheme}
(define (a-pythagorean-triple-between low high)
  (let ((i (an-integer-between low high))
        (hsq (* high high)))
    (let ((j (an-integer-between i high)))
      (let ((ksq (+ (* i i) (* j j))))
        (require (>= hsq ksq))
        (let ((k (sqrt ksq)))
          (require (integer? k))
          (list i j k))))))
\end{scheme}
\end{quote}

\subsection{非決定性プログラムの例}
\label{Section 4.3.2}
\link{Section 4.3.3}は\code{amb}評価機の実装を説明します。しかし最初にそれがどのように使用できるかに
ついて、いくつかの例を与えます。非決定性プログラミングの利点は探索がどのように実行されるのか
についての詳細を隠すことができることです。従って抽象の高いレベルにてプログラムを表現できます。

\subsubsection*{論理パズル}


以下のパズル(\link{Dinesman 1968}から拝借しました)は典型的な大きなクラスの簡単な
論理パズルです。

\begin{quote}
Baker, Cooper, Fletcher, Miller, それにSmithは同じ5階しかないアパートの異なる階に
住んでいます。Bakerは最上階には住んでいません。Cooperは最下階には住んでいません。
Fletcherは最上階にも最下階にも住んでいません。MillerはCooperよりも高い階に住んでいます。
SmithはFletcherのすぐ隣の階には住んでいません。FletcherはCooperのすぐ隣の階には住んでいません。
皆はどの階に住んでいるでしょうか?
\end{quote}

\noindent
誰がどの階に住んでいるかを、全ての可能性を列挙し与えられた制約を与える簡単な方法で
決定することができます。\footnote{
私達のプログラムは以下の手続をリストの要素が識別可能
であるか判断するために使用しています。

\begin{smallscheme}
(define (distinct? items)
  (cond ((null? items) true)
        ((null? (cdr items)) true)
        ((member (car items) (cdr items)) false)
        (else (distinct? (cdr items)))))
\end{smallscheme}

\noindent
\code{member}は\code{memq}と同様ですがこれは\code{eq?}の代わりに\code{equal?}を
等価性のテストに用いています。}

\begin{scheme}
(define (multiple-dwelling)
  (let ((baker    (amb 1 2 3 4 5)) (cooper (amb 1 2 3 4 5))
        (fletcher (amb 1 2 3 4 5)) (miller (amb 1 2 3 4 5))
        (smith    (amb 1 2 3 4 5)))
    (require
     (distinct? (list baker cooper fletcher miller smith)))
    (require (not (= baker 5)))
    (require (not (= cooper 1)))
    (require (not (= fletcher 5)))
    (require (not (= fletcher 1)))
    (require (> miller cooper))
    (require (not (= (abs (- smith fletcher)) 1)))
    (require (not (= (abs (- fletcher cooper)) 1)))
    (list (list 'baker baker)       (list 'cooper cooper)
          (list 'fletcher fletcher) (list 'miller miller)
          (list 'smith smith))))
\end{scheme}

\noindent
式\code{(multiple\-/dwelling)}を評価すると結果を生成します。

\begin{scheme}
((baker 3) (cooper 2) (fletcher 4) (miller 5) (smith 1))
\end{scheme}

\noindent
この簡単な手続はうまく行きますが、とても遅いです。\link{Exercise 4.39}と
\link{Exercise 4.40}はいくらかの可能な改善法について議論します。

\begin{quote}
\heading{\phantomsection\label{Exercise 4.38}Exercise 4.38:}
multiple-dwelling手続を変更し、SmithとFletcherが隣接する階に住んでいないという
要件を取り除く。この変更したパズルにはいくつの解が存在するか?
\end{quote}

\begin{quote}
\heading{\phantomsection\label{Exercise 4.39}Exercise 4.39:}
multiple-dwelling手続内の制約の順は解に影響するだろうか? 回答を見つけるのにかかる
時間には影響を与えるだろうか? もしそれが重要であると考えるのなら、制約の順を変える
ことで与えられる物から得られるより速いプログラムを実演せよ。
もしそれが問題ではないと考えるのなら、あなたの考えを論ぜよ。
\end{quote}

\begin{quote}
\heading{\phantomsection\label{Exercise 4.40}Exercise 4.40:}
複数の住居の問題において、人を階へと割り当てる方法は、requirementの前と後で階の割り当てが
区別可能な方法でいくつあるか? 全ての人から階への可能な割り当てを生成してからその次に
それらを排除するためにbacktrackにまかせることは非常に非効率である。
例えば制約のほとんどは一つか二つの人と階の変数を持ち、従って全ての人に対して階が
選択される前に制約を与えることができる。この問題を先の制約により既に排除されたもの
以外の可能性のみを生成することに基づいて解くずっと効率の良い非決定性手続を書き、実演せよ。
\end{quote}

\begin{quote}
\heading{\phantomsection\label{Exercise 4.41}Exercise 4.41:}
複数住居パズルを解く通常のSchemeプログラムを書け。
\end{quote}

\begin{quote}
\heading{\phantomsection\label{Exercise 4.42}Exercise 4.42:}
次の``嘘つき''パズル(\link{Phillips 1934}から)を解け

5人の女生徒が試験のために座っている。彼女達は、彼女等の両親が
結果に過大な興味を見せていると考えている。従って彼女達は次のことを合意した。
それぞれの少女が1つの正しい文と嘘の文を作り、各家庭に試験についての手紙を書く。
以下は彼女らの手紙の該当する一節である。

\begin{itemize}

\item
Betty: ``Kittyが試験では二位だった。私だけが3位だった。''

\item
Ethel: ``喜んで、私がトップ。Joanが2位だった。''

\item
Joan: ``私が3番。可哀想なEthelは最下位だった。''

\item
Kitty: ``私が2番。Maryが単独で4位。''

\item
Mary: ``私が4位。トップはBettyが取ったわ。''

\end{itemize}

実際にはどの順に5人の女の子は並べられるか?
\end{quote}

\begin{quote}
\heading{\phantomsection\label{Exercise 4.43}Exercise 4.43:}
\code{amb}評価機を用いて以下のパズルを解け\footnote{これは1960年代にLitton Industriesにより出版された
``問題の多いレクリエーション''と呼ばれる小冊子から引用しました。
\textit{Kansas State Engineer}著。}

Mary Ann Mooreの父はヨットを持っており、彼の4人の友達、
Downing大佐, Hallさん, Barnacle Hood卿, Dr. Parkerもそれぞれが持っていました。
5人のそれぞれに一人の娘がおり、それぞれが各自のヨットに他人の娘の名を取って付けていました。
Barnacle卿のヨットはGabrielleで、MooreさんのはLornaです。HallさんのはRosalindです。
Downing大佐のMelissaはBarnacle卿の娘の名を取って付けました。
Gabrielleの父のヨットはDr. Parkerの娘からです。Lornaの父は誰でしょう?

効率良く実行されるプログラムを書くように努めること(\link{Exercise 4.40}参照)。
また、もしMary Annの家族名がMooreであることを伝えなければ、いくつの解が存在するだろうか?
\end{quote}

\begin{quote}
\heading{\phantomsection\label{Exercise 4.44}Exercise 4.44:}
\link{Exercise 2.42}はチェス盤に8つのクイーンをどの2つもお互いに攻撃することが無いように
置く``8クイーンパズル''について説明した。このパズルを解く非決定性プログラムを書け。
\end{quote}

\subsubsection*{自然言語の構文解析}

自然言語を入力として受け入れるよう設計されたプログラムは通常、その入力を
\newterm{parse}(\jnewterm{パース}、\jnewterm{構文解析})することから始めます。つまり入力を
ある文法構造に対して合わせることです。例えば冠詞と続く名詞、続く動詞から成る
簡単な文、''The cat eats.``(猫は食べる)のような物を認識しようとしていると
します。そのような分析を達成するためには個別の単語の品詞を判別できなければ
なりません。多種の単語を判別できるいくつかのリストから始めることが
できるでしょう。\footnote{ここでは各リストの最初の要素はリストの残りの単語の
品詞を示すという仕様を用いています。}

\begin{scheme}
(define nouns '(noun student professor cat class))
(define verbs '(verb studies lectures eats sleeps))
(define articles '(article the a))
\end{scheme}

\noindent
また\newterm{grammar}(\jnewterm{文法})も必要です。つまり、文法上の要素がどのようにより簡単な要素から
組み立てられるのかを説明するルールの集合です。とても簡単な文法は文は常に2つの要素---名詞句と
それに続く動詞---により成ると規定することができるかもしれません。そして名詞句は定冠詞と
それに続く名詞から成り立ちます。この文法を用いて、文``The cat eats.''は以下のように
パースされます。

\begin{scheme}
(sentence (noun-phrase (article the) (noun cat))
          (verb eats))
\end{scheme}

\noindent
そのようなパースを、手続を各文法上のルールに分割する簡単なプログラムを用いて生成することができます。
文をパースするためには、それを構成する2つの要素を判別し、これらの2つの要素のリストを
シンボル\code{sentence}のタグを付けて返します。


\begin{scheme}
(define (parse-sentence)
  (list 'sentence
         (parse-noun-phrase)
         (parse-word verbs)))
\end{scheme}

\noindent
名詞句も同様に定冠詞とそれに続く名詞を見つけることでパースされます。

\begin{scheme}
(define (parse-noun-phrase)
  (list 'noun-phrase
        (parse-word articles)
        (parse-word nouns)))
\end{scheme}

\noindent
最も低いレベルでは、パースとは繰り返し次のパースされていない単語が必要とされる品詞の
ための単語のリストのメンバであるかをチェックすることだとまとめられます。
これを実装するために、私達はグローバル変数\code{*unparsed*}を持ちます。これは
まだパースされていない入力です。単語をチェックする各時点で\code{*unparsed*}が空では
ないことを要求し、また指定されたリストの単語で始まることを要求します。もしそうで
あるならその単語を\code{*unparsed*}から削除し、その単語をその品詞(これはリストの先頭に
見つかります)と共に返します。\footnote{\code{parse\-/word}がparseされていない入力リストの
変更するのに\code{set!}を用いているおおに注意して下さい。これをうまく行うためには、
\code{amb}評価機は\code{set!}命令の効果をバックトラックする時に取り消しできなければいけません。}

\begin{scheme}
(define (parse-word word-list)
  (require (not (null? *unparsed*)))
  (require (memq (car *unparsed*) (cdr word-list)))
  (let ((found-word (car *unparsed*)))
    (set! *unparsed* (cdr *unparsed*))
    (list (car word-list) found-word)))
\end{scheme}

\noindent
パースを始めるために行わなければいけないこと全ては\code{*unparsed*}に入力全体を設定し、
文のパースを試み、何も残っていないことをチェックすることです。

\begin{scheme}
(define *unparsed* '())
(define (parse input)
  (set! *unparsed* input)
  (let ((sent (parse-sentence)))
    (require (null? *unparsed*)) sent))
\end{scheme}

\noindent
これでパーザ(parser、パースを行うプログラム)を試し、簡単なテスト文に対してうまく働くこと
を確認することができます。

\begin{scheme}
~\textit{;;; Amb-Eval input:}~
(parse '(the cat eats))
~\textit{;;; Starting a new problem}~
~\textit{;;; Amb-Eval value:}~
\end{scheme}
\begin{smallscheme}
~\textit{(sentence (noun-phrase (article the) (noun cat)) (verb eats))}~
\end{smallscheme}

\noindent
\code{amb}評価機はここでとても役立ちます。\code{require}の助けを用いてパースする上での
制約を表現するのにとても便利なためです。しかし、自動的な探索とバックトラックが
本当に効果を生むのはより複雑な文法について考えた時に、1つの単位の分解方法に
多数の選択肢が存在する場合です。


私達の文法に前置詞を追加してみましょう。

\begin{scheme}
(define prepositions '(prep for to in by with))
\end{scheme}

\noindent
そして前置詞句(例えば``for the cat''(猫のために))を名詞句の前の前置詞として定義します。

\begin{scheme}
(define (parse-prepositional-phrase)
  (list 'prep-phrase
        (parse-word prepositions)
        (parse-noun-phrase)))
\end{scheme}

\noindent
これで文は名詞句に動詞句が続くと定義でき、動詞句は動詞か、または前置詞句で拡張された動詞句
となります。\footnote{この定義が再帰的であることに注意して下さい。動詞には任意の数の
前置詞句が続けられます。}

\begin{scheme}
(define (parse-sentence)
  (list 'sentence (parse-noun-phrase) (parse-verb-phrase)))
(define (parse-verb-phrase)
  (define (maybe-extend verb-phrase)
    (amb verb-phrase
         (maybe-extend
          (list 'verb-phrase
                verb-phrase
                (parse-prepositional-phrase)))))
  (maybe-extend (parse-word verbs)))
\end{scheme}

\noindent
ここまでを行っている間に、名詞句の定義に``a cat in the class''(クラスの猫)のような
物を認めるよう詳細を詰めることができます。今まで名詞句と呼んできた物はこれからは
シンプルな名詞句と呼びます。そして名詞句はこれからシンプルな名詞句か前置詞句で
拡張した名詞句となります。

\begin{scheme}
(define (parse-simple-noun-phrase)
  (list 'simple-noun-phrase
        (parse-word articles)
        (parse-word nouns)))
(define (parse-noun-phrase)
  (define (maybe-extend noun-phrase)
    (amb noun-phrase
         (maybe-extend
          (list 'noun-phrase
                noun-phrase
                (parse-prepositional-phrase)))))
  (maybe-extend (parse-simple-noun-phrase)))
\end{scheme}

\noindent
私達の新しい文法はより複雑な文をパースできます。例えば、

\begin{scheme}
(parse '(the student with the cat sleeps in the class))
\end{scheme}

\noindent
(猫と一緒にその生徒はクラスで寝ている)は以下を生成します。

\begin{scheme}
(sentence
 (noun-phrase
  (simple-noun-phrase (article the) (noun student))
  (prep-phrase
   (prep with)
   (simple-noun-phrase (article the) (noun cat))))
 (verb-phrase
  (verb sleeps)
  (prep-phrase
   (prep in)
   (simple-noun-phrase (article the) (noun class)))))
\end{scheme}

\noindent
与えられた入力が二つ以上の有効な分析結果を持つかもしれないことを確認して下さい。
文``The professor lectures to the student with the cat''はprofessor(教授)が猫と一緒に
講義をしている場合と、学生が猫を持っている場合が有り得ます。
私達の非決定性プログラムは両方の可能性を見つけます。

\begin{scheme}
(parse '(the professor lectures to the student with the cat))
\end{scheme}

\noindent
は以下を生成します。

\begin{scheme}
(sentence
 (simple-noun-phrase (article the) (noun professor))
 (verb-phrase
  (verb-phrase
   (verb lectures)
   (prep-phrase
    (prep to)
    (simple-noun-phrase (article the) (noun student))))
  (prep-phrase
   (prep with)
   (simple-noun-phrase (article the) (noun cat)))))
\end{scheme}

\noindent
評価機にもう一度試行を命ずると以下を生じます。

\begin{scheme}
(sentence
 (simple-noun-phrase (article the) (noun professor))
 (verb-phrase
  (verb lectures)
  (prep-phrase
   (prep to)
   (noun-phrase
    (simple-noun-phrase (article the) (noun student))
    (prep-phrase
     (prep with)
     (simple-noun-phrase (article the) (noun cat)))))))
\end{scheme}

\begin{quote}
\heading{\phantomsection\label{Exercise 4.45}Exercise 4.45:}
上で与えられた文法を用いて次の文は5通りにパースできる。 ``The professor
lectures to the student in the class with the cat''。5通りの結果を与えて
それらの間の様々な意味の違いを説明せよ。

\end{quote}

\begin{quote}
\heading{\phantomsection\label{Exercise 4.46}Exercise 4.46:} 
\link{Section 4.1}と\link{Section 4.2}は評価機はどの順でオペランドが評価されるかを決定しない。
我々は\code{amb}評価機がそれらを左から右へと評価するのを見るだろう。
なぜ我々のパーザはオペランドが何らかの他の順で評価されたならうまく働かないのか
説明せよ。
\end{quote}

\begin{quote}
\heading{\phantomsection\label{Exercise 4.47}Exercise 4.47:}
Louis Reasonerは動詞句は動詞か前置詞句が続く動詞句であるのだから、
手続\code{parse\-/verb\-/phrase}を以下のように(そして同様に名詞句に対しても)
定義すればずっと簡単になるのではないかと提案した。

\begin{scheme}
(define (parse-verb-phrase)
  (amb (parse-word verbs)
       (list 'verb-phrase
             (parse-verb-phrase)
             (parse-prepositional-phrase))))
\end{scheme}

これはうまく行くだろうか? このプログラムの振舞はもし\code{amb}内の式の順を置き換えたら
変わるだろうか?
\end{quote}

\begin{quote}
\heading{\phantomsection\label{Exercise 4.48}Exercise 4.48:}
上で与えられた文法をより複雑な文を取り扱うように拡張せよ。例えば、名詞句と動詞句を
拡張し形容詞と福祉を含める、または複合文を取り扱えるようできるだろう。\footnote{この種の
文法は任意の複雑さに成り得ます。しかし現実の言語の理解を考える限りはただの玩具に
過ぎません。現実の自然言語のコンピュータによる理解は構文解析と意味解釈の念入りな
混合が要求されます。一方で、例えおもちゃのパーザでもプログラムのための柔軟な命令言語、
例えば情報取得システム等をサポートするには実用的です。\link{Winston 1992}は
計算機による自然言語理解への取り組みと共に簡単な文法の命令言語のアプリケーションについても
議論しています。}
\end{quote}

\begin{quote}
\heading{\phantomsection\label{Exercise 4.49}Exercise 4.49:}
Alyssa P. Hackerはパースするよりも面白い文を生成することにより興味を持っている。
彼女は手続\code{parse\-/word}を簡単に変更してそれが``入力文''を無視し、その代わりに
常に成功して適切な単語を生成するようにすれば、このパースのために構築したプログラムを
用いて代わりに生成を行えるのではないかと推測した。Alyssaの考えを実装せよ。そして
生成された文の最初の半ダースかそこらを示せ。\footnote{Alyssaのアイデアはちゃんと
うまく働きますが(そして驚くほど簡単ですが)、それが生成する文は少しつまらないです。
それらはこの言語の可能な文からとても面白い様には抽出はしません。実際に文法は
多くの場所で高度に再帰し、Alyssaの技術はこれらの再帰の1つに``falls into''(陥り)、
抜け出せなくなります。これに対処する方法は\link{Exercise 4.50}を参照して下さい。}
\end{quote}

\subsection{\code{Amb}評価機の実装}
\label{Section 4.3.3}


通常のScheme式の評価機は値を返すか、永遠に停止しないか、またはエラーを発します。
非決定性Schemeでは式の評価はそれに加えて探索が行き止まりに帰結します。
その場合には評価機は依然の選択点へバックトラックしなければなりません。
非決定性Schemeの解釈はこの特別な場合により複雑になります。


私達は非決定性Schemeのための\code{amb}評価機を、\link{Section 4.1.7}の分析評価機を変更することで
構築します。\footnote{私達は\link{Section 4.2}の遅延評価機を\link{Section 4.1.1}の通常のメタ循環評価機に
対する変更として実装することを選択しました。対照的に、\link{Section 4.1.7}の分析評価機を
\code{amb}評価機の基にします。その評価機内の実行手続がバックトラックを実装するのに
便利なフレームワークを提供するためです。}分析評価機のように、式の評価は式の分析により
生成される実行手続を呼ぶことにより達成されます。通常のSchemeの解釈と非決定性Schemeの
解釈との違いは完全に実行手続の中に存在します。

\subsubsection*{手続と継続の実行}

通常の評価機の実行手続が1つの引数、実行の環境を取ることを思い出して下さい。
対照的に、\code{amb}評価機の実行手続は3つの引数を取ります。環境と
\newterm{continuation procedures}(\jnewterm{継続手続})と呼ばれる2つの手続です。
式の評価はこれらの2つの継続の1つを呼ぶことで完了します。もし評価の結果が
値に帰結するならば、\newterm{success continuation}(\jnewterm{成功継続})がその値と共に
呼ばれます。もし評価が行き止まりの発見に帰結したのであれば、
\newterm{failure continuation}(\jnewterm{失敗継続})が呼ばれます。適切な継続の構築と呼出が
非決定性評価機のバックトラックが実装される仕組みです。

値を受け取り計算を続行することが成功継続の仕事です。その値と共に、成功継続は別の
失敗継続も渡されます。これはその後にもしその値の使用が行き止まりに導いたなら
呼び出されます。


非決定性処理の他の分岐を試すのは失敗継続の仕事です。非決定性言語の本質は
式が選択肢の間の選択を表現するだろうという事実の中に存在します。
そのような式の評価は例え前もってどの選択肢受け入れ可能な結果に導くか知らなくても
指示された代替となる選択の一つを用いて続行しなければなりません。これを処理するため
には、評価機は選択肢から1つを取り出しこの値を成功継続に渡します。この値と共に、
評価機は後で異なる選択肢を選択するために呼び出し可能な失敗継続を構築し、一緒に渡します。


失敗は評価の間に引き起こされます。(言い換えれば失敗継続が呼ばれます)。それはユーザプログラムが
明示的に現在の一連の取り組みを拒絶した場合に起こります。(例えば、\code{require}の呼出は結果として
\code{(amb)}が実行される場合があります。これは常に失敗する式です。---\link{Section 4.3.1}参照)。
その時点で手中にある失敗継続が最も最近の選択点に他の選択肢を選択させます。
もしもうその選択点にて考えられる他の選択肢が無い場合には、直前の選択点の失敗が引き起こされます。
以下その繰り返しです。失敗継続はまた式の別の値を見つけるためのドライバループによる\code{try\-/again}要求への
応答としても起動されます。


加えて、もし副作用命令(変数への代入等)がある選択の結果としての分岐処理上で起こったならば、
処理が行き止まりを見つけた時に、新しい選択を行う前にその副作用を取り消しする必要が
あるかもしれません。これは副作用命令に副作用を取り消し失敗を伝播させる失敗継続を
生成させることで達成されます。


まとめとして、失敗継続は以下により構築されます。

\begin{itemize}

\item
\code{amb}式---\code{amb}式により行われた現在の選択が行き止まりに導いた場合に別の選択を行う
仕組みを提供します

\item
トップレベルドライバ---選択肢が枯渇した時に失敗を報告する仕組みを提供します

\item
代入---失敗に割り込み、バックトラックの間に代入を取り消します

\end{itemize}

\noindent
失敗は行き止まりに遭遇したその時のみ起動されます。これは以下の場合に起こります。

\begin{itemize}

\item
ユーザプログラムが\code{(amb)}を実行した

\item
ユーザがトップレベルドライバにて\code{try\-/again}を入力した

\end{itemize}

\noindent
失敗継続はまた失敗処理の間にも呼ばれます。

\begin{itemize}

\item
代入が副作用の取消を完了させることで失敗継続が作成された時に、それは割り込んだ
失敗継続を、失敗を伝播させこの代入に導いた選択点に戻すために、またはトップレベルに戻すために
呼びます。

\item
\code{amb}に対する失敗継続が選択肢を使い切った時、、
失敗を直前の選択点かトップレベルに伝播させるために、\code{amb}に対して元々与えられた失敗継続を呼び出します。

\end{itemize}

\subsubsection*{評価機の構造}


\code{amb}評価機に対する文法とデータの表現手続、また基本的な\code{analyze}手続は、
\link{Section 4.1.7}の評価機のそれらに等しい物です。しかし私達が追加の構文手続を\code{amb}の
特殊形式を認識するために必要とすることが異なります。\footnote{評価機は\code{let}を
サポートすると想定しています。(\link{Exercise 4.22}参照)。私達はこれを非決定性
プログラム内にて利用してきました。}

\begin{scheme}
(define (amb? exp)
  (tagged-list? exp 'amb))
(define (amb-choices exp) (cdr exp))
\end{scheme}

\noindent
\code{analyze}内にこの特殊形式を認識し、適切な実行手続を生成する呼出を追加しなければなりません。

\begin{scheme}
((amb? exp) (analyze-amb exp))
\end{scheme}

\noindent
トップレベル手続\code{ambeval}(\link{Section 4.1.7}で与えられた\code{eval}の版に似た物)は
与えられた式を分析し、実行手続を与えられた環境に対し、2つの与えられた継続と一緒に
適用しなければなりません。

\begin{scheme}
(define (ambeval exp env succeed fail)
  ((analyze exp) env succeed fail))
\end{scheme}

\noindent
成功継続は2引数の手続です。2引数は得られたばかりの値とその値がその後に失敗へと導いたなら
使用される別の失敗継続です。失敗継続は引数無しの手続です。そのため実行手続の
一般的な形は以下のようになります。

\begin{scheme}
(lambda (env succeed fail)
  ~\textrm{;; \code{succeed} is \code{(lambda (value fail) \( \dots \))}}~
  ~\textrm{;; \code{fail} is \code{(lambda () \( \dots \))}}~
  ~\( \dots \)~)
\end{scheme}

\noindent
例えば、以下を実行すると、

\begin{scheme}
(ambeval ~\( \dark \langle \)~~\var{\dark exp}~~\( \dark \rangle \)~
         the-global-environment
         (lambda (value fail) value)
         (lambda () 'failed))
\end{scheme}

\noindent
与えられた式を評価し、式の値(評価が成功した場合)かシンボル\code{failed}(評価が失敗
した場合)を返します。以下で示されるドライバループ内での\code{ambeval}の呼出はより
多くの複雑な継続手続を使用します。これらはループを継続し\code{try\-/again}要求を
サポートします。

\code{amb}の複雑さの多くは実行手続がお互いを呼ぶに従い、継続をたらい回しにすることから来ています。
以下のコードを通して読むに当って、それぞれの実行手続を\link{Section 4.1.7}で与えられた通常の評価機のための
対応する手続と比べて下さい。

\subsubsection*{単純な式}

最も単純な種類の式に対する実行手続は本質的に通常の評価機に対するものと同じです。
ただし、継続を管理する必要があることが異なります。これらの実行手続は式の値と共に単純に
成功し、渡された失敗継続をそのまま手渡します。

\begin{scheme}
(define (analyze-self-evaluating exp)
  (lambda (env succeed fail)
    (succeed exp fail)))
(define (analyze-quoted exp)
  (let ((qval (text-of-quotation exp)))
    (lambda (env succeed fail)
      (succeed qval fail))))
(define (analyze-variable exp)
  (lambda (env succeed fail)
    (succeed (lookup-variable-value exp env)
             fail)))
(define (analyze-lambda exp)
  (let ((vars (lambda-parameters exp))
        (bproc (analyze-sequence (lambda-body exp))))
    (lambda (env succeed fail)
      (succeed (make-procedure vars bproc env)
               fail))))
\end{scheme}

\noindent
変数の検索が常に`成功'することに注意して下さい。もし\code{lookup\-/variable\-/value}が
変数を見つけるのに失敗した場合、それはいつも通りにエラーを発します。
そのような``失敗''はプログラムのバグ---未束縛な変数への参照---を示します。
これは現在試行中の物の代わりに別の非決定性選択を試すことを示してはいません。

\subsubsection*{条件文と列}


条件文もまた通常の評価機と同様に取り扱われます。\code{analyze\-/if}により生成される
実行手続は述語実行手続\code{pproc}を成功継続と共に起動します。成功継続は
述語の値が真であるかチェックし、結果部(consequent)か代替部(alternative)を実行します。
もし\code{pproc}の実行が失敗したなら\code{if}式に対する元の失敗継続が呼ばれます。

\begin{scheme}
(define (analyze-if exp)
  (let ((pproc (analyze (if-predicate exp)))
        (cproc (analyze (if-consequent exp)))
        (aproc (analyze (if-alternative exp))))
    (lambda (env succeed fail)
      (pproc env
             ~\textrm{;; 述語を評価するための成功継続を置く}~
             ~\textrm{;; \code{pred\-/value}を得るため}~
             (lambda (pred-value fail2)
               (if (true? pred-value)
                   (cproc env succeed fail2)
                   (aproc env succeed fail2)))
             ~\textrm{;; 述語を評価するための失敗継続}~
             fail))))
\end{scheme}

\noindent
列もまた以前の評価機と同様に取り扱われます。ただし継続を渡すために必要とされる
内部手続\code{sequentially}内の企みが異なります。具体的には\code{a}を実行し、次に\code{b}と順に
行うために、\code{a}を成功継続と共に呼び、成功継続が\code{b}を呼びます。

\begin{scheme}
(define (analyze-sequence exps)
  (define (sequentially a b)
    (lambda (env succeed fail)
      (a env
         ~\textrm{;; \code{a}を呼ぶための成功継続}~
         (lambda (a-value fail2)
           (b env succeed fail2))
         ~\textrm{;; \code{a}を呼ぶための失敗継続}~
         fail)))
  (define (loop first-proc rest-procs)
    (if (null? rest-procs)
        first-proc
        (loop (sequentially first-proc
                            (car rest-procs))
              (cdr rest-procs))))
  (let ((procs (map analyze exps)))
    (if (null? procs)
        (error "Empty sequence: ANALYZE"))
    (loop (car procs) (cdr procs))))
\end{scheme}

\subsubsection*{定義と代入}

定義は継続を管理するために手間をかけなければいけない一例です。定義の値(definition-value)の式を
実際に新しい値を定義する前に評価する必要があるためです。これを達成するためには
定義値実行手続\code{vproc}が環境、成功継続、失敗継続と共に呼ばれます。
もし\code{vproc}の実行が成功したなら定義値のための値\code{val}を取得し、変数が定義され
成功が伝播されます。

\begin{scheme}
(define (analyze-definition exp)
  (let ((var (definition-variable exp))
        (vproc (analyze (definition-value exp))))
    (lambda (env succeed fail)
      (vproc env
             (lambda (val fail2)
               (define-variable! var val env)
               (succeed 'ok fail2))
             fail))))
\end{scheme}

\noindent
代入はもっと面白いです。これは継続をたらい回しにするのではなく、本当に継続を使用する最初の場所です。
代入のための実行手続は定義のためのものと同様に開始します。最初に変数に代入される新しい値を取得しようと
試みます。もしこの\code{vproc}の評価が失敗したら代入は失敗します。


しかし\code{vproc}が成功し代入を行なおうとした場合には、この計算の分岐が後に失敗する
可能性について考えねばなりません。この場合には代入から外れてバックトラックする必要があります。
従ってこの代入をバックトラック処理の一部分として取消する準備をしなければなりません。\footnote{定義の
取消については心配しません。内部定義は走査されたことが想定可能なためです。(\link{Section 4.1.6})}


これは\code{vproc}に(下でコメント``*1*''が記された)成功継続を与えることで達成されます。
この成功継続は新しい値を代入し、結果として存在する前に、変数の古い値を保存します。
代入値と共に渡された(下でコメント``*2*''が記された)失敗継続は失敗を続ける前に変数の古い値を再格納します。
つまり、代入の成功は後の失敗に割り込む失敗継続を提供します。そうでなければ\code{fail2}を呼んでいたはずのどんな失敗も
この手続を代わりに呼ぶことで、実際に\code{fail2}を呼ぶ前に代入を取り消します。

\begin{scheme}
(define (analyze-assignment exp)
  (let ((var (assignment-variable exp))
        (vproc (analyze (assignment-value exp))))

    (lambda (env succeed fail)
      (vproc env
             (lambda (val fail2)        ~\textrm{; *1*}~
               (let ((old-value
                      (lookup-variable-value var env)))
                 (set-variable-value! var val env)
                 (succeed 'ok
                          (lambda ()    ~\textrm{; *2*}~
                            (set-variable-value!
                             var old-value env)
                            (fail2)))))
             fail))))
\end{scheme}

\subsubsection*{手続の適用}


適用のための実行手続は新しいアイデアを含んではいません。ただし継続を管理する
技術的な複雑さが異なります。この複雑さは\code{analyze\-/application}の中で
オペランドを評価するに従い成功と失敗の継続を追跡する必要があるために、浮上します。
私達は通常の評価機の中の様に単純に\code{map}を使うのでなく、手続\code{get\-/args}を用いてオペランドの
リストを評価しています。

\begin{scheme}
(define (analyze-application exp)
  (let ((fproc (analyze (operator exp)))
        (aprocs (map analyze (operands exp))))
    (lambda (env succeed fail)
      (fproc env
             (lambda (proc fail2)
               (get-args aprocs
                         env
                         (lambda (args fail3)
                           (execute-application
                            proc args succeed fail3))
                         fail2))
             fail))))
\end{scheme}

\noindent
\code{get\-/args}の中では、どのようにして\code{aproc}実行手続のリストを\code{cdr}で下り、そして
結果の\code{args}のリストを全て\code{cons}するかについて注意して下さい。
これはリスト中の全ての\code{aproc}を、
再帰的に\code{get\-/args}を呼ぶ成功継続と共に呼ぶことにより達成されます。
これらの\code{get\-/args}に対する全ての再帰的呼出は、蓄積された引数のリストの上に
新しく取得された引数を\code{cons}した値を返す成功継続を持っています。

\begin{scheme}
(define (get-args aprocs env succeed fail)
  (if (null? aprocs)
      (succeed '() fail)
      ((car aprocs)
       env
       ;;~\textrm{この\code{aproc}のための成功継続}~
       (lambda (arg fail2)
         (get-args
          (cdr aprocs)
          env
          ;;~\textrm{\code{get\-/args}の再帰呼出のための}~
          ;;~\textrm{成功継続}~
          (lambda (args fail3)
            (succeed (cons arg args) fail3))
          fail2))
       fail)))
\end{scheme}

\noindent
\code{exe@\-/cute\-/application}により実行される実際の手続適用は通常の評価機に対する物と
同じ方法にて達成されます。ただし継続の管理の必要が異なります。

\begin{scheme}
(define (execute-application proc args succeed fail)
  (cond ((primitive-procedure? proc)
         (succeed (apply-primitive-procedure proc args)
                  fail))
        ((compound-procedure? proc)
         ((procedure-body proc)
          (extend-environment
           (procedure-parameters proc)
           args
           (procedure-environment proc))
          succeed
          fail))
        (else (error "Unknown procedure type:
                      EXECUTE-APPLICATION"
                     proc))))
\end{scheme}

\subsubsection*{\code{amb}式の評価}

\code{amb}の特殊形式は非決定性言語の鍵となる要素です。ここでは逐次翻訳処理の本質と
継続を追跡する理由について学びます。\code{amb}に対する実行手続はループ\code{try\-/next}を
定義します。これは全ての\code{amb}式の可能な値のために、全実行手続を通して実行します。
各実行手続は次の実行手続を試す失敗継続と共に呼ばれます。試行する選択肢が無くなった
時には、\code{amb}式全体が失敗します。

\begin{scheme}
(define (analyze-amb exp)
  (let ((cprocs (map analyze (amb-choices exp))))
    (lambda (env succeed fail)
      (define (try-next choices)
        (if (null? choices)
            (fail)
            ((car choices)
             env
             succeed
             (lambda () (try-next (cdr choices))))))
      (try-next cprocs))))
\end{scheme}

\subsubsection*{ドライバループ}

\code{amb}評価機のドライバループは複雑です。ユーザに式の評価を再試行(try-again)
することを可能にする仕組みのためです。ドライバは\code{internal\-/loop}と呼ばれる
手続を使用します。これは引数として手続\code{try\-/again}を取ります。これの意図は
\code{try\-/again}の呼出は次のまだ試行されていない非決定性評価における選択肢へ
続けなければならないことです。\code{internal\-/loop}はユーザのドライバループでの
\code{try\-/again}の入力への応答として\code{try\-/again}を呼ぶか、またはそうでなければ
\code{ambeval}を呼ぶことにより新しい評価を開始します。

\code{ambeval}へのこの呼出のための失敗継続はユーザにもう値は残っていないと伝え、
ドライバループを再起動する。

\code{ambeval}への呼出のための成功継続はより微妙です。獲得した値を表示し、次に内部ループを
再び起動します。起動には次の選択肢を試行可能な\code{try\-/again}手続を伴ないます。
この\code{next\-/alternative}手続は成功継続に二番目の引数として渡されます。通常はこの二番目の
引数はもし現在の評価分岐が後に失敗場合に利用される失敗継続として考えます。
今回の場合はしかし、評価を成功裏に完了しました。そのため``失敗''の代替分岐を追加の成功する
評価を探すために起動することができます。

\begin{scheme}
(define input-prompt  ";;; Amb-Eval input:")
(define output-prompt ";;; Amb-Eval value:")
(define (driver-loop)
  (define (internal-loop try-again)
    (prompt-for-input input-prompt)
    (let ((input (read)))
      (if (eq? input 'try-again)
          (try-again)
          (begin
            (newline)
            (display ";;; Starting a new problem ")
            (ambeval
             input
             the-global-environment
             ;; ~\textrm{\code{ambeval} success}~
             (lambda (val next-alternative)
               (announce-output output-prompt)
               (user-print val)
               (internal-loop next-alternative))
             ;; ~\textrm{\code{ambeval} failure}~
             (lambda ()
               (announce-output
                ";;; There are no more values of")
               (user-print input)
               (driver-loop)))))))
  (internal-loop
   (lambda ()
     (newline)
     (display ";;; There is no current problem")
     (driver-loop))))
\end{scheme}

\noindent
\code{internal\-/loop}の最初の呼出では現在、問題が無いと不服を述べ、ドライバループを再開します。
これはユーザが\code{try\-/again}を入力し評価に進展が無い場合に起こる振舞です。

\begin{quote}
\heading{\phantomsection\label{Exercise 4.50}Exercise 4.50:}
\code{amb}と似ているが、次の選択肢を左から右へでなくランダムな順で探索する新しい特殊形式\code{ramb}を実装せよ。
\link{Exercise 4.49}におけるAlyssaの問題をどのように助けるか示せ。
\end{quote}

\begin{quote}
\heading{\phantomsection\label{Exercise 4.51}Exercise 4.51:}
失敗時に取り消されない\code{permanent\-/set!}と呼ばれる新しい種類の代入を実装せよ。
例えば、以下の様に2つの区別可能な要素をリストから選択し、成功した選択に必要とした試行の数を
カウントする。

\begin{scheme}
(define count 0)
(let ((x (an-element-of '(a b c)))
      (y (an-element-of '(a b c))))
  (permanent-set! count (+ count 1))
  (require (not (eq? x y)))
  (list x y count))
~\textit{;;; Starting a new problem}~
~\textit{;;; Amb-Eval value:}~
~\textit{(a b 2)}~
~\textit{;;; Amb-Eval input:}~
try-again
~\textit{;;; Amb-Eval value:}~
~\textit{(a c 3)}~
\end{scheme}


ここで\code{permanent\-/set!}の代わりに\code{set!}を使ったらどんな値が表示されるだろうか?
\end{quote}

\begin{quote}
\heading{\phantomsection\label{Exercise 4.52}Exercise 4.52:}
ユーザに式の失敗を捕獲させることを可能にする新しいコンストラクタ\code{if\-/fail}を実装せよ。
\code{if\-/fail}は2つの式を取る。最初の式を通常通りに評価し評価が成功したら普通に戻る。
しかし、もし評価が失敗したら2つ目の式の値が以下の例の様に返される。

\begin{scheme}
~\textit{;;; Amb-Eval input:}~
(if-fail (let ((x (an-element-of '(1 3 5))))
           (require (even? x))
           x)
         'all-odd)
~\textit{;;; Starting a new problem}~
~\textit{;;; Amb-Eval value:}~
~\textit{all-odd}~

~\textit{;;; Amb-Eval input:}~
(if-fail (let ((x (an-element-of '(1 3 5 8))))
           (require (even? x))
           x)
         'all-odd)
~\textit{;;; Starting a new problem}~
~\textit{;;; Amb-Eval value:}~
~\textit{8}~
\end{scheme}
\end{quote}

\begin{quote}
\heading{\phantomsection\label{Exercise 4.53}Exercise 4.53:}
\link{Exercise 4.51}で説明した\code{permanent\-/set!}と\link{Exercise 4.52}の
\code{if\-/fail}を用いて以下を評価した時どのような結果になるか。

\begin{scheme}
(let ((pairs '()))
  (if-fail 
   (let ((p (prime-sum-pair '(1 3 5 8)
                            '(20 35 110))))
     (permanent-set! pairs (cons p pairs))
     (amb))
   pairs))
\end{scheme}
\end{quote}

\begin{quote}
\heading{\phantomsection\label{Exercise 4.54}Exercise 4.54:}
もし\code{require}が\code{amb}を用いた通常の手続として実装できることに気付かなかった場合、
ユーザにより非決定性プログラムの一部として定義されるために、それを特殊形式として
実装する必要があったはずである。これは以下の構文手続を必要としたであろう。

\begin{scheme}
(define (require? exp)
  (tagged-list? exp 'require))
(define (require-predicate exp) 
  (cadr exp))
\end{scheme}

\noindent
そして\code{analyze}内の呼出に新しい節が必要となった。

\begin{scheme}
((require? exp) (analyze-require exp))
\end{scheme}

\noindent
また\code{require}式を取り扱う手続\code{analyze\-/require}も必要となった。
以下の\code{analyze\-/require}の定義を完成させよ。

\begin{scheme}
(define (analyze-require exp)
  (let ((pproc (analyze (require-predicate exp))))
    (lambda (env succeed fail)
      (pproc env
             (lambda (pred-value fail2)
               (if ~\( \dark \langle \)~~\var{\dark ??}~~\( \dark \rangle \)~
                   ~\( \dark \langle \)~~\var{\dark ??}~~\( \dark \rangle \)~
                   (succeed 'ok fail2)))
             fail))))

\end{scheme}
\end{quote}


\section{論理プログラミング}
\label{Section 4.4}

\link{Chapter 1}では計算機科学は命令型(どうするか)の知識を扱い、一方、数学は宣言型(何であるか)の
知識を扱うと強調しました。実際に、プログラミング言語はプログラマが特定の問題を解くために、
段階的な手法を示す形式により、知識を表現することを要求します。
一方、高級言語は言語の実装の一部としてユーザを、指定された演算がどのように進められるかについての詳細から
解放する、十分な量の方法論的知識を提供します。

Lispを含む多くのプログラミング言語は数学上の関数の値の演算の周りに体系化されています。
式指向の言語(例えばLisp、Fortran、Algol)は関数の値を記述する式がその値を求める
手段としても解釈されるという``多義性''を十分に活用しています。このため、多くの
プログラミング言語は単向性演算(明確な入力と出力を持つ演算)に向けて強く偏っています。
しかし、この偏りを緩和する完全に異なるプログラミング言語も存在します。そのような言語の
例を\link{Section 3.3.5}で見ました。そこでは計算オブジェクトは数値的な制約でした。制約システムでは
演算の向きと順はあまり明らかには指定されません。従って、演算の実行においてシステムはより
詳細な``行い方''の知識を、通常の数値演算による場合よりも多く提供しなければなりません。
しかし、これはユーザが命令型の知識を提供する責任から完全に解放されることは意味しません。
同じ制約の集合を実装する制約ネットワークは数多く存在し、ユーザは数学的に等価なネットワークの中から
特定の演算を指定するのに適切なネットワークを選択せねばなりません。

\link{Section 4.3}の非決定性プログラム評価機もまたプログラミングとは一方向性関数のための
アルゴリズムを構築することであるという視点から離れています。非決定性言語においては、
式は2つ以上の値を持つことができ、結果として演算とは単一の値の関数ではなく関係性を
取り扱う物になります。論理プログラミングはプログラミングの関係性の視点と
\newterm{unification}(\jnewterm{ユニフィケーション}、単一化)と呼ばれる強力な種類の記号パターン
マッチングとを組み合わせることでこの考えを拡張します。\footnote{論理プログラミングは
自動定理証明の研究の長い歴史から成長しました。早期の定理証明プログラムはあまり目的を
達成することができませんでした。可能な証明空間を網羅的に探索するためです。受け入れ可能な
探索を行える打開策の主な物は1960年代早期の\newterm{unification algorithm}(\jnewterm{ユニフィケーションアルゴリズム})と
\newterm{resolution principle}(\jnewterm{導出原理})(\link{Robinson 1965})の発見でした。
例えば導出は\link{Green and Raphael (1968)}により(\link{Green 1969}も参照)
演繹的質問応答システムの基盤として使用されました。この時期の多くの間、研究者達は
証明が存在するのであれば見つけられることが保証されたアルゴリズムについて集中していました。
そのようなアルゴリズムはコントロールして証明に向かわせることが難しいものでした。
\link{Hewitt (1969)}はプログラミング言語のコントロール構造と論理操作システムとの
結合の可能性を認識し、\link{Section 4.3.1}(\link{Footnote 4.47})で述べられた自動探索の成果へと
導きました。同時期に、マルセイユのColmerauerは自然言語を扱うルールベースシステム
(\link{Colmerauer et al. 1973})により同じ事を達成しました。彼はPrologと呼ばれる
プログラミング言語を開発しそれらのルールを表現しました。\link{Kowalski (1973; 1979)}は
エディンバラにて、Prologプログラムの実行は(線形ホーン節導出と呼ばれる
証明のテクニックを用いて)定理証明として解釈できることを認めました。
最後の2つの糸を縒り合わせることが論理プログラミング運動へと導きました。従って
論理プログラミングの開発に対して功績を与えることにおいて、フランス人はマルセイユ大学
でのPrologの起源を指摘することができ、一方、イギリス人はエディンバラ大学の成果を
強調することができます。\acronym{MIT}の人々に言わせれば、論理プログラミングはこれらの
グループにより、Hewittがその才能ある、しかし頑迷な博士論文にて何を伝えていたかを
解き明かす試みにより開発されました。論理プログラミングの歴史については
\link{Robinson 1983}を参照して下さい。}

この取り組み方は、うまく行く場合には、プログラムを書くのにとても強力な方法となります。
その力の一部は単一の``何であるか''という事実が、異なる``行い方''の要素を持つかもしれない
いくつかの異なる問題を解決するのに使用できるという事実から来ています。
例として、\code{append}命令について考えましょう。これは2つのリストを引数として取り、それらの
要素を結合して単一のリストを形成します。Lispのような手続型言語では\code{append}を
基本的なリストコンストラクタ\code{cons}を用いて、\link{Section 2.2.1}で行ったように定義することができました。

\begin{scheme}

(define (append x y)
  (if (null? x) y (cons (car x) (append (cdr x) y))))
\end{scheme}

\noindent
この手続は以下の2つのルールによりLispへと翻訳したと捕えることができます。
最初のルールは1つ目のリストが空である場合を扱い、2つ目のルールは空でないリスト、
つまり2つの部分による\code{cons}の場合を扱います。

\begin{itemize}

\item
任意のリスト\code{y}に対し、空リストと\code{y}の\code{append}は\code{y}を形成する。

\item
任意の\code{u}, \code{v}, \code{y}, \code{z}に対し、\code{(cons u v)}と\code{y}の\code{append}は
もし\code{v}と\code{y}の\code{append}が\code{z}を形成するならば\code{(cons u z)}を形成する。\footnote{ルールと
手続の間の対応を見るためには、手続における\code{x}(\code{x}が空でない場合)を
ルールの\code{(cons u v)}に対応させます。次にルールの\code{z}は\code{(cdr x)}と\code{y}の
\code{append}に対応します。}

\end{itemize}

\noindent
\code{append}手続を用いることで、私達は次のような質問に答えることができます。

\begin{quote}
\code{(a b)}と\code{(c d)}の\code{append}を求めよ。
\end{quote}

\noindent
しかし同じ2つのルールがまた以下のような種類の質問に答えるためにも十分です。
これらは手続では答えられません。

\begin{quote}
\code{(a b)}と\code{append}すると\code{(a b c d)}を生成するリスト\code{y}を求めよ。


\code{append}すると\code{(a b c d)}を生成する全ての\code{x}と\code{y}を求めよ。
\end{quote}

\noindent
論理プログラミング言語ではプログラマは\code{append}``手続''を上で与えられた\code{append}に関する
2つのルールを提示することにより記述します。``行い方''の知識は自動的にインタプリタにより
提供されこの単一ペアのルールが3つ全てのタイプの\code{append}に関する質問に対して答えることを
可能にします。\footnote{これは確かにユーザをどのように回答を求めるかという問題全体からは
解放しません。\code{append}の関係を形式化するための数学的に等価なルールは数多く存在します。
それらのいくつかのみが任意の方向の演算に対する効果的な手段と成り得ます。
付け加えて、時々、``何であるか''という情報は``どのように''回答を求めるかについて
何の手掛かりも与えない場合があります。例えば\( y^2 = x \)となる\( y \)を求める
問題について考えてみて下さい。}


現代の論理プログラミング言語(ここで私達が実装しているものを含めて)にはかなりの量の
不足がそれらの一般的な``行い方''の手法について存在します。このことが偽の無限ループを
引き起したり、他の望ましくない振舞へと導いてしまします。論理プログラミングは計算機
科学において活発な研究領域です。\footnote{論理プログラミングへの興味は80年代早期に
日本政府が論理プログラミング言語を実行するのに最適化されたとても速い計算機を構築する
ことを狙った大望あるプロジェクトを開始した時にピークを迎えました。そのような計算機の
スピードは通常のFLOPS(FLoating-point Operations Per Second)でなく
LIPS(Logical Inferences Per Second)で計られます。プロジェクトはハードウェアとソフトウェアの
開発において元々の計画通りに成功しましたが、国際的なコンピュータ業界は異なる方向へと
向かいました。日本のプロジェクトの評価の概観については\link{Feigenbaum and Shrobe 1993}を
参照して下さい。論理プログラミングコミュニティもまた、\link{Section 3.3.5}の制約伝播システムで
説明されたような数値値上の制約を取り扱う能力の様な単純なパターンマッチングではない
技術を基盤にしたリレーショナルプログラミングへと移行しました。}


この章の最初では私達はインタプリタの実装技術を探求しLispの様な言語のための
インタプリタに対して(実際に、任意の従来の言語に対して)本質である要素を説明しました。
今から私達はこれらの考えを応用し論理プログラミング言語のためのインタプリタについて議論します。
この言語を\newterm{query language}(\jnewterm{クエリ言語})と呼ぶことにします。言語で内で表現される、
\newterm{queries}(\jnewterm{クエリ})、つまり質問を定式化することによりデータベースから情報を
取得することに対してとても便利なためです。
クエリ言語はLispと全く違うであるにも係らず、私達がここまで利用してきた同じ一般的なフレームワークを
用いてこの言語を説明することがとても都合が良いことを理解するでしょう。
このフレームワークはプリミティブな要素の集合として、簡単な要素を組み合わせることでより複雑な要素を作ることを
可能にする組み合わせの手段と、複雑な要素を単純な概念の単位として見做すことを可能にする
抽象化の手段とを一緒に用いました。
論理プログラミング言語向けインタプリタはLispのような言語のインタプリタよりも大幅に複雑です。
それでも、私達のクエリ言語インタプリタが\link{Section 4.1}のインタプリタ内にて見つけた多くの同じ
要素を含むことを学ぶでしょう。
具体的には式を方に従って分類する``eval''のパートが存在し、そして言語の抽象化の仕組み(Lispの
場合では手続であり、論理プログラミングの場合ではルール)を実装する``apply''のパートが存在します。
また、中心的な役割はフレームデータ構造により実装の中で演じられます。このフレームデータ構造は
シンボルとそれらに関連する値の間の対応を決定します。クエリ言語の実装の追加の面白い側面の1つは、
\link{Chapter 3}で紹介したストリームを大量に使用することです。



\subsection{演繹的情報検索}
\label{Section 4.4.1}

論理プログラミングは情報取得のためのデータベースに対するインターフェイスの提供に
おいて秀でています。私達がこの章で実装するクエリ言語はこのように使用されるよう
設計されています。


クエリシステムが何を行うかを説明するために、ボストン区域に存在する成長中の
ハイテク企業、Microshaftの社員情報のデータベースを管理するために、クエリシステムが
どのように利用できるかについて示します。この言語はパターンにより示される社員情報への
アクセスを提供し、また論理的演繹法を行うための一般的なルールの利点をも得ることができます。

\subsubsection*{サンプルデータベース}

Microshaftの社員情報データベースは会社の全社員に関する\newterm{assertions}(\jnewterm{アサーション}、表明)を
保持します。以下に常駐のコンピュータウィザード、Ben Bitdiddleに関する情報を挙げます。

\begin{scheme}
(address (Bitdiddle Ben) (Slumerville (Ridge Road) 10))
(job (Bitdiddle Ben) (computer wizard))
(salary (Bitdiddle Ben) 60000)
\end{scheme}

\noindent
各アサーションはリスト(この場合3つ組)で、その要素はそれ自体がリストに成り得ます。


常駐のウィザードとして、Benは会社のコンピュータ部門を管理し、二人のプログラマと一人の
技術者を監督します。以下に部下に関する情報を挙げます。

\begin{scheme}
(address (Hacker Alyssa P) (Cambridge (Mass Ave) 78))
(job (Hacker Alyssa P) (computer programmer))
(salary (Hacker Alyssa P) 40000)
(supervisor (Hacker Alyssa P) (Bitdiddle Ben))

(address (Fect Cy D) (Cambridge (Ames Street) 3))
(job (Fect Cy D) (computer programmer))
(salary (Fect Cy D) 35000)
(supervisor (Fect Cy D) (Bitdiddle Ben))

(address (Tweakit Lem E) (Boston (Bay State Road) 22))
(job (Tweakit Lem E) (computer technician))
(salary (Tweakit Lem E) 25000)
(supervisor (Tweakit Lem E) (Bitdiddle Ben))
\end{scheme}

\noindent
Alyssaに監督されているプログラマ見習いもいます。

\begin{scheme}
(address (Reasoner Louis) (Slumerville (Pine Tree Road) 80))
(job (Reasoner Louis) (computer programmer trainee))
(salary (Reasoner Louis) 30000)
(supervisor (Reasoner Louis) (Hacker Alyssa P))
\end{scheme}

\noindent
これらの人々全てはコンピュータ部門に属し、彼等の職位(job)記述の最初の項目である
単語\code{computer}により示されています。


Benは高位の従業員です。彼の監督者は会社の有力者である彼自身です。

\begin{scheme}
(supervisor (Bitdiddle Ben) (Warbucks Oliver))
(address (Warbucks Oliver) (Swellesley (Top Heap Road)))
(job (Warbucks Oliver) (administration big wheel))
(salary (Warbucks Oliver) 150000)
\end{scheme}

\noindent
コンピュータ部門がBenに監督されているのに加えて、会社には会計士長とそのアシスタントから
成る経理部門があります。

\begin{scheme}
(address (Scrooge Eben) (Weston (Shady Lane) 10))
(job (Scrooge Eben) (accounting chief accountant))
(salary (Scrooge Eben) 75000)
(supervisor (Scrooge Eben) (Warbucks Oliver))

(address (Cratchet Robert) (Allston (N Harvard Street) 16))
(job (Cratchet Robert) (accounting scrivener))
(salary (Cratchet Robert) 18000)
(supervisor (Cratchet Robert) (Scrooge Eben))
\end{scheme}

\noindent
また重役のための秘書もいます。

\begin{scheme}
(address (Aull DeWitt) (Slumerville (Onion Square) 5))
(job (Aull DeWitt) (administration secretary))
(salary (Aull DeWitt) 25000)
(supervisor (Aull DeWitt) (Warbucks Oliver))
\end{scheme}

\noindent
データベースはまたどの職種が他の職種を持つ人々により行われることができるかに関する
アサーションも含みます。例えばコンピュータウィザードはコンピュータプログラマと
コンピュータ技術者の両方の職を行うことができます。

\begin{scheme}
(can-do-job (computer wizard) (computer programmer))
(can-do-job (computer wizard) (computer technician))
\end{scheme}

\noindent
コンピュータプログラマは見習いを埋めることができるでしょう。

\begin{scheme}
(can-do-job (computer programmer)
            (computer programmer trainee))
\end{scheme}

\noindent
また良く知られているように以下も言えます。

\begin{scheme}
(can-do-job (administration secretary)
            (administration big wheel))
\end{scheme}

\subsubsection*{単純なクエリ}


クエリ言語はユーザにシステムプロンプトに対する応答として
クエリを提示させることで、データベースから情報を取得することを許します。

\begin{scheme}
~\textit{;;; Query input:}~
(job ?x (computer programmer))
\end{scheme}

\noindent
システムは以下の項目を返します。

\begin{scheme}
~\textit{;;; Query results:}~
(job (Hacker Alyssa P) (computer programmer))
(job (Fect Cy D) (computer programmer))
\end{scheme}

\noindent
入力クエリはある種のパターンにマッチするデータベース内のエントリを探すことを指示します。
この例では、パターンは3つの項目から成るエントリを指定しています。最初が文字シンボルの
\code{job}、2つ目は任意の値に成り得て、3番目は文字のリスト\code{(computer programmer)}です。
マッチングリスト内の2つ目の項目に成り得る``任意項''は\newterm{pattern variable}(\jnewterm{パターン変数})\code{?x}で
指定されます。パターン変数の一般的な形式はクエスチョンマークを前に置いた、変数の名前と
して取られるシンボルです。以下では、なぜこのことが単に\code{?}を``任意''を表すパターンに
置くのではなく、パターン変数のために名前を指定することが便利であるかを学びます。
システムは簡単なクエリに指定されたパターンにマッチするデータベース内の全てのエントリを
表示することで応答します。

パターンは複数の変数を持つことができます。例えば、以下のクエリ

\begin{scheme}
(address ?x ?y)
\end{scheme}

\noindent
は全ての従業員の住所を並べます。

パターンはクエリが単純にパターンがデータベース内のエントリであるかどうかを決定する
場合には変数を持つことができません。もしそうであれば1つの一致が存在します。
そうでなければ1つも一致は存在しません。

同じパターン変数が1つのクエリ内に複数存在することができ、同じ``任意項''が各位置に
現われなければいけないこと指定します。これがなぜ変数が名前を持つのかの理由です。
例えば、

\begin{scheme}
(supervisor ?x ?x)
\end{scheme}

\noindent
上のクエリは自分自身を監督する全ての人々を見つけます。(しかし私達のサンプル
データベース内のアサーションにはそのようなエントリがありません。)

以下のクエリは、

\begin{scheme}
(job ?x (computer ?type))
\end{scheme}

\noindent
3つ目の項目が二要素リストでありその1つ目の要素が\code{computer}である全ての職種エントリに
適合します。

\begin{scheme}
(job (Bitdiddle Ben) (computer wizard))
(job (Hacker Alyssa P) (computer programmer))
(job (Fect Cy D) (computer programmer))
(job (Tweakit Lem E) (computer technician))
\end{scheme}

\noindent
この同じパターンが以下にはマッチ``しません''。

\begin{scheme}
(job (Reasoner Louis) (computer programmer trainee))
\end{scheme}

\noindent
なぜならエントリの3つ目の項目が3要素のリストであり、パターンの3つ目の項目がそこは
2要素でなければならないと指定しているためです。もし私達がパターンを変更し3つ目の
項目が\code{computer}で始まる任意のリストでも良いようにしたければ、以下のように
指定可能です。\footnote{これは\link{Exercise 2.20}で紹介されたドット付き末尾記述を
用いています。}

\begin{scheme}
(job ?x (computer . ?type))
\end{scheme}

\noindent
例えば、以下のクエリは、

\begin{scheme}
(computer . ?type)
\end{scheme}

\noindent
次のデータに適合します。

\begin{scheme}
(computer programmer trainee)
\end{scheme}

\noindent
この時\code{?type}はリスト\code{(programmer trainee)}になります。これはまた
次のデータにも適合します。

\begin{scheme}
(computer programmer)
\end{scheme}

\noindent
この時\code{?type}はリスト\code{(programmer)}になります。さらに以下のデータにも
適合します。

\begin{scheme}
(computer)
\end{scheme}

\noindent
この時\code{?type}は空リスト\code{()}です。


クエリ言語の簡単なクエリの処理は以下のように説明できます。

\begin{itemize}

\item
システムはクエリパターン内の変数に対する、パターンを満たす全ての割り当てを見つけます---つまり、
パターン変数が値によりインスタンス化されるような(例示されるような)、つまり値により置き換えられるような
変数に対する値の全ての集合です。結果はデータベース内に存在します。

\item
システムはクエリに対し、パターンを満たす変数割り当てと共に、クエリパターンの
全てのインスタンス(事例)を列挙することで応答します。

\end{itemize}

\noindent
もしパターンに変数が無い場合、クエリはそのパターンがデータベース内に存在するかどうか
の決定に簡約されることに注意して下さい。もしそうならば、変数に何の値も割り当てない
空割り当てがデータベースに対するそのパターンを満たします。

\begin{quote}
\heading{\phantomsection\label{Exercise 4.55}Exercise 4.55:}
以下の情報をデータベースから取り出す簡単なクエリを与えよ。

\begin{enumerate}

\item
Ben Bitdiddleにより監督される(supervisor)全ての人

\item
経理部門に属す全ての人の名前(name)と職種(job)

\item
Slumervilleに済む全ての人の名前と住所(address)

\end{enumerate}
\end{quote}

\subsubsection*{複合クエリ}


単純なクエリはクエリ言語のプリミティブな命令を形成します。複雑な命令を形成する
ためには、クエリ言語は組み合わせの手段を提供します。クエリ言語を論理プログラミング言語と
成す物の1つに組み合わせの手段が論理式を形成するのに用いられる組み合わせの手段に酷似することが
あげられます。\code{and}, \code{or}, \code{not}です。(ここでは\code{and}, \code{or}, \code{not}は
Lispのプリミティブではありません。クエリ言語の組込命令です。)

\noindent
\code{and}を以下の様に用いて全てのコンピュータプログラマの住所を見つけることができます。

\begin{scheme}
(and (job ?person (computer programmer))
     (address ?person ?where))
\end{scheme}

\noindent
結果の出力は以下の通りです。

\begin{scheme}
(and (job (Hacker Alyssa P) (computer programmer))
     (address (Hacker Alyssa P) (Cambridge (Mass Ave) 78)))
(and (job (Fect Cy D) (computer programmer))
     (address (Fect Cy D) (Cambridge (Ames Street) 3)))
\end{scheme}

\noindent
一般的に、

\begin{scheme}
(and ~\( \dark \langle \)~~\( \dark query_1 \)~~\( \dark \rangle \)~ ~\( \dark \langle \)~~\( \dark query_2 \)~~\( \dark \rangle \)~ ~\( \dots \)~ ~\( \dark \langle \)~~\( \dark query_n \)~~\( \dark \rangle \)~)
\end{scheme}

\noindent
上の式はパターン変数に対する全ての値の集合が同時に
\( \langle \)\( query_1 \)\( \rangle \) \( \dots \) \( \langle \)\( query_n \)\( \rangle \)を満す時に
満たされます。

\noindent
簡単にクエリに関しては、システムはクエリを満たすパターン変数への全ての割り当てを見つける
ことにより複合クエリを処理します。そしてそれらの値によるクエリのインスタンスを
表示します。

\noindent
複合クエリを構築する別の手段として\code{or}を通す方法があります。例えば、

\begin{scheme}
(or (supervisor ?x (Bitdiddle Ben))
    (supervisor ?x (Hacker Alyssa P)))
\end{scheme}

\noindent
上の式はBen Bitdiddle、またはAlyssa P.  Hackerに監督される従業員全てを見つけます。

\begin{scheme}
(or (supervisor (Hacker Alyssa P) (Bitdiddle Ben))
    (supervisor (Hacker Alyssa P) (Hacker Alyssa P)))
(or (supervisor (Fect Cy D) (Bitdiddle Ben))
    (supervisor (Fect Cy D) (Hacker Alyssa P)))
(or (supervisor (Tweakit Lem E) (Bitdiddle Ben))
    (supervisor (Tweakit Lem E) (Hacker Alyssa P)))
(or (supervisor (Reasoner Louis) (Bitdiddle Ben))
    (supervisor (Reasoner Louis) (Hacker Alyssa P)))
\end{scheme}

\noindent
一般的に、

\begin{scheme}
(or ~\( \dark \langle \)~~\( \dark query_1 \)~~\( \dark \rangle \)~ ~\( \dark \langle \)~~\( \dark query_2 \)~~\( \dark \rangle \)~ ~\( \dots \)~ ~\( \dark \langle \)~~\( \dark query_n \)~~\( \dark \rangle \)~)
\end{scheme}

\noindent
上の式はパターン変数に対する全ての値の集合が、
\( \langle \)\( query_1 \)\( \rangle \) \( \dots \) \( \langle \)\( query_n \)\( \rangle \)の内、少くとも1つを
満たす場合に満たされます。

\noindent
複合クエリはまた\code{not}を用いても形成できます。例えば、

\begin{scheme}
(and (supervisor ?x (Bitdiddle Ben))
     (not (job ?x (computer programmer))))
\end{scheme}

\noindent
上の式はBen Bitdiddleに監督されるが、コンピュータプログラマではない全ての人を見つけます。
一般的に、

\begin{scheme}
(not ~\( \dark \langle \)~~\( \dark query_1 \)~~\( \dark \rangle \)~)
\end{scheme}

\noindent
上の式はパターン変数に対する全ての割り当てが\( \langle \)\( query_1 \)\( \rangle \)を
満たさない場合に満たされます。\footnote{実際にはこの\code{not}の説明は簡単な場合に対してのみ有効です。
本当の\code{not}の振舞はより複雑です。\code{not}の奇妙な点については節\link{Section 4.4.2}と
\link{Section 4.4.3}にて調査します。}

最後の組み合わせ形式は\code{lisp\-/value}と呼ばれます。lisp-value
がパターンの最初の要素の時、
次の要素は(インスタンス化された)残りの要素を引数として適用されるLispの述語であることを意味します。
一般的に、

\begin{scheme}
(lisp-value ~\( \dark \langle \)~~\var{\dark predicate}~~\( \dark \rangle \)~ ~\( \dark \langle \)~~\( \dark arg_1 \)~~\( \dark \rangle \)~ ~\( \dots \)~ ~\( \dark \langle \)~~\( \dark arg_n \)~~\( \dark \rangle \)~)
\end{scheme}

\noindent
上の式は\( \langle \)\var{predicate}\( \rangle \)がパターン変数に対してインスタンス化された
\( \langle \)\( arg_1 \)\( \rangle \) \( \dots \) \( \langle \)\( arg_n \)\( \rangle \)に適用された
時の値がtrueになる場合の割り当てにより満たされます。

\begin{scheme}
(and (salary ?person ?amount) (lisp-value > ?amount 30000))
\end{scheme}


\begin{quote}
\heading{\phantomsection\label{Exercise 4.56}Exercise 4.56:}
以下の情報を取得する複合クエリを定式化せよ。

\begin{enumerate}[a]

\item
Ben Bitdiddleに監督される全ての人の名前と住所を共に

\item
Ben Bitdiddleよりも給料(salary)が安い全ての人をその給料とBen Bitdiddleの給料と共に

\item
コンピュータ部門ではない人に監督されている全ての人をその上司の名前と職種と共に

\end{enumerate}
\end{quote}

\subsubsection*{ルール}


プリミティブなクエリと複合クエリに加えて、クエリ言語はクエリを抽象化する手段を
提供します。これらは\newterm{rules}(\jnewterm{ルール})により提供されます。
以下のルールは、

\begin{scheme}
(rule (lives-near ?person-1 ?person-2)
      (and (address ?person-1 (?town . ?rest-1))
           (address ?person-2 (?town . ?rest-2))
           (not (same ?person-1 ?person-2))))
\end{scheme}

\noindent
二人の人が同じ街に住んでいるのなら、お互いに近くに住んでいると指定しています。
最後の\code{not}節はこのルールが全ての人がその人自身の近くに住んでいると言うことを
防ぎます。\code{same}リレーションはとても簡単なルールにより定義されます。\footnote{2つの
物が同じであるようにするためには\code{same}は必要ではないことに注意して下さい。
単に同じパターン変数をそれぞれに使用するだけです。実際に、最初から2つの物でなく
1つの物しか持ちません。例として\code{lives\-/near}ルールの\code{?town}や下記の\code{wheel}ルールの
\code{?middle\-/manager}を参照して下さい。\code{same}は2つの物が異なることを強制する場合に
便利です。例えば\code{lives\-/near}ルールの\code{?person\-/1}と\code{?person\-/2}です。
同じパターン変数をクエリの2つの部分に使うことは両方の場所に同じ値が現れることを
強制しますが、異なるパターン変数を用いることは異なる値が現れることを強制しません。
(異なるパターン変数に割り当てられた値は同じにも違う値にも成り得ます。)}

\begin{scheme}
(rule (same ?x ?x))
\end{scheme}

\noindent
以下のルールはある人が監督する人が同様に監督者である場合に組織内での``wheel''(重要人物)であると
宣言します。

\begin{scheme}
(rule (wheel ?person)
      (and (supervisor ?middle-manager ?person)
           (supervisor ?x ?middle-manager)))
\end{scheme}

\noindent
ルールの一般的な形式は以下となります。

\begin{scheme}
(rule ~\( \dark \langle \)~~\var{\dark conclusion}~~\( \dark \rangle \)~ ~\( \dark \langle \)~~\var{\dark body}~~\( \dark \rangle \)~)
\end{scheme}

\noindent
\( \langle \)\var{conclusion}\( \rangle \)がパターンであり\( \langle \)\var{body}\( \rangle \)が
任意のクエリです。\footnote{私達はsameの様にボディの無いルールも認めます。また
そのようなルールは、ルールの結論(conclusion)が任意の変数の値により満たされたことを意味する
と解釈します。}ルールは大きな(例え無限でも)アサーションの集合を表現するものとして考えることができます。
即ち、ルールのボディを満たす変数の割り当てを用いたルールの結果の全てのインスタンスです。
簡単なクエリ(パターン)を説明した時、変数への割り当ては、インスタンス化されたパターンがデータベース内に
存在する場合にパターンが満たされると説明しました。しかし、パターンは明示的にアサーションとして
データベース内に存在する必要はありません。ルールにより暗示される暗黙的なアサーションに成り得ます。
例えば、以下のクエリは、

\begin{scheme}
(lives-near ?x (Bitdiddle Ben))
\end{scheme}

\noindent
次の結果を生みます。

\begin{scheme}
(lives-near (Reasoner Louis) (Bitdiddle Ben))
(lives-near (Aull DeWitt) (Bitdiddle Ben))
\end{scheme}

\noindent
Ben Bitdiddleの近くに住む全てのコンピュータプログラマを見つけるためには、
以下のように質問することができます。

\begin{scheme}
(and (job ?x (computer programmer))
     (lives-near ?x (Bitdiddle Ben)))
\end{scheme}

\noindent
複合手続の場合と同様に、ルールは他のルールの一部分として(上記の\code{lives\-/near}ルールで
見たように)使用可能です。または再帰的に定義することさえもできます。例として、
以下のルールは、

\begin{scheme}
(rule (outranked-by ?staff-person ?boss)
      (or (supervisor ?staff-person ?boss)
          (and (supervisor ?staff-person ?middle-manager)
               (outranked-by ?middle-manager ?boss))))
\end{scheme}

\noindent
もしボスがスタッフの上司であるか、(再帰的に)スタッフの上司よりボスが上役(outranked)であるならば
ボスはスタッフより地位が上であると言えます。

\begin{quote}
\heading{\phantomsection\label{Exercise 4.57}Exercise 4.57:}
人(person)その1が人その2を置き換えられるとは人その1が人その2と同じ仕事をしているか、
または第三者(someone)が人その1と同じ仕事をしつつ、かつ人その2の仕事も行え、そして
人その1と人その2が異なる人である場合であると述べるルールを定義せよ。
そのルールを用いて以下の条件を見つけるクエリを与えよ。

\begin{enumerate}[a]

\item
Cy D. Fectを置き換えられる全ての人

\item
自分より給料の高い誰かを置き換えられる全ての人を二人の給料と一緒に。

\end{enumerate}
\end{quote}

\begin{quote}
\heading{\phantomsection\label{Exercise 4.58}Exercise 4.58:}
ある人が自分が働いている同じ部署に上司(監督者)がいない場合にその人を``big shot''(有力者)である
と述べるルールを定義せよ。
\end{quote}

\begin{quote}
\heading{\phantomsection\label{Exercise 4.59}Exercise 4.59:}
Ben Bitdiddleはある会議を何度も欠席してしまった。彼の会議を忘れる癖は仕事を失う
恐れがある。Benは何かしなければならないと決心した。彼は会社の週次ミーティング全てを
Microshaftデータベースに以下のアサーションとして加えた。

\begin{scheme}
(meeting accounting (Monday 9am))
(meeting administration (Monday 10am))
(meeting computer (Wednesday 3pm))
(meeting administration (Friday 1pm))
\end{scheme}

各アサーションは部門の全体ミーティングのためのものだ。Benはまた全ての部門に渡る
全社会議のエントリを追加した。会社の全従業員がこの会議に参加する。

\begin{scheme}
(meeting whole-company (Wednesday 4pm))
\end{scheme}

\begin{enumerate}[a]

\item
金曜の朝に、Benはその日の全ての会議をデータベースからクエリしたいと思った。
彼のクエリはどのような物になるか?

\item
Alyssa P. Hackerは感心しなかった。彼女は自分の名前を指定することで彼女の会議を
尋ねることができればより便利になるだろうと考えた。そのため彼女はある人の会議は
全ての\code{whole\-/company}(全社)会議に加えてその人の部門会議を全て含むと言うルールを
設計した。Alyssaのルールのボディを埋めよ。

\begin{scheme}
(rule (meeting-time ?person ?day-and-time)
      ~\( \dark \langle \)~~\var{\dark rule-body}~~\( \dark \rangle \)~)
\end{scheme}

\item
Alyssaは水曜の朝に仕事場に到着し、その日に何の会議があるかについて考えた。
上記のルールを定義した上で、彼女のがこのことを見つけるためにはどのようなクエリを
行うべきか?

\end{enumerate}
\end{quote}

\begin{quote}
\heading{\phantomsection\label{Exercise 4.60}Exercise 4.60:}
以下のクエリを与えることにより、

\begin{scheme}
(lives-near ?person (Hacker Alyssa P))
\end{scheme}

Alyssa P. Hackerは仕事場に相乗りできる、彼女の近所に住む人を見つけることができる。
一方で、お互いが近所に住んでいる全ての人々のペアを見つけたい場合には以下のクエリを
用いる。

\begin{scheme}
(lives-near ?person-1 ?person-2)
\end{scheme}

\noindent
彼女はお互いに近所に住んでいる人々の各ペアが二度づつ挙げられていることに気付いた。
例えば、

\begin{scheme}
(lives-near (Hacker Alyssa P) (Fect Cy D))
(lives-near (Fect Cy D) (Hacker Alyssa P))
\end{scheme}

なぜこれが起こるのか?  お互いに近くに住んでいる人々のリストを各ペアが一度しか現れないように
見つける方法は存在するか? 説明せよ。
\end{quote}

\subsubsection*{プログラムとしての論理}


ルールを論理的意味合いの一種であると見做すことができます。\emph{もし}パターン変数に
対する値の割り当てがボディを満たす場合、\emph{それならば}結論を満たします。必然的に、
クエリ言語はルールを基にした\newterm{logical deductions}(\jnewterm{論理的推理})を実行する能力を
有すると見做すことができます。例として、\link{Section 4.4}の始めに説明した\code{append}命令について
考えてみましょう。既に述べたように、\code{append}は以下の2つのルールにて特徴づけられます。

\begin{itemize}

\item
任意のリスト\code{y}に対し、空リストと\code{y}の\code{append}は\code{y}を形成する。

\item
任意の\code{u}, \code{v}, \code{y}, \code{z}に対し、\code{(cons u v)}と\code{y}の\code{append}は
もし\code{v}と\code{y}\code{append}が\code{z}を形成する場合、\code{(cons u z)}を形成する。

\end{itemize}

\noindent
これを私達のクエリ言語で表現するために、以下の関係に対する2つのルールを定義します。

\begin{scheme}
(append-to-form x y z)
\end{scheme}

\noindent
上の関係は``\code{x}と\code{y}の\code{append}は\code{z}を形成する''ことを意味すると解釈できます。

\begin{scheme}
(rule (append-to-form () ?y ?y))
(rule (append-to-form (?u . ?v) ?y (?u . ?z))
      (append-to-form ?v ?y ?z))
\end{scheme}

\noindent
最初のルールにはボディがありません。これは結果部分が\code{?y}の任意の値を保持することを
意味します。2つ目のルールがどのようにドット付き末尾記述をリストの\code{car}と\code{cdr}に
名前を付けるために使用しているかについて注意して下さい。


これら2つのルールを与えられることで、2つのリストに対する\code{append}を求めるクエリを
定式化することができます。

\begin{scheme}
~\textit{;;; Query input:}~
(append-to-form (a b) (c d) ?z)
~\textit{;;; Query results:}~
(append-to-form (a b) (c d) (a b c d))
\end{scheme}

\noindent
より印象的なのは、同じルールを``\code{(a b)}に対し\code{append}したら\code{(a b c d)}になるリストは何''
という質問に使用できることです。これは以下のように行われます。

\begin{scheme}
~\textit{;;; Query input:}~
(append-to-form (a b) ?y (a b c d))
~\textit{;;; Query results:}~
(append-to-form (a b) (c d) (a b c d))
\end{scheme}

\noindent
\code{append}すると\code{(a b c d)}を形成する全てのリストのペアを尋ねることも可能です。

\begin{scheme}
~\textit{;;; Query input:}~
(append-to-form ?x ?y (a b c d))
~\textit{;;; Query results:}~
(append-to-form () (a b c d) (a b c d))
(append-to-form (a) (b c d) (a b c d))
(append-to-form (a b) (c d) (a b c d))
(append-to-form (a b c) (d) (a b c d))
(append-to-form (a b c d) () (a b c d))
\end{scheme}

\noindent
上記のクエリに対する答を推論するルールを用いることにておいて、クエリシステムは
かなりの知性を示すように見えるかもしれません。実際には次の節で学ぶように、
システムはルールをときほぐす明確なアルゴリズムに従っているに過ぎません。
残念ながら、システムが\code{append}の場合では見事な程うまく行きますが、
一般的な手法はより複雑な場合に分解されるかもしれません。このことは\link{Section 4.4.3}で学びます。

\begin{quote}
\heading{\phantomsection\label{Exercise 4.61}Exercise 4.61:}
以下のルールはリストの直前の要素を見つける関係\code{next\-/to}を実装する。

\begin{scheme}
(rule (?x next-to ?y in (?x ?y . ?u)))
(rule (?x next-to ?y in (?v . ?z))
      (?x next-to ?y in ?z))
\end{scheme}


以下のクエリの結果を答えよ。

\begin{scheme}
(?x next-to ?y in (1 (2 3) 4))
(?x next-to  1 in (2 1 3 1))
\end{scheme}
\end{quote}

\begin{quote}
\heading{\phantomsection\label{Exercise 4.62}Exercise 4.62:}
\link{Exercise 2.17}の\code{last\-/pair}命令を実装するルールを定義せよ。
これは空ではないリストの最後の要素を含むリストを返す。あなたのルールを
\code{(last\-/pair (3) ?x)}, \code{(last\-/pair (1 2 3) ?x)}, \code{(last\-/pair (2 ?x) (3))}のような
クエリにて確認せよ。あなたのルールは\code{(last\-/pair ?x (3))}の様なクエリに対し
正しく動作するだろうか?
\end{quote}

\begin{quote}
\heading{\phantomsection\label{Exercise 4.63}Exercise 4.63:}
以下のデータベース(創世記第4章を参照せよ)はAdaの子孫の家系をCainを経由してAdamまで
戻りながら辿っている。

\begin{scheme}
(son Adam Cain)
(son Cain Enoch)
(son Enoch Irad)
(son Irad Mehujael)
(son Mehujael Methushael)
(son Methushael Lamech)
(wife Lamech Ada)
(son Ada Jabal)
(son Ada Jubal)
\end{scheme}

``もし\( S \)が\( f \)の息子であり、かつ、\( f \)が\( G \)の息子ならば、\( S \)は\( G \)
の孫である''と``もし\( W \)が\( M \)の妻であり、かつ、\( S \)が\( W \)の息子ならば、\( S \)は
\( M \)の息子である''(これは恐らく今日より聖書の時代にはより正確であっただろう)のルールを
定式化せよ。これらはクエリシステムに対しCainの孫、Lamechの息子、Methushaelの孫を見つけることを可能にする。
(より複雑な関係を推論するいくつかのルールについては\link{Exercise 4.69}を参照せよ。)

\end{quote}

\subsection{クエリシステムの働き方}
\label{Sec. 4.4.2}
\label{Section 4.4.2}

\link{Section 4.4.4}ではクエリインタプリタを手続の集合として紹介します。この節では低レベルの
実装上の詳細からは独立したシステムの一般的な構造について説明する概観を与えます。
インタプリタの実装を説明した後に、私達はインタプリタのいくつかの限界と記号論理学の演算とは異なる
クエリ言語の論理演算のいくつかの微妙な行い方を理解できる位置に辿り着きます。

クエリ評価機がクエリをデータベース内の事実とルールに対してマッチさせるために
ある種の探索を実行せねばならないことは明らかでしょう。これを行う1つの方法はクエリシステムを
\link{Section 4.3}の\code{amb}評価機を用いて非決定性プログラムとして実装することになります(\link{Exercise 4.78}参照)。
別の可能性にはストリームの助けを用いて探索を管理する方法があります。私達の実装は
この2つ目のアプローチに従います。

クエリシステムは2つの中心となる演算、\newterm{pattern matching}(\jnewterm{パターンマッチング})と
\newterm{unification}(\jnewterm{ユニフィケーション}、単一化)の周りに体系化されます。
最初にパターンマッチングについて記述し、この演算がフレームのストリームを用いた情報体系と
共にどのように単純クエリと複合クエリの両方を実装可能にするのか説明します。
次に私達はユニフィケーション、つまりルールを実装ために必要なパターンマッチングの
一般化について議論します。最後に、
\link{Section 4.1}で説明されたインタプリタのために\code{eval}が式を分類する方法と同様の方法で、
式を分類する手続を通してクエリインタプリタ全体がどのように組み合わされるかについて示します。

\subsubsection*{パターンマッチング}

\newterm{pattern matcher}(\jnewterm{パターンマッチャ})はあるデータが指定されたパターンに適合するかどうかを
試すプログラムです。例えばデータリスト\code{((a b) c (a b))}はパターン\code{(?x c ?x)}に対し
パターン変数\code{?x}が\code{(a b)}に束縛されることで適合します。
同じデータリストがパターン\code{(?x ?y ?z)}に対し\code{?x}と\code{?z}の両者が\code{(a b)}に束縛され、
\code{?y}が\code{c}に束縛されることで適合します。これはまたパターン\code{((?x ?y) c (?x ?y))}に
対しても\code{?x}が\code{a}に、\code{?y}が\code{b}に束縛されることで適合します。
しかし、これはパターン\code{(?x a ?y)}には適合しません。このパターンが2つ目の要素がシンボル\code{a}で
あるリストを指定しているためです。

パターンマッチャはクエリシステムにより使用されます。クエリシステムは入力として
パターン、データ、\newterm{frame}(\jnewterm{フレーム})を取ります。フレームはさまざまな
パターン変数に対する束縛を指定します。パターンマッチャはデータがフレームに既に
存在する束縛と一致する状態でパターンに適合するかどうかをチェックします。
もしそうであれば、その適合により決定された任意の束縛を増やしたフレームを返します。
そうでなければ、適合が失敗したことを示します。

例えば、パターン\code{(?x ?y ?x)}を用いて\code{(a b a)}に空のフレームを与えられた
場合に適合を行うと\code{?x}が\code{a}に、\code{?y}が\code{b}に束縛されることを指定する
フレームを返します。同じパターン、同じデータで\code{?y}が\code{a}に束縛されている
と指定するフレームを用いて適合を行うと失敗します。同じパターン、同じデータで
\code{?y}が\code{b}に束縛され\code{?x}が未束縛であるフレームを用いて適合を行えば
与えられたフレームに\code{?x}の\code{a}への束縛を増やした物が返されます。

パターンマッチャはルールを含まない単純なクエリを処理するのに必要な仕組みの全てです。
例えば、以下のクエリを処理する場合、

\begin{scheme}
(job ?x (computer programmer))
\end{scheme}

\noindent
データベース内の全てのアサーションを探索し、最初は空のフレームを考慮してパターンに
適合する物を選択します。探索を行った各適合に対して、適合により返されたフレームを
用いてパターンを\code{?x}の値と共にインスタンス化します。

\subsubsection*{フレームのストリーム}

フレームに対してパターンのテストを行うことはストリームの使用を通して体系化されています。
単一のフレームを与えられて、マッチング処理はデータベースのエントリを1つづつ通して実行します。
各データベースエントリに対して、マッチャは適合が失敗したことを示す特別なシンボルか、
フレームに対する拡張を生成します。全てのデータベースエントリに対する結果はストリーム内に
集められ、フィルタを通すことで失敗が取り除かれます。結果は与えられた
フレームを適合を通すことでデータベース内のあるアサーションに拡張した全てのフレームの
ストリームです。\footnote{マッチングは一般的にとても重いので、完全なマッチャをデータベース
の全ての要素に対して適用することは防ぎたいと考えます。これは通常は高速で粗い適合と
最終適合の部品に分解することで準備します。粗い適合はデータベースをフィルタし、最終適合の
ための候補の小さな集合を生成します。手間をかけて、粗い適合のいくつかの成果がデータベースが
候補を選択したい時ではなく、構築された時に行えるようにデータベースを事前に
準備することができます。データベースの索引の仕組みの周りには莫大な技術が構築されています。
私達の実装は\link{Section 4.4.4}で説明されているように、そのような最適化のあまり賢くはない形態を
含んでいます。}

私達のシステムでは\link{Figure 4.4}で示されるように、クエリはフレームの入力ストリームを取り、
ストリーム内の各フレームに対して上記のマッチング処理を実行します。
言い替えれば、入力ストリーム内の各フレームに対して、クエリはデータベース内のアサーションに対する
適合による、全てのフレームの拡張から成る新しいストリームを生成します。
これらのストリームの全ては次に組み合わされて1つの大きなストリームを形成します。
これは入力ストリーム内の各フレームの全ての可能な拡張を含んでいます。
このストリームがクエリの出力です。

単純なクエリに答えるためにはクエリを単一の空フレームから成る入力ストリームと共に
用います。結果としての出力ストリームは空にフレームに対する全ての拡張を含んでいます(言い換えれば、
クエリに対する全ての答を含みます)。このフレームのストリームは次に、元々のクエリのパターンと
各フレーム内の値でインスタンス化された変数のコピーのストリームを生成するのに利用されます。
そしてこれが最終的に表示されるストリームです。

\begin{figure}[tb]
\phantomsection\label{Figure 4.4}
\centering
\begin{comment}
\heading{Figure 4.4:} A query processes a stream of frames.

\begin{example}
                                  output stream
  input stream   +-------------+  of frames,
  of frames      |    query    |  filtered and extended
---------------->|             +------------------------->
                 | (job ?x ?y) |
                 +-------------+
                        ^
                        |
               stream of assertions
                  from data base
\end{example}
\end{comment}
\includegraphics[width=102mm]{fig/chap4/Fig4.4.pdf}
\par\bigskip
\noindent
\heading{Figure 4.4:} フレームのストリームを処理するクエリ
\end{figure}

\subsubsection*{複合クエリ}


フレームのストリーム実装の真に優雅な点は複合クエリを扱う時に明白になります。
複合クエリの処理は適合の結果が指定されたフレームに一致するという私達の
マッチャが要求する能力を利用します。例えば、2つのクエリの\code{and}を取り扱う
以下のようなクエリでは

\begin{scheme}
(and (can-do-job ?x (computer programmer trainee))
     (job ?person ?x))
\end{scheme}

\noindent
(簡単に言えば、``コンピュータプログラマ見習いの職を行える全ての人を見つけろ'')
まず以下のパターンに適合する全てのエントリを見つけます。

\begin{scheme}
(can-do-job ?x (computer programmer trainee))
\end{scheme}

\noindent
これはフレームのストリームを生成します。各フレームは\code{?x}に対する束縛を含んでいます。
次にストリーム内の各フレームに対し、与えられた\code{?x}に対する束縛に一致する様に、
以下のパターンに適合する全てのエントリを探します。

\begin{scheme}
(job ?person ?x)
\end{scheme}

\noindent
そのような適合のそれぞれは\code{?x}と\code{?person}に対する束縛を含むフレームを生成します。
2つのクエリの\code{and}は\link{Figure 4.5}に示されるように、一連の2つのクエリのコンポーネントの
組み合わせであると見做すことができます。最初のクエリフィルタを通過するフレームは
フィルタをかけられ、2つ目のクエリにてさらに拡張されます。

\begin{figure}[tb]
\phantomsection\label{Figure 4.5}
\centering
\begin{comment}
\heading{Figure 4.5:} The \code{and} combination of two queries is produced 
by operating on the stream of frames in series.

\begin{example}
                +----------------------+
                |       (and A B)      |
  input stream  |                      |  output stream
  of frames     |   +---+       +---+  |  of frames
------------------->| A +------>| B +-------------------->
                |   +---+       +---+  |
                |     ^           ^    |
                |     |           |    |
                |     +-----*-----+    |
                +-----------|----------+
                            |
                        data base
\end{example}
\end{comment}
\includegraphics[width=93mm]{fig/chap4/Fig4.5.pdf}
\begin{quote}
\heading{Figure 4.5:} 2つのクエリの\code{and}の組合せはフレームのストリーム上での連続した操作により生成される
\end{quote}
\end{figure}

\begin{figure}[tb]
\phantomsection\label{Figure 4.6}
\centering
\begin{comment}
\heading{Figure 4.6:} The \code{or} combination of two queries is produced 
by operating on the stream of frames in parallel and merging the results.

\begin{example}
           +---------------------------+
           |          (or A B)         |
           |    +---+                  |
input      | +->| A |------------+     |  output
stream of  | |  +---+            V     |  stream of
frames     | |    ^          +-------+ |  frames
-------------*    |          | merge +--------------->
           | |    |          +-------+ |
           | |    |              ^     |
           | |    |   +---+      |     |
           | +------->| B +------+     |
           |      |   +---+            |
           |      |     ^              |
           |      |     |              |
           |      +--*--+              |
           +---------|-----------------+
                     |
                 data base
\end{example}
\end{comment}
\includegraphics[width=107mm]{fig/chap4/Fig4.6.pdf}
\begin{quote}
\heading{Figure 4.6:} 2つのクエリのcode{or}の組合せはフレームのストリームを並列に操作しその結果をマージする
ことで生成される
\end{quote}
\end{figure}

\noindent
\link{Figure 4.6}は2つのクエリの\code{or}を2つのクエリコンポーネント
の並列な組み合わせとして求めるための類似の手法を示しています。
フレームの入力ストリームは各クエリにより別々に拡張されます。2つの結果ストリームは
次にマージされ最終の出力ストリームを生成します。

この高いレベルの記述からでも複合クエリの処理が遅くなることがはっきりとわかります。
例えば、クエリは各入力フレームに対して複数の出力ストリームを生成するかもしれません。
そして各クエリも同様です。最悪の場合にはクエリ数の指数関数となる多数のマッチングを
実行しなければなりません(\link{Exercise 4.76}参照)。\footnote{しかし、この種の
指数関数爆発は\code{and}クエリでは一般的ではありません。追加された条件が生成されるフレームの
数を増やすのでなく、減らす傾向があるためです。}単純なクエリのみを扱うシステムのほうが
とても実用的ではありますが、複合クエリを扱うことは極めて難しいのです。\footnote{複合
クエリをどのように効果的に扱うかに関連するデータベース管理システムの多数の文献が
存在します。}

フレームのストリームの視点から、あるクエリの\code{not}はクエリが満たされる
全てのフレームを取り除くフィルタとして働きます。例えば、以下のパターンを与えらえると、

\begin{scheme}
(not (job ?x (computer programmer)))
\end{scheme}

\noindent
入力ストリームの各フレームに対して\code{(job ?x (computer programmer))}を満たす
拡張フレームの生成を試みます。入力ストリームからそのような拡張が存在する
全てのフレームを削除します。結果はフレーム中の\code{?x}の束縛が
\code{(job ?x (computer programmer))}を満たさないフレームのみから成るストリームと
なります。例えば以下のクエリの処理においては、

\begin{scheme}
(and (supervisor ?x ?y)
     (not (job ?x (computer programmer))))
\end{scheme}

\noindent
最初の節は\code{?x}と\code{?y}に対する束縛を持つフレームを生成します。次に
\code{not}節はこれらから\code{?x}に対する束縛が\code{?x}がコンピュータプログラマである
という制約を満たす全てのフレームを削除することでフィルタリングします。\footnote{この
\code{not}のフィルタ実装と、記号論理学における通常の意味での\code{not}の間には
微妙な違いが存在します。\link{Section 4.4.3}を参照して下さい。}


\code{lisp\-/value}特殊形式はフレームのストリーム上の同様なフィルタとして実装されます。
ストリーム内の各フレームをパターン内の任意の変数をインスタンス化するために用い、
そしてLisp手続を適用します。入力ストリームから述語が失敗する全てのフレームを削除します。

\subsubsection*{ユニフィケーション}


クエリ言語内のルールを扱うために、ルールの結果が与えられたクエリパターンに適合する
ルールを見付けられねばなりません。ルールの結果はアサーションに似ていますが、変数を
含められる所が異なります。そのためパターンマッチングの
一般化---\newterm{unification}(\jnewterm{ユニフィケーション})と呼ばれます---を必要とし、
その中で``パターン''と``データ''の両方が変数を持ち得ます。

ユニファイアは2つの定数と変数を含むパターンを取り、2つのパターンを等しくする
変数への値の割り当てが可能であるかどうかを決定します。もしそうであれば、
これらの束縛を含むフレームを返します。例えば\code{(?x a ?y)}と\code{(?y ?z a)}の
ユニフィケーションは\code{?x}, \code{?y}, \code{?z}が全て\code{a}に束縛されなければならない
フレームを指示します。一方で、\code{(?x ?y a)}と\code{(?x b ?y)}のユニフィケーションは
失敗します。2つのパターンを等しくできる\code{?y}の値が存在しないためです。(両方のパターンの
2つ目の要素が等しくなるためには\code{?y}は\code{b}にならなければなりません。
しかし、3番目の要素が等しくなるためには\code{?y}が\code{a}になるしかありません)。
クエリシステムで用いられるユニファイアはパターンマッチャの様に、フレームを入力として取り
このフレームと一致するユニフィケーションを実行します。

ユニフィケーションアルゴリズムはクエリシステムで最も技術的に難しい部分です。
複雑なパターンを共なうため、ユニフィケーションの実行は演繹を必要とするように見えるかもしれません。
例えば、\code{(?x ?x)}と\code{((a ?y c) (a b ?z))}をユニフィケーションするためには
アルゴリズムは\code{?x}は\code{(a b c)}に、\code{?y}は\code{b}に、\code{?z}は\code{c}にならなければ
いけないことを推論しなければなりません。この処理はパターンコンポーネント間の
等式の集合を解くこととして考えることができます。一般的には、これらは連立方程式であり、
これを解くためには大量の操作が必要となるでしょう。\footnote{一方向のパターンマッチングでは、
全てのパターン変数を含む等式は明白で未知数(パターン変数)について既に解かれています。}
例えば、\code{(?x ?x)}と\code{((a ?y c) (a b ?z))}のユニフィケーションは以下の連立方程式を
指定することだと考えられるでしょう。

\begin{scheme}
?x  =  (a ?y c)
?x  =  (a b ?z)
\end{scheme}

\noindent
これらの方程式は以下を暗示します。

\begin{scheme}
(a ?y c)  =  (a b ?z)
\end{scheme}

\noindent
これは順に次を暗示します。

\begin{scheme}
 a  =  a, 
?y  =  b, 
 c  =  ?z,
\end{scheme}

\noindent
従って以下の通りです。

\begin{scheme}
?x  =  (a b c)
\end{scheme}

\noindent
パターンマッチが成功する場合、全てのパターン変数は束縛され、それらに束縛される
値は定数のみを持ちます。これはまたここまで見てきた全てのユニフィケーションの例に
対しても真です。しかし一般的に、ユニフィケーションが成功する場合には変数の値が
完全には決定されるとは限りません。いくつかの変数は未束縛のままで、他は変数を
含む値に束縛されます。

\code{(?x a)}と\code{((b ?y) ?z)}のユニフィケーションについて考えます。
\code{?x = (b ?y)}であり\code{a = ?z}であると推論できます。しかしそれ以上\code{?x}と\code{?y}に
ついて解くことはできません。このユニフィケーションは失敗はしません。確かに
2つのパターンを\code{?x}と\code{?y}に値を割り当てることで等しくすることは可能なためです。
この適合が\code{?y}の取り得る値を全く限定しないため、結果フレームに\code{?y}の束縛は
全く入りません。しかしこの適合は\code{?x}の値は限定します。\code{?y}がどのような値を
取っても、\code{?x}は必ず\code{(b ?y)}になります。従って\code{?x}の\code{(b ?y)}への束縛は
フレームへ入れられます。もし\code{?y}の値が(パターンマッチ、またはこのフレームに
一致する必要のあるユニフィケーションにより)後に決定されフレームに追加されたなら、
その前に束縛された\code{?x}はこの値を参照することになります。\footnote{ユニフィケーションについて
考えるもう1つの方法は、二つの入力パターンの特殊化である最も一般的なパターンを生成するということです。
言い換えれば、\code{(?x a)}と\code{((b ?y) ?z)}のユニフィケーションは\code{((b ?y) a)}であり、
上で議論した \code{(?x a ?y)}と\code{(?y ?z a)}のユニフィケーションは\code{(a a a)}です。
私達の実装に対しては、ユニフィケーションの結果をパターンではなく、フレームとして
考えたほうがより便利です。}

\subsubsection*{ルールの適用}


ユニフィケーションはルールから推論を行わせるクエリシステムのコンポーネントに対する
鍵です。これがどのように達成されるかについて学ぶためには、ルールの適用を含むクエリの
処理について考えてみましょう。例えば、以下について考えます。

\begin{scheme}
(lives-near ?x (Hacker Alyssa P))
\end{scheme}

\noindent
このクエリを処理するためには、最初に通常の上で説明されたパターンマッチ手続を用いて
このパターンに適合するアサーションがデータベース内に存在するかどうかを見ます。(この場合には
存在しません。私達のデータベースには誰が誰の近くに住んでいるかについての直接の
アサーションが全く含まれていないためです)。次のステップはクエリパターンと各ルールの結果との
ユニフィケーションを試みることです。このパターンは以下のルールの結果とユニフィケーションする
と、

\begin{scheme}
(rule (lives-near ?person-1 ?person-2)
      (and (address ?person-1 (?town . ?rest-1))
           (address ?person-2 (?town . ?rest-2))
           (not (same ?person-1 ?person-2))))
\end{scheme}

\noindent
結果としてフレームに\code{?person\-/2}が\code{(Hacker Alyssa P)}に束縛され、
\code{?x}が(同じ値として)\code{?person\-/1}に束縛されなければならないことの指定が
入ることを発見します。これで、このフレームに関連して、このルールのボディにより
与えられた複合クエリを評価します。適合が成功すればこのフレームは\code{?person\-/1}に
対する束縛を与えることで拡張され、その結果として\code{?x}の値も決定し、元々のクエリ
パターンをインスタンス化するのに利用することができます。


一般的に、クエリ評価機は以下の手法を用いて、パターン変数に対する束縛を指定する
フレーム内のクエリパターンを定めようとする時に、ルールを適用します。

\begin{itemize}

\item
クエリをルールの結論とユニフィケーションすることで(成功すれば)元のフレームの
拡張を形成する

\item
拡張されたフレームを参照しながら、ルールのボディにより形成されたクエリを評価する

\end{itemize}

\noindent
これがどれほどLispの\code{eval}/\code{apply}評価機内での手続適用のための手法に似ているか
について注意して下さい。

\begin{itemize}

\item
手続のパラメタをその引数に束縛することで元々の手続環境を拡張するフレームを形成する

\item
拡張された環境を参照しながら、手続のボディにより形成された式を評価する

\end{itemize}

\noindent
2つの評価機の間の類似度は驚くべきことではありません手続定義がLispにおける抽象化の
手段であるように、ルール定義はクエリ言語の抽象化の手段です。それぞれの場合において、
適切な束縛を作成し、ルール、または手続のボディをこれらの束縛を参照することで
抽象化を巻き戻します。

\subsubsection*{単純なクエリ}


私達はこの節の始めにルールを欠いた単純なクエリをどのように評価するかについて学びました。
今ではルールの適用の仕方も学んだため、単純なクエリをルールとアサーションの両方を用いて
どのように評価するかについても説明することができます。


クエリパターンとフレームのストリームを与えられた時、入力ストリーム内の各フレームに対して
2つのストリームを生成します。

\begin{itemize}

\item
(パターンマッチャを用いて)データベース内の全てのアサーションに対してパターンの適合を行うことにより
得られた拡張フレームのストリーム

\item
(ユニファイアを用いて)全ての可能なルールを適用することにより得られた
拡張フレームのストリーム\footnote{ユニフィケーションはマッチングの一般化であるため、
ユニファイアを用いて両方のストリームを生成することによりシステムを簡略化することが
できました。しかし、簡単な場合を単純なマッチャで取り扱うことはマッチング(適合)が
どのように(本格的なユニフィケーションとは逆に)それ自身の正しさにおいて便利であることを
説明します。}

\end{itemize}

\noindent
これらの2つのストリームをアペンドすることにより、与えられたパターンを元のフレームに
一致して満たすことができる全ての方法により成り立つストリームを生成します。
これらのストリーム(入力ストリームの各フレームに対して1つ)はこれで全てが1つの
巨大なストリームに接続されます。従ってこの巨大ストリームは元の入力ストリーム内の
任意のフレームを与えられたパターンへの適合を生成するために拡張した全ての方法から
成り立っています。

\subsubsection*{クエリ評価機とドライバループ}

潜在的なマッチング操作の複雑さに係らず、システムは任意の言語のための評価機と
そっくりに体系化されます。マッチング操作を統合する手続は\code{qeval}と呼ばれ、
Lispの\code{eval}手続の役割と同様な役割を演じます。\code{qeval}は入力としてクエリと
フレームのストリームを取ります。その出力はフレームのストリームであり、クエリパターンへの
成功したマッチングに相応します。これは\link{Figure 4.4}で示されるよう入力ストリームの
いくつかを拡張しています。\code{eval}と同様に、\code{qeval}は異なる型の式(クエリ)を分類し、
それぞれに対する適切な手続を呼び出します。各特殊形式(\code{and}, \code{or}, \code{not}, \code{lisp\-/value})に
手続が存在し、また単純なクエリにも手続が存在します。

この章の他の評価機のための\code{driver\-/loop}手続と同様のドライバループが端末から
クエリを読み出します。各クエリに対して、ドライバループは\code{qeval}をそのクエリと
1つの空フレームと共に呼び出します。これにより全ての可能な適合(全ての可能な空フレームに
対する拡張)のストリームが生成されます。結果としてのストリームの各フレームに対して、
ドライバループは元のクエリをフレーム内で見つかった変数の値を用いてインスタンス化します。
次にこのインスタンス化されたクエリのストリームは表示されます。\footnote{私達が
フレームの(リストではなく)ストリームを使う理由は、ルールの再帰的適用は
クエリを満たす無限の数の値を生成することができるからです。ストリームに組込まれた
遅延化された評価がここでは重要です。システムは応答を1つづつそれらが生成された順に、
有限か無限の数の応答があるかに係らずに表示します。}

ドライバはまた特別なコマンド\code{assert!}をチェックします。これは入力がクエリではなく
データベースに追加するアサーション、またはルールでることを示します。例えば、

\begin{scheme}
(assert! (job (Bitdiddle Ben)
              (computer wizard)))
(assert! (rule (wheel ?person)
               (and (supervisor ?middle-manager ?person)
                    (supervisor ?x ?middle-manager))))
\end{scheme}


\subsection{論理プログラミングは記号論理学なのか?}
\label{Sec. 4.4.3}
\label{Section 4.4.3}

クエリ言語内で使用される組み合わせの手段は最初は記号論理学の\code{and}, \code{or}, \code{not}命令と
同じに見えるかもしれません。実際にクエリ言語のルールの適用は、推論という、まともな手段を通して
達成されます。\footnote{推論の特定の手段がまともであるということは自明な主張ではありません。
もし真となる前提で開始したのであれば、真となる結論のみが導き出されることを証明しな
ければなりません。ルール適用で表現された推論の手法は\newterm{modus ponens}(\jnewterm{肯定式})という
親しみある推論の手法であり、もし\( A \)が真でありかつ\emph{A implies B}(AならばB)が真である
ならば、\( B \)は真であると結論づけることができます。}
しかし、このクエリ言語の記号論理学を用いた同定は実際には有効ではありません。クエリ言語が
論理的な命題を手続的に解釈する\newterm{control structure}(\jnewterm{制御構造})を提供するためです。
私達は頻繁にこの制御構造を活用することができます。例えばプログラマの監督者全てを見るけるためには
以下の2つの論理的に等価な形式のどちらかをクエリとして策定することができます。

\begin{scheme}
(and (job ?x (computer programmer)) (supervisor ?x ?y))
\end{scheme}

\noindent
または

\begin{scheme}
(and (supervisor ?x ?y) (job ?x (computer programmer)))
\end{scheme}

\noindent
もし会社に(通常の場合として)プログラマより多くの監督者が存在するのであれば、
2つ目よりも最初の形式を用いたほうが良いです。なぜならデータベースは\code{and}の
最初の節により生成された中間結果(フレーム)全てに対して探索されねばならないためです。

論理プログラミングの目的はプログラマに演算問題を2つの分離された問題、``何''が求めらるべきかと
``どのように''これが求められるべきかに分解する技術を与えることです。
これは記号論理学の命題の部分集合を選択することで達成されます。これは人が演算したい
対象全てを記述するのに十分に強く、けれども制御可能な手続的解釈を行うに十分に弱い
物です。一方で、ここでの意図は論理プログラミング言語で指示されたプログラムは
計算機により実行され得る実効的なプログラムでなければなりません。
制御(``どのように''演算するか)は言語の評価順の使用に影響を受けます。
私達は節の順と各節の中の下位目標の順とを操作し、演算が実効的、かつ効率的であると
考えれられる順で行われるようにせねばなりません。

私達のクエリ言語は単なるそのような手続的に解釈可能な記号論理学の部分集合であると
見做すことができます。アサーションは単純な事実(アトミックな命題)を表現します。
ルールはルールのボディが持つ複数の場合に対する、ルールの結論が持つ推測の結果を表現します。
ルールは自然な手続的解釈を持ちます。ルールの結論を成立させるためには、ルールのボディを
定めます。従って、ルールは演算を提示しています。しかし、ルールはまた記号論理学の命題であるとも
見做すことができるため、同じ結果が全体的に記号論理学の中で働くことにより得られることを
主張することで、論理プログラムにより遂行された任意の``推論''を正当化することができます。\footnote{私達は
この命題を以下に同意することで制限しなければなりません。``推論''が論理プログラムにより
正当化されるに言及するにおいて、私達は演算が停止することを前提としています。
残念なことに、例えこの制限された命題もクエリ言語の私達の実装においては正しくありません。
(そして同時にPrologのプログラムにとっても、そして他のほとんどの現在の論理プログラミング
言語においてもこれは正しくありません)。原因は私達の\code{not}と\code{lisp\-/value}の使用のためです。
この先で議論するように、クエリ言語で実装された\code{not}は常に記号論理学の\code{not}と一致
しません。そして\code{lisp\-/value}は複雑さを増します。私達は
単純に\code{not}と\code{lisp\-/value}を言語から削除し、プログラムを単純なクエリ, \code{and}, \code{or}のみを用いて
書くことに同意することで、記号論理学と一致する言語を実装することができます。
しかし、これは言語の表現力を大きく制限してしまいます。論理プログラミングにおける主要な
研究課題の1つは過度に表現力を犠牲にすることなく、記号論理学とより一致する方法を
見つけることです。}

\subsubsection*{無限ループ}


論理プログラムの手続的な解釈の結果は絶望的に非効率なプログラムを一部の問題に
対して構築することが有り得ることです。極端に非効率な場合にはシステムは演繹を
行う無限ループに落ち込んでしまいます。簡単な例として、縁組のデータベースを
構築したと考えてみましょう。以下を含みます。

\begin{scheme}
(assert! (married Minnie Mickey))
\end{scheme}

\noindent
ここで以下を尋ねた場合、

\begin{scheme}
(married Mickey ?who)
\end{scheme}

\noindent
応答は有りません。なぜならシステムはもし\( A \)が\( B \)に結婚した場合、\( B \)が
\( A \)に結婚することになることを知らないためです。そのため以下のルールを
宣言します。

\begin{scheme}
(assert! (rule (married ?x ?y) (married ?y ?x)))
\end{scheme}

\noindent
そして再び質問します。

\begin{scheme}
(married Mickey ?who)
\end{scheme}

\noindent
残念ながら、これはシステムを無限ループに追いやります。以下のとおりです。

\begin{itemize}

\item
システムは\code{married}ルールが適用可能であることを見つけます。言い換えれば、
ルールの結論\code{(married ?x ?y)}は成功裏にクエリパターン\code{(married Mickey ?who)}と
単一化し、\code{?x}が\code{Mickey}に、\code{?y}が\code{?who}に束縛されるフレームを生成します。

\item
1つの答は直接データベース内のアサーションとして現れます: \code{(married
Minnie Mickey)}

\item
\code{married}ルールもまた適用可能です。そのためインタプリタは再度ルールのボディを
評価し、今回は\code{(married Mickey ?who)}に等しくなります。

\end{itemize}

\noindent
これでシステムは無限ループの中です。実際に、システムが簡単な答、\code{(married Minnie Mickey)}を
ループに入る前に見つけるかどうかは、システムがデータベース内のアイテムをチェックする順に
関連する実装上の詳細に依存します。これは起こり得るループのとても単純な種類の例です。
相互に関連するルールの蓄積は予想することがより難しいループへと導きます。そしてループの
出現は\code{and}内の節の順(\link{Exercise 4.64}参照)か、またはシステムがクエリを処理する順に
関連する低レベルの詳細に依存します。\footnote{これは論理の問題ではなく、私達のインタプリタにより
提供される手続的な解釈の問題です。ここでループに陥らないインタプリタを書くこともできました。
例えばアサーションとルールから導きだせる全ての証明を深さ優先探索でなく、幅優先探索で
列挙することもできました。しかし、そのようなシステムは私達のプログラムの中における推論の
順序を活用することがより難しくなります。そのようなプログラムの中に洗練された制御を構築する
試みが\link{deKleer et al. 1977}に説明されています。そのような
深刻な制御上の問題に導かない別のテクニックとして、特定の種類のループの検知器のような
特別な知識を組込むことがあります(\link{Exercise 4.67})。しかし、推論の実行において
無限の小道を下ることから確実にシステムを防ぐ一般的な理論体系は有りません。
``\( P(x) \)が真であることを示すためには、\( P(f(x)) \)が真であることを示せ''という様式の
悪魔のルールをいくつかの適切に選択された関数\( f \)に対して想像してみて下さい。}

\subsubsection*{\code{not}の問題}


もう1つのクエリシステムの予測できない出来事は\code{not}に関連します。\link{Section 4.4.1}の
データベースを受け取った時、以下の2つのクエリについて考えてみます。

\begin{scheme}
(and (supervisor ?x ?y)
     (not (job ?x (computer programmer))))
(and (not (job ?x (computer programmer)))
     (supervisor ?x ?y))
\end{scheme}

\noindent
これらの2つのクエリは同じ結果を生成しません。最初のクエリはデータベース中の
\code{(supervisor ?x ?y)}に適合する全てのエントリを見つけ、次に結果のフレームから
\code{?x}の値が\code{(job ?x (computer programmer))}を満たす物を削除します。
2つ目のクエリは入力フレームから\code{(job ?x (computer programmer))}を満たす物を
消すフィルタから開始します。入力フレームだけでは空であるため、データベースから
\code{(job ?x (computer programmer))}を満たするパターンが存在するか確認します。
通常はこの形式のエントリが存在するので、\code{not}節は空のフレームを取り除き、
空のフレームのストリームを返します。結果として、複合クエリ全体が空ストリームを
返します。

問題は\code{not}の私達の実装は本当に変数の値上のフィルタとしての役目を果たす
ことを意図しています。もし\code{not}節がいくつかの束縛されていない変数を持つ
フレームと処理された場合(上記の例における\code{?x}が行うように)、システムは
予想外の結果を生成します。同様の問題が\code{lisp\-/value}の使用でも起こります。
Lispの述語はその引数のいくつかが未束縛な場合働くことができません。
\link{Exercise 4.77}を参照して下さい。

クエリ言語の\code{not}が記号論理学の\code{not}と異なるずっと深刻な部分があります。
論理学では命題``not \( P \)''を\( P \)は真ではないことを意味すると解釈します。
しかし、クエリシステムでは``not \( P \)''は\( P \)がデータベース内の知識から
推論不可能であることを意味しています。例えば、\link{Section 4.4.1}の社員情報データベース
を与えられた場合、システムは幸いにも全ての種類の\code{not}命令を推論することができるでしょう。
例えばBen Bitdiddleは野球のファンではない、外で雨は振っていない、2 + 2 は4ではないなどです。\footnote{クエリ
\code{(not (baseball\-/fan (Bitdiddle Ben)))}について考えてみましょう。システムは
データベースに\code{(baseball\-/fan (Bitdiddle Ben))}が無いことを知り、そのため空フレームは
パターンを満たさず初期値のフレームのストリームから取り除かれません。クエリの結果は
従って空フレームであり、これが入力クエリのインスタンス化に用いられ、
\code{(not (baseball\-/fan (Bitdiddle Ben)))}が生成されます。}

\begin{quote}
\heading{\phantomsection\label{Exercise 4.64}Exercise 4.64:}
Louis Reasonerは誤って\code{outranked\-/by}ルール(\link{Section 4.4.1})をデータベースから削除して
しまった。彼はこのことに気付いた時、直ぐに再インストールした。残念なことに、
彼はルールにわずかな変更を行い、以下のように入力した。

\begin{scheme}
(rule (outranked-by ?staff-person ?boss)
      (or (supervisor ?staff-person ?boss)
          (and (outranked-by ?middle-manager ?boss)
               (supervisor ?staff-person 
                           ?middle-manager))))
\end{scheme}


Louisがこの情報をシステムに入力して直ぐに、DeWitt AullがやってきてBen Bitdiddleの
上司は誰かを調べようとした。彼は以下のクエリを入力した。

\begin{scheme}
(outranked-by (Bitdiddle Ben) ?who)
\end{scheme}


回答を行った後、システムは無限ループへと陥った。何故であるか、説明せよ。
\end{quote}

\begin{quote}
\heading{\phantomsection\label{Exercise 4.65}Exercise 4.65:}
組織内での昇進の日を待ち望んでいるCy D. Fectは全ての重役を探すクエリを入力
してみた(\link{Section 4.4.1}のルール\code{wheel}を用いた)。

\begin{scheme}
(wheel ?who)
\end{scheme}


驚いたことにシステムは以下の内容を応答した。

\begin{scheme}
~\textit{;;; Query results:}~
(wheel (Warbucks Oliver))
(wheel (Bitdiddle Ben))
(wheel (Warbucks Oliver))
(wheel (Warbucks Oliver))
(wheel (Warbucks Oliver))
\end{scheme}

何故、Oliver Warbucksは4度表示されたのか?
\end{quote}

\begin{quote}
\heading{\phantomsection\label{Exercise 4.66}Exercise 4.66:}
Benはクエリシステムを一般化し会社に関する統計を提供する。例えば、全ての
コンピュータプログラマの給料の合計を求めるためには、以下のように入力することが
できるだろう。

\begin{scheme}
(sum ?amount (and (job ?x (computer programmer))
                  (salary ?x ?amount)))
\end{scheme}


全般に、Benの新しいシステムは以下の形式の式を可能にする。

\begin{scheme}
(accumulation-function ~\( \dark \langle \)~~\var{\dark variable}~~\( \dark \rangle \)~ ~\( \dark \langle \)~~\var{\dark query pattern}~~\( \dark \rangle \)~)
\end{scheme}

\noindent
ここで\code{accumulation\-/function}は\code{sum}, \code{average}, または\code{maximum}のような物である。
Benはこれを実装するのは簡単なはずだと考えた。単純にクエリパターンを\code{qeval}に追加するだろう。
これはフレームのストリームを生成するだろう。すると彼はこのストリームをmap関数を通すことで
ストリーム内の各フレームから指定した変数の値を抽出し、結果の値のストリームをaccumulation(集積)
関数へと与えるだろう。Benが実装を完成し、丁度試験を行おうとした時にCyが依然として\link{Exercise 4.65}の
\code{wheel}クエリの結果に悩みながら歩いてきた。CyがBenにシステムの応答を見せた時、Benはうなってから
``なんてこった。私の簡単な集積の仕組みは動かない!''と述べた。


Benは何に気付いたのか? この状況を救い出すため用いられる手段の要点を述べよ。
\end{quote}

\begin{quote}
\heading{\phantomsection\label{Exercise 4.67}Exercise 4.67:}
クエリシステムにループ検知器をインストールし、テキストと\link{Exercise 4.64}で
説明されたような単純なループを防ぐための手段を工夫せよ。
一般的なアイデアは、システムに現在の推論の連鎖のある種の履歴を管理させ、既に取り組んでいる
クエリの処理を始めないようにすることである。どのような種類の情報(パターンとフレーム)が
この履歴に含まれるか、そしてどのように検査が行われるべきかについて説明せよ。(\link{Section 4.4.4}に
おけるクエリシステムの実装の詳細を学んだ後に、あなたはシステムを変更してループ検知器を
入れたいと思うだろう)。
\end{quote}

\begin{quote}
\heading{\phantomsection\label{Exercise 4.68}Exercise 4.68:}
\link{Exercise 2.18}の\code{reverse}命令を実装するルールを定義せよ。これは与えられたリストの
逆順で同じ要素を含むリストを返す。(ヒント:\code{append\-/to\-/form}を使用せよ)。あなたの
ルールは\code{(reverse (1 2 3) ?x)}と\code{(reverse ?x (1 2 3))}の両方に回答することができるだろうか?
\end{quote}

\begin{quote}
\heading{\phantomsection\label{Exercise 4.69}Exercise 4.69:}
\link{Exercise 4.63}で策定したデータベースとルールから始めて、孫の関係に``great''を
追加するためのルールを工夫せよ。これはシステムに対しIradがAdamのgreat-grandson(ひ孫)である
こと、またJabalとJubalがgreat-great-great-great-great-grandsons(ひひひひひ孫)であることを
推論することを可能にしなければならない。(ヒント:例えばIradに関する事実を
\code{((great grandson) Adam Irad)}として表現する。リストの終端が単語\code{grandson}であるかを
決定するルールを書け。これを用いて\code{?rel}が\code{grandson}で終わるリストである場合に、
関係\code{((great .  ?rel) ?x ?y)}を導き出すことが可能なルールを表現せよ)。
あなたのルールを\code{((great grandson) ?g ?ggs)}と\code{(?relationship Adam Irad)}のような
クエリを用いて確認せよ。
\end{quote}

\subsection{クエリシステムの実装}
\label{Sec. 4.4.4}
\label{Section 4.4.4}


\link{Section 4.4.2}はどのようにクエリシステムが働くかについて説明した。ここでは完全なシステムの実装
を公開することにより詳細を知らせる。



\subsubsection{ドライバループとインスタンス化}
\label{Section 4.4.4.1}


クエリシステムのためのドライバループは繰り返し入力式を読み込みます。もし式が追加されるべき
ルールかアサーションであるのならばその情報が追加されます。そうでなければ式はクエリであると
見做されます。ドライバはこのクエリを評価機\code{qeval}に単一の空のフレームから成る初期フレーム
ストリームと共に渡されます。評価の結果はクエリをデータベース内で見つかった変数の値で
満たすことにより生成されたフレームのストリームです。これらのフレームは、フレームのストリームに
より提供された値を用いて変数がインスタンス化された元のクエリのコピーから
成る新しいストリームを形成するのに用いられます。そしてこの最終的なストリームが
端末に表示されます。

\begin{scheme}
(define input-prompt  ";;; Query input:")
(define output-prompt ";;; Query results:")

(define (query-driver-loop)
  (prompt-for-input input-prompt)
  (let ((q (query-syntax-process (read))))
    (cond ((assertion-to-be-added? q)
           (add-rule-or-assertion! (add-assertion-body q))
           (newline)
           (display "Assertion added to data base.")
           (query-driver-loop))
          (else
           (newline)
           (display output-prompt)
           (display-stream
            (stream-map
             (lambda (frame)
               (instantiate
                q
                frame
                (lambda (v f)
                  (contract-question-mark v))))
             (qeval q (singleton-stream '()))))
           (query-driver-loop)))))
\end{scheme}

\noindent
ここで、この章の他の評価機と同様に、クエリ言語の式に対して抽象構文を用います。
式の構文の実装は述語\code{assertion\-/to\-/be\-/added?}とセレクタ\code{add\-/assertion\-/body}を
含めて、\link{Section 4.4.4.7}にて与えられます。\code{add\-/rule\-/or\-/assertion!}は\link{Section 4.4.4.5}で
定義されます。


入力式のどんな処理を行う前にも、ドライバループは処理をより効率的にする形式へと
構文的に変換します。これはパターン変数の表現の変更を含みます。クエリが初期化される
時、未束縛である任意の変数は表示される前に入力時の表現に戻されます。これらの
変換は2つの手続、\code{query\-/syntax\-/process}と\code{contract\-/question\-/mark}により実行されます(\link{Section 4.4.4.7})。


式をインスタンス化するためにはまずコピーを行い、式中の全ての変数を与えられたフレーム内の
それらの値にて置き換えます。値はそれら自身がインスタンス化されます。それらが変数を含む
可能性があるためです(例えば、式の中の\code{?x}がユニフィケーションの結果として\code{?y}に
束縛され、\code{?y}が同様に5に束縛されている場合)。変数がインスタンス化できない場合に取るべき
行動は手続\code{instantiate}の引数に渡されます。

\begin{scheme}
(define (instantiate exp frame unbound-var-handler)
  (define (copy exp)
    (cond ((var? exp)
           (let ((binding (binding-in-frame exp frame)))
             (if binding
                 (copy (binding-value binding))
                 (unbound-var-handler exp frame))))
          ((pair? exp)
           (cons (copy (car exp)) (copy (cdr exp))))
          (else exp)))
  (copy exp))
\end{scheme}

\noindent
束縛を操作する手続は\link{Section 4.4.4.8}で定義されます。

\subsubsection{評価機}
\label{Section 4.4.4.2}


\code{query\-/driver\-/loop}により呼ばれる\code{qeval}手続はクエリシステムの基本的な評価機です。
入力としてクエリとフレームのストリームを取り、拡張されたフレームのストリームを返します。
\link{Chapter 2}で総称的な命令を実装したのと同様に、
\code{get}と\code{put}を用いたデータ適従による呼出により特殊形式を判別します。
特殊形式とは判別されない任意のクエリは単純なクエリと見做され\code{simple\-/query}により処理されます。

\begin{scheme}
(define (qeval query frame-stream)
  (let ((qproc (get (type query) 'qeval)))
    (if qproc
        (qproc (contents query) frame-stream)
        (simple-query query frame-stream))))
\end{scheme}

\noindent
\code{type}と\code{contents}は\link{Section 4.4.4.7}で定義され、特殊形式の抽象構文を実装します。

\subsubsection*{単純なクエリ}


\code{simple\-/query}手続は単純なクエリを扱います。引数として単純なクエリ(パターン)をフレームの
ストリームと共に取り、クエリのデータベースへの適合全てにより各フレームを拡張することにより
形成されたストリームを返します。

\begin{scheme}
(define (simple-query query-pattern frame-stream)
  (stream-flatmap
   (lambda (frame)
     (stream-append-delayed
      (find-assertions query-pattern frame)
      (delay (apply-rules query-pattern frame))))
   frame-stream))
\end{scheme}

\noindent
入力ストリーム中の各フレームに対し、\code{find\-/assertions}(\link{Section 4.4.4.3})を用いて
データベース内の全てのアサーションに対してパターンを適合し、拡張フレームの
ストリームを生成します。そして\code{apply\-/rules}(\link{Section 4.4.4.4})を用いて全ての
可能なルールを適用し、拡張フレームのもう1つのストリームを生成します。
これらの2つのストリームは(\code{stream\-/append\-/delayed}(\link{Section 4.4.4.6})を用いて)
接続され、与えられたパターンが元のフレームに一致して満たされることが可能な
全ての手段でストリームを作ります(\link{Exercise 4.71}参照)。個別の入力フレームに
対するストリームは\code{stream\-/flatmap} (\link{Section 4.4.4.6})を用いて接続され、
元の入力ストリーム内の任意のフレームが与えられたパターンを用いて
適合を生成するために拡張されることができる全ての手段により、1つの
巨大なストリームが形成されます。

\subsubsection*{複合クエリ}

\code{and}クエリは\link{Figure 4.5}にて説明されているように\code{conjoin}手続により
扱われます。\code{conjoin}は入力として結合(conjuncts)とフレームのストリームを
取り、拡張されたフレームのストリームを返します。最初に\code{conjoin}はフレームの
ストリームを処理し、結合内の最初のクエリを満たす全ての可能なフレームの拡張の
ストリームを探します。次に、これを新しいフレームのストリームとして用いて、
再帰的にクエリの残りに対して\code{conjoin}を適用します。

\begin{scheme}
(define (conjoin conjuncts frame-stream)
  (if (empty-conjunction? conjuncts)
      frame-stream
      (conjoin (rest-conjuncts conjuncts)
               (qeval (first-conjunct conjuncts)
                      frame-stream))))
\end{scheme}

\noindent
以下の式は

\begin{scheme}
(put 'and 'qeval conjoin)
\end{scheme}

\noindent
\code{qeval}に対し、\code{and}の型に遭遇した場合に\code{conjoin}を呼び出すように設定します。

\code{or}クエリも同様に、\link{Figure 4.6}に示されるように扱われます。
\code{or}の多様な選言肢に対する出力ストリームは別々に求められ、\link{Section 4.4.4.6}の
\code{interleave\-/delayed}手続を用いて結合されます。(\link{Exercise 4.71}と\link{Exercise 4.72}を参照)

\begin{scheme}
(define (disjoin disjuncts frame-stream)
  (if (empty-disjunction? disjuncts)
      the-empty-stream
      (interleave-delayed
       (qeval (first-disjunct disjuncts)
              frame-stream)
       (delay (disjoin (rest-disjuncts disjuncts)
                       frame-stream)))))
(put 'or 'qeval disjoin)
\end{scheme}

\noindent
論理積(conjuncts)と論理和(disjuncts)の構文のための述語とセレクタは\link{Section 4.4.4.7}で
提供されます。

\subsubsection*{フィルタ}


\code{not}は\link{Section 4.4.2}にて概説された手法により扱われます。入力ストリーム内の各フレームを
否定されたクエリを満たすように拡張することを試みます。そして拡張できない場合にのみ
出力ストリームに与えられたフレームを含めます。

\begin{scheme}
(define (negate operands frame-stream)
  (stream-flatmap
   (lambda (frame)
     (if (stream-null?
          (qeval (negated-query operands)
                 (singleton-stream frame)))
         (singleton-stream frame)
         the-empty-stream))
   frame-stream))
(put 'not 'qeval negate)
\end{scheme}

\noindent
\code{lisp\-/value}は\code{not}に似たフィルタです。ストリーム内の各フレームはパターン内の
変数をインスタンス化するために用いられ、指定された述語が適用され、述語が偽を返した
フレームは入力ストリームから取り除かれます。未束縛なパターン変数が存在する場合には
結果はエラーとなります。

\begin{scheme}
(define (lisp-value call frame-stream)
  (stream-flatmap
   (lambda (frame)
     (if (execute
          (instantiate
           call
           frame
           (lambda (v f)
             (error "Unknown pat var: LISP-VALUE"
                    v))))
         (singleton-stream frame)
         the-empty-stream))
   frame-stream))
(put 'lisp-value 'qeval lisp-value)
\end{scheme}

\noindent
\code{execute}は述語を引数に適用しますが、述語式を評価し適用する手続を得なければなりません。
しかし引数は評価してはいけません。なぜならそれらは既に実際の引数であり、その(Lispにおける)
評価が引数を生成する式ではないためです。\code{execute}が基礎を成すLispシステムの
\code{eval}と\code{apply}を使用して実装されていることに注意して下さい。

\begin{scheme}
(define (execute exp)
  (apply (eval (predicate exp)
               user-initial-environment)
         (args exp)))
\end{scheme}

\noindent
特殊形式\code{always\-/true}はクエリに対し常に満たされた状態を与えます。これはその
中身(通常は空)を無視し、単純に入力ストリームの全てのフレームを通します。
\code{always\-/true}は\code{rule\-/body}セレクタ(\link{Section 4.4.4.7})により利用され、ボディ成しで
定義されたルールに対しボディを提供します。(言い換えれば、その結果部分が常に
満たされます。)

\begin{scheme}
(define (always-true ignore frame-stream) frame-stream)
(put 'always-true 'qeval always-true)
\end{scheme}

\noindent
\code{not}と\code{lisp\-/value}の構文を定義するセレクタは\link{Section 4.4.4.7}で提供されます。

\subsubsection{パターンマッチングによりアサーションを見つける}
\label{Sec. 4.4.4.3}
\label{Section 4.4.4.3}

\code{find\-/assertions}は\code{simple\-/query} (\link{Section 4.4.4.2})により呼ばれ、
入力としてパターンとフレームを取ります。フレームのストリームを返し、各フレームは
与えられた物を与えらえたパターンへのデータベースの適合により拡張されています。
\code{fetch\-/assertions} (\link{Section 4.4.4.5})を用いてデータベース内の全てのアサーションの
ストリームを得ます。これはパターンとフレームに対して適合するか確認されなければなりません。
ここで\code{fetch\-/assertions}する理由は、私達は良く簡単なテストをここで適用するためです。
このテストは適合を成功する候補のプールからデータベース内のエントリを数多く削減することが
できます。システムは例え\code{fetch\-/assertions}を削除して単純にデータベース内の全ての
アサーションのストリームを確認するだけでも動くでしょう。しかし演算は効率的ではなくなります。
より多くのマッチャに対する呼出を行わねばならなくなるためです。

\begin{scheme}
(define (find-assertions pattern frame)
  (stream-flatmap 
    (lambda (datum) 
      (check-an-assertion datum pattern frame))
    (fetch-assertions pattern frame)))
\end{scheme}

\noindent
\code{check\-/an\-/assertion}は引数としてパターン、データオブジェクト(アサーション)、フレームを取り、
拡張されたフレームを含む1要素のストリームか、適合を失敗した場合に\code{the\-/empty\-/stream}を返します。

\begin{scheme}
(define (check-an-assertion
         assertion query-pat query-frame)
  (let ((match-result
         (pattern-match query-pat assertion query-frame)))
    (if (eq? match-result 'failed)
        the-empty-stream
        (singleton-stream match-result))))
\end{scheme}

\noindent
基本的なパターンマッチャはシンボル\code{failed}か、与えられたフレームの拡張を返します。
マッチャの基本的な考えはパターンをデータに対して要素毎に確認し、パターン変数に対する
束縛を集積します。もしパターンとデータオブジェクトが同じであるなら、適合は成功しそこまで
集積された束縛のフレームを返します。そうでなければ、もしパターンが変数ならば、変数を
データに対して束縛することで現在のフレームを拡張することをフレーム内に既に存在する
束縛に一致するまで行います。もしパターンとデータの両方がペアであるなら、(再帰的に)パターンの
\code{car}をデータの\code{car}に対して適合を行いフレームを生成します。次にこのフレームの中で
パターンの\code{cdr}をデータのに対して適合を行います。もしこれらの場合全てが当て嵌らない
場合、適合は失敗し、シンボル\code{failed}を返します。

\begin{scheme}
(define (pattern-match pat dat frame)
  (cond ((eq? frame 'failed) 'failed)
        ((equal? pat dat) frame)
        ((var? pat) (extend-if-consistent pat dat frame))
        ((and (pair? pat) (pair? dat))
         (pattern-match 
          (cdr pat)
          (cdr dat)
          (pattern-match (car pat) (car dat) frame)))
        (else 'failed)))
\end{scheme}

\noindent
次が、フレーム内に既に存在している束縛に一致するなら、新しい束縛を追加することによりフレームを拡張する手続です。

\begin{scheme}
(define (extend-if-consistent var dat frame)
  (let ((binding (binding-in-frame var frame)))
    (if binding
        (pattern-match
         (binding-value binding) dat frame)
        (extend var dat frame))))
\end{scheme}

\noindent
もしフレーム内の変数に対する束縛が無い場合、単純に変数のデータに対する束縛を追加します。
そうでなければこのフレーム内で、データをフレーム内の変数の値に対して適合を行います。
もし格納されていた値が定数のみを持つならば、つまり\code{extend\-/if\-/consistent}により
パターンマッチングの間に格納されたのであれば、適合は単純に格納されていた値と新しい
値が同じであるかどうかを確認します。もしそうならば、フレームを変更せずに返します。
そうでないならば、失敗を示す印を返します。しかし格納されたいた値は、それがユニフィケーションの
間に格納されたのであればパターン変数を含む場合があります(\link{Section 4.4.4.4}参照)。
格納されたパターンの新しいデータに対する再帰的な適合はこのパターン内の変数に対する
束縛の追加、または確認を行います。例えば、\code{?x}が\code{(f ?y)}に束縛され\code{?y}が未束縛である
フレームを持っているとしましょう。そしてこのフレームを\code{?x}の\code{(f b)}への束縛で
拡大させたいとします。私達は\code{?x}を探し、それが\code{(f ?y)}に束縛されているのを見つけます。
このことがこの同じフレームの中で提案された新しい値\code{(f b)}に対して\code{(f ?y)}を適合させる
ことへと導きます。最終的に、この適合は\code{?y}から\code{b}への束縛を追加する
ことによりこのフレームを拡張します。\code{?x}は\code{(f ?y)}への束縛を維持します。
格納されていた束縛を変更することはありません。また与えられた変数に対して複数の
束縛を格納することもありません。


\code{extend\-/if\-/consistent}により使用される束縛を操作するための複数の手続は\link{Section 4.4.4.8}で
定義されます。

\subsubsection*{末尾ドット付きパターン}
\label{Sec. 4.4.4.4}
\label{Section 4.4.4.4}

パターンがドットとそれに続くパターン変数を含む場合、そのパターン変数はデータリストの
(次の要素ではなく)残りに適合します。誰かが予想するように\link{Exercise 2.20}にて
説明されたドット付き末尾記述と同様です。私達が実装したばかりのパターンマッチャは
ドットを探しませんが、私達が望むとおりに振舞います。これは\code{query\-/driver\-/loop}で
用いられるLispの\code{read}プリミティブがクエリを読み込みリスト構造として表現する時に
ドットを特別な方法で扱うためです。

\code{read}がドットを見た時、次の項目をリストの次の要素にするのではなく
(\code{cons}の\code{car}のこと、\code{cdr}はリストの残り)、リスト構造の\code{cdr}を
次の項目にします。例えば、パターン\code{(computer ?type)}に対する\code{read}により
生成されるリスト構造は式\code{(cons 'computer (cons '?type '()))}を評価することにより
構築されます。またパターン\code{(computer . ?type)}に対する場合は式
\code{(cons 'computer '?type)}を評価することにより構築されます。

従って\code{pattern\-/match}が再帰的にデータリストとドットを持つパターンの\code{car}と\code{cdr}を
比較するにつれ、最終的にはドットの後ろの変数(パターンの\code{cdr})がデータリストの
部分リストに対して適合され、そのリストに対してその変数が束縛されます。例えば、
パターン\code{(computer . ?type)}を\code{(programmer trainee)}に適合することは
\code{?type}をリスト\code{(programmer trainee)}に適合させます。

\subsubsection{ルールとユニフィケーション}

\code{apply\-/rules}は\code{find\-/assertions}の類似のルールです(\link{Section 4.4.4.3})。
入力としてパターンとフレームを取り、データベースからルールを適用することにより
拡張フレームのストリームを形成します。\code{stream\-/flatmap}は\code{apply\-/a\-/rule}を
(\code{fetch\-/rules}により選択された(\link{Section 4.4.4.5}))恐らく適用可能な
ルールのストリームに対しmapし、結果のフレームのストリーム群を結合します。

\begin{scheme}
(define (apply-rules pattern frame)
  (stream-flatmap (lambda (rule)
                    (apply-a-rule rule pattern frame))
                  (fetch-rules pattern frame)))
\end{scheme}

\noindent
\code{apply\-/a\-/rule}は\link{Section 4.4.2}で概説された手法を用いてルールを適用します。
最初にルールの結論を与えられたフレーム内のパターンとユニフィケーションを行うことで
引数フレームを増大させます。これが成功したならこの新しいフレーム内でルールのボディを
評価します。

しかしこの全てが起こる前に、プログラムはルール内の全ての変数を個別の新しい名前に
変更します。この理由は異なるルールの適用に対する変数が御互いに混同されることを
防ぐためです。例えば、もし2つのルールの両方が\code{?x}と名付けられた変数を用いる場合、
それぞれが適用された時に\code{?x}に対する束縛をフレームに追加するかもしれません。
これら2つの\code{?x}は御互いに関係がありません。そして私達は2つの束縛が一致するはずだと
考えるように惑わされてはいけません。変数名を変えるのでなく、より賢い環境構造を
工夫することもできるでしょう。しかし、私達がここで選択した改名による取り組み方は
最も効率的ではないとしても、最も簡単です(\link{Exercise 4.79}参照)。
以下が\code{apply\-/a\-/rule}手続です。

\begin{scheme}
(define (apply-a-rule rule query-pattern query-frame)
  (let ((clean-rule (rename-variables-in rule)))
    (let ((unify-result
           (unify-match query-pattern
                        (conclusion clean-rule)
                        query-frame)))
      (if (eq? unify-result 'failed)
          the-empty-stream
          (qeval (rule-body clean-rule)
                 (singleton-stream unify-result))))))
\end{scheme}

\noindent
セレクタ\code{rule\-/body}と\code{conclusion}はルールの部分を抜き出します。
これは\link{Section 4.4.4.7}で定義されます。

私達はユニークな(unique、独自の)識別子(例えば番号)を各ルールの適用に関連付けし、この識別子を元の
変数名に接続することで、ユニークな変数名を生成します。例えば、もしルール適用識別子が7なら、
ルール内の各\code{?x}を\code{?x\-/7}に、各\code{?y}を\code{?y\-/7}に変更するでしょう。
(\code{make\-/new\-/variable}と\code{new\-/rule\-/application\-/id}は\link{Section 4.4.4.7}の構文手続に含まれます。)

\begin{scheme}
(define (rename-variables-in rule)
  (let ((rule-application-id (new-rule-application-id)))
    (define (tree-walk exp)
      (cond ((var? exp)
             (make-new-variable
              exp rule-application-id))
            ((pair? exp)
             (cons (tree-walk (car exp))
                   (tree-walk (cdr exp))))
            (else exp)))
    (tree-walk rule)))
\end{scheme}

\noindent
ユニフィケーションアルゴリズムは手続として実装され、入力として2つのパターンとフレームを取り、
拡張されたフレームかシンボル\code{failed}を返します。ユニファイアはパターンマッチャに似ていますが、
対照的であることが異なります---つまり、変数が適合の両サイドに存在することが許されます。
\code{unify\-/match}は基本的には\code{pattern\-/match}と同じですが、(以下で``\code{***}''のマークを付けた)
拡張コードの存在が異なります。これは適合の右側のオブジェクトが変数である場合を扱います。

\begin{scheme}
(define (unify-match p1 p2 frame)
  (cond ((eq? frame 'failed) 'failed)
        ((equal? p1 p2) frame)
        ((var? p1) (extend-if-possible p1 p2 frame))
        ((var? p2) (extend-if-possible p2 p1 frame))  ~\textrm{; ***}~
        ((and (pair? p1) (pair? p2))
         (unify-match (cdr p1)
                      (cdr p2)
                      (unify-match (car p1)
                                   (car p2)
                                   frame)))
        (else 'failed)))
\end{scheme}

\noindent
ユニフィケーションにおいては一方向マッチングのように、既存の束縛に一致する場合のみ
提案されたフレームの拡張を受け入れたいです。手続\code{extend\-/if\-/possible}はユニフィケーションに
おいて使用され、パターンマッチにて利用される\code{extend\-/if\-/consistent}と同じですが、
下記のプログラムで``\code{***}''がマークされている、2つの特別なチェックが異なります。
最初のケースでは、もし適合を試す変数が未束縛であり、かつそれに対して適合させようと
している値それ自体が(異なる)変数である場合に、その値が束縛されているかを確認する
必要があります。そしてもしそうであれば、その値を適合する必要があります。
もし適合の両側が共に未束縛である場合、それぞれを御互いに束縛します。

2つ目のチェックは変数を、変数を含むパターンに対して束縛する試みを取り扱います。
そのような状況は変数が両方のパターン内で繰り返される場合に常に起こり得ます。
例えば2つのパターン、\code{(?x ?x)}と\code{(?y <\var{\code{?y}を含む式}>)}を、
\code{?x}と\code{?y}の両方が未束縛である場合のフレーム内にてユニフィケーションを
行う場合について考えてみて下さい。最初の\code{?x}は\code{?y}に対して適合し、\code{?x}から\code{?y}への
束縛を作成します。次に同じ\code{?x}が与えられた\code{?y}を含む式に対して適合されます。
\code{?x}は既に\code{?y}に対して束縛されているため、これは結果として\code{?y}をその式に対して
適合することになります。もし私達がユニファイアを2つのパターンを同じにするパターン変数に
対する値の集合を見つけるものとして考えているなら、これらのパターンは\code{?y}が\code{?y}を
含む式に等しいような\code{?y}を見付ける命令を暗示します。そのような方程式を解く一般的な
手法は存在しませんので、私達はそのような束縛を却下します。このような場合が述語\code{depends\-/on?}に
より認識されます。\footnote{
一般的に、\code{?y}を\code{?y}を含む式にユニフィケーションを行う場合には、
方程式\code{?y} = <\var{expression involving \code{?y}}>の不動点を見つけられなければなりません。
偶に解が存在する式を構文的に形成することが可能です。例えば、\code{?y} = \code{(f ?y)}は不動点
\code{(f (f (f \( \dots \) )))}を持つように見えます。これは式\code{(f ?y)}で始め、繰り返し\code{?y}を
\code{(f ?y)}で置き換えることで生成できます。残念ながら全てのそのような方程式が意味のある
不動点を持つわけではありません。ここで浮かび上がる問題は数学で無限級数を扱う場合の
問題と似ています。例えば、私達は2が方程式\( y = 1 + y / 2 \)の解であることを知っています。
式\( 1 + y / 2 \)で始めて、繰り返し\( y \)を\( 1 + y / 2 \)で置き換えていくと
以下の様になります。
\begin{comment}
\begin{example}
2 = y = 1 + y/2 = 1 + (1 + y/2)/2 = 1 + 1/2 + y/4 = ...
\end{example}
\end{comment}
\begin{displaymath}
 2 = y = 1 + {y \over 2} = 1 + {1\over2}\left(1 + {y \over 2}\right) = 
	1 + {1\over2} + {y \over 4} = \dots , 
\end{displaymath}
\noindent
これは以下の式へと導きます。
\begin{comment}

\begin{example}
2 = 1 + 1/2 + 1/4 + 1/8 + ...
\end{example}

\end{comment}
\begin{displaymath}
 2 = 1 + {1\over2} + {1\over4} + {1\over8} + \dots. 
\end{displaymath}
\noindent
しかし、もし同じ操作を式\( y = 1 + 2y \)の解が-1であるという結果から始めると、
\begin{comment}

\begin{example}
-1 = y = 1 + 2y = 1 + 2(1 + 2y) = 1 + 2 + 4y = ...
\end{example}

\end{comment}
\begin{displaymath}
 -1 = y = 1 + 2y = 1 + 2(1 + 2y) = 1 + 2 + 4y = \dots, 
\end{displaymath}
\noindent
これは以下の式へと導きます。
\begin{comment}

\begin{example}
-1 = 1 + 2 + 4 + 8 + ...
\end{example}

\end{comment}
\begin{displaymath}
 -1 = 1 + 2 + 4 + 8 + \dots. 
\end{displaymath}
\noindent
これらの2つの等式を導き出した形式的な操作は同一であるにも係らず、最初の結果は
無限級数に関して有効な正しい主張となりますが、2つ目はそうではありません。
同様に、私達のユニフィケーションの結果に対して無計画に構文に従い構築された
式はエラーへと繋るでしょう。
}


一方で、変数をそれ自体へ束縛する試みを拒否したくはありません。例として、
\code{(?x ?x)}と\code{(?y ?y)}のユニフィケーションについて考えてみましょう。
二度目の\code{?x}を\code{?y}へ束縛する試行は\code{?y}(\code{?x}の新しい値)に対する
\code{?y}(\code{?x}に格納された値)に適合します。これは\code{unify\-/match}の
\code{equal?}節により担当されます。

\begin{scheme}
(define (extend-if-possible var val frame)
  (let ((binding (binding-in-frame var frame)))
    (cond (binding
           (unify-match
            (binding-value binding) val frame))
          ((var? val)                      ~\textrm{; ***}~
           (let ((binding (binding-in-frame val frame)))
             (if binding
                 (unify-match
                  var (binding-value binding) frame)
                 (extend var val frame))))
          ((depends-on? val var frame)     ~\textrm{; ***}~
           'failed)
          (else (extend var val frame)))))
\end{scheme}

\noindent
\code{depends\-/on?}はパターン変数の値であると提案された式がその変数に依存するかを確認します。
これは現在のフレームと比較して行われなければなりません。式がテスト変数に依存する値を
既に持つ変数の存在を含むかもしれないためです。\code{depends\-/on?}の構造は簡単な再帰木の
探索であり、この中で必要な場合いつでも変数の値を置き換えます。

\begin{scheme}
(define (depends-on? exp var frame)
  (define (tree-walk e)
    (cond ((var? e)
           (if (equal? var e)
               true
               (let ((b (binding-in-frame e frame)))
                 (if b
                     (tree-walk (binding-value b))
                     false))))
          ((pair? e)
           (or (tree-walk (car e))
               (tree-walk (cdr e))))
          (else false)))
  (tree-walk exp))
\end{scheme}

\subsubsection{データベースの保守}
\label{Sec. 4.4.4.5}
\label{Section 4.4.4.5}

論理プログラミング言語の設計における重要な問題の1つは、与えられたパターンの確認に
おいてできる限り少ないデータベースのエントリが検査されるように物事を準備することです。
私達のシステムでは、全てのアサーションを1つの大きなストリームに格納することに加えて、
\code{car}が静的なシンボルである全てのアサーションをそのシンボルで索引付けられたテーブル内の
分離されたストリームに格納します。パターンに適合するかもしれないアサーションを取り出すためには、
最初にパターンの\code{car}が静的なシンボルであるかを確認します。
もしそうならば、(マッチャを用いて確認するため)同じ\code{car}を持つ全ての格納された
アサーションを返します。もしパターンの\code{car}が静的なシンボルでない場合には、格納された
アサーションを全て返します。より賢い方法ではフレーム内の情報も活用するか、パターンの
\code{car}が静的なシンボルでない場合にも最適化を行うことに挑むことができるでしょう。
私達は検索作成の基準(\code{car}を用いる、静的シンボルの場合のみを扱う)をこのプログラムの中に
構築することを避けました。その代わりに私達の基準を具現する述語とセレクタを呼び出します。

\begin{scheme}
(define THE-ASSERTIONS the-empty-stream)
(define (fetch-assertions pattern frame)
  (if (use-index? pattern)
      (get-indexed-assertions pattern)
      (get-all-assertions)))
(define (get-all-assertions) THE-ASSERTIONS)
(define (get-indexed-assertions pattern)
  (get-stream (index-key-of pattern) 'assertion-stream))
\end{scheme}

\noindent
\code{get\-/stream}はテーブル内のストリームを探し、そこに何にも格納されていない場合には
空ストリームを返します。

\begin{scheme}
(define (get-stream key1 key2)
  (let ((s (get key1 key2)))
    (if s s the-empty-stream)))
\end{scheme}

\noindent
ルールも同様にルールの結論の\code{car}を用いて格納されます。しかしルールの結論は任意の
パターンであるため、変数を含められることがアサーションとは異なります。\code{car}が静的なシンボルである
パターンは結論が変数で始まるルールと、結論が同じ\code{car}を持つルールにも適合できます。
従って、\code{car}が静的なシンボルであるパターンに適合するかもしれないルールを取り出す場合、
結論が変数で始まる全てのルールと、結論がそのパターンと同じ\code{car}を持つルールを取り出します。
この目的のために、結論が変数で始まる全てのルールをテーブル内の分離されたストリームに、
シンボル\code{?}で索引付けして格納します。

\begin{scheme}
(define THE-RULES the-empty-stream)
(define (fetch-rules pattern frame)
  (if (use-index? pattern)
      (get-indexed-rules pattern)
      (get-all-rules)))
(define (get-all-rules) THE-RULES)
(define (get-indexed-rules pattern)
  (stream-append
   (get-stream (index-key-of pattern) 'rule-stream)
   (get-stream '? 'rule-stream)))
\end{scheme}

\noindent
\code{add\-/rule\-/or\-/assertion!}は\code{query\-/driver\-/loop}により使用されアサーションと
ルールとデータベースに追加します。各アイテムは適切であればインデックスに格納され、
データベース内の全てのアサーション、またはルールのストリームに格納されます。

\begin{scheme}
(define (add-rule-or-assertion! assertion)
  (if (rule? assertion)
      (add-rule! assertion)
      (add-assertion! assertion)))
(define (add-assertion! assertion)
  (store-assertion-in-index assertion)
  (let ((old-assertions THE-ASSERTIONS))
    (set! THE-ASSERTIONS
          (cons-stream assertion old-assertions))
    'ok))
(define (add-rule! rule)
  (store-rule-in-index rule)
  (let ((old-rules THE-RULES))
    (set! THE-RULES (cons-stream rule old-rules))
    'ok))
\end{scheme}

\noindent
実際にアサーション、またはルールを格納するためには、索引を付けられるかを確認します。
もしそうであれば、適切なストリームに格納します。

\begin{scheme}
(define (store-assertion-in-index assertion)
  (if (indexable? assertion)
      (let ((key (index-key-of assertion)))
        (let ((current-assertion-stream
               (get-stream key 'assertion-stream)))
          (put key
               'assertion-stream
               (cons-stream
                assertion
                current-assertion-stream))))))
(define (store-rule-in-index rule)
  (let ((pattern (conclusion rule)))
    (if (indexable? pattern)
        (let ((key (index-key-of pattern)))
          (let ((current-rule-stream
                 (get-stream key 'rule-stream)))
            (put key
                 'rule-stream
                 (cons-stream rule
                              current-rule-stream)))))))
\end{scheme}

\noindent
以下の手続はデータベースのインデックス(索引)がどのように使用されかについて定義します。
パターン(アサーション、またはルールの結論)が変数、または静的なシンボルで始まる場合に
テーブルに格納されます。

\begin{scheme}
(define (indexable? pat)
  (or (constant-symbol? (car pat))
      (var? (car pat))))
\end{scheme}

\noindent
パターンがその下に格納されるテーブル内のキーは(変数で始まる場合には)\code{?}、または
パターンの始めの静的なシンボルです。

\begin{scheme}
(define (index-key-of pat)
  (let ((key (car pat)))
    (if (var? key) '? key)))
\end{scheme}

\noindent
インデックスはパターンが静的なシンボルで始まる場合、パターンにマッチするかもしれない
アイテムを取得するために利用されます。

\begin{scheme}
(define (use-index? pat) (constant-symbol? (car pat)))
\end{scheme}

\begin{quote}
\heading{\phantomsection\label{Exercise 4.70}Exercise 4.70:}
手続\code{add\-/assertion!}と\code{add\-/rule!}内の\code{let}の束縛の目的は何か?
以下の\code{add\-/assertion!}の実装の誤りは何か? ヒント:\link{Section 3.5.2}における1の
無限ストリームの定義を思い出せ:\code{(define ones (cons\-/stream 1 ones))}

\begin{scheme}
(define (add-assertion! assertion)
  (store-assertion-in-index assertion)
  (set! THE-ASSERTIONS
        (cons-stream assertion THE-ASSERTIONS))
  'ok)
\end{scheme}
\end{quote}

\subsubsection{ストリーム命令}
\label{Section 4.4.4.6}

クエリシステムは\link{Chapter 3}には存在しなかったいくつかのストリーム命令を用います。


\code{stream\-/append\-/delayed}と\code{interleave\-/delayed}は\code{stream\-/append}と\\
\code{interleave}(\link{Section 3.5.3})と
同じですが、それらが(\link{Section 3.5.4}の\code{integral}の様に)遅延化された引数を取ることが異なります。
これはいくつかの場合においてループを先送りします。(\link{Exercise 4.71}参照)

\begin{scheme}
(define (stream-append-delayed s1 delayed-s2)
  (if (stream-null? s1)
      (force delayed-s2)
      (cons-stream
       (stream-car s1)
       (stream-append-delayed
        (stream-cdr s1)
        delayed-s2))))
(define (interleave-delayed s1 delayed-s2)
  (if (stream-null? s1)
      (force delayed-s2)
      (cons-stream
       (stream-car s1)
       (interleave-delayed
        (force delayed-s2)
        (delay (stream-cdr s1))))))
\end{scheme}

\noindent
\code{stream\-/flatmap}はクエリ評価機を通して使用され手続をフレームのストリーム上に対してmapし、
結果としての複数のフレームのストリームを接続します。
\code{stream\-/flatmap}は\link{Section 2.2.3}にて通常のリストのために導入された\code{flatmap}手続の
ストリーム向け類似品です。しかし通常の\code{flatmap}と異なり、単純にストリームをappendしていく
のではなく、相互配置処理により蓄積します。(\link{Exercise 4.72}と\link{Exercise 4.73}参照)
\begin{scheme}
(define (stream-flatmap proc s)
  (flatten-stream (stream-map proc s)))

(define (flatten-stream stream)
  (if (stream-null? stream)
      the-empty-stream
      (interleave-delayed
       (stream-car stream)
       (delay (flatten-stream (stream-cdr stream))))))
\end{scheme}

\noindent
評価機はまた以下の単純な手続を用いて単一要素から成るストリームを生成します。

\begin{scheme}
(define (singleton-stream x)
  (cons-stream x the-empty-stream))
\end{scheme}

\subsubsection{クエリ構文手続}
\label{Section 4.4.4.7}


\code{qeval}(\link{Section 4.4.4.2})により用いられる\code{type}と\code{contents}は、
特殊形式がその\code{car}に存在するシンボルにより判別されることを指示します。
これらは\link{Section 2.4.2}の\code{type\-/tag}と\code{contents}手続と同じですが、エラーメッセージが
異なります。

\begin{scheme}
(define (type exp)
  (if (pair? exp)
      (car exp)
      (error "Unknown expression TYPE" exp)))
(define (contents exp)
  (if (pair? exp)
      (cdr exp)
      (error "Unknown expression CONTENTS" exp)))
\end{scheme}

\noindent
以下の手続は\link{Section 4.4.4.1}の\code{query\-/driver\-/loop}にて使用されます。これはルールと
アサーションがデータベースに\code{(assert! <\var{rule\-/or\-/assertion}>)}の形式の式により
追加されることを指示します。

\begin{scheme}
(define (assertion-to-be-added? exp)
  (eq? (type exp) 'assert!))
(define (add-assertion-body exp) (car (contents exp)))
\end{scheme}

\noindent
以下は特殊形式\code{and}, \code{or}, \code{not}, \code{lisp\-/value}のための構文定義です。(\link{Section 4.4.4.2})

\begin{scheme}
(define (empty-conjunction? exps) (null? exps))
(define (first-conjunct exps) (car exps))
(define (rest-conjuncts exps) (cdr exps))
(define (empty-disjunction? exps) (null? exps))
(define (first-disjunct exps) (car exps))
(define (rest-disjuncts exps) (cdr exps))
(define (negated-query exps) (car exps))
(define (predicate exps) (car exps))
(define (args exps) (cdr exps))
\end{scheme}

\noindent
以下の3つの手続はルールの構文を定義します。

\begin{scheme}
(define (rule? statement)
  (tagged-list? statement 'rule))
(define (conclusion rule) (cadr rule))
(define (rule-body rule)
  (if (null? (cddr rule)) '(always-true) (caddr rule)))
\end{scheme}

\noindent
\code{query\-/driver\-/loop}(\link{Section 4.4.4.1})は\code{query\-/syntax\-/process}を呼び、
\code{?symbol}の形態を持つ式のパターン変数を内部形式\code{(? symbol)}に変形します。
これは言ってみれば、\code{(job ?x ?y)}のようなパターンが実際には内部的にシステムにより
\code{(job (? x) (? y))}と表現されているということです。これによりクエリ処理の効率が
良くなります。システムが式がパターン変数であるかを確認するのにシンボルから文字を
抽出する必要が無しに、式の\code{car}がシンボル\code{?}であるかどうかを確認することにより
確認できることを意味するためです。構文変形は以下の手続により達成されます。\footnote{
多くの
Lispシステムは通常の\code{read}手続を\newterm{reader macro characters}(\jnewterm{リーダ
マクロキャラクタ})を定義することにより変更し、そのような変形を実行させる能力を
ユーザにに与えています。クォートされた式は既にこのような取り扱われています。
リーダーは評価機が式を見る前に自動的に\code{'expression}を\code{(quote expression)}に
変形します。私達は同様に\code{?expression}が\code{(? expression)}に変形されるように
準備することも可能でした。しかし、明快さのために、私達はここに明示的に変形手続を
含めました。

\code{expand\-/question\-/mark}と\code{contract\-/question\-/mark}は名前に\code{string}を持つ
いくつかの手続を使用します。これらはSchemeのプリミティブです。}

\begin{scheme}
(define (query-syntax-process exp)
  (map-over-symbols expand-question-mark exp))
(define (map-over-symbols proc exp)
  (cond ((pair? exp)
         (cons (map-over-symbols proc (car exp))
               (map-over-symbols proc (cdr exp))))
        ((symbol? exp) (proc exp))
        (else exp)))
(define (expand-question-mark symbol)
  (let ((chars (symbol->string symbol)))
    (if (string=? (substring chars 0 1) "?")
        (list '?
              (string->symbol
               (substring chars 1 (string-length chars))))
        symbol)))
\end{scheme}

\noindent
一旦、変数がこのように変形されれば、パターン内の変数は\code{?}で始まるリストであり、
静的なシンボル(データベースの索引付けのために必要、\link{Section 4.4.4.5})はただのシンボルです。

\begin{scheme}
(define (var? exp) (tagged-list? exp '?))
(define (constant-symbol? exp) (symbol? exp))
\end{scheme}

\noindent
他とは異なる変数がルールの適用の間に以下の手続を用いて構築されます(\link{Section 4.4.4.4})。
ルール適用のための独自識別子は数値であり、ルールが適用される度にインクリメントされます。

\begin{scheme}
(define rule-counter 0)
(define (new-rule-application-id)
  (set! rule-counter (+ 1 rule-counter))
  rule-counter)
(define (make-new-variable var rule-application-id)
  (cons '? (cons rule-application-id (cdr var))))
\end{scheme}

\noindent
\code{query\-/driver\-/loop}が回答を表示するためにクエリをインスタンス化する時、
全ての未束縛のパターン変数を表示に適した形式に以下を用いて戻します。

\begin{scheme}
(define (contract-question-mark variable)
  (string->symbol
   (string-append "?"
     (if (number? (cadr variable))
         (string-append (symbol->string (caddr variable))
                        "-"
                        (number->string (cadr variable)))
         (symbol->string (cadr variable))))))
\end{scheme}

\subsubsection{フレームと束縛}
\label{Section 4.4.4.8}


フレームは変数と値のペアである束縛のリストとして表現されます。

\begin{scheme}
(define (make-binding variable value)
  (cons variable value))
(define (binding-variable binding) (car binding))
(define (binding-value binding) (cdr binding))
(define (binding-in-frame variable frame)
  (assoc variable frame))
(define (extend variable value frame)
  (cons (make-binding variable value) frame))
\end{scheme}

\begin{quote}
\heading{\phantomsection\label{Exercise 4.71}Exercise 4.71:}

Louis Reasonerはなぜ\code{simple\-/query}と\code{disjoin}の手続(\link{Section 4.4.4.2})は
以下のような定義ではなく、明示的な\code{delay}命令を用いて実装されたのか不思議だった。

\begin{scheme}
(define (simple-query query-pattern frame-stream)
  (stream-flatmap
   (lambda (frame)
     (stream-append
      (find-assertions query-pattern frame)
      (apply-rules query-pattern frame)))
   frame-stream))
(define (disjoin disjuncts frame-stream)
  (if (empty-disjunction? disjuncts)
      the-empty-stream
      (interleave
       (qeval (first-disjunct disjuncts)
              frame-stream)
       (disjoin (rest-disjuncts disjuncts)
                frame-stream))))
\end{scheme}

あなたはこれらのより簡単な定義を望まない振舞いへと導くクエリの例を与えることができるか?
\end{quote}

\begin{quote}
\heading{\phantomsection\label{Exercise 4.72}Exercise 4.72:}
なぜ\code{disjoin}と\code{stream\-/flatmap}は単純にそれらをappendせずに相互配置するのか?
なぜ相互配置のほうがより良く働くのかを説明する例を与えよ。(ヒント:なぜ私達は
\link{Section 3.5.3}において\code{interleave}を使用したのか?)
\end{quote}

\begin{quote}
\heading{\phantomsection\label{Exercise 4.73}Exercise 4.73:}
なぜ\code{flatten\-/stream}は明示的に\code{delay}を用いるのか?
以下のように定義した場合に何が間違っているのか?

\begin{scheme}
(define (flatten-stream stream)
  (if (stream-null? stream)
      the-empty-stream
      (interleave
       (stream-car stream)
       (flatten-stream (stream-cdr stream)))))
\end{scheme}
\end{quote}

\begin{quote}
\heading{\phantomsection\label{Exercise 4.74}Exercise 4.74:}
Alyssa P. Hackerはより簡単な版の\code{stream\-/flatmap}を\code{negate}, \code{lisp\-/value},
\code{find\-/assertions}の中で使用することを提案した。彼女はフレームのストリーム上に
mapされる手続はこれらの場合において常に空ストリームか、単一要素のストリームを
生成する。そのためこれらのストリームを接続する場合、相互配置する必要が無いと
気付いた。

\begin{enumerate}[a]

\item
Alyssaのプログラムに欠けている式を埋めよ。

\begin{scheme}
(define (simple-stream-flatmap proc s)
  (simple-flatten (stream-map proc s)))
(define (simple-flatten stream)
  (stream-map ~\( \dark \langle \)~~\var{\dark ??}~~\( \dark \rangle \)~
              (stream-filter ~\( \dark \langle \)~~\var{\dark ??}~~\( \dark \rangle \)~ stream)))
\end{scheme}

\item
クエリシステムの振舞はこのように変更した場合に変化するだろうか?

\end{enumerate}
\end{quote}

\begin{quote}
\heading{\phantomsection\label{Exercise 4.75}Exercise 4.75:}
クエリ言語に対して新しい特殊形式\code{unique}を実装せよ。\code{unique}は正確に
データベース内の1つの項目が指定されたクエリを満たす場合に成功しなければならない。
例えば、

\begin{scheme}
(unique (job ?x (computer wizard)))
\end{scheme}

\noindent
上の式は1つの項目のストリームを表示しなければならない。

\begin{scheme}
(unique (job (Bitdiddle Ben) (computer wizard)))
\end{scheme}

\noindent
Benはただ1人のコンピュータウィザードであるためである。
次に、

\begin{scheme}
(unique (job ?x (computer programmer)))
\end{scheme}

\noindent
上は空ストリームを表示しなければならない。複数のコンピュータプログラマが存在するためである。
さらに、

\begin{scheme}
(and (job ?x ?j) (unique (job ?anyone ?j)))
\end{scheme}

\noindent
上はただ1人により埋められた役職とその人達を全て表示しなければならない。


\code{unique}を実装するには2つの部分が存在する。1つ目はこの特殊形式を扱う手続を
書くことであり、2つ目は\code{qeval}にその手続を呼出させることである。2つ目の部分は
自明だ。\code{qeval}はその呼出をデータ適従の方法に従うためである。もしあなたの
手続が\code{uniquely\-/asserted}という名前であるなら、やらなければいけないことは以下である。

\begin{scheme}
(put 'unique 'qeval uniquely-asserted)
\end{scheme}

\noindent
これで\code{qeval}は型(\code{car})がシンボル\code{unique}である全てのクエリに対して
この手続を呼び出す。

真の問題は手続\code{uniquely\-/asserted}を書くことである。これは入力として\code{unique}クエリの
\code{contents}(\code{cdr})をフレームのストリームと共に受け取る。ストリームの各フレームに対し、
\code{qeval}を用いて与えられたクエリを満たすフレームの全ての拡張のストリームを見つけなければ
ならない。正確に1つのアイテムのみを持たないストリームは全て取り除かれなければならない。
残ったストリームは\code{unique}クエリの結果である1つの巨大なストリームに蓄積するために
戻されなければならない。これは特殊形式\code{not}の実装に似ている。

あなたの実装を正確に1人だけを監督する全ての人々を並べるクエリを形成することによりテストせよ。
\end{quote}

\begin{quote}
\heading{\phantomsection\label{Exercise 4.76}Exercise 4.76:}
一連のクエリの結合としての\code{and}の実装(\link{Figure 4.5})は洗練されているが非効率だ。
\code{and}の2つ目のクエリの処理において最初のクエリにより生成された各フレームに対して
データベースを走査しなければならないためである。もしデータベースが\( n \)個の要素を持ち、
典型的なクエリが\( n \)に比例した数(仮に\( n / k \)個)の出力フレームを生成する場合、
最初のクエリにより生成された各フレームに対するデータベースの走査は\( n^2\! / k \)の
パターンマッチャの呼出を必要とする。別の取り組み方としては\code{and}の2つの節を分離して処理し、
矛盾のない出力フレームの全てのペアを探すことになるだろう。もし各クエリが\( n / k \)個の
出力フレームを生成するなら、これは\( n^2\! / k^2 \)回の無矛盾テストを実行しなければ
ならないことを意味する。\( k \)の係数が現在の手法で必要な適合数よりもより少ない。


この戦略を用いる\code{and}の実装を工夫せよ。入力として2つのフレームを取り、
両フレームの中の束縛が無矛盾であることを確認しなければならない。もしそうであるなら
束縛の2つの集合をマージするフレームを生成する。この操作はユニフィケーションに似ている。
\end{quote}

\begin{quote}
\heading{\phantomsection\label{Exercise 4.77}Exercise 4.77:}
\link{Section 4.4.3}において\code{not}と\code{lisp\-/value}がクエリ言語に対し
もしこれらのフィルタリング命令が変数が束縛されていないフレームに適用された場合に
``間違った''回答を与えることがあることを学んだ。この欠陥を直す方法を工夫せよ。
1つの考えはフィルタリングを``遅延''の様式で実行することだ。フレームに``プロミス''を
追加することで十分な変数がその操作を可能にする場合にのみそれを果たすようにする。
フィルタリングの実行は全ての他の命令が実行を終えるまで待つことができる。しかし、
効率のために生成される中間フレームの数を削減できるようフィルタリングをできるだけ早く実行したい。
\end{quote}

\begin{quote}
\heading{\phantomsection\label{Exercise 4.78}Exercise 4.78:}
クエリ言語をストリーム処理ではなく非決定性プログラムとして、\link{Section 4.3}の評価機を用いて実装されるように
再設計せよ。この取り組み方においては、各クエリは(全ての回答のストリームではなく)単一の回答を生成し、
ユーザは\code{try\-/again}を入力することでより多くの回答を見ることができる。
この節で構築した仕組みの多くは非決定性探索とバックトラックにより組込まれていることに気付かなければ
ならない。しかし、新しいクエリ言語の振舞にここで実装されたものからわずかな違うことにも気付くだろう。
この違いを説明する例を見つけることができるだろうか?
\end{quote}

\begin{quote}
\heading{\phantomsection\label{Exercise 4.79}Exercise 4.79:}
\link{Section 4.1}でLisp評価機を実装した時に、どのようにローカル環境を使用して
手続のパラメタ間の名前衝突を防ぐかについて学んだ。例えば以下を評価する場合に
おいて、

\begin{scheme}
(define (square x) (* x x))
(define (sum-of-squares x y)
  (+ (square x) (square y)))
(sum-of-squares 3 4)
\end{scheme}

\noindent
\code{square}の\code{x}と\code{sum\-/of\-/squares}の\code{x}の間に混乱は無い。なぜなら
各手続のボディをローカル変数のための束縛を含めるために特別に構築した環境の
中で評価するからである。クエリシステムではルール適用における名前衝突を
避けるために異なる戦略を用いた。ルールを適用する度に変数名を唯一であることを
保証された新しい名前に変えている。Lisp評価機に対する同様の戦略はローカルな
環境を排除し、手続を適用する度に手続のボディの中の変数を改名することになるだろう。

クエリ言語に対して改名でなく、環境を用いるルール適用手法を実装せよ。
あなたの環境構造上に巨大システムを取り扱うために、クエリ言語内にブロック構造化された
手続に同等なルールのような構成概念を作るために構築できるか確かめよ。
これの何かを文脈中での推論を行う問題に、問題解決の手段として関連付けることはできるだろうか?
(例えば``もし\( P \)が真であるとするならば、\( A \)と\( B \)を推論することができる'')。

(この問題には明確な回答やルールは存在しない。良い回答は恐らく博士号の価値があるだろう。)
\end{quote}


\chapter{レジスタマシンによる演算}
\label{Chapter 5}

\vspace{0.2em}

\begin{quote}
私の目的は天の機械は神からの授かり物や生き物ではなく、時計仕掛のような物であることを
示すことです。
(そして時計が魂を持つと信ずる人はその理由をその仕事に対するメーカーの栄光に帰するでしょう。)
それはほとんど全ての多様な運動が最も単純な物質の力により引き起こされるとする限りにおいて、
時計の全ての動作が1つの錘により引き起こされるのと全く同じように。

---Johannes Kepler (Herwart von Hohenburgへの手紙, 1605)
\end{quote}

\vspace{1.0em}

\noindent
私達はこの本をプロセスを学ぶことと、プロセスをLispで書かれた手続を用いて説明することにより
始めました。これらの手続の意味を説明するために、いくつかの評価モデルを用いました。
\link{Chapter 1}の置換モデル、\link{Chapter 3}の環境モデル、\link{Chapter 4}のメタ循環評価機です。
私達のメタ循環評価機の調査は特にLispのような言語がどのように解釈されるのかについての
謎の大部分を氷解させました。しかしメタ循環評価機ですらも重要な疑問を未知の状態に残します。
Lispシステム中の制御の仕組みは明らかにしないためです。例えば、この評価機は
部分式の評価がこの式の値を用いる式にどのようにその値を返すのかについて説明しません。
またこの評価機は、ある再帰関数が反復プロセス(つまり、定量的な記憶域で評価されるもの)を生成するのに
対し、一方で他の再帰関数が再帰プロセスを生成することもまた説明しません。
これらの疑問は未解決のままです。なぜならメタ循環評価機はそれ自身がLispプログラムであり、
それ故に根底に存在するLispシステムの制御構造を引き継ぐためです。より完全なLisp評価機の
制御構造の説明を与えるためには、Lispそれ自身よりもよりプリミティブなレベルについて
取り組まねばなりません。

この章ではプロセスを旧来の計算機の個々の操作を用いて説明します。そのような計算機、つまり
\newterm{register machine}(\jnewterm{レジスタマシン})は\newterm{registers}(\jnewterm{レジスタ})と呼ばれる
固定長の記憶要素の集合の中身を操作する\newterm{instructions}(\jnewterm{命令})を順に実行します。
典型的なレジスタマシンの命令はプリミティブな操作をいくつかのレジスタの中身に対して
適用し、その結果を他のレジスタに割り当てます。レジスタマシンにより実行される
プロセスの私達の説明は伝統的な計算機向けの``機械語''にとても良く似ているでしょう。
しかし、何らかの特定の計算機の機械語に注力する代わりに、私達はいくつかのLisp手続を
調査し、各手続を実行するための特定のレジスタマシンを設計します。従って私達はこの目的に
機械語のコンピュータプログラマではなく、ハードウェアアーキテクトの視点から取り組みます。
レジスタマシンの設計において、私達は再帰のような重要なプログラミング構造を実装するための
仕組みを開発します。またレジスタマシンの設計を記述するための言語も与えます。
\link{Section 5.2}ではこれらの記述を用いて設計したマシンをシミュレートするLispプログラムを
実装します。

私達のレジスタマシンのプリミティブな命令の多くはとても簡単です。例えばある命令は2つの
レジスタから取得した数値を足し、結果を生成して3つ目のレジスタに格納します。
そのような命令は簡単に記述されたハードウェアにより実行されることができます。
しかし、リスト構造を取り扱うためにはメモリ操作命令\code{car}, \code{cdr}, \code{cons}もまた
使用します。これは複雑なストレージ(記憶領域)獲得の仕組みを必要とします。
\link{Section 5.3}でより初歩的な命令を用いてのそれらの実装について学びます。

\link{Section 5.4}ではレジスタマシンによる簡単な手続の形式化についての経験を貯めた後に、
\link{Section 4.1}のメタ循環評価機により説明されたアルゴリズムを実行するマシンを設計します。
これがSchemeがどのように解釈されるのかについての私達の理解のギャップを、評価機の
制御の仕組みに対する明確なモデルを与えることにより、埋めることでしょう。
\link{Section 5.5}ではSchemeプログラムを評価機のレジスタマシンのレジスタと命令を用いて直接実行可能な
一連の命令に変換する簡単なコンパイラについて学びます。



\section{レジスタマシンの設計}
\label{Section 5.1}

レジスタマシンを設計するためには、その\newterm{data paths}(\jnewterm{データパス})(レジスタと命令)と
これらの命令を順序付ける\newterm{controller}(\jnewterm{コントローラ})を設計する必要があります。
簡単なレジスタマシンの設計を説明するために、2つの整数の最大公約数(\acronym{GCD})を求めるために使用した
ユークリッドのアルゴリズムを検討しましょう。\link{Section 1.2.5}で学んだように、ユークリッドの
アルゴリズムは反復プロセスにて以下の手続にて指定されるように実行されることができます。

\begin{scheme}
(define (gcd a b)
  (if (= b 0)
      a
      (gcd b (remainder a b))))
\end{scheme}

\noindent
このアルゴリズムを実行する機械は2つの数値、\( a \)と\( b \)を追跡しなければいけません。
そうすることでこれらの数値がそれらの名前と共に2つのレジスタに格納されることが推測できます。
必要とされる基本的な命令はレジスタ\code{b}の値が0であるかどうかを確認し、レジスタ\code{a}の中身を
レジスタ\code{b}の中身で割った余りを求めます。剰余の命令は複雑な処理ですが、当座は剰余を求める
プリミティブな手法が存在すると仮定します。\acronym{GCD}アルゴリズムの各サイクルにおいて、
レジスタ\code{a}の中身はレジスタ\code{b}の中身で置き換えられ、レジスタ\code{b}の中身は\code{a}の古い中身を
\code{b}の古い中身で割った場合の余りで置き換えられなければなりません。
もしこれらの置換が同時に行われれば便利でしょう。しかし私達のレジスタマシンのモデルでは
ただ1つのレジスタのみが各ステップで新しい値を割り当てられることができます。
置換を達成するためには、私達の機械は3つ目の``temporary''(一時的な)レジスタを使用します。
これを\code{t}と呼びます。(最初に剰余は\code{t}に置かれます。次に\code{b}の中身が\code{a}に置かれます。
最後に\code{t}に格納されている剰余が\code{b}に置かれます。)

この機械のレジスタと命令を\link{Figure 5.1}に示されるデータパス図を用いて説明することができます。
この図では、レジスタ(\code{a}, \code{b}, \code{t})は長方形で表現されます。値をレジスタに割り当てる方向は
\code{X}が頭の後ろにあり、データの元からレジスタを指す矢印により示されます。\code{X}は押された時に
元の値が指定されたレジスタに``flow''する(流れる)ボタンだと考えることができます。
各ボタンの横にあるラベルはそのボタンを参照するのに使われる名前です。この名前は
自由で、かつ記憶を助ける値を持つことを選択することができます。(例えば、\code{a<\-/b}は
ボタンを押すとレジスタ\code{b}の中身を\code{a}に割り当てることを意味します)。
レジスタに対するデータ元は別のレジスタであることも可能で(\code{a<\-/b}の代入のように)、また
(\code{t<\-/r}の代入のように)命令の結果や、定数(変更できない組込の値、データパス図では
定数を持つ三角形で表現される)にもなり得ます。

\begin{figure}[tb]
\phantomsection\label{Figure 5.1}
\centering
\begin{comment}
\heading{Figure 5.1:} Data paths for a \acronym{GCD} machine.

\begin{example}
                              ___
+-----+          +-----+     /   \
|  a  |<--(X)----|  b  +--->|  =  |
+--+--+   a<-b   +-+---+     \___/
   |               |  ^        ^
   +------+   +----+  |        |
          |   |      (X) b<-t  |
       .--+---+--.    |       / \
        \  rem  /     |      / O \
         \_____/      |     +-----+
            |         |
           (X) t<-r   |
            |         |
            V         |
         +-----+      |
         |  t  +------+
         +-----+
\end{example}
\end{comment}
\includegraphics[width=58mm]{fig/chap5/Fig5.1a.pdf}
\par\bigskip
\noindent
\heading{Figure 5.1:} \acronym{GCD}マシンのデータパス
\end{figure}

\noindent
定数とレジスタの中身から値を求める命令はデータパス図では命令に対する名前を持つ
台形により表現されます。例えば\link{Figure 5.1}で\code{rem}と印された箱はそれに
取り付けられたレジスタ\code{a}と\code{b}の中身の剰余を求める命令を表します。
ボタンの無い矢印は入力レジスタと定数から箱へと指し、別の矢印は命令の出力値からレジスタへと
接続しています。テストはそのテストを表す名前を持つ円で表現されます。例えば、
私達の\acronym{GCD}マシンはレジスタ\code{b}の中身がゼロであるかをテストする命令を持ちます。
テストはまたその入力レジスタと定数からの矢印を持ちます。しかし出力の矢印を持ちません。
その値はデータパスでなくコントローラにより使用されます。全体としては、データパス図は
機械にとって必要とされるレジスタと命令と、それらがどのように接続されるべきかを示しています。
もし私達が矢印を配線に、\code{X}ボタンをスイッチだと見れば、データパス図は電子部品から
構築することができる機械の配線図にとても似ています。

データパスに対し実際に\acronym{GCD}を求めるためには、複数のボタンが正しい順序で
押される必要があります。私達はこの順序を\link{Figure 5.2}で図示されるコントローラ図を
用いて説明します。コントローラ図の要素はデータパスのコンポーネントがどのように
操作されるべきかであるかを示します。コントローラ図の長方形の箱は押されるべきデータパスの
ボタンを判別します。そして矢印はあるステップから次への順を示します。図の中のひし形は
選択を表現します。ひし形内で確認されたデータパスのテストの値に依存し、2つの順路矢印の1つに
従います。私達はコントローラを物質的なアナロジーを用いて解釈することができます。この図を
ビー玉が転がっている迷路だと考えるのです。ビー玉が箱に転がり込んだ時に、箱により名付けら
れたデータパスボタンを押します。ビー玉が(\code{b} = 0のテストのような)決断点に転がり込んだ
時には、示されたテストの結果により決定された道に乗りその点を去ります。
これらをもとに、データパスとコントローラは完全に\acronym{GCD}を求めるための機械を説明します。
私達はコントローラ(転がるビー玉)を\code{start}と印された地点から、レジスタ\code{a}と\code{b}に
数値を置いてから開始します。コントローラが\code{done}に辿り着いた時、\acronym{GCD}の値は
レジスタ\code{a}の中に見つかります。

\begin{figure}[tb]
\phantomsection\label{Figure 5.2}
\centering
\begin{comment}
\heading{Figure 5.2:} Controller for a \acronym{GCD} machine.

\begin{example}
     start
       |
       V
      / \ yes
+--->< = >-----> done
|     \ /
|      | no
|      V
|  +------+
|  | t<-r |
|  +---+--+
|      |
|      V
|  +------+
|  | a<-b |
|  +---+--+
|      |
|      V
|  +------+
+--+ b<-t |
   +------+
\end{example}
\end{comment}
\includegraphics[width=41mm]{fig/chap5/Fig5.2.pdf}
\par\bigskip
\noindent
\heading{Figure 5.2:} \acronym{GCD}マシンのコントローラ
\end{figure}


\begin{quote}
\heading{\phantomsection\label{Exercise 5.1}Exercise 5.1:}
以下の手続で指定される反復アルゴリズムを用いて階乗を求めるレジスタマシンを設計せよ。
このマシンに対するデータパスとコントローラの図を描け。

\begin{scheme}
(define (factorial n)
  (define (iter product counter)
    (if (> counter n)
        product
        (iter (* counter product)
              (+ counter 1))))
  (iter 1 1))
\end{scheme}
\end{quote}



\subsection{レジスタマシンを記述するための言語}
\label{Section 5.1.1}


データパスとコントローラの図は\acronym{GCD}の様な簡単な機械を表現するには適切です。
しかしそれらはLispインタプリタのような大きな機械を記述するには扱いにくい物です。
複雑な機械を扱うことを可能にするために、私達はテキスト形式でデータパスとコントローラの図
により与えられる全ての情報を表現する言語を作成することにします。まずは直接図を写し取る
表記法から始めます。

機械のデータパスをレジスタと命令を記述することにより定義します。レジスタを記述するために、
それに名前を与え、それに対する代入をコントロールするボタンを指定します。
これらのボタン全てに名前を与え、ボタンのコントロールの下にレジスタに入れられる
データの代入元を指定します。(代入元はレジスタ、定数、または命令です)。命令を
記述するために、それに名前を与え、その入力(レジスタ、または定数)を指定します。

機械のコントローラを\newterm{instructions}(\jnewterm{命令})の列として、その列の
\newterm{entry points}(\jnewterm{エントリポイント}、入口)を特定する\newterm{labels}(\jnewterm{ラベル})と共に
定義します。

\begin{itemize}

\item
レジスタに値を割り当てるために押すデータパスボタンの名前。(これはコントローラ図の
箱に対応する)

\item
\code{test}(テスト)命令、特定のテストを実行する。

\item
直前のテストの結果に基づくコントローララベルにより示された地点への条件分岐(\code{branch}命令)。
(テストと分岐は共にコントローラ図のひし形に対応する)。もしテストが偽であれば、
コントローラは命令列の次の命令へと続ける。そうでなければ、コントローラはラベルの
次の命令から続ける。

\item
無条件分岐(\code{goto}命令)は実行を続ける地点にコントローララベルを名付ける

\end{itemize}

\noindent
機械はコントローラの命令列の初めから開始し、列の終わりに辿り付いた時に実行を停止する。
ただし分岐が制御の流れを変更した場合、命令はそれが並べられた順に向かい実行される。

\begin{quote}
\heading{\phantomsection\label{Figure 5.3}Figure 5.3:} \( \downarrow \) A specification of the \acronym{GCD}
machine.

\begin{scheme}
(data-paths
 (registers
  ((name a)
   (buttons ((name a<-b) (source (register b)))))
  ((name b)
   (buttons ((name b<-t) (source (register t)))))
  ((name t)
   (buttons ((name t<-r) (source (operation rem))))))
 (operations
  ((name rem) (inputs (register a) (register b)))
  ((name =) (inputs (register b) (constant 0)))))
(controller
 test-b                           ~\textrm{; label}~
   (test =)                       ~\textrm{; test}~
   (branch (label gcd-done))      ~\textrm{; conditional branch}~
   (t<-r)                         ~\textrm{; button push}~
   (a<-b)                         ~\textrm{; button push}~
   (b<-t)                         ~\textrm{; button push}~
   (goto (label test-b))          ~\textrm{; unconditional branch}~
 gcd-done)                        ~\textrm{; label}~
\end{scheme}

\end{quote}

\noindent
\link{Figure 5.3}はこの方法で記述された\acronym{GCD}マシンを示します。この例はこれらの
記述の一般性を暗示しているに過ぎません。\acronym{GCD}マシンはとても単純な場合であるからです。
各レジスタはたった1つのボタンしか持たず、各ボタンとテストはコントローラにより
ただ1度しか利用されていません。

残念なことに、このような記述を読むことは難しいことです。コントローラの命令を理解する
ためには、常にボタンの名前と命令の名前の定義に戻らねばならず、またボタンが何をするのか
理解するためには命令の名前の定義を参照する必要があるでしょう。従って私達はこの表記法を
変形し、データパスとコントローラの記述からの情報を組み合わせることで全てを一緒に
見られるようにします。

記述のこの形式を得るために、自由裁量なボタンと命令の名前をそれらの振舞の定義により
置き換えます。つまり、(コントローラの中で)``ボタン\code{t<\-/r}を押せ''と言い、別に
(データパスの中で)``ボタン\code{t<\-/r}は\code{rem}命令の値をレジスタ\code{t}に代入''と
``\code{rem}命令の入力はレジスタ\code{a}と\code{b}の中身''と言う代わりに、これからは
(コントローラの中で)``レジスタ\code{a}と\code{b}の中身上での\code{rem}命令の値をレジスタ\code{t}に
代入するボタンを押せ''と言うことにします。同様に、(コントローラの中で)``\code{=}テストを
実行せよ''と言い、別に(データパスの中で)``\code{=}テストはレジスタ\code{b}の中身と定数0の上で
動作する''と言う代わりに、これからは``\code{=}テストをレジスタ\code{b}の中と定数0の上で実行せよ''と
言います。データパスの記述は省略し、コントローラの命令列のみを残します。従って、
\acronym{GCD}マシンは以下のように記述されます。

\begin{scheme}
(controller
 test-b
   (test (op =) (reg b) (const 0))
   (branch (label gcd-done))
   (assign t (op rem) (reg a) (reg b))
   (assign a (reg b))
   (assign b (reg t))
   (goto (label test-b))
 gcd-done)
\end{scheme}

\noindent
この記述の形式は\link{Figure 5.3}で説明されたものよりも読み易いでしょう。
しかし同時に欠点も持ちます。

\begin{itemize}

\item
大きな機械に対してはより冗長である。データパス要素の複雑な記述がその要素が
コントローラ命令列内で触れられる度に繰り返されるため。(これは\acronym{GCD}の例では
問題にならない。命令とボタンのそれぞれがただ1度しか使用されないため)。さらに、
データパス記述の繰り返しが実際の機械のデータパス構造を分かりにくくする。
大きな機械にとっていくつのレジスタ、命令、ボタンが存在し、それらがどのように相互接続されて
いるのかは自明では無い。

\item
機械の定義内のコントローラの命令はLisp式の様に見えるため、それらが自由裁量なLisp式
ではないことを簡単に忘れてしまう。それらは正式な機械の命令のみを記述できる。
例えば、命令は直接には定数とレジスタの中身のみに対して操作ができる。他の命令の
結果に対してはできない。

\end{itemize}

\noindent
これらの欠点にも係らず、私達はこのレジスタマシンの言語をこの章を通して使用します。
データパスの要素と接続を理解することよりもコントローラを理解することにより関係
していくためです。しかし、私達はデータパスの設計は実際の機械の設計において、とても
重要であることを肝に命じておかねばなりません。

\begin{quote}
\heading{\phantomsection\label{Exercise 5.2}Exercise 5.2:}
レジスタマシン言語を用いて\link{Exercise 5.1}の反復階乗機械を記述せよ。

\end{quote}

\subsubsection*{アクション}


\acronym{GCD}マシンを変更して、\acronym{GCD}が欲しい数値を入力し、端末に答が表示される
ようにしてみましょう。私達は読み込みや表示ができる機械をどのように作るかについては
議論しません。しかし(私達がSchemeにて\code{read}と\code{display}を使う時に行うように)
それらがプリミティブな命令として既に存在すると仮定します。\footnote{この仮定は
多量の複雑さを言い繕っています。通常、Lispシステムの実装の大きな部分が読み込みと
表示を可能にすることに関してささげられています。}

\begin{figure}[tb]
\phantomsection\label{Figure 5.4}
\centering
\begin{comment}
\heading{Figure 5.4:} A \acronym{GCD} machine that reads inputs and prints results.

\begin{example}
                   .--------.
                    \ read /
                     \____/
                       |
               +-------*------+
               |              |
        a<-rd (X)            (X) b<-rd
               |              |
               V              V           ___
            +-----+        +-----+       /   \
            |  a  |<--(X)--+  b  +----->|  =  |
            +-+-+-+  a<-b  +-+---+       \___/
              | |            |  ^          ^
           +--+ +----+    +--+  |          |
           |         |    |    (X) b<-t   / \
           V         V    V     |        / O \
      .---------.  .---------.  |       /_____\
--(X)->\ print /    \  rem  /   |
   P    \_____/      \_____/    |
                        |       |
                       (X) t<-r |
                        |       |
                        V       |
                     +-----+    |
                     |  t  +----+
                     +-----+
\end{example}

\begin{scheme}
 (controller
  gcd-loop
    (assign a (op read))
    (assign b (op read))
  test-b
    (test (op =) (reg b) (const 0))
    (branch (label gcd-done))
    (assign t (op rem) (reg a) (reg b))
    (assign a (reg b))
    (assign b (reg t))
    (goto (label test-b))
  gcd-done
    (perform (op print) (reg a))
    (goto (label gcd-loop)))
\end{scheme}

\end{comment}
\includegraphics[width=107mm]{fig/chap5/Fig5.4b.pdf}
\par\bigskip
\noindent
\heading{Figure 5.4:} 入力を読み込み結果を表示する\acronym{GCD}マシン
\end{figure}

\noindent
\code{read}は私達が使用してきた、その中でレジスタに格納することができる値を生成する
命令のような物です。しかし\code{read}は入力をどのレジスタからも取得しません。その値は
私達が設計している機械の外側の部品で起こる何かに依存しています。私達は私達の機械の
命令にそのようは振舞を持つことを許します。従って\code{read}の使用を描き、記述することを
他の任意の値を求める命令と全く同様に行います。

一方で、\code{print}は私達が使用してきた命令とは基本的な意味において異なります。これは
レジスタに格納できる出力の値を生成しません。この種の命令は\newterm{action}(\jnewterm{アクション})として
参照することにします。データパス図ではアクションは値を求める命令と同じように、アクションの
名前を含む台形として表現します。矢印は任意の入力(レジスタ、または定数)からアクションの箱へと
指します。またボタンをアクションと関連付けることもします。ボタンを押すとアクションが起こります。
コントローラにアクションボタンを押させるために、\code{perform}(パフォーム、実行)と呼ばれる
新しい種類の命令を用います。従ってレジスタ\code{a}の中身を表示するアクションはコントローラの
命令列の中でその命令により表現されます。

\begin{scheme}
(perform (op print) (reg a))
\end{scheme}

\noindent
\link{Figure 5.4}は新しい\acronym{GCD}マシンのデータパスとコントローラを示しています。
回答を表示した後にマシンをストップさせる代わりに、再開させています。そのため数値の
ペアを読み込み、それらの\acronym{GCD}を計算し、結果を表示することを繰り返します。
この構造は\link{Chapter 4}のインタプリタにて使用したドライバループに似ています。

\subsection{機械設計における抽象化}
\label{Section 5.1.2}

私達はこれから頻繁に、実際には複雑な``プリミティブな''命令を含む機械を定義します。例えば
\link{Section 5.4}と\link{Section 5.5}ではSchemeの環境の操作をプリミティブとして扱います。
そのような抽象化はそれにより機械の部品の詳細を無視することを可能にし、設計の他の
側面に集中することを可能にするため有益です。しかし、私達が数多くの複雑さを敷物の下に
隠してしまった事実は機械設計が非現実的であることを意味しません。
私達は常に複雑な``プリミティブ''をより簡単なプリミティブな命令で置き換えることができます。

\acronym{GCD}マシンについて考えます。マシンはレジスタ\code{a}と\code{b}の中身の剰余を
求める、結果をレジスタ\code{t}に割り当てる命令を持ちます。もし\acronym{GCD}マシンを
プリミティブな剰余命令を使用すること無しに構築したい場合、より単純な命令、例えば
引き算を用いてどのように剰余を求めるのかを指定しなければなりません。実際に、この
方法で剰余を見つけるSchemeの手続を描くことができます。

\begin{scheme}
(define (remainder n d)
  (if (< n d) n (remainder (- n d) d)))
\end{scheme}

従って\acronym{GCD}マシンのデータパス内の剰余命令を引き算命令と比較テストで置き換える
ことができます。\link{Figure 5.5}は緻密化されたマシンのデータパスとコントローラを
示します。\acronym{GCD}コントローラ定義内の以下の命令は、

\begin{scheme}
(assign t (op rem) (reg a) (reg b))
\end{scheme}

\noindent
\link{Figure 5.6}に示されるように、ループを含む一連の命令により置き換えることができます。

\begin{figure}[tp]
\phantomsection\label{Figure 5.5}
\centering
\begin{comment}
\heading{Figure 5.5:} Data paths and controller for the elaborated \acronym{GCD} machine.

\begin{example}
                                    ___
+-----+         +-----+            /   \
|  a  |<--(X)---+  b  +-------*-->|  =  |
+--+--+   a<-b  +-+---+       |    \___/
   |              |  ^        |
  (X) t<-a        |  |        |
   |              | (X) b<-t  |
   V              |  |       _V_
+-----+           |  |      /   \
|  t  +-------*---|--*-----|  <  |
+-----+       |   |         \___/
   ^          V   V
   |        ---------
  (X) t<-d   \  -  /
   |          --+--
   |            |
   +------------+


   start
     |
     V
    / \ yes            +-------+
+->< = >----> done     | t<-d  |<--+
|   \ /                +---+---+   |
|    | no                  |       |
|    |                     V       |
|    |   +------+         / \ no   |
|    +-->| t<-a +------->< < >-----+
|        +------+         \ /
|                          | yes
|      +-------------------+
|      V
|  +-------+
|  | a<-b  |
|  +---+---+
|      |
|      V
|  +-------+
+--+ b<-t  |
   +-------+
\end{example}
\end{comment}
\includegraphics[width=67mm]{fig/chap5/Fig5.5a.pdf}
\begin{quote}
\heading{Figure 5.5:} 精緻な\acronym{GCD}マシンのデータパスとコントローラ
\end{quote}
\end{figure}


\noindent
\heading{\phantomsection\label{Figure 5.6}\mbox{Figure 5.6:}} \( \downarrow \) \link{Figure 5.5}の\acronym{GCD}マシンのコントローラの命令列

\begin{quote}
\begin{scheme}
(controller test-b
              (test (op =) (reg b) (const 0))
              (branch (label gcd-done))
              (assign t (reg a))
            rem-loop
              (test (op <) (reg t) (reg b))
              (branch (label rem-done))
              (assign t (op -) (reg t) (reg b))
              (goto (label rem-loop))
            rem-done
              (assign a (reg b))
              (assign b (reg t))
              (goto (label test-b))
            gcd-done)
\end{scheme}
\end{quote}

\begin{quote}
\heading{\phantomsection\label{Exercise 5.3}Exercise 5.3:}
平方根を求める機械を\link{Section 1.1.7}で説明されたようにニュートン法を用いて設計せよ。

\begin{scheme}
(define (sqrt x)
  (define (good-enough? guess)
    (< (abs (- (square guess) x)) 0.001))
  (define (improve guess)
    (average guess (/ x guess)))
  (define (sqrt-iter guess)
    (if (good-enough? guess)
        guess
        (sqrt-iter (improve guess))))
  (sqrt-iter 1.0))
\end{scheme}


\code{good\-/enough?}と\code{improve}命令はプリミティブとして存在するとして始めよ。
次にこれらを算術演算子を用いてどのように展開するか示せ。\code{sqrt}マシン設計の各版を
データパス図を描き、レジスタマシンのコントローラ定義を記述することで説明せよ。
\end{quote}

\subsection{サブルーチン}
\label{Section 5.1.3}

演算を実行する機械を設計する時、私達は良くコンポーネントを複製するのではなく、
演算の異なる部品により共有されるコンポーネントを準備することを好みます。
2つの\acronym{GCD}演算を含む機械についえ考えてみましょう。1つはレジスタ\code{a}と\code{b}の
中身の\acronym{GCD}を求め、もう1つはレジスタ\code{c}と\code{d}の\acronym{GCD}を求めます。
私達はまずプリミティブな\code{gcd}命令を持つと仮定することから始め、次に2つの\code{gcd}の
インスタンスをよりプリミティブな命令を用いて展開するでしょう。\link{Figure 5.7}は
結果としての機械のデータパスの\acronym{GCD}の部分を、それらが機械の残りの部分にどのように
接続されていかを除いて示しています。この図はまた機械のコントローラシーケンス(命令列)の
対応する部分も示しています。

\begin{figure}[tb]
\phantomsection\label{Figure 5.7}
\centering
\begin{comment}
\heading{Figure 5.7:} Portions of the data paths and controller sequence for a machine with two \acronym{GCD} computations.

\begin{example}
                            ___                                 ___  
+-----+        +-----+     /   \    +-----+        +-----+     /   \ 
|  a  |<-(X)---+  b  |--->|  =  |   |  c  |<-(X)---+  d  |--->|  =  |
+--+--+  a<-b  ++----+     \___/    +--+--+  c<-d  ++----+     \___/ 
   |            |  ^         ^         |            |  ^         ^   
   `----.   .---'  |         |         `----.   .---'  |         |   
        V   V     (X) b<-t   |              V   V     (X) d<-t   |   
       -------     |        / \            -------     |        / \  
       \ rem /     |       /_0_\           \ rem /     |       /_0_\ 
        --+--      |                        --+--      |             
          |        |                          |        |             
         (X) t<-r  |                         (X) s<-r  |             
          |        |                          |        |             
          V        |                          V        |             
       +-----+     |                       +-----+     |             
       |  t  +-----'                       |  s  +-----'             
       +-----+                             +-----+                   
\end{example}

\begin{scheme}
gcd-1
 (test (op =) (reg b) (const 0))
 (branch (label after-gcd-1))
 (assign t (op rem) (reg a) (reg b))
 (assign a (reg b))
 (assign b (reg t))
 (goto (label gcd-1))
after-gcd-1
   ~\( \dots \)~
gcd-2
 (test (op =) (reg d) (const 0))
 (branch (label after-gcd-2))
 (assign s (op rem) (reg c) (reg d))
 (assign c (reg d))
 (assign d (reg s))
 (goto (label gcd-2))
after-gcd-2
\end{scheme}

\end{comment}
\includegraphics[width=105mm]{fig/chap5/Fig5.7b.pdf}
\begin{quote}
\heading{Figure 5.7:} 2つの\acronym{GCD}演算を持つ機械のデータパスとコントローラシーケンスの一部
\end{quote}
\end{figure}

\noindent
この機械は2つの剰余命令の箱と2つの等値テストの箱を持っています。もし複製された
コンポーネントが剰余の箱のように複雑なら、これは機械を構築するのに経済的な方法では
ありません。私達はより大きな機械の演算に影響を与えないように与えられた場合に、
同じコンポーネントを両方の\acronym{GCD}演算に用いることでデータパスコンポーネントの
複製を防ぎます。もしレジスタ\code{a}と\code{b}の値がコントローラが\code{gcd\-/2}に取り掛かって
いる時に必要無いのであれば(またはもしこれらの値が安全のために他のレジスタに移動して
おくことができるのならば)、機械を変更し、レジスタ\code{c}と\code{d}でなく、レジスタ\code{a}と
\code{b}を2つ目の\acronym{GCD}を1つ目と同じに求めるおおができます。もしこれを行うなら、
\link{Figure 5.8}に示されるコントローラシーケンスを得ます。
\newpage

\begin{quote}
\heading{\phantomsection\label{Figure 5.8}Figure 5.8:} \( \downarrow \) 2つの異なる\acronym{GCD}演算に対して同じデータパスコンポーネントを使用する機械のコントローラシーケンスの一部
\begin{scheme}
gcd-1
 (test (op =) (reg b) (const 0))
 (branch (label after-gcd-1))
 (assign t (op rem) (reg a) (reg b))
 (assign a (reg b))
 (assign b (reg t))
 (goto (label gcd-1))
after-gcd-1
  ~\( \dots \)~
gcd-2
 (test (op =) (reg b) (const 0))
 (branch (label after-gcd-2))
 (assign t (op rem) (reg a) (reg b))
 (assign a (reg b))
 (assign b (reg t))
 (goto (label gcd-2))
after-gcd-2
\end{scheme}
\end{quote}

\noindent
私達はデータパスコンポーネントの複製を削除しました。(そうすることでデータパスは
\link{Figure 5.1}の状態に戻りました)。しかしコントローラは今ではそれらのエントリポイントの
ラベルのみが異なる2つの\acronym{GCD}シーケンスを持ちます。これら2つのシーケンスを
1つのシーケンス---\code{gcd} \newterm{subroutine}(\jnewterm{サブルーチン})---への分岐により置き換えた
ほうが良くなるでしょう。サブルーチンの終わりにメインの命令列の正しい場所へと戻ります。
これを次のように達成することができます。\code{gcd}に分岐する前に、(0か1のような)識別するための値を
特別なレジスタ、\code{continue}に置きます。\link{Figure 5.9}は結果としてのコントローラシーケンスの
関連する部分を示しています。これはただ1つの\code{gcd}命令列のコピーを含みます。

\begin{quote}
\heading{\phantomsection\label{Figure 5.9}Figure 5.9:} \( \downarrow \) \link{Figure 5.8}でコントローラシーケンスの重複を防ぐため\code{continue}レジスタを用いる
\begin{scheme}
gcd
 (test (op =) (reg b) (const 0))
 (branch (label gcd-done))
 (assign t (op rem) (reg a) (reg b))
 (assign a (reg b))
 (assign b (reg t))
 (goto (label gcd))
gcd-done
 (test (op =) (reg continue) (const 0))
 (branch (label after-gcd-1))
 (goto (label after-gcd-2))
  ~\( \dots \)~
 ~\textrm{;; \code{gcd}を必要とする場所からその場所へと分岐する前に}~
 ~\textrm{;; レジスタ\code{continue}に0を置く}~
 (assign continue (const 0))
 (goto (label gcd))
after-gcd-1
  ~\( \dots \)~

 ~\textrm{;; \code{gcd}の二度目の使用の前にはレジスタ\code{continue}に1を置く}~
 (assign continue (const 1))
 (goto (label gcd))
after-gcd-2
\end{scheme}
\end{quote}

\noindent
これは小さな問題に対応するのに妥当な取り組み方です。しかしもし数多くの\acronym{GCD}演算が
コントローラシーケンスの中にある場合には困ったことになりそうです。
\acronym{gcd}サブルーチンの後に実行をどこで続けるかを決定するために、データパス内のテストと
コントローラ内に分岐命令が\acronym{gcd}を置く全ての場所に対して必要となるでしょう。
サブルーチンを実装するためのより強力な手法は、\code{continue}レジスタにサブルーチンが終了した時に
実行が続行しなければならない場所のコントローラシーケンス内のエントリポイントのラベルを持たせることです。
この戦略の実装にはレジスタマシンのデータパスとコントローラの間に新しい種類のコネクションが必要です。
ラベルの値をレジスタから取得し指定されたエントリポイントから実行を再開するのに使用できるような方法のため、
レジスタにコントローラシーケンス内のラベルを代入するための方法が必要です。

この能力を反映するために、レジスタマシン言語の\code{assign}命令を拡張し、レジスタに値として
ラベルをコントローラシーケンスから(特別な種類の中身として)代入することを許可する拡張を行います。
また\code{goto}命令にも静的ラベルにより記述されたエントリポイントのみでなく、レジスタの中により
表されたエントリポイントから実行を続行することを許可する拡張を行います。
これらの新しい構造物を用いることで、\code{continue}レジスタ内に格納された
場所に分岐することにより、\code{gcd}サブルーチンを停止することができます。
これは\link{Figure 5.10}に示されたコントローラシーケンスへと導きます。

\begin{quote}
\heading{\phantomsection\label{Figure 5.10}Figure 5.10:} \( \downarrow \) Assigning labels to the
\code{continue} register simplifies and generalizes the strategy shown in
\link{Figure 5.9}.
\begin{scheme}
gcd
 (test (op =) (reg b) (const 0))
 (branch (label gcd-done))
 (assign t (op rem) (reg a) (reg b))
 (assign a (reg b))
 (assign b (reg t))
 (goto (label gcd))
gcd-done
 (goto (reg continue))
   ~\( \dots \)~
 ~\textrm{;; \code{gcd}を呼ぶ前に、\code{continue}に\code{gcd}が戻るべきラベルを代入します}~
 (assign continue (label after-gcd-1))
 (goto (label gcd))
after-gcd-1
   ~\( \dots \)~
 ~\textrm{;; 異なる継続を持つ2つ目の\code{gcd}呼出}~
 (assign continue (label after-gcd-2))
 (goto (label gcd))
after-gcd-2
\end{scheme}
\end{quote}

\noindent
複数のサブルーチンを持つマシンは複数の継続レジスタ(例えば\code{gcd\-/continue}, \code{factorial\-/continue})を
用いるか、または全てのサブルーチンが単一の\code{continue}レジスタを共有することができるでしょう。
共有はより経済的ですが、別のサブルーチン(\code{sub2})を呼び出すサブルーチン(\code{sub1})を持っていないか
注意しなければなりません。\code{sub1}が\code{continue}の中身を何か他のレジスタに、\code{continue}を
\code{sub2}の呼出のために設定する前に保存しなければ、 \code{sub1}は完了した時点で
どこに行けば良いのか知ることができません。次の節で開発される再帰を扱う仕組みは
この入れ子のサブルーチン呼出の問題にもより良い解法を提供します。

\subsection{再帰実装にスタックを使用する}
\label{Section 5.1.4}

ここまでに説明されたアイデアを用いて、そのプロセスの各状態変数に
対応するレジスタを持つレジスタマシンを指定することにより、任意の反復プロセスを
実装することができます。この機械はレジスタの中身を変更しながら、繰り返しコントローラの
ループを、ある停止条件が満たされるまで実行します。コントローラシーケンスの各地点に
おいて、(反復プロセスの状態を表現する)機械の状態はレジスタの状態(状態変数の値)により
完全に決定されます。

しかし、再帰プロセスを実装する場合には追加の仕組みを必要とします。
以下の階乗を求めるための再帰手法について考えましょう。これは\link{Section 1.2.1}で
最初に調査しました。

\begin{scheme}
(define (factorial n)
  (if (= n 1) 1 (* (factorial (- n 1)) n)))
\end{scheme}

\noindent
この手続から見てとれるように、\( n! \)の演算は\( (n - 1)! \)の演算を必要とします。
私達の\acronym{GCD}は以下の手続からモデル化されていますが、

\begin{scheme}
(define (gcd a b)
  (if (= b 0) a (gcd b (remainder a b))))
\end{scheme}

\noindent
同様に別の\acronym{GCD}を求める必要があります。しかし、元の演算から新しい{GCD}演算へと
簡約する{GCD}手続と、部分問題として別の階乗を求める必要がある\code{factorial}の間には
重要な違いがあります。\acronym{GCD}においては新しい\acronym{GCD}演算に対する答は
元の問題の答です。次の\acronym{GCD}を求めるためには、単純に新しい引数を\acronym{GCD}マシンの
入力レジスタに置き、機械のデータパスを同じコントローラシーケンスを実行することにより
再利用します。機械が最後の\acronym{GCD}問題を解くことを完了した時には、演算全体を
完了したことになります。

階乗の場合(または任意の再帰プロセス)においては新しい階乗の部分問題の回答は元の問題の
回答ではありません。\( (n - 1)! \)に対して得られた値は最終回答を得るために\( n \)で
乗算しなければなりません。もし\acronym{GCD}の設計を真似し、階乗の部分問題をレジスタ\code{n}を
デクリメント(1引く)し、階乗マシンに戻るこのより解決したいとしても、その結果に乗算を
行う有効な古い\code{n}の値は既に存在しません。従って部分問題上で働くための2つ目の
階乗マシンが必要です。この2つ目の階乗の演算はそれ自身が階乗の部分問題を持ち、それは
3つ目の階乗マシンを必要とし、以下繰り返されます。各階乗マシンがその中に別の階乗マシンを
持つため、総計の機械は同様な機械の無限の入れ子を含み、従って固定長の有限数な部品から
構築することはできません。

\begin{figure}[tp]
\phantomsection\label{Figure 5.11}
\centering
\begin{comment}
\heading{Figure 5.11:} A recursive factorial machine.

\begin{example}
                             ___
                            /   \
    +----------*-----------|  =  |
    |          |            \___/
   (X)         |              ^
    |          |              |
    V          |          +---+---+   sn    +-------+
+-------+      |          |       +---(X)-->|       |
|  val  |<-(X)-|----------+   n   |         | stack |
+-----+-+      |          |       |<--(X)---+       |
  ^   |        |          +-------+   rn    +-+-----+
  |   |        |            ^                 |   ^
 (X)  |        |            |                 |   |
  |   |   +----|--------*  (X)                |  (X) sc
  |   |   |    |        |   |             rc (X)  |
  |   |   |    *----.   |   |                 |   |
  |   V   V    |    V   V   |                 V   |
  |  -------   |   -------  |              +------+-+
  |  \  *  /   |   \  -  /  |              |continue+--> controller
  |   --+--    |    --+--   |              +--------+
  |     |      |      |     |               ^      ^
  +-----+      |      +-----+               |      |
               |                           (X)    (X)
               |                            |      |
              / \                   after- / \    / \  fact-
             /_1_\                  fact  /___\  /___\ done
\end{example}

\begin{smallscheme}
(controller
   (assign continue (label fact-done))     ~\textrm{; set up final return address}~
 fact-loop
   (test (op =) (reg n) (const 1))
   (branch (label base-case))
   ~\textrm{;; Set up for the recursive call by saving \code{n} and \code{continue}.}~
   ~\textrm{;; Set up \code{continue} so that the computation will continue}~
   ~\textrm{;; at \code{after\-/fact} when the subroutine returns.}~
   (save continue)
   (save n)
   (assign n (op -) (reg n) (const 1))
   (assign continue (label after-fact))
   (goto (label fact-loop))
 after-fact
   (restore n)
   (restore continue)
   (assign val (op *) (reg n) (reg val))   ~\textrm{; \code{val} now contains}~ ~\( n(n-1)! \)~
   (goto (reg continue))                   ~\textrm{; return to caller}~
 base-case
   (assign val (const 1))                  ~\textrm{; base case: }~1! = 1
   (goto (reg continue))                   ~\textrm{; return to caller}~
 fact-done)
\end{smallscheme}

\end{comment}
\includegraphics[width=106mm]{fig/chap5/Fig5.11a.pdf}
\par\bigskip
\noindent
\heading{Figure 5.11:} 再帰階乗マシン
\end{figure}

\noindent
それにもかかわらず、もし機械の各入れ子のインスタンスが同じコンポーネントを使用するように
準備ができれば階乗プロセスをレジスタマシンとして実装できます。具体的に言えば、
\( n! \)を求める機械は\( (n - 1)! \)を求める部分問題、\( (n - 2)! \)の部分問題、以下
繰り返しの仕事に同じコンポーネントを使用せねばなりません。これはもっともらしく見えます。
例え階乗プロセスが同じ機械のコピーの未束縛の数値が演算を実行するのに必要だと指図したとしても、
これらのコピーのただ1つが一度に有効になる必要があるためです。この機械が再帰の部分問題に
遭遇した時に、メインの問題上の仕事を中断し、同じ物理部品を部分問題上の仕事に再利用し、
そして中断した演算を続けることが可能です。

部分問題の中では、レジスタの中身はメインの問題の中の物と異なります。(この場合には
レジスタ\code{n}はデクリメントされます)。中断された演算を続けることを可能にするために、
機械は部分問題が解決した後に必要となる全てのレジスタの中身を保存しなければなりません。
そうすることで、中断した演算を続ける時にこれらの値が再格納されることができます。
階乗の場合には、デクリメントされたレジスタ\code{n}の階乗の演算が完了した時に再格納されるように
\code{n}の古い値を保存します。\footnote{古い\code{n}を保存する必要は無いと主張する人が
いるかもしれません。デクリメントし、部分問題を解決した後に、単純に古い値を回復するために
インクリメントすることができると思われるでしょう。例えこの戦略が階乗に対しては
働いたとしても、それは一般的にはうまく行きません。レジスタの古い値が常に
新しい値から求められるとは限らないためです。}

予測可能な限界が入れ子の再帰呼出の深さには存在しないため、任意の数のレジスタ値を
保存する必要があるでしょう。これらの値は保存された順の逆順に再格納されねばなりません。
入れ子の再帰では突入する最後の部分問題が最初に完了するためです。このことが
\newterm{stack}(\jnewterm{スタック})、つまり``last in, first out''(LIFO, 後入れ先出し)データ構造を
レジスタ値の保存への使用することを指示しています。レジスタマシン言語を拡張し、
2つの種類の命令を追加することでスタックを含めることができます。値はスタックに
\code{save}命令を用いて置かれて、\code{restore}命令を用いてスタックから再格納されます。
スタック上に一連の値が\code{save}された後に、連続した\code{restore}がこれらの値を逆順に
取り出します。\footnote{\link{Section 5.3}において,よりプリミティブな命令を用いてどのように
スタックを実装するかについて学びます。}

スタックの助けを借りることで階乗マシンの各階乗部分問題のために、データパスの
単一のコピーを再利用することができます。同様なデータパスを操作するコントローラシーケンスの
再利用についても同様の設計上の問題が存在します。階乗演算を再実行するためには,
コントローラは単純には最初に反復プロセスのようにループバックすることはできません。
\( (n - 1)! \)を解いた後には機械は依然としてその結果と\( n \)を掛ける必要があるためです。
コントローラは\( n! \)の演算を中断し、部分問題\( (n - 1)! \)を解き、そして\( n! \)の演算を
続けなければなりません。階乗演算のこの見方は\link{Section 5.1.3}で説明されたサブルーチンの仕組みの
使用を推奨しており、これはコントローラにレジスタ\code{continue}を使用させて部分問題を解く
列の一部へと移動し、そしてメイン問題を中止した場所から続行します。このようにして
\code{continue}レジスタに格納されたエントリポイントに帰る階乗のサブルーチンを作ることができます。
各サブルーチン呼出の周りでは、\code{continue}を\code{n}レジスタに行うのと同じように保存し
再格納します。階乗演算の各``レベル''が同じ\code{continue}レジスタを利用するためです。
つまり、階乗サブルーチンはそれが自分自身を部分問題として呼び出す時に、新しい値を\code{continue}に
設定しなければいけません。しかし部分問題を解くために呼び出した場所に戻るために
古い値が必要となるのです。

\link{Figure 5.11}は再帰\code{factorial}手続を実装する機械のためのデータパスとコントローラを
示しています。この機械はスタックと3つのレジスタ、\code{n}, \code{val}, \code{continue}を持ちます。
データパス図を単純化するために、レジスタ代入ボタンには名前を付けず、スタック命令ボタン
(レジスタを保存する\code{sc}と\code{sn}、レジスタに戻す\code{rc}と\code{rn})のみに付けています。
機械を運用するには、レジスタ\code{n}に階乗を求めたい数を入れ、それから機械を開始します。
機械が\code{fact\-/done}に辿り着いた時に演算は完了し、答はレジスタ\code{val}に見つかります。
コントローラシーケンスでは\code{n}と\code{continue}が各再帰呼出の前に保存され、その呼出から
戻る時に再格納されます。呼出からの復帰は\code{continue}に格納された場所に分岐することにより
達成されます。\code{continue}は機械が開始した時に最後の復帰が\code{fact\-/done}に向かうように
初期化されます。階乗演算の結果を持つ\code{val}レジスタは再帰呼出の前に保存されません。
\code{val}の古い中身はサブルーチンから復帰後には役に立たないためです。部分問題により
生成された新しい値のみが必要とされます。

例え原理上は階乗演算が無限の機械を必要とするとしても、\link{Figure 5.11}の機械は
実際には限りが無いかもしれないスタックを除けば有限です。しかし、スタックのどんな特定の
物理実装も有限のサイズを持ち、このことが機械により扱うことが可能な再帰呼出の深さを
制限します。この階乗の実装は再帰アルゴリズムをスタックで容量が増加された通常のレジスタマシン
として実現するための一般的な戦略を説明します。再帰部分問題に遭遇した時には
その現在の値が部分問題が解決された後に必要とされるレジスタをスタック上に保存します。
次に再帰部分問題を解決し、保存されたレジスタを戻してメイン問題の実行を続行します。
\code{continue}レジスタは常に保存されなければなりません。保存する必要の有るレジスタが
他に存在するかどうかは機械に依存します。全ての再帰演算が部分問題の解決の間に変更される
レジスタの元の値を必要とはしないためです。(\link{Exercise 5.4}参照)。

\subsubsection*{二重再帰}


より複雑な再帰プロセス、\link{Section 1.2.2}で紹介したフィボナッチ数の木再帰演算について調査してみましょう。

\begin{scheme}
(define (fib n)
  (if (< n 2)
      n
      (+ (fib (- n 1)) (fib (- n 2)))))
\end{scheme}

\noindent
階乗と同じ様に、再帰フィボナッチ演算をレジスタマシンとしてレジスタ\code{n}, \code{val}, \code{continue}と用いて
実装することができます。この機械は階乗のものよりも、より複雑です。コントローラシーケンスの中に
二箇所の再帰呼出の実行が必要な箇所が存在するためです。一度目は\( {\rm Fib}(n - 1) \)を求めるために、
二度目は\( {\rm Fib}(n - 2) \)を求めるためです。
これらの各呼出に準備するために、後にその値が必要となるレジスタを保存し、レジスタ\code{n}に
再帰的に求める(\( n - 1 \)または\( n - 2 \))必要のあるフィボナッチ数を設定します。
そして\code{continue}に戻り先のメインシーケンスのエントリポイント(それぞれ\code{afterfib\-/n\-/1}または\code{afterfib\-/n\-/2})を
割り当てます。そうしたら\code{fib\-/loop}へと飛びます。再帰呼出から帰る時には、回答は\code{val}の
中にあります。\link{Figure 5.12}はこの機械のためのコントローラシーケンスを示しています。

\begin{quote}
\heading{\phantomsection\label{Figure 5.12}Figure 5.12:} \( \downarrow \) Controller for a machine to compute
Fibonacci numbers.

\begin{scheme}
(controller
   (assign continue (label fib-done))
 fib-loop
   (test (op <) (reg n) (const 2))
   (branch (label immediate-answer))
   ~\textrm{;; Fib\( (n-1) \)を求める準備}~
   (save continue)
   (assign continue (label afterfib-n-1))
   (save n)                 ~\textrm{; \code{n}の古い値を保存}~
   (assign n (op -) (reg n) (const 1)) ~\textrm{; \code{n}を\code{n\-/1}で上書き}~
   (goto (label fib-loop))  ~\textrm{; 再帰呼出の実行}~
 afterfib-n-1     ~\textrm{; リターン時に, \code{val}がFib\( (n-1) \)を持つ}~
   (restore n)
   (restore continue)
   ~\textrm{;; Fib\( (n - 2) \)を求める準備}~
   (assign n (op -) (reg n) (const 2))
   (save continue)
   (assign continue (label afterfib-n-2))
   (save val)               ~\textrm{; Fib\( (n-1) \)を保存}~
   (goto (label fib-loop))
 afterfib-n-2     ~\textrm{; リターン時に, \code{val}がFib\( (n-2) \)を持つ}~
   (assign n (reg val))     ~\textrm{; \code{n}がここでFib\( (n-2) \)を持つ}~
   (restore val)            ~\textrm{; \code{val}がここでFib\( (n-1) \)を持つ}~
   (restore continue)

   (assign val              ~\textrm{; Fib\( (n-1) \) + Fib\( (n-2) \)}~
           (op +) (reg val) (reg n))
   (goto (reg continue))    ~\textrm{; 呼び出しから戻る, 答は\code{val}の中にある}~
 immediate-answer
   (assign val (reg n))     ~\textrm{; 基底の場合: Fib\( (n) = n \)}~
   (goto (reg continue))
 fib-done)
\end{scheme}

\end{quote}

\begin{quote}
\heading{\phantomsection\label{Exercise 5.4}Exercise 5.4:}
次の手続のそれぞれを実装するレジスタマシンを指定せよ。各マシンに対して、
コントローラ命令列を書き、データパスを示す図を描け。

\begin{enumerate}[a]

\item
再帰指数計算

\begin{scheme}
(define (expt b n)
  (if (= n 0)
      1
      (* b (expt b (- n 1)))))
\end{scheme}

\item
反復指数計算

\begin{scheme}
(define (expt b n)
  (define (expt-iter counter product)
    (if (= counter 0)
        product
        (expt-iter (- counter 1)
                   (* b product))))
  (expt-iter n 1))
\end{scheme}

\end{enumerate}
\end{quote}

\begin{quote}
\heading{\phantomsection\label{Exercise 5.5}Exercise 5.5:}
階乗とフィボナッチの機械をいくつかの非自明な入力を用いて手動でシミュレートせよ。
(少なくとも1回の再帰呼出を必要とする)。実行中の各重要な地点におけるスタックの中身を示せ。
\end{quote}

\begin{quote}
\heading{\phantomsection\label{Exercise 5.6}Exercise 5.6:}
Ben Bitdiddleはフィボナッチマシンのコントローラシーケンスが余分な\code{save}と\code{restore}を持ち、
より速いマシンを作るために取り除くことができることに気付いた。これらの命令はどこにあるか?
\end{quote}

\subsection{命令の要約}
\label{Section 5.1.5}

私達のレジスタマシン言語のコントローラ命令は以下の形式の内1つを持ち、各\( \langle \)\( input_i \)\( \rangle \)は
\code{(reg<\var{register\-/name}>)}か\code{(const <\var{constant\-/value}>)}の何れかです。
これらの命令は\link{Section 5.1.1}で導入されました。

\begin{scheme}
(assign ~\( \dark \langle \)~~\var{\dark register-name}~~\( \dark \rangle \)~ (reg ~\( \dark \langle \)~~\var{\dark register-name}~~\( \dark \rangle \)~))
(assign ~\( \dark \langle \)~~\var{\dark register-name}~~\( \dark \rangle \)~ (const ~\( \dark \langle \)~~\var{\dark constant-value}~~\( \dark \rangle \)~))
(assign ~\( \dark \langle \)~~\var{\dark register-name}~~\( \dark \rangle \)~ 
        (op ~\( \dark \langle \)~~\var{\dark operation-name}~~\( \dark \rangle \)~) 
        ~\( \dark \langle \)~~\( \dark input_1 \)~~\( \dark \rangle \)~ ~\( \dots \)~ ~\( \dark \langle \)~~\( \dark input_n \)~~\( \dark \rangle \)~)
(perform (op ~\( \dark \langle \)~~\var{\dark operation-name}~~\( \dark \rangle \)~) ~\( \dark \langle \)~~\( \dark input_1 \)~~\( \dark \rangle \)~ ~\( \dots \)~ ~\( \dark \langle \)~~\( \dark input_n \)~~\( \dark \rangle \)~)
(test (op ~\( \dark \langle \)~~\var{\dark operation-name}~~\( \dark \rangle \)~) ~\( \dark \langle \)~~\( \dark input_1 \)~~\( \dark \rangle \)~ ~\( \dots \)~ ~\( \dark \langle \)~~\( \dark input_n \)~~\( \dark \rangle \)~)
(branch (label ~\( \dark \langle \)~~\var{\dark label-name}~~\( \dark \rangle \)~))
(goto (label ~\( \dark \langle \)~~\var{\dark label-name}~~\( \dark \rangle \)~))
\end{scheme}

\noindent
レジスタを用いてラベルを保存することは\link{Section 5.1.3}で導入されました。

\begin{scheme}
(assign ~\( \dark \langle \)~~\var{\dark register-name}~~\( \dark \rangle \)~ (label ~\( \dark \langle \)~~\var{\dark label-name}~~\( \dark \rangle \)~))
(goto (reg ~\( \dark \langle \)~~\var{\dark register-name}~~\( \dark \rangle \)~))
\end{scheme}

\noindent
スタックを使用する命令は\link{Section 5.1.4}で導入されました。

\begin{scheme}
(save ~\( \dark \langle \)~~\var{\dark register-name}~~\( \dark \rangle \)~)
(restore ~\( \dark \langle \)~~\var{\dark register-name}~~\( \dark \rangle \)~)
\end{scheme}

\noindent
ここまでで見た\( \langle \)\var{constant-value}\( \rangle \)の種類は数値のみです。しかし
後程、文字列、シンボル、それにリストを使用します。
\begin{scheme}
(const "abc") ~\textrm{は文字列}~ "abc",
(const abc) ~\textrm{はシンボル}~ abc,
(const (a b c)) ~\textrm{はリスト}~ (a b c),
(const ()) ~\textrm{は空リスト}~
\end{scheme}

\section{レジスタマシンシミュレータ}
\label{Section 5.2}

レジスタマシンの設計を良く理解するために、私達は設計した機械を期待通りに実行されるか
確認するためにテストをする必要があります。設計のテストを行う1つの方法として
\link{Exercise 5.5}と同様にコントローラの命令を手動でシミュレートする方法があります。
しかしこれは簡単な機械を除いてとんでもなく退屈な方法です。この節ではレジスタマシン言語で
記述された機械のためのシミュレータを構築します。このシミュレータは4つのインターフェイス
手続を持つSchemeのプログラムです。1つ目はレジスタマシンの記述をマシンのモデルを構築する
ために利用します(データ構造の部品がシミュレートされるマシンの部品に対応します)。
残りの3つがモデルを操作することにより機械のシミュレーションを可能にします。

\begin{quote}

\begin{scheme}
(make-machine ~\( \dark \langle \)~~\var{\dark register-names}~~\( \dark \rangle \)~ ~\( \dark \langle \)~~\var{\dark operations}~~\( \dark \rangle \)~ ~\( \dark \langle \)~~\var{\dark controller}~~\( \dark \rangle \)~)
\end{scheme}

\noindent
与えられたレジスタ、命令、コントローラを持つ機械のモデルを構築し、返します。

\begin{scheme}
(set-register-contents! ~\( \dark \langle\kern0.08em \)~~\var{\dark machine-model}~~\( \dark \rangle \)~ 
                        ~\( \dark \langle \)~~\var{\dark register-name}~~\( \dark \rangle \)~ 
                        ~\( \dark \langle \)~~\var{\dark value}~~\( \dark \rangle \)~)
\end{scheme}

\noindent
与えられた機械でシミュレートされるレジスタに値を格納します。

\begin{scheme}
(get-register-contents ~\( \dark \langle\kern0.08em \)~~\var{\dark machine-model}~~\( \dark \rangle \)~ ~\( \dark \langle \)~~\var{\dark register-name}~~\( \dark \rangle \)~)
\end{scheme}

\noindent
与えられた機械のシミュレートされるレジスタの中身を返す。

\begin{scheme}
(start ~\( \dark \langle\kern0.08em \)~~\var{\dark machine-model}~~\( \dark \rangle \)~)
\end{scheme}

\noindent
与えられた機械の実行をシミュレートする。コントローラシーケンスの最初から開始し、
シーケンスの最後に辿り着いた時に停止する。
\end{quote}

\noindent
これらの手続がどのように利用されるかの例として、\link{Section 5.1.1}の\acronym{GCD}マシンの
モデルとなる\code{gcd\-/machine}を以下のように定義します。

\begin{scheme}
(define gcd-machine
  (make-machine
   '(a b t)
   (list (list 'rem remainder) (list '= =))
   '(test-b (test (op =) (reg b) (const 0))
            (branch (label gcd-done))
            (assign t (op rem) (reg a) (reg b))
            (assign a (reg b))
            (assign b (reg t))
            (goto (label test-b))
            gcd-done)))
\end{scheme}

\noindent
\code{make\-/machine}に対する最初の引数はレジスタ名のリストです。次の引数は各命令名と
その命令を実装するScheme手続(つまり、同じ入力値を与えられて同じ出力値を生成します)を
ペアにするテーブル(2要素リストのリスト)です。最後の引数は\link{Section 5.1}にあるように
ラベルと機械の命令(機械語)のリストとしてのコントローラを指定します。


この機械を用いて\acronym{GCD}を求めるために、入力レジスタを設定し、機械を開始し、
シミュレーションが停止した時に結果を検査します。

\begin{scheme}
(set-register-contents! gcd-machine 'a 206)
~\textit{done}~
(set-register-contents! gcd-machine 'b 40)
~\textit{done}~
(start gcd-machine)
~\textit{done}~
(get-register-contents gcd-machine 'a)
~\textit{2}~
\end{scheme}

\noindent
この演算はSchemeで書かれた\code{gcd}手続よりもとても遅く実行します。なぜなら
\code{assign}のような低レベルの機械語をより複雑な命令によりシミュレートするためです。

\begin{quote}
\heading{\phantomsection\label{Exercise 5.7}Exercise 5.7:}
シミュレータを用いて\link{Exercise 5.4}で自分で設計した機械をテストせよ。
\end{quote}



\subsection{マシンモデル}
\label{Section 5.2.1}

\code{make\-/machine}にて生成された機械のモデルは\link{Chapter 3}で開発された
メッセージパッシングの技術を用いた局所状態を持つ手続として表現されています。
このモデルを構築するために、\code{make\-/machine}は手続\code{make\-/new\-/machine}を呼び
全てのレジスタマシンに対して共通なマシンモデルの部品を構築することから始めます。
\code{make\-/new\-/machine}により構築されるこの基本的な機械のモデルは本質的には
いくつかのレジスタとスタックと、コントローラ命令を1つずつ処理する実行の
仕組みを一緒にしたコンテナです。

\code{make\-/machine}は次にこの基本的なモデルを(それに対してメッセージを送ることで)
拡張し、レジスタ、命令、定義される特定の機械のコントローラを含めます。
最初に新しい機械の中に与えられた各レジスタ名に対するレジスタを獲得し、指定された
命令をその機械にインストール(導入)します。次に\newterm{assembler}(\jnewterm{アセンブラ})(下記の
\link{Section 5.2.2}で説明されます)を用いてコントローラリストを新しい機械に対する命令に変換し、
これらを機械の命令列としてインストールします。\code{make\-/machine}はその値として変更された
機械のモデルを返します。

\begin{scheme}
(define (make-machine register-names ops controller-text)
  (let ((machine (make-new-machine)))
    (for-each
     (lambda (register-name)
       ((machine 'allocate-register) register-name))
     register-names)
    ((machine 'install-operations) ops)
    ((machine 'install-instruction-sequence)
     (assemble controller-text machine))
    machine))
\end{scheme}

\subsubsection*{レジスタ}

レジスタは\link{Chapter 3}の様に局所状態を持つ手続として表現されます。手続\code{make\-/register}は
アクセスと変更が可能な値を持つレジスタを作成します。

\begin{scheme}
(define (make-register name)
  (let ((contents '*unassigned*))
    (define (dispatch message)
      (cond ((eq? message 'get) contents)
            ((eq? message 'set)
             (lambda (value) (set! contents value)))
            (else
             (error "Unknown request: REGISTER" message))))
    dispatch))
\end{scheme}

\noindent
以下の手続はレジスタにアクセスするために使用されます。

\begin{scheme}
(define (get-contents register) (register 'get))
(define (set-contents! register value)
  ((register 'set) value))
\end{scheme}

\subsubsection*{スタック}


スタックもまた局所状態を持つ手続として表現されます。手続\code{make\-/stack}は
局所状態がスタック上のアイテム(項目)のリストから成るスタックを作成します。
スタックはスタック上にアイテムを\code{push}とスタックから最上位の
アイテムを取り去りそれを返す\code{pop}、スタックを空に初期化する\code{initialize}の
リクエストを受け付けます。

\begin{scheme}
(define (make-stack)
  (let ((s '()))
    (define (push x) (set! s (cons x s)))
    (define (pop)
      (if (null? s)
          (error "Empty stack: POP")
          (let ((top (car s)))
            (set! s (cdr s))
            top)))
    (define (initialize)
      (set! s '())
      'done)
    (define (dispatch message)
      (cond ((eq? message 'push) push)
            ((eq? message 'pop) (pop))
            ((eq? message 'initialize) (initialize))
            (else (error "Unknown request: STACK"
                         message))))
    dispatch))
\end{scheme}

\noindent
以下の手続はスタックへのアクセスに使用されます。

\begin{scheme}
(define (pop stack) (stack 'pop))
(define (push stack value) ((stack 'push) value))
\end{scheme}

\subsubsection*{基本的な機械}

\link{Figure 5.13}に示す\code{make\-/new\-/machine}手続は局所状態がスタック、初期値が空の
命令列、初期値がスタックを初期化する命令を持つ命令のリスト、初期値として2つのレジスタ
\code{flag}(フラグ)と\code{pc}(``program counter''、プログラムカウンタ)を持つ
\newterm{register table}(\jnewterm{レジスタテーブル})から成り立ちます。内部手続\code{lookup\-/register}は
テーブル内のレジスタを探します。

\code{flag}レジスタはシミュレートされる機械にて分岐をコントロールするために使用されます。
\code{test}命令は\code{flag}の中身にテストの結果(真、または、偽)を設定します。\code{branch}命令は
分岐するかしないかを\code{flag}の中身を調査して決定します。

\code{pc}レジスタは機械が実行する命令の順序付けを決定します。この順序付けは内部手続\code{execute}により
実装されています。シミュレーションモデルでは各機械命令は
\newterm{instruction execution procedure}(\jnewterm{命令実行手続})と呼ばれる引数無しの
手続を含むデータ構造であり、この手続を呼ぶことにより命令の実行をシミュレートします。
シミュレーションが実行されるにつれ、\code{pc}は次に実行される命令から始まる命令列の地点を
指します。\code{execute}はその命令を得て、それを命令実行手続を呼ぶことにより実行し、
このサイクルを実行する命令が無くなるまで(すなわち、\code{pc}が命令列の最後を
指すまで)繰り返します。

\newpage
\noindent
\heading{\phantomsection\label{Figure 5.13}Figure 5.13:}
\( \downarrow \) 基本の機械モデルを実装する\code{make\-/new\-/machine}手続

\begin{scheme}
(define (make-new-machine)
  (let ((pc (make-register 'pc))
        (flag (make-register 'flag))
        (stack (make-stack))
        (the-instruction-sequence '()))
    (let ((the-ops
           (list (list 'initialize-stack
                       (lambda () (stack 'initialize)))))
          (register-table
           (list (list 'pc pc) (list 'flag flag))))
      (define (allocate-register name)
        (if (assoc name register-table)
            (error "Multiply defined register: " name)
            (set! register-table
                  (cons (list name (make-register name))
                        register-table)))
        'register-allocated)
      (define (lookup-register name)
        (let ((val (assoc name register-table)))
          (if val
              (cadr val)
              (error "Unknown register:" name))))
      (define (execute)
        (let ((insts (get-contents pc)))
          (if (null? insts)
              'done
              (begin
                ((instruction-execution-proc (car insts)))
                (execute)))))
      (define (dispatch message)
        (cond ((eq? message 'start)
               (set-contents! pc the-instruction-sequence)
               (execute))
              ((eq? message 'install-instruction-sequence)
               (lambda (seq)
                 (set! the-instruction-sequence seq)))
              ((eq? message 'allocate-register) 
               allocate-register)
              ((eq? message 'get-register)
               lookup-register)
              ((eq? message 'install-operations)
               (lambda (ops)
                 (set! the-ops (append the-ops ops))))
              ((eq? message 'stack) stack)
              ((eq? message 'operations) the-ops)
              (else (error "Unknown request: MACHINE"
                           message))))
      dispatch)))
\end{scheme}

\noindent
工程の一部として、各命令の実行手続は\code{pc}を変更し次に実行される命令を指すようにします。
\code{branch}と\code{goto}命令は\code{pc}を変更し新しい行き先を指すようにします。
全ての他の命令は単純に\code{pc}を進めて列の次の命令を指すようにします。
各\code{execute}の呼出が\code{execute}を再び呼び出すことに中止して下さい。
これはしかし無限ループにはなりません。命令実行手続の実行は\code{pc}の中身を
変更するためです。

\code{make\-/new\-/machine}は\code{dispatch}手続を返します。これは内部の状態にアクセスする
メッセージパッシングを実装します。機械の開始は\code{pc}に命令列の最初を設定し、
\code{execute}を呼ぶことにより達成されることに注意して下さい。

利便性のために、機械の\code{start}命令の代替となる手続のインターフェイスを提供します。
同様に、レジスタの中身の設定、試験の手続も\link{Section 5.2}の最初にて指示されたように
提供します。

\begin{scheme}
(define (start machine) (machine 'start))
(define (get-register-contents machine register-name)
  (get-contents (get-register machine register-name)))
(define (set-register-contents! machine register-name value)
  (set-contents! (get-register machine register-name)
                 value)
  'done)
\end{scheme}

\noindent
これらの手続(と\link{Section 5.2.2}と\link{Section 5.2.3}の多くの手続)は以下を用いて与えられた機械と
レジスタ名のレジスタを探します。

\begin{scheme}
(define (get-register machine reg-name)
  ((machine 'get-register) reg-name))
\end{scheme}

\subsection{アセンブラ}
\label{Section 5.2.2}

アセンブラはコントローラの機械のための式の列を対応する機械の命令のリストへと変形します。
各命令はその実行手続を持ちます。概して、アセンブラは\link{Chapter 4}で学習した評価機に
とても似ています。入力言語が存在し(この場合にはレジスタマシン言語)、言語の式の各型に
対して適切なアクションを実行しなければなりません。

各命令のための実行手続を生成する技術は\link{Section 4.1.7}で実行時に実行から分析を分離する
ことで高速化するために用いたのものと同じです。\link{Chapter 4}で学んだように、
Schemeの式の多くの実用的な分析は変数の実際の値を知らなくとも実行することができました。
ここでも同様に、レジスタマシン言語の式の多くの実用的な分析が実際の機械のレジスタの
値を知ることなしに実行することができます。例えばレジスタへの参照をレジスタオブジェクトへの
ポインタにより置き換えたり、ラベルをラベルが指定する命令列内の地点へのポインタで
置き換えることができます。

アセンブラが命令実行手続を生成する前に、全てのテーブルが何を参照するのか知っておく
必要がります。そのためコントローラテキストを走査し命令からラベルを分離することから
始めます。アセンブラがテキストを走査するにつれ、命令のリストと各ラベルをそのリスト内部を指すポインタと
関連付けるテーブルの両方を構築します。そうしたらアセンブラは命令リストを各命令に
対する実行手続を挿入することで増補します。

\code{assemble}手続はアセンブラに対する主な入口です。コントローラテキストとマシンモデルを引数として
取り、モデルに格納するべき命令列を返します。\code{assemble}は\code{extract\-/labels}を呼び初期命令リストと
与えられたコントローラテキストからラベルテーブルを構築します。\code{extract\-/labels}の2つ目の引数は
これらの結果を処理するために呼ばれるべきものです。この手続は\code{update\-/insts!}を用いて命令実行
手続を再生し、それらを命令リストの中に挿入し、変更されたリストを返します。

\begin{scheme}
(define (assemble controller-text machine)
  (extract-labels
   controller-text
   (lambda (insts labels)
     (update-insts! insts labels machine)
     insts)))
\end{scheme}

\noindent
\code{extract\-/labels}は引数としてリスト\code{text}(コントローラ命令式の列)と
\code{receive}手続を取ります。\code{receive}は2つの値と共に呼び出されます。
(1) 命令データ構造のリスト\code{insts}はそれぞれが\code{text}からの命令を含みます。
(2) テーブル\code{labels}は\code{text}からの各ラベルとそのラベルが指定するリスト\code{insts}内の
位置とを関連付けします。

\begin{scheme}
(define (extract-labels text receive)
  (if (null? text)
      (receive '() '())
      (extract-labels
       (cdr text)
       (lambda (insts labels)
         (let ((next-inst (car text)))
           (if (symbol? next-inst)
               (receive insts
                        (cons (make-label-entry next-inst
                                                insts)
                              labels))
               (receive (cons (make-instruction next-inst)
                              insts)
                        labels)))))))
\end{scheme}

\noindent
\code{extract\-/labels}は連続して\code{text}の要素を走査し、\code{insts}と\code{labels}を集積する
ことで働きます。もし要素がシンボル(従ってラベル)なら適切なエントリが\code{labels}テーブルに
追加されます。そうでなければその要素は\code{insts}リスト上に集積されます。\footnote{
\code{receive}手続をここで使用するのは\code{extract\-/labels}を得て、効率的に2つの値、\code{labels}と
\code{insts}をそれを保持する複合データ構造を明示的に作ること無しに返すための方法です。
代替となる、明示的に値のペアを返す実装は以下の通りです。

\begin{smallscheme}
(define (extract-labels text)
  (if (null? text)
      (cons '() '())
      (let ((result (extract-labels (cdr text))))
        (let ((insts (car result)) (labels (cdr result)))
          (let ((next-inst (car text)))
            (if (symbol? next-inst)
                (cons insts
                      (cons (make-label-entry next-inst insts) 
                            labels))
                (cons (cons (make-instruction next-inst) insts)
                      labels)))))))
\end{smallscheme}

\noindent
これは\code{assemble}により以下のように呼び出されます。

\begin{smallscheme}
(define (assemble controller-text machine)
  (let ((result (extract-labels controller-text)))
    (let ((insts (car result)) (labels (cdr result)))
      (update-insts! insts labels machine)
      insts)))
\end{smallscheme}

\noindent
\code{receive}の使用は複数の値を返す洗練された手法の実演、または単純にプログラミング上のトリックを
見せ付けるための言い訳として考えることができます。\code{receive}のような次に実行されるべき手続引数は
``継続''と呼ばれます。\link{Section 4.3.3}で私達が継続を\code{amb}評価機のバックトラック制御構造の実装に
用いたのを思い出して下さい。
}


\code{update\-/insts!}は命令リストを変更します。これは初期値としては命令のテキストのみを
含みますが、対応する実行手続を含むようになります。

\begin{scheme}
(define (update-insts! insts labels machine)
  (let ((pc (get-register machine 'pc))
        (flag (get-register machine 'flag))
        (stack (machine 'stack))
        (ops (machine 'operations)))
    (for-each
     (lambda (inst)
       (set-instruction-execution-proc!
        inst
        (make-execution-procedure
         (instruction-text inst)
         labels machine pc flag stack ops)))
     insts)))
\end{scheme}

\noindent
マシン語データ構造は単純に命令テキストと対応する実行手続のペアを作ります。
実行手続は\code{extract\-/labels}が命令を構築した時にはまだ存在せず、
後に\code{update\-/insts!}により挿入されます。

\begin{scheme}
(define (make-instruction text) (cons text '()))
(define (instruction-text inst) (car inst))
(define (instruction-execution-proc inst) (cdr inst))
(define (set-instruction-execution-proc! inst proc)
  (set-cdr! inst proc))
\end{scheme}

\noindent
命令テキストはシミュレータでは使用されません。しかし、デバッグのために
手元に置いておくと便利です。(\link{Exercise 5.16}参照)


ラベルテーブルの要素はペアです。

\begin{scheme}
(define (make-label-entry label-name insts)
  (cons label-name insts))
\end{scheme}

\noindent
テーブル内の要素は以下により検索されます。

\begin{scheme}
(define (lookup-label labels label-name)
  (let ((val (assoc label-name labels)))
    (if val
        (cdr val)
        (error "Undefined label: ASSEMBLE"
               label-name))))
\end{scheme}

\begin{quote}
\heading{\phantomsection\label{Exercise 5.8}Exercise 5.8:}
以下のレジスタマシンのコードは曖昧である。ラベル\code{here}が複数回、定義されているためである。

\begin{scheme}
start
  (goto (label here))
here
  (assign a (const 3))
  (goto (label there))
here
  (assign a (const 4))
  (goto (label there))
there
\end{scheme}


シミュレータが書かれているままの状態で、レジスタ\code{a}の中身はコントローラが
\code{there}に辿り着いた時に何になるか? 手続\code{extract\-/labels}を変更し、同じラベル名が
2つの異なる地点を指し示すのに使用された場合にエラーを発するようにせよ。
\end{quote}

\subsection{各命令に対する実行手続の生成}
\label{Section 5.2.3}

アセンブラは命令の実行手続を生成するために\code{make\-/execution\-/procedure}を呼びます。
\link{Section 4.1.7}の評価機の\code{analyze}手続と同様に、これは適切な実行手続を生成するために
命令の型に従い呼出を行います。

\begin{scheme}
(define (make-execution-procedure 
         inst labels machine pc flag stack ops)
  (cond ((eq? (car inst) 'assign)
         (make-assign inst machine labels ops pc))
        ((eq? (car inst) 'test)
         (make-test inst machine labels ops flag pc))
        ((eq? (car inst) 'branch)
         (make-branch inst machine labels flag pc))
        ((eq? (car inst) 'goto)
         (make-goto inst machine labels pc))
        ((eq? (car inst) 'save)
         (make-save inst machine stack pc))
        ((eq? (car inst) 'restore)
         (make-restore inst machine stack pc))
        ((eq? (car inst) 'perform)
         (make-perform inst machine labels ops pc))
        (else
         (error "Unknown instruction type: ASSEMBLE"
                inst))))
\end{scheme}

\noindent
レジスタマシンの言語の命令の各型に対し、適切な実行手続を構築する生成器が存在します。
これらの手続の詳細がレジスタマシン言語の構文と個別の命令の意味の両方を決定します。
データ抽象化を用いることで全体的な実行の仕組みからレジスタマシンの式の詳細な構文を
分離しています。これは\link{Section 4.1.2}で評価機に対して行ったのと同様で、構文手続を用いて
命令の部分を抽出し、分類することによります。

\subsubsection*{\code{assign}命令}


\code{make\-/assign}手続は\code{assign}命令を扱います。

\begin{scheme}
(define (make-assign inst machine labels operations pc)
  (let ((target
         (get-register machine (assign-reg-name inst)))
        (value-exp (assign-value-exp inst)))
    (let ((value-proc
           (if (operation-exp? value-exp)
               (make-operation-exp
                value-exp machine labels operations)
               (make-primitive-exp
                (car value-exp) machine labels))))
      (lambda ()   ~\textrm{; \code{assign}に対する実行手続}~
        (set-contents! target (value-proc))
        (advance-pc pc)))))
\end{scheme}

\noindent
\code{make\-/assign}はターゲットとなるレジスタ名(命令の2つ目の要素)と値の式(命令を
構成するリストの残りの部分)を\code{assign}命令からセレクタを用いて抽出します。

\begin{scheme}
(define (assign-reg-name assign-instruction)
  (cadr assign-instruction))
(define (assign-value-exp assign-instruction)
  (cddr assign-instruction))
\end{scheme}

\noindent
レジスタ名が\code{get\-/register}を用いて検索され目的のレジスタオブジェクトを生成します。
値の式はもし値が命令の結果であるのなら\code{make\-/operation\-/exp}に渡され、そうでなければ
\code{make\-/primitive\-/exp}に渡されます。これらの手続(以下に示されます)は値の式を構文解析し
その値に対する実行手続を生成します。これは引数無しの手続で\code{value\-/proc}と呼ばれ、
シミュレーションの間にレジスタに代入される実際の値を生成するために評価されます。
レジスタ名の検索と値の式の構文解析の仕事はただ一度、アセンブリ時(アセンブラ実行時)に実行されることに注意して下さい。
その命令がシミュレートされる度に毎回ではありません。この仕事量の削減こそが私達が
実行手続を使用する理由です。そしてこれが直接\link{Section 4.1.7}の評価機において、実行から
プログラム分析を分離することにより仕事量の削減を得たことに対応します。

\code{make\-/assign}により返される結果は\code{assign}命令のための実行手続です。
この手続が(マシンモデルの\code{execute}手続により)呼ばれた時に、\code{value\-/proc}手続を
実行することにより得られた結果を目的のレジスタの中身に設定します。その次に\code{pc}を
以下の手続を実行することにより次の命令へと進めます。

\begin{scheme}
(define (advance-pc pc)
  (set-contents! pc (cdr (get-contents pc))))
\end{scheme}

\noindent
\code{advance\-/pc}は\code{branch}と\code{goto}を除く全ての命令に対する通常の終わりです。

\subsubsection*{\code{Test}, \code{branch}, \code{goto}命令}

\code{make\-/test}は\code{test}命令を同様な方法で扱います。これはテストされる条件を指定する
式を抽出し、それに対する実行手続を生成します。シミュレーション時に、条件のための
手続が呼ばれ、その結果が\code{flag}レジスタに割り当てられ、\code{pc}が進められます。

\begin{scheme}

(define (make-test inst machine labels operations flag pc)
  (let ((condition (test-condition inst)))
    (if (operation-exp? condition)
        (let ((condition-proc
               (make-operation-exp
                condition machine labels operations)))
          (lambda ()
            (set-contents! flag (condition-proc))
            (advance-pc pc)))
        (error "Bad TEST instruction: ASSEMBLE" inst))))
(define (test-condition test-instruction)
  (cdr test-instruction))
\end{scheme}

\noindent
\code{branch}命令のための実行手続は\code{flag}レジスタの中身をチェックし、\code{pc}の中身に
分岐の目的地を設定するか(分岐が選択された場合)、または単に\code{pc}を進めます
(分岐が選択されなかった場合)。\code{branch}命令内で指定された目的値はラベルでなければ
ならず、\code{make\-/branch}手続がこのことを強制することに注意して下さい。またラベルは
アセンブリ時に検索され、\code{branch}命令がシミュレートされる時に毎回検索される
訳ではないことにも注意して下さい。

\begin{scheme}
(define (make-branch inst machine labels flag pc)
  (let ((dest (branch-dest inst)))
    (if (label-exp? dest)
        (let ((insts
               (lookup-label
                labels 
                (label-exp-label dest))))
          (lambda ()
            (if (get-contents flag)
                (set-contents! pc insts)
                (advance-pc pc))))
        (error "Bad BRANCH instruction: ASSEMBLE" inst))))
(define (branch-dest branch-instruction)
  (cadr branch-instruction))
\end{scheme}

\noindent
\code{goto}命令は\code{branch}に似ていますが、目的地がラベルか、またはレジスタにより指定される
ことが異なります。また条件分岐ではありません。\code{pc}は常に新しい目的地に設定されます。

\begin{scheme}
(define (make-goto inst machine labels pc)
  (let ((dest (goto-dest inst)))
    (cond ((label-exp? dest)
           (let ((insts (lookup-label
                         labels
                         (label-exp-label dest))))
             (lambda () (set-contents! pc insts))))
          ((register-exp? dest)
           (let ((reg (get-register
                       machine
                       (register-exp-reg dest))))
             (lambda ()
               (set-contents! pc (get-contents reg)))))
          (else (error "Bad GOTO instruction: ASSEMBLE"
                       inst)))))
(define (goto-dest goto-instruction)
  (cadr goto-instruction))
\end{scheme}

\subsubsection*{他の命令}

スタック命令の\code{save}と\code{restore}は単純にスタックを指定したレジスタと共に
用いて、\code{pc}を進めます。

\begin{scheme}
(define (make-save inst machine stack pc)
  (let ((reg (get-register machine
                           (stack-inst-reg-name inst))))
    (lambda ()
      (push stack (get-contents reg))
      (advance-pc pc))))
(define (make-restore inst machine stack pc)
  (let ((reg (get-register machine
                           (stack-inst-reg-name inst))))
    (lambda ()
      (set-contents! reg (pop stack))
      (advance-pc pc))))
(define (stack-inst-reg-name stack-instruction)
  (cadr stack-instruction))
\end{scheme}

\noindent
\code{make\-/perform}で扱われる最後の命令型は実行されるべきアクションのための実行手続を
生成します。シミュレーション時にこのアクション手続が実行され\code{pc}は進められます。

\begin{scheme}
(define (make-perform inst machine labels operations pc)
  (let ((action (perform-action inst)))
    (if (operation-exp? action)
        (let ((action-proc
               (make-operation-exp
                action machine labels operations)))
          (lambda () (action-proc) (advance-pc pc)))
        (error "Bad PERFORM instruction: ASSEMBLE" inst))))
(define (perform-action inst) (cdr inst))
\end{scheme}

\subsubsection*{部分式の実行手続}

\code{reg}, \code{label}, または\code{const}式の値はレジスタへの代入(\code{make\-/assign})のため、
または演算命令の入力(下記の\code{make\-/operation\-/exp})のために必要になるかもしれません。
以下の手続はこれらの式のための値をシミュレーションの間に生成するための実行手続を
生成します。

\begin{scheme}
(define (make-primitive-exp exp machine labels)
  (cond ((constant-exp? exp)
         (let ((c (constant-exp-value exp)))
           (lambda () c)))
        ((label-exp? exp)
         (let ((insts (lookup-label
                       labels
                       (label-exp-label exp))))
           (lambda () insts)))
        ((register-exp? exp)
         (let ((r (get-register machine
                                (register-exp-reg exp))))
           (lambda () (get-contents r))))
        (else
         (error "Unknown expression type: ASSEMBLE" exp))))
\end{scheme}

\noindent
\code{reg}, \code{label}, \code{const}式の構文は以下により決定されます。

\begin{scheme}
(define (register-exp? exp) (tagged-list? exp 'reg))
(define (register-exp-reg exp) (cadr exp))
(define (constant-exp? exp) (tagged-list? exp 'const))
(define (constant-exp-value exp) (cadr exp))
(define (label-exp? exp) (tagged-list? exp 'label))
(define (label-exp-label exp) (cadr exp))
\end{scheme}

\noindent
\code{assign}, \code{perform}, \code{test}命令は(\code{op}式により指定される)機械の演算命令の
(\code{reg}と\code{const}式により指定される)いくつかのオペランドへの適用を含むかも
しれません。以下の手続は``演算命令式''---命令からの演算命令とオペランドの式を
含むリスト---に対する実行手続を生成します。

\begin{scheme}
(define (make-operation-exp exp machine labels operations)
  (let ((op (lookup-prim (operation-exp-op exp)
                         operations))
        (aprocs
         (map (lambda (e)
                (make-primitive-exp e machine labels))
              (operation-exp-operands exp))))
    (lambda ()
      (apply op (map (lambda (p) (p)) aprocs)))))
\end{scheme}

\noindent
演算命令式の構文は以下により決定されます。

\begin{scheme}
(define (operation-exp? exp)
  (and (pair? exp) (tagged-list? (car exp) 'op)))
(define (operation-exp-op operation-exp)
  (cadr (car operation-exp)))
(define (operation-exp-operands operation-exp)
  (cdr operation-exp))
\end{scheme}

\noindent
演算命令式の処理が\link{Section 4.1.7}の評価機において各オペランドに対して実行手続を生成したことにおいて
\code{analyze\-/application}手続による手続の適用の処理にとても似ていることに注意して下さい。
シミュレーション時に、オペランド手続を呼び、結果となる値に対して演算をシミュレートする
Scheme手続を適用します。シミュレーション手続は演算命令の名前を機械の演算命令テーブルから
検索することで見つかります。

\begin{scheme}
(define (lookup-prim symbol operations)
  (let ((val (assoc symbol operations)))
    (if val
        (cadr val)
        (error "Unknown operation: ASSEMBLE"
               symbol))))
\end{scheme}

\begin{quote}
\heading{\phantomsection\label{Exercise 5.9}Exercise 5.9:}
上記の機械の演算命令の取扱はそれらにラベル、定数、レジスタの中身上での演算を可能にする。
式を処理する手続を変更し、演算命令がレジスタと定数のみに対して使用できるような条件を
強制するようにせよ。
\end{quote}

\begin{quote}
\heading{\phantomsection\label{Exercise 5.10}Exercise 5.10:}
レジスタマシンの命令に新しい構文を設計し、シミュレータを変更してその新しい構文を使用せよ。
シミュレータの内、この節の構文手続以外を変更せずにあなたの新しい構文を実装することができるだろうか?
\end{quote}

\begin{quote}
\heading{\phantomsection\label{Exercise 5.11}Exercise 5.11:}
\link{Section 5.1.4}で\code{save}と\code{restore}を導入した時、以下の順の様に最後に保存した物ではない
レジスタに戻した場合に何が起こるのかは指定しなかった。

\begin{scheme}
(save y)  (save x)  (restore y)
\end{scheme}


\code{restore}の意味に対してはいくつかの妥当な可能性が存在する。

\begin{enumerate}[a]

\item
\code{(restore y)}はスタック上に最後に保存された値を、どのレジスタからその値が来たのか
関係無しに\code{y}に入れる。これが私達のシミュレータの振舞である。この振舞の利点の
活用法を示すため、\link{Section 5.1.4}のフィボナッチマシンから1つ命令を削減して見せよ。(\link{Figure 5.12})

\item
\code{(restore y)}はスタック上に最後に保存された値を\code{y}に入れる。しかしその
値が\code{y}から保存された場合のみである。そうでなければエラーを発する。
シミュレータを変更してこのように振る舞うようにせよ。\code{save}を変更して
スタック上に値と共にレジスタ名を保存しなければならない。

\item
\code{(restore y)}は\code{y}の後に他のどのレジスタが保存され、取り出されていなくても
最後に\code{y}から保存した値を\code{y}に入れる。シミュレータをこのように振る舞うように変更せよ。
分離されたスタックを各レジスタに関連付けする必要がある。また\code{initialize\-/stack}命令に
全てのレジスタのスタックを初期化させなければならない。

\end{enumerate}
\end{quote}

\begin{quote}
\heading{\phantomsection\label{Exercise 5.12}Exercise 5.12:}
シミュレータは与えられたコントローラと共に機械を実装するために必要とされるデータパスを
決定することを手助けするために利用することが可能である。アセンブラを拡張し以下の情報を
マシンモデルに格納せよ。

\begin{itemize}

\item
全ての命令のリストを重複を削除し、命令の型でソートする(\code{assign}, \code{goto}等)

\item
エントリポイントを持つのに使用されたレジスタの(重複の無い)リスト。(これらは
\code{goto}命令で参照されたレジスタである)

\item
\code{save}または\code{restore}されたレジスタの(重複の無い)リスト

\item
各レジスタに対し、代入元の(重複の無い)リスト。(例えば\link{Figure 5.11}の階乗マシンの
レジスタ\code{val}の入力元は\code{(const 1)}と\code{((op *) (reg n) (reg val))}).

\end{itemize}


メッセージパッシングの機械へのインターフェイスを拡張し、この新しい情報へのアクセスを提供せよ。
あなたの分析器をテストするために\link{Figure 5.12}のフィボナッチマシンを定義し、
構築されたリストを試験せよ。
\end{quote}

\begin{quote}
\heading{\phantomsection\label{Exercise 5.13}Exercise 5.13:}
シミュレータを変更することで、\code{make\-/machine}に対する引数としてレジスタのリストを
要求するのではなく、コントローラシーケンスを使用して機械がどんなレジスタを持つのか
決定するようにせよ。\code{make\-/machine}の中でレジスタを事前に獲得しておく代わりに、
命令のアセンブリ時の間に初めて現れた時に1つづつレジスタを獲得するようにせよ。
\end{quote}

\subsection{機械のパフォーマンスの監視}
\label{Section 5.2.4}

シミュレーションは提案された機械設計の正しさを確認するためだけではなく、
機械のパフォーマンスを計るためにも便利です。例えば、私達のシミュレータに演算中に
使用されるスタック命令の数を計る``メーター''を導入することができます。これを行うためには、
シミュレーションを行うスタックを変更しスタック上にレジスタが保存された回数とスタックが
到達した最大の深さを追跡するにし、スタックのインターフェイスにメッセージを追加し
以下のように統計を表示するようにします。また\code{make\-/new\-/machine}内の\code{the\-/ops}を以下の様に
初期化することで、基本的なマシンモデルにスタックの統計を表示する命令を追加します。

\begin{scheme}
(list (list 'initialize-stack
            (lambda () (stack 'initialize)))
      (list 'print-stack-statistics
            (lambda () (stack 'print-statistics))))
\end{scheme}

\noindent
以下が新しい版の\code{make\-/stack}です。

\begin{scheme}
(define (make-stack)
  (let ((s '())
        (number-pushes 0)
        (max-depth 0)
        (current-depth 0))
    (define (push x)
      (set! s (cons x s))
      (set! number-pushes (+ 1 number-pushes))
      (set! current-depth (+ 1 current-depth))
      (set! max-depth (max current-depth max-depth)))
    (define (pop)
      (if (null? s)
          (error "Empty stack: POP")
          (let ((top (car s)))
            (set! s (cdr s))
            (set! current-depth (- current-depth 1))
            top)))
    (define (initialize)
      (set! s '())
      (set! number-pushes 0)
      (set! max-depth 0)
      (set! current-depth 0)
      'done)
    (define (print-statistics)
      (newline)
      (display (list 'total-pushes  '= number-pushes
                     'maximum-depth '= max-depth)))
    (define (dispatch message)
      (cond ((eq? message 'push) push)
            ((eq? message 'pop) (pop))
            ((eq? message 'initialize) (initialize))
            ((eq? message 'print-statistics)
             (print-statistics))
            (else
             (error "Unknown request: STACK" message))))
    dispatch))
\end{scheme}

\noindent
\link{Exercise 5.15}から\link{Exercise 5.19}はレジスタマシンシミュレータに追加できる
他の便利な監視とデバッグの機能を説明します。

\begin{quote}
\heading{\phantomsection\label{Exercise 5.14}Exercise 5.14:}
\link{Figure 5.11}で示された階乗マシンを用いて様々な小さな値\( n \)に対する\( n! \)の
演算に必要とされるpushの数とスタックの最大深さを計れ。データから任意\( n > 1 \)に
対する\( n! \)を求めるのに使用されたpush命令の総数とスタックの最大深度に対する
\( n \)を用いた方程式を決定せよ。それぞれが\( n \)の線形関数であり、従って2つの
定数により決定されることに注意せよ。統計が表示されるために、階乗マシンをスタックを初期化する
命令と統計を表示する命令を拡張しなければならない。また機械を変更することで\( n \)に対する
値を繰り返し読み込み、階乗を求め、結果を表示できるようにしたいと思うかもしれないだろう(我々が
\link{Figure 5.4})で丁度\acronym{GCD}マシンに対して行ったように)。そうすることで繰り返し
\code{get\-/register\-/contents},\code{set\-/register\-/contents!}, \code{start}を起動する必要が無くなる。

\end{quote}

\begin{quote}
\heading{\phantomsection\label{Exercise 5.15}Exercise 5.15:}
レジスタマシンのシミュレータに\newterm{instruction counting}(\jnewterm{命令数カウンタ})を
追加せよ。これはマシンモデルに対して実行された命令数を追跡させる。マシンモデルの
インターフェイスを拡張し、命令カウンタの値を表示する物とカウンタをゼロにリセットする
新しいメッセージを受け入れるようにせよ。
\end{quote}

\begin{quote}
\heading{\phantomsection\label{Exercise 5.16}Exercise 5.16:}
シミュレータを拡張し\newterm{instruction tracing}(\jnewterm{命令トレーサ})を追加せよ。
これは各命令が実行される前に、シミュレータが命令のテキストを表示する。
マシンモデルに対しトレーサをon/offする\code{trace\-/on}と\code{trace\-/off}メッセージを受け入れるようにせよ。
\end{quote}

\begin{quote}
\heading{\phantomsection\label{Exercise 5.17}Exercise 5.17:}
\link{Exercise 5.16}の命令トレーサを拡張し命令を表示する前にシミュレータが
コントローラシーケンス内でその命令の直前のラベル表示するにせよ。
命令数カウンタ(\link{Exercise 5.15})に干渉しない方法で行うように注意すること。
シミュレータに必要なラベル情報を維持するようにすることが必要だろう。
\end{quote}

\begin{quote}
\heading{\phantomsection\label{Exercise 5.18}Exercise 5.18:}
\link{Section 5.2.1}の\code{make\-/register}手続を変更し、レジスタをトレース可能にせよ。
レジスタがトレースのon、offを行うメッセージを受け入れなければならない。
レジスタがトレースされている時、そのレジスタに対する代入はレジスタの名前、
レジスタの古い値、代入される新しい値が表示されること。マシンモデルへの
インターフェイスを拡張し指定された機械のレジスタに対するトレーサのon、offを
可能にせよ。
\end{quote}

\begin{quote}
\heading{\phantomsection\label{Exercise 5.19}Exercise 5.19:}
Alyssa P. Hackerはシミュレータ内に\newterm{breakpoint}(\jnewterm{ブレイクポイント})の機能を
欲しいと思った。それにより彼女の機械設計を手助けするためである。あなたが彼女のために
この機能を導入するために雇用された。彼女はコントローラシーケンス内でシミュレータが
停止する場所を指定し、機械の状態を調査することができるようにして欲しかった。
あなたは以下の手続を実装しようとしている。

\begin{scheme}
(set-breakpoint ~\( \dark \langle\kern0.08em \)~~\var{\dark machine}~~\( \dark \rangle \)~ ~\( \dark \langle \)~~\var{\dark label}~~\( \dark \rangle \)~ ~\( \dark \langle \)~~\var{\dark n}~~\( \dark \rangle \)~)
\end{scheme}

\noindent
これは与えられたラベルの後ろの\( n \)番目の命令の直前にブレイクポイントを設定する。
例えば、

\begin{scheme}
(set-breakpoint gcd-machine 'test-b 4)
\end{scheme}

\noindent
上の式はブレイクポイントを\code{gcd\-/machine}のレジスタ\code{a}への代入の直前にブレイクポイントを
導入する。シミュレータがブレイクポイントに到達する時、ラベルとブレークポイントのオフセットを
表示し、命令の実行を停止しなければなりません。するとAlyssaは\code{get\-/register\-/contents}と
\code{set\-/register\-/contents!}を用いてシミュレートされている機械の状態を操作することが可能になる。
次に彼女は以下を入力することで実行を続行できなければならない。

\begin{scheme}
(proceed-machine ~\( \dark \langle\kern0.08em \)~~\var{\dark machine}~~\( \dark \rangle \)~)
\end{scheme}


また特定のブレイクポイントを以下を用いて削除できなければならない。

\begin{scheme}
(cancel-breakpoint ~\( \dark \langle\kern0.08em \)~~\var{\dark machine}~~\( \dark \rangle \)~ ~\( \dark \langle \)~~\var{\dark label}~~\( \dark \rangle \)~ ~\( \dark \langle \)~~\var{\dark n}~~\( \dark \rangle \)~)
\end{scheme}

\noindent
または全てのブレイクポイントを削除するためには以下を用いる。

\begin{scheme}
(cancel-all-breakpoints ~\( \dark \langle\kern0.08em \)~~\var{\dark machine}~~\( \dark \rangle \)~)
\end{scheme}
\end{quote}

\section{記憶域の割当とガベージコレクション}
\label{Section 5.3}

\link{Section 5.4}ではレジスタマシンとしてのScheme評価機をどのように実装するかを示します。
議論を簡易化するために、私達のレジスタマシンは\newterm{list-structured memory}(\jnewterm{リスト構造メモリ})を
供えていると仮定します。この機械ではリスト構造のデータを操作する命令はプリミティブです。
そのようなメモリが存在するという仮定はSchemeインタプリタの制御の仕組みに集中する場合には
有用な抽象化です。しかしこれは現在のコンピュータの実際のプリミティブなデータ操作の
現実の光景を反映してはいません。Lispシステムがどのように動作するかのより完全な理解を得るためには、
リスト構造がどのように旧来のコンピュータのメモリに互換性のある方法で表現されるかについて
調査しなければなりません。

リスト構造の実装には2つの考慮点が存在します。1つは純粋に表現上の問題です。
Lispのペアによる``箱とポインタ''構造をストレージと典型的なコンピュータのメモリの
アドレス指定能力を用いてどのように表現するか。2つ目の問題は演算が進行するにつれての
メモリ管理に関係します。Lispシステムの動作は決定的に、継続して新しいデータオブジェクトを
作る能力に依存しています。これらは逐次実行されるLisp手続により明示的に作成されるオブジェクトと
同様に、インタプリタ自身により作成される環境や引数リストのような構造も含みます。
持続的な新しいデータオブジェクトの作成は無限の容量でかつ、高速にアドレス指定できるメモリを
持つコンピュータ上では問題を起こさないでしょうが、コンピュータのメモリは有限な量しか
ありません(残念なことに)。Lispシステムは従って無限のメモリという空想をサポートする
\newterm{automatic storage allocation}(\jnewterm{自動記憶域割当})の設備を提供します。
データオブジェクトが既に必要でなくなった時に、それに割り当てられたメモリは自動的に
リサイクルされ新しく構築されるデータオブジェクトに利用されます。そのような自動的な
記憶域割当を提供する多様な技術が存在します。この節で私達が議論する手法は
\newterm{garbage collection}(\jnewterm{ガベージコレクション}、ゴミ拾い)と呼ばれます。



\subsection{ベクタとしてのメモリ}
\label{Section 5.3.1}

伝統的なコンピュータのメモリは小さな部屋の配列だと考えることができます。
各部屋は情報の一片を入れることができます。各部屋は\newterm{address}(\jnewterm{アドレス})または
\newterm{location}(\jnewterm{位置})と呼ばれる個有の名前を持ちます。典型的なメモリシステムは
2つのプリミティブな命令を提供します。1つは指定された位置に格納されたデータを取り出し、
もう1つは指定された位置に新しいデータを割り当てます。メモリアドレスはある部屋の集合に
シーケンシャル(順)なアクセスをサポートするためにインクリメントすることができます。
より一般的には、多くの重要なデータの操作はメモリアドレスをデータして扱うことを要求します。
このデータはメモリ上の位置に格納でき、機械のレジスタ上で操作できなければいけません。
リスト構造の表現はそのような\newterm{address arithmetic}(\jnewterm{アドレス演算})の一つの応用です。

コンピュータメモリをモデル化するためには、\newterm{vector}(\jnewterm{ベクタ})と呼ばれる新しい
種類のデータ構造を用います。抽象的には、ベクタは複合データオブジェクトであり、その
個別の要素が整数の索引を用いて、索引から独立した時間量でアクセスすることができます。\footnote{メモリを
項目のリストとして表現することはできます。しかし、アクセス時間はその場合、索引から独立しません。
リストの\( n \)番目の要素へのアクセスが\( n - 1 \)回の\code{cdr}命令を必要とするためです。}
メモリ操作を説明するために、ベクタを扱うための2つのプリミティブなScheme手続を使用します。

\begin{itemize}

\item
\code{(vector\-/ref <\var{vector}> <\var{n}>)}はベクタの\( n \)番目の要素を返す。

\item
\code{(vector\-/set! <\var{vector}> <\var{n}> <\var{value}>)}はベクタの\( n \)番目の要素に
指定された値を設定する。

\end{itemize}

\noindent
例えば、\code{v}がベクタであるならば、\code{(vector\-/ref v 5)}はベクタ\code{v}の5番目の項目を
取得し、\code{(vector\-/set! v 5 7)}はベクタ\code{v}の5番目の項目の値を7に変更します。\footnote{完全にするには、
ベクタを構築する\code{make\-/vector}命令を指定するべきです。しかし、現在のアプリケーションではベクタを
コンピュータメモリの固定区域をモデル化するためにのみ使用します。}コンピュータメモリに対して、
このアクセスはアドレス演算を用いて、メモリ内のベクタの開始位置を指定する
\newterm{base address}(\jnewterm{ベース(基底)アドレス})とベクタの特定の項目のオフセットを指定する
\newterm{index}(\jnewterm{インデックス}、索引)を組み合わせることで実装することができます。

\subsubsection*{Lispデータの表現}

ベクタを用いてリスト構造メモリに対する基本的なペア構造を実装することができます。
コンピュータメモリが2つのベクタに分割されている所を想像してみましょう。
\code{the\-/cars}と\code{the\-/cdrs}です。私達は次のようにリスト構造を表現します。
ペアに対するポイントは2つのベクタへの索引です。ペアの\code{car}は\code{the\-/cars}に
指定した索引を用いた項目です。そしてペアの\code{cdr}は指定された索引を用いた\code{the\-/cdrs}の
項目です。またペア以外のオブジェクト(例えば数値やシンボル)に対する表現とデータの種類をお互いに見分ける
ための手法も必要になります。これを達成する方法は多数存在しますが、しかしそれらは
全て\newterm{typed pointers}(\jnewterm{型付きポインタ})の使用へと帰します。これはつまり、
``ポインタ''の概念を拡張しデータの型の情報を含めることです。\footnote{これは正確に
\link{Chapter 2}で紹介したジェネリック(総称)な命令を扱うための``タグ付きデータ''と同じ考えです。
ここではしかし、データの型はリストの使用を通して構築されるのでなく、プリミティブな機械レベルにて
含まれます。}データの型はシステムにペアのポインタ(``ペア''データ型とメモリベクタを
指す索引から成り立つ)を他の種類のデータへのポインタ(何らかの他のデータ型とその型を
表現するために利用された何かにより成り立つ)を見分けることを可能にします。
2つのデータオブジェクトはそれらのポインタが全く同じである場合に同じ(\code{eq?})だと
判断されます。\footnote{型情報はLispシステムが実装される機械の詳細に依存して多様な
方法でエンコード(encode、符号化)されるでしょう。Lispプログラムの実行効率はこの選択
がどれだけ明確に行われたかに強く依存します。しかし良い選択のための一般的な設計ルールを
形式化することは難しいことです。型付きポインタを実装する最も簡単な方法は固定長の
ビット集合を各ポインタの中でデータ型をエンコードする\newterm{type field}(\jnewterm{型フィールド})とする様に
割り当てしておくことです。そのような表現を設計するにおいて解決すべき重要な問題は
次を含みます。いくつの型ビットが必要とされるか? ベクタの索引の長さはどれだけ必要か?
どれだけ効率良くプリミティブな機械語命令がポインタの型フィールドの操作に使用できるか?
型フィールドを効率良く扱うための特別なハードウェアを含む機械は\newterm{tagged
architectures}(\jnewterm{タグアーキテクチャ})を持つと言われます。}\link{Figure 5.14}はこの手法を
用いてリスト\code{((1 2) 3 4)}を表現する場合を図示しています。その箱とポインタ図もまた
示されています。私達は文字接頭辞をデータ型情報を示すために使用しています。従って、
ペアに対する索引5を伴なうポインタは\code{p5}と示されます。空リストはポインタ\code{e0}で示されます。
そして数値4へのポインタは\code{n4}として示されます。箱とポインタ図において各ペアの左隅に
ペアの\code{car}と\code{cdr}がどこに格納されるかを指定するベクタの索引を表示しました。

数値へのポインタ、例えば\code{n4}は数値データを示す型と実際の数値4の表現から成り立つでしょう。\footnote{この
数値の表現上の決断はポインタの等価性をテストする\code{eq?}が数値の等値性の試験に使用できるか
どうかを決定します。もしポインタが数値それ自身を含む場合、等しい数値は同じポインタを持ちます。
しかしもしポインタが数値が格納される位置の索引を持つ場合、私達が同じ数を複数の位置に
格納しないことに注意しない場合に限り等しい数値が同じポインタになることが保証されます。}
単一のポインタのために獲得された固定長のメモリの中で表現されるには大き過ぎる数値を扱うためには、
独特な\newterm{bignum}(\jnewterm{ビッグナンバー})データ型を使うことができるでしょう。このためのポインタは
格納される数値の部分が格納されるリストを指定します。\footnote{これは丁度数値を数字の列として
書くのに似ています。ただし各``桁''が0から単一のポインタに格納できる最大の数の間になることが
異なります。}

\begin{figure}[tb]
\phantomsection\label{Figure 5.14}
\centering
\begin{comment}
\heading{Figure 5.14:} Box-and-pointer and memory-vector representations of the list \code{((1 2) 3 4)}.

\begin{example}
               +---+---+               +---+---+    +---+---+
((1 2) 3 4) -->| * | *-+-------------->| * | *-+--->| * | / |
               +-|-+---+               +-|-+---+    +-|-+---+
              1  |                    2  |         4  |
                 V                       V            V
               +---+---+    +---+---+  +---+        +---+
               | * | *-+--->| * | / |  | 3 |        | 4 |
               +-|-+---+    +-|-+---+  +---+        +---+
              5  |         7  |
                 V            V
               +---+        +---+
               | 1 |        | 2 |
               +---+        +---+

   Index   0    1    2    3    4    5    6    7    8    ...
         +----+----+----+----+----+----+----+----+----+----
the-cars |    | p5 | n3 |    | n4 | n1 |    | n2 |    | ...
         +----+----+----+----+----+----+----+----+----+----
the-cdrs |    | p2 | p4 |    | e0 | p7 |    | e0 |    | ...
         +----+----+----+----+----+----+----+----+----+----
\end{example}
\end{comment}
\includegraphics[width=91mm]{fig/chap5/Fig5.14a.pdf}
\begin{quote}
\heading{Figure 5.14:} リスト\code{((1 2) 3 4)}の``箱とポインタ''とメモリベクタの表現
\end{quote}
\end{figure}

\noindent
シンボルはその表示内容を形成する文字の列を指定する型付きポインタとして表現されることが
できるでしょう。この列はLispのreaderにより、最初に入力の中の文字列に出くわした時に
構築されます。2つのシンボルのインスタンスが\code{eq?}により``同じ''シンボルであると
認識されて欲しいことと、\code{eq?}にポインタの等価性のための簡単なテストになって欲しいこと
から、もしreaderが同じ文字列を2回見た場合、(同じ文字列に対する)同じポインタを両方の
出現に対して表現するために利用することを保証しなければなりません。これを達成するためには、
readerは伝統的に\newterm{obarray}(\jnewterm{オブジェクト配列})と呼ばれる出会った全てのシンボルの
表を管理します。readerが文字列に遭遇しシンボルを構築しようとする時、obarrayをチェックし
同じ文字列を以前に見ていないか確認します。もし初見であれば、文字列を用いて新しいシンボル
(新しい文字列に対する型付きポインタ)を構築し、このポインタをobarrayに挿入します。
もしreaderが既にその文字列を見ていれば、obarrayに格納されているシンボルのポインタを返します。
この文字列を一意なポインタで置き換える処理はシンボルの\newterm{interning}(\jnewterm{抑留})と呼ばれます。

\subsubsection*{プリミティブなリスト命令の実装}

上記の表現の構想を与えられた時に、レジスタマシンの各``プリミティブ''なリスト命令を
複数のプリミティブなベクタ命令で置き換えることができます。2つのレジスタ\code{the\-/cars}と
\code{the\-/cdrs}を用いてメモリベクタを特定し、\code{vector\-/ref}と\code{vector\-/set!}がプリミティブな
命令として有効であると仮定します。またポインタ上の演算命令(例えばポインタをインクリメントする、
ペアのポインタを用いてベクタを索引付けする、または2つの数値を足す)は型付きポインタの
索引部分しか利用しません。

\noindent
例えば、次の命令をサポートするレジスタマシンをその下の条件の下で作成することができます。

\begin{scheme}
(assign ~\( \dark \langle \)~~\( \dark reg_1 \)~~\( \dark \rangle \)~ (op car) (reg ~\( \dark \langle \)~~\( \dark reg_2 \)~~\( \dark \rangle \)~))
(assign ~\( \dark \langle \)~~\( \dark reg_1 \)~~\( \dark \rangle \)~ (op cdr) (reg ~\( \dark \langle \)~~\( \dark reg_2 \)~~\( \dark \rangle \)~))
\end{scheme}

\noindent
上の命令のそれぞれに対しこれらが実装されているとします。

\begin{scheme}
(assign ~\( \dark \langle \)~~\( \dark reg_1 \)~~\( \dark \rangle \)~ (op vector-ref) (reg the-cars) (reg ~\( \dark \langle \)~~\( \dark reg_2 \)~~\( \dark \rangle \)~))
(assign ~\( \dark \langle \)~~\( \dark reg_1 \)~~\( \dark \rangle \)~ (op vector-ref) (reg the-cdrs) (reg ~\( \dark \langle \)~~\( \dark reg_2 \)~~\( \dark \rangle \)~))
\end{scheme}

\noindent
以下の命令は、

\begin{scheme}
(perform (op set-car!) (reg ~\( \dark \langle \)~~\( \dark reg_1 \)~~\( \dark \rangle \)~) (reg ~\( \dark \langle \)~~\( \dark reg_2 \)~~\( \dark \rangle \)~))
(perform (op set-cdr!) (reg ~\( \dark \langle \)~~\( \dark reg_1 \)~~\( \dark \rangle \)~) (reg ~\( \dark \langle \)~~\( \dark reg_2 \)~~\( \dark \rangle \)~))
\end{scheme}

\noindent
次のように実装されます。

\begin{scheme}
(perform
 (op vector-set!) (reg the-cars) (reg ~\( \dark \langle \)~~\( \dark reg_1 \)~~\( \dark \rangle \)~) (reg ~\( \dark \langle \)~~\( \dark reg_2 \)~~\( \dark \rangle \)~))
(perform
 (op vector-set!) (reg the-cdrs) (reg ~\( \dark \langle \)~~\( \dark reg_1 \)~~\( \dark \rangle \)~) (reg ~\( \dark \langle \)~~\( \dark reg_2 \)~~\( \dark \rangle \)~))
\end{scheme}

\noindent
\code{cons}は未使用の索引を割り当て、\code{cons}の引数を\code{the\-/cars}と\code{the\-/cdrs}の中で
索引付けられたベクタの位置に格納します。私達は特別なレジスタ、\code{free}が存在し、
常に次に使用可能な索引を持つペアポインタを保つと仮定します。そしてそのポインタの
索引部分をインクリメントすることで次の空き位置を探すことができます。\footnote{空きの
記憶域を探す他の方法も存在します。例えば、全ての未使用のペアをリンクして
\newterm{free list}(\jnewterm{空きリスト})にすることもできたでしょう。私達の空き位置は連続的
(従ってポインタをインクリメントすることでアクセス可能であるため)です。なぜなら
私達が圧縮GCを用いているためです。また\link{Section 5.3.2}も参照して下さい。}例えば、以下の命令は

\begin{scheme}
(assign ~\( \dark \langle \)~~\( \dark reg_1 \)~~\( \dark \rangle \)~ (op cons) (reg ~\( \dark \langle \)~~\( \dark reg_2 \)~~\( \dark \rangle \)~) (reg ~\( \dark \langle \)~~\( \dark reg_3 \)~~\( \dark \rangle \)~))
\end{scheme}

\noindent
次の一連のベクタ命令として実装されます。\footnote{これは本質的に\link{Section 3.3.1}で説明した\code{set\-/car!}と
\code{set\-/cdr!}を用いた\code{cons}の実装です。その実装内で使用された命令\code{get\-/new\-/pair}は
ここでは\code{free}ポインタにより実現されています。}

\begin{scheme}
(perform
 (op vector-set!) (reg the-cars) (reg free) (reg ~\( \dark \langle \)~~\( \dark reg_2 \)~~\( \dark \rangle \)~))
(perform
 (op vector-set!) (reg the-cdrs) (reg free) (reg ~\( \dark \langle \)~~\( \dark reg_3 \)~~\( \dark \rangle \)~))
(assign ~\( \dark \langle \)~~\( \dark reg_1 \)~~\( \dark \rangle \)~ (reg free))
(assign free (op +) (reg free) (const 1))
\end{scheme}

\noindent
以下の\code{eq?}命令は

\begin{scheme}
(op eq?) (reg ~\( \dark \langle \)~~\( \dark reg_1 \)~~\( \dark \rangle \)~) (reg ~\( \dark \langle \)~~\( \dark reg_2 \)~~\( \dark \rangle \)~)
\end{scheme}

\noindent
単純にレジスタ内の全ての項目の等価性をテストします。そして
\code{pair?}, \code{null?}, \code{symbol?}, \code{number?}等のような述語は型フィールドのみを
確認する必要があります。

\subsubsection*{スタックの実装}

私達のレジスタマシンはスタックを用いますが、ここでは特に特別なことを行う必要がありません。
スタックはリストを用いてモデル化することができるためです。スタックは保存した値のリストと
することができ、特別なレジスタ\code{the\-/stack}により指し示されます。従って\code{
(save <\var{reg}>)}は以下のように実装することができます。

\begin{scheme}
(assign the-stack (op cons) (reg ~\( \dark \langle \)~~\var{\dark reg}~~\( \dark \rangle \)~) (reg the-stack))
\end{scheme}

\noindent
同様に、\code{(restore <\var{reg}>)}は次のように実装することができます。

\begin{scheme}
(assign ~\( \dark \langle \)~~\var{\dark reg}~~\( \dark \rangle \)~ (op car) (reg the-stack))
(assign the-stack (op cdr) (reg the-stack))
\end{scheme}

\noindent
そして\code{(perform (op initialize\-/stack))}は以下のように実装することができます。

\begin{scheme}
(assign the-stack (const ()))
\end{scheme}

\noindent
これらの命令は上で与えられたベクタ命令を用いてさらに伸展されます。しかし、伝統的な計算機
アーキテクチャにおいてはスタックを別のベクタとして割り当てることは通常は好都合です。
そうすれば、スタックにpushやpopを行うことはベクタに対する索引をインクリメント、
デクリメントすることにより達成することができます。

\begin{quote}
\heading{\phantomsection\label{Exercise 5.20}Exercise 5.20:}
以下の式から生成されるリスト構造の表現と(\link{Figure 5.14}にあるような)メモリ-ベクタ表現の
箱とポインタ図を描け。

\begin{scheme}
(define x (cons 1 2))
(define y (list x x))
\end{scheme}

\noindent
ただし、\code{free}ポインタの初期値は\code{p1}とする。\code{free}の最終的な値は何か?
どんなポインタが\code{x}と\code{y}の値を表現するか?
\end{quote}

\begin{quote}
\heading{\phantomsection\label{Exercise 5.21}Exercise 5.21:}
以下の手続のためのレジスタマシンを実装せよ。リスト構造のメモリ命令は機械のプリミティブと
して使用可能だと仮定せよ。

\begin{enumerate}[a]

\item
再帰\code{count\-/leaves}:

\begin{scheme}
(define (count-leaves tree)
  (cond ((null? tree) 0)
        ((not (pair? tree)) 1)
        (else (+ (count-leaves (car tree))
                 (count-leaves (cdr tree))))))
\end{scheme}

\item
明示的なカウンタを用いた再帰\code{count\-/leaves}

\begin{scheme}
(define (count-leaves tree)
  (define (count-iter tree n)
    (cond ((null? tree) n)
          ((not (pair? tree)) (+ n 1))
          (else
           (count-iter (cdr tree)
                       (count-iter (car tree) 
                                   n)))))
  (count-iter tree 0))
\end{scheme}
\end{enumerate}
\end{quote}

\begin{quote}
\heading{\phantomsection\label{Exercise 5.22}Exercise 5.22:}
\link{Section 3.3.1}の\link{Exercise 3.12}は2つのリストを接続し1つの新しいリストを形成する\code{append}手続と、
2つのリストを一緒に繋ぎ合わせる\code{append!}手続を紹介した。これらの手続それぞれを実装するレジスタ
マシンを設計せよ。リスト構造のメモリ命令はプリミティブな命令として使用可能と前提せよ。
\end{quote}

\subsection{無限のメモリの幻想を維持する}
\label{Section 5.3.2}

\link{Section 5.3.1}で概観した表現手法はリスト構造の実装上の問題を解決しましたが、無限の容量の
メモリを持っている場合という条件付きでした。実際のコンピュータではいつかは新しいペア
を構築するための空き容量を使い切ってしまいます。\footnote{これはいつかは正しくはなく
なるかもしれません。なぜならメモリが十分に大きくなればコンピュータの生存時間の間には
空きメモリを使い切ることは不可能になるかもしれないからです。例えば一年は
\( 3\cdot10^{\hbox{13}} \)マイクロ秒ですから、もし1マイクロ秒に1回\code{cons}を行うので
あれば、30年間はメモリを使い切ることのないコンピュータを構築するのには
約\( 10^{\hbox{15}} \)セルのメモリを必要とします。それだけのメモリは今日の標準では話にならない程大きく
見えますが、しかし物理的に不可能ではありません。一方で、プロセッサはより速くなりつつあり
未来のコンピュータは数多くのプロセッサを並列に単一のメモリ上で作動するかもしれません。
従って私達の前提よりもよい早くメモリを使い切ることが可能かもしれません。}しかし、
典型的な演算により生成されるペアの多くは中間結果を保つためだけに使用されます。これらの
結果がアクセスされた後には、それらのペアはもう必要ありません。それらは\newterm{garbage}(\jnewterm{ゴミ})です。
例えば、以下の演算は

\begin{scheme}
(accumulate + 0 (filter odd? (enumerate-interval 0 n)))
\end{scheme}

\noindent
2つのリストを構築します。enumaration(列挙)と列挙をフィルタリングした結果です。
accumulation(集積)が完了した時に、これらのリストはもう必要ありません。そして
割り当てられたメモリは返還要求できます。もし全てのゴミを定期的に回収する準備を
行えるのであれば、そしてもしこれが新しいペアを構築するのと大体同じ比率でメモリを
リサイクルすることになれば、無限の容量のメモリが存在するという錯覚を維持することが
できます。

ペアをリサイクルするためには、どの割り当てられたペアが必要でないか(それらの中身が
その後将来の演算に影響しないという意味で)決定する方法を持たねばなりません。
これを達成するために調査する手法は\newterm{garbage collection}(\jnewterm{ガベージコレクション}、GC)として
知られています。ガベージコレクションはLispの逐次実行における任意の時点で、将来の
演算に影響を与えることができるオブジェクトは現状で機械のレジスタ内に存在する
ポインタにより辿り着くことができるオブジェクトのみであるという観察結果に基いています。\footnote{ここでは
スタックは\link{Section 5.3.1}で説明されたリストとして表現されていると仮定しています。そのため
スタック上の項目はスタックレジスタ内のポインタを通してアクセスすることができます。}
そのようにアクセスできないどのメモリセルもリサイクルして良いでしょう。

ガベージコレクションを実行する方法は数多く存在します。ここで調査する手法は
\newterm{stop-and-copy}と呼ばれます。基本的な考えはメモリを2つに割ります。
``ワーキングメモリ''と``空きメモリ''です。\code{cons}がペアを構築する時、ワーキングメモリに
割り当てます。ワーキングメモリに空きが無い時、ワーキングメモリ内の使い道のある全てのペアを探し出し、
これらをフリーメモリ内の連続した位置にコピーすることでガベージコレクションを実行します。
(使い道のあるペアは機械のレジスタから始めて、全ての\code{car}と\code{cdr}のポインタを
追跡することにより探し出します)。ゴミはコピーしないため、推定上、新しいペアを
割り当てるための利用できる追加の空きメモリが存在するはずです。加えて、ワーキング
メモリ内の全てが必要ありません。その中の使い道のあるペアはコピーされています。
従ってワーキングメモリと空きメモリの役割を交換すれば、処理を続けることができます。
新しいペアは新しいワーキングメモリ(空きメモリだった物)の中に割り当てられます。
これがいっぱいになったなら、使い道のあるペアを新しい空きメモリ(ワーキングメモリだったもの)の
中にコピーできます。\footnote{
この考えはMinsky(ミンスキー)により発明され、\acronym{MIT}研究所の
電子工学ラボのPDP-1に対するLispの実装の一部として実装されました。
\link{Fenichel and Yochelson (1969)}によりMultics時分割システムのLisp実装で使用するために、
さらに開発が進められました。後に、\link{Baker (1978)}はこの手法の``リアルタイム''版を開発しました。
これはガベージコレクションの間に演算を停止する必要がありません。Bakerの考えは
Hewitt, Lieberman, Moonにより拡張され(\link{Lieberman and Hewitt 1983}参照)、
ある構造はvolatile(揮発性)であり、別の構造はより永続的であるといった事実を
活用する様になりました。

一般に利用される代替的なガベージコレクションの技術は\newterm{mark-sweep}(\jnewterm{マーク アンド スイープ})の
手法です。これは回帰のレジスタからアクセス可能な全ての構造の追跡と辿り着く各ペアへのマーキングから
成り立ちます。次に全てのメモリを走査し、マークの無い全てのメモリはゴミとして``掃き出し''、
再使用可能とされます。マーク アンド スイープの十分な議論は\link{Allen 1978}の中に見つけられます。

Minsky-Fenichel-Yochelsonアルゴリズムは巨大なメモリシステムに対する使用における
支配的なアルゴリズムです。メモリの使い道のある部分のみを調査するためです。
これはスイープの段階で全てのメモリを確認しなければならないmark-and-sweepとは対照的です。
stop-and-copyの2つ目の強みは\newterm{compacting}(\jnewterm{圧縮})ガベージコレクタであることです。
つまり、ガベージコレクションの段階の終わりには使い道のあるデータは連続したメモリ位置に
移動され、全てのゴミペアは圧縮の仮定で外に出されます。このことが仮想メモリを使用する
機械におけるパフォーマンス上の考慮において非常に重要と成り得ます。仮想メモリを使用する
機械は広範囲に分離されたメモリアドレスへのアクセスに余計なページング処理が必要となるかも
しれません。
}

\subsubsection*{stop-and-copyガベージコレクタの実装}

今から私達はレジスタマシン言語を用いてstop-and-copyアルゴリズムをより詳細に
記述します。私達は\code{root}と呼ばれるレジスタが存在し、ある構造体へのポインタを
保持し、そのポインタから最終的には全てのアクセス可能なデータを指し示すことが
できるという前提を行います。これはガベージコレクションを行う直前に全てのレジスタの
中身を事前に割り当てられたリストに格納し、\code{root}により指し示させることで
準備が行えます。\footnote{このレジスタのリストは記憶域割当システムのレジスタ---\code{root},
\code{the\-/cars}, \code{the\-/cdrs}, それにこの節で紹介される他のレジスタは含みません。}
私達はまた現在のワーキングメモリに加えて、使い道のあるデータをコピーできる空きメモリが
存在すると前提します。現在のワーキングメモリはベースアドレスが\code{the\-/cars}と\code{the\-/cdrs}と
呼ばれるレジスタに格納されるベクタから成り立ち、そして空きメモリは同様に
\code{new\-/cars}と\code{new\-/cdrs}と呼ばれるレジスタに格納されます。

ガベージコレクションは現在のワーキングメモリ内の空きセルが枯渇した時に引き起こされます。
それはつまり、\code{cons}命令が\code{free}ポインタをメモリベクタの終端を越えてインクリメント
しようとした時です。ガベージコレクションの処理が完了した時、\code{root}ポインタは
新しいメモリの中を指し示し、\code{root}からアクセス可能な全てのオブジェクトは新しいメモリに
移動されています。そして\code{free}ポインタは新しいメモリ内の新しいペアを割り当てられる
次の位置を示します。加えて、ワーキングメモリと新しいメモリの役割が交換されます。
新しいペアは\code{free}により指し示される位置から始まる新しいメモリ内に構築され、
(以前の)ワーキングメモリは次のガベージコレクションに対する新しいメモリとして
使用可能となります。\link{Figure 5.15}はガベージコレクション直前、直後のメモリの
割り振りを示します。

\begin{figure}[tp]
\phantomsection\label{Figure 5.15}
\centering
\begin{comment}
\begin{quote}
\heading{Figure 5.15:} Reconfiguration of memory by the garbage-collection process.

\begin{example}
             Just before garbage collection

         +------------------------------------+
the-cars |                                    | working
         | mixture of useful data and garbage | memory
the-cdrs |                                    |
         +------------------------------------+
                                            ^
                                            | free

         +------------------------------------+
new-cars |                                    | free   
         |            free memory             | memory
new-cdrs |                                    |
         +------------------------------------+

             Just after garbage collection

         +------------------------------------+
new-cars |                                    | new  
         |          discarded memory          | free  
new-cdrs |                                    | memory
         +------------------------------------+

         +------------------+-----------------+
the-cars |                  |                 | new
         |   useful data    |    free area    | working
the-cdrs |                  |                 | memory
         +------------------+-----------------+
                              ^
                              | free
\end{example}
\end{quote}
\end{comment}
\includegraphics[width=91mm]{fig/chap5/Fig5.15a.pdf}
\begin{quote}
\heading{Figure 5.15:} ガベージコレクションプロセスによるメモリの再構成
\end{quote}
\end{figure}

\noindent
ガベージコレクション処理の状態は2つのポインタを管理することによりコントロールされています。
\code{free}と\code{scan}です。これらは新しいメモリの開始位置を指し示すように初期化されます。
アルゴリズムは\code{root}により指し示されるペアの新しいメモリの開始位置への再配置から開始されます。
ペアはコピーされ、\code{root}ポインタは新しい位置を指すように調整されます。そして
\code{free}ポインタがインクリメントされます。併せて、ペアの古い位置はその中身が移動された
ことを示すマークが付けられます。このマーキングは次のように行われます。\code{car}の位置には
これが既に移動されたオブジェクトであることを示す特別なタグを置きます。(そのような
オブジェクトは伝統的に\newterm{broken heart}(失恋)と呼ばれます。)\footnote{\emph{broken heart}と
いう用語はDavid Cresseyにより作られました。彼は1970年代初期の間に\acronym{MIT}で開発された
Lispの方言、MDLのためにガベージコレクタを書きました。}\code{cdr}の位置には
\newterm{forwarding address}(\jnewterm{転送先})を置きます。これはオブジェクトの移動先の位置を指し示します。

\code{root}の再配置の後に、ガベージコレクタは基本となるサイクルに入ります。
アルゴリズムの各ステップにおいて、\code{scan}ポインタ(初期値として再配置後の\code{root}を指す)は、
新しいメモリに移動されたがその\code{car}と\code{cdr}のポインタが依然として古いメモリ内の
オブジェクトを参照しているペアを指します。これらのオブジェクトはそれぞれが再配置され、
\code{scan}ポインタはインクリメントされます。オブジェクト(例えば走査しているペアの\code{car}ポインタにより
指されたオブジェクト)を再配置するためにはそのオブジェクトが既に移動されていないかを
(そのオブジェクトの\code{car}の位置内にbroken-heartタグが存在するこにより示されていないか)確認します。
もしオブジェクトがまだ移動されていなければ、それを\code{free}により示される位置にコピーし、
\code{free}を更新し、オブジェクトの古い位置にbroken-heartを設定し、そのオブジェクトへの
ポインタを(この礼では、走査しているペアの\code{car}ポインタを)更新し、新しい位置を指すようにします。
もしオブジェクトが既に移動されている場合には、(broken heartの\code{cdr}の位置に見つかる)その移動先は
走査中のペアのポインタに置き換えられます。最終的には、\code{scan}
ポインタが\code{free}ポインタを追い越す時点まで、全てのアクセス可能なオブジェクトは移動され、
精査されます。そして処理は停止します。

stop-and-copyアルゴリズムをレジスタマシンの命令列として記述することができます。
オブジェクトの再配置の基本的なステップは\code{relocate\-/old\-/result\-/in\-/new}と呼ばれるサブルーチンにて
達成されます。このサブルーチンはその引数として再配置するオブジェクトのポインタを\code{old}という名の
レジスタから取得します。これは指定されたオブジェクトを再配置し、(処理の間に\code{free}をインクリメントし)、
再配置されたオブジェクトを指すポインタを\code{new}と呼ばれるレジスタに入れます。
そして\code{relocate\-/continue}レジスタに格納されたエントリポイントへ分岐することで帰ります。
ガベージコレクションを始めるために、このサブルーチンを起動して、\code{free}と\code{scan}を初期化した後に
\code{root}ポインタを再配置します。\code{root}の再配置が完了した時に、\code{new}ポインタを新しい\code{root}と
して導入し、ガベージコレクタのメインループに入ります。

\begin{scheme}
begin-garbage-collection
  (assign free (const 0))
  (assign scan (const 0))
  (assign old (reg root))
  (assign relocate-continue (label reassign-root))
  (goto (label relocate-old-result-in-new))
reassign-root
  (assign root (reg new))
  (goto (label gc-loop))
\end{scheme}

\noindent
ガベージコレクタのメインループでは走査すべきオブジェクトが残っているのか決定しなければ
なりません。これを\code{scan}ポインタが\code{free}ポインタと一致するかどうかを試験することで
行います。もしポインタが等しければ、全てのアクセス可能なオブジェクトの再配置は完了し、
\code{gc\-/flip}へと分岐します。ここは後片付けを行い、割り込みが行われた演算を継続します。
もしまだ走査すべきペアが残っているのであれば、再配置(relocate)のサブルーチンを呼び出し
次のペアの\code{car}を(\code{old}内の\code{car}ポインタを配置することで)再配置します。
\code{relocate\-/continue}レジスタの設定によりサブルーチンは\code{car}ポインタを更新するために
帰ります。

\begin{scheme}
gc-loop
  (test (op =) (reg scan) (reg free))
  (branch (label gc-flip))
  (assign old (op vector-ref) (reg new-cars) (reg scan))
  (assign relocate-continue (label update-car))
  (goto (label relocate-old-result-in-new))
\end{scheme}

\noindent
\code{update\-/car}にて、精査しているペアの\code{car}ポインタを変更します。次にペアの
\code{cdr}を再配置するために向かいます。再配置が完了すると\code{update\-/cdr}に帰ってきます。
再配置と\code{cdr}の更新の後に、そのペアの精査を完了しメインループを継続します。

\begin{scheme}
update-car
  (perform (op vector-set!)
           (reg new-cars)
           (reg scan)
           (reg new))
  (assign old (op vector-ref) (reg new-cdrs) (reg scan))
  (assign relocate-continue (label update-cdr))
  (goto (label relocate-old-result-in-new))

update-cdr
  (perform (op vector-set!)
           (reg new-cdrs)
           (reg scan)
           (reg new))
  (assign scan (op +) (reg scan) (const 1))
  (goto (label gc-loop))
\end{scheme}

\noindent
サブルーチン\code{relocate\-/old\-/result\-/in\-/new}はオブジェクトを次のように再配置します。
もし(\code{old}により指し示される)再配置すべきオブジェクトがペアでないなら、そのオブジェクトへの
同じポインタを変更無しで(\code{new}の中で)返します。(例えば、\code{car}が数値の4であるペアを精査して
いるとします。もし\link{Section 5.3.1}にて説明されているように\code{n4}で\code{car}を表現するのなら、
``再配置された''\code{car}のポインタも依然として\code{n4}であって欲しいと願うはずです)。
そうでなければ、再配置を実行しなければなりません。もしペアの再配置すべき\code{car}の位置に
broken-heartタグを持つのならば、そのペアは実際には既に移動されています。従って
(broken-heartの\code{cdr}の位置から)移動先を取得し、これを\code{new}に入れて返します。
もし\code{old}内のポインタがまだ移動されていないペアを指す場合、そのペアを(\code{free}が指し示す)
新しいメモリの最初の空きセルに移動させ、broken-heartタグと移動先を元の位置に格納することで
broken-heartを設定します。\code{relocate\-/old\-/result\-/in\-/new}はレジスタ\code{oldcr}を用いて
\code{old}により指し示されるオブジェクトの\code{car}または\code{cdr}を保持します。\footnote{ガベージ
コレクタは低レベルの述語\code{pointer\-/to\-/pair?}をリスト構造\code{pair?}命令の代わりに使用します。
実際のシステムでは様々な物がガベージコレクションの目的のためにペアとして扱われるためです。
例えば、\acronym{IEEE}標準に準拠するSchemeシステムでは手続オブジェクトは特別な種類の``ペア''
として実装されても良くこれは述語\code{pair?}は満たしません。シミュレーションの目的には、
\code{pointer\-/to\-/pair?}は\code{pair?}として実装できます。}

\begin{scheme}
relocate-old-result-in-new
  (test (op pointer-to-pair?) (reg old))
  (branch (label pair))
  (assign new (reg old))
  (goto (reg relocate-continue))
pair
  (assign oldcr (op vector-ref) (reg the-cars) (reg old))
  (test (op broken-heart?) (reg oldcr))
  (branch (label already-moved))
  (assign new (reg free)) ~\textrm{; ペアの新しい位置}~
  ~\textrm{;; \code{free}ポインタを更新する}~
  (assign free (op +) (reg free) (const 1))
  ~\textrm{;; \code{car}と\code{cdr}を新しいメモリにコピーする.}~
  (perform (op vector-set!)
           (reg new-cars) (reg new) (reg oldcr))
  (assign oldcr (op vector-ref) (reg the-cdrs) (reg old))
  (perform (op vector-set!)
           (reg new-cdrs) (reg new) (reg oldcr))
  ~\textrm{;; ブロークンハートの構築}~
  (perform (op vector-set!)
           (reg the-cars) (reg old) (const broken-heart))
  (perform
   (op vector-set!) (reg the-cdrs) (reg old) (reg new))
  (goto (reg relocate-continue))
already-moved
  (assign new (op vector-ref) (reg the-cdrs) (reg old))
  (goto (reg relocate-continue))
\end{scheme}

\noindent
ガベージコレクション処理の最後に、メモリの新旧の役割をポインタを交換することにより
交代します。\code{the\-/cars}と\code{new\-/cars}、そして\code{the\-/cdrs}を\code{new\-/cdrs}を交換します。
これで次回メモリが枯渇した時にもう一度ガベージコレクションを行う準備ができます。

\begin{scheme}
gc-flip
  (assign temp (reg the-cdrs))
  (assign the-cdrs (reg new-cdrs))
  (assign new-cdrs (reg temp))
  (assign temp (reg the-cars))
  (assign the-cars (reg new-cars))
  (assign new-cars (reg temp))
\end{scheme}

\section{明示的制御評価機}
\label{Section 5.4}
\link{Section 5.1}では簡単なSchemeプログラムをどのようにレジスタマシンの記述に変形するかについて学びました。
ここではこの変形をより複雑なプログラム上で実行します。\link{Section 4.1.1}--\link{Section 4.1.4}のメタ循環評価機です。
メタ循環評価機はSchemeインタプリタの振舞が手続\code{eval}と\code{apply}と用いてどのように説明できるかを示しました。
この節で開発する\newterm{explicit-control evaluator}(\jnewterm{明示的制御評価機})は評価過程にて使用される
潜在的な手続呼出と引数受け渡しの仕組みがレジスタとスタックの命令を用いてどのように説明できるかを
示します。付け加えて、明示的制御評価機はSchemeインタプリタの実装としての役割を果たすことができ、
従来の計算機の生来の機械語ととてもそっくりな言語で書かれています。この評価機は\link{Section 5.2}のレジスタ
マシンシミュレータにより実行することができます。あるいは、Scheme評価機の機械語実装を構築するための
開始点として使用することができます。またはSchemeの式を評価するための特殊用途の機械にすら使用できるでしょう。
\link{Figure 5.16}はそのようなハードウェア実装を示しています。Schemeの評価機として働くシリコンチップです。
このチップの設計者はこの節の中で説明される評価機に似たレジスタマシンに対するデータパスとコントローラの
仕様から開始しました。そしてIC(integrated-circuit、統合回路)を構築するための設計自動化プログラムを
使用しました。\footnote{このチップとその設計手法についてのより多くの情報については\link{Batali et al. 1982}を
参照して下さい。}

\begin{figure}[tb]
\phantomsection\label{Figure 5.16}
\centering
\begin{comment}
\heading{Figure 5.16:} A silicon-chip implementation of an evaluator for Scheme.

[A photograph of chip layout]

\end{comment}
\includegraphics[width=91mm]{fig/chap5/chip.jpg}
\begin{quote}
\heading{Figure 5.16:} Scheme評価機のシリコンチップ実装
\end{quote}
\end{figure}

\subsubsection*{レジスタと命令}

明示的制御評価機の設計では、私達のレジスタマシンで使用される命令を指定しなければなりません。
私達は\code{quoted?}や\code{make\-/procedure}のような手続を用いることで抽象構文を用いたメタ循環評価機を
説明しました。レジスタマシンの実装においてはこれらの手続を初歩的なリスト構造メモリの命令列に
展開することができました。そしてこれらの命令を私達のレジスタマシン上に実装しました。
しかし、これは私達の評価機の基本的な構造をその詳細によりわかりにくくしながら、とても長くして
しまいます。その表現を明快にするために、レジスタマシンのプリミティブな命令として
\link{Section 4.1.2}で与えられた構文手続と環境を表現するための手続、それに\link{Section 4.1.3}と\link{Section 4.1.4}で与えられた
実行時データを含めることにします。低レベルのマシン語でプログラミングできる、またはハードウェア
にて実装できる評価機の完全な仕様化を行うために、\link{Section 5.3}で説明したリスト構造の実装を用いて
これらの命令をより基本的な命令により置き換えることができるでしょう。

私達のScheme評価機レジスタマシンはスタックと7つのレジスタを含みます。
\code{exp}, \code{env}, \code{val}, \code{continue}, \code{proc}, \code{argl}, \code{unev}です。
\code{exp}は評価される式の保持に仕様され、\code{env}は評価がその中で実行される環境を
持ちます。評価の終わりには、\code{val}が指定された環境における式の評価により
得られた値を保持します。\code{continue}レジスタは\link{Section 5.1.4}で説明されたように
再帰の実装に用いられます。(評価機はそれ自身を再帰的に呼び出す必要があります。
式の評価はその部分式の評価を必要とするためです)。レジスタ\code{proc}, \code{argl}, \code{unev}は
組み合わせの評価に用いられます。

私達はデータパス図を評価機のレジスタと命令がどのように接続されているかを示すために
提供はしません。また機械の命令の完全なリストの提供も行ないません。これらは評価機の
コントローラに暗黙的に存在し、コントローラの詳細が与えられます。



\subsection{明示制御評価機の核}
\label{Section 5.4.1}

評価機の中心的な要素は\code{eval\-/dispatch}で始まる命令列です。これは\link{Section 4.1.1}で説明された
メタ循環評価機の\code{eval}手続に対応します。コントローラが\code{eval\-/dispatch}から開始する時、
\code{exp}により指定された式を、\code{env}により指定された環境にて評価します。
評価が完了した時には、コントローラは\code{continue}に格納されたエントリポイントに飛びます。
その時、\code{val}レジスタが式の値を保持しています。メタ循環の\code{eval}と同様に、
\code{eval\-/dispatch}の構造は評価される式の構文型上の事例分析です。\footnote{私達のコントローラに
おいては、ディスパッチ(dispatch、割り振り)は\code{test}と\code{branch}の命令列として書かれています。
代替法として、データ適従スタイルで書くこともできるでしょう(そして実際のシステムは恐らくそうされて
いるでしょう)。連続したテストの実行の必要を防ぎ、新しい式の型の定義を用意にするためです。
Lispを実行するように設計された機械は恐らく\code{dispatch\-/on\-/type}命令を含むことでしょう。
これはそのようなデータに従った割り振りを効率的に実行します。}

\begin{scheme}
eval-dispatch
  (test (op self-evaluating?) (reg exp))
  (branch (label ev-self-eval))
  (test (op variable?) (reg exp))
  (branch (label ev-variable))
  (test (op quoted?) (reg exp))
  (branch (label ev-quoted))
  (test (op assignment?) (reg exp))
  (branch (label ev-assignment))
  (test (op definition?) (reg exp))
  (branch (label ev-definition))
  (test (op if?) (reg exp))
  (branch (label ev-if))
  (test (op lambda?) (reg exp))
  (branch (label ev-lambda))
  (test (op begin?) (reg exp))
  (branch (label ev-begin))
  (test (op application?) (reg exp))
  (branch (label ev-application))
  (goto (label unknown-expression-type))
\end{scheme}

\subsubsection*{単純な式の評価}

数値と文字列(これらは自己評価です)、変数、クォーテーション、そして\code{lambda}式は
評価するべき部分式がありません。これらのために、評価機は単純に正しい値を\code{val}レジスタに
配置し、\code{continue}により指定されたエントリポイントから実行を継続します。
単純な式の評価は以下のコントローラのコードにより実行されます。

\begin{scheme}
ev-self-eval
  (assign val (reg exp))
  (goto (reg continue))
ev-variable
  (assign 
    val (op lookup-variable-value) (reg exp) (reg env))
  (goto (reg continue))
ev-quoted
  (assign val (op text-of-quotation) (reg exp))
  (goto (reg continue))
ev-lambda
  (assign unev (op lambda-parameters) (reg exp))
  (assign exp (op lambda-body) (reg exp))
  (assign val (op make-procedure)
              (reg unev) (reg exp) (reg env))
  (goto (reg continue))
\end{scheme}

\noindent
\code{ev\-/lambda}がどのように\code{unev}と\code{exp}レジスタを用いてラムダ式のパラメタと
ボディを保持し、\code{env}の中の環境と共に\code{make\-/procedure}命令に引き渡されるのか観察して下さい。

\subsubsection*{手続適用の評価}

手続の適用はオペレータとオペランドを含む組み合わせにより指定します。
オペレータはその値が手続となる部分式であり、オペランドはその値が引数となる部分式で、
その引数に対して手続が適用されねばなりません。メタ循環の\code{eval}は適用をそれ自身を
再帰的に呼び出すことで扱い、組み合わせの各要素を評価し、そして結果を\code{apply}に渡します。
これが実際の手続適用を実行します。明示的制御評価機も同じことを行います。
これらの再帰呼出は\code{goto}命令と共に、スタックを使用して再帰呼出から戻った時に
再格納されるようにレジスタを保存することで実装されます。各呼出の前にどのレジスタが
保存されなければならないのかの確認に注意をしなければなりません。(なぜならこれらの値が
後で必要になるからです)。\footnote{これは重要ですが、アルゴリズムをLispの様な手続型の言語から
レジスタマシンの言語へ翻訳する場合において微妙な点です。必要な物だけを保存することの
代替法として、各再帰呼出の前に全てのレジスタ(\code{val}を除く)を保存することもできます。
これは\newterm{framed-stack}(\jnewterm{スタックフレーム})の統制と呼ばれます。これはうまく行きますが
しかし必要以上のレジスタを保存します。このことはスタック命令が高価であるというシステム内の
懸念点に成り得ます。後に使用される必要のないレジスタの保存はまた使用価値の無いデータを
手放さないことにも成り得ます。これはそうでなければガベージコレクションされ、再使用される
ために領域が解法されたはずです。}

適用の評価はオペレータを評価し手続を生成することから開始します。手続は後に評価された
オペランドに適用されます。オペレータを評価するためには、それを\code{exp}レジスタに
移動させ、\code{eval\-/dispatch}へ飛びます。\code{env}レジスタ内の環境は既にその中でオペレータを
評価するために適切な物になっていますが、それでも\code{env}を保存します。オペランドの評価にも
必要なためです。またオペランドを\code{unev}の中に展開し、スタック上にこれを保存します。
\code{continue}に対し\code{eval\-/dispatch}がオペレータの評価が完了した後に\code{ev\-/appl\-/did\-/operator}にて
resume(再開)できるように設定します。しかし、最初に\code{continue}の古い値は保存します。
これがコントローラに対し適用後にどこから続行するのかを告げるためです。

\begin{scheme}
ev-application
  (save continue)
  (save env)
  (assign unev (op operands) (reg exp))
  (save unev)
  (assign exp (op operator) (reg exp))
  (assign continue (label ev-appl-did-operator))
  (goto (label eval-dispatch))
\end{scheme}

\noindent
オペレータ部分式の評価からの帰還すると、組み合わせのオペランドの評価と、
結果としての引数を\code{argl}に保持されるリストの中への蓄積へと進みます。
最初に未評価のオペランドと環境を戻します。\code{argl}を空リストに初期化します。
そして\code{proc}レジスタにオペレータの評価により生成された手続を割り当てます。
もしオペランドが無ければ、直接\code{apply\-/dispatch}へと進みます。そうでなければ、
\code{proc}をスタックに保存し引数評価ループを開始します。\footnote{
\link{Section 4.1.3}の評価機データ構造の手続に以下の2つの手続を引数リストの操作のために追加します。

\begin{smallscheme}
(define (empty-arglist) '())
(define (adjoin-arg arg arglist) (append arglist (list arg)))
\end{smallscheme}


また追加の構文手続を使用して組み合わせの最後のオペランドであるかのテストを行います。

\begin{smallscheme}
(define (last-operand? ops) (null? (cdr ops)))
\end{smallscheme}
}

\begin{scheme}
ev-appl-did-operator
  (restore unev)                       ~\textrm{; the operands}~
  (restore env)
  (assign argl (op empty-arglist))
  (assign proc (reg val))              ~\textrm{; the operator}~
  (test (op no-operands?) (reg unev))
  (branch (label apply-dispatch))
  (save proc)
\end{scheme}

\noindent
引数評価ループの各サイクルは\code{unev}の中のリストからオペランドを評価し、その結果を\code{argl}の
中に蓄積します。オペランドを評価するために、それを\code{exp}レジスタの中に入れ、
実行が引数蓄積段階から再開できるよう\code{continueを}設定した後に\code{eval\-/dispatch}に飛びます。
しかし、最初に私達はそれまでに(\code{argl}に保持されている)蓄積された引数、環境(\code{env}に維持)、
評価されていない残りのオペランド(\code{unev}が保持)を保存します。最後のオペランドの評価は
特別な場合として扱われ\code{ev\-/appl\-/last\-/arg}により取り扱われます。

\begin{scheme}
ev-appl-operand-loop
  (save argl)
  (assign exp (op first-operand) (reg unev))
  (test (op last-operand?) (reg unev))
  (branch (label ev-appl-last-arg))
  (save env)
  (save unev)
  (assign continue (label ev-appl-accumulate-arg))
  (goto (label eval-dispatch))
\end{scheme}

\noindent
オペランドが評価された時に、その値は\code{argl}にて保持されるリストの中に蓄積されます。
そのオペランドはその後\code{unev}中の未評価オペランドのリストから消され、引数評価が
続行されます。

\begin{scheme}
ev-appl-accumulate-arg
  (restore unev)
  (restore env)
  (restore argl)
  (assign argl (op adjoin-arg) (reg val) (reg argl))
  (assign unev (op rest-operands) (reg unev))
  (goto (label ev-appl-operand-loop))
\end{scheme}

\noindent
最後の引数の評価は異なる扱いを受けます。環境や未評価のオペランドのリストを
\code{eval\-/dispatch}に飛ぶ前に保存する必要がありません。最後のオペランドが評価された後に
それらは必要が無いためです。従って評価から特別なエントリポイント\code{ev\-/appl\-/accum\-/last\-/arg}に
帰ります。これは引数リストを戻し、新しい引数を蓄積し、保存された手続を戻し、適用
を実行するために飛びます。\footnote{最後のオペランドの処理の特別な最適化は
\newterm{evlis tail recursion}(\jnewterm{エブリス末尾再帰})として知られています(\link{Wand 1980}参照)。
最初のオペランドも特別な場合とすれば、私達は引数評価ループをいくらかより効率的良くできたでしょう。
これは\code{argl}の初期化を最初のオペランドの評価の後まで延期することができ、この場合に\code{argl}を
保存することを防げたでしょう。\link{Section 5.5}のコンパイラはこの最適化を実行します。
(\link{Section 5.5.3}の\code{construct\-/arglist}手続と比較して下さい。)}

\begin{scheme}
ev-appl-last-arg
  (assign continue (label ev-appl-accum-last-arg))
  (goto (label eval-dispatch))
ev-appl-accum-last-arg
  (restore argl)
  (assign argl (op adjoin-arg) (reg val) (reg argl))
  (restore proc)
  (goto (label apply-dispatch))
\end{scheme}

\noindent
引数評価ループの詳細はインタプリタが組み合わせのオペランドを評価する順を決定します。(例えば、
左から右や右から左---\link{Exercise 3.8}参照)。この順はメタ循環評価機では決定されません。
メタ循環評価機はその制御構造をその基礎を成し実装を行うSchemeから継承します。\footnote{メタ循環評価機の
オペランドの評価順は\link{Section 4.1.1}の手続\code{list\-/of\-/values}内の\code{cons}への引数の評価順により決定されます
(\link{Exercise 4.1}参照)。}(\code{ev\-/appl\-/operand\-/loop}内で一連のオペランドを\code{unev}から抽出するために使用された)
\code{first\-/operand}セレクタは\code{car}として実装され、\code{rest\-/operands}は\code{cdr}として実装され、
明示的制御評価機は組み合わせのオペランドを左から右への順で評価します。

\subsubsection*{手続適用}

エントリポイント\code{apply\-/dispatch}はメタ循環評価機の\code{apply}手続に対応します。
\code{apply\-/dispatch}に到達する時に、\code{proc}レジスタは適用するための手続を持ち、
\code{argl}は適用するべき評価された引数のリストを持ちます。
(元々は\code{eval\-/dispatch}に渡され、\code{ev\-/application}で保存された)\code{continue}の保存された値は
手続適用の結果と共に帰る場所を伝えますが、スタック上に存在します。適用が完了した時に、
コントローラは保存された\code{continue}により指示されたエントリポイントへ、\code{val}内の
適用の結果と共に移動します。メタ循環の\code{apply}と同様に、考慮すべき2つの場合が存在します。
適用すべき手続はプリミティブであるか、または複合手続であるかです。

\begin{scheme}
apply-dispatch
  (test (op primitive-procedure?) (reg proc))
  (branch (label primitive-apply))
  (test (op compound-procedure?) (reg proc))
  (branch (label compound-apply))
  (goto (label unknown-procedure-type))
\end{scheme}

\noindent
各プリミティブは引数を\code{argl}から取得し、その結果を\code{val}内に置くように実装されていると
想定されます。機械がどのようにプリミティブを扱うかを指定するためには、それぞれのプリミティブを
実装するための一連のコントローラ命令を提供しなければならず、\code{proc}の中身により判別された
プリミティブのための命令への割り振りを行うように\code{primitive\-/apply}を準備しなければなりません。
私達はプリミティブの詳細ではなく、評価処理の構造に興味があるため、それらの代わりに
単に\code{apply\-/primitive\-/procedure}を使用します。これは\code{proc}内の手続を\code{argl}内の引数に対して
適用します。\link{Section 5.2}のシミュレータを用いて評価機のシミュレーションを行う目的のために、
私達は手続\code{apply\-/primitive\-/procedure}を使用します。これは根底にあるSchemeシステムを
適用を実行するために呼び出します。私達が\link{Section 4.1.4}のメタ循環評価機で行ったのと全く同じです。
プリミティブの適用の値を計算した後に、\code{continue}を戻して指定されたエントリポイントに飛びます。

\begin{scheme}
primitive-apply
  (assign val (op apply-primitive-procedure)
              (reg proc)
              (reg argl))
  (restore continue)
  (goto (reg continue))
\end{scheme}

\noindent
複合手続を適用するためには、メタ循環評価機と全く同様に進行します。
手続のパラメタを引数に束縛するフレームを構築し、このフレームを用いて手続により
運ばれた環境を拡張し、この拡張環境の中で手続のボディを形成する式の列を評価します。
\link{Section 5.4.2}で説明される\code{ev\-/sequence}は列の評価を取り扱います。

\begin{scheme}
compound-apply
  (assign unev (op procedure-parameters) (reg proc))
  (assign env (op procedure-environment) (reg proc))
  (assign env (op extend-environment)
              (reg unev) (reg argl) (reg env))
  (assign unev (op procedure-body) (reg proc))
  (goto (label ev-sequence))
\end{scheme}

\noindent
\code{compound\-/apply}は\code{env}レジスタが新しい値を割り当てられるインタプリタ内で唯一の場所です。
メタ循環評価機と同様に、新しい環境は手続により運ばれた環境から引数リストと対応する束縛される
変数のリストと共に構築されます。

\subsection{列の評価と末尾再帰}
\label{Section 5.4.2}

明示的制御評価機の\code{ev\-/sequence}の部分はメタ循環評価機の\code{eval\-/sequence}手続と同等です。
手続のボディ内の式、または明示的な\code{begin}式内の列を取り扱います。

明示的な\code{begin}式は\code{unev}内に評価されるべき式の列を配置し、\code{continue}をスタック上に保存し、
\code{ev\-/sequence}に飛ぶことで評価されます。

\begin{scheme}
ev-begin
  (assign unev (op begin-actions) (reg exp))
  (save continue)
  (goto (label ev-sequence))
\end{scheme}

\noindent
手続のボディ内の暗黙的な列は\code{compound\-/apply}から\code{ev\-/sequence}へと飛ぶことで
扱われます。この時点で\code{continue}は既に\code{ev\-/application}で保存され、スタック上に存在します。

\code{ev\-/sequence}と\code{ev\-/sequence\-/continue}のエントリポイントはループを形成し、
連続して列内の各式を評価します。未評価の式のリストは\code{unev}に保持されています。
各式の評価の前に、列内にさらなる評価すべき式が存在しないかどうか確認します。
もしそうであれば、(\code{unev}に保持された)未評価の式の残りと、(\code{env}に保持された)
式の残りが評価される環境を保存し、その式を評価するために\code{eval\-/dispatch}を呼びます。
2つの保存されたレジスタはこの評価からの帰還時に\code{ev\-/sequence\-/continue}にて戻されます。

列内の最後の式は\code{ev\-/sequence\-/last\-/exp}にて異なる取扱を行います。
この後には評価すべき式は無いため、\code{unev}と\code{env}を\code{eval\-/dispatch}に行く前に
保存する必要はありません。列全体の値は最後の式の値であるため、最後の式の評価の後に
スタック上に現時点で保存されている(\code{ev\-/application}または\code{ev\-/begin}にて保存された)
エントリポイントから続行すること以外に必要なことはありません。
\code{continue}を設定して\code{eval\-/dispatch}からここに帰るように準備し、次にスタックから
\code{continue}の値を戻してそのエントリポイントから続行するのではなく、\code{eval\-/dispatch}へ
行く前にスタックから\code{continue}を戻します。そうすることで\code{eval\-/dispatch}は
式を評価した後にそのエントリポイントから続行します。

\begin{scheme}
ev-sequence
  (assign exp (op first-exp) (reg unev))
  (test (op last-exp?) (reg unev))
  (branch (label ev-sequence-last-exp))
  (save unev)
  (save env)
  (assign continue (label ev-sequence-continue))
  (goto (label eval-dispatch))
ev-sequence-continue
  (restore env)
  (restore unev)
  (assign unev (op rest-exps) (reg unev))
  (goto (label ev-sequence))
ev-sequence-last-exp
  (restore continue)
  (goto (label eval-dispatch))
\end{scheme}

\subsubsection*{末尾再帰}


\link{Chapter 1}にて以下のような手続により記述されるプロセスは、

\begin{scheme}
(define (sqrt-iter guess x)
  (if (good-enough? guess x)
      guess
      (sqrt-iter (improve guess x)
                 x)))
\end{scheme}

\noindent
反復プロセスだと述べました。例えこの手続が構文的に(それ自身の単語の定義において)再帰であっても、
論理的には評価機にとって、1つの\code{sqrt\-/iter}の呼出から次の呼出への横断において情報を保存する
必要がありません。\footnote{\link{Section 5.1}にてそのようなプロセスをどのようにスタックを持たない
レジスタマシンにて実装するのかを学びました。プロセスの状態は固定長のレジスタ集合に格納されます。}
\code{sqrt\-/iter}のような手続を、手続がそれ自身を呼び出すことを続けるに従い記憶域を増加させる
必要無しに実行することが可能な評価機は、\newterm{tail-recursive}(\jnewterm{末尾再帰})評価機と呼ばれます。
\link{Chapter 4}のメタ循環評価機の実装は評価機が末尾再帰であるかどうかを指定しませんでした。
その評価機が状態を保存するための仕組みをその基礎に横たわるSchemeから継承していたためです。
しかし明示的制御評価機と用いる場合、私達は評価の過程を追跡し、いつ手続呼出が正味の情報集積を
スタック上に引き起すのかを確認することができます。

私達の評価機は末尾再帰です。なぜなら列の最後の式を評価するために、スタック上に
何の情報も保存すること無く\code{eval\-/dispatch}へと直接飛びます。従って、
列の最後の式---例えもしそれが手続呼出であっても(\code{sqrt\-/iter}のように、手続のボディの最後の
式が\code{if}式であっても、\code{sqrt\-/iter}への呼出へと簡約されます)---の評価がスタック上に
何の情報の蓄積も起こしません。\footnote{この\code{ev\-/sequence}における末尾再帰の実装は
多くのコンパイラで使用されている良く知られた最適化の技術の一種です。手続呼出で
終了する手続のコンパイルでは、呼出を呼び出された手続のエントリポイントへのジャンプで
置き換えることができます。この節で行ったように、この戦略をインタプリタの中に構築することは
言語の至る所に均一に最適化を提供します。}

もしこの場合に情報を保存する必要が無いという事実を活用することを考えなかった場合、
列内の全ての式を同じように取り扱うように\code{eval\-/sequence}を実装していたことでしょう。
レジスタの保存、式の評価、レジスタを戻すために帰る、これらを全ての式が評価されるまで
繰り返したことでしょう。\footnote{We can
define \code{no\-/more\-/exps?} as follows:

\code{no\-/more\-/exps?}を以下のように定義することができます。

\begin{smallscheme}
(define (no-more-exps? seq) (null? seq))
\end{smallscheme}
}

\begin{scheme}
ev-sequence
  (test (op no-more-exps?) (reg unev))
  (branch (label ev-sequence-end))
  (assign exp (op first-exp) (reg unev))
  (save unev)
  (save env)
  (assign continue (label ev-sequence-continue))
  (goto (label eval-dispatch))
ev-sequence-continue
  (restore env)
  (restore unev)
  (assign unev (op rest-exps) (reg unev))
  (goto (label ev-sequence))
ev-sequence-end
  (restore continue)
  (goto (reg continue))
\end{scheme}

\noindent
恐らくこれは列の評価のための以前のコードに対する軽微な変更のように見えるでしょう。
唯一の違いは保存と再格納のサイクルを他と同様に列の最後の式でも通すことです。
インタプリタは依然としてどの式に対しても同じ値を与えます。しかし、この変更は
末尾再帰の実装に対しては致命的です。なぜなら、これで私達は列の最後の式の評価の後にも
(使用価値の無い)レジスタの保存を戻すために帰らねばなりません。
これらの余分な保存は入れ子の手続の呼出の間で蓄積されます。その結果として、\code{sqrt\-/iter}の
ようなプロセスは一定容量を必要とするのでなく、繰り返しの回数に比例する記憶域を必要と
します。この違いは重大事に成り得ます。例えば、末尾再帰を用いれば、無限ループは
手続呼出の仕組みだけを用いて表現できます。

\begin{scheme}
(define (count n)
  (newline) (display n) (count (+ n 1)))
\end{scheme}

\noindent
末尾再帰が無ければ、そのような手続はいつかはスタック領域を使いつくします。
そして真に反復を表現することは手続呼出以外の何らかの制御の仕組みを必要とします。

\subsection{条件文、代入、定義}
\label{Section 5.4.3}

メタ循環評価機と同様に、特殊形式は選択的に式の部分部分を評価することで取り扱われます。
\code{if}式に対しては、述語を評価して、その値を元に、結果部(consequent)と代替部(alternative)の
どちらを評価するか決定します。

述語を評価する前に、\code{if}式自身を保存します。そうすることで後に結果部か代替部を抽出することが
できます。また後に結果部か代替部を評価するため必要となるので環境を保存します。
そして後に\code{if}の値を待っている式の評価に戻るために必要なため\code{continue}も保存します。

\begin{scheme}
ev-if
  (save exp)                    ~\textrm{; 後のため式を保存する}~
  (save env)
  (save continue)
  (assign continue (label ev-if-decide))
  (assign exp (op if-predicate) (reg exp))
  (goto (label eval-dispatch))  ~\textrm{; 述語を評価する}~
\end{scheme}

\noindent
述語の評価から戻る時、真か偽であるかをテストし、結果に依り\code{eval\-/dispatch}に飛ぶ前に
\code{exp}に結果部か代替部を配置します。\code{env}と\code{continue}をここで戻すことが
\code{eval\-/dispatch}に正しい環境を持たせ、正しい場所から継続し\code{if}式の値を受けとるように
設定していることに注意して下さい。

\begin{scheme}
ev-if-decide
  (restore continue)
  (restore env)
  (restore exp)
  (test (op true?) (reg val))
  (branch (label ev-if-consequent))
ev-if-alternative
  (assign exp (op if-alternative) (reg exp))
  (goto (label eval-dispatch))
ev-if-consequent
  (assign exp (op if-consequent) (reg exp))
  (goto (label eval-dispatch))
\end{scheme}

\subsubsection*{代入と定義}

代入と定義は\code{ev\-/assignment}により扱われます。ここには\code{eval\-/dispatch}から代入式が\code{exp}の
中にある状態で到達します。\code{ev\-/assignment}の最初のコードは式の部分の値を評価し、次に
新しい値を環境に導入します。\code{Set\-/variable\-/value!}が機械語命令として必要可能であると
前提します。

\begin{scheme}
ev-assignment
  (assign unev (op assignment-variable) (reg exp))
  (save unev)                   ~\textrm{; 後のため変数を保存}~
  (assign exp (op assignment-value) (reg exp))
  (save env)
  (save continue)
  (assign continue (label ev-assignment-1))
  (goto (label eval-dispatch))  ~\textrm{; 代入値を評価する}~
ev-assignment-1
  (restore continue)
  (restore env)
  (restore unev)
  (perform
   (op set-variable-value!) (reg unev) (reg val) (reg env))
  (assign val (const ok))
  (goto (reg continue))
\end{scheme}

\noindent
定義も同様に扱われます。

\begin{scheme}
ev-definition
  (assign unev (op definition-variable) (reg exp))
  (save unev)                   ~\textrm{; 変数を後のため保存}~
  (assign exp (op definition-value) (reg exp))
  (save env)
  (save continue)
  (assign continue (label ev-definition-1))
  (goto (label eval-dispatch))  ~\textrm{; 定義値を評価する}~
ev-definition-1
  (restore continue)
  (restore env)
  (restore unev)
  (perform
   (op define-variable!) (reg unev) (reg val) (reg env))
  (assign val (const ok))
  (goto (reg continue))
\end{scheme}

\begin{quote}
\heading{\phantomsection\label{Exercise 5.23}Exercise 5.23:}
評価機を拡張し、\code{cond}, \code{let}, 等の派生式を取り扱うようにせよ(\link{Section 4.1.2})。
\code{cond\-/>if}の様な構文変換器が機械語命令として使用可能と前提して``インチキ''しても良い。\footnote{これは
本当はインチキではありません。実際のゼロからの実装においても、Schemeを解釈する明示的制御評価機を用いて
\code{cond\-/>if}のようなソースレベル変換を実行前の構文フェーズにて実行するでしょう。}
\end{quote}

\begin{quote}
\heading{\phantomsection\label{Exercise 5.24}Exercise 5.24:}
\code{cond}を新しい基本的な特殊形式として\code{if}に簡約すること無しに実装せよ。
連続する\code{cond}節の述語を真になるものを見つけるまでテストするループを構築する必要がある。
次に\code{ev\-/sequence}を使用してその節のアクションを評価する。
\end{quote}

\begin{quote}
\heading{\phantomsection\label{Exercise 5.25}Exercise 5.25:}
評価機を変更し、\link{Section 4.2}の遅延評価機を基にした正規順評価を使用するようにせよ。
\end{quote}

\subsection{評価機を実行する}
\label{Section 5.4.4}

明示的制御評価機の実装と共に、私達は\link{Chapter 1}から始まった開発の終わりにやってきました。
ここまで私達は引き続きより的確な評価過程のモデルを探求してきました。比較的、略式な
置換モデルから開始し、次にこれを\link{Chapter 3}で環境モデルに拡張しました。これは
状態と変更を扱うことを可能にしました。\link{Chapter 4}のメタ循環評価機ではScheme自身を
式の評価の間に構築される、より明確な環境構造のための言語として使用しました。
ここでは、レジスタマシンを用いてメモリ管理、引数渡し、制御のための評価機の仕組みについて
つぶさに見てきました。それぞれの新しいレベルの説明にて、直前の、明確さで劣る評価処理の
見ることはできない曖昧さに関して問題を提起し、解決する必要がありました。
明示的制御評価機の振舞を理解するために、そのシミュレーションを行い、パフォーマンスを
監視することができます。

私達の評価機にドライバループを導入します。これは\link{Section 4.1.4}の\code{driver\-/loop}手続の
役割を果たします。この評価機は繰り返しプロンプトを表示し、式を読み込み、\code{eval\-/dispatch}へ
飛ぶことで式を評価し、結果を表示します。以下の命令は明示的制御評価機のコントローラシーケンスの
開始を形づくります。\footnote{
ここでは\code{read}と多様な表示命令がプリミティブな機械語命令として
使用可能であると前提します。このことは私達のシミュレーションには便利ですが、実際には完全に
非現実的です。これらは本当はかなり複雑な命令です。実際には、それらは単一の文字を端末との間で
双方向に転送するような低レベルの入出力命令を用いて実装されるでしょう。


\code{get\-/global\-/environment}命令をサポートするためには以下を定義します。

\begin{smallscheme}
(define the-global-environment (setup-environment))
(define (get-global-environment) the-global-environment)
\end{smallscheme}
}

\begin{scheme}
read-eval-print-loop
  (perform (op initialize-stack))
  (perform
   (op prompt-for-input) (const ";;EC-Eval input:"))
  (assign exp (op read))
  (assign env (op get-global-environment))
  (assign continue (label print-result))
  (goto (label eval-dispatch))
print-result
  (perform (op announce-output) (const ";;EC-Eval value:"))
  (perform (op user-print) (reg val))
  (goto (label read-eval-print-loop))
\end{scheme}

\noindent
手続の中で(\code{apply\-/dispatch}で指摘される``未知の手続型エラー''の様な)エラーに遭遇した時、
エラーメッセージを表示し、ドライバループへと戻ります。\footnote{インタプリタに取り扱って
欲しいと願うかもしれない他のエラーも存在します。しかしこれらはあまり単純ではありません。
\link{Exercise 5.30}を参照して下さい。}

\begin{scheme}
unknown-expression-type
  (assign val (const unknown-expression-type-error))
  (goto (label signal-error))
unknown-procedure-type
  (restore continue)    ~\textrm{; clean up stack (from \code{apply\-/dispatch})}~
  (assign val (const unknown-procedure-type-error))
  (goto (label signal-error))
signal-error
  (perform (op user-print) (reg val))
  (goto (label read-eval-print-loop))
\end{scheme}

\noindent
シミュレーションの目的のために、ドライバループを通る度にスタックを初期化します。
(未定義変数の様な)エラーが評価を割り込みした後には空でない可能性があるためです。\footnote{スタックの
初期化をエラーの後にのみ行うことも可能でしょう。しかしドライバループの中で行うことは
評価機のパフォーマンスを監視するために便利です。この先で説明されます。}

\link{Section 5.4.1}から\link{Section 5.4.4}の間のコードの断片を組合せれば、\link{Section 5.2}のレジスタマシン
シミュレータを用いて実行することができる評価機の機械モデルを作ることができます。

\begin{scheme}
(define eceval
  (make-machine
   '(exp env val proc argl continue unev)
   eceval-operations
   '(read-eval-print-loop
     ~\( \dark \langle \)~~\emph{entire machine controller as given above}~~\( \dark \rangle \)~ )))
\end{scheme}

\noindent
評価機によりプリミティブとして使用される命令をシミュレートするためのScheme手続を定義しなければなりません。
これらは\link{Section 4.1}でメタ循環評価機のために使用したものと同じ手続と、\link{Section 5.4}の至る所の脚注にて
定義されたいくつかの追加の物があります。

\begin{scheme}
(define eceval-operations
  (list (list 'self-evaluating? self-evaluating)
        ~\( \dark \langle \)~~\emph{complete list of operations for eceval machine}~~\( \dark \rangle \)~))
\end{scheme}

\noindent
最後に、グローバル環境を初期化し、評価機を実行します。

\begin{scheme}
(define the-global-environment (setup-environment))
(start eceval)
~\textit{;;; EC-Eval input:}~
(define (append x y)
  (if (null? x) y (cons (car x) (append (cdr x) y))))
~\textit{;;; EC-Eval value:}~
~\textit{ok}~
~\textit{;;; EC-Eval input:}~
(append '(a b c) '(d e f))
~\textit{;;; EC-Eval value:}~
~\textit{(a b c d e f)}~
\end{scheme}

\noindent
もちろん、この方法の式の評価はSchemeに直接入力した場合よりもずっと長くかかります。
複数レベルのシミュレーションが関与するためです。式は明示的制御評価器の機械により評価されます。
これはSchemeプログラムによりシミュレートされ、Schemeプログラム自身はSchemeインタプリタにより
評価されています。

\subsubsection*{評価機のパフォーマンスの監視}

シミュレーションは評価器の実装を案内するのに強力なツールです。
シミュレーションはレジスタマシンの設計の多様性を探求することだけでなく、シミュレートされた
評価器のパフォーマンスを観察することも簡単にします。例えば、パフォーマンスにおける
1つの重要な要因はどれだけ効率良く評価機がスタックを使用するかがあります。
スタック利用上の統計を集めるシミュレータの版を用いて評価機のレジスタマシンを設計することと、
評価機の\code{print\-/result}エントリポイントに統計を表示する命令を追加することでにより、
様々な式を評価するのに必要とされるスタック命令の数を観察することができます(\link{Section 5.2.4})。

\begin{scheme}
print-result
  (perform (op print-stack-statistics))   ~\textrm{; 追加された命令}~
  (perform
   (op announce-output) (const ";;; EC-Eval value:"))
  ~\( \dots \)~ ~\textrm{; 以前と同じ}~
\end{scheme}

\noindent
評価機との応答はこれで以下のように見えます。

\begin{scheme}
~\textit{;;; EC-Eval input:}~
(define (factorial n)
  (if (= n 1) 1 (* (factorial (- n 1)) n)))
~\textit{(total-pushes = 3 maximum-depth = 3)}~
~\textit{;;; EC-Eval value:}~
~\textit{ok}~
~\textit{;;; EC-Eval input:}~
(factorial 5)
~\textit{(total-pushes = 144 maximum-depth = 28)}~
~\textit{;;; EC-Eval value:}~
~\textit{120}~
\end{scheme}

\noindent
評価機のドライバループが全ての応答の開始にスタックを再度初期化することに注意して下さい。
それにより表示された統計は直前の式の評価のために使用されたスタック命令のみを参照します。

\begin{quote}
\heading{\phantomsection\label{Exercise 5.26}Exercise 5.26:}
監視付きのスタックを用いて評価機(\link{Section 5.4.2})の末尾再帰の特性を調査せよ。
評価機を開始し、\link{Section 1.2.1}の反復\code{factorial}手続を定義せよ。

\begin{scheme}
(define (factorial n)
  (define (iter product counter)
    (if (> counter n)
        product
        (iter (* counter product) (+ counter 1))))
  (iter 1 1))
\end{scheme}


手続をいくつかの\( n \)の小さな値で実行せよ。これらの値に対する\( n! \)を求めるのに必要な
最大スタック深度とpushの数を記録せよ。

\begin{enumerate}[a]

\item
\( n! \)を評価するために必要な最大深度が\( n \)から独立していることを発見するだろう。
この深さは何か?

\item
あなたのデータから任意の\( n \ge 1 \)に対して\( n! \)を評価するのに使用されるpush命令の
総数を求める\( n \)の方程式を求めよ。使用される命令数は\( n \)の線形関数であり、従って
2つの定数から決定されることに注意せよ。

\end{enumerate}
\end{quote}

\begin{quote}
\heading{\phantomsection\label{Exercise 5.27}Exercise 5.27:}
\link{Exercise 5.26}との比較として、以下の階乗を再帰的に求める手続の振舞を調査せよ。

\begin{scheme}
(define (factorial n)
  (if (= n 1)
      1
      (* (factorial (- n 1)) n)))
\end{scheme}

この手続を監視付きのスタックを用いて実行し、任意の\( n \ge 1 \)に対して\( n! \)を
評価するために使用される、スタックの最大深度とプッシュの総数を\( n \)の関数として求めよ。
(再び、これらの関数は線形になる)。あなたの経験を以下の表に適切な\( n \)の式を埋めることで
まとめよ。

\begin{comment}

\begin{example}
               Maximum depth       Number of pushes

Recursive
factorial

Iterative
factorial
\end{example}

\end{comment}
\begin{displaymath}
\vbox{
\offinterlineskip
\halign{
\strut 	\hfil \quad #\quad \hfil & \vrule 
	\hfil \quad #\quad \hfil & \vrule
	\hfil \quad #\quad \hfil \cr

				& 最大深度 & pushの総数 \cr
	\noalign{\hrule}
	再帰      		&  		&  \cr
	階乗      		&  		&  \cr
	\noalign{\hrule}
	反復      		&  		&  \cr
	階乗      		&  		&  \cr
}
}
\end{displaymath}
最大深度は演算の実行において評価機により使用された記憶域の量の尺度である。
pushの総数は必要な時間に良く関連している。
\end{quote}

\begin{quote}
\heading{\phantomsection\label{Exercise 5.28}Exercise 5.28:}
\link{Section 5.4.2}にて説明されているように\code{eval\-/sequence}を変えることで評価機の定義を変更し、
評価機がもはや末尾再帰ではないようにせよ。\link{Exercise 5.26}と\link{Exercise 5.27}の
実験を再実行し、\code{factorial}手続の両版が今では必要とされる記憶域がそれらの入力に対し
線形に増加することを実演せよ。
\end{quote}

\begin{quote}
\heading{\phantomsection\label{Exercise 5.29}Exercise 5.29:}
木再帰フィボナッチ数の演算におけるスタック命令を監視せよ。

\begin{scheme}
(define (fib n)
  (if (< n 2)
      n
      (+ (fib (- n 1)) (fib (- n 2)))))
\end{scheme}

\begin{enumerate}[a]

\item
\( n \ge 2 \)に対して\( {\rm Fib}(n) \)を求まる場合に必要なスタックの最大深度に
対する\( n \)の方程式を与えよ。ヒント:\link{Section 1.2.2}において私達は
この処理により使用される記憶域は\( n \)に対し線形に増加することを議論した。

\item
\( n \ge 2 \)に対して\( {\rm Fib}(n) \)を求めるのに使用されたpushの総数に対する
方程式を与えよ。(費やされた時間に良く関連する)pushの総数は\( n \)の指数関数的に
増加することを見付けなければならない。ヒント:\( S(n) \)を\( {\rm Fib}(n) \)を
求めるのに使用されたpushの総数とせよ。\( S(n - 1) \), \( S(n - 2) \), それにある
固定の``オーバヘッド''として\( n \)から独立した定数\( k \)を用いて\( S(n) \)を
表現する方程式が存在することを主張することができるはずだ。方程式を与えて、\( k \)が
何であるかを述べよ。次に\( S(n) \)が\( a\cdot{\rm Fib}(n + 1) + b \)として表現できる
ことを示し、\( a \)と\( b \)の値を与えよ。

\end{enumerate}
\end{quote}

\begin{quote}
\heading{\phantomsection\label{Exercise 5.30}Exercise 5.30:}
私達の評価機は現在は2つの種類のエラー---未知の型の式と未知の型の手続---のみを発し
キャッチする。他のエラーは評価機のREPLから抜けてしまう。評価機をレジスタマシン
シミュレータを用いて実行した時に、これらのエラーはその下に横たわるSchemeシステムに
よりキャッチされるこれはユーザプログラムがエラーを発生させた時にコンピュータが
強制終了するのと同様である。\footnote{残念ながら、C言語のような従来のコンパイラベース
言語システムにおいてはこれが普通の状況です。\acronym{UNIX}(tm)ではシステムは
``コア(core)をダンプ(dump)''し、\acronym{DOS}/Windows(tm)ではcatatonic(硬まる、フリーズ)します。
Macintosh(tm)は---もしラッキーな場合は---爆発する爆弾の絵を表示しコンピュータをリブートするよう提案します。}
本物のエラーシステムを働かせることは大きなプロジェクトである。しかし、ここで
何が関与しているのかを理解する努力には大きな価値がある。

\begin{enumerate}[a]

\item
未定義の変数にアクセスを試すような評価プロセスで発生するエラーはlookup命令を
変更し識別可能な状態コードを返すように変更することで捕まえることができるだろう。
状態コードは全てのユーザ変数が取り得ない値でなければならない。評価機はこの状態コードに対し
テストを行い、\code{signal\-/error}に飛ぶために必要なことを行う。評価機の中でそのような
変更が必要な全ての箇所を見つけ修正せよ。これはとても大変な作業だ。

\item
ゼロで割ることやシンボルから\code{car}を抽出するような試行により、プリミティブな
手続の適用上で発せられるエラーの取り扱いの問題はずっと酷い物だ。
専門的に記述された高品質なシステムにおいては、プリミティブの適用全てはプリミティブの
一部として安全性が確認されている。例えば全ての\code{car}の呼出は最初に引数がペアであるかを
確認する。もし引数がペアでなければ、適用は区別可能な状態コードを評価機に返す。すると
評価機は失敗を報告する。私達はこれを私達のレジスタマシンシミュレータに全てのプリミティブ
手続の適用性チェックを行い適切な識別可能な状態コードを失敗時に返すようにすることで
手筈を整えることができるだろう。すると評価機の\code{primitive\-/apply}コードが状態コードを
チェックし必要なら\code{signal\-/error}へ飛ぶことができる。この構造を構築し、働くようにせよ。
これは巨大なプロジェクトである。

\end{enumerate}
\end{quote}

\section{コンパイル}
\label{Section 5.5}

\link{Section 5.4}の明示的制御評価機はコントローラがSchemeプログラムを解釈するレジスタマシンです。
この節ではSchemeプログラムをコントローラがSchemeインタプリタではないレジスタマシン上にて
どのように実行するのかについて学びます。

明示的制御評価機械は普遍的です。Schemeで記述できるどんな演算プロセスも実行できます。
評価機のコントローラは望んだ演算を実行するためのデータパスの使用を調整します。
従って、評価機のデータパスは普遍的です。適切なコントローラを与えられた場合に、
私達が望む任意の演算を実行するのに十分です。\footnote{これは理論的な発言です。
この評価機のデータパスが一般的なコンピュータのために特に便利な、または効率的な
データパスの集合であると主張している訳ではありません。例えば、これらは高いパフォーマンスの
浮動小数点演算や、激しくビットベクタを操作する演算の実装にはあまり向いていません。}

商業上の汎用なコンピュータはレジスタと効率的で便利なデータパスの普遍的な集合を
構成する命令の周りに構築されるレジスタマシンです。汎用目的の機械は私達がここまで
使用しているようなレジスタマシン言語のためのインタプリタです。この言語は
機械の\newterm{native language}(\jnewterm{ネイティブ言語})、または単純に\newterm{machine language}(\jnewterm{機械語})と
呼ばれます。機械語で記述されたプログラムはその機械のデータパスを用いた命令列です。
例えば、明示的制御評価機の命令列は専門のインタプリタマシンのためのコントローラではなく、
汎用目的のコンピュータのための機械語プログラムだと考えることができます。

2つの共通な戦略が高水準の言語とレジスタマシンの言語の間のギャップを橋渡しします。
明示的制御評価機は逐次翻訳(interpretation)上の戦略を説明します。機械のネイティブ言語で
書かれたインタプリタは、評価を実行する機械のネイティブ言語とは異なっても良い
ある言語(\newterm{source language}(\jnewterm{ソース言語}))で書かれたプログラムを実行するように
機械を構成します。ソース言語のプリミティブ手続は与えられた機械のネイティブ言語により
記述されたサブルーチンのライブラリとして実装されます。
(\newterm{source program}(\jnewterm{ソースプログラム})と呼ばれる)逐次翻訳するプログラムは
データ構造として表現されます。インタプリタはこのデータ構造を横断し、ソースプログラムを
分析します。それを行うにつれ、ソースプログラムの意図された振舞を適切なプリミティブの
サブルーチンをライブラリから呼ぶことによりシミュレートします。

この節では、\newterm{compilation}(\jnewterm{コンパイル})という代替的な戦略を探求します。
与えられたソース言語と機械に対するコンパイラはソースプログラムを機械のネイティブ言語で書かれた
(\newterm{object program}(\jnewterm{オブジェクトプログラム})と呼ばれる)等価なプログラムに
翻訳します。この節で実装するコンパイラはSchemeで書かれたプログラムを明示的制御評価機の
データパスを用いて実行される命令列へと翻訳します。\footnote{実際には、コンパイルされた
コードを実行する機械はインタプリタマシンよりもより単純に成り得ます。\code{exp}と
\code{unev}のレジスタを使用しないためです。これらを使用するインタプリタは未評価の式の
部分を保持します。しかしコンパイラを用いる場合には、これらの式はレジスタマシンが実行する
コンパイルされたコードの中に組込まれます。同じ理由により、式の構文を扱う機械語命令を
必要としません。しかしコンパイルされたコードは明示的制御評価機械では存在しなかった
いくつかの追加の(コンパイルされた手続オブジェクトを表現するための)機械語命令を使用します。}

逐次翻訳と比べた時、コンパイルはプログラム実行の効率性において大きな向上を与えられます。
このことは下記にてコンパイラの概観において説明して行きます。一方で、インタプリタは
より強力な環境を対話式のプログラム開発とデバッグのために提供します。実行する
ソースプログラムが実行時にも試験し、変更するために使用可能なためです。
それに加えて、プリミティブのライブラリ全体が存在し、新しいプログラムがデバッグの間に
構築し、追加することができることも挙げられます。

コンパイルと逐次翻訳の相補的な利点の視点において、最新のプログラム開発環境は
入り交じった戦略を追求しています。Lispインタプリタは一般的に逐次翻訳された手続と
コンパイルされた手続がお互いを呼びだせるように構築されています。
これはプログラマがデバッグすることを想定されているこれらのプログラムの部品を
コンパイルすることを可能にします。従ってコンパイルの効率上の利点を得ながら、
プログラムのそれらの部品に対して対話式開発とデバッグの流動的な、実行の解釈的な
モードを維持することもできます。\link{Section 5.5.7}において、コンパイラを実装した後には
インタプリタとどのように接続して統合的なインタプリタ・コンパイラ開発システムを
生成するかを示します。

\subsubsection*{コンパイラの概要}

私達のコンパイラは私達のインタプリタに両者のその構造と実行する機能において
とても良く似ています。従って、コンパイラにより式の解析のために使用される仕組みは
インタプリタにて使用されたものと同様になります。さらに、コンパイルされたコードと
逐次翻訳されたコードの接続を簡単にするために、インタプリタと同じ
レジスタ使用法の仕様に従うコードを生成する様にコンパイラを設計します。
環境は\code{env}レジスタに保持され、引数リストは\code{argl}に蓄積され、適用される
手続は\code{proc}に入り、手続はそれらの回答を\code{val}に入れて戻り、手続が戻らなければ
いけない位置は\code{continue}に維持されます。一般的に、コンパイラはソースプログラムを
インタプリタが同じソースプログラムを評価する場合に行うのと本質的に同じレジスタ命令
を実行するオブジェクトプログラムに翻訳します。

この説明ではとても基本的なコンパイラを実装するための戦略を提案します。式をインタプリタと
同じ方法で横断します。インタプリタが式の評価で実行するだろうレジスタ命令に遭遇したら、
その命令を実行はしませんがその代わりに列に蓄積します。結果としての命令列は
オブジェクトコードになります。逐次翻訳に対するコンパイルの効率上の利点を注意して下さい。
インタプリタが式、例えば\code{(f 84 96)}を評価する度に、式の分類(手続の適用であるかを見出す)と
オペランドリストの終端の検査(2つのオペランドが残っているかを見出す)を行います。
コンパイラを用いる場合、式は命令列がコンパイル時に生成された時に一度しか解析されません。
コンパイラにより生成されたオブジェクトコードはオペレータと2つのオペランドを評価する命令しか
含んでおらず、引数リストを組み立て、(\code{proc}内の)手続を(\code{argl}内の)その引数に適用します。

これは\link{Section 4.1.7}の解析評価機で実装したものと同じ種類の最適化です。しかし、コンパイルされた
コード中で効率を良くするためのさらなる機会が存在します。インタプリタが実行するにしたがって、
インタプリタは言語の任意の式に必ず当てはまる過程を追います。対照的に、与えられたコンパイル済みコードの断片は
ある特定の式を実行することを意味します。これは例えばスタックを用いてレジスタを保存する場合等に
大きな違いを生みます。インタプリタが式を評価する時には、任意の偶発性に対して準備をしなければ
なりません。部分式を評価する前に、インタプリタは後で必要となる全てのレジスタを保存します。
部分式が無原則な評価を要求するかもしれないためです。一方、コンパイラは処理対象の特定の式の構造を利用して
不必要なスタック命令を回避するコードを生成することができます。

その一例として、組み合わせ\code{(f 84 96)}について考えてみます。インタプリタが組み合わせの
オペレータを評価する前に、値が後で必要になるオペランドと環境を持つレジスタを保存することで
この評価のための準備を行います。次にインタプリタはオペレータを評価してその結果を\code{val}に
取得し、保存したレジスタを戻し、最後に結果を\code{val}から\code{proc}に移します。しかし、
私達が評価しているこの式では、オペレータがシンボルの\code{f}であり、その評価は機械語の
\code{lookup\-/variable\-/value}にて達成され、これはどのレジスタの値も変化させません。
この節で実装するコンパイラはこの事実を活用し、オペレータをこの命令を使用して評価するコードを
生成します。

\begin{scheme}
(assign proc (op lookup-variable-value)
             (const f)
             (reg env))
\end{scheme}

\noindent
このコードは不必要な保存と復元を回避するだけでなく、lookupの値を直接\code{proc}に割り当てます。
一方でインタプリタは結果を\code{val}の中に取得し、その後に\code{proc}へと移します。

コンパイラはまた環境へのアクセスを最適化することができます。コードを解析した後に、
コンパイラは多くの場合において、どのフレームの中に特定の変数が位置するかを知り、
\code{lookup\-/variable\-/value}による検索を実行するのではなしに、直接アクセスすることができます。
そのような変数のアクセスをどのように実装するかについての議論は\link{Section 5.5.6}にて行います。



\subsection{コンパイラの構造}
\label{Section 5.5.1}

\link{Section 4.1.7}において、私達は元のメタ循環インタプリタを変更して分析を実行から分離しました。
各式を分析して環境を引数として取り必要とされる命令を実行する実行手続を生成しました。
私達のコンパイラでは、本質的には同じ分析を行います。しかし、実行手続を生成する代わりに、
私達のレジスタマシンにより実行される命令列を生成します。

手続\code{compile}はコンパイラ内でのトップレベルの割り振りです。これは\link{Section 4.1.1}の
\code{eval}手続、\link{Section 4.1.7}の\code{analyze}手続、そして\link{Section 5.4.1}の明示的制御評価機の
エントリポイント\code{eval\-/dispatch}に対応します。コンパイラはインタプリタと同様に、
\link{Section 4.1.2}における式の構文手続を用います。\footnote{しかし、私達のコンパイラが
Schemeプログラムであり、式を操作するためにそれが用いる構文手続がメタ循環評価機により
仕様される実際のScheme手続であることに注意して下さい。一方で、明示的制御評価機では
等価な構文命令がレジスタマシンに対する命令として使用可能であると前提しました。
(もちろん、Schemeでレジスタマシンをシミュレートした時には、実際のSchemeの手続を
使用しました。)}\code{compile}はコンパイルされる式の構文の型の事例分析を実行します。
各式の型に対し、特別な\newterm{code generator}(\jnewterm{コード生成器})を割り振ります。

\begin{scheme}
(define (compile exp target linkage)
  (cond ((self-evaluating? exp)
         (compile-self-evaluating exp target linkage))
        ((quoted? exp) (compile-quoted exp target linkage))
        ((variable? exp)
         (compile-variable exp target linkage))
        ((assignment? exp)
         (compile-assignment exp target linkage))
        ((definition? exp)
         (compile-definition exp target linkage))
        ((if? exp) (compile-if exp target linkage))
        ((lambda? exp) (compile-lambda exp target linkage))
        ((begin? exp)
         (compile-sequence
          (begin-actions exp) target linkage))
        ((cond? exp) 
         (compile (cond->if exp) target linkage))
        ((application? exp)
         (compile-application exp target linkage))
        (else
         (error "Unknown expression type: COMPILE" exp))))
\end{scheme}

\subsubsection*{ターゲットとリンク記述子}

\code{compile}とそれが呼ぶコード生成器はコンパイル対象の式に加えて2つの引数を取ります。
コンパイルされたコードがその中で式の値を返すレジスタを指定する\newterm{target}(\jnewterm{ターゲット})と
実行が完了した時に、式のコンパイルの結果としてのコードがどのように続けるべきかを
説明する\newterm{linkage descriptor}(\jnewterm{リンク記述子})です。リンク記述子はコードが
以下の3つの内1つを行うよう要求することができます。

\begin{itemize}

\item
列の次の命令を続ける(これはリンク記述子\code{next}により指定されます)

\item
コンパイルしている手続から戻る(これはリンク記述子\code{return}により指定されます)

\item
名前付きエントリポイントへ飛ぶ(これはリンク記述子として指定したラベルを用いて指定します)

\end{itemize}

\noindent
例えば、式\code{5}(これは自己評価型)をターゲットをレジスタ\code{val}、リンク記述子を\code{next}で
コンパイルする時、以下の命令を生成しなければなりません。

\begin{scheme}
(assign val (const 5))
\end{scheme}

\noindent
同じ式をリンク記述子\code{return}でコンパイルする時には以下の命令を生成しなければなりません。

\begin{scheme}
(assign val (const 5))
(goto (reg continue))
\end{scheme}

\noindent
最初の場合には、実行は列内の次の命令と共に続行します。2つ目の場合には、手続呼出から
戻ります。両者の場合において、式の値はターゲットレジスタ\code{val}に配置されます。

\subsubsection*{命令列とスタックの使用法}

各コード生成器は式のために生成したオブジェクトコードを含む\newterm{instruction sequence}(\jnewterm{命令列})を
返します。複合式に対するコード生成は部分式のためのより単純なコード生成器からの出力を組み合わせる
ことにより達成されます。これは複合式の評価が部分式を評価することにより達成されるのと同じです。

命令列を組み合わせる最も単純な手法は\code{append\-/instruction\-/sequences}という手続です。
これは引数として順に実行されるべき任意の数の命令列を取り、それらを接続し、組み合わされた列を返します。
つまり、もし\( \langle \)\( seq_1 \)\( \rangle \)と\( \langle \)\( seq_2 \)\( \rangle \)が
命令列であるならば、以下の評価は、

\begin{scheme}
(append-instruction-sequences ~\( \dark \langle \)~~\( \dark seq_1 \)~~\( \dark \rangle \)~ ~\( \dark \langle \)~~\( \dark seq_2 \)~~\( \dark \rangle \)~)
\end{scheme}

\noindent
次の列を生成します。

\begin{scheme}
~\( \dark \langle \)~~\( \dark seq_1 \)~~\( \dark \rangle \)~
~\( \dark \langle \)~~\( \dark seq_2 \)~~\( \dark \rangle \)~
\end{scheme}

\noindent
レジスタが保存される必要がある度に、コンパイラのコード生成器は\code{preserving}を使用します。
これは命令列を組み立てるための、より芸が細かい手法です。\code{preserving}は3つの引数を
取ります。レジスタの集合と2つの命令列です。これは列をレジスタ集合内の各レジスタの中身が、
2つ目の列の実行に必要ならば、最初の列の実行の間は維持(preserve)されるような方法で接続します。
言い換えれば、もし最初の命令列がレジスタを変更し、2つ目の列が実際にそのレジスタの元の中身を
必要とするならば、\code{preserving}は列を接続する前に最初の列をそのレジスタの\code{save}と\code{restore}で
包みます。そうでなければ、\code{preserving}は単純に接続した命令列を返します。
従って、例えば
\code{(preserving (list}\( \;\;\langle{reg_1}\rangle\;\;\langle{reg_2}\rangle \)\code{)}\( \;\;\langle{seg_1}\rangle\;\;\langle{seg_2}\rangle \)\code{)}
は、
\( \langle \)\( seq_1 \)\( \rangle \)と\( \langle \)\( seq_2 \)\( \rangle \)がどのように
\( \langle \)\( reg_1 \)\( \rangle \)と\( \langle \)\( reg_2 \)\( \rangle \)を使用するかに依存して
以下の4つの命令列の内1つを生成します。

\begin{comment}

\begin{smallexample}
<seq_1> | (save <reg_1>)    | (save <reg_2>)    | (save <reg_2>)
<seq_2> | <seq_1>           | <seq_1>           | (save <reg_1>)
        | (restore <reg_1>) | (restore <reg_2>) | <seq_1>
        | <seq_2>           | <seq_2>           | (restore <reg_1>)
        |                   |                   | (restore <reg_2>)
        |                   |                   | <seq_2>
\end{smallexample}

\end{comment}
\begin{displaymath}
\vbox{
\offinterlineskip
\halign{
\strut 	\kern0.8em # \kern0.4em \hfil & \vrule 
	\kern0.8em # \kern0.4em \hfil & \vrule
	\kern0.8em # \kern0.4em \hfil & \vrule
	\kern0.8em # \kern0.4em \hfil \cr

$\langle{seq_1}\rangle$ 			& \hbox{\tt (save} $\langle{reg_1}\rangle${\tt)} & 
\hbox{\tt (save} $\langle{reg_2}\rangle${\tt)} 	& \hbox{\tt (save} $\langle{reg_2}\rangle${\tt)} \cr

$\langle{seq_2}\rangle$ 			& $\langle{seq_1}\rangle$	& 
$\langle{seq_1}\rangle$ 			& \hbox{\tt (save} $\langle{reg_1}\rangle${\tt)} \cr

						& \hbox{\tt (restore} $\langle{reg_1}\rangle${\tt)} & 
\hbox{\tt (restore} $\langle{reg_2}\rangle${\tt)} 	& $\langle{seq_1}\rangle$ \cr

						& $\langle{seq_2}\rangle$ & 
$\langle{seq_2}\rangle$				& \hbox{\tt (restore} $\langle{reg_1}\rangle${\tt)} \cr

						&  &
			      			& \hbox{\tt (restore} $\langle{reg_2}\rangle${\tt)} \cr

						&  &
			      			& $\langle{seq_2}\rangle$ \cr
}
}
\end{displaymath}

\noindent
\code{preserving}を用いて命令列を組み立てることにより、コンパイラは不必要なスタック命令を
回避することが可能になります。これはまた\code{save}と\code{restore}の命令を\code{preserving}手続の
中で生成するか、しないかの詳細を分離し、個別のコード生成器それぞれを書く場合に
浮かび上がる考慮点から隔離します。実際に\code{save}と\code{restore}の命令は明示的には
コード生成器により生成されることはありません。

原理上は、命令列を単純に命令のリストとして表現できるでしょう。
\code{append\-/instruction\-/sequences}はそうすると命令列の組み立てを通常のリストの
\code{append}にて行うことができます。するとしかし、\code{preserving}は複雑な命令に
なります。それが各命令列に対し、レジスタをどのように使用するかの分析を行わな
ければならないためです。また複雑であると同様に\code{preserving}が非効率にもなります。
各命令列の引数をも分析しなければならなくなるためです。例えこれらの列自身が
\code{preserving}の呼出により構築されていて、それらの部品が既に分析されていてもです。
そのような分析の繰り返しを防ぐために、各命令列とそのレジスタ使用に関する情報とを
結び付けます。基本的な命令列を構築する時に、私達はこの情報を明示的に与えます。
そして命令列を接続する手続は列の組み合わせのために、レジスタ使用の情報を
構成部品である列に結び付けられた情報から引き出します。


命令列は3つの情報を持ちます。

\begin{itemize}

\item
命令列内の命令が実行される前に初期化しなければならないレジスタ集合(これらのレジスタは
命令列により\newterm{needed}(必要とされる)と述べられる)

\item
列内の命令によりその値が変更されるレジスタ集合

\item
列内の実際の命令(\newterm{statements}(\jnewterm{命令文})とも呼ばれる)

\end{itemize}

\noindent
命令列をその3つの部品として表現します。命令列のコンストラクタは従って以下のようになります。

\begin{scheme}
(define (make-instruction-sequence
         needs modifies statements)
  (list needs modifies statements))
\end{scheme}

\noindent
喩えば、現在の環境内で変数\code{x}の値を探し、その結果を\code{val}に割り当てて戻る2つの命令の列は
レジスタ\code{env}と\code{continue}が初期化される必要があり、そしてレジスタ\code{val}を変更します。
この列は従って以下のように構築されます。

\begin{scheme}
(make-instruction-sequence 
 '(env continue)
 '(val)
 '((assign val
           (op lookup-variable-value) (const x) (reg env))
   (goto (reg continue))))
\end{scheme}

\noindent
時々、命令文が無い命令列を構築する必要があります。

\begin{scheme}
(define (empty-instruction-sequence)
  (make-instruction-sequence '() '() '()))
\end{scheme}

\noindent
命令列を組み立てる手続は\link{Section 5.5.4}に示します。

\begin{quote}
\heading{\phantomsection\label{Exercise 5.31}Exercise 5.31:}
手続適用を評価する場合において、明示的制御評価機は常にオペレータの評価の周りで
\code{env}レジスタの保存と復元を行う。また各オペランドの評価の評価の周りでも(最後の1つを除いて)
\code{env}の保存と復元を行う。そえいオペランド列の評価の周りでは\code{proc}の保存と
復元を行う。以下の各組み合わせに対し、これらの\code{save}と\code{restore}命令のどれが
余分であり、従ってコンパイラの\code{preserving}の仕組みにより削減できるかを述べよ。

\begin{scheme}
(f 'x 'y)
((f) 'x 'y)
(f (g 'x) y)
(f (g 'x) 'y)
\end{scheme}
\end{quote}

\begin{quote}
\heading{\phantomsection\label{Exercise 5.32}Exercise 5.32:}
\code{preserving}の仕組みを用いた場合、コンパイラは組み合わせのオペレータの評価の周りで
オペレータがシンボルの場合に\code{env}の保存と復元を削減することができる。
またそのような最適化を評価機の中に構築することもできるだろう。実際に、\link{Section 5.4}の
明示的制御評価機は既に似たような最適化をオペランドの無い組み合わせを特別な場合として
扱うことで実行している。

\begin{enumerate}[a]

\item
明示的制御評価機を拡張しオペレータがシンボルである組み合わせを別のクラスの
式として認識するようにせよ。そしてこの事実をそのような式の評価において活用
するようにせよ。

\item
Alyssa P. Hackerは評価機を拡張し、組込むことができる全てのコンパイラの最適化を
より多くの特別な場合として認識することで、コンパイルの利点全体を
無くすことができると提案した。あなたはこの考えをどう思うか?

\end{enumerate}
\end{quote}

\subsection{式のコンパイル}
\label{Section 5.5.2}


この節と次の節では、\code{compile}手続が割り振るコード生成器を実装します。

\subsubsection*{リンクコードのコンパイル}

一般的に、各コード生成器の出力は手続\code{compile\-/linkage}により生成された、
要求されたリンク記述子を実装した命令で終わります。もしリンク記述子が\code{return}なら、
命令\code{(goto (reg continue))}を生成せねばなりません。これは\code{continue}レジスタを
必要とし、他のレジスタを変更はしません。もしリンク記述子が\code{next}なら、何の追加の
命令も必要ありません。さもなければ、リンク記述子はラベルであり、そのラベルへの
\code{goto}を生成します。この命令はレジスタを必要とせず、変更もしません。\footnote{
この手続は
\newterm{backquote}(\jnewterm{バッククォート}、または\newterm{quasiquote}(\jnewterm{擬似クォート}))と呼ばれるLispの機能を使用します。
これはリストを構築するのに便利です。リストの前にバッククォート記号を置くことは
クォートすることにとても似ていますが、リスト内のカンマで合図された物全てを評価することが
異なります。


例えば、もし\code{linkage}の値がシンボル\code{branch25}の場合、以下の式は

\begin{smallscheme}
`((goto (label ,linkage)))
\end{smallscheme}


次のリストとして評価されます。

\begin{smallscheme}
((goto (label branch25)))
\end{smallscheme}


同様に、もし\code{x}の値がリスト\code{(a b c)}ならば、以下の式は

\begin{smallscheme}
`(1 2 ,(car x))
\end{smallscheme}


次のリストとして評価されます。

\begin{smallscheme}
(1 2 a)
\end{smallscheme}
}

% For example, if the value of \code{linkage} is the symbol\\ \code{branch25},
% then the expression\\ \code{`((goto (label ,linkage)))}\\ evaluates to the
% list\\ \code{((goto (label branch25)))}.\\ Similarly, if the value of \code{x}
% is the list \code{(a b c)}, then\\ \code{`(1 2 ,(car x))}\\ evaluates to the
% list\\ \code{(1 2 a)}.}

\begin{scheme}
(define (compile-linkage linkage)
  (cond ((eq? linkage 'return)
         (make-instruction-sequence '(continue) '()
          '((goto (reg continue)))))
        ((eq? linkage 'next)
         (empty-instruction-sequence))
        (else
         (make-instruction-sequence '() '()
          `((goto (label ,linkage)))))))
\end{scheme}

\noindent
リンクのコードが命令列に対し\code{preserving}により\code{continue}レジスタを維持しながら
追加されます。リンク記述子\code{return}が\code{continue}レジスタを必要とするためです。
もし与えられた命令列が\code{continue}を変更し、リンクのコードがそれを必要とする場合、
\code{continue}は保存と復元が行われます。

\begin{scheme}
(define (end-with-linkage linkage instruction-sequence)
  (preserving '(continue)
   instruction-sequence
   (compile-linkage linkage)))
\end{scheme}

\subsubsection*{単純な式のコンパイル}

自己評価型式、クォート、変数に対するコード生成器は
必要な値をターゲットのレジスタに割り当てリンク記述子により指示されたように進める
命令列を構築します。

\begin{scheme}
(define (compile-self-evaluating exp target linkage)
  (end-with-linkage linkage
   (make-instruction-sequence '() (list target)
    `((assign ,target (const ,exp))))))
(define (compile-quoted exp target linkage)
  (end-with-linkage linkage
   (make-instruction-sequence '() (list target)
    `((assign ,target (const ,(text-of-quotation exp)))))))
(define (compile-variable exp target linkage)
  (end-with-linkage linkage
   (make-instruction-sequence '(env) (list target)
    `((assign ,target
              (op lookup-variable-value)
              (const ,exp)
              (reg env))))))
\end{scheme}

\noindent
これら全ての代入命令はターゲットレジスタを変更します。また変数の検索を行う物は
\code{env}レジスタを必要とします。

代入と定義はインタプリタの物と同様に扱われます。再帰的に変数に割り当てられる値を求める
コードを生成し、それに対して実際に変数の設定、または定義を行う物と式全体の値(シンボル\code{ok})を割り当てる
物の2つの命令列を接続します。再帰的なコンパイルはターゲット\code{val}とリンク記述子\code{next}を持つので
コードはその結果を\code{val}に入れ、その後に接続されたコードを用いて続けられます。接続は
\code{env}を維持(preserving)している間に行われます。環境が変数の設定、または定義のため
必要なためです。また変数の値のためのコードは複雑な式のコンパイルと成り得るため任意の方法で
レジスタを変更する可能性があります。

\begin{scheme}
(define (compile-assignment exp target linkage)
  (let ((var (assignment-variable exp))
        (get-value-code
         (compile (assignment-value exp) 'val 'next)))
    (end-with-linkage linkage
     (preserving '(env)
      get-value-code
      (make-instruction-sequence '(env val) (list target)
       `((perform (op set-variable-value!)
                  (const ,var)
                  (reg val)
                  (reg env))
         (assign ,target (const ok))))))))

(define (compile-definition exp target linkage)
  (let ((var (definition-variable exp))
        (get-value-code
         (compile (definition-value exp) 'val 'next)))
    (end-with-linkage linkage
     (preserving '(env)
      get-value-code
      (make-instruction-sequence '(env val) (list target)
       `((perform (op define-variable!)
                  (const ,var)
                  (reg val)
                  (reg env))
         (assign ,target (const ok))))))))
\end{scheme}

\noindent
接続された2つの命令列は\code{env}と\code{val}を必要とし、ターゲットを変更します。
例え\code{env}をこの列のために維持したとしても、\code{val}は維持しません。
\code{get\-/value\-/code}が明示的にその結果をこの命令列のために\code{val}に配置するように
設計されているためです。(実際には、もし\code{val}を維持した場合、バグを持つことになります。
これにより直前の\code{val}の中身が\code{get\-/value\-/code}の実行直後に復元されるためです。)

\subsubsection*{条件式のコンパイル}

与えられたターゲットとリンク記述子と共にコンパイルされる\code{if}式のためのコードは
以下の形式を持ちます。

\begin{scheme}
~\( \dark \langle \)~~\emph{述語のコンパイル, ターゲット \code{val}, リンク記述子 \code{next}}~~\( \dark \rangle \)~
 (test (op false?) (reg val))
 (branch (label false-branch))
true-branch
 ~\( \dark \langle \)~~\emph{結果部 与えられたターゲット
  与えられたリンク記述子 又は \code{after\-/if}}~~\( \dark \rangle \)~
false-branch
 ~\( \dark \langle \)~~\emph{代替部 与えられたターゲット リンク記述子}~~\( \dark \rangle \)~
after-if
\end{scheme}

\noindent
このコードを生成するために、述語、結果部、代替部をコンパイルし、結果のコードを
述語の結果をテストするコードと新しく生成された真と偽の分岐をマークするラベルと
条件文の最後と共に組み立てます。\footnote{
私達は単にラベル\code{true\-/branch},
\code{false\-/branch}, \code{after\-/if}を上で示されたように使うことはできません。なぜなら
プログラム中に\code{if}文は複数存在する可能性があるためです。\code{make\-/label}は
シンボルを引数として、与えられたシンボルで始まる取り新しいシンボルを返します。
例えば、\code{(make\-/label 'a)}に対する連続した呼出は\code{a1}, \code{a2}, \( \dots \)を返します。
\code{make\-/label}はクエリ言語における一意の変数名の生成と同様に、以下の様に実装する
ことができます。

\begin{smallscheme}
(define label-counter 0)
(define (new-label-number)
  (set! label-counter (+ 1 label-counter))
  label-counter)
(define (make-label name)
  (string->symbol
    (string-append (symbol->string name)
                   (number->string (new-label-number)))))
\end{smallscheme}
}
このコードの準備では、テストが偽である場合、真の分岐へと飛ぶ必要があります。
唯一、微妙に複雑なのは真の分岐がリンク記述子をどのように扱うかについてです。
もし条件文のリンク記述子が\code{return}、またはラベルの場合、真と偽の分岐は両方共
これと同じリンク記述子を用います。もしリンク記述子が\code{next}なら、真の分岐は
偽の分岐を飛び越し条件文の最後へと飛ぶコードで終わります。

\begin{scheme}
(define (compile-if exp target linkage)
  (let ((t-branch (make-label 'true-branch))
        (f-branch (make-label 'false-branch))
        (after-if (make-label 'after-if)))
    (let ((consequent-linkage
           (if (eq? linkage 'next) after-if linkage)))
      (let ((p-code (compile (if-predicate exp) 'val 'next))
            (c-code
             (compile
              (if-consequent exp) target 
                                  consequent-linkage))
            (a-code
             (compile (if-alternative exp) target linkage)))
        (preserving '(env continue)
         p-code
         (append-instruction-sequences
          (make-instruction-sequence '(val) '()
           `((test (op false?) (reg val))
             (branch (label ,f-branch))))
          (parallel-instruction-sequences
           (append-instruction-sequences t-branch c-code)
           (append-instruction-sequences f-branch a-code))
          after-if))))))
\end{scheme}

\noindent
\code{env}は述語コードの間維持されます。真と偽の分岐で必要になるかもしれないためです。
そして\code{continue}もそれら分岐内でリンクのためのコードにて使用されるかもしれないため維持されます。
真と偽の分岐のためのコード(順には実行されません)は\link{Section 5.5.4}で説明される
専用の結合器、\code{parallel\-/instruction\-/sequences}を使用して接続されます。

\code{cond}は派生式であることに注意してください。そのためコンパイラが取り扱いのために
必要なこと全ては(\link{Section 4.1.2}の)\code{cond\-/>if}変換器を適用して、結果の\code{if}式を
コンパイルするだけです。

\subsubsection*{列のコンパイル}

列のコンパイル(手続のボディ、または明示的な\code{begin}式)はそれらの評価を並列化します。
列の各式は次の条件でコンパイルされます。最後の式は列に対して指示されたリンク記述子を
用いて。他の式はリンク記述子\code{next}を用いて(列の残りを実行するために)。
個別の式の命令列は接続され単一の命令列を形成します。(列の残りのために必要な)\code{env}と
(列の終わりのリンクコードで必要な可能性のある)\code{continue}は維持されます。

\begin{scheme}
(define (compile-sequence seq target linkage)
  (if (last-exp? seq)
      (compile (first-exp seq) target linkage)
      (preserving '(env continue)
       (compile (first-exp seq) target 'next)
       (compile-sequence (rest-exps seq) target linkage))))
\end{scheme}

\subsubsection*{\code{lambda}式のコンパイル}

\code{lambda}式は手続を構築します。\code{lambda}式のためのオブジェクトコードは
以下の形式に従わねばなりません。

\begin{scheme}
~\( \dark \langle \)~~\emph{手続オブジェクトの構築}~ 
 ~\emph{ターゲットレジスタにそれを割り当てる}~~\( \dark \rangle \)~
~\( \dark \langle \)~~\var{\dark リンク}~~\( \dark \rangle \)~
\end{scheme}

\noindent
\code{lambda}式をコンパイルする時、手続のボディのためのコードも生成します。
例えボディが手続構築時に実行されなくても、オブジェクトコードの中の
\code{lambda}式のコードの直後に挿入しておくことは便利です。
もし\code{lambda}式に対するリンク記述子がラベルか\code{return}ならば、このことに問題はありません。
しかし、もしリンク記述子が\code{next}ならば、手続のボディの後ろに挿入されたラベルへ飛ぶリンク記述子を
使用することによりボディに対するコードを回避する必要があります。従ってオブジェクト
コードは以下の形式になります。

\begin{scheme}
~\( \dark \langle \)~~\emph{手続オブジェクトの構築}~ 
 ~\emph{それをターゲットレジスタに割当}~~\( \dark \rangle \)~
 ~\( \dark \langle \)~~\emph{与えられたリンク記述子に対するコード}~~\( \dark \rangle \)~ ~\emph{または}~ ~\code{(goto (label after\-/lambda))}~
 ~\( \dark \langle \)~~\emph{手続ボディのコンパイル後コード}~~\( \dark \rangle \)~
after-lambda
\end{scheme}

\noindent
\code{compile\-/lambda}は手続のボディのコードが続く手続オブジェクトを構築するためのコードを
生成します。手続オブジェクトは実行時に現在の環境(定義時点での環境)をコンパイルされた
手続ボディのエントリポイント(新しく生成されたラベル)と共に
組み立てることで構築されます。\footnote{\label{Footnote 38}
\link{Section 4.1.3}で説明した
複合手続のための構造と同様に、コンパイル後の手続を表現するためのデータ構造を
実装するための機械語命令を必要とします。

\begin{smallscheme}
(define (make-compiled-procedure entry env)
  (list 'compiled-procedure entry env))
(define (compiled-procedure? proc)
  (tagged-list? proc 'compiled-procedure))
(define (compiled-procedure-entry c-proc) (cadr c-proc))
(define (compiled-procedure-env c-proc) (caddr c-proc))
\end{smallscheme}
}

\begin{scheme}
(define (compile-lambda exp target linkage)
  (let ((proc-entry (make-label 'entry))
        (after-lambda (make-label 'after-lambda)))
    (let ((lambda-linkage
           (if (eq? linkage 'next) after-lambda linkage)))
      (append-instruction-sequences
       (tack-on-instruction-sequence
        (end-with-linkage lambda-linkage
         (make-instruction-sequence '(env) (list target)
          `((assign ,target
                    (op make-compiled-procedure)
                    (label ,proc-entry)
                    (reg env)))))
        (compile-lambda-body exp proc-entry))
       after-lambda))))
\end{scheme}

\noindent
\code{compile\-/lambda}は\code{append\-/instruction\-/sequences}(\link{Section 5.5.4})ではなく、
特別な結合器\code{tack\-/on\-/instruction\-/sequence}を手続のボディと\code{lambda}式のコードを
接続するのに利用します。ボディは組み立てられた列が入力された時に実行される命令列の一部では
ないためです。そうではなく、それはただ、そこに置くことが便利だから、その列の中にあります。

\code{compile\-/lambda\-/body}は手続のボディのためのコードを構築します。
このコードはエントリポイントに対するラベルで開始します。次に来るのは
実行時の環境を手続のボディの評価を評価するために正しい環境へとスイッチする
命令列です。即ち、手続の定義環境であり、これは手続が呼ばれる時に利用される引数に
対する形式パラメタの束縛を含むように拡張されています。
これの後には、式の列のコードが来ます。これが手続のボディを作り上げます。
この列はリンク記述子\code{return}とターゲット\code{val}と共にコンパイルさるため、
手続の結果は\code{val}に入れらた状態で手続から戻ることで終わります。

\begin{scheme}
(define (compile-lambda-body exp proc-entry)
  (let ((formals (lambda-parameters exp)))
    (append-instruction-sequences
     (make-instruction-sequence '(env proc argl) '(env)
      `(,proc-entry
        (assign env
                (op compiled-procedure-env)
                (reg proc))
        (assign env
                (op extend-environment)
                (const ,formals)
                (reg argl)
                (reg env))))
     (compile-sequence (lambda-body exp) 'val 'return))))
\end{scheme}

\subsection{組み合わせのコンパイル}
\label{Section 5.5.3}

コンパイル処理の本質は手続適用のコンパイルです。与えられたターゲットとリンク記述子と
共にコンパイルされた組み合わせのコードは以下の形式を持ちます。

\begin{scheme}
~\( \dark \langle \)~~\emph{演算子のコンパイル, ターゲット \code{proc}, リンク記述子 \code{next}}~~\( \dark \rangle \)~
~\( \dark \langle \)~~\emph{オペランドを評価し、\code{argl}内に引数リストを構築}~~\( \dark \rangle \)~
~\( \dark \langle \)~~\emph{手続呼出のコンパイル}~ 
 ~\emph{与えられたターゲットとリンク記述子と共に}~~\( \dark \rangle \)~
\end{scheme}

\noindent
レジスタ\code{env}, \code{proc}, \code{argl}はオペレータ(演算子)とオペランドの評価の間に
保存と復元を行う必要があるかもしれません。ここだけがこのコンパイラにおいて
\code{val}以外のターゲットが指定される箇所であることに注意して下さい。

必要なコードは\code{compile\-/application}により生成されます。これは再帰的にオペレータを
コンパイルして\code{proc}に適用する手続を配置するコードを生成し、オペランドをコンパイルして
個別の適用の個々のオペランドを評価するコードを生成します。オペランドの命令列は
(\code{construct\-/arglist}により)\code{argl}に引数リストを構築するコードと共に組み合わされます。
そして結果となる引数リストのコードは手続のコードと(\code{compile\-/procedure\-/call}により生成された)
手続呼出を実行するコードと共に組み合わされます。コードの列の接続において、
\code{env}レジスタはオペレータの評価の周りにおいて維持(preserving)されなければなりません。
(オペレータの評価がオペランドの評価で必要とな\code{env}を変更する
可能性があるため)。そして\code{proc}レジスタは引数リストの周りで維持されなければなりません。
(オペランドの評価が実際の手続適用に必要な\code{proc}レジスタを変更するかもしれないため)。
\code{Continue}もまたその間中、維持されなければなりません。手続呼出のリンクコードが
必要とするためです。

\begin{scheme}
(define (compile-application exp target linkage)
  (let ((proc-code (compile (operator exp) 'proc 'next))
        (operand-codes
         (map (lambda 
                (operand) (compile operand 'val 'next))
              (operands exp))))
    (preserving '(env continue)
     proc-code
     (preserving '(proc continue)
      (construct-arglist operand-codes)
      (compile-procedure-call target linkage)))))
\end{scheme}

\noindent
引数リストを構築するためのコードは\code{val}内に評価して、次にその値を\code{argl}に蓄積される
引数リスト上に\code{cons}します。\code{argl}上に順に引数を\code{cons}するため、最後の引数から
開始し、最初のもので終わらなければなりません。そうすることで引数は結果リストの中に
最初から最後の順で現れることになります。この一連の評価のための設定を行うため、\code{argl}を
空に初期化することで命令を無駄にするのではなく、\code{argl}の初期値を構築する最初の
コード列を作成します。従って、引数リスト構築の一般的な形式は以下になります。

\begin{scheme}
~\( \dark \langle \)~~\emph{最後のオペランドのコンパイル, ターゲットは\code{val}}~~\( \dark \rangle \)~
(assign argl (op list) (reg val))
~\( \dark \langle \)~~\emph{次のオペランドのコンパイル, ターゲットは\code{val}}~~\( \dark \rangle \)~
(assign argl (op cons) (reg val) (reg argl))
~\( \dots \)~
~\( \dark \langle \)~~\emph{最初のオペランドのコンパイル, ターゲットは\code{val}}~~\( \dark \rangle \)~
(assign argl (op cons) (reg val) (reg argl))
\end{scheme}

\noindent
\code{argl}は各オペランドの評価の間、最初の1つを除いて維持しなければなりません。
(そうすることで、そこまで蓄積した引数を失わないように)。そして\code{env}は
(続きのオペランド評価での使用のため)各オペランドの評価の周りで、
最後の1つを除いて維持されなければなりません。

この引数コードのコンパイルは少しだけ巧妙です。評価する最初のオペランドの特別な
扱いと、\code{argl}と\code{env}を異なる箇所にて維持する必要性のためです。
\code{construct\-/arglist}手続は引数として個々のオペランドを評価するコードを取ります。
もしオペランドが全く無ければ、単純に以下の命令を発行します。

\begin{scheme}
(assign argl (const ()))
\end{scheme}

\noindent
そうでなければ、\code{construct\-/arglist}は\code{argl}を最後の引数で初期化するコードを
生成し、引数の残りを評価するコードを接続し、それらを相次いで\code{argl}の中に隣接させていきます。
引数を最後から最初へ処理するために、オペランドのコード列のリストを
\code{compile\-/application}により提供された順から逆順(reverse)にする必要があります。

\begin{scheme}
(define (construct-arglist operand-codes)
  (let ((operand-codes (reverse operand-codes)))
    (if (null? operand-codes)
        (make-instruction-sequence '() '(argl)
         '((assign argl (const ()))))
        (let ((code-to-get-last-arg
               (append-instruction-sequences
                (car operand-codes)
                (make-instruction-sequence '(val) '(argl)
                 '((assign argl (op list) (reg val)))))))
          (if (null? (cdr operand-codes))
              code-to-get-last-arg
              (preserving '(env)
               code-to-get-last-arg
               (code-to-get-rest-args
                (cdr operand-codes))))))))

(define (code-to-get-rest-args operand-codes)
  (let ((code-for-next-arg
         (preserving '(argl)
          (car operand-codes)
          (make-instruction-sequence '(val argl) '(argl)
           '((assign argl
              (op cons) (reg val) (reg argl)))))))
    (if (null? (cdr operand-codes))
        code-for-next-arg
        (preserving '(env)
         code-for-next-arg
         (code-to-get-rest-args (cdr operand-codes))))))
\end{scheme}

\subsubsection*{手続の適用}

組み合わせの要素を評価した後に、コンパイルされたコードは\code{proc}内の手続を
\code{argl}内の引数に適用しなければなりません。このコードは本質的に\link{Section 4.1.1}の
メタ循環評価機の\code{apply}手続、または\link{Section 5.4.1}の明示的制御評価機の
\code{apply\-/dispatch}エントリポイントと同じ割り振りを実行します。
適用する手続がプリミティブな手続であるか複合手続であるかを確認します。
プリミティブな手続に対しては、\code{apply\-/primitive\-/procedure}を使用します。
簡潔にこれがどのようにコンパイルされた手続を取り扱うのかについて見ていきます。
手続適用のコードは以下の形式を持ちます。

\begin{scheme}
(test (op primitive-procedure?) (reg proc))
 (branch (label primitive-branch))
compiled-branch
 ~\( \dark \langle \)~~\emph{与えられたターゲットと適切なリンク記述子と共に 
  手続をコンパイルするコード}~~\( \dark \rangle \)~
primitive-branch
 (assign ~\( \dark \langle \)~~\var{\dark target}~~\( \dark \rangle \)~
         (op apply-primitive-procedure)
         (reg proc)
         (reg argl))
 ~\( \dark \langle \)~~\var{\dark リンクコード}~~\( \dark \rangle \)~
after-call
\end{scheme}

\noindent
コンパイルされた分岐はprimitive-branchをスキップしなければならないことに注意して下さい。
従って、もし元の手続呼出のリンク記述子が\code{next}ならば、複合分岐はprimitive-branchの後に
挿入されたラベルへと飛ぶリンクコードを使用しなければなりません。(これは\code{compile\-/if}に
おいて、真の分岐のために使用されたリンクコードと同様です。)

\begin{scheme}
(define (compile-procedure-call target linkage)
  (let ((primitive-branch (make-label 'primitive-branch))
        (compiled-branch (make-label 'compiled-branch))
        (after-call (make-label 'after-call)))

    (let ((compiled-linkage
           (if (eq? linkage 'next) after-call linkage)))
      (append-instruction-sequences
       (make-instruction-sequence '(proc) '()
        `((test (op primitive-procedure?) (reg proc))
          (branch (label ,primitive-branch))))
       (parallel-instruction-sequences
        (append-instruction-sequences
         compiled-branch
         (compile-proc-appl target compiled-linkage))
        (append-instruction-sequences
         primitive-branch
         (end-with-linkage linkage
          (make-instruction-sequence '(proc argl)
                                     (list target)
           `((assign ,target
                     (op apply-primitive-procedure)
                     (reg proc)
                     (reg argl)))))))
       after-call))))
\end{scheme}

\noindent
\code{compile\-/if}の真と偽の分岐のような、プリミティブかつ、複合な分岐は通常の
\code{append\-/instruction\-/sequences}ではなく\code{parallel\-/instruction\-/sequences}を
用いて接続されます。それらは順には実行されないためです。

\subsubsection*{コンパイル済み手続の適用}

手続の適用を取り扱うコードはコンパイラの最も微妙な部分です。例えそれが生成する命令列が
とても短くても変わりません。(\code{compile\-/lambda}により構築されたような)コンパイルされた手続は
手続が開始する場所を指定するラベルであるエントリポイントを持ちます。
このエントリポイントにてコードは\code{val}に結果を求め、命令\code{(goto (reg continue))}を実行すること
により戻ります。従って与えられたターゲットとリンク記述子を伴なう(\code{compile\-/proc\-/appl}により生成される)
コンパイルされた手続の適用はリンク記述子がラベルであれば以下のようになります。

\begin{scheme}
(assign continue (label proc-return))
 (assign val (op compiled-procedure-entry) (reg proc))
 (goto (reg val))
proc-return
 (assign ~\( \dark \langle \)~~\var{\dark target}~~\( \dark \rangle \)~ (reg val))   ~\textrm{; ターゲットが\code{val}でなければ含まれる}~
 (goto (label ~\( \dark \langle \)~~\var{\dark リンク記述子}~~\( \dark \rangle \)~))   ~\textrm{; リンクコード}~
\end{scheme}

\noindent
またはリンク記述子が\code{return}の場合は次のとおりです。

\begin{scheme}
(save continue)
 (assign continue (label proc-return))
 (assign val (op compiled-procedure-entry) (reg proc))
 (goto (reg val))
proc-return
 (assign ~\( \dark \langle \)~~\var{\dark target}~~\( \dark \rangle \)~ (reg val))   ~\textrm{; ターゲットが\code{val}でなければ含まれる}~
 (restore continue)
 (goto (reg continue))         ~\textrm{; リンクコード}~
\end{scheme}

\noindent
このコードは手続が\code{proc\-/return}に戻るように\code{continue}を設定し、手続の
エントリポイントへと飛びます。\code{proc\-/return}のコードは手続の結果を\code{val}から
ターゲットレジスタへと(もし必要なら)転送し、次にリンク記述子により指定された
位置へと飛びます。(リンク記述子は常に\code{return}かラベルです。なぜなら
\code{compile\-/procedure\-/call}が複合手続の分岐のためのリンク記述子\code{next}を
\code{after\-/call}ラベルに置き換えるためです。)

実際には、もしターゲットが\code{val}でなければ、それはまさに私達のコンパイラが
生成するコードです。\footnote{実際に、ターゲットが\code{val}でなく、リンク記述子が
\code{return}である場合にはエラーを発します。私達がリンク記述子\code{return}を要求する
箇所は手続のコンパイル内のみです。そして私達の仕様は、手続はその値を\code{val}にて返す、です。}
しかし、通常はターゲットは\code{val}であり(コンパイラが異なるレジスタを指定する唯一の場合は
オペレータの評価のターゲットを\code{proc}にする時です)、そのため手続の結果は直接
ターゲットレジスタに入れられ、コピーを行う特別な位置へ戻る必要はありません。その代わりに、
手続が直接呼び出し元のリンク記述子により指定される場所へ直接``戻る''ように\code{continue}を
設定します。

\begin{scheme}
~\( \dark \langle \)~~\emph{\code{continue}にリンク記述子を設定}~~\( \dark \rangle \)~
(assign val (op compiled-procedure-entry) (reg proc))
(goto (reg val))
\end{scheme}

\noindent
もしリンク記述子がラベルならば、手続がそのラベルに戻るように\code{continue}を設定します。
(つまり、上記の\code{proc\-/return}において手続の終端\code{(goto (reg continue))}が
\code{(goto (label <\var{linkage}>))}と等価になります。)

\begin{scheme}
(assign continue (label ~\( \dark \langle \)~~\var{\dark リンク記述子}~~\( \dark \rangle \)~))
(assign val (op compiled-procedure-entry) (reg proc))
(goto (reg val))
\end{scheme}

\noindent
もしリンク記述子が\code{return}なら、\code{continue}を設定する必要は全くありません。
それは既に望まれた位置を持っています。(言い換えれば、手続の終端\code{(goto (reg
continue))}は\code{proc\-/return}の\code{(goto (reg continue))}が飛ぶはずだった場所へ
直接飛びます。)

\begin{scheme}
(assign val (op compiled-procedure-entry) (reg proc))
(goto (reg val))
\end{scheme}

\noindent
このリンク記述子\code{return}の実装を用いて、コンパイラは末尾再帰のコードを生成します。
手続のボディの最後ステップとしての手続の呼出は直接移動を行いスタック上にどのような
情報も保存しません。

その代わりに手続呼出の場合をリンク記述子\code{return}とターゲット\code{val}を用いて、
上記で示されたように\code{val}以外のターゲットに対しても取り扱ったと仮定します。
これは末尾再帰を損うでしょう。それでも、私達のシステムは任意の式に対して同じ値を
与えます。しかし、私達が手続を呼ぶ度に、\code{continue}を保存し、呼出の後に
(必要の無い)保存の取消を呼び出すことになります。これらの余分な保存が入れ子の手続
呼出の間に蓄積されます。\footnote{コンパイラに末尾再帰のコードを生成させることは
簡単な考えのように見えるかもしれません。しかし一般的な言語のための多くのコンパイラは
C言語とPascalを含めて、これを行いません。従ってこれらの言語は反復プロセスを手続
呼出のみを用いて表現することができません。これらの言語における末尾再帰の困難さは
それらの実装がスタックを用いて手続の引数とローカル変数と同様にリターンアドレスを
も格納しているためです。この本で説明されているSchemeの実装は引数と変数をガベージ
コレクションされるようにメモリに保存します。変数と引数に対してスタックを使用する
理由は他のやり方によりガベージコレクションの必要の無い言語内で、その必要性を回避するからです。
そして一般的にはより効率的になると信じられています。実際には、最新のLispコンパイラは
末尾再帰を無効化せずにスタックを引数のために使用することができます。(このことの説明に
関しては\link{Hanson 1990}を参照して下さい)。またスタックの割当がそもそもガベージ
コレクションより効率的であるかどうかについての討論もいくつか存在します。しかし、
詳細はコンピュータアーキテクチャの委細に依存しているように見えます。(この問題の
反対の立場からの視点については\link{Appel 1987}と\link{Miller and Rozas 1994}を参照して下さい。)}

code{compile\-/proc\-/appl}は上記の手続適用のコードを生成します。これは呼出のためのターゲットが
\code{val}であるか、そしてリンク記述子が\code{return}であるかについてに依存する4つの場合について
考慮します。命令列が全てのレジスタを変更するために宣言されることについて注意して下さい。
手続のボディの実行が自由な形でレジスタを変更することができるためです。\footnote{
変数\code{all\-/regs}は全てのレジスタの名前のリストに対して束縛されます。

\begin{smallscheme}
(define all-regs '(env proc val argl continue))
\end{smallscheme}
}
またターゲットが\code{val}であり、リンク記述子が\code{return}の場合に対するコードの列は
\code{continue}を必要とすると宣言されていることに注意して下さい。例え\code{continue}が
明示的に2つの命令列の中で使用されていなくとも、私達がコンパイルされた手続を入力した
時に\code{continue}が正しい値を持つことを確実にしなければなりません。

\begin{scheme}
(define (compile-proc-appl target linkage)
  (cond ((and (eq? target 'val) (not (eq? linkage 'return)))
         (make-instruction-sequence '(proc) all-regs
           `((assign continue (label ,linkage))
             (assign val (op compiled-procedure-entry)
                         (reg proc))
             (goto (reg val)))))
        ((and (not (eq? target 'val))
              (not (eq? linkage 'return)))
         (let ((proc-return (make-label 'proc-return)))
           (make-instruction-sequence '(proc) all-regs
            `((assign continue (label ,proc-return))
              (assign val (op compiled-procedure-entry)
                          (reg proc))
              (goto (reg val))
              ,proc-return
              (assign ,target (reg val))
              (goto (label ,linkage))))))
        ((and (eq? target 'val) (eq? linkage 'return))
         (make-instruction-sequence 
          '(proc continue) 
          all-regs
          '((assign val (op compiled-procedure-entry)
                        (reg proc))
            (goto (reg val)))))
        ((and (not (eq? target 'val))
              (eq? linkage 'return))
         (error "return linkage, target not val: COMPILE"
                target))))
\end{scheme}

\subsection{命令列のコンパイル}
\label{Sec. 5.5.4}
\label{Section 5.5.4}

この節では命令列がどのように表現され、組み合わされるのかについての詳細を説明します。
\link{Section 5.5.1}から命令列が必要なレジスタのリスト、変更されるレジスタ、実際の命令のリスト
として表現されたことを思い出して下さい。またラベル(シンボル)を命令列の退化した場合だと
考慮します。これはどのレジスタも必要とせず、また変更しません。故に、命令列により必要とされる、または
変更されるレジスタを決定するために以下のセレクタを用います。

\begin{scheme}
(define (registers-needed s)
  (if (symbol? s) '() (car s)))
(define (registers-modified s)
  (if (symbol? s) '() (cadr s)))
(define (statements s)
  (if (symbol? s) (list s) (caddr s)))
\end{scheme}

\noindent
また与えられた命令列が与えられたレジスタを必要とするか、変更するかを決定するために
以下の述語を用います。

\begin{scheme}
(define (needs-register? seq reg)
  (memq reg (registers-needed seq)))
(define (modifies-register? seq reg)
  (memq reg (registers-modified seq)))
\end{scheme}

\noindent
これらの述語とセレクタを用いて、コンパイラを通して使用される様々な命令列の
結合器(combiner)を実装することができます。

基本的な結合器は\code{append\-/instruction\-/sequences}です。これは引数として順に実行される
任意の数の命令列を取り、命令文(statement)が全ての命令列の命令文を一緒に接続した命令文
である命令列を返します。結果の命令列により必要とされる、または変更されるレジスタの
決定が繊細な点になります。これは命令列のどれかにより変更されるレジスタが変更されます。
またこれは最初の命令列が実行する前に初期化されなければならないレジスタ(最初の命令列で
必要とされるレジスタ)に加えて、それに続く命令列により初期化されない(変更されない)
他の命令列により必要とされるレジスタ全てです。

命令列は\code{append\-/2\-/sequences}により一度に2つが接続されます。これは2つの命令列
\code{seq1}と\code{seq2}を取り、命令文が\code{seq1}の命令文の後に\code{seq2}の命令文が置かれる命令列を
返します。これの変更されたレジスタは\code{seq1}か\code{seq2}のどちらかにより変更された
レジスタです。そして必要とされるレジスタは\code{seq1}により必要とされるレジスタと
\code{seq2}で必要とされ\code{seq1}で変更されないレジスタを加えたものです。
(集合の命令を用いて、必要なレジスタの新しい集合は\code{seq1}により必要とされるレジスタの
集合と、\code{seq2}により必要とされるレジスタと\code{seq1}により変更されたレジスタの
差集合との、和集合です。)

\begin{scheme}
(define (append-instruction-sequences . seqs)
  (define (append-2-sequences seq1 seq2)
    (make-instruction-sequence
     (list-union
      (registers-needed seq1)
      (list-difference (registers-needed seq2)
                       (registers-modified seq1)))
     (list-union (registers-modified seq1)
                 (registers-modified seq2))
     (append (statements seq1) (statements seq2))))
  (define (append-seq-list seqs)
    (if (null? seqs)
        (empty-instruction-sequence)
        (append-2-sequences
         (car seqs)
         (append-seq-list (cdr seqs)))))
  (append-seq-list seqs))
\end{scheme}

\noindent
この手続はリストとして表現された集合を操作するためのいくつかの簡単な命令を使います。
\link{Section 2.3.3}で説明された(順序無し)集合表現と同様です。

\begin{scheme}
(define (list-union s1 s2)
  (cond ((null? s1) s2)
        ((memq (car s1) s2) (list-union (cdr s1) s2))
        (else (cons (car s1) (list-union (cdr s1) s2)))))
(define (list-difference s1 s2)
  (cond ((null? s1) '())
        ((memq (car s1) s2) (list-difference (cdr s1) s2))
        (else (cons (car s1)
                    (list-difference (cdr s1) s2)))))
\end{scheme}

\noindent
\code{preserving}は2つ目の主な命令列結合器ですが、レジスタのリスト\code{regs}と順に実行する2つの命令列
\code{seq1}と\code{seq2}を取ります。これは\code{seq1}の命令文(statements)のその後に
\code{seq2}の命令文が続く命令文を持つ命令列を返します。この命令文には\code{seq1}により
変更されるが\code{seq2}で必要とされる\code{regs}内のレジスタを守るために\code{seq1}の周りに
適切な\code{save}と\code{restore}が追加されます。これを達成するために、\code{preserving}は
最初に必要とされる\code{save}とそれに続く\code{seq1}、それに続く必要とされる\code{restore}を
持つ命令列を作ります。この命令列は\code{seq1}により必要とされるレジスタに加えて
レジスタの保存と復元を必要とします。そして\code{seq1}で変更されたレジスタを保存と
回復が行われるものを除いて変更します。次に、この増補された命令列と\code{seq2}が通常の
方法で接続されます。以下の手続はこの戦略を、維持されるべきレジスタのリストを横断しながら
再帰的に実装します。\footnote{\code{preserving}が\code{append}を3つの引数と共に呼び出すことに
注意して下さい。この本に表われる\code{append}の定義は2つの引数しか受け付けませんが、
Schemeの標準は任意の数の引数を取る\code{append}手続を提供します。}

\begin{scheme}
(define (preserving regs seq1 seq2)
  (if (null? regs)
      (append-instruction-sequences seq1 seq2)
      (let ((first-reg (car regs)))
        (if (and (needs-register? seq2 first-reg)
                 (modifies-register? seq1 first-reg))
            (preserving (cdr regs)
             (make-instruction-sequence
              (list-union (list first-reg)
                          (registers-needed seq1))
              (list-difference (registers-modified seq1)
                               (list first-reg))
              (append `((save ,first-reg))
                      (statements seq1)
                      `((restore ,first-reg))))
             seq2)
            (preserving (cdr regs) seq1 seq2)))))
\end{scheme}

\noindent
別の命令列結合器である\code{tack\-/on\-/instruction\-/sequence}は\code{compile\-/lambda}により
手続のボディを他の命令列に接続するために使用されます。手続のボディは組み合わされた列の
一部として実行されるための``インライン''形式ではないため、それによるレジスタの使用は
それが組込まれる命令列のレジスタ使用に影響を与えません。従って手続ボディの必要な、また
変更されるレジスタの集合は別の命令列に接続する時に無視されます。

\begin{scheme}
(define (tack-on-instruction-sequence seq body-seq)
  (make-instruction-sequence
   (registers-needed seq)
   (registers-modified seq)
   (append (statements seq)
           (statements body-seq))))
\end{scheme}

\noindent
\code{compile\-/if}と\code{compile\-/procedure\-/call}は\code{parallel\-/instruction\-/sequences}と
呼ばれる特別な結合器を使用してテストに続く二者択一の分岐を接続します。2つの分岐は
絶対に順には実行されません。どんなテストの評価に対しても、一方か、別の一方に入ります。
このため、2つ目の分岐により必要とされるレジスタは
例えもしこれらが1つ目の分岐により変更されようとも依然として結合後の命令列でも必要とします。

\begin{scheme}
(define (parallel-instruction-sequences seq1 seq2)
  (make-instruction-sequence
   (list-union (registers-needed seq1)
               (registers-needed seq2))
   (list-union (registers-modified seq1)
               (registers-modified seq2))
   (append (statements seq1)
           (statements seq2))))
\end{scheme}

\subsection{コンパイルされたコードの例}
\label{Section 5.5.5}

これでコンパイラの全ての要素について学び終えました。
ここまでのものがどのように御互いに組合せられるのかを見るためにコンパイル済みのコードの例を
試してみましょう。再帰\code{factorial}手続の定義を\code{compile}を呼ぶことでコンパイルしてみます。

\begin{scheme}
(compile
 '(define (factorial n)
    (if (= n 1)
        1
        (* (factorial (- n 1)) n)))
 'val
 'next)
\end{scheme}

\noindent
\code{define}式の値はレジスタ\code{val}に配置されなければならないと指定しました。
私達は\code{define}を実行した後にコンパイル済みコードが何を行うのか気にしません。
そのためリンク記述子に対する\code{next}の選択は気まぐれです。

\code{compile}は式が定義であるかを判断します。そのため\code{compile\-/definition}を
呼び出し(ターゲット\code{val}に対し)割り当てられるべき値を求めるコードをコンパイルします。
続いて定義を導入するコード、さらに\code{define}の値(シンボル\code{ok})をターゲットレジスタに
入れるコード、最後にリンクコードが続きます。\code{env}は値の演算の周りで維持されます。
定義の導入のために必要とされるためです。今回のリンク記述子は\code{next}ですから、
リンクコードは存在しません。従ってコンパイルされたコードの骨格は以下のようになります。

\begin{scheme}
~\( \dark \langle \)~~\emph{値を求めるコードで変更されるなら\code{env}を保存}~~\( \dark \rangle \)~
  ~\( \dark \langle \)~~\emph{定義値、ターゲット\code{val}、
   リンク記述子\code{next}のコンパイル}~~\( \dark \rangle \)~
  ~\( \dark \langle \)~~\emph{上で保存したなら\code{env}の復元}~~\( \dark \rangle \)~
  (perform (op define-variable!)
           (const factorial)
           (reg val)
           (reg env))
  (assign val (const ok))
\end{scheme}

\noindent
変数\code{factorial}に対する値を生成するためにコンパイルされる式は、値が階乗を計算する
手続である\code{lambda}式です。\code{compile}は\code{compile\-/lambda}を呼ぶことによりこれを
扱います。\code{compile\-/lambda}は手続のボディをコンパイルし、それに新しいエントリポイントと
してラベル付けを行い、新しいエントリポイントの手続ボディを実行時環境と組み合わせ、
結果を\code{val}に割り当てるコードを生成します。次に命令列はこの時点で挿入された、
このコンパイルされたコードをスキップします。手続のコードそれ自体は手続定義環境を
形式パラメタ\code{n}を手続の引数に束縛するフレームにより拡張することから始めます。
その次に実際の手続のボディが来ます。変数の値のためのこのコードは\code{env}レジスタを
変更しませんので、上で示された任意の\code{save}と\code{restore}は生成されません。
(\code{entry2}における手続のコードはこの時点では実行されません。そのため、その\code{env}の
使用は無関係です)。従って、コンパイルされたコードの骨組は以下のようになります。

\begin{scheme}
  (assign val
          (op make-compiled-procedure)
          (label entry2)
          (reg env))
  (goto (label after-lambda1))
entry2
  (assign env (op compiled-procedure-env) (reg proc))
  (assign env
          (op extend-environment)
          (const (n))
          (reg argl)
          (reg env))
  ~\( \dark \langle \)~~\emph{手続ボディのコンパイル}~~\( \dark \rangle \)~
after-lambda1
  (perform (op define-variable!)
           (const factorial)
           (reg val)
           (reg env))
  (assign val (const ok))
\end{scheme}

\noindent
手続のボディは常に(\code{compile\-/lambda\-/body}により)、ターゲット\code{val}と
リンク記述子\code{return}を用いる命令列としてコンパイルされます。今回の場合の
命令列は単一の\code{if}式から成り立ちます。

\begin{scheme}
(if (= n 1)
    1
    (* (factorial (- n 1)) n))
\end{scheme}

\noindent
\code{compile\-/if}は最初に述語を演算し(ターゲットは\code{val})、次にその結果を確認して
述語が偽であれば真の分岐を回避します。\code{env}と\code{continue}が述語のコードの周りで
維持されます。それらが\code{if}式の残りの部分で必要となる可能性があるためです。
\code{if}式が手続のボディを構成する命令列内の最後の式であるため(そしてただ1つの式であるため)、
そのターゲットは\code{val}で、リンク記述子は\code{return}になります。そのため真と偽の両方の
分岐がターゲット\code{val}とリンク記述子\code{return}と共にコンパイルされます。
(言い換えれば、どちらかの分岐により値が演算される条件文の値がその手続の値です。)

\begin{scheme}
~\( \dark \langle \)~~\emph{述語により変更され、分岐により必要とされるなら 
 \code{continue}, \code{env}を保存する}~~\( \dark \rangle \)~
  ~\( \dark \langle \)~~\emph{述語, ターゲット\code{val}, リンク記述子\code{next}のコンパイル}~~\( \dark \rangle \)~
  ~\( \dark \langle \)~~\emph{上で保存したなら\code{continue}, \code{env}を復元する}~~\( \dark \rangle \)~
  (test (op false?) (reg val))
  (branch (label false-branch4))
true-branch5
  ~\( \dark \langle \)~~\emph{真の分岐, ターゲット\code{val}, リンク記述子\code{return}のコンパイル}~~\( \dark \rangle \)~
false-branch4
  ~\( \dark \langle \)~~\emph{偽の分岐, ターゲット\code{val}, リンク記述子\code{return}のコンパイル}~~\( \dark \rangle \)~
after-if3
\end{scheme}

\noindent
述語\code{(= n 1)}は手続の呼出です。これはオペレータ(シンボル\code{=})を探し、その値を
\code{proc}内に配置します。次に引数\code{1}と変数\code{n}を\code{argl}に集めます。そして
\code{proc}がプリミティブ、または複合手続を含むかどうかをテストし、それに応じて
プリミティブの分岐か複合の分岐へ飛びます。両方の分岐がラベル\code{after\-/call}にて
再開します。オペレータとオペランドの評価の周りでレジスタを維持する必要性は
どのレジスタも保存することにはなりません。今回の場合はそれらの評価は問題となる
レジスタを変更しないためです。

\begin{scheme}
  (assign proc
          (op lookup-variable-value) (const =) (reg env))
  (assign val (const 1))
  (assign argl (op list) (reg val))
  (assign val
          (op lookup-variable-value) (const n) (reg env))
  (assign argl (op cons) (reg val) (reg argl))
  (test (op primitive-procedure?) (reg proc))
  (branch (label primitive-branch17))
compiled-branch16
  (assign continue (label after-call15))
  (assign val (op compiled-procedure-entry) (reg proc))
  (goto (reg val))
primitive-branch17
  (assign val
          (op apply-primitive-procedure)
          (reg proc)
          (reg argl))
after-call15
\end{scheme}

\noindent
真の分岐は定数1ですが、(ターゲット\code{val}とリンク記述子\code{return}と共に)以下の
ようにコンパイルされます。

\begin{scheme}
(assign val (const 1))
(goto (reg continue))
\end{scheme}

\noindent
偽の分岐のコードは別の手続呼出です。手続はシンボル\code{*}で、その引数は\code{n}と
別の手続呼出の結果(\code{factorial}の呼出)です。これらの呼出の全てが\code{proc}と\code{argl}、
それ自身のプリミティブと複合の分岐の準備を行います。\link{Figure 5.17}は手続
\code{factorial}の定義の完全なコンパイルを示します。述語の周りで可能性のある\code{continue}と\code{env}の
\code{save}と\code{restore}が実際に生成されていることに注意して下さい。これらのレジスタが
述語内の手続呼出にて変更され、また分岐内の手続呼び出しと\code{return}のリンクコードにより
必要とされるためです。

\begin{quote}
\heading{\phantomsection\label{Exercise 5.33}Exercise 5.33:}
上で与えられたものとは微妙に異なる以下の階乗手続の定義について考えよ。

\begin{scheme}
(define (factorial-alt n)
  (if (= n 1)
      1
      (* n (factorial-alt (- n 1)))))
\end{scheme}


この手続をコンパイルし結果のコードを\code{factorial}に対して生成されたコードと比べよ。
見つけた全ての違いについて説明せよ。どちらのプログラムが他方よりもより効率的に
実行するだろうか?
\end{quote}

\begin{quote}
\heading{\phantomsection\label{Exercise 5.34}Exercise 5.34:}
反復階乗手続をコンパイルせよ

\begin{scheme}
(define (factorial n)
  (define (iter product counter)
    (if (> counter n)
        product
        (iter (* counter product)
              (+ counter 1))))
  (iter 1 1))
\end{scheme}


結果のコードに注釈を付け、一方のプロセスがスタック領域を増進させ、他方が一定の
スタック領域で実行される元となる、\code{factorial}の反復版と再帰版のコードの間の
本質的な違いを示せ。
\end{quote}

\begin{quote}
\heading{\phantomsection\label{Figure 5.17}Figure 5.17:} \( \downarrow \) \code{factorial}手続定義のコンパイル結果

\begin{smallscheme}
~\textrm{;; 手続を構築し、手続のボディのコードを飛ばす}~
  (assign val
          (op make-compiled-procedure) 
          (label entry2) 
          (reg env))
  (goto (label after-lambda1))
entry2     ~\textrm{; \code{factorial}の呼出はここから入る}~
  (assign env (op compiled-procedure-env) (reg proc))
  (assign env
          (op extend-environment) 
          (const (n)) 
          (reg argl) 
          (reg env))
~\textrm{;; 実際の手続のボディを開始する}~
  (save continue)
  (save env)
~\textrm{;; \code{(= n 1)}を求める}~
  (assign proc 
          (op lookup-variable-value) 
          (const =) 
          (reg env))
  (assign val (const 1))
  (assign argl (op list) (reg val))
  (assign val 
          (op lookup-variable-value) 
          (const n) 
          (reg env))
  (assign argl (op cons) (reg val) (reg argl))
  (test (op primitive-procedure?) (reg proc))
  (branch (label primitive-branch17))
compiled-branch16
  (assign continue (label after-call15))
  (assign val (op compiled-procedure-entry) (reg proc))
  (goto (reg val))
primitive-branch17
  (assign val 
          (op apply-primitive-procedure) 
          (reg proc) 
          (reg argl))
after-call15   ~\textrm{; ここで\code{val}は\code{(= n 1)}の結果を持つ}~
  (restore env)
  (restore continue)
  (test (op false?) (reg val))
  (branch (label false-branch4))
true-branch5  ~\textrm{; return 1}~
  (assign val (const 1))
  (goto (reg continue))
false-branch4
~\textrm{;; \code{(* (factorial (- n 1)) n)}を求めて返す}~
  (assign proc 
          (op lookup-variable-value) 
          (const *) 
          (reg env))
  (save continue)
  (save proc)   ~\textrm{; \code{*}手続を保存する}~
  (assign val 
          (op lookup-variable-value) 
          (const n) 
          (reg env))
  (assign argl (op list) (reg val))
  (save argl)   ~\textrm{; \code{*}の引数リストの一部を保存}~
~\textrm{;; \code{(factorial (- n 1))}を求める。これは\code{*}のもう一方の引数}~
  (assign proc
          (op lookup-variable-value) 
          (const factorial) 
          (reg env))
  (save proc)  ~\textrm{; \code{factorial}手続を保存}~
~\textrm{;; \code{(- n 1)}を求める。これは\code{factorial}に対する引数}~
  (assign proc 
          (op lookup-variable-value)
          (const -) 
          (reg env))
  (assign val (const 1))
  (assign argl (op list) (reg val))
  (assign val 
          (op lookup-variable-value) 
          (const n) 
          (reg env))
  (assign argl (op cons) (reg val) (reg argl))
  (test (op primitive-procedure?) (reg proc))
  (branch (label primitive-branch8))
compiled-branch7
  (assign continue (label after-call6))
  (assign val (op compiled-procedure-entry) (reg proc))
  (goto (reg val))
primitive-branch8
  (assign val 
          (op apply-primitive-procedure) 
          (reg proc) 
          (reg argl))
after-call6   ~\textrm{; ここで\code{val}は\code{(- n 1)}の結果を持つ}~
  (assign argl (op list) (reg val))
  (restore proc) ~\textrm{; \code{factorial}に戻す}~
~\textrm{;; \code{factorial}の適用}~
  (test (op primitive-procedure?) (reg proc))
  (branch (label primitive-branch11))
compiled-branch10
  (assign continue (label after-call9))
  (assign val (op compiled-procedure-entry) (reg proc))
  (goto (reg val))
primitive-branch11
  (assign val 
          (op apply-primitive-procedure) 
          (reg proc) 
          (reg argl))
after-call9      ~\textrm{; ここで\code{val}は\code{(factorial (- n 1))}の結果を持つ}~
  (restore argl) ~\textrm{; \code{*}の引数リストの一部を復元}~
  (assign argl (op cons) (reg val) (reg argl))
  (restore proc) ~\textrm{; \code{*}に戻す}~
  (restore continue)
~\textrm{;; \code{*}を適用しその値を返す}~
  (test (op primitive-procedure?) (reg proc))
  (branch (label primitive-branch14))
compiled-branch13
~\textrm{;; ここの複合手続は末尾再帰で呼ばれることに注意すること}~
  (assign val (op compiled-procedure-entry) (reg proc))
  (goto (reg val))
primitive-branch14
  (assign val 
          (op apply-primitive-procedure) 
          (reg proc) 
          (reg argl))
  (goto (reg continue))
after-call12
after-if3
after-lambda1
~\textrm{;; 手続を変数\code{factorial}に割り当てる}~
  (perform (op define-variable!) 
           (const factorial) 
           (reg val) 
           (reg env))
  (assign val (const ok))
\end{smallscheme}

\end{quote}

\begin{quote}
\heading{\phantomsection\label{Exercise 5.35}Exercise 5.35:}
どの式がコンパイルされると\link{Figure 5.18}に示されるコードを生成するか?
\end{quote}

\begin{quote}
\heading{\phantomsection\label{Figure 5.18}Figure 5.18:} \( \downarrow \) コンパイラ出力の例。\link{Exercise 5.35}参照

\begin{smallscheme}
(assign val 
        (op make-compiled-procedure) 
        (label entry16) 
        (reg env))
  (goto (label after-lambda15))
entry16
  (assign env (op compiled-procedure-env) (reg proc))
  (assign env
          (op extend-environment) 
          (const (x)) 
          (reg argl) 
          (reg env))
  (assign proc 
          (op lookup-variable-value) 
          (const +) 
          (reg env))
  (save continue)
  (save proc)
  (save env)
  (assign proc 
          (op lookup-variable-value) 
          (const g) 
          (reg env))
  (save proc)
  (assign proc 
          (op lookup-variable-value) 
          (const +) 
          (reg env))
  (assign val (const 2))
  (assign argl (op list) (reg val))
  (assign val 
          (op lookup-variable-value) 
          (const x) 
          (reg env))
  (assign argl (op cons) (reg val) (reg argl))
  (test (op primitive-procedure?) (reg proc))
  (branch (label primitive-branch19))
compiled-branch18
  (assign continue (label after-call17))
  (assign val (op compiled-procedure-entry) (reg proc))
  (goto (reg val))
primitive-branch19
  (assign val 
          (op apply-primitive-procedure) 
          (reg proc) 
          (reg argl))
after-call17
  (assign argl (op list) (reg val))
  (restore proc)
  (test (op primitive-procedure?) (reg proc))
  (branch (label primitive-branch22))
compiled-branch21
  (assign continue (label after-call20))
  (assign val (op compiled-procedure-entry) (reg proc))
  (goto (reg val))
primitive-branch22
  (assign val 
          (op apply-primitive-procedure) 
          (reg proc) 
          (reg argl))
after-call20
  (assign argl (op list) (reg val))
  (restore env)
  (assign val 
          (op lookup-variable-value) 
          (const x) 
          (reg env))
  (assign argl (op cons) (reg val) (reg argl))
  (restore proc)
  (restore continue)
  (test (op primitive-procedure?) (reg proc))
  (branch (label primitive-branch25))
compiled-branch24
  (assign val 
          (op compiled-procedure-entry) 
          (reg proc))
  (goto (reg val))
primitive-branch25
  (assign val 
          (op apply-primitive-procedure) 
          (reg proc) 
          (reg argl))
  (goto (reg continue))
after-call23
after-lambda15
  (perform (op define-variable!) 
           (const f) 
           (reg val) 
           (reg env))
  (assign val (const ok))
\end{smallscheme}

\end{quote}

\begin{quote}
\heading{\phantomsection\label{Exercise 5.36}Exercise 5.36:}
私達のコンパイラが生成する組み合わせのオペランドに対する評価の順はどれか?
左から右であるか、右から左であるか、または何らかの他の順であるか?
コンパイラの中のどこがこの順を決定するか? コンパイラを変更し、それが何らかの
別の評価順を生成するようにせよ。(\link{Section 5.4.1}における明示的制御評価機の評価順の
議論を参考にせよ)。オペランドの評価順を変更することが引数リストを構築するコードの
効率にどのような影響があるか?
\end{quote}

\begin{quote}
\heading{\phantomsection\label{Exercise 5.37}Exercise 5.37:}
スタック使用の最適化のためのコンパイラの\code{preserving}の仕組みを理解する1つの方法は
この考えを用いなかった場合にどんな余分な命令が生成されるかを見てみることだ。
\code{preserving}を変更し、常に\code{save}と\code{restore}の命令を生成するようにせよ。
いくつかの簡単な式をコンパイルし、生成された不必要なスタック命令を確認せよ。
\code{preserving}の仕組みが失われていないものから生成されたコードと比較せよ。
\end{quote}

\begin{quote}
\heading{\phantomsection\label{Exercise 5.38}Exercise 5.38:}
私達のコンパイラは不必要なスタック命令を防ぐことに関して賢いものだ。しかし、
機械により提供されるプリミティブな命令を用いて言語のプリミティブな手続の
呼出をコンパイルすることに関しては全く賢くない。例えば、\code{(+ a 1)}を求める
ためにどれだけのコードがコンパイルされるか考えてみる。このコードは引数リストを
\code{argl}に準備し、(環境内でシンボル\code{+}を探すことにより見つけた)プリミティブな加算手続を
\code{proc}に入れる。そしてこの手続がプリミティブであるか複合であるかをテストする。
コンパイラは常にこのテストを実行するコードと、同様にプリミティブと複合の分岐の
ためのコード(内、一方のみが実行される)が生成される。私達はコントローラのプリミティブを
実装する部品を示さなかった。しかし、これらの命令が機械のデータパス内のプリミティブな
数値演算命令を利用することは仮定した。もしコンパイラがプリミティブを\newterm{open-code}
できたらどれだけ少ないコードが生成されたか考えよ。これはつまり、
もしこれらのプリミティブな機械語命令を直接使用するコードを生成することができれば、である。
式\code{(+ a 1)}は以下と同じくらい単純なものにコンパイルされるだろう。\footnote{私達は
同じシンボル\code{+}をソース言語の手続と機械語命令の両方を示すためにここで使用しました。
一般的に、ソース言語のプリミティブと機械のプリミティブの間に1対1の対応はありません。}

\begin{scheme}
(assign
 val (op lookup-variable-value) (const a) (reg env))
(assign val (op +) (reg val) (const 1))
\end{scheme}

この課題では私達のコンパイラを拡張し、選択されたプリミティブのopen-codeをサポートする。
特別な目的のコードがこれらのプリミティブな手続の呼出に対し、一般的な手続適用のコードの
代わりに生成される。これをサポートするためには、私達の機械に特別な引数レジスタ、
\code{arg1}と\code{arg2}を追加する。機械のプリミティブな数値演算子は入力を\code{arg1}と\code{arg2}から
得る。その結果は\code{val}, \code{arg1}, \code{arg2}のどれかに入れて良い。

コンパイラはソースプログラム内のopen-codeなプリミティブの適用を認識できなければ
ならない。\code{compile}手続に割り振りを追加し、
現在認識可能な予約語(特殊形式)に加えてこれらのプリミティブの名前を認識できるようにする。\footnote{プリミティブを
予約語に入れることは一般的には悪い考えです。そうするとユーザがこれらの名前を異なる
手続に束縛し直すことができなくなるためです。さらに、もし使用中のコンパイラに予約語を
追加すると、これらの名前で手続を定義した既存のプログラムが動作しなくなります。
この問題をどのように回避するかの見解については\link{Exercise 5.44}を参照して下さい。}
特殊形式のそれぞれに対してコンパイラはコード生成器を持つ。この課題ではopen-codeな
プリミティブのためのコード生成器の仲間を構築する。

\begin{enumerate}[a]

\item
open-codeなプリミティブ全ては特殊形式とは異なり、オペランドが評価されることを
必要とする。全てのopen-codeのコード生成器から使用されるコード生成器\code{spread\-/arguments}を書け。
\code{spread\-/arguments}はオペランドのリストを取り、与えられたオペランドを次に続く引数
レジスタをターゲットにコンパイルしなければならない。オペランドがopen-codeなプリミティブへの
呼出を含んでも良いことに注意すること。そのため引数レジスタはオペランド評価の間は
維持されなければならない。

\item
プリミティブな手続\code{=}, \code{*}, \code{-}, \code{+}のそれぞれに対してそのオペレータと
ターゲット、リンク記述子の組み合わせを取り引数をレジスタに入れ、与えられた
ターゲットをターゲットに取り、与えられたリンク記述子と共に命令を実行するコードを
生成するコード生成器を書け。2つのオペランドを扱う式を扱うのみで良い。
これらのコード生成器に対する割り振りを作成せよ。

\item
貴方の新しいコンパイラを階乗の例を用いて試してみよ。結果のコードをopen-code無しで
生成した結果と比較せよ。

\item
\code{+}と\code{*}のコード生成器を拡張し任意の数のオペランドを持つ式を取り扱えるようにせよ。
3つ以上のオペランドを持つ式は、それぞれが2つだけ入力を持つ命令の列にコンパイルしなけれ
ばならない。

\end{enumerate}
\end{quote}

\subsection{レキシカルアドレッシング}
\label{Section 5.5.6}

コンパイラにより実行される最も一般的な最適化の1つは変数検索の最適化です。
ここまで実装した私達のコンパイラは評価機の\code{lookup\-/variable\-/value}命令を用いる
コードを生成します。これは実行時環境を通してフレーム毎に取り組みながら、
変数を現在束縛されている全ての変数と比較することで変数の検索を行う。
この検索はもしフレームが深く入れ子になったり、変数の数が多い場合には高コストに
成り得ます。例えば以下の式を評価した結果の適用において、式\code{(* x y z)}の評価の
間に\code{x}の値を探す問題について考えみましょう。

\begin{scheme}
(let ((x 3) (y 4))
  (lambda (a b c d e)
    (let ((y (* a b x)) (z (+ c d x)))
      (* x y z))))
\end{scheme}

\noindent
\code{let}式は\code{lambda}の組み合わせのための単なる糖衣構文ですので、この式は以下と
等価です。

\begin{scheme}
((lambda (x y)
   (lambda (a b c d e)
     ((lambda (y z) (* x y z))
      (* a b x)
      (+ c d x))))
 3
 4)
\end{scheme}

\noindent
\code{lookup\-/variable\-/value}が\code{x}を検索する度に、シンボル\code{x}は \code{y}、または\code{z}に
\code{eq?}でないことを(最初のフレームで)確認しなければなりません。また
(2つ目のフレームにて)\code{a}, \code{b}, \code{c}, \code{d}, \code{e}についても同様に必要です。
差し当たり、私達のプログラムは\code{define}を使用しないと仮定します。つまり変数は
\code{lambda}の使用にのみ束縛されます。私達の言語はレキシカルスコープであるため、
任意の式のための実行時環境は式が現れるプログラムのレキシカルな(語彙的な)構造を
並列化する構造を持ちます。\footnote{これはもし内部定義を許可するのであれば、
それら全てを走査しない限りは正しくありません。\link{Exercise 5.43}を参照して下さい。}
従って、コンパイラは上の式を分析した時に、手続が適用される度に\code{(* x y z)}内の
変数\code{x}が現在のフレームから2つ外のフレームの最初の変数として見つかることを知ることができます。

私達は新しい種類の変数検索命令、\code{lexical\-/address\-/lookup}を発明することにより、この
事実を利用することができます。この命令は引数として環境と2つの数値から成る
\newterm{lexical address}(\jnewterm{レキシカルアドレス})を取ります。2つの数値は、いくつのフレームを
見送るかを指定する\newterm{frame number}とそのフレーム内でいくつの変数を見送るかを
指定する\newterm{displacement number}です。\code{lexical\-/address\-/lookup}は現在のフレームに
対して相対的なレキシカルアドレスに格納された変数の値を生成します。
もし私達の機械に\code{lexical\-/address\-/lookup}命令を追加したなら、コンパイラに対して
\code{lookup\-/variable\-/value}ではなく、この命令を使用して変数を参照するコードを生成させる
ことができます。同様に、コンパイルされたコードは\code{set\-/variable\-/value!}の代わりに
新しい\code{lexical\-/address\-/set!}命令を使用することができます。

そのようなコードを生成するためには、コンパイラは参照をコンパイルしようとする変数の
レキシカルアドレスを決定できなければなりません。プログラム中の変数のレキシカルアドレスは
それがコードのどこにあるのかに依存します。例えば、以下のプログラムでは式
\( \langle \)\var{e1}\( \kern0.08em\rangle \)のアドレスは(2, 0)です。
つまり、2フレーム後ろでそのフレームの最初の変数です。同じ地点で\code{y}はアドレス(0, 0)であり、
\code{c}はアドレス(1, 2)に存在します。式\( \langle \)\var{e2}\( \kern0.09em\rangle \)においては、
\code{x}は(1, 0)に、\code{y}は(1, 1)に、\code{c}は(0, 2)に存在します。

\begin{scheme}
((lambda (x y)
   (lambda (a b c d e)
     ((lambda (y z) ~\( \dark \langle \)~~\var{\dark e1}~~\( \dark \rangle \)~)
      ~\( \dark \langle \)~~\var{\dark e2}~~\( \dark \rangle \)~
      (+ c d x))))
 3
 4)
\end{scheme}

\noindent
コンパイラにとってレキシカルアドレスを使用する1つの方法は
\newterm{compile-time environment}(\jnewterm{コンパイル時環境})と呼ばれるデータ構造を管理することです。
これは実行時環境内にて特定の変数アクセス命令が実行された時に、どの変数がどのフレーム内の
どの位置に存在することになるのかを追跡します。コンパイル時環境はフレームのリストであり、
各フレームが変数の変数のリストを保持します。(もちろん値が束縛されない変数も存在します。
値はコンパイル時には計算されないためです)。コンパイル時環境は\code{compile}の追加の引数に
なり、各コード生成器に渡されます。\code{lambda}のボディがコンパイルされる時、
\code{compile\-/lambda\-/body}がコンパイル時環境を手続のパラメータを持つフレームにより拡張し、
ボディを構成する命令列がその拡張された環境を用いてコンパイルされます。コンパイルの各
時点にて、\code{compile\-/variable}と\code{compile\-/assignment}は適切なレキシカルアドレスを
生成するためにコンパイル時環境を使用します。

\link{Exercise 5.39}から\link{Exercise 5.43}は
コンパイラにレキシカルな検索を組込むためにこのレキシカルアドレス付けの戦略の草案を
どのようにして完了させるかについて説明します。\link{Exercise 5.44}はコンパイル時環境の
別の使用法を説明します。

\begin{quote}
\heading{\phantomsection\label{Exercise 5.39}Exercise 5.39:}
新しい検索命令を実装する\code{lexical\-/address\-/lookup}手続を書け。2つの引数、レキシカル
アドレスと実行時環境を取ること。そして指定したレキシカルアドレスに格納された
変数の値を返すこと。\code{lexical\-/address\-/lookup}はもし変数の値がシンボル\code{*unassigned*}ならば
エラーを発する。\footnote{これはもし内部定義を削除するためにこの検索手法を実装するので
あれば必要となる、変数検索に対する変更です(\link{Exercise 5.43})。レキシカルアドレスを
うまく動かすためにはこれらの定義を排除する必要があります。}また指定したレキシカル
アドレスの変数の値を変更する操作を実装する手続\code{lexical\-/address\-/set!}を書け。
\end{quote}

\begin{quote}
\heading{\phantomsection\label{Exercise 5.40}Exercise 5.40:}
コンパイラを変更し、上で説明されたコンパイル時環境を保存するようにせよ。
つまり、\code{compile}と多様なコード生成器の引数にcompile-time-environmentを
追加し、それを\code{compile\-/lambda\-/body}の中で拡張せよ。
\end{quote}

\begin{quote}
\heading{\phantomsection\label{Exercise 5.41}Exercise 5.41:}
引数として変数とコンパイル時環境を取り、その環境に関するその変数のレキシカルアドレスを
返す手続\code{find\-/variable}を書け。例えば、上で示されたプログラムの断片において、
式\( \langle \)\var{e1}\( \kern0.08em\rangle \)をコンパイルしている間の
コンパイル時環境は\code{((y z) (a b c d e) (x y))}である。\code{find\-/variable}は以下を
生成しなければならない。

\begin{scheme}
(find-variable 'c '((y z) (a b c d e) (x y)))
~\textit{(1 2)}~
(find-variable 'x '((y z) (a b c d e) (x y)))
~\textit{(2 0)}~
(find-variable 'w '((y z) (a b c d e) (x y)))
~\textit{not-found}~
\end{scheme}
\end{quote}

\begin{quote}
\heading{\phantomsection\label{Exercise 5.42}Exercise 5.42:}
\link{Exercise 5.41}の\code{find\-/variable}を使用して、\code{compile\-/variable}と
\code{compile\-/assignment}を書き直し、レキシカルアドレス命令を出力するようにせよ。
\code{find\-/variable}が\code{not\-/found}を返す場合においては(つまり、変数がコンパイル時
環境内には存在しない場合には)、コード生成器に対して以前と同じ環境命令を使用させ
ることで束縛を検索させなければならない。(コンパイル時に変数が見つからない唯一の
場所はグローバル環境である。これは実行時環境の一部であり、コンパイル時環境の
一部ではない。\footnote{レキシカルアドレスはグローバル環境内の変数をアクセスする
ためには利用できません。なぜなら、これらの名前は対話形式的に任意の時点で定義と
再定義が可能なためです。\link{Exercise 5.43}の内部定義走査を用いてコンパイラが
知ることができる定義は、グローバル環境に従うトップレベルのものだけです。
定義のコンパイルは、定義された名前がコンパイル時環境に入れることにはなりません。}
従って、もしあなたが望むなら、それらに対し\code{env}内の全ての実行時環境を探させる代わりに、
環境の命令に、命令\code{(op get\-/global\-/environment)}により獲得できるグローバル環境を
直接探させてもかまわない)。変更したコンパイラをこの節の最初の入れ子の\code{lambda}の
組み合わせのような、いくつかの簡単な事例を用いてテストせよ。
\end{quote}

\begin{quote}
\heading{\phantomsection\label{Exercise 5.43}Exercise 5.43:}
\link{Section 4.1.6}においてブロック構造に対する内部定義は``実際の''\code{define}だと
考慮されるべきでないと主張した。そうではなく、手続のボディは通常の
\code{set!}を用いて正しい値に初期化された\code{lambda}の
変数のように、内部変数定義が導入されたかのように解釈されるべきである。
\link{Section 4.1.6}と\link{Exercise 4.16}はどのようにメタ循環インタプリタを変更して
内部定義を走査することで、これを達成するかを示した。コンパイラを変更し、
手続のボディをコンパイルする前にこれと同じ変形を実行するようにせよ。
\end{quote}

\begin{quote}
\heading{\phantomsection\label{Exercise 5.44}Exercise 5.44:}
この節ではレキシカルアドレスを生成するためのコンパイル時環境の使用に焦点を合わせた。
しかしコンパイル時環境の他の使用法も存在する。例として、\link{Exercise 5.38}では
コンパイルされたコードの効率をopen-codeなプリミティブ手続により向上させた。
私達の実装はopen-codeな手続を予約語として扱った。もしプログラムがそのような名前を
再束縛するなら、\link{Exercise 5.38}にて説明された仕組みは依然としてプリミティブとして
open-codeし、新しい束縛を無視するだろう。例えば、以下の手続について考えてみる。

\begin{scheme}
(lambda (+ * a b x y)
  (+ (* a x) (* b y)))
\end{scheme}

\noindent
これは\code{x}と\code{y}の一次結合を求める。これを引数\code{+matrix}, \code{*matrix}、それに
4つの行列(matrix)と共に呼ぶこともあるだろう。しかし、open-codeなコンパイラは
依然として\code{(+ (* a x) (* b y))}内の\code{+}と\code{*}をプリミティブな\code{+}と\code{*}として
open-codeしてしまうだろう。open-codeなコンパイラを変更し、プリミティブな手続の
名前を含む式に対して正しいコードをコンパイルするために、コンパイル時環境を参考に
するようにせよ。(このコードはプログラムがこれらの名前に対して\code{define}や\code{set!}を
行わない限り正しく動くようになる。)
\end{quote}

\subsection{コンパイル済みコードと評価機の連結}
\label{Section 5.5.7}

私達はまだコンパイルされたコードを評価機にどのようにロードするか、またはどのように
実行するかについて説明していません。ここでは明示的制御評価機が\link{Section 5.4.4}の時点にて定義された
状態であると仮定します。\link{Footnote 38}で指定された追加の命令も含みます。
Scheme式をコンパイルし、結果としてのオブジェクトコードを評価機にロードし、評価機に
グローバル環境の中で実行させ、結果を表示し、評価機のドライバループへと入る手続
\code{compile\-/and\-/go}を実装します。また評価機を変更し、逐次翻訳された式がコンパイルされた
手続を逐次翻訳されたものと同じように呼ぶことができるようにもします。
するとコンパイルされた手続を機械に入れてそれを呼び出すことができます。

\begin{scheme}
(compile-and-go
 '(define (factorial n)
    (if (= n 1)
        1
        (* (factorial (- n 1)) n))))
~\textit{;;; EC-Eval value:}~
~\textit{ok}~
~\textit{;;; EC-Eval input:}~
(factorial 5)
~\textit{;;; EC-Eval value:}~
~\textit{120}~
\end{scheme}

\noindent
評価機にコンパイルされた手続の取り扱いを可能にするには(例えば上記の\code{factorial}の呼出を評価すること)、
\code{apply\-/dispatch}(\link{Section 5.4.1})のコードを変更して、それがコンパイルされた手続を
(複合、またはプリミティブな手続から区別可能なものとして)認識し、制御を直接コンパイルされた
コードのエントリポイントへと移動させる必要があります。\footnote{もちろん、
逐次翻訳された手続と同様にコンパイルされた手続も複合(compound, 非プリミティブ)です。明示的
制御評価機で使用された用語との互換性のために、この節では``複合''を逐次翻訳された
(コンパイルされたの逆を)意味するものとして使用します。}

\begin{scheme}
apply-dispatch
  (test (op primitive-procedure?) (reg proc))
  (branch (label primitive-apply))
  (test (op compound-procedure?) (reg proc))
  (branch (label compound-apply))
  (test (op compiled-procedure?) (reg proc))
  (branch (label compiled-apply))
  (goto (label unknown-procedure-type))
compiled-apply
  (restore continue)
  (assign val (op compiled-procedure-entry) (reg proc))
  (goto (reg val))
\end{scheme}

\noindent
\code{compiled\-/apply}での\code{continue}の復元に注意して下さい。評価機は\code{apply\-/dispatch}にて
継続がスタックの一番上になるように準備されています。一方で、コンパイルされたコードの
エントリポイントは継続が\code{continue}の中にあることを期待しています。そのため、
\code{continue}はコンパイルされたコードが実行される前に復元されなければなりません。

評価機を開始した時にいくつかのコンパイルされたコードを実行することを可能にするために、
\code{branch}命令を評価機の最初に追加します。これはもし\code{flag}レジスタが設定されて
いれば、機械を新しいエントリポイントへと飛ばします。\footnote{
今や評価機は\code{branch}を
用いて開始するので、私達は常に評価機を開始する前に\code{flag}レジスタを初期化しなければ
なりません。機械を通常のREPLにて開始するためには、以下を用いることができます。

\begin{smallscheme}
(define (start-eceval)
  (set! the-global-environment (setup-environment))
  (set-register-contents! eceval 'flag false)
  (start eceval))
\end{smallscheme}
}

\begin{scheme}
(branch (label external-entry))      ~\textrm{; \code{flag}が立っていれば飛ぶ}~
read-eval-print-loop
  (perform (op initialize-stack))
  ~\( \dots \)~
\end{scheme}

\noindent
\code{external\-/entry}は機械が結果を\code{val}に入れ\code{(goto (reg continue))}で
終わる命令列の位置を持つ\code{val}と共に開始すると仮定します。
このエントリポイントで開始する場合、\code{val}で指定された位置へ飛びます。
しかし、最初に\code{continue}に実行が\code{print\-/result}に戻るように設定します。
\code{print\-/result}は\code{val}内の値を表示し、次に評価機のREPLの最初へと
飛びます。\footnote{
コンパイルされた手続はシステムが表示しようとするかもしれない
オブジェクトであるため、システムの表示命令(\link{Section 4.1.4}の)\code{user\-/print}も
変更し、コンパイルされた手続の構成部品を表示しようとしないようにします。

\begin{smallscheme}
(define (user-print object)
  (cond ((compound-procedure? object)
         (display (list 'compound-procedure
                        (procedure-parameters object)
                        (procedure-body object)
                        '<procedure-env>)))
        ((compiled-procedure? object)
         (display '<compiled-procedure>))
        (else (display object))))
\end{smallscheme}
}

\begin{scheme}
external-entry
  (perform (op initialize-stack))
  (assign env (op get-global-environment))
  (assign continue (label print-result))
  (goto (reg val))
\end{scheme}

\noindent
これで以下手続を用いて手続定義をコンパイルし、コンパイルされたコードを実行し、
手続を試行することができるようREPLを実行することができます。コンパイルされた
コードに\code{continue}内の位置に、\code{val}内の結果を持って戻って欲しいため、
式をターゲット\code{val}とリンク記述子\code{return}を用いてコンパイルします。
コンパイラにより生成されたオブジェクトコードを評価機で実行可能な命令に変形する
ために、レジスタマシンシミュレータ(\link{Section 5.2.2})の手続\code{assemble}を使用します。
次に\code{val}レジスタを命令のリストを指すように初期化し、\code{flag}を評価機が
\code{external\-/entry}へ飛ぶように設定し、評価機を開始します。

\begin{scheme}
(define (compile-and-go expression)
  (let ((instructions
         (assemble
          (statements
           (compile expression 'val 'return))
          eceval)))
    (set! the-global-environment (setup-environment))
    (set-register-contents! eceval 'val instructions)
    (set-register-contents! eceval 'flag true)
    (start eceval)))
\end{scheme}

\noindent
もし\link{Section 5.4.4}の終わりのようにスタック監視を設定したなら、
コンパイルされたコードのスタック使用量を調査できます。

\begin{scheme}
(compile-and-go
 '(define (factorial n)
    (if (= n 1)
        1
        (* (factorial (- n 1)) n))))
~\textit{(total-pushes = 0 maximum-depth = 0)}~
~\textit{;;; EC-Eval value:}~
~\textit{ok}~
~\textit{;;; EC-Eval input:}~
(factorial 5)
~\textit{(total-pushes = 31 maximum-depth = 14)}~
~\textit{;;; EC-Eval value:}~
~\textit{120}~
\end{scheme}

\noindent
この例を、\link{Section 5.4.4}の終わりで示された同じ手続の逐次翻訳された版を
用いた\code{(factorial 5)}の評価と比べてみて下さい。
逐次翻訳された版は144回のpushと最大スタック深度28を必要としました。
これは私達のコンパイル戦略に起因する最適化を説明しています。

\subsubsection*{逐次翻訳とコンパイル}

この節のプログラムを用いることで、今では逐次翻訳とコンパイルの代替的な実行戦略
を実験することができます。\footnote{コンパイラを拡張してコンパイルされたコードに
逐次翻訳された手続の呼び出しを許可することでさらにうまく行うことができます。
\link{Exercise 5.47}を参照して下さい。}インタプリタは機械をユーザプログラムのレベルへと
上げます。コンパイラはユーザプログラムを機械語のレベルへと下げます。
私達はScheme言語を(またはどんなプログラミング言語も)機械語の上に構築された体系化の
目的を同じとした仲間だと見做すことができます。インタプリタは対話的なプログラム開発と
デバッグに最適です。プログラムのステップの実行がこれらの抽象化を用いて組織化され、
そのため、プログラマにとってより理解しやすくなります。コンパイルされたコードは
より速く実行することができます。プログラムのステップの実行が機械語を利用して体系化され、
コンパイラは自由に高いレベルの抽象化を近道する最適化を作ることができます。\footnote{
実行
戦略とは独立して、もしユーザプログラムを実行した場合にエラーに遭遇した時にシステムを
殺すことや間違った値を生成するおおを許可するのではなく、エラーが発見され、その旨が
伝えられることを望むのならば、明らかなオーバヘッドを経験することになります。
例えば、配列の境界外参照は実行する前に参照の有効性をチェックすることで発見することが
できます。しかし、チェックのオーバヘッドは配列参照自体の何倍ものコストに成り得ます。
そしてプログラマはそのようなチェックが望ましいかの決定において安全性よりもスピードに
重きを置きます。良いコンパイラはそのようなチェックを行うコードを生成することが
可能であるべきです。また冗長なチェックは回避し、プログラマにコンパイルされたコード内での
エラーチェックの範囲と型を制御できるようにするべきです。

CやC++のような人気のある言語のコンパイラはほとんど何も実行コードの中にエラーチェックの
命令を挿入しません。可能な限り速く実行するためです。結果として、プログラマに対して
明示的にエラーチェックを提供させることに陥ります。残念なことに、人々は良くこのことを
軽視します。例えスピードが制約ではない重要なアプリケーションにおいてもです。
こうのような人々のプログラムは高速、かつ危険な生活へと導きます。例えば、1988年に
インターネットを麻痺させた悪名高い``Worm''(ワーム)は\acronym{UNIX}(tm) OS
(オペレーティングシステム)のfingerデーモンにおける入力バッファがオーバフローしたか
どうかのチェックミスを利用しました。(\link{Spafford 1989}参照)
}

逐次翻訳とコンパイルの代替もまた、新しいコンピュータへ言語を移植するための異なる
戦略へと導きます。新しい機械にLispを実装したいと願っていると仮定します。
1つの戦略は\link{Section 5.4}の明示的制御評価機と共に始めて、その命令を新しい機械の命令へと
翻訳することです。異なる戦略はコンパイラと共に始めてコード生成器を変更し、新しい
機械のコードを生成するようにします。2つ目の戦略はどんなLispプログラムも最初に
元のLispシステム上で動くコンパイラを用いてコンパイルし、実行時ライブラリの
コンパイル済みの版とリンクすることにより、新しい機械の上で実行させることが
可能になります。\footnote{もちろん、逐次翻訳とコンパイルの戦略のどちらを用いても、
新しい機械の記憶域割り当て、入出力(I/O)、そして評価機とコンパイラの議論において
``プリミティブ''として扱った全ての多彩な命令もまた新しい機械のために実装しなければ
なりません。ここで仕事量を最小化するための1つの方法としてはこれらの命令を可能な限り
Lispで書き、次に新しい機械のためにコンパイルすることが上げられます。究極的には、
全てが新しい機械のために手で書かれた(ガベージコレクションや実際の機械の
プリミティブを適用する仕組みの様な)小さなカーネルに縮小されます。}
もっと良いことには、コンパイラそれ自身をコンパイルすることができます。そしてこれを
新しい機械の上で他のLispプログラムをコンパイルするために実行するのです。\footnote{この
戦略は、コンパイルされたコンパイラを用いた、新しい機械上でのプログラムのコンパイルが
元のLispシステム上のプログラムのコンパイルと同一であるかどうかという、コンパイラの
正確性の楽しいテストへと至ります。違いの原因の追跡は楽しいのですが、しばしばイライラも
させます。その結果はとても小さな詳細に非常に敏感なためです。}または、\link{Section 4.1}の
インタプリタの内1つをコンパイルして新しい機械上で実行できるインタプリタを
生成することもできます。

\begin{quote}
\heading{\phantomsection\label{Exercise 5.45}Exercise 5.45:}
コンパイルされたコードにより使用されたスタック命令を同じ演算のための評価機により
使用されたスタック命令と比較することで、コンパイラのスタック使用の最適化の範囲を
速さ(スタック命令の総数の削減)と記憶域(最大スタック深度の削減)の両方において
判断することができる。この最適化されたスタックの使用を、同じ演算のための特別な目的の機械
と比較することでコンパイルの品質の何らかの指標を与えることができる。

\begin{enumerate}[a]

\item
\link{Exercise 5.27}は、評価機が上で与えられた再帰階乗手続を用いて
\( n! \)を求めるのに必要なプッシュの数と最大スタック深度を\( n \)の関数として決定するよう
求めた。\link{Exercise 5.14}は\link{Figure 5.11}で示された特別な目的の階乗マシンに
対しする同じ測定を求めた。ここでは同じ分析をコンパイルした\code{factorial}手続を用いて
実行する。

コンパイルされた版のプッシュの数と逐次翻訳された版のプッシュの数との比率を取得せよ
次に同じ事を最大スタック深度に対しても行なえ。\( n! \)を求めるために使用される命令数と
スタック深度は\( n \)の線形であるために、これらの比率は\( n \)が巨大になるにつれ
定数へと収束するはずである。これらの定数は何か? 同様に、特定目的マシンの使用量と
逐次翻訳の版の使用量との比率も求めよ。

特定目的と逐次翻訳されたコードとの間の比率と、コンパイルされたコードと逐次翻訳された
コードとの間の比率を比較せよ。特定目的マシンがコンパイルされたコードよりもとても良い
ことに気付くはずだ。手作りのコントローラのコードは基本的な汎用目的のコンパイラにより
生成されたものよりとても優れているはずだからである。

\item
パフォーマンスにおいて手作り版により近いコードを生成することを手助けする、
コンパイラに対する改善を提案できるだろうか?

\end{enumerate}
\end{quote}

\begin{quote}
\heading{\phantomsection\label{Exercise 5.46}Exercise 5.46:}
\link{Exercise 5.45}のような分析を木再帰フィボナッチ手続のコンパイルの効果を
判断するために実行せよ。

\begin{scheme}
(define (fib n)
  (if (< n 2)
      n
      (+ (fib (- n 1))
         (fib (- n 2)))))
\end{scheme}

\noindent
\link{Figure 5.12}の特定目的フィボナッチマシンを用いた場合の効果と
比較せよ。(逐次翻訳のパフォーマンスの測定のために、\link{Exercise 5.29}を参照せよ)。
フィボナッチ数では、使用された時間的リソースは\( n \)の線形にはならない。
\end{quote}

\begin{quote}
\heading{\phantomsection\label{Exercise 5.47}Exercise 5.47:}
この節では逐次翻訳されたコードがコンパイルされたコードを呼び出すことができるように
するために、明示的制御評価機をどのように変更するかを説明した。コンパイルされた手続が
プリミティブとコンパイルされた手続のみでなく、逐次翻訳された手続も同様に呼び出すこと
ができるようにするために、コンパイラをどのように変更するのか示せ。
これは\code{compile\-/procedure\-/call}を複合(逐次翻訳)の場合を取り扱うように変更する必要がある。
全ての同じ\code{target}と\code{linkage}の組み合わせを\code{compile\-/proc\-/appl}が行うように
取り扱うよう気をつけよ。実際に手続適用を行うためには、コードは評価機の\code{compound\-/apply}
エントリポイントへ飛ぶ必要がある。このラベルはオブジェクトコードの中では直接参照することが
できない。(アセンブラが全てのラベルに対し、それがアセンブルしている、そこで定義される
コードにより参照されることを要求するためである)。従って、\code{compapp}と呼ばれるレジスタを
評価機に追加し、このエントリポイントを持たせて、これを初期化する命令を追加する。

\begin{scheme}
 (assign compapp (label compound-apply))
 (branch (label external-entry)) ~\textrm{;\code{flag}が立っていれば飛ぶ}~
read-eval-print-loop ~\( \dots \)~
\end{scheme}

あなたのコードをテストするために、手続\code{g}を呼ぶ手続\code{f}を定義することから始めよ。
\code{compile\-/and\-/go}を用いて\code{f}の定義をコンパイルし、評価機を開始せよ。ここから
評価機に対し入力を行い\code{g}を定義し\code{f}の呼出を試せ。
\end{quote}

\begin{quote}
\heading{\phantomsection\label{Exercise 5.48}Exercise 5.48:}
この節で実装された\code{compile\-/and\-/go}インターフェイスは扱いにくい。コンパイラを
(評価機が開始された時に)一度しか呼ぶことができないためだ。
以下のように明示的制御評価機の中から呼び出すことができる\code{compile\-/and\-/run}を
追加することでコンパイラ-インタプリタ間のインターフェイスを増補せよ。

\begin{scheme}
~\textit{;;; EC-Eval input:}~
(compile-and-run
 '(define (factorial n)
    (if (= n 1) 1 (* (factorial (- n 1)) n))))
~\textit{;;; EC-Eval value:}~
~\textit{ok}~
~\textit{;;; EC-Eval input:}~
(factorial 5)
~\textit{;;; EC-Eval value:}~
~\textit{120}~
\end{scheme}
\end{quote}

\begin{quote}
\heading{\phantomsection\label{Exercise 5.49}Exercise 5.49:}
明示的制御評価機のREPLを用いる代わりとして、read-compile-execute-print loopを
実行するレジスタマシンを設計せよ。言い換えれば、このマシンは式を読み込み、それを
コンパイルし、その結果のコードをアセンブルして実行し、その結果を表示するループを
実行する。これは私達のシミュレートされた構成内で簡単に実行できる。なぜなら、
手続\code{compile}と\code{assemble}を``レジスタマシンの命令''として呼ぶことを
手配できるからだ。
\end{quote}

\begin{quote}
\heading{\phantomsection\label{Exercise 5.50}Exercise 5.50:}
コンパイラを用いて\link{Section 4.1}のメタ循環評価機をコンパイルし、レジスタマシン
シミュレータと用いてこのプログラムを実行せよ。(一度に複数の定義をコンパイルするために、
\code{begin}の中に定義を詰めることができる)。結果としてのインタプリタの実行は
複数レベルの逐次翻訳のため、とても遅い。しかし、実行の詳細全てを理解することは
教育的な課題である。
\end{quote}

\begin{quote}
\heading{\phantomsection\label{Exercise 5.51}Exercise 5.51:}
C言語(またはあなたが選んだ何らかの他の低レベルな言語)によるSchemeの基本的な実装を、
\link{Section 5.4}の明示的制御評価機をC言語に翻訳することで開発せよ。このコードを実行するためには、
適切なメモリ割当ルーチンと他の実行時サポートも提供する必要がある。
\end{quote}

\begin{quote}
\heading{\phantomsection\label{Exercise 5.52}Exercise 5.52:}
\link{Exercise 5.51}に対する好対照として、コンパイラを変更してSchemeの手続を
C言語の命令列へとコンパイルするようにせよ。\link{Section 4.1}のメタ循環評価機をコンパイルして
C言語で書かれたSchemeインタプリタを生成せよ。
\end{quote}


%=======================================================================================================

\backmatter

\chapter*{参考文献}
\addcontentsline{toc}{chapter}{参考文献}
\label{References}

\phantomsection \label{Abelson et al. 1992}
Abelson, Harold, Andrew Berlin, Jacob Katzenelson, William McAllister,
Guillermo Rozas, Gerald Jay Sussman, and Jack Wisdom. 1992.  The Supercomputer
Toolkit: A general framework for special-purpose computing.
\textit{International Journal of High-Speed Electronics} 3(3): 337-361.
\href{http://www.hpl.hp.com/techreports/94/HPL-94-30.html}{\code{(Onl)}}

\phantomsection \label{Allen 1978}
Allen, John.  1978.  \textit{Anatomy of Lisp}. New York: McGraw-Hill.

\phantomsection \label{ANSI 1994}
\acronym{ANSI} X3.226-1994. \textit{American National Standard for Information
Sys\-tems---Programming Language---Common Lisp}.

\phantomsection \label{Appel 1987}
Appel, Andrew W.  1987.  Garbage collection can be faster than stack
allocation.  \textit{Information Processing Letters} 25(4): 275-279.
\href{http://citeseer.ist.psu.edu/viewdoc/summary?doi=10.1.1.39.8219}{\code{(Online)}}

\phantomsection \label{Backus 1978}
Backus, John.  1978.  Can programming be liberated from the von Neumann style?
\textit{Communications of the \acronym{ACM}} 21(8): 613-641.
\href{http://www.stanford.edu/class/cs242/readings/backus.pdf}{\code{(Online)}}

\phantomsection \label{Baker (1978)}
Baker, Henry G., Jr.  1978.  List processing in real time on a serial computer.
\textit{Communications of the \acronym{ACM}} 21(4): 280-293.
\href{http://dspace.mit.edu/handle/1721.1/41976}{\code{(Online)}}

\phantomsection \label{Batali et al. 1982}
Batali, John, Neil Mayle, Howard Shrobe, Gerald Jay Sussman, and Daniel Weise.
1982.  The Scheme-81 architecture---System and chip.  In \textit{Proceedings of
the \acronym{MIT} Conference on Advanced Research in \acronym{VLSI}}, edited by
Paul Penfield, Jr. Dedham, MA: Artech House.

\phantomsection \label{Borning (1977)}
Borning, Alan.  1977.  ThingLab---An object-oriented system for building
simulations using constraints. In \textit{Proceedings of the 5th International
Joint Conference on Artificial Intelligence}.
\href{http://ijcai.org/Past\%20Proceedings/IJCAI-77-VOL1/PDF/085.pdf}{\code{(Online)}}

\phantomsection \label{Borodin and Munro (1975)}
Borodin, Alan, and Ian Munro.  1975.  \textit{The Computational Complexity of
Algebraic and Numeric Problems}. New York: American Elsevier.

\phantomsection \label{Chaitin 1975}
Chaitin, Gregory J.  1975.  Randomness and mathematical proof.
\textit{Scientific American} 232(5): 47-52.

\phantomsection \label{Church (1941)}
Church, Alonzo.  1941.  \textit{The Calculi of Lambda-Conversion}.  Princeton,
N.J.: Princeton University Press.

\phantomsection \label{Clark (1978)}
Clark, Keith L.  1978.  Negation as failure.  In \textit{Logic and Data Bases}.
New York: Plenum Press, pp. 293-322.
\href{http://www.doc.ic.ac.uk/~klc/neg.html}{\code{(Online)}}

\phantomsection \label{Clinger (1982)}
Clinger, William.  1982.  Nondeterministic call by need is neither lazy nor by
name. In \textit{Proceedings of the \acronym{ACM} Symposium on Lisp and
Functional Programming}, pp. 226-234.

\phantomsection \label{Clinger and Rees 1991}
Clinger, William, and Jonathan Rees.  1991.  Macros that work.  In
\textit{Proceedings of the 1991 \acronym{ACM} Conference on Principles of
Programming Languages}, pp. 155-162.
\href{http://mumble.net/~jar/pubs/macros_that_work.ps}{\code{(Online)}}

\phantomsection \label{Colmerauer et al. 1973}
Colmerauer A., H. Kanoui, R. Pasero, and P. Roussel.  1973.  Un syst\`eme de
communication homme-machine en fran\c{c}ais.  Technical report, Groupe
Intelligence Artificielle, Universit\'e d'Aix Marseille, Luminy.

\phantomsection \label{Cormen et al. 1990}
Cormen, Thomas, Charles Leiserson, and Ronald Rivest.  1990. \textit{Introduction
to Algorithms}. Cambridge, MA: \acronym{MIT} Press.

\phantomsection \label{Darlington et al. 1982}
Darlington, John, Peter Henderson, and David Turner.  1982.  \textit{Functional
Programming and Its Applications}. New York: Cambridge University Press.

\phantomsection \label{Dijkstra 1968a}
Dijkstra, Edsger W. 1968a.  The structure of the ``\acronym{THE}''
multiprogramming system.  \textit{Communications of the \acronym{ACM}}
11(5): 341-346.
\href{http://www.cs.utexas.edu/users/EWD/ewd01xx/EWD196.PDF}{\code{(Online)}}

\phantomsection \label{Dijkstra 1968b}
Dijkstra, Edsger W. 1968b.  Cooperating sequential processes.  In
\textit{Programming Languages}, edited by F. Genuys. New York: Academic Press,
pp.  43-112.
\href{http://www.cs.utexas.edu/users/EWD/ewd01xx/EWD123.PDF}{\code{(Online)}}

\phantomsection \label{Dinesman 1968}
Dinesman, Howard P.  1968.  \textit{Superior Mathematical Puzzles}.  New York:
Simon and Schuster.

\phantomsection \label{deKleer et al. 1977}
deKleer, Johan, Jon Doyle, Guy Steele, and Gerald J. Sussman.  1977.
\acronym{AMORD}: Explicit control of reasoning.  In \textit{Proceedings of the
\acronym{ACM} Symposium on Artificial Intelligence and Programming Languages},
pp.  116-125.
\href{http://dspace.mit.edu/handle/1721.1/5750}{\code{(Online)}}

\phantomsection \label{Doyle (1979)}
Doyle, Jon. 1979. A truth maintenance system. \textit{Artificial Intelligence}
12: 231-272.
\href{http://dspace.mit.edu/handle/1721.1/5733}{\code{(Online)}}

\phantomsection \label{Feigenbaum and Shrobe 1993}
Feigenbaum, Edward, and Howard Shrobe. 1993. The Japanese National Fifth
Generation Project: Introduction, survey, and evaluation.  In \textit{Future
Generation Computer Systems}, vol. 9, pp. 105-117.

\phantomsection \label{Feeley (1986)}
Feeley, Marc.  1986.  Deux approches \`a l'implantation du language
Scheme.  Masters thesis, Universit\'e de Montr\'eal.

\phantomsection \label{Feeley and Lapalme 1987}
Feeley, Marc and Guy Lapalme.  1987.  Using closures for code generation.
\textit{Journal of Computer Languages} 12(1): 47-66.
\href{http://citeseerx.ist.psu.edu/viewdoc/summary?doi=10.1.1.90.6978}{\code{(Online)}}

Feller, William.  1957.  \textit{An Introduction to Probability Theory and Its
Applications}, volume 1. New York: John Wiley \& Sons.

\phantomsection \label{Fenichel and Yochelson (1969)}
Fenichel, R., and J. Yochelson.  1969.  A Lisp garbage collector for virtual
memory computer systems.  \textit{Communications of the \acronym{ACM}}
12(11): 611-612.

\phantomsection \label{Floyd (1967)}
Floyd, Robert. 1967. Nondeterministic algorithms. \textit{\acronym{JACM}},
14(4): 636-644.

\phantomsection \label{Forbus and deKleer 1993}
Forbus, Kenneth D., and Johan deKleer.  1993. \textit{Building Problem
Solvers}. Cambridge, MA: \acronym{MIT} Press.

\phantomsection \label{Friedman and Wise (1976)}
Friedman, Daniel P., and David S. Wise.  1976.  \acronym{CONS} should not
evaluate its arguments. In \textit{Automata, Languages, and Programming: Third
International Colloquium}, edited by S. Michaelson and R.  Milner, pp. 257-284.
\href{https://www.cs.indiana.edu/cgi-bin/techreports/TRNNN.cgi?trnum=TR44}{\code{(Online)}}

\phantomsection \label{Friedman et al. 1992}
Friedman, Daniel P., Mitchell Wand, and Christopher T. Haynes. 1992.
\textit{Essentials of Programming Languages}.  Cambridge, MA: \acronym{MIT}
Press/ McGraw-Hill.

\phantomsection \label{Gabriel 1988}
Gabriel, Richard P. 1988.  The Why of \emph{Y}.  \textit{Lisp Pointers}
2(2): 15-25.
\href{http://www.dreamsongs.com/Files/WhyOfY.pdf}{\code{(Online)}}

Goldberg, Adele, and David Robson.  1983.  \textit{Smalltalk-80: The Language and
Its Implementation}. Reading, MA: Addison-Wesley.

\phantomsection \label{Gordon et al. 1979}
Gordon, Michael, Robin Milner, and Christopher Wadsworth.  1979.
\textit{Edinburgh LCF}. Lecture Notes in Computer Science, volume 78. New York:
Springer-Verlag.

\phantomsection \label{Gray and Reuter 1993}
Gray, Jim, and Andreas Reuter. 1993. \textit{Transaction Processing: Concepts and
Models}. San Mateo, CA: Morgan-Kaufman.

\phantomsection \label{Green 1969}
Green, Cordell.  1969.  Application of theorem proving to problem solving.  In
\textit{Proceedings of the International Joint Conference on Artificial
Intelligence}, pp. 219-240.
\href{http://citeseer.ist.psu.edu/viewdoc/summary?doi=10.1.1.81.9820}{\code{(Online)}}

\phantomsection \label{Green and Raphael (1968)}
Green, Cordell, and Bertram Raphael.  1968.  The use of theorem-proving
techniques in question-answering systems.  In \textit{Proceedings of the
\acronym{ACM} National Conference}, pp. 169-181.

\phantomsection \label{Griss 1981}
Griss, Martin L.  1981.  Portable Standard Lisp, a brief overview.  Utah
Symbolic Computation Group Operating Note 58, University of Utah.

\phantomsection \label{Guttag 1977}
Guttag, John V.  1977.  Abstract data types and the development of data
structures.  \textit{Communications of the \acronym{ACM}} 20(6): 396-404.
\href{http://www.unc.edu/~stotts/comp723/guttagADT77.pdf}{\code{(Online)}}

\phantomsection \label{Hamming 1980}
Hamming, Richard W.  1980.  \textit{Coding and Information Theory}.  Englewood
Cliffs, N.J.: Prentice-Hall.

\phantomsection \label{Hanson 1990}
Hanson, Christopher P.  1990.  Efficient stack allocation for tail-recur\-sive
languages.  In \textit{Proceedings of \acronym{ACM} Conference on Lisp and
Functional Programming}, pp. 106-118.

\phantomsection \label{Hanson 1991}
Hanson, Christopher P.  1991.  A syntactic closures macro facility.  \textit{Lisp
Pointers}, 4(3).
\href{http://groups.csail.mit.edu/mac/ftpdir/scheme-reports/synclo.ps}{\code{(Online)}}

\phantomsection \label{Hardy 1921}
Hardy, Godfrey H.  1921.  Srinivasa Ramanujan.  \textit{Proceedings of the London
Mathematical Society} XIX(2).

\phantomsection \label{Hardy and Wright 1960}
Hardy, Godfrey H., and E. M. Wright.  1960.  \textit{An Introduction to the
Theory of Numbers}.  4th edition.  New York: Oxford University Press.

\phantomsection \label{Havender (1968)}
Havender, J. 1968. Avoiding deadlocks in multi-tasking systems. \textit{IBM
Systems Journal} 7(2): 74-84.

\phantomsection \label{Hearn 1969}
Hearn, Anthony C.  1969.  Standard Lisp.  Technical report \acronym{AIM}-90,
Artificial Intelligence Project, Stanford University.
\href{http://www.softwarepreservation.org/projects/LISP/stanford/Hearn-StandardLisp-AIM-90.pdf}{\code{(Online)}}

\phantomsection \label{Henderson 1980}
Henderson, Peter. 1980.  \textit{Functional Programming: Application and
Implementation}. Englewood Cliffs, N.J.: Prentice-Hall.

\phantomsection \label{Henderson 1982}
Henderson. Peter. 1982. Functional Geometry. In \textit{Conference Record of the
1982 \acronym{ACM} Symposium on Lisp and Functional Programming}, pp. 179-187.
\href{http://pmh-systems.co.uk/phAcademic/papers/funcgeo.pdf}{\code{(Online)}}
\href{http://eprints.soton.ac.uk/257577/1/funcgeo2.pdf}{\code{(2002 version)}}

\phantomsection \label{Hewitt (1969)}
Hewitt, Carl E.  1969.  \acronym{PLANNER}: A language for proving
theorems in robots.  In \textit{Proceedings of the International Joint
Conference on Artificial Intelligence}, pp. 295-301.
\href{http://dspace.mit.edu/handle/1721.1/6171}{\code{(Online)}}

\phantomsection \label{Hewitt (1977)}
Hewitt, Carl E.  1977.  Viewing control structures as patterns of passing
messages.  \textit{Journal of Artificial Intelligence} 8(3): 323-364.
\href{http://dspace.mit.edu/handle/1721.1/6272}{\code{(Online)}}

\phantomsection \label{Hoare (1972)}
Hoare, C. A. R. 1972.  Proof of correctness of data representations.
\textit{Acta Informatica} 1(1).

\phantomsection \label{Hodges 1983}
Hodges, Andrew. 1983.  \textit{Alan Turing: The Enigma}. New York: Simon and
Schuster.

\phantomsection \label{Hofstadter 1979}
Hofstadter, Douglas R.  1979.  \textit{G\"odel, Escher, Bach: An Eternal Golden
Braid}. New York: Basic Books.

\phantomsection \label{Hughes 1990}
Hughes, R. J. M.  1990.  Why functional programming matters.  In \textit{Research
Topics in Functional Programming}, edited by David Turner.  Reading, MA:
Addison-Wesley, pp. 17-42.
\href{http://www.cs.kent.ac.uk/people/staff/dat/miranda/whyfp90.pdf}{\code{(Online)}}

\phantomsection \label{IEEE 1990}
\acronym{IEEE} Std 1178-1990.  1990.  \textit{\acronym{IEEE} Standard for the
Scheme Programming Language}.

\phantomsection \label{Ingerman et al. 1960}
Ingerman, Peter, Edgar Irons, Kirk Sattley, and Wallace Feurzeig; assisted by
M. Lind, Herbert Kanner, and Robert Floyd.  1960.  \acronym{THUNKS}: A way of
compiling procedure statements, with some comments on procedure declarations.
Unpublished manuscript.  (Also, private communication from Wallace Feurzeig.)

\phantomsection \label{Kaldewaij 1990}
Kaldewaij, Anne. 1990.  \textit{Programming: The Derivation of Algorithms}. New
York: Prentice-Hall.

\phantomsection \label{Knuth (1973)}
Knuth, Donald E.  1973.  \textit{Fundamental Algorithms}. Volume 1 of \textit{The
Art of Computer Programming}.  2nd edition. Reading, MA: Addison-Wesley.

\phantomsection \label{Knuth 1981}
Knuth, Donald E.  1981.  \textit{Seminumerical Algorithms}. Volume 2 of \textit{The
Art of Computer Programming}.  2nd edition. Reading, MA: Addison-Wesley.

\phantomsection \label{Kohlbecker 1986}
Kohlbecker, Eugene Edmund, Jr. 1986.  Syntactic extensions in the programming
language Lisp.  Ph.D. thesis, Indiana University.
\href{http://www.ccs.neu.edu/scheme/pubs/dissertation-kohlbecker.pdf}{\code{(Online)}}

\phantomsection \label{Konopasek and Jayaraman 1984}
Konopasek, Milos, and Sundaresan Jayaraman.  1984.  \textit{The TK!Solver Book: A
Guide to Problem-Solving in Science, Engineering, Business, and
Education}. Berkeley, CA: Osborne/McGraw-Hill.

\phantomsection \label{Kowalski (1973; 1979)}
Kowalski, Robert.  1973.  Predicate logic as a programming language.  Technical
report 70, Department of Computational Logic, School of Artificial
Intelligence, University of Edinburgh.
\href{http://www.doc.ic.ac.uk/~rak/papers/IFIP\%2074.pdf}{\code{(Online)}}

Kowalski, Robert.  1979.  \textit{Logic for Problem Solving}. New York:
North-Holland.

\phantomsection \label{Lamport (1978)}
Lamport, Leslie. 1978.  Time, clocks, and the ordering of events in a
distributed system.  \textit{Communications of the \acronym{ACM}} 21(7): 558-565.
\href{http://www.stanford.edu/class/cs240/readings/lamport.pdf}{\code{(Online)}}

\phantomsection \label{Lampson et al. 1981}
Lampson, Butler, J. J. Horning, R.  London, J. G. Mitchell, and G. K.  Popek.
1981.  Report on the programming language Euclid.  Technical report, Computer
Systems Research Group, University of Toronto.
\href{http://www.bitsavers.org/pdf/xerox/parc/techReports/CSL-81-12_Report_On_The_Programming_Language_Euclid.pdf}{\code{(Online)}}

\phantomsection \label{Landin (1965)}
Landin, Peter.  1965.  A correspondence between Algol 60 and Church's lambda
notation: Part I.  \textit{Communications of the \acronym{ACM}} 8(2): 89-101.

\phantomsection \label{Lieberman and Hewitt 1983}
Lieberman, Henry, and Carl E. Hewitt. 1983. A real-time garbage collector based
on the lifetimes of objects. \textit{Communications of the \acronym{ACM}}
26(6): 419-429.
\href{http://dspace.mit.edu/handle/1721.1/6335}{\code{(Online)}}

\phantomsection \label{Liskov and Zilles (1975)}
Liskov, Barbara H., and Stephen N. Zilles.  1975.  Specification techniques for
data abstractions.  \textit{\acronym{IEEE} Transactions on Software Engineering}
1(1): 7-19.
\href{http://csg.csail.mit.edu/CSGArchives/memos/Memo-117.pdf}{\code{(Online)}}

\phantomsection \label{McAllester (1978; 1980)}
McAllester, David Allen.  1978.  A three-valued truth-maintenance system.  Memo
473, \acronym{MIT} Artificial Intelligence Laboratory.
\href{http://dspace.mit.edu/handle/1721.1/6296}{\code{(Online)}}

McAllester, David Allen.  1980.  An outlook on truth maintenance.  Memo 551,
\acronym{MIT} Artificial Intelligence Laboratory.
\href{http://dspace.mit.edu/handle/1721.1/6327}{\code{(Online)}}

\phantomsection \label{McCarthy 1960}
McCarthy, John.  1960.  Recursive functions of symbolic expressions and their
computation by machine.  \textit{Communications of the \acronym{ACM}}
3(4): 184-195.
\href{http://innovation.it.uts.edu.au/projectjmc/articles/recursive.html}{\code{(Online)}}

\phantomsection \label{McCarthy 1963}
McCarthy, John.  1963.  A basis for a mathematical theory of computation.  In
\textit{Computer Programming and Formal Systems}, edited by P. Braffort and
D. Hirschberg.  North-Holland.
\href{http://innovation.it.uts.edu.au/projectjmc/articles/basis.html}{\code{(Online)}}

\phantomsection \label{McCarthy 1978}
McCarthy, John.  1978.  The history of Lisp.  In \textit{Proceedings of the
\acronym{ACM} \acronym{SIGPLAN} Conference on the History of Programming
Languages}.
\href{http://innovation.it.uts.edu.au/projectjmc/articles/lisp.html}{\code{(Online)}}

\phantomsection \label{McCarthy et al. 1965}
McCarthy, John, P. W. Abrahams, D. J. Edwards, T. P. Hart, and M. I.  Levin.
1965.  \textit{Lisp 1.5 Programmer's Manual}.  2nd edition.  Cambridge, MA:
\acronym{MIT} Press.
\href{http://www.softwarepreservation.org/projects/LISP/book/LISP\%201.5\%20Programmers\%20Manual.pdf/view}{\code{(Online)}}

\phantomsection \label{McDermott and Sussman (1972)}
McDermott, Drew, and Gerald Jay Sussman.  1972. Conniver reference manual.
Memo 259, \acronym{MIT} Artificial Intelligence Laboratory.
\href{http://dspace.mit.edu/handle/1721.1/6203}{\code{(Online)}}

\phantomsection \label{Miller 1976}
Miller, Gary L.  1976.  Riemann's Hypothesis and tests for primality.
\textit{Journal of Computer and System Sciences} 13(3): 300-317.
\href{http://www.cs.cmu.edu/~glmiller/Publications/b2hd-Mi76.html}{\code{(Online)}}

\phantomsection \label{Miller and Rozas 1994}
Miller, James S., and Guillermo J. Rozas. 1994.  Garbage collection is fast,
but a stack is faster.  Memo 1462, \acronym{MIT} Artificial Intelligence
Laboratory.
\href{http://dspace.mit.edu/handle/1721.1/6622}{\code{(Online)}}

\phantomsection \label{Moon 1978}
Moon, David.  1978.  MacLisp reference manual, Version 0.  Technical report,
\acronym{MIT} Laboratory for Computer Science.
\href{http://www.softwarepreservation.org/projects/LISP/MIT/Moon-MACLISP_Reference_Manual-Apr_08_1974.pdf/view}{\code{(Online)}}

\phantomsection \label{Moon and Weinreb 1981}
Moon, David, and Daniel Weinreb.  1981.  Lisp machine manual.  Technical
report, \acronym{MIT} Artificial Intelligence Laboratory.
\href{http://www.unlambda.com/lmman/index.html}{\code{(Online)}}

\phantomsection \label{Morris et al. 1980}
Morris, J. H., Eric Schmidt, and Philip Wadler.  1980.  Experience with an
applicative string processing language.  In \textit{Proceedings of the 7th Annual
\acronym{ACM} \acronym{SIGACT}/\acronym{SIGPLAN} Symposium on the Principles of
Programming Languages}.

\phantomsection \label{Phillips 1934}
Phillips, Hubert.  1934. \textit{The Sphinx Problem Book}.  London: Faber and
Faber.

\phantomsection \label{Pitman 1983}
Pitman, Kent. 1983. The revised MacLisp Manual (Saturday evening edition).
Technical report 295, \acronym{MIT} Laboratory for Computer Science.
\href{http://maclisp.info/pitmanual}{\code{(Online)}}

\phantomsection \label{Rabin 1980}
Rabin, Michael O. 1980. Probabilistic algorithm for testing primality.
\textit{Journal of Number Theory} 12: 128-138.

\phantomsection \label{Raymond 1993}
Raymond, Eric.  1993. \textit{The New Hacker's Dictionary}. 2nd edition.
Cambridge, MA: \acronym{MIT} Press.
\href{http://www.outpost9.com/reference/jargon/jargon_toc.html}{\code{(Online)}}

Raynal, Michel. 1986. \textit{Algorithms for Mutual Exclusion}.  Cambridge, MA:
\acronym{MIT} Press.

\phantomsection \label{Rees and Adams 1982}
Rees, Jonathan A., and Norman I. Adams IV. 1982.  T: A dialect of Lisp or,
lambda: The ultimate software tool.  In \textit{Conference Record of the 1982
\acronym{ACM} Symposium on Lisp and Functional Programming}, pp.  114-122.
\href{http://people.csail.mit.edu/riastradh/t/adams82t.pdf}{\code{(Online)}}

Rees, Jonathan, and William Clinger (eds). 1991.  The \( \rm revised^4 \) report on the
algorithmic language Scheme.  \textit{Lisp Pointers}, 4(3).
\href{http://people.csail.mit.edu/jaffer/r4rs_toc.html}{\code{(Online)}}

\phantomsection \label{Rivest et al. (1977)}
Rivest, Ronald, Adi Shamir, and Leonard Adleman.  1977.  A method for obtaining
digital signatures and public-key cryptosystems. Technical memo LCS/TM82,
\acronym{MIT} Laboratory for Computer Science.
\href{http://people.csail.mit.edu/rivest/Rsapaper.pdf}{\code{(Online)}}

\phantomsection \label{Robinson 1965}
Robinson, J. A. 1965.  A machine-oriented logic based on the resolution
principle.  \textit{Journal of the \acronym{ACM}} 12(1): 23.

\phantomsection \label{Robinson 1983}
Robinson, J. A. 1983.  Logic programming---Past, present, and future.
\textit{New Generation Computing} 1: 107-124.

\phantomsection \label{Spafford 1989}
Spafford, Eugene H.  1989.  The Internet Worm: Crisis and aftermath.
\textit{Communications of the \acronym{ACM}} 32(6): 678-688.
\href{http://citeseerx.ist.psu.edu/viewdoc/download?doi=10.1.1.123.8503&rep=rep1&type=pdf}{\code{(Online)}}

\phantomsection \label{Steele 1977}
Steele, Guy Lewis, Jr.  1977.  Debunking the ``expensive procedure call'' myth.
In \textit{Proceedings of the National Conference of the \acronym{ACM}},
pp. 153-62.
\href{http://dspace.mit.edu/handle/1721.1/5753}{\code{(Online)}}

\phantomsection \label{Steele 1982}
Steele, Guy Lewis, Jr.  1982.  An overview of Common Lisp.  In
\textit{Proceedings of the \acronym{ACM} Symposium on Lisp and Functional
Programming}, pp. 98-107.

\phantomsection \label{Steele 1990}
Steele, Guy Lewis, Jr.  1990.  \textit{Common Lisp: The Language}. 2nd edition.
Digital Press.
\href{http://www.cs.cmu.edu/Groups/AI/html/cltl/cltl2.html}{\code{(Online)}}

\phantomsection \label{Steele and Sussman 1975}
Steele, Guy Lewis, Jr., and Gerald Jay Sussman.  1975.  Scheme: An interpreter
for the extended lambda calculus.  Memo 349, \acronym{MIT} Artificial
Intelligence Laboratory.
\href{http://dspace.mit.edu/handle/1721.1/5794}{\code{(Online)}}

\phantomsection \label{Steele et al. 1983}
Steele, Guy Lewis, Jr., Donald R. Woods, Raphael A. Finkel, Mark R.  Crispin,
Richard M. Stallman, and Geoffrey S. Goodfellow.  1983.  \textit{The Hacker's
Dictionary}. New York: Harper \& Row.
\href{http://www.dourish.com/goodies/jargon.html}{\code{(Online)}}

\phantomsection \label{Stoy 1977}
Stoy, Joseph E.  1977.  \textit{Denotational Semantics}. Cambridge, MA:
\acronym{MIT} Press.

\phantomsection \label{Sussman and Stallman 1975}
Sussman, Gerald Jay, and Richard M. Stallman.  1975.  Heuristic techniques in
computer-aided circuit analysis.  \textit{\acronym{IEEE} Transactions on Circuits
and Systems} CAS-22(11): 857-865.
\href{http://dspace.mit.edu/handle/1721.1/5803}{\code{(Online)}}

\phantomsection \label{Sussman and Steele 1980}
Sussman, Gerald Jay, and Guy Lewis Steele Jr.  1980.  Constraints---A language
for expressing almost-hierachical descriptions.  \textit{AI Journal} 14: 1-39.
\href{http://dspace.mit.edu/handle/1721.1/6312}{\code{(Online)}}

\phantomsection \label{Sussman and Wisdom 1992}
Sussman, Gerald Jay, and Jack Wisdom.  1992. Chaotic evolution of the solar
system.  \textit{Science} 257: 256-262.
\href{http://groups.csail.mit.edu/mac/users/wisdom/ss-chaos.pdf}{\code{(Online)}}

\phantomsection \label{Sussman et al. (1971)}
Sussman, Gerald Jay, Terry Winograd, and Eugene Charniak.  1971.  Microplanner
reference manual.  Memo 203A, \acronym{MIT} Artificial Intelligence Laboratory.
\href{http://dspace.mit.edu/handle/1721.1/6184}{\code{(Online)}}

\phantomsection \label{Sutherland (1963)}
Sutherland, Ivan E.  1963.  \acronym{SKETCHPAD}: A man-machine graphical
communication system.  Technical report 296, \acronym{MIT} Lincoln Laboratory.
\href{http://citeseer.ist.psu.edu/viewdoc/summary?doi=10.1.1.10.4290}{\code{(Onl.)}}

\phantomsection \label{Teitelman 1974}
Teitelman, Warren.  1974.  Interlisp reference manual.  Technical report, Xerox
Palo Alto Research Center.

\phantomsection \label{Thatcher et al. 1978}
Thatcher, James W., Eric G. Wagner, and Jesse B. Wright. 1978.  Data type
specification: Parameterization and the power of specification techniques. In
\textit{Conference Record of the Tenth Annual \acronym{ACM} Symposium on Theory
of Computing}, pp. 119-132.

\phantomsection \label{Turner 1981}
Turner, David.  1981.  The future of applicative languages.  In
\textit{Proceedings of the 3rd European Conference on Informatics}, Lecture Notes
in Computer Science, volume 123. New York: Springer-Verlag, pp.  334-348.

\phantomsection \label{Wand 1980}
Wand, Mitchell.  1980.  Continuation-based program transformation strategies.
\textit{Journal of the \acronym{ACM}} 27(1): 164-180.
\href{http://citeseerx.ist.psu.edu/viewdoc/summary?doi=10.1.1.83.8567}{\code{(Online)}}

\phantomsection \label{Waters (1979)}
Waters, Richard C.  1979.  A method for analyzing loop programs.
\textit{\acronym{IEEE} Transactions on Software Engineering} 5(3): 237-247.

Winograd, Terry.  1971.  Procedures as a representation for data in a computer
program for understanding natural language.  Technical report AI TR-17,
\acronym{MIT} Artificial Intelligence Laboratory.
\href{http://dspace.mit.edu/handle/1721.1/7095}{\code{(Online)}}

\phantomsection \label{Winston 1992}
Winston, Patrick. 1992. \textit{Artificial Intelligence}.  3rd edition.  Reading,
MA: Addison-Wesley.

\phantomsection \label{Zabih et al. 1987}
Zabih, Ramin, David McAllester, and David Chapman.  1987.  Non-deterministic
Lisp with dependency-directed backtracking.  \textit{\acronym{AAAI}-87},
pp. 59-64.
\href{http://www.aaai.org/Papers/AAAI/1987/AAAI87-011.pdf}{\code{(Online)}}

\phantomsection \label{Zippel (1979)}
Zippel, Richard.  1979.  Probabilistic algorithms for sparse polynomials.
Ph.D. dissertation, Department of Electrical Engineering and Computer Science,
\acronym{MIT}.

\phantomsection \label{Zippel 1993}
Zippel, Richard.  1993.  \textit{Effective Polynomial Computation}.  Boston, MA:
Kluwer Academic Publishers.

\chapter*{課題リスト}
\addcontentsline{toc}{chapter}{課題リスト}
\label{List of Exercises}

% 
\input{exercises}

\chapter*{図一覧}
\addcontentsline{toc}{chapter}{図一覧}
\label{List of Figures}

% 
\input{figures}


\setindexprenote{\normalsize \begin{quote}
この索引内のどんな間違いも、コンピュータの手助けにより準備された
という事実により説明できるだろう。

---Donald E. Knuth, \textit{Fundamental Algorithms}\\ 
(Volume 1 of \textit{The Art of Computer Programming}) \end{quote}}

\printindex

\chapter*{奥付}
\addcontentsline{toc}{chapter}{奥付}
\label{Colophon}

表紙は1588年、Agostino Ramelliのブックホイールのメカニズムです。これは初期のハイパーテキスト ナビゲーション支援と
見ることができるのではないでしょうか。この版画のイメージは
\href{http://newgottland.com/2012/02/09/before-the-ereader-there-was-the-wheelreader/ramelli_bookwheel_1032px/}{New Gottland}. 
のJ. E. Johnsonにより提供されています。

タイプフェイスは本文はLinux Libertineで、見出しはLinux Biolinumです。両方ともPhilipp H. Pollの手によります。
タイプライターフェイスはRaph LevienによるInconsolataであり、Dimosthenis KaponisとTakashi Tanigawaにより補完された
Inconsolata LGCの形式で利用しています。

(日本語版では漢字にIPAフォントを使用させて頂いてます。)

グラフィックデザインとタイポグラフィはAndres Rabaにより行われました。TexinfoのソースはPerlスクリプトにより
LaTeXに変換され、XeLaTeXにより\acronym{PDF}にコンパイルされています。図はInkscapeを用いて描かれました。

\end{document}
