\documentclass[oneside]{book}
\usepackage{zxjatype}

\setjamainfont[BoldFont=IPAゴシック]{IPA明朝}
\setjasansfont{IPAゴシック}
\setjamonofont{IPAゴシック}
\renewcommand\indexname{索引} 
\renewcommand{\contentsname}{目次}

% New line height: 1.05 * 1.2 = 1.26
\renewcommand{\baselinestretch}{1.05}

\newfontfamily\greekfont[Mapping=TeX]{CMU Serif} % Computer Modern Unicode
\usepackage{polyglossia}
\setotherlanguage{greek}

% To be able to use '\-/' in place of '-' inside \code{}
% so that long function names containing hyphens 
% can be broken up after the hyphen:
\usepackage[shortcuts]{extdash} 

% So that file names with multiple dots don't confuse 
% graphicx package when using \includegraphics command:
\usepackage[multidot]{grffile}
\usepackage{graphicx}

\usepackage[usenames,dvipsnames,x11names]{xcolor}
\usepackage{amsmath}

% Workaround to fix mismatched left and right math delimiters. Taken from: 
% http://tex.stackexchange.com/questions/63410/parentheses-differ-xelatex-fontspec-newtxmath-libertine
\DeclareSymbolFont{parenthesis}{T1}{fxl}{m}{n}
\DeclareMathDelimiter{(}{\mathopen}{parenthesis}{"28}{largesymbols}{"00}
\DeclareMathDelimiter{)}{\mathclose}{parenthesis}{"29}{largesymbols}{"01}
\DeclareMathDelimiter{[}{\mathopen}{parenthesis}{"5B}{largesymbols}{"02} 
\DeclareMathDelimiter{]}{\mathclose}{parenthesis}{"5D}{largesymbols}{"03} 
\DeclareMathDelimiter{\lbrace}{\mathopen}{parenthesis}{"7B}{largesymbols}{"08} 
\DeclareMathDelimiter{\rbrace}{\mathclose}{parenthesis}{"7D}{largesymbols}{"09}

\usepackage{fancyvrb}
\usepackage{imakeidx}
\usepackage[totoc,font=footnotesize]{idxlayout}
\usepackage{fancyhdr}
\pagestyle{plain}
\usepackage[final]{pdfpages} % inserts pages from a pdf file

% Page geometry for 10-inch tablets:
\usepackage[papersize={148mm,197mm},
            top=21mm,
            textwidth=111mm,
            textheight=148mm,
            hcentering,
]{geometry}

\usepackage{titlesec}  % to change the appearance of section titles 
\usepackage{listings}  % for syntax highlighted code listings
\usepackage{verbatim}  % for simple verbatim and comment environments
\usepackage{enumerate} % allows customized labels in enumerations
\usepackage{hyperref}  % makes cross references and URLs clickable 
\definecolor{LinkRed}{HTML}{80171F}
\hypersetup{
  pdfauthor={Harold Abelson, Gerald Jay Sussman, Julie Sussman},
  pdftitle={Structure and Interpretation of Computer Programs, 2nd ed.},
  pdfsubject={computer science, programming, abstraction},
  colorlinks=true,
  linkcolor=LinkRed,
  urlcolor=LinkRed,
}

% Document colors 
\definecolor{SchemeLight}  {HTML} {686868}
\definecolor{SchemeSteel}  {HTML} {787878}
\definecolor{SchemeDark}   {HTML} {262626}
\definecolor{SchemeBlue}   {HTML} {4172A3}
\definecolor{SchemeGreen}  {HTML} {487818}
\definecolor{SchemeBrown}  {HTML} {A07040}
\definecolor{SchemeRed}    {HTML} {AD4D3A}
\definecolor{SchemeViolet} {HTML} {7040A0}
\definecolor{DropCapGray}  {HTML} {A8A8A8}
\definecolor{ChapterGray}  {HTML} {C8C8C8}

\usepackage{lettrine}  % adds commands that make drop capitals
\renewcommand{\LettrineFontHook}{\rmfamily\bfseries\color{DropCapGray}}
\renewcommand{\DefaultLraise}{0.00}
\renewcommand{\DefaultLoversize}{0.02}
\renewcommand{\DefaultLhang}{0.12}
\setlength{\DefaultFindent}{1pt}
\setlength{\DefaultNindent}{0em}

\lstset{%
  % Scheme syntax highlighter
    columns=fixed,
    extendedchars=true,
    upquote=true,
    showstringspaces=false,
    sensitive=false,
    mathescape=true,
    escapechar=~,
    alsodigit={>,<,/,-,=,!,?,*},
    alsoletter=',
    morestring=[b]",
    morecomment=[l];,
    % Keyword list taken form functional.py in Pygments package:
    morekeywords={lambda, define, if, else, cond, and, or, case,%
      let, let*, letrec, begin, do, delay, set!, =>, quote,%
      quasiquote, unquote, unquote-splicing, define-syntax, let-syntax,%
      letrec-syntax, syntax-rules},
    % If keywords are quoted, they must not be highlighted:
    emph={'lambda, 'define, 'if, 'else, 'cond, 'and, 'or, 'case,%
      'let, 'let*, 'letrec, 'begin, 'do, 'delay, 'set!, '=>, 'quote,%
      'quasiquote, 'unquote, 'unquote-splicing, 'define-syntax, 'let-syntax,%
      'letrec-syntax, 'syntax-rules}, 
    emphstyle=\color{SchemeDark},
    % Paint error red:
    emph={[2]error},emphstyle=[2]\color{SchemeRed},%
    % Builtins taken from functional.py:
    emph={[3]*, +, -, /, <, <=, =, >, >=, abs, acos, angle,
        append, apply, asin, assoc, assq, assv, atan,
        boolean?, caaaar, caaadr, caaar, caadar, caaddr, caadr,
        caar, cadaar, cadadr, cadar, caddar, cadddr, caddr,
        cadr, call-with-current-continuation, call-with-input-file,
        call-with-output-file, call-with-values, call/cc, car,
        cdaaar, cdaadr, cdaar, cdadar, cdaddr, cdadr, cdar,
        cddaar, cddadr, cddar, cdddar, cddddr, cdddr, cddr,
        cdr, ceiling, char->integer, char-alphabetic?, char-ci<=?,
        char-ci<?, char-ci=?, char-ci>=?, char-ci>?, char-downcase,
        char-lower-case?, char-numeric?, char-ready?, char-upcase,
        char-upper-case?, char-whitespace?, char<=?, char<?, char=?,
        char>=?, char>?, char?, close-input-port, close-output-port,
        complex?, cons, cos, current-input-port, current-output-port,
        denominator, display, dynamic-wind, eof-object?, eq?,
        equal?, eqv?, eval, even?, exact->inexact, exact?, exp,
        expt, floor, for-each, force, gcd, imag-part,
        inexact->exact, inexact?, input-port?, integer->char,
        integer?, interaction-environment, lcm, length, list,
        list->string, list->vector, list-ref, list-tail, list?,
        load, log, magnitude, make-polar, make-rectangular,
        make-string, make-vector, map, max, member, memq, memv,
        min, modulo, negative?, newline, not, null-environment,
        null?, number->string, number?, numerator, odd?,
        open-input-file, open-output-file, output-port?, pair?,
        peek-char, port?, positive?, procedure?, quotient,
        rational?, rationalize, read, read-char, real-part, real?,
        remainder, reverse, round, scheme-report-environment,
        set-car!, set-cdr!, sin, sqrt, string, string->list,
        string->number, string->symbol, string-append, string-ci<=?,
        string-ci<?, string-ci=?, string-ci>=?, string-ci>?,
        string-copy, string-fill!, string-length, string-ref,
        string-set!, string<=?, string<?, string=?, string>=?,
        string>?, string?, substring, symbol->string, symbol?,
        tan, transcript-off, transcript-on, truncate, values,
        vector, vector->list, vector-fill!, vector-length,
        vector-ref, vector-set!, vector?, with-input-from-file,
        with-output-to-file, write, write-char, zero?},
    emphstyle=[3]\color{SchemeViolet},%
    %
    basicstyle=\color{SchemeLight}\ttfamily,
    keywordstyle=\color{SchemeBlue}\bfseries,
    identifierstyle=\color{SchemeDark},
    stringstyle=\color{SchemeGreen},
    commentstyle=\color{SchemeLight}\itshape,
}
  
\newcommand{\acronym}[1]{\textsc{\MakeLowercase{#1}}}
\newcommand{\newterm}[1]{\index{#1}\emph{#1}}
\newcommand{\jnewterm}[1]{\index{#1}{\bf #1}}
\newcommand{\strong}[1]{\textbf{#1}}
\newcommand{\var}[1]{\textsl{#1}}
\newcommand{\code}[1]{\texttt{#1}}
\newcommand{\link}[1]{\hyperref[#1]{#1}}
\newcommand{\heading}[1]{{\sffamily\bfseries #1}}
\newcommand{\dark}{\color{SchemeDark}}

\newenvironment{example}%
  {\verbatim\small}%
  {\endverbatim}

\newenvironment{smallexample}%
  {\verbatim\footnotesize}%
  {\endverbatim}

\lstnewenvironment{scheme}[1][]
{\lstset{basicstyle=\ttfamily\small\color{SchemeLight},#1}}
{}

\lstnewenvironment{smallscheme}[1][]
{\lstset{basicstyle=\ttfamily\footnotesize\color{SchemeLight},#1}}
{}

\titleformat{\chapter}[display]
  {\color{SchemeDark}\normalfont\sffamily\bfseries\LARGE}
  {\filright \color{ChapterGray}\fontsize{3em}{0em}\selectfont
    \oldstylenums{\thechapter}}
  {1em}
  {\filright}
  
\titleformat{\section}
{\color{SchemeDark}\normalfont\Large\sffamily\bfseries}
{\color{SchemeSteel}\thesection}{0.8em}{}

\titleformat{\subsection}
{\color{SchemeDark}\normalfont\large\sffamily\bfseries}
{\color{SchemeSteel}\thesubsection}{0.8em}{}

\titleformat{\subsubsection}
{\color{black}\normalfont\normalsize\sffamily\bfseries}
{\color{SchemeSteel}\thesubsubsection}{0.8em}{}

\setcounter{secnumdepth}{3}
\setcounter{tocdepth}{3}

\frenchspacing
\makeindex

%====================%
%  End of preamble.  %
%====================%

\begin{document}
\pagenumbering{roman}
\VerbatimFootnotes

%             **********************************************************
%         sicp
%            Structure and Interpretation of Computer Programs, 2e
%             Unofficial Texinfo Format
%
% utfversion      2.andresraba5.2 
% utfversiondate  February 10, 2014
%
%             This file is licensed under a Creative Commons 
%             Attribution-ShareAlike 3.0 Unported License 
%             http://creativecommons.org/licenses/by-sa/3.0/
%             
%             This is a Texinfo file.  To convert it to Info hypertext
%             format, you will need the `makeinfo' program from the GNU
%             Texinfo package.  To produce a PDF, use `texi2pdf'. 
%             For more information about this file,
%             see the text under `\label{UTF' below.}
%             
%             Various versions of sicp.texi and preformatted sicp.info
%             can be found at the following Web pages:
%             
%                 http://www.neilvandyke.org/sicp-texi/
%                 http://sicpebook.wordpress.com/
%                 [add your own here]
%             
%             **********************************************************

% HISTORY:
%
% * Version 1 (April, 2001) by Lytha Ayth.
%
% * Version 2 (April 20, 2001) by Lytha Ayth.
%
% * Version 2.nwv1 (March 11, 2002) by Neil W. Van Dyke.
%   Cosmetic change to heading in Info format, and comment changes.
% 
% * Version 2.neilvandyke1 (February 10, 2003) by Neil W. Van Dyke
%   Correction to Exercise 1.39 formula, spotted by Steve VanDevender.
%   Added URL of Abelson and Sussman video lectures.
%
% * Version 2.neilvandyke2 (unreleased)
%
% * Version 2.neilvandyke3 (April 20, 2006) by Neil W. Van Dyke
%   Pedro Kr\"oger patch to add missing Lisp example.
%
% * Version 2.neilvandyke4 (January 10, 2007) by Neil W. Van Dyke
%   Brad Walker patch to add \code{@dircategory} and \code{@direntry}.
%
% * Version 2.andresraba1 (May 23, 2011) by Andres Raba.
%   Mathematics typeset in TeX, figures redrawn in vector graphics,
%   typeface changed, cross-references improved, hyperlinks added,
%   known errors and typos corrected.
%
% * Version 2.andresraba2 (November 21, 2011) by Andres Raba.
%   Minor change to the appearance of diagrams. Adjusted page layout.
%   Fixed some typos. License changed from CC BY-NC to CC BY-SA.
%
% * Version 2.andresraba3 (November 22, 2012) by Andres Raba.
%   Improved layout and pagination. Included list of figures.
%   Added punctuation to displayed math. Updated citation links.

% * (Version 2.andresraba4 is a pocket version, described in
%   sicp-pocket.texi.)

% * Version 2.andresraba5 (September 20, 2013) by Andres Raba.
%   Texinfo source is converted to LaTeX. Pages are redesigned.

% The Algorithmic Language Scheme

\frontmatter

\includepdf[scale=0.92]{coverpage.pdf}

\pagebreak

\vspace*{\fill}
\thispagestyle{empty}

\begin{small}

\noindent
{\copyright}1996 by The Massachusetts Institute of Technology

\vspace{1.26em}
\noindent
Structure and Interpretation of Computer Programs,\\
second edition

\vspace{1.26em}
\noindent
Harold Abelson and Gerald Jay Sussman\\
with Julie Sussman, foreword by Alan J. Perlis

\vspace{1.6em}
\noindent
\includegraphics[width=25mm]{fig/icons/by-nc-sa.pdf}

\vspace{0.4em}
\noindent
This work is licensed under a Creative Commons\\ 
Attribution-NonCommercial-ShareAlike 3.0 Unported License\\
(\href{http://creativecommons.org/licenses/by-nc-sa/3.0/}{\acronym{CC BY-NC-SA} 3.0}).
Based on a work at \href{http://mitpress.mit.edu/sicp/}{mitpress.mit.edu}.

\vspace{1.26em}
\noindent
The \acronym{MIT} Press\\
Cambridge, Massachusetts\\ 
London, England

\vspace{1.26em}
\noindent
McGraw-Hill Book Company\\
New York, St. Louis, San Francisco,\\ 
Montreal, Toronto

\vspace{1.26em}
\noindent
Unofficial Texinfo Format \href{http://sicpebook.wordpress.com}{2.andresraba5.2} (February 10, 2014),\\ 
based on \href{http://www.neilvandyke.org/sicp-texi/}{2.neilvandyke4} (January 10, 2007).

\vspace{1.26em}
\noindent
日本語: by \href{http://github.com/minghai/sicp-pdf/}{minghai} based on 2.andresraba5.2 (March 31, 2014).

\end{small}

\pagebreak

\tableofcontents

\small  % Added by minghai. (Japanese fonts looks too big.)

%=======================================================================================================

\chapter*{非公式Texinfoフォーマット}
\addcontentsline{toc}{chapter}{非公式Texinfoフォーマット}
\label{UTF}

これは\acronym{SICP}の第二版非公式Texinfo版です。

あなたは恐らくこれをEmacsのInfoモードの様なハイパーテキストブラウザで読んで
いることでしょう。他にも{\TeX}で組版した物を画面や印刷して読んでいるかもしれませんが
それはバカバカしい上に高くつきます。


公式に無料で公開された\acronym{HTML}-and-\acronym{GIF}版を
Lytha Aythが最初に私的に、2001年4月の長いEmacs Lovefest Weekendの間に
非公式Texinfo版(\acronym{UTF})バージョン1へと変換しました。



\acronym{UTF}は\acronym{HTML}版よりも検索がより簡単です。また寄付された古い386の様な
質素な計算機上で行う人々にとってよりアクセスが容易です。386は理論的にはLinux、Emacs、
Schemeインタプリタを同時に実行できます。しかし多くの386は恐らくNetscapeと必要なX Window
Systemを事前に芽の出かけた資金不足の若いハッカーに\newterm{thrashing}(\jnewterm{スラッシング})の
概念を教えることなしに動かすことはできないでしょう。UTFはまた圧縮無しでも1.44\acronym{MB}
のフロッピーディスケットに収まります。これはインターネットやLANへの接続環境の無いPC
にインストールする場合に役立つでしょう。



Texinfoへの変換は可能な範囲での直接的な翻字でした。{\TeX}-to-\acronym{HTML}変換の様に
ある程度の破れが含まれること無しにはできませんでした。非公式TexInfo形式においては
図が「失なわれた技術」であるアスキーアートによる下手糞な"復活"を被りました。また
多量の上付き文字と下付き文字のいくつかの変換の間に不明瞭さによる変換の
失敗が含まれてしまった可能性が大いにあります。読者への課題として残されたと予測します。
しかし、最低でも\emph{``以上''}の記号を\texttt{<u>\&gt;</u>}と符号化することで
我等の勇敢な宇宙飛行士を危険に晒すようなことはしませんでした。



もしあなたが\texttt{sicp.texi}を変更しエラーを訂正したり、アスキーアートを向上させたなら
\code{@set utfversion {utfversion}}の行を更新し、あなたの修正を反映して下さい。
例えば、もしあなたがLythaのバージョン\code{1}で開始し、あなたの名前がBobなら、改訂版は
\code{1.bob1}, \code{1.bob2}, \dots , \code{1.bob\textit{n}}です。また\code{utfversiondate}も更新
して下さい。もしあなたが自分の改訂版をWeb上で配布したいのなら文字列``sicp.texi''を
ファイルやWebページのどこかに埋め込んでおけば人々にとってWeb検索エンジンから
探すことが簡単になるでしょう。



非公式Texinfo形式は寛大にも自由の下に配布された\acronym{HTML}版の魂を引き継いで
いると信じられています。しかし、いつ誰かの法律家の大艦隊が良心に基づく小さな事に対して非常に腹を立て
何かを行わなければならなくなるかもしれません。ですのであなたのフルネームを
使ったり、あなたのアカウントやマシン名を含むInfo, \acronym{DVI}, PostScript, \acronym{PDF}形式
を配布する前に良く良く考えて下さい。

\noindent
\textit{Peath, Lytha Ayth}

\vspace{1.0em}
\noindent
\textbf{付録:}AbelsonとSussmanによる\acronym{SICP}のビデオレクチャーもご覧下さい。\\
\href{http://groups.csail.mit.edu/mac/classes/6.001/abelson-sussman-lectures/}{\acronym{MIT CSAIL}}, 
\href{http://ocw.mit.edu/courses/electrical-engineering-and-computer-science/6-001-structure-and-interpretation-of-computer-programs-spring-2005/video-lectures/}{\acronym{MIT OCW}}.

\vspace{0.5em}
\noindent 
\textbf{付録2:} 上記は2001年の元の\acronym{UTF}の紹介です。
10年後、\acronym{UTF}は一変しました。数学上の記号と式は適切に組版され、図は
ベクターグラフィックにより描かれています。元のテキスト形式とアスキーアートの図
は今でもTexinfoのソースに残っていますが、Info形式でコンパイルした場合のみ
表示されます。電子書籍リーダーとタブレットの夜明けに画面上で\acronym{PDF}を
読むことは正式に、最早バカバカしいことでは無くなりました。楽しんで下さい!

\vspace{0.5em}
\noindent
\textit{A.R, May, 2011}

%============================================================================================================
\chapter*{非公式日本語版}
\addcontentsline{toc}{chapter}{非公式日本語版}
\label{Unofficail Japanese Edition}

SICPはかつて第一版、第二版共に日本にて公式に翻訳が商業出版されていました。
第二版を出版していたピアソン桐原が2013年8月に
\href{http://slashdot.jp/story/13/08/09/0517250/}{ピアソングループから撤退し技術書の取扱を終了したため}、
日本語でSICPを読む機会は失われました。
このことがこの翻訳を行うことの契機となりました。

実際にはその後、2014年1月付近に、寛大にも第二版の訳者、和田英一先生がオンライン上にてSICPの訳書、
\href{http://sicp.iijlab.net/}{「計算機プログラムの構造と解釈」}全文を公開して下さいました。この時点でこの非公式日本語版の
価値は随分と小さくなりました。

しかし、その時、既に3章まで翻訳していたこと、そして非公式TexInfo版が2013年11月に大改訂を行い、
当初の日本語には正式に対応していないtexi2pdfから変更を行い、XeLaTeXを採用したために、
日本語でも美しい組版ができる可能性が出てきたことが、この原稿を廃棄することを押し止めました。

SICPのライセンスについてはインターネットアーカイブにて調べてみました。
2001年1月にMITがSICPを寛大にもオンラインで無料で読むことができるように公開された時にはライセンスが指定されていませんでした。

2008年4月にMITはSICPのライセンスをCC BY-NCと指定しました。その後ライセンスは2011年10月に一旦CC BY-SAに変更されます。
そして2年後の2013年9月に再びCC BY-NCへと戻されました。この事実がSICP原文のライセンスの解釈を難しくしています。
ライセンスの変更はオーナーの自由ですが、ライセンシーはコンテンツ取得時のライセンスを尊重すれば良いからです。

最初に非公式TexInfo版を作成したLytha Aythはライセンス指定の無いSICP公開をWeb文化に基づくものだと理解しました。
次にLaTeXの組版を開発したAndres RabaはCC BY-SAに基き正式な許諾の下、PDF版を作成しました。
私の翻訳はPDF版のライセンスであるCC BY-SAに従うことが求められます。しかし、現在のMITが非商業を求めて
いることを鑑みて、Raba氏に許可を頂いた上で非商業制約を追加した
\href{http://creativecommons.org/licenses/by-nc-sa/3.0/}{\acronym{CC BY-NC-SA} 3.0}
にてリリースすることにしました。

CC BY-NC、及びBY-SAは共に翻訳の許可を明記しています。従ってこの翻訳にはLythaが心配したような法的問題は
起こらないと信じています。しかし同時に、法的問題は常に一方的に起こされることがあることもまた現実です。
従って読者の皆様には常にネットワーク上のデータは(そしてプログラムも!)消えてなくなってしまうシャボン玉で
あることを忘れずに御用心願います。

TeX、LaTeX環境の日本語対応を進めて下さった全ての関係者の皆様に感謝します。
特に最新の情報を常に更新し続けて下さっている\href{http://oku.edu.mie-u.ac.jp/~okumura/texwiki/}{TeX Wiki}の奥村~晴彦氏、
W32TeXを自動でインストールし更新可能な
\href{http://www.math.sci.hokudai.ac.jp/~abenori/soft/abtexinst.html}{TeXインストーラ}作者の阿部~紀行氏、
XeLaTeX向け日本語パッケージ
\href{http://zrbabbler.sp.land.to/zxjatype.html}{``ZXjatype''}
を開発して下さった八登~崇之氏に感謝致します。

海外ではSICPの新しい形の開発が非常に盛んです。PDFはもちろん、epubやインタラクティブ版、Kindle版(mobi形式)、ClojureやJavaScriptに
よるSICP等が公開されています。この翻訳はCC BY-NC-SAですので非商業であればそのような
派生や翻案に利用することが可能です。日本でもSICPの世界が広がっていくことを期待しています。

\vspace{1em}
\noindent
※ 校正御協力者様 (順不同、敬称略)

\begin{itemize}

\item \href{https://github.com/kei-s}{Kei Shiratsuchi}

\item \href{https://github.com/kimurakoichi}{Kimura, Koichi}

\item \href{https://github.com/nna774}{のな}

\end{itemize}

%============================================================================================================


\chapter*{献辞}
\addcontentsline{toc}{chapter}{献辞}
\label{Dedication}



この本を、尊敬と賛美を込めて、コンピュータの中に住む妖精に捧げます。

\begin{quote}
``コンピュータサイエンスに関わる私達にとってコンピュータを使用することを
楽しむことはとても大事だと私は考えます。コンピュータサイエンスが始まった時、それは
とても多くの楽しみに溢れていました。
ご存知のとおり、お金を払うお客様達は時折酷く騙されました。そして暫くして私達は
彼らの不満を真面目に受け取り始めてしまいました。
私達は考え始めてしまったのです。成功裏に、障害の無い完全なコンピュータの使用法に
ついて私達に責任があるのではないかと。
私はそうは思いません。
私は、私達がコンピュータサイエンスを伸展し、新しい方向に向かわせ、
そして仲間達と共に楽しむことに責任があると考えます。
私はコンピュータサイエンスの現場が楽しむことの感覚を失わないことを望みます。
さらに、我々が伝道師になることは望みません。
自分が聖書のセールスマンだとは思わないで下さい。
世界には既にそのような人が溢れています。
あなたが他の人々が学ぶコンピュータ利用法について何を知っているでしょう。
コンピュータ利用に成功する鍵があなたの手の中にのみあるとは決っして思わないで
下さい。
 私が思うに、そして期待することは、あなたの手の中にあるものは知性です。
それはあなたが初めて計算機に出会った時よりもより多くのことを知ることができる
能力であり、それはより多くのことを生むことができるのです。''

\noindent
---Alan J. Perlis (April 1, 1922 -- February 7, 1990)
\end{quote}

%===========================================================================================================


\chapter*{前書き}
\addcontentsline{toc}{chapter}{前書き}
\label{Foreword}

\vspace{-0.6em}


教育者、将軍、栄養士、精神分析医、そして両親はプログラムします。軍隊、学生、そして
いくつかの社会はプログラムされます。大きな問題に対する解決は一連のプログラムを
利用します。それらのほとんどは途中でひょっこり表れます。これらのプログラムは手近な
問題に特化されて現れる成果に溢れています。プログラミングを独立した知的な活動として
理解するためにはあなたはコンピュータプログラミングに向かわねばなりません。
コンピュータプログラムを読み、書かねばなりません。それも数多くです。そのプログラムが何に
ついてであるか、またはどのような適用を担うのかは多くは関係ありません。
重要なことはそれらがどのように実行され、どれだけ滑らかに他のプログラムに対してより
大きなプログラムの作成のために適合するのかです。プログラマは部分の完全性と集合の
妥当性の両方を追求せねばなりません。この本では``プログラム''の使用はデジタル計算機上にて
実行されるためのLispの方言で書かれたプログラムの創造、実行、それに学習に焦点を当てて
います。Lispの使用はプログラム記述の表記法のみを制約、制限し、私達が何をプログラムするか
については影響を与えません。


この本の主題は3つの事象に焦点を当てます。人の心、コンピュータプログラムの集合、そして
コンピュータです。全てのコンピュータプログラムは人の心の中で生まれる現実の、または
精神的な過程のモデルです。これらの過程は人の経験と思考から浮かび上がり、数はとても
多く、詳細は入り組んで、いつでも部分的にしか理解されません。それらはコンピュータ
プログラムにより稀にしか永遠の充足としてモデル化されることはありません。従って、
例え私達のプログラムが注意深く手作りされた別個の記号の集合だとしても、連動する機能の
寄せ集めだとしても、それらは絶えず発展します。私達のモデルの知覚がより深まるにつれ、
増えるにつれ、一般化されるにつれ、モデルが究極的に準安定な位置に逹っするまで変更を
行い、その中には依然として私達が格闘するモデルが存在します。コンピュータプログラミングに
関連する歓喜の源はプログラムとして表現された仕組みの心の中とコンピュータ上で絶え間無く
続く発展であり、それにより生まれる知力の爆発です。もし技巧が私達の夢を解釈するならば、
コンピュータはプログラムとして現わされるそれらを実行するのです!




その力全てに対して、コンピュータは厳しい親方です。そのプログラムは正しくなければ
なりません。私達が伝えたいと望む事柄は委細全て正確に伝えられねばなりません。
全ての他の象徴的な活動と同じく、私達は議論を通してプログラムの心理を確信するように
なります。Lispそれ自身に意味論を割り当てることも可能です。(ところでこれはまた別の
モデルです)。そしてもしプログラムの機能を指定できるのなら、例えば述語論理においてなら、
論理の証明方法が容認可能な正確性の議論に使用できます。残念なことにプログラムが巨大で
複雑になるにつれ、そしてほとんど常にそうなるのですが、仕様の妥当性、一貫性、正確さそれら
自身が疑わしくなります。そのため完全に形式化された正確さの議論は巨大なプログラムには
伴いません。巨大プログラムは小さな物から成長するため正確さに確信を持てる標準的な
プログラム構造の武器庫を開発することは重要です。私達はこれをidiom(イディオム)と呼びます。
そしてそれらを組み合わせて価値が検証された構成技術を用いてより大きな構造にすることを
学びます。これらの技術はこの本の中で長々と扱われます。そしてそれらを理解することは
プログラミングと呼ばれるプロメテウスの進取性(Promethean enterprise)に参加するのに
絶対に必要なことです。他の何事でもなく、強力な構成技術を暴き熟達することは巨大で
重要なプログラムを作成する能力を加速します。反対に、巨大なプログラムを書くことはとても
苦労が多いため、私達は多大な機能や詳細を巨大プログラムに合うように減らす新しい手法
を開発することを促されています。



プログラムとは異なり、コンピュータは物理法則に従わなければなりません。もし
それらを迅速に動かしたいのならば---状態変更当たり2、3ナノ秒で---コンピュータは
電子を極小の距離で転送せねばなりません(高々\(1{1\over2}\) フィート)。
巨大な数の端子により生じる熱は空間に集中しますがこれは取り除かねばなりません。
精緻な工学の技芸が機能の多重度と端子の密度の間のバランスを取るために開発されました。
任意のイベントにおいて、ハードウェアは常に私達がプログラムを行うのに気にするよりも
よりプリミティブなレベルで動作します。私達のLispプログラムを``機械の''プログラムに
変換する処理はそれ自体が私達がプログラムする抽象モデルです。それらの学習と作成は
とても多くの見識をプログラミングの自由裁量なモデルに関連する組織的なプログラムに
対して与えます。もちろんコンピュータそれ自身もそのようにモデル化可能です。そのことを
考えてみましょう。最小の物理スイッチング要素の振舞は量子力学でモデル化され、微分
方程式により記述され、その詳細な振舞は近似値の数値演算により獲得され、それは
コンピュータプログラムにより表現され、それはコンピュータ上で実行され、それは
組み立てられ\dots !



3つの焦点を別々に判別することは戦術上の利便性の問題でしかありません。
例え良く言われるように全てが頭の中にあるとしても、この論理的分割はこれらの焦点の
間の記号的通信量の加速を引き起します。焦点の豊かさ、活力、潜在力は人間の
経験の中で人生自体の発展により増加します。最良時には焦点の間の関係は準安定に
なります。コンピュータは絶対に十分に大きく、速くはなりません。ハードウェア技術の
全ての飛躍的進歩がより大規模なプログラミング計画、新しい組織化原理、抽象モデルの
向上へと導きます。読者の全員が自身に対し繰り返し``どの終点に向かって? どの終端に向かって?''と
問わねばなりません。しかしあまり問い過ぎてもいけません。ほろ苦い哲学の便秘のために
プログラミングの楽しさを逸っしてしまいます。



私達が書くプログラムの間で、いくつか(しかし絶対に十分ではない)は厳格な数学上の
関数、例えばソートや数列の最大値を見つける、素数性判定、平方根を求める等が実行されます。
私達はそのようなプログラムをアルゴリズムと呼びます。多数の物がそれらの最適な振舞を、
特に2つの重要なパラメタである実行時間とデータストレージの必要量に関して知られています。
プログラマは良いアルゴリズムとイディオムを獲得しなければなりません。
例えいくつかのプログラムが厳格な仕様に反しても、それらのパフォーマンスに関して
見積り、常に改善に努めることはプログラマの責務です。


Lispは``生存者''であり約四半世紀の間利用されてきました。活発なプログラミング言語の中で
FortranのみがLispより長い人生を経ています。LispとFortranはどちらもアプリケーションの重要な領域の
プログラミング上の必要性に対処してきました。すなわちFortranは科学計算や工学計算に対して、
Lispは人工知能に対してです。これらの2つの領域は重要で有り続けており、そこに携わっている
プログラマ達はこれら2つの言語に専念しているため、LispとFortranは少なくとももう四半期は
活発に使われ続けることでしょう。


Lispは変化します。このテキストで使用されるScheme方言はオリジナルのLispから発展し
いくつかの重要な手法に関して異なっています。違いには変数束縛に対する静的スコーピングや
関数の値として関数の生成を許可している点等が含まれます。その意味構造においてSchemeは
初期のLispと同等にAlgol 60に近い物です。Algol 60は再び現役となることはないでしょうが、
SchemeとPascalの遺伝子に受け継がれています。
これらの2つの言語の周りに集った言語よりも、もう2つの異なる文化の流通貨幣としての
2つの言語を見つけることのほうが難しいでしょう。
Pascalはピラミッドを建築するための物です---印象的で、息を飲むような、軍隊が重い
ブロックを所定の位置に押すことで建築された静的な構造物です。Lispは有機体を構築
するための物です---印象的で、息を飲むような、小分隊が不安定で無数のより単純な有機体を
所定の位置に嵌め込むことで構築された動的な構築物です。使用された体系化の原則は
両者の場合で同じです。ただし並外れて重要な違いが1つあります。個々のLispプログラマに
委ねられた任意のエクスポート可能な機能の数はPascalの進取性の中に見つかるそれらよりも桁違いに多いのです。
Lispプログラムは機能のライブラリを膨らませます。その機能の実用性はそれらを生成した
アプリケーションを越えます。Lisp生来のデータ構造であるリストがそのような実用性の成長の
大きな原因です。簡単な構造と自然なリストの適用可能性が驚くべき程に非特異的に機能に
反映されています。Pascalでは宣言可能なデータ構造の過剰さがカジュアルな連携を抑止し、ペナルティを科す
機能の中に特殊化することを促しています。1つのデータ構造の上で操作する100の機能を持つほうが
10のデータ構造の上で操作する10の機能を持つよりも優れています。結果としてピラミッドは1000年の間
変わらぬままでいるに違いありませんが、有機体は発展できなければ滅んでしまうのです。


この違いを説明するためにはこの本の中にある教材と課題の扱いを任意の初級課程の
Pascalを用いるテキストのそれと比べてみて下さい。\acronym{MIT}だけが消費できる、
そこで見つかる血統書付きの良馬のためのものという幻想の下で苦悩しないで下さい。学生が
誰であるかとかどこで利用されるかが問題ではありません。まさに、
Lispプログラミングに対して真剣な本はどんな物であるべきかが問題です。



これはプログラミングに関するテキストであることに注意して下さい。人工知能の仕事のための
予習に使われる他の多くのLispの本とは違います。結局、ソフトウェア工学と人工知能の重大な
プログラミングの課題は研究がより大きくなるにつれシステムとして融合する傾向にあります。
このことがなぜそのようなLispへの興味が人工知能の外側で大きくなっているのかを説明します。



誰かがそのゴールから予測したように、人工知能研究は多くの明確なプログラミング上の問題を
生成しました。他のプログラミング文化ではこの相次ぐ問題は新しい言語を生みます。実際に
どんなとても大きなプログラミングタスクにおいても効果的な体系化原理はタスクモジュール内の
情報量を言語の発明を通してコントロールし、分離することです。これらの言語は
私達、人間が最も良く操作を行うシステムの境界へと辿り着くに従いプリミティブではなくなっていく
傾向にあります。結果として、そのようなシステムは何度も複製された複雑な言語処理機能を含みます。
Lispはとてもシンプルな文法と意味論を持ち、パースが初歩的なタスクとして扱えます。
従ってパースの技術はLispプログラムにおいてはほとんどルール無用の役割を演じます。そして
言語処理機の構築は巨大なLispシステムの変化と成長の程度に対しほとんど障害になりません。
最後に、全てのLispプログラマにより負われている義務と自由に対して責任を持つものこそが
このとても単純な文法と意味論です。数行のサイズを越えるLispプログラムなら自由裁量による
関数で満たすことなく書くことはできません。開発し、合わせる。合わせて、また開発する!
括弧の入れ子の中に自身の考えを記述するLispプログラマに乾杯。

\vspace{0.5em}
\noindent
Alan J. Perlis\\
New Haven, Connecticut

%=========================================================================================================


\chapter*{第二版~序文}
\addcontentsline{toc}{chapter}{第二版~序文}
\label{Preface}

\begin{quote}
ソフトウェアが他の何物にも似ていないと言うことはできるでしょうか。それが捨てられる
べき物だと。つまり、常にシャボン玉だと見なすことだと。

---Alan J. Perlis
\end{quote}

\vspace{0.7em}

\noindent
この本の中の教材は1980年から\acronym{MIT}の入門者レベルの計算機科学の科目の中心となる物です。
私達はこの教材を4年間、最初の版が出版された時点で教えてきました。そしてこの第二版が出現する
までにさらに12年が経過しました。私達の成果が広く受け入れられ、他のテキストに取り込まれている
ことを喜ばしく思っています。私達の生徒がこの本の考えとプログラムを学び新しい計算機システムと
言語の核としてそれらを組み込んでいるのを見てきました。古代のタルムードの多義語の文字認識では、
私達の生徒が開発者になってくれました。そのような能力有る学生と熟練した開発者を得たことは
とても幸運なことでした。



この版を準備するにあたって、私達自身の教育上の経験と\acronym{MIT}や他の同僚達からの
コメントにより提案された幾百もの説明を統合しました。この本の中の主なプログラミングシステムの
多くを包括的数値演算システム、インタプリタ、レジスタマシンシミュレータ、コンパイラを含めて再設計しました。
そして全てのプログラム例を、任意の\acronym{IEEE} Scheme標準(\link{IEEE 1990})に従うScheme実装が
そられのコードを実行できることを確実にするために、書き直しました。


この版はいくつかの新しいテーマを重視しています。これらの内、最も重要なものは
計算モデル内での時間を取り扱うための異なる取り組みにより演じられる中心的な役割です。
状態を伴うオブジェクト、並行プログラミング、関数型プログラミング、遅延評価、そして
非決定性プログラミングです。私達は並行性と非決定性に関わる新しい節を含め、そして
このテーマをこの本を通してまとめることを試みました。



この本の第一版は\acronym{MIT}の一学期の科目の講義概要を密接に追っていました。
第二版の全ての新しい教材により、一学期で全てをカバーすることは不可能となりました。
そのためインストラクタは選択をしなければなりません。私達自身の教育現場では、
時々論理プログラミング(\link{Section 4.4})を飛ばします。学生にはレジスタマシンの
シミュレータを使用させるのでその実装(\link{Section 5.2})はカバーしません。
そしてコンパイラ(\link{Section 5.5})は概観のみを大雑把に教えています。それでもこれは
依然として強烈な授業です。何人かのインストラクタは最初の3章から4章のみをカバーし、
他の教材を続きの授業に残したいと願うでしょう。



World-Wide-Webサイト \href{http://mitpress.mit.edu/sicp}{http://mitpress.mit.edu/sicp} はこの本のユーザへのサポートを提供します。
これにはこの本のプログラム、プログラミング課題のサンプル、補助教材、ダウンロード可能なLispの
Scheme方言の実装が含まれます。

%============================================================================================================

\chapter*{第一版~序文}
\addcontentsline{toc}{chapter}{第一版~序文}
\label{Preface 1e}

% \vspace{-0.6em}
\begin{quote}
コンピュータはヴァイオリンのような物です。初心者が最初に蓄音機、そして次に
ヴァイオリンを試すことを想像して下さい。彼は後者の音は酷いと言います。
これが人間主義者と多くの計算機科学者から聞こえてくる議論です。
計算機のプログラムは特定の目的には良い物だ、しかし柔軟性が無いと彼らは言います。
ヴァイオリンやタイプライタだって同じです。あなたがその使い方を学ぶまでは。

---Marvin Minsky, ``Why Programming Is a Good Medium for Expressing
Poorly-Understood and Sloppily-Formulated Ideas''
\end{quote}

\vspace{0.8em}

\noindent
``The Structure and Interpretation of Computer Programs''(SICP, 計算機プログラムの構造と解釈)は
マサチューセッツ工科大学(MIT)での入門者レベルの
計算機科学の科目です。\acronym{MIT}にて電気工学、または計算機工学を専攻する全ての
学生が``共通コアカリキュラム''の4つの内の1つとして履修しなければなりません。
共通コアカリキュラムは回路と線形システムについて2つの科目とデジタルシステムの
設計についての科目を含みます。私達はこの科目の開発を1978年から行なってきました。
そしてこの教材を現行様式として1980年の秋から、600名から700名の学生に毎年、教えて
きました。これらの学生の多くは少ししか、または全く事前に公式な計算機利用についての
トレーニングを受けてはいませんでした。ただし、多くは事前に計算機で少々遊んだ経験が
有り、ほんの少数は広範囲のプログラミングの経験やハードウェア設計の経験がありました。

私達のこの計算機科学の入門科目の設計は2つの主な関心事を反映しています。1つは、
コンピュータ言語はコンピュータに命令を実行させるための単なる方法等ではなく、
新しい種類の方法論に関する考えを表現するための公式なメディアであるという考えを
証明することです。従ってプログラムは人々が読むために書かれねばならず、そして
ただ偶然に機械にとって実行する物でなければなりません。2つ目は、このレベルの
科目により扱われる本質的な教材とは、特定のプログラミング言語が構築する構文ではなく、
また特定の関数を効率的に演算するための賢いアルゴリズムでもなく、増してアルゴリズムと
演算基盤の数理解析でないという信念です。そうではなく、大きなソフトウェアシステムの
知的な複雑性をコントロールするために用いる技術でなければなりません。

私達の目標は、この教科を完了した学生がプログラミングの美学とスタイルの原理に対して
必ず良い感触を得ることです。学生達が大きなシステムの複雑性をコントロールするための
主な技術の能力を得られなければなりません。学生達が50ページの長さのプログラムを、
それが模範的なスタイルで書かれているのならば、読めるようにならなければなりません。
学生達がプログラムの変更を行う時に、元の作者の魂とスタイルを維持しながら安心できな
ければなりません。

これらのスキルは決してコンピュータプログラミングに対して独自なことではありません。
私達が教え、利用する技術は全ての工学設計に対して共通な物です。私達は
適切な場合に、詳細を隠す抽象概念を構築することにより複雑性をコントロールします。
標準的な、良く理解された部品を``mix and match''(様々な物をうまく組み合わせる方法)の方法により
組み合わせることにより、システムを構築することを可能にする慣習的なインターフェイスを
確立することで、複雑性をコントロールします。私達は設計を記述するための新しい言語を
確立することで複雑性をコントロールします。そして各言語は設計の特定の側面を重要視し、
他の側面の重要性を緩和します。

私達のこの教科に対する取り組み方の根底を成す物は、``計算機科学''は科学ではなく、
その意義は計算機とは関係が無いという信念です。計算機革命とは私達の考え方と
私達の考えの表現方法における革命です。この変化の本質を恐らく最もうまく言い表わす
のは\newterm{procedural epistemology}(\jnewterm{手続的認識論})---古典的な数学上の主題により
取られるより宣言的な視点に対立する、命令型の視点からの知識構造の研究---の出現でしょう。
数学は``何であるか''の概念を正確に扱うためのフレームワークを提供します。計算機の使用は
``行い方''の概念を正確に扱うためのフレームワークを提供します。

私達の教材を教えるにあたって、プログラミング言語Lispの一方言を使用します。
私達は正式にこの言語を教えることはしません。する必要がないからです。
ただそれを使用し、そして学生は2、3日で習熟してしまいます。これはLispの様な
言語の1つの利点です。これらの言語は複合式を形成する方法があまり多くありません。
そしてほとんど構文構造が存在しません。形式的な特性の全ては一時間もあれば
カバーできます。まるでチェスのルールの様なものです。少しの時間の後にはこの
言語の構文上の詳細を忘れてしまいます。(ほとんど存在しないからです)。そして
本当の問題---私達が演算したい物を把握すること、どのように問題を扱いやすい
部分へと分解するか、そしてどのようにその部品上で働くかについて取り掛かります。
Lispのもう1つの利点は私達が知っている他のどの言語よりもプログラムを分解した
モジュラに対するより多くの大規模な戦略をサポートする(しかし強制はしない)ことです。
手続化とデータ抽象化を行い、公開関数を用いて処理の共通なパターンを獲得し、代入と
データの変更を用いて局所状態のモデル化を行い、プログラムの部品をストリームと遅延評価に
結び付け、簡単に組込言語を実装することができます。これら全てがインタラクティブ(相互作用)な
環境にインクリメンタル(漸増的な)プログラム設計、構築、テスト、デバッグのための
優れたサポートと共に組込まれています。私達は前例の無い力と洗練さを供えた素晴しいツールを
創り出したJohn McCarthyを始めとする全ての世代のLisp wizard(ウィザード、魔法使い、
最上級のプログラマの賞賛を込めた呼び名)に感謝します。

私達が用いるLispの方言、SchemeはLispとAlgolの力と洗練を一緒にもたらそうとしました。
Lispからは単純な構文から導き出されるメタ言語の力、データオブジェクトとしてのプログラムの
単一の表現、ガベージコレクションを持つヒープ上に取得されるデータを得ました。
AlgolからはAlgol委員会に在籍したプログラム設計の開拓者からの贈り物である
レキシカルスコープとブロック構造を得ました。私達はJohn ReynoldsとPeter Landinの
Church(チャーチ)の\(lambda\)-calculus(ラムダ計算)のプログラミング言語の構造に
対する関係についての彼等の洞察に対して言及したいと願います。
またコンピュータがこの世界に現れる何十年も前にこの領域を偵察された数学者達に対する
恩義も忘れておりません。これらの開拓者にはAlonzo Church, Barkley Rosser, Stephen Kleene,
Haskell Curry等が含まれております。

%============================================================================================================

\chapter*{謝辞}
\addcontentsline{toc}{chapter}{謝辞}
\label{Acknowledgements}


この本とこのカリキュラムの開発を手助けして下さった多くの人々に感謝致します。



私達の教科は明らかに1960年代の終わりに\acronym{MIT}にてJack Wozencraftと
Arthur Evans, Jr.により教えられたプログラミング言語学と\( \lambda \)計算上の
素晴しい科目、``6.231''の知的末裔です。



私達はRobert Fanoに大きな借りがあります。彼は\acronym{MIT}の電気工学と計算機科学の
導入部のカリキュラムを再編成し、工学設計の原理を重視しました。彼はこの進取性への
着手に導き、またこの本への発展の元となる最初の教科ノートのまとめを記述しました。



私達が教えようとするプログラミングのスタイルと美学の多くは
Guy Lewis Steele Jr.の協力の下に開発されました。彼は初期のSchemeの開発において
Gerald Jay Sussmanと協力を行いました。加えてDavid Turner, Peter Henderson, 
Dan Friedman, David Wise, Will Clingerが私達にこの本の中に現れる関数型プログラミングの
テクニックの多くを教えてくれました。



Joel Mosesは私達に巨大システムの構造化について教えてくれました。彼の記号演算のための
Macsymaシステムにおける経験が、人は制御の複雑性を回避し、データの体系化に集中して
モデル化されていく世界の真の構造を反映するべきだという見識を与えてくれました。


Marvin MinskyとSeymour Papertは私達のプログラミングに関する態度の多くと、
私達の知的な生活内にそれの場所を形作りました。彼等に対して、考えを探求するための
式の意味を演算が与えることについての理解に借りがあります。そうでなければ、
正確に取り扱うためには複雑過ぎることになってしまいます。彼らは学生のプログラムを
書き、変更する能力が、その中で探求が自然な活動になる強力なメディアを提供すると
強調します。



私達はまたプログラミングは大いに楽しく、このプログラミングの楽しみをサポートする
ために十分に注意しなければならない点についてAlan Perlisに強く同意します。
この楽しみの一部は作業中の偉大な職人達を観察することから得られます。
私達は幸運なことに、Bill GosperとRichard Greenblattの下で見習いプログラマで
いることができました。



私達のカリキュラムの開発に貢献して下さった全ての人々を特定することは難しいことです。
私達は過去15年私達と共に働き、多くの時間を私達の教科に費してくれた全ての講師、
口答の指導者、チューターに、
特に、Bill Siebert, Albert Meyer,
Joe Stoy, Randy Davis, Louis Braida, Eric Grimson, Rod Brooks, Lynn Stein and
Peter Szolovitsに感謝します。
私達は特に卓越した教育上の貢献として現在はウェルズリーのFranklyn Turbakに感謝します。
彼の学部生向け指導要項は私達皆が目指す基準を打ち立てました。
Jerry SaltzerとJim Millerには私達が並行性のミステリーに取り組むのを手助けして下さった
ことに感謝します。そしてPeter SzolovitsとDavid McAllesterには\link{Chapter
4}における非決定性評価の説明に対する貢献に感謝します。



多くの人々は他大学でこの資料を紹介するのに大きな努力を費してくださいました。
私達が親密に働いたそれらの人々の幾人かはイスラエル工科大学のJacob Katzenelson、
カリフォルニア大学アーバイン校のHardy Mayer、オックスフォード大学のJoe Stoy、
パデュー大学のElisha Sacks、ノルウェー技術科学大学のJan Komorowskiです。
私達は他大学においてこの科目を受け入れることで主要な教育の賞を受けた同僚達を
非常に誇りに思います。この中にはイェール大学のKenneth Yip、カリフォルニア大学
バークリー校のBrian Harvey、コーネル大学のDan Huttenlocherを含みます。


Al Moy\'eは私たちのためにこの教材をHPの技術者達に教える手筈とこのレクチャーのビデオ
テープの製品化を準備してくれました。私たちはまた才能あるインストラクター達にも
感謝致します。具体的にはJim Miller, Bill Siebert, Mike Eisenbergです。彼等は
これらのテープを組み込んで生涯教育のコースを設計し、世界中の大学と業界にて
教育を行いました。



他国の多くの教育者が多大な時間を第一版の翻訳に費して下さいました。
Michel Briand, Pierre Chamard, and Andr\'e Picはフランス語版をプロデュースして下さいました。
Susanne Daniels-Heroldはドイツ語版をプロデュースして下さいました。
元吉文男は日本語版をプロデュースして下さいました。私たちはどなたが中国語版を
プロデュースして下さったのか知りません。しかし``未許可''の翻訳の題材として
選ばれたことを光栄に思います。



私たちが教育の目的のために使用するSchemeシステムの開発に技術的な貢献をされた全ての
人々を列挙することは難しいことです。Guy Steeleに加えて、主要なウィザードの中にはChris Hanson, Joe
Bowbeer, Jim Miller, Guillermo Rozas, Stephen Adamsが含まれます。
多大な時間を費して下さった他の人々はRichard Stallman, Alan Bawden, Kent Pitman, Jon Taft,
Neil Mayle, John Lamping, Gwyn Osnos, Tracy Larrabee, George Carrette, Soma
Chaudhuri, Bill Chiarchiaro, Steven Kirsch, Leigh Klotz, Wayne Noss, Todd Cass,
Patrick O'Donnell, Kevin Theobald, Daniel Weise, Kenneth Sinclair, Anthony
Courtemanche, Henry M. Wu, Andrew Berlin, それにRuth Shyuです。


\acronym{MIT}の実装を越えて、私たちは\acronym{IEEE}のScheme標準仕様について
働いた多くの人々に感謝したいと思います。\( \rm R^4RS \)を編集したWilliam Clingerと
Jonathan Rees、\acronym{IEEE}標準を準備したChris Haynes, David Bartley,
Chris Hanson, Jim Millerを含みます。


Dan Friedmanは長い間Schemeコミュニティのリーダーでした。コミュニティの広範な仕事は
言語設計の問題を越えて、Schemer's Inc.によるEdSchemeを基にした高校生向けカリキュラムや
Mike EisenbergやBrian HarveyとMatthew Wrightによる素晴しい本のような、特筆すべき
教育上のイノベーションを含むまでに至りました。



私たちはこの本を現実にすることに貢献して下さった人々の働きに感謝致します。
特に\acronym{MIT}出版のTerry Ehling, Larry Cohen, Paul Bethgeです。
Ella Mazelは素晴しいカバーの絵を見つけてくれました。第二版に対しては特にこの本の
デザインを助けてくれたBernardとEllaのMazel夫妻、非凡な{\TeX}ウィザードである
David Jonesに感謝致します。私たちはまた新しいドラフトに対し洞察力のあるコメントを
して下さった読者の方々、Jacob Katzenelson, Hardy Mayer, Jim Miller, そして特に
Brian Harveyに対して、Julieが彼の本\textit{Simply Scheme}に行ったように、この本に
行ってくれたことを感謝致します。



最後に、何年にも渡ったこの仕事を励まして下さった組織のサポートに感謝したいと思います。
Hewlett-Packardからのサポートを可能にして下さったIra GoldsteinとJoel Birnbaum、
それに\acronym{DARPA}からのサポートを可能にして下さったBob Kahnを含みます。

%============================================================================================================
%\pagenumbering{arabic}

\mainmatter

\input{jsicp/chapter1}

%=======================================================================================================
%=======================================================================================================
%=======================================================================================================

\input{jsicp/chapter2}

%=======================================================================================================
%=======================================================================================================
%=======================================================================================================

\input{jsicp/chapter3}

\input{jsicp/chapter4}

\input{jsicp/chapter5}

%=======================================================================================================

\backmatter

\chapter*{参考文献}
\addcontentsline{toc}{chapter}{参考文献}
\label{References}

\phantomsection \label{Abelson et al. 1992}
Abelson, Harold, Andrew Berlin, Jacob Katzenelson, William McAllister,
Guillermo Rozas, Gerald Jay Sussman, and Jack Wisdom. 1992.  The Supercomputer
Toolkit: A general framework for special-purpose computing.
\textit{International Journal of High-Speed Electronics} 3(3): 337-361.
\href{http://www.hpl.hp.com/techreports/94/HPL-94-30.html}{\code{(Onl)}}

\phantomsection \label{Allen 1978}
Allen, John.  1978.  \textit{Anatomy of Lisp}. New York: McGraw-Hill.

\phantomsection \label{ANSI 1994}
\acronym{ANSI} X3.226-1994. \textit{American National Standard for Information
Sys\-tems---Programming Language---Common Lisp}.

\phantomsection \label{Appel 1987}
Appel, Andrew W.  1987.  Garbage collection can be faster than stack
allocation.  \textit{Information Processing Letters} 25(4): 275-279.
\href{http://citeseer.ist.psu.edu/viewdoc/summary?doi=10.1.1.39.8219}{\code{(Online)}}

\phantomsection \label{Backus 1978}
Backus, John.  1978.  Can programming be liberated from the von Neumann style?
\textit{Communications of the \acronym{ACM}} 21(8): 613-641.
\href{http://www.stanford.edu/class/cs242/readings/backus.pdf}{\code{(Online)}}

\phantomsection \label{Baker (1978)}
Baker, Henry G., Jr.  1978.  List processing in real time on a serial computer.
\textit{Communications of the \acronym{ACM}} 21(4): 280-293.
\href{http://dspace.mit.edu/handle/1721.1/41976}{\code{(Online)}}

\phantomsection \label{Batali et al. 1982}
Batali, John, Neil Mayle, Howard Shrobe, Gerald Jay Sussman, and Daniel Weise.
1982.  The Scheme-81 architecture---System and chip.  In \textit{Proceedings of
the \acronym{MIT} Conference on Advanced Research in \acronym{VLSI}}, edited by
Paul Penfield, Jr. Dedham, MA: Artech House.

\phantomsection \label{Borning (1977)}
Borning, Alan.  1977.  ThingLab---An object-oriented system for building
simulations using constraints. In \textit{Proceedings of the 5th International
Joint Conference on Artificial Intelligence}.
\href{http://ijcai.org/Past\%20Proceedings/IJCAI-77-VOL1/PDF/085.pdf}{\code{(Online)}}

\phantomsection \label{Borodin and Munro (1975)}
Borodin, Alan, and Ian Munro.  1975.  \textit{The Computational Complexity of
Algebraic and Numeric Problems}. New York: American Elsevier.

\phantomsection \label{Chaitin 1975}
Chaitin, Gregory J.  1975.  Randomness and mathematical proof.
\textit{Scientific American} 232(5): 47-52.

\phantomsection \label{Church (1941)}
Church, Alonzo.  1941.  \textit{The Calculi of Lambda-Conversion}.  Princeton,
N.J.: Princeton University Press.

\phantomsection \label{Clark (1978)}
Clark, Keith L.  1978.  Negation as failure.  In \textit{Logic and Data Bases}.
New York: Plenum Press, pp. 293-322.
\href{http://www.doc.ic.ac.uk/~klc/neg.html}{\code{(Online)}}

\phantomsection \label{Clinger (1982)}
Clinger, William.  1982.  Nondeterministic call by need is neither lazy nor by
name. In \textit{Proceedings of the \acronym{ACM} Symposium on Lisp and
Functional Programming}, pp. 226-234.

\phantomsection \label{Clinger and Rees 1991}
Clinger, William, and Jonathan Rees.  1991.  Macros that work.  In
\textit{Proceedings of the 1991 \acronym{ACM} Conference on Principles of
Programming Languages}, pp. 155-162.
\href{http://mumble.net/~jar/pubs/macros_that_work.ps}{\code{(Online)}}

\phantomsection \label{Colmerauer et al. 1973}
Colmerauer A., H. Kanoui, R. Pasero, and P. Roussel.  1973.  Un syst\`eme de
communication homme-machine en fran\c{c}ais.  Technical report, Groupe
Intelligence Artificielle, Universit\'e d'Aix Marseille, Luminy.

\phantomsection \label{Cormen et al. 1990}
Cormen, Thomas, Charles Leiserson, and Ronald Rivest.  1990. \textit{Introduction
to Algorithms}. Cambridge, MA: \acronym{MIT} Press.

\phantomsection \label{Darlington et al. 1982}
Darlington, John, Peter Henderson, and David Turner.  1982.  \textit{Functional
Programming and Its Applications}. New York: Cambridge University Press.

\phantomsection \label{Dijkstra 1968a}
Dijkstra, Edsger W. 1968a.  The structure of the ``\acronym{THE}''
multiprogramming system.  \textit{Communications of the \acronym{ACM}}
11(5): 341-346.
\href{http://www.cs.utexas.edu/users/EWD/ewd01xx/EWD196.PDF}{\code{(Online)}}

\phantomsection \label{Dijkstra 1968b}
Dijkstra, Edsger W. 1968b.  Cooperating sequential processes.  In
\textit{Programming Languages}, edited by F. Genuys. New York: Academic Press,
pp.  43-112.
\href{http://www.cs.utexas.edu/users/EWD/ewd01xx/EWD123.PDF}{\code{(Online)}}

\phantomsection \label{Dinesman 1968}
Dinesman, Howard P.  1968.  \textit{Superior Mathematical Puzzles}.  New York:
Simon and Schuster.

\phantomsection \label{deKleer et al. 1977}
deKleer, Johan, Jon Doyle, Guy Steele, and Gerald J. Sussman.  1977.
\acronym{AMORD}: Explicit control of reasoning.  In \textit{Proceedings of the
\acronym{ACM} Symposium on Artificial Intelligence and Programming Languages},
pp.  116-125.
\href{http://dspace.mit.edu/handle/1721.1/5750}{\code{(Online)}}

\phantomsection \label{Doyle (1979)}
Doyle, Jon. 1979. A truth maintenance system. \textit{Artificial Intelligence}
12: 231-272.
\href{http://dspace.mit.edu/handle/1721.1/5733}{\code{(Online)}}

\phantomsection \label{Feigenbaum and Shrobe 1993}
Feigenbaum, Edward, and Howard Shrobe. 1993. The Japanese National Fifth
Generation Project: Introduction, survey, and evaluation.  In \textit{Future
Generation Computer Systems}, vol. 9, pp. 105-117.

\phantomsection \label{Feeley (1986)}
Feeley, Marc.  1986.  Deux approches \`a l'implantation du language
Scheme.  Masters thesis, Universit\'e de Montr\'eal.

\phantomsection \label{Feeley and Lapalme 1987}
Feeley, Marc and Guy Lapalme.  1987.  Using closures for code generation.
\textit{Journal of Computer Languages} 12(1): 47-66.
\href{http://citeseerx.ist.psu.edu/viewdoc/summary?doi=10.1.1.90.6978}{\code{(Online)}}

Feller, William.  1957.  \textit{An Introduction to Probability Theory and Its
Applications}, volume 1. New York: John Wiley \& Sons.

\phantomsection \label{Fenichel and Yochelson (1969)}
Fenichel, R., and J. Yochelson.  1969.  A Lisp garbage collector for virtual
memory computer systems.  \textit{Communications of the \acronym{ACM}}
12(11): 611-612.

\phantomsection \label{Floyd (1967)}
Floyd, Robert. 1967. Nondeterministic algorithms. \textit{\acronym{JACM}},
14(4): 636-644.

\phantomsection \label{Forbus and deKleer 1993}
Forbus, Kenneth D., and Johan deKleer.  1993. \textit{Building Problem
Solvers}. Cambridge, MA: \acronym{MIT} Press.

\phantomsection \label{Friedman and Wise (1976)}
Friedman, Daniel P., and David S. Wise.  1976.  \acronym{CONS} should not
evaluate its arguments. In \textit{Automata, Languages, and Programming: Third
International Colloquium}, edited by S. Michaelson and R.  Milner, pp. 257-284.
\href{https://www.cs.indiana.edu/cgi-bin/techreports/TRNNN.cgi?trnum=TR44}{\code{(Online)}}

\phantomsection \label{Friedman et al. 1992}
Friedman, Daniel P., Mitchell Wand, and Christopher T. Haynes. 1992.
\textit{Essentials of Programming Languages}.  Cambridge, MA: \acronym{MIT}
Press/ McGraw-Hill.

\phantomsection \label{Gabriel 1988}
Gabriel, Richard P. 1988.  The Why of \emph{Y}.  \textit{Lisp Pointers}
2(2): 15-25.
\href{http://www.dreamsongs.com/Files/WhyOfY.pdf}{\code{(Online)}}

Goldberg, Adele, and David Robson.  1983.  \textit{Smalltalk-80: The Language and
Its Implementation}. Reading, MA: Addison-Wesley.

\phantomsection \label{Gordon et al. 1979}
Gordon, Michael, Robin Milner, and Christopher Wadsworth.  1979.
\textit{Edinburgh LCF}. Lecture Notes in Computer Science, volume 78. New York:
Springer-Verlag.

\phantomsection \label{Gray and Reuter 1993}
Gray, Jim, and Andreas Reuter. 1993. \textit{Transaction Processing: Concepts and
Models}. San Mateo, CA: Morgan-Kaufman.

\phantomsection \label{Green 1969}
Green, Cordell.  1969.  Application of theorem proving to problem solving.  In
\textit{Proceedings of the International Joint Conference on Artificial
Intelligence}, pp. 219-240.
\href{http://citeseer.ist.psu.edu/viewdoc/summary?doi=10.1.1.81.9820}{\code{(Online)}}

\phantomsection \label{Green and Raphael (1968)}
Green, Cordell, and Bertram Raphael.  1968.  The use of theorem-proving
techniques in question-answering systems.  In \textit{Proceedings of the
\acronym{ACM} National Conference}, pp. 169-181.

\phantomsection \label{Griss 1981}
Griss, Martin L.  1981.  Portable Standard Lisp, a brief overview.  Utah
Symbolic Computation Group Operating Note 58, University of Utah.

\phantomsection \label{Guttag 1977}
Guttag, John V.  1977.  Abstract data types and the development of data
structures.  \textit{Communications of the \acronym{ACM}} 20(6): 396-404.
\href{http://www.unc.edu/~stotts/comp723/guttagADT77.pdf}{\code{(Online)}}

\phantomsection \label{Hamming 1980}
Hamming, Richard W.  1980.  \textit{Coding and Information Theory}.  Englewood
Cliffs, N.J.: Prentice-Hall.

\phantomsection \label{Hanson 1990}
Hanson, Christopher P.  1990.  Efficient stack allocation for tail-recur\-sive
languages.  In \textit{Proceedings of \acronym{ACM} Conference on Lisp and
Functional Programming}, pp. 106-118.

\phantomsection \label{Hanson 1991}
Hanson, Christopher P.  1991.  A syntactic closures macro facility.  \textit{Lisp
Pointers}, 4(3).
\href{http://groups.csail.mit.edu/mac/ftpdir/scheme-reports/synclo.ps}{\code{(Online)}}

\phantomsection \label{Hardy 1921}
Hardy, Godfrey H.  1921.  Srinivasa Ramanujan.  \textit{Proceedings of the London
Mathematical Society} XIX(2).

\phantomsection \label{Hardy and Wright 1960}
Hardy, Godfrey H., and E. M. Wright.  1960.  \textit{An Introduction to the
Theory of Numbers}.  4th edition.  New York: Oxford University Press.

\phantomsection \label{Havender (1968)}
Havender, J. 1968. Avoiding deadlocks in multi-tasking systems. \textit{IBM
Systems Journal} 7(2): 74-84.

\phantomsection \label{Hearn 1969}
Hearn, Anthony C.  1969.  Standard Lisp.  Technical report \acronym{AIM}-90,
Artificial Intelligence Project, Stanford University.
\href{http://www.softwarepreservation.org/projects/LISP/stanford/Hearn-StandardLisp-AIM-90.pdf}{\code{(Online)}}

\phantomsection \label{Henderson 1980}
Henderson, Peter. 1980.  \textit{Functional Programming: Application and
Implementation}. Englewood Cliffs, N.J.: Prentice-Hall.

\phantomsection \label{Henderson 1982}
Henderson. Peter. 1982. Functional Geometry. In \textit{Conference Record of the
1982 \acronym{ACM} Symposium on Lisp and Functional Programming}, pp. 179-187.
\href{http://pmh-systems.co.uk/phAcademic/papers/funcgeo.pdf}{\code{(Online)}}
\href{http://eprints.soton.ac.uk/257577/1/funcgeo2.pdf}{\code{(2002 version)}}

\phantomsection \label{Hewitt (1969)}
Hewitt, Carl E.  1969.  \acronym{PLANNER}: A language for proving
theorems in robots.  In \textit{Proceedings of the International Joint
Conference on Artificial Intelligence}, pp. 295-301.
\href{http://dspace.mit.edu/handle/1721.1/6171}{\code{(Online)}}

\phantomsection \label{Hewitt (1977)}
Hewitt, Carl E.  1977.  Viewing control structures as patterns of passing
messages.  \textit{Journal of Artificial Intelligence} 8(3): 323-364.
\href{http://dspace.mit.edu/handle/1721.1/6272}{\code{(Online)}}

\phantomsection \label{Hoare (1972)}
Hoare, C. A. R. 1972.  Proof of correctness of data representations.
\textit{Acta Informatica} 1(1).

\phantomsection \label{Hodges 1983}
Hodges, Andrew. 1983.  \textit{Alan Turing: The Enigma}. New York: Simon and
Schuster.

\phantomsection \label{Hofstadter 1979}
Hofstadter, Douglas R.  1979.  \textit{G\"odel, Escher, Bach: An Eternal Golden
Braid}. New York: Basic Books.

\phantomsection \label{Hughes 1990}
Hughes, R. J. M.  1990.  Why functional programming matters.  In \textit{Research
Topics in Functional Programming}, edited by David Turner.  Reading, MA:
Addison-Wesley, pp. 17-42.
\href{http://www.cs.kent.ac.uk/people/staff/dat/miranda/whyfp90.pdf}{\code{(Online)}}

\phantomsection \label{IEEE 1990}
\acronym{IEEE} Std 1178-1990.  1990.  \textit{\acronym{IEEE} Standard for the
Scheme Programming Language}.

\phantomsection \label{Ingerman et al. 1960}
Ingerman, Peter, Edgar Irons, Kirk Sattley, and Wallace Feurzeig; assisted by
M. Lind, Herbert Kanner, and Robert Floyd.  1960.  \acronym{THUNKS}: A way of
compiling procedure statements, with some comments on procedure declarations.
Unpublished manuscript.  (Also, private communication from Wallace Feurzeig.)

\phantomsection \label{Kaldewaij 1990}
Kaldewaij, Anne. 1990.  \textit{Programming: The Derivation of Algorithms}. New
York: Prentice-Hall.

\phantomsection \label{Knuth (1973)}
Knuth, Donald E.  1973.  \textit{Fundamental Algorithms}. Volume 1 of \textit{The
Art of Computer Programming}.  2nd edition. Reading, MA: Addison-Wesley.

\phantomsection \label{Knuth 1981}
Knuth, Donald E.  1981.  \textit{Seminumerical Algorithms}. Volume 2 of \textit{The
Art of Computer Programming}.  2nd edition. Reading, MA: Addison-Wesley.

\phantomsection \label{Kohlbecker 1986}
Kohlbecker, Eugene Edmund, Jr. 1986.  Syntactic extensions in the programming
language Lisp.  Ph.D. thesis, Indiana University.
\href{http://www.ccs.neu.edu/scheme/pubs/dissertation-kohlbecker.pdf}{\code{(Online)}}

\phantomsection \label{Konopasek and Jayaraman 1984}
Konopasek, Milos, and Sundaresan Jayaraman.  1984.  \textit{The TK!Solver Book: A
Guide to Problem-Solving in Science, Engineering, Business, and
Education}. Berkeley, CA: Osborne/McGraw-Hill.

\phantomsection \label{Kowalski (1973; 1979)}
Kowalski, Robert.  1973.  Predicate logic as a programming language.  Technical
report 70, Department of Computational Logic, School of Artificial
Intelligence, University of Edinburgh.
\href{http://www.doc.ic.ac.uk/~rak/papers/IFIP\%2074.pdf}{\code{(Online)}}

Kowalski, Robert.  1979.  \textit{Logic for Problem Solving}. New York:
North-Holland.

\phantomsection \label{Lamport (1978)}
Lamport, Leslie. 1978.  Time, clocks, and the ordering of events in a
distributed system.  \textit{Communications of the \acronym{ACM}} 21(7): 558-565.
\href{http://www.stanford.edu/class/cs240/readings/lamport.pdf}{\code{(Online)}}

\phantomsection \label{Lampson et al. 1981}
Lampson, Butler, J. J. Horning, R.  London, J. G. Mitchell, and G. K.  Popek.
1981.  Report on the programming language Euclid.  Technical report, Computer
Systems Research Group, University of Toronto.
\href{http://www.bitsavers.org/pdf/xerox/parc/techReports/CSL-81-12_Report_On_The_Programming_Language_Euclid.pdf}{\code{(Online)}}

\phantomsection \label{Landin (1965)}
Landin, Peter.  1965.  A correspondence between Algol 60 and Church's lambda
notation: Part I.  \textit{Communications of the \acronym{ACM}} 8(2): 89-101.

\phantomsection \label{Lieberman and Hewitt 1983}
Lieberman, Henry, and Carl E. Hewitt. 1983. A real-time garbage collector based
on the lifetimes of objects. \textit{Communications of the \acronym{ACM}}
26(6): 419-429.
\href{http://dspace.mit.edu/handle/1721.1/6335}{\code{(Online)}}

\phantomsection \label{Liskov and Zilles (1975)}
Liskov, Barbara H., and Stephen N. Zilles.  1975.  Specification techniques for
data abstractions.  \textit{\acronym{IEEE} Transactions on Software Engineering}
1(1): 7-19.
\href{http://csg.csail.mit.edu/CSGArchives/memos/Memo-117.pdf}{\code{(Online)}}

\phantomsection \label{McAllester (1978; 1980)}
McAllester, David Allen.  1978.  A three-valued truth-maintenance system.  Memo
473, \acronym{MIT} Artificial Intelligence Laboratory.
\href{http://dspace.mit.edu/handle/1721.1/6296}{\code{(Online)}}

McAllester, David Allen.  1980.  An outlook on truth maintenance.  Memo 551,
\acronym{MIT} Artificial Intelligence Laboratory.
\href{http://dspace.mit.edu/handle/1721.1/6327}{\code{(Online)}}

\phantomsection \label{McCarthy 1960}
McCarthy, John.  1960.  Recursive functions of symbolic expressions and their
computation by machine.  \textit{Communications of the \acronym{ACM}}
3(4): 184-195.
\href{http://innovation.it.uts.edu.au/projectjmc/articles/recursive.html}{\code{(Online)}}

\phantomsection \label{McCarthy 1963}
McCarthy, John.  1963.  A basis for a mathematical theory of computation.  In
\textit{Computer Programming and Formal Systems}, edited by P. Braffort and
D. Hirschberg.  North-Holland.
\href{http://innovation.it.uts.edu.au/projectjmc/articles/basis.html}{\code{(Online)}}

\phantomsection \label{McCarthy 1978}
McCarthy, John.  1978.  The history of Lisp.  In \textit{Proceedings of the
\acronym{ACM} \acronym{SIGPLAN} Conference on the History of Programming
Languages}.
\href{http://innovation.it.uts.edu.au/projectjmc/articles/lisp.html}{\code{(Online)}}

\phantomsection \label{McCarthy et al. 1965}
McCarthy, John, P. W. Abrahams, D. J. Edwards, T. P. Hart, and M. I.  Levin.
1965.  \textit{Lisp 1.5 Programmer's Manual}.  2nd edition.  Cambridge, MA:
\acronym{MIT} Press.
\href{http://www.softwarepreservation.org/projects/LISP/book/LISP\%201.5\%20Programmers\%20Manual.pdf/view}{\code{(Online)}}

\phantomsection \label{McDermott and Sussman (1972)}
McDermott, Drew, and Gerald Jay Sussman.  1972. Conniver reference manual.
Memo 259, \acronym{MIT} Artificial Intelligence Laboratory.
\href{http://dspace.mit.edu/handle/1721.1/6203}{\code{(Online)}}

\phantomsection \label{Miller 1976}
Miller, Gary L.  1976.  Riemann's Hypothesis and tests for primality.
\textit{Journal of Computer and System Sciences} 13(3): 300-317.
\href{http://www.cs.cmu.edu/~glmiller/Publications/b2hd-Mi76.html}{\code{(Online)}}

\phantomsection \label{Miller and Rozas 1994}
Miller, James S., and Guillermo J. Rozas. 1994.  Garbage collection is fast,
but a stack is faster.  Memo 1462, \acronym{MIT} Artificial Intelligence
Laboratory.
\href{http://dspace.mit.edu/handle/1721.1/6622}{\code{(Online)}}

\phantomsection \label{Moon 1978}
Moon, David.  1978.  MacLisp reference manual, Version 0.  Technical report,
\acronym{MIT} Laboratory for Computer Science.
\href{http://www.softwarepreservation.org/projects/LISP/MIT/Moon-MACLISP_Reference_Manual-Apr_08_1974.pdf/view}{\code{(Online)}}

\phantomsection \label{Moon and Weinreb 1981}
Moon, David, and Daniel Weinreb.  1981.  Lisp machine manual.  Technical
report, \acronym{MIT} Artificial Intelligence Laboratory.
\href{http://www.unlambda.com/lmman/index.html}{\code{(Online)}}

\phantomsection \label{Morris et al. 1980}
Morris, J. H., Eric Schmidt, and Philip Wadler.  1980.  Experience with an
applicative string processing language.  In \textit{Proceedings of the 7th Annual
\acronym{ACM} \acronym{SIGACT}/\acronym{SIGPLAN} Symposium on the Principles of
Programming Languages}.

\phantomsection \label{Phillips 1934}
Phillips, Hubert.  1934. \textit{The Sphinx Problem Book}.  London: Faber and
Faber.

\phantomsection \label{Pitman 1983}
Pitman, Kent. 1983. The revised MacLisp Manual (Saturday evening edition).
Technical report 295, \acronym{MIT} Laboratory for Computer Science.
\href{http://maclisp.info/pitmanual}{\code{(Online)}}

\phantomsection \label{Rabin 1980}
Rabin, Michael O. 1980. Probabilistic algorithm for testing primality.
\textit{Journal of Number Theory} 12: 128-138.

\phantomsection \label{Raymond 1993}
Raymond, Eric.  1993. \textit{The New Hacker's Dictionary}. 2nd edition.
Cambridge, MA: \acronym{MIT} Press.
\href{http://www.outpost9.com/reference/jargon/jargon_toc.html}{\code{(Online)}}

Raynal, Michel. 1986. \textit{Algorithms for Mutual Exclusion}.  Cambridge, MA:
\acronym{MIT} Press.

\phantomsection \label{Rees and Adams 1982}
Rees, Jonathan A., and Norman I. Adams IV. 1982.  T: A dialect of Lisp or,
lambda: The ultimate software tool.  In \textit{Conference Record of the 1982
\acronym{ACM} Symposium on Lisp and Functional Programming}, pp.  114-122.
\href{http://people.csail.mit.edu/riastradh/t/adams82t.pdf}{\code{(Online)}}

Rees, Jonathan, and William Clinger (eds). 1991.  The \( \rm revised^4 \) report on the
algorithmic language Scheme.  \textit{Lisp Pointers}, 4(3).
\href{http://people.csail.mit.edu/jaffer/r4rs_toc.html}{\code{(Online)}}

\phantomsection \label{Rivest et al. (1977)}
Rivest, Ronald, Adi Shamir, and Leonard Adleman.  1977.  A method for obtaining
digital signatures and public-key cryptosystems. Technical memo LCS/TM82,
\acronym{MIT} Laboratory for Computer Science.
\href{http://people.csail.mit.edu/rivest/Rsapaper.pdf}{\code{(Online)}}

\phantomsection \label{Robinson 1965}
Robinson, J. A. 1965.  A machine-oriented logic based on the resolution
principle.  \textit{Journal of the \acronym{ACM}} 12(1): 23.

\phantomsection \label{Robinson 1983}
Robinson, J. A. 1983.  Logic programming---Past, present, and future.
\textit{New Generation Computing} 1: 107-124.

\phantomsection \label{Spafford 1989}
Spafford, Eugene H.  1989.  The Internet Worm: Crisis and aftermath.
\textit{Communications of the \acronym{ACM}} 32(6): 678-688.
\href{http://citeseerx.ist.psu.edu/viewdoc/download?doi=10.1.1.123.8503&rep=rep1&type=pdf}{\code{(Online)}}

\phantomsection \label{Steele 1977}
Steele, Guy Lewis, Jr.  1977.  Debunking the ``expensive procedure call'' myth.
In \textit{Proceedings of the National Conference of the \acronym{ACM}},
pp. 153-62.
\href{http://dspace.mit.edu/handle/1721.1/5753}{\code{(Online)}}

\phantomsection \label{Steele 1982}
Steele, Guy Lewis, Jr.  1982.  An overview of Common Lisp.  In
\textit{Proceedings of the \acronym{ACM} Symposium on Lisp and Functional
Programming}, pp. 98-107.

\phantomsection \label{Steele 1990}
Steele, Guy Lewis, Jr.  1990.  \textit{Common Lisp: The Language}. 2nd edition.
Digital Press.
\href{http://www.cs.cmu.edu/Groups/AI/html/cltl/cltl2.html}{\code{(Online)}}

\phantomsection \label{Steele and Sussman 1975}
Steele, Guy Lewis, Jr., and Gerald Jay Sussman.  1975.  Scheme: An interpreter
for the extended lambda calculus.  Memo 349, \acronym{MIT} Artificial
Intelligence Laboratory.
\href{http://dspace.mit.edu/handle/1721.1/5794}{\code{(Online)}}

\phantomsection \label{Steele et al. 1983}
Steele, Guy Lewis, Jr., Donald R. Woods, Raphael A. Finkel, Mark R.  Crispin,
Richard M. Stallman, and Geoffrey S. Goodfellow.  1983.  \textit{The Hacker's
Dictionary}. New York: Harper \& Row.
\href{http://www.dourish.com/goodies/jargon.html}{\code{(Online)}}

\phantomsection \label{Stoy 1977}
Stoy, Joseph E.  1977.  \textit{Denotational Semantics}. Cambridge, MA:
\acronym{MIT} Press.

\phantomsection \label{Sussman and Stallman 1975}
Sussman, Gerald Jay, and Richard M. Stallman.  1975.  Heuristic techniques in
computer-aided circuit analysis.  \textit{\acronym{IEEE} Transactions on Circuits
and Systems} CAS-22(11): 857-865.
\href{http://dspace.mit.edu/handle/1721.1/5803}{\code{(Online)}}

\phantomsection \label{Sussman and Steele 1980}
Sussman, Gerald Jay, and Guy Lewis Steele Jr.  1980.  Constraints---A language
for expressing almost-hierachical descriptions.  \textit{AI Journal} 14: 1-39.
\href{http://dspace.mit.edu/handle/1721.1/6312}{\code{(Online)}}

\phantomsection \label{Sussman and Wisdom 1992}
Sussman, Gerald Jay, and Jack Wisdom.  1992. Chaotic evolution of the solar
system.  \textit{Science} 257: 256-262.
\href{http://groups.csail.mit.edu/mac/users/wisdom/ss-chaos.pdf}{\code{(Online)}}

\phantomsection \label{Sussman et al. (1971)}
Sussman, Gerald Jay, Terry Winograd, and Eugene Charniak.  1971.  Microplanner
reference manual.  Memo 203A, \acronym{MIT} Artificial Intelligence Laboratory.
\href{http://dspace.mit.edu/handle/1721.1/6184}{\code{(Online)}}

\phantomsection \label{Sutherland (1963)}
Sutherland, Ivan E.  1963.  \acronym{SKETCHPAD}: A man-machine graphical
communication system.  Technical report 296, \acronym{MIT} Lincoln Laboratory.
\href{http://citeseer.ist.psu.edu/viewdoc/summary?doi=10.1.1.10.4290}{\code{(Onl.)}}

\phantomsection \label{Teitelman 1974}
Teitelman, Warren.  1974.  Interlisp reference manual.  Technical report, Xerox
Palo Alto Research Center.

\phantomsection \label{Thatcher et al. 1978}
Thatcher, James W., Eric G. Wagner, and Jesse B. Wright. 1978.  Data type
specification: Parameterization and the power of specification techniques. In
\textit{Conference Record of the Tenth Annual \acronym{ACM} Symposium on Theory
of Computing}, pp. 119-132.

\phantomsection \label{Turner 1981}
Turner, David.  1981.  The future of applicative languages.  In
\textit{Proceedings of the 3rd European Conference on Informatics}, Lecture Notes
in Computer Science, volume 123. New York: Springer-Verlag, pp.  334-348.

\phantomsection \label{Wand 1980}
Wand, Mitchell.  1980.  Continuation-based program transformation strategies.
\textit{Journal of the \acronym{ACM}} 27(1): 164-180.
\href{http://citeseerx.ist.psu.edu/viewdoc/summary?doi=10.1.1.83.8567}{\code{(Online)}}

\phantomsection \label{Waters (1979)}
Waters, Richard C.  1979.  A method for analyzing loop programs.
\textit{\acronym{IEEE} Transactions on Software Engineering} 5(3): 237-247.

Winograd, Terry.  1971.  Procedures as a representation for data in a computer
program for understanding natural language.  Technical report AI TR-17,
\acronym{MIT} Artificial Intelligence Laboratory.
\href{http://dspace.mit.edu/handle/1721.1/7095}{\code{(Online)}}

\phantomsection \label{Winston 1992}
Winston, Patrick. 1992. \textit{Artificial Intelligence}.  3rd edition.  Reading,
MA: Addison-Wesley.

\phantomsection \label{Zabih et al. 1987}
Zabih, Ramin, David McAllester, and David Chapman.  1987.  Non-deterministic
Lisp with dependency-directed backtracking.  \textit{\acronym{AAAI}-87},
pp. 59-64.
\href{http://www.aaai.org/Papers/AAAI/1987/AAAI87-011.pdf}{\code{(Online)}}

\phantomsection \label{Zippel (1979)}
Zippel, Richard.  1979.  Probabilistic algorithms for sparse polynomials.
Ph.D. dissertation, Department of Electrical Engineering and Computer Science,
\acronym{MIT}.

\phantomsection \label{Zippel 1993}
Zippel, Richard.  1993.  \textit{Effective Polynomial Computation}.  Boston, MA:
Kluwer Academic Publishers.

\chapter*{課題リスト}
\addcontentsline{toc}{chapter}{課題リスト}
\label{List of Exercises}

% 
\input{exercises}

\chapter*{図一覧}
\addcontentsline{toc}{chapter}{図一覧}
\label{List of Figures}

% 
\input{figures}


\setindexprenote{\normalsize \begin{quote}
この索引内のどんな間違いも、コンピュータの手助けにより準備された
という事実により説明できるだろう。

---Donald E. Knuth, \textit{Fundamental Algorithms}\\ 
(Volume 1 of \textit{The Art of Computer Programming}) \end{quote}}

\printindex

\chapter*{奥付}
\addcontentsline{toc}{chapter}{奥付}
\label{Colophon}

表紙は1588年、Agostino Ramelliのブックホイールのメカニズムです。これは初期のハイパーテキスト ナビゲーション支援と
見ることができるのではないでしょうか。この版画のイメージは
\href{http://newgottland.com/2012/02/09/before-the-ereader-there-was-the-wheelreader/ramelli_bookwheel_1032px/}{New Gottland}. 
のJ. E. Johnsonにより提供されています。

タイプフェイスは本文はLinux Libertineで、見出しはLinux Biolinumです。両方ともPhilipp H. Pollの手によります。
タイプライターフェイスはRaph LevienによるInconsolataであり、Dimosthenis KaponisとTakashi Tanigawaにより補完された
Inconsolata LGCの形式で利用しています。

(日本語版では漢字にIPAフォントを使用させて頂いてます。)

グラフィックデザインとタイポグラフィはAndres Rabaにより行われました。TexinfoのソースはPerlスクリプトにより
LaTeXに変換され、XeLaTeXにより\acronym{PDF}にコンパイルされています。図はInkscapeを用いて描かれました。

\end{document}
